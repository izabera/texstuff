\parte{PARTE SESTA}\label{parte-sesta} 
\pagestyle{pagina}

\capitolo{I}Dar'ja Aleksandrovna passava l'estate coi bambini a Pokrovskoe da sua sorella Kitty Levina. Nella sua proprietà la casa era crollata del tutto. Levin e sua moglie l'avevano convinta a passare l'estate da loro. Stepan Arkad'ic aveva approvato molto questa organizzazione. Diceva che gli spiaceva che l'ufficio gli impedisse di passare l'estate in campagna con la famiglia, cosa che per lui rappresentava la più grande felicità; intanto, rimanendo a Mosca, veniva di rado in campagna per un giorno o due. Oltre agli Oblonskij, con i bambini e la governante, era ospite quell'estate dei Levin anche la vecchia principessa, che riteneva suo dovere sorvegliare la figlia inesperta, in stato interessante. Inoltre Varen'ka, l'amica di Kitty a Soden, aveva mantenuto la promessa di recarsi da lei appena Kitty si fosse sposata, ed era ospite dell'amica. Questi erano tutti parenti e amici della moglie di Levin. Ed egli, pur volendo bene a tutti, si rammaricava un po' e per il suo mondo e per l'ordine delle cose ``leviniano'' che veniva soffocato da quell'inondazione dell'``elemento šcerbackiano'', come diceva lui. Dei suoi parenti era loro ospite, quell'estate, il solo Sergej Ivanovic, e anche costui non era un uomo dalla struttura leviniana, ma koznyševiana, così che lo spirito leviniano veniva completamente annientato. 

Nella casa dei Levin, così a lungo vuota, c'era, adesso, tanta gente che quasi tutte le stanze erano occupate, e quasi ogni giorno la vecchia principessa, sedendosi a tavola, doveva contar tutti e isolare il tredicesimo nipotino su di un tavolino a parte. E per Kitty, che si occupava accuratamente della casa, c'era non poco lavoro a procurarsi galline e tacchini e anitre, di cui, con l'appetito estivo degli ospiti e dei bambini, occorreva gran numero. 

Tutta la famiglia era seduta a tavola. I bambini di Dolly con la governante e Varen'ka, facevano progetti per andare a cogliere funghi. Sergej Ivanovic, che fra gli ospiti, per intelligenza e cultura, godeva di un prestigio che giungeva quasi all'adorazione, sorprese tutti con l'intervenire nel discorso sui funghi. 

- Prendete anche me con voi. Mi piace molto andare a cercar funghi. - disse, guardando Varen'ka; - penso che sia una bella occupazione. 

- Ma certo, siamo molto contenti - rispose Varen'ka, facendosi rossa. Kitty scambiò un'occhiata significativa con Dolly. La proposta di Sergej Ivanovic, persona di cultura e d'ingegno, di andare a cercar funghi con Varen'ka, confermava alcune supposizioni di Kitty, che negli ultimi tempi l'avevano molto interessata. Si affrettò a parlare con la madre, per non far notare la sua occhiata. Dopo pranzo Sergej Ivanovic sedette con la sua tazza di caffè accanto alla finestra nel salotto, continuando una conversazione incominciata col fratello e guardando di tanto in tanto la porta dalla quale dovevano venir fuori i bambini, pronti per andare a cercar funghi. Levin si accoccolò sulla finestra accanto al fratello. 

Kitty stava in piedi vicino al marito, evidentemente aspettando la fine della conversazione che non era interessante, per dirgli qualcosa. 

- Sei cambiato in molte cose da che sei sposato, e in meglio - disse, sorridendo a Kitty, Sergej Ivanovic, poco interessato evidentemente alla conversazione avviata - ma sei rimasto fedele alla tua passione di difendere temi paradossali. 

- Katja, non ti fa bene restare in piedi - le disse il marito, accostandole una seggiola e guardandola significativamente. 

- Eh sì, il resto a dopo, non c'è neanche tempo - soggiunse Sergej Ivanovic, vedendo i bambini venir fuori correndo. 

Innanzi a tutti, di fianco, a galoppo, con le sue calze tese, agitando un cestino e il cappello di Sergej Ivanovic, proprio verso di lui correva Tanja. 

Correndo verso Sergej Ivanovic, con gli occhi scintillanti, tanto simili ai bellissimi occhi del padre, ella porse a Sergej Ivanovic il cappello e fece finta di volerglielo mettere, mitigando la libertà che s'era presa con un sorriso timido e delicato. 

- Varen'ka aspetta - disse, mettendogli cauta il cappello, dopo aver visto dal sorriso di Sergej Ivanovic che si poteva. 

Varen'ka stava in piedi sulla porta, in abito di seta gialla, con un fazzoletto bianco annodato in capo. 

- Vengo, vengo, Varvara Andreevna - diceva Sergej Ivanovic, finendo di bere la tazza di caffè e distribuendo per le tasche il fazzoletto e il portasigari. 

- Ma che incanto è la mia Varen'ka! Eh? - disse Kitty al marito, non appena Sergej Ivanovic si fu alzato. Lo disse in modo che Sergej Ivanovic potesse sentirla, cosa che evidentemente desiderava. - E com'è bella, nobilmente bella! Varen'ka! - gridò Kitty - andate nel bosco, al mulino? Verremo anche noi da voi. 

- Tu decisamente dimentichi il tuo stato - prese a dire la vecchia principessa, uscendo in fretta sulla porta. - Non devi gridare così. 

Varen'ka, udita la voce di Kitty e il rimprovero della madre, si accostò a Kitty a passi rapidi e leggeri. La velocità del movimento, il colorito che le copriva il viso animato, tutto faceva vedere che in lei accadeva qualcosa di eccezionale. Kitty sapeva cos'era questo qualcosa di eccezionale, e la osservava attenta. Adesso, aveva chiamato Varen'ka solo per benedirla mentalmente, in vista di un avvenimento così importante che, secondo Kitty, doveva compiersi in quel giorno, dopo pranzo, nel bosco. 

- Varen'ka, sarò molto contenta se accadrà una certa cosa - disse sottovoce, baciandola. 

- E voi verrete con noi? - disse Varen'ka confusa a Levin, facendo finta di non aver sentito quello che le era stato detto. 

- Verrò solo fino all'aia e rimarrò là. 

- Ma che gusto ci provi? - disse Kitty. 

- Bisogna guardare e verificare i carri nuovi - disse Levin. - E tu dove resterai? 

- Sulla terrazza. 

\capitolo{II}Sulla terrazza s'era riunita tutta la compagnia delle donne di casa. In genere piaceva loro sedere là dopo pranzo, ma quel giorno c'era anche da fare. Oltre la confezione delle camicine e delle fasce a maglia, di cui tutte si occupavano, quel giorno si confezionava la marmellata secondo un metodo nuovo per Agaf'ja Michajlovna, senza, cioè, aggiungervi acqua. Kitty introduceva questo metodo nuovo usato in casa sua. Agaf'ja Michajlovna, alla quale in un primo momento era stato affidato questo lavoro, ritenendo che tutto quello che si faceva in casa Levin non potesse essere mal fatto, aveva, nonostante tutto, versato dell'acqua sulle fragole del giardino e su quelle selvatiche, convinta che era impossibile fare altrimenti; ma era stata sorpresa in questo, e ora si cuocevano i lamponi alla presenza di tutti. Agaf'ja Michajlovna avrebbe dovuto convincersi che, anche senza acqua, la marmellata sarebbe riuscita bene. 

Agaf'ja Michajlovna, col viso rosso e crucciato, i capelli arruffati e le braccia magre nude fino al gomito, faceva dondolare circolarmente la casseruolina sul braciere e guardava torva i lamponi, desiderando con tutta l'anima che si rappigliassero e che non finissero di cuocere. La principessa, sentendo che l'ira di Agaf'ja Michajlovna era diretta contro di lei, quale consulente più esperta nella cottura dei lamponi, cercava di aver l'aria di attendere ad altro e di non interessarsi affatto dei lamponi; parlava di cose estranee ma di tanto in tanto sbirciava il braciere. 

- Io compro sempre da me al mercato i vestiti per le ragazze - diceva la principessa, continuando un discorso incominciato.- Non si deve levare adesso la schiuma, amica mia! - soggiungeva rivolta ad Agaf'ja Michajlovna. - Ma non lo devi fare tu, fa caldo - disse, trattenendo Kitty. 

- Lo farò io - disse Dolly e, alzatasi, cominciò a passare col cucchiaio su per lo zucchero schiumoso, battendolo, di tanto in tanto, per staccarne quel che vi s'era appiccicato, sul piatto già coperto di schiuma variopinta giallorosa, con lo sciroppo sanguigno che colava sotto. ``Come se lo leccheranno col tè!'' pensava dei suoi bambini, ricordando come lei stessa, nell'infanzia, si stupiva che i grandi non mangiassero il meglio, la schiuma. 

- Stiva dice che è preferibile dare in denaro - continuava intanto Dolly l'argomento interessante incominciato sul modo migliore di far regali alla servitù - ma\ldots{} 

- Come si può dar denaro! - cominciarono a dire a una voce la principessa e Kitty. - Esse apprezzano questo. 

- Io, per esempio, l'anno scorso, ho comprato per la nostra Matrëna Semënovna non popeline, ma una specie - disse la principessa. 

- Mi ricordo, ce l'aveva al vostro onomastico. 

- Un disegno molto grazioso; così semplice e distinto. Me lo sarei fatto anch'io, se non l'avesse avuto lei. Una specie di quello di Varen'ka. Così carino e a buon prezzo. 

- Su, adesso mi pare che sia pronta - disse Dolly, facendo colar giù lo sciroppo dal cucchiaio. 

- Quando è caramellato, allora è pronto. Cuocete ancora, Agaf'ja Michajlovna. 

- Quante mosche! - disse, rabbiosa, Agaf'ja Michajlovna. - Sarà sempre lo stesso - aggiunse. 

- Ah, com'è carino, non lo spaventate! - disse inaspettatamente Kitty, guardando un passerotto che s'era posato sulla balaustrata e che, rigirato un rametto di lampone, aveva preso a beccarlo. 

- Sì, ma sta' lontana dal braciere - disse la madre. 

- A propos de Varen'ka - disse Kitty in francese, come del resto parlavano sempre perché Agaf'ja Michajlovna non capisse. - Voi sapete, maman, che oggi, chissà perché, mi aspetto una decisione. Voi capite quale. Come sarebbe bello! 

- Ma che abile pronuba! - disse Dolly. - Con che accortezza e con che abilità li mette insieme\ldots{} 

- No, ditemi, maman, cosa ne pensate? 

- Ma cosa pensare? Lui - ``lui'' era naturalmente Sergej Ivanovic - ha sempre potuto rappresentare il miglior partito della Russia; ora non è più tanto giovane, tuttavia, molte lo sposerebbero\ldots{} Lei è molto buona, ma egli potrebbe\ldots{} 

- No, mamma, capitemi, per lui e per lei non si può pensare nulla di meglio. Per prima cosa - disse Kitty piegando un dito - lei è una delizia! 

- Lei gli piace molto, questo è vero - confermò Dolly. 

- Poi, egli occupa una posizione tale in società che non ha assolutamente bisogno né del patrimonio né della posizione sociale della moglie. Ha bisogno di una cosa sola: di una moglie buona, simpatica, tranquilla. 

- Sì, con lei davvero si può esser tranquilli - confermò Dolly. 

- Terza cosa, che lei lo ami. E questo c'è\ldots{} Sarebbe così bello!\ldots{} Ecco, io aspetto che spuntino fuori dal bosco e che tutto si decida. Me ne accorgerò subito dagli occhi. Sarei così contenta. Che ne pensi, Dolly? 

- Ma tu non agitarti. Tu non devi agitarti per nulla - disse la madre. 

- Ma io non mi agito, mamma. Mi pare solo che quest'oggi egli farà la sua proposta. 

- Ah, è così strano, quando un uomo fa la sua dichiarazione\ldots{} C'è una certa barriera e a un tratto questa si rompe - disse Dolly, sorridendo pensierosa e ricordando il proprio passato con Stepan Arkad'ic. 

- Mamma, come vi ha fatto papà la sua domanda? - chiese a un tratto Kitty. 

- Non c'è stato nulla di straordinario, molto semplicemente - disse la principessa, ma il suo viso s'illuminò tutto al ricordo. 

- No, ma come? Lo amavate, però, prima che vi permettessero di parlare? 

Kitty provava un incanto particolare nel poter parlare adesso con la madre come ad una sua pari, di queste importanti questioni della vita di una donna. 

- S'intende che l'amavo; veniva da noi in campagna. 

- Ma come si decise la cosa, mamma? 

- Credete forse di aver inventato qualcosa di nuovo? È sempre la stessa cosa: si risolve con gli occhi, con il sorriso\ldots{} 

- Come l'avete detto bene, mamma! Proprio con gli occhi, con il sorriso - ripeté Dolly. 

- Ma quali parole diceva? 

- Quali ti diceva Kostja? 

- Lui le scriveva col gesso. Che cosa strana\ldots{} Come mi sembra lontano tutto ciò! - ella disse. 

E le tre donne cominciarono a pensare alla stessa cosa. Kitty ruppe il silenzio per prima. Le era venuto in mente l'ultimo inverno prima del suo matrimonio e la sua esaltazione per Vronskij. 

- Una cosa sola\ldots{} la passione precedente di Varen'ka - disse, ricordandosi di questo per naturale associazione di idee. - Volevo dirlo, in qualche modo, a Sergej Ivanovic, prepararlo. Loro, tutti gli uomini - aggiunse - sono terribilmente gelosi del nostro passato. 

- Non tutti - disse Dolly. - Tu giudichi da tuo marito. Ancora adesso lui si tormenta al ricordo di Vronskij. Non è vero, forse? 

- È vero - rispose Kitty sorridendo, pensosa, con gli occhi. 

- Ma io non so - disse la principessa in difesa della sua protezione materna sulla figlia - quale tuo passato potesse mai agitarlo! Forse il fatto che Vronskij ti avesse fatto la corte? Questo succede a ogni ragazza. 

- Ma non parliamo di questo - disse Kitty, arrossendo. 

- No, permetti - continuò la madre - e poi tu stessa non volevi che io parlassi a Vronskij. Ricordi? 

- Ah, mamma! - disse Kitty con un'espressione di sofferenza. 

- Adesso, non si riesce a trattenervi\ldots{} I tuoi rapporti non potevano andare oltre quel che si conviene; io stessa l'avrei richiamato. Inoltre a te, anima mia, non fa bene agitarti; non dimenticarlo, ti prego e sta' tranquilla. 

- Sono perfettamente calma, Ma. 

- Come fu bene per Kitty che venisse Anna! - disse Dolly - e come è stato male per lei! Ecco, proprio tutto al contrario! - aggiunse, colpita dalla propria idea. - Allora Anna era così felice e Kitty si considerava infelice! Come è ora tutto al contrario! Io penso spesso a lei. 

- Non hai a che pensare! una donna odiosa, ripugnante, senza cuore - disse la madre, che non poteva dimenticare che Kitty non aveva sposato Vronskij ma Levin. 

- Che gusto c'è a parlar di questo? - disse Kitty con dispetto - io non ci penso e non ci voglio pensare\ldots{} E non ci voglio pensare - disse, prestando orecchio ai passi noti del marito per la scala della terrazza. 

- E che cosa significa questo ``e non ci voglio pensare''? - chiese Levin, entrando sulla terrazza. 

Ma nessuno gli rispose, ed egli non ripeté la domanda. 

- Mi dispiace di aver disturbato il vostro regno femminile - disse, dopo aver osservato tutte, contrariato, e dopo aver capito che si stava parlando di una cosa che non avrebbero ripetuto a lui. 

Per un attimo sentì che condivideva il sentimento di Agaf'ja Michajlovna, lo scontento perché i lamponi si cuocevano senza acqua, e in genere l'influenza estranea degli Šcerbackij. Sorrise, tuttavia, e si accostò a Kitty. 

- Be', come va? - chiese, guardandola con quella stessa espressione con cui adesso tutti si volgevano a lei. 

- Niente, sto benissimo - disse sorridendo Kitty - e da te come si va? 

- Ne portano tre volte di più di un carro. Allora devo andare a prendere i bambini? Ho detto di attaccare. 

- Ma come! vuoi portare Kitty in calesse? - chiese la madre in tono di rimprovero. 

- Ma se va al passo, principessa. 

Levin non chiamava mai la principessa maman, come fanno i generi, e questo dispiaceva alla principessa. Levin, pur amando e rispettando la principessa, non poteva chiamarla così senza offendere il proprio sentimento verso la madre morta. 

- Venite con noi, maman - disse Kitty. 

- Non voglio assistere a queste sconsideratezze. 

- Via, andrò a piedi. Mi fa bene del resto. - Kitty si alzò, si avvicinò al marito e lo prese per mano. 

- Fa bene, ma tutto con misura - disse la principessa. 

- Be', Agaf'ja Michajlovna, è pronta la marmellata? - disse Levin, sorridendo ad Agaf'ja Michajlovna e desiderando rallegrarla. - Va bene col nuovo metodo? 

- Deve andar bene. Per noi è troppo cotta. 

- È anche meglio, Agaf'ja Michajlovna, non andrà a male, e qui da noi ormai il ghiaccio si è già sciolto e non si sa dove conservarla - disse Kitty, dopo aver capito subito l'intenzione del marito, e rivolgendosi alla vecchia col medesimo intento. - In compenso la vostra salamoia è così buona che la mamma dice di non averne mangiata in nessun posto una simile - soggiunse, sorridendo e accomodandole addosso lo scialletto. 

Agaf'ja Michajlovna guardò Kitty con rabbia. 

- Non mi consolate, signora. Ecco, vi guardo con lui e mi rallegro - ella disse, e questo rozzo modo di esprimersi commosse Kitty. 

- Venite con noi a cercar funghi, ci farete vedere i posti. - Agaf'ja Michajlovna sorrise, scosse il capo, quasi a dire: ``Sarei anche contenta d'arrabbiarmi con voi, ma non si può''. 

- Fate secondo il mio consiglio, per favore - disse la vecchia principessa; - sopra la marmellata mettete un pezzetto di carta e bagnatelo di rum: anche quando non sarà più freddo, non si formerà la muffa. 

\capitolo{III}Kitty fu particolarmente felice di trovarsi sola con suo marito, perché aveva notato come un'ombra di disappunto fosse passata sul suo viso, che rispecchiava tutto così vivacemente, nel momento in cui era entrato sulla terrazza e aveva domandato di che si parlava e non gli era stato risposto. 

Quando si avviarono a piedi, avanti agli altri, e sparvero dalla vista della casa sulla strada battuta, polverosa e cosparsa di spighe e di granelli di segala, ella si appoggiò fortemente al braccio di lui e lo strinse a sé. Egli aveva già dimenticato l'impressione spiacevole di un attimo prima e ora, solo con lei, mentre il pensiero della sua gravidanza non lo lasciava neppure per un momento, provava quella soddisfazione, ancora nuova e gioiosa per lui, del tutto scevra di sensualità, della vicinanza della donna amata. Non c'era nulla da dire, ma egli voleva ascoltare la voce di lei, come pure vederne lo sguardo, cambiatosi ora con la gravidanza. Nella voce, come pure nello sguardo c'erano una dolcezza e una serietà simili a quelli che hanno le persone continuamente concentrate in un'unica opera amata. 

- Allora non ti stancherai? Appoggiati di più - disse. 

- No, sono così contenta d'esser rimasta per caso con te; ti confesso che, per quanto bene stia con loro, rimpiango le nostre serate invernali a due. 

- Allora era bene, e adesso è ancora meglio. Tutte e due le cose sono buone - disse lui, stringendole il braccio. 

- Sai di che parlavamo quando sei entrato? 

- Della marmellata? 

- Sì, anche della marmellata; ma poi della maniera di far la dichiarazione. 

- Ah! - disse Levin, ascoltando più il suono della voce di lei che non le parole, badando tutto il tempo alla strada, che in quel punto attraversava un bosco, ed evitando di passare nei punti dove ella avrebbe potuto mettere un piede in fallo. 

- E di Sergej Ivanovic e Varen'ka. Hai fatto caso? Io lo desidero molto - ella proseguì: - Cosa ne pensi? - E lo guardò in viso. 

- Non so cosa pensare - rispose, sorridendo, Levin. - Sergej Ivanovic, sotto questo riguardo, è molto strano per me. Perché t'ho raccontato\ldots{} 

- Sì, che era innamorato di quella ragazza che è morta\ldots{} 

- Successe quando ero bambino; lo so per quel che m'hanno detto. Me lo ricordo allora. Era straordinariamente simpatico. Ma da allora lo osservo nei suoi rapporti con le donne: è gentile, alcune gli piacciono, ma senti che per lui sono semplicemente degli esseri, non delle donne. 

- Sì, ma adesso con Varen'ka\ldots{} pare che qualcosa ci sia\ldots{} 

- Può darsi che ci sia\ldots{} Ma bisogna conoscerlo\ldots{} È un uomo speciale, straordinario. Vive di sola vita spirituale. È un essere troppo puro e di animo elevato. 

- Come? Questo forse lo abbasserà? 

- No, ma è così abituato a vivere di sola vita spirituale, che non può riconciliarsi con la realtà, e Varen'ka, tuttavia, è la realtà. 

Levin, adesso, s'era già abituato a dire coraggiosamente il proprio pensiero, senza darsi la pena di rivestirlo di parole precise; sapeva che la moglie, in momenti così teneri come quello, avrebbe capito ciò ch'egli voleva dire da un accenno, ed ella lo capì. 

- Sì, ma in lei non c'è tanta realtà quanta in me; capisco come egli non mi amerebbe mai. Lei è tutta spirito. 

- Eh no, lui ti vuol così bene, e a me fa sempre tanto piacere che i miei te ne vogliano\ldots{} 

- Sì, egli è buono con me, ma\ldots{} 

- Ma non è così come col povero Nikolen'ka\ldots{} vi volevate bene l'un l'altro - terminò Levin. - Perché non parlarne? - aggiunse. - A volte, me lo rimprovero: si finisce col dimenticare. Ah, che uomo terribile e delizioso che era\ldots{} Sì, allora, di che stavamo parlando? - proseguì Levin, dopo un momento di silenzio. 

- Tu pensi dunque ch'egli non possa innamorarsi? - disse Kitty, traducendo nel suo linguaggio. 

- Non ch'egli non possa innamorarsi - disse Levin, sorridendo - ma non ha quella debolezza necessaria\ldots{} Io l'ho sempre invidiato e anche adesso, pur essendo così felice, tuttavia lo invidio. 

- Lo invidi perché non riesce a innamorarsi? 

- Lo invidio perché è migliore di me - disse Levin, sorridendo. - Egli non vive per sé. Tuttavia la sua vita è sottoposta al dovere. E lo invidio perché lui sì, che può essere tranquillo e soddisfatto. 

- E tu? - disse Kitty con un sorriso amabile, scherzoso. 

Ella non avrebbe potuto esprimere in nessun modo il corso dei pensieri che la faceva sorridere, ma l'ultima deduzione era questa, che suo marito esaltando il fratello e abbassando se stesso, non era sincero. Kitty sapeva che questa mancanza di sincerità derivava dall'amore per il fratello, da un senso di vergogna della propria felicità e, in particolare, dal desiderio, che non l'abbandonava mai, di essere migliore, e questo le piaceva in lui; perciò sorrideva. 

- E tu? Di che cosa mai sei scontento? - chiese lei, sempre con quel sorriso. 

L'incredulità di lei verso la sua insoddisfazione lo rallegrava ed egli la spingeva inconsciamente a manifestare le ragioni della propria incredulità. 

- Io sono felice, ma scontento di me\ldots{} - egli disse. 

- Allora come puoi mai esser scontento, se sei felice? 

- Cioè, come dirti? Io sinceramente non desidero nulla di più, tranne che tu non inciampi. Ah, ma non si può mica saltare così - egli interruppe il suo dire con un rimprovero per il movimento troppo rapido ch'ella aveva fatto nel saltare un ramo che si trovava sul viottolo. - Ma quando mi giudico e mi paragono agli altri, in particolare a mio fratello, allora sento di essere cattivo. 

- Ma in che cosa? - continuò Kitty con un sorriso. - Non lavori anche tu per gli altri? E la tua azienda, la tua fattoria, e il tuo libro? 

- No, lo sento in modo particolare adesso: tu ne hai colpa - egli disse, stringendole il braccio - che le cose non vadano bene. Lavoro così, alla leggera. Se io potessi amare tutto questo lavoro come amo te\ldots{} invece in quest'ultimo tempo, lo faccio come una lezione assegnata. 

- E allora cosa dirai mai di papà? - chiese Kitty. - Che anche lui è cattivo, perché non ha fatto nulla per il lavoro comune? 

- Lui? no. Bisogna avere quella semplicità, quella chiarezza, quella bontà di tuo padre e io le posseggo forse? Io non faccio nulla e mi tormento. Tutto questo l'hai fatto succedere tu. Quando tu non c'eri e non c'era codesto - egli disse guardando i fianchi di lei in modo ch'ella capisse - io mettevo tutte le mie energie nel lavoro; invece ora non posso e me ne vergogno; lo faccio proprio come una lezione assegnata, fingo\ldots{} 

- Su, via, e vorresti fare il cambio con Sergej Ivanyc - disse Kitty. - Vorresti compiere questo lavoro comune e amare questa lezione assegnata come lui, e basta? 

- S'intende che no - disse Levin. - Del resto, io sono così felice che non capisco nulla. Allora tu credi che oggi egli faccia addirittura la sua proposta di matrimonio? - egli soggiunse dopo aver taciuto un po'. 

- Lo credo e non lo credo. Lo desidero tanto. Ecco, aspetta. - Ella si chinò e strappò dal ciglio della strada una margherita selvatica. - Su, conta: sarà o non sarà - ella disse, dandogli il fiore. 

- Sarà, non sarà - diceva Levin, strappando i petali bianchi, stretti e scanalati. 

- No, no! - lo fermò, afferrandolo per la mano, Kitty che aveva seguito con agitazione il movimento delle dita. - Ne hai strappati due insieme. 

- Su, in compenso, questo piccolo qui non conta - disse Levin, strappando un petalo corto non cresciuto. - Ecco, anche il calesse ci ha raggiunto. 

- Non sei stanca, Kitty? - gridò la principessa. 

- Per nulla. 

- Altrimenti sali su, i cavalli sono tranquilli e vanno al passo. 

Ma non valeva la pena di salir su, erano quasi arrivati e tutti andarono a piedi. 

\capitolo{IV}Varen'ka, con un fazzoletto bianco sui capelli neri, circondata dai bambini di cui si occupava con allegra cordialità, visibilmente agitata dalla possibilità di una spiegazione con un uomo che le piaceva, era molto attraente. Sergej Ivanovic le camminava a fianco e non cessava d'ammirarla. Guardandola, ricordava tutti i discorsi di lode che aveva sentito su di lei, tutto quello che di buono sapeva di lei, e riconosceva sempre più che il sentimento ch'egli provava per lei era qualcosa di particolare, da lui provato tanto e tanto tempo fa e solamente una volta, nella prima giovinezza. Il senso di gioia per la vicinanza di lei, divenendo sempre più forte, giunse al punto che, porgendole nel cestino un enorme fungo prugnolo dalle estremità accartocciate, col gambo sottile, la guardò negli occhi e, notando che un rossore di gioiosa e spaventata agitazione le ricopriva il viso, si confuse egli stesso e le sorrise in silenzio con un sorriso fin troppo eloquente. 

``Se è così - egli disse - devo riflettere e decidere, e non abbandonarmi, come un ragazzo, all'esaltazione d'un momento''. 

- Adesso andrò a cercar funghi da solo, altrimenti le mie conquiste non si notano - disse, e si avviò, solo, dall'estremità del bosco, dove camminavano sull'erba bassa e morbida, fra le betulle vecchie e rade, verso il folto del bosco, dove, in mezzo ai tronchi bianchi delle betulle, apparivano i fusti grigi delle tremule e i cespugli scuri dei nocciuoli. Allontanatosi di una quarantina di passi e oltrepassato un cespuglio di fusaro in pieno fiore, con le pannocchie color rosa, Sergej Ivanovic, sapendo di non essere visto, si fermò. Intorno a lui c'era un silenzio assoluto. Solo in cima alle betulle sotto alle quali si trovava, ronzavano incessantemente le mosche come uno sciame di api, e di tanto in tanto giungevano le voci dei bambini. Ad un tratto, non lontano dal limitare del bosco, risonò la voce di contralto di Varen'ka, che chiamava Griša, e un sorriso gioioso apparve sul viso di Sergej Ivanovic. Avuta la percezione di questo sorriso, Sergej Ivanovic scosse il capo, disapprovando la propria situazione e, tirato fuori un sigaro, prese ad accenderlo. Per un pezzo non riuscì ad accendere il fiammifero contro il tronco di una betulla. Lo strato sottile della corteccia bianca aderiva al fosforo e la fiamma si spegneva. Alla fine uno di questi fiammiferi si accese, e il fumo odoroso del sigaro, come un largo drappo ondeggiante, si allungò in forme precise, al di sopra del cespuglio e oltre, sotto i rami pendenti della betulla. Seguendo con gli occhi la striscia di fumo, Sergej Ivanovic si avviò con passo tranquillo, riflettendo alla propria posizione. 

``E perché no? - egli pensava. - Se questo fosse un capriccio o una passione, se provassi solo una simpatia, una simpatia reciproca (posso ben dire reciproca), e sentissi che non va d'accordo con tutto il mio ordine di vita, se sentissi, abbandonandomi a questa simpatia, di tradire la mia vocazione e il mio dovere\ldots{} ma questo non è. L'unica cosa che posso dire in contrario è che, avendo perduto Marie, mi dicevo che sarei rimasto fedele alla sua memoria. Questa sola cosa posso dire contro il mio sentimento\ldots{} È importante - diceva Sergej Ivanovic, sentendo nello stesso tempo che tale considerazione, per lui personalmente, non poteva avere nessuna importanza, salvo forse a sciupare agli occhi degli altri la propria parte poetica. - Ma al di fuori di questo, per quanto io cerchi, non trovo nulla da poter dire contro il mio sentimento. Se avessi scelto con la sola ragione, non avrei potuto trovar di meglio''. 

Per quante donne e fanciulle note ricordasse, non poteva ricordare una ragazza che riunisse fino a tal punto tutte le qualità ch'egli, ragionando freddamente, desiderava di vedere in sua moglie. Ella aveva tutto l'incanto e la freschezza della giovinezza, ma non era una fanciulla, e se lo amava, lo amava coscientemente, come deve amare una donna: e questa era già una cosa. Un'altra: ella era non solo lontana dalla mondanità, ma, evidentemente, aveva il disprezzo del mondo, e nello stesso tempo lo conosceva, e aveva tutte le maniere della donna di buona società, senza le quali per Sergej Ivanovic era inconcepibile la compagna della propria vita. Terza cosa: ella era religiosa, ma non religiosa e buona senza rendersene conto, come può esserlo un bambino, come, per esempio, era Kitty; la sua vita era fondata su convinzioni religiose. Perfino nei particolari Sergej Ivanovic trovava in lei tutto quello che avrebbe desiderato in una moglie: era povera e sola, sicché non avrebbe portato con sé un nugolo di parenti e la loro influenza in casa del marito, come egli vedeva nel caso di Kitty; ma avrebbe dovuto tutto al marito, cosa ch'egli aveva sempre desiderato per la propria futura vita familiare. E questa ragazza che riuniva in sé queste qualità, lo amava. Egli era modesto, ma non poteva non accorgersene. Anch'egli l'amava. Sola considerazione sfavorevole era la propria età. Ma la sua razza era longeva, egli non aveva neppure un capello bianco, nessuno gli dava quarant'anni, e ricordava che Varen'ka aveva detto che soltanto in Russia gli uomini a cinquant'anni si considerano vecchi, mentre in Francia un uomo di cinquant'anni si considera dans la force de l'âge, e uno di quaranta, un jeune homme. Ma che significava il conto degli anni, quand'egli si sentiva giovane d'animo, come vent'anni prima? Non era forse giovinezza il sentimento che provava ora, quando, uscito di nuovo sul limitare del bosco, dall'altra parte, aveva visto, nella luce viva dei raggi obliqui del sole, la figura graziosa di Varen'ka, col vestito giallo e il cestello, camminare con passo leggero accanto al tronco di una vecchia betulla, e quando questa impressione della vista di Varen'ka si era fusa in uno con la vista d'un campo di avena gialla, splendido, inondato dai raggi obliqui, e, di là del campo, un vecchio bosco lontano, screziato d'oro, che si perdeva nella lontananza azzurra? Il cuore gli si strinse per la gioia. Un senso di commozione lo afferrò. Sentì che si era deciso. Varen'ka, che si era appena abbassata per cogliere un fungo, con un movimento agile si levò e si voltò a guardarlo. Gettato via il sigaro, Sergej Ivanovic si diresse verso di lei a passi decisi. 

\capitolo{V}``Varvara Andreevna, quando ero ancora molto giovane mi sono formato l'ideale della donna che avrei voluto amare ed essere felice di chiamare mia moglie. Ho vissuto una lunga vita e, adesso, per la prima volta ho incontrato in voi quello che cercavo. Vi amo e vi offro la mia mano''. 

Sergej Ivanovic si diceva questo, mentre era già a dieci passi da Varen'ka. In ginocchio e difendendo con le mani un fungo da Griša, Varen'ka chiamava la piccola Maša. 

- Qua, qua piccoli! ce ne sono tanti! - ella diceva con la sua simpatica voce di petto. 

Scorto Sergej Ivanovic che si avvicinava, non si levò e non cambiò posizione; ma tutto a lui diceva ch'ella sentiva il suo avvicinarsi e che ne era felice. 

- Ebbene, avete trovato qualcosa? - ella chiese, volgendo verso di lui, di là dal fazzoletto bianco, il suo bel viso dolcemente sorridente. 

- Neppure uno - disse Sergej Ivanovic. - E voi? 

Ella non gli rispose, occupata con i bambini che la circondavano. 

- Ancora questo, accanto al ramo - ed ella mostrò alla piccola Maša una minuscola rossola tagliata per traverso, nel suo piccolo, morbido cappello rosa, da un filo d'erba secco, da sotto al quale si liberava. Varen'ka si alzò quando Maša, spezzatala in due metà bianche, ebbe tirata su la rossola. - Questo mi ricorda l'infanzia - ella soggiunse, allontanandosi dai bambini a fianco di Sergej Ivanovic. 

Fecero alcuni passi in silenzio. Varen'ka vedeva ch'egli voleva parlare; indovinava di che cosa e veniva meno per l'agitazione, dovuta alla felicità e al timore. Erano ormai così lontani che nessuno avrebbe potuto sentirli; egli tuttavia non cominciava a parlare. Per Varen'ka era meglio tacere. Dopo un silenzio era più facile dire quello ch'essi volevano, che non dopo i discorsi sui funghi; ma contro la sua volontà, come per caso, Varen'ka disse: 

- Allora non avete trovato nulla? Del resto, in mezzo al bosco, ce n'è sempre meno. 

Sergej Ivanovic sospirò e non rispose nulla. Era urtato che si fosse messa a parlare di funghi. Voleva ricondurla alle prime parole ch'ella aveva dette sulla sua infanzia, ma come contro la sua volontà, dopo un po' di silenzio, egli stesso fece un'osservazione sulle ultime parole di lei. 

- Ho sentito dire che solo gli ovoli sono di preferenza ai margini, sebbene io non sappia distinguere un ovolo. 

Passarono ancora alcuni minuti, essi erano andati ancora più lontani dai bambini ed erano completamente soli. Il cuore di Varen'ka batteva tanto ch'ella ne udiva i colpi e sentiva di arrossire, d'impallidire e di arrossire di nuovo. 

Essere la moglie di un uomo come Koznyšev, dopo la sua posizione presso la signora Stahl, le appariva il colmo della felicità. Inoltre era quasi sicura d'esserne innamorata. E ora questo avrebbe dovuto decidersi. Ella ne aveva timore. Era terribile e quello ch'egli avrebbe detto e quello che non avrebbe detto. 

Bisognava spiegarsi adesso o mai più: questo Sergej Ivanovic lo sentiva. Tutto, nello sguardo, nel colorito acceso, negli occhi bassi di Varen'ka tutto rivelava un'aspettazione morbosa. Sergej Ivanovic lo vedeva e aveva pena per lei. Sentiva perfino che non dir nulla, adesso, significava offenderla. Si ripeteva anche le parole con cui voleva esprimerle la sua proposta; ma, invece di queste parole, per un certo pensiero inatteso che gli venne in mente, chiese a un tratto. 

- E che differenza c'è tra un ovolo e un prugnolo? 

Le labbra di Varen'ka tremarono per l'agitazione quando rispose: 

- Nel cappello non c'è quasi differenza, ma nel gambo sì. 

E non appena queste parole furono dette, lui e lei capirono che tutto era finito, che quello che si doveva dire non si sarebbe detto, e la loro agitazione, che prima di questo aveva raggiunto l'acme, cominciò a placarsi. 

- Il fungo prugnolo ricorda nel gambo la barba non rasata di due giorni di un uomo bruno - disse, ormai calmo, Sergej Ivanovic. 

- Già, è vero - rispose sorridendo Varen'ka, e involontariamente la direzione della loro passeggiata cambiò. Cominciarono ad avvicinarsi ai bambini. Varen'ka provava pena e vergogna, ma nello stesso tempo anche un senso di sollievo. 

Ritornando a casa ed esaminando tutti gli argomenti, Sergej Ivanovic scoprì che non aveva ragionato in modo giusto. Non poteva tradire la memoria di Marie. 

- Piano, bambini, piano! - gridò perfino sgarbato Levin, ponendosi davanti alla moglie per difenderla, quando la folla dei bambini volò loro incontro con un grido di gioia. 

Dopo i bambini, uscirono dal bosco Sergej Ivanovic e Varen'ka. Kitty non ebbe bisogno di interrogare Varen'ka: dall'espressione calma e un po' vergognosa di tutti e due i visi capì che i suoi programmi non si erano avverati. 

- Su, ebbene? - le domandò il marito nel tornare di nuovo a casa. 

- Non attacca - disse Kitty, rassomigliando al padre nel sorriso e nel modo di parlare, cosa che Levin notava spesso in lei con piacere. 

- Come non attacca? 

- Ecco, così - disse lei, prendendo la mano del marito, portandola alla bocca e toccandola con le labbra chiuse. - Come si bacia la mano al vescovo. 

- E chi è che non attacca? - disse lui, ridendo. 

- Tutti e due. E deve esser così\ldots{} 

- Vengono i contadini\ldots{} 

- No, non hanno visto. 

\capitolo{VI}Durante il tè dei bambini, i grandi erano seduti sul balcone e conversavano come se nulla fosse accaduto, sebbene tutti, e in particolare Sergej Ivanovic e Varen'ka, sapessero molto bene che era accaduto un avvenimento sia pure negativo, ma molto importante. Tutti e due provavano una sensazione identica, simile a quella che prova uno scolaro dopo un esame andato male, per cui o deve ripetere o viene per sempre escluso dall'istituto. Tutti i presenti, sentendo pure che era successo qualcosa, parlavano animatamente di argomenti estranei. Levin e Kitty si sentivano particolarmente felici e pieni d'amore, quella sera. E il fatto che fossero felici del loro amore, racchiudeva in sé un'allusione spiacevole per coloro che sentivano la stessa cosa e non vi erano riusciti, e se ne vergognavano. 

- Ricordate le mie parole: Alexandre non verrà - disse la vecchia principessa. 

Quella sera si aspettava l'arrivo col treno di Stepan Arkad'ic e il vecchio principe scriveva che forse anche lui sarebbe venuto. 

- E io lo so perché - continuò la principessa - egli dice che i giovani sposi devono essere lasciati soli, nei primi tempi. 

- Ma anche così papà ci ha lasciato. Non l'abbiamo più visto - disse Kitty. - E che giovani siamo mai? Siamo già così vecchi! 

- Se lui non verrà, vi saluterò anch'io, ragazzi - disse la principessa, dopo aver sospirato con tristezza. 

- Ma cosa avete, mamma! - l'investirono le due figlie insieme. 

- Ma pensa, chi sa come sta? Del resto ora\ldots{} 

E a un tratto, del tutto inaspettatamente, la voce della vecchia principessa tremò. Le figlie tacquero e si guardarono: ``Mamma ha sempre qualcosa di triste'' si dissero con lo sguardo. Non sapevano che, per quanto bene si trovasse la principessa dalla figlia, per quanto necessaria vi si sentisse, provava una tormentosa tristezza e per sé e per il marito da che avevano maritato l'ultima figlia prediletta e il nido familiare era rimasto vuoto. 

- Di che cosa avete bisogno, Agaf'ja Michajlovna? - chiese a un tratto Kitty ad Agaf'ja Michajlovna che si era fermata con un'aria misteriosa e un viso significativo. 

- Per la cena. 
\enlargethispage*{1\baselineskip}

- Sì, ecco, va benissimo - disse Dolly - tu va' a dare gli ordini, e io andrò con Griša a risentirgli la lezione. Se no, quest'oggi, non ha fatto nulla. 

- Ma questa lezione è per me! No, Dolly, andrò io - disse Levin, scattando. 

Griša, che era entrato al ginnasio, d'estate doveva ripetere le lezioni. Dar'ja Aleksandrovna, che già a Mosca aveva studiato insieme col figlio il latino, venuta dai Levin s'era imposta come regola di ripetere con lui, almeno una volta al giorno, le materie più difficili, aritmetica e latino. Levin si era offerto di sostituirla, ma la madre, ascoltando una volta la lezione di Levin e notando che non era simile alla ripetizione dell'insegnante di Mosca, confusamente e cercando di non offendere Levin, gli aveva detto con risolutezza che bisognava studiare secondo il testo, come faceva l'insegnante, e che piuttosto l'avrebbe fatto di nuovo lei. Levin era urtato e contro Stepan Arkad'ic, perché per la sua incuria non lui, ma la madre si occupava di sorvegliare gli studi di cui non capiva nulla, e contro i professori che insegnavano così male ai ragazzi; ma aveva promesso alla cognata di far lezione così come voleva lei. E aveva continuato a studiare con Griša non più secondo il proprio metodo, ma secondo il testo, e perciò senza voglia, dimenticando spesso l'ora della lezione. Così era stato anche quel giorno. 

- No, vado io, Dolly, e tu resta a sedere - disse Levin. - Faremo tutto bene, secondo il libro. Soltanto quando arriverà Stiva, andremo a caccia, e allora salteremo le lezioni. - E Levin andò da Griša. 

Lo stesso disse Varen'ka a Kitty. Varen'ka anche nella casa felice, ben organizzata dei Levin, aveva saputo rendersi utile. 

- Ordinerò io la cena, e voi restate a sedere - disse e si alzò per andare da Agaf'ja Michajlovna. 

- Già, forse non hanno trovato i polli. Allora dei nostri\ldots{} - disse Kitty. 

- Ne ragioneremo con Agaf'ja Michajlovna - e Varen'ka sparve con lei. 

- Che cara ragazza! - disse la principessa. 

- Non cara, maman, ma una tale delizia, come non ce n'è al mondo. 

- Allora, quest'oggi aspettate Stepan Arkad'ic - disse Sergej Ivanovic, non desiderando evidentemente continuare il discorso su Varen'ka. - È difficile trovare due cognati meno somiglianti fra di loro - disse con un sorriso fine: - uno, mobile, che vive soltanto in società come un pesce nell'acqua; l'altro, il nostro Kostja, vivace, svelto, sensibile a tutto, ma che appena è in società, si gela o si dibatte insensatamente, come un pesce tirato a secco. 

- Già, è proprio insensato - disse la principessa, rivolgendosi a Sergej Ivanovic. - Proprio di questo volevo pregarvi, di dirgli che per lei - e indicò Kitty - è impossibile restare qui, è assolutamente necessario andare a Mosca, invece. Lui dice che farà venire il dottore\ldots{} 

- Maman, egli farà tutto, è d'accordo su tutto - disse Kitty, con stizza verso la madre che aveva chiamato a giudice, in questa faccenda, Sergej Ivanovic. 

Nel più bello della loro conversazione si sentì per il viale uno sbuffar di cavalli e un rumore di ruote sulla ghiaia. 

Dolly non aveva ancora fatto in tempo ad alzarsi per andare incontro al marito, che di sotto, dalla finestra dove studiava Griša, saltò fuori Levin e fece scendere Griša. 

- È Stiva! - gridò Levin da sotto al balcone. - Abbiamo finito, Dolly, non aver paura! - soggiunse e si mise a correre incontro alla carrozza come un ragazzo. 

- Is, ea, id, eius, eius, eius! - gridava Griša, saltellando per il viale. 
%\enlargethispage*{1\baselineskip}

- E qualcun altro ancora. Forse papà! - gridò Levin fermandosi all'ingresso del viale. - Kitty, non andare per la scala ripida, fa' il giro. 

Ma Levin s'era sbagliato nello scambiare la persona seduta in carrozza per il vecchio principe. Quando si fu accostato alla carrozza, vide, accanto a Stepan Arkad'ic, non già il principe, ma un bel giovane robusto, con un berretto irlandese con due lunghi nastri dietro. Era Vasen'ka Veslovskij, cugino in terzo grado degli Šcerbackij, giovane elegante di Pietroburgo e di Mosca, ``ottimo ragazzo e appassionato cacciatore'', come lo presentò Stepan Arkad'ic. 

Per nulla sconcertato dalla delusione provocata per essersi sostituito al vecchio principe, Veslovskij salutò allegramente Levin, ricordando la sua conoscenza di un tempo, e preso Griša in carrozza, lo fece passare di là dal pointer che Stepan Arkad'ic aveva portato con sé. 

Levin non montò in carrozza, ma tenne dietro. Era urtato perché non era venuto il vecchio principe, cui voleva tanto più bene quanto più lo imparava a conoscere, e perché era apparso Vasen'ka Veslovskij, persona del tutto estranea e superflua. Gli apparve ancor più estraneo e superfluo quando, avvicinatosi alla scalinata, dove s'era riunita tutta l'animata compagnia dei grandi e dei piccoli, vide che Vasen'ka Veslovskij baciava la mano a Kitty con un'aria particolarmente affabile e galante. 

- E noi siamo cousins con vostra moglie, e anche vecchi amici - disse Vasen'ka Veslovskij, stringendo di nuovo fortemente la mano di Levin. 

- Be', com'è la caccia? - chiese a Levin Stepan Arkad'ic che aveva appena fatto in tempo a salutare tutti. - Ecco, io e lui abbiamo le intenzioni più feroci. Ma come, maman, da allora non sono ancora andati a Mosca! Ecco, Tanja, questo è per te! Tiralo fuori dalla vettura, di dietro - diceva da tutte le parti. 

- Come ti sei ringiovanita, Dollen'ka! - diceva alla moglie, baciandole ancora una volta la mano, trattenendola nella propria, e dandovi sopra dei colpetti con l'altra. 

Levin, che un minuto prima era nella più amena disposizione d'animo, adesso guardava tutti torvo, e niente più gli piaceva. 

``Chi sa chi ha baciato ieri con queste labbra!'' pensava, guardando le tenerezze di Stepan Arkad'ic con la moglie. Guardò Dolly e neanche lei gli piacque. 

``Lei non crede mica al suo amore. Allora perché mai è così contenta? È disgustosa!'' pensava Levin. 

Guardò la principessa, che un minuto prima era stata così cara, e non gli piacque il modo con cui salutava e accoglieva in casa sua questo Vasen'ka con i suoi nastri. 

Perfino Sergej Ivanovic, uscito anche lui sulla scalinata, non gli riuscì gradito per quella finta benevolenza con cui accolse Stepan Arkad'ic, quando egli sapeva che suo fratello non amava Oblonskij e non lo stimava. 

E Varen'ka, anche lei, gli era antipatica per il suo modo di far conoscenza, con quell'aria di sainte nitouche, con quel signore, quando invece pensava soltanto al modo di prendere marito. 
%\enlargethispage*{1\baselineskip}

E più antipatica di tutti gli era Kitty, per la maniera con la quale si era sottomessa al tono di allegria con cui quel signore considerava il proprio arrivo in campagna come una festa per sé e per gli altri, e gli era particolarmente sgradita per quello speciale sorriso con cui rispondeva ai sorrisi di lui. 

Discorrendo rumorosamente, entrarono in casa; ma non appena tutti si furono seduti, Levin si voltò e uscì. 

Kitty si accorse che qualcosa era accaduto al marito. Voleva trovare un momento per parlargli a solo, ma egli si affrettò ad allontanarsi da lei, dicendo che doveva passare in amministrazione. Da tempo gli affari dell'azienda non gli apparivano così importanti come quel giorno. ``Per loro è sempre festa - pensava - ma qui ci sono gli affari tutt'altro che festivi, che non aspettano indugio e senza i quali non si può vivere''. 

\capitolo{VII}Levin tornò a casa solo quando lo mandarono a chiamare per la cena. Sulla scala stavano Kitty e Agaf'ja Michajlovna che si consigliavano sui vini per la cena. 

- Ma perché fate un tale fuss? Servite quello d'ogni giorno. 

- No, Stiva non lo beve\ldots{} Kostja, aspetta, che t'è successo? - disse Kitty, seguendolo, ma egli se ne andò senza aspettarla, inesorabile, a grandi passi, in sala da pranzo ed entrò subito nell'animata conversazione generale sostenuta da Vasen'ka Veslovskij e Stepan Arkad'ic. 

- Su, dunque, domani andiamo a caccia? - disse Stepan Arkad'ic. 

- Andiamo, vi prego - disse Veslovskij, mettendosi a sedere di fianco e ripiegando sotto di sé una delle sue gambe grasse. 

- Io sono contentissimo, andiamo. E voi siete già andato a caccia quest'anno? - chiese Levin a Veslovskij, esaminando attentamente la gamba di lui, ma con quella falsa affabilità che Kitty conosceva così bene e che gli si addiceva così poco. - Beccacce bottaie non so se ne troveremo, ma di beccaccini ce n'è tanti. Solo che bisogna mettersi in cammino presto. Non vi stancherete? Non sei stanco, Stiva? 

- Io stanco? Non sono ancora mai stato stanco. Vogliamo non dormire tutta la notte? Andiamo a passeggio. 

- Davvero vogliamo non dormire? benissimo! - confermò Veslovskij. 

- Oh, di questo siamo sicuri, che tu possa non dormire e non lasciar dormire gli altri - disse Dolly al marito con quell'ironia appena percettibile con cui adesso trattava quasi sempre il marito. - E secondo me, adesso, è già ora\ldots{} io vado, non ceno. 

- No, rimani a sedere, Dollen'ka - disse Stepan Arkad'ic, passando dalla sua parte alla tavola grande alla quale cenavano. - Ti racconterò ancora tante cose. 

- Probabilmente, nulla. 

- E sai, Veslovskij è stato da Anna. E va di nuovo da loro. Perché sono a settanta verste da voi, e io pure ci andrò certamente. Veslovskij, vieni qua. 

Vasen'ka passò accanto alle signore e si sedette vicino a Kitty 

- Ah, raccontate per favore, siete stato da lei? come sta? - gli si rivolse Dar'ja Aleksandrovna. 

Levin era rimasto all'altra estremità della tavola e, senza smettere di parlare con la principessa e con Varen'ka, vedeva che fra Stepan Arkad'ic, Dolly, Kitty e Veslovskij si svolgeva un'animata, misteriosa conversazione. Non solo c'era una conversazione misteriosa, ma egli scorgeva sul viso di sua moglie l'espressione di un sentimento serio, mentre guardava senza abbassare gli occhi, il bel viso di Vasen'ka che raccontava con animazione qualcosa. 

- Da loro si sta proprio bene - raccontava Veslovskij di Vronskij e Anna. - Io, naturalmente, non voglio giudicare, ma in casa loro ci si sente come in famiglia. 

- Che cosa pensano di fare? 

- Pare che per l'inverno vogliano andare a Mosca. 

- Come sarebbe bene andare insieme da loro! Tu quando vai? - chiese Stepan Arkad'ic a Vasen'ka. 

- Passerò da loro il mese di luglio. 

- E tu andrai? - si rivolse Stepan Arkad'ic alla moglie. 

- È tanto tempo che voglio andare e ci andrò certamente - disse Dolly. - Lei mi fa pena, io la conosco. È una carissima donna. Andrò da sola, quando tu sarai già andato via, e così non darò fastidio a nessuno. Anzi, è meglio senza di te. 

- E va bene - disse Stepan Arkad'ic. - E tu Kitty? 

- Io? e perché dovrei andare? - disse Kitty, infiammandosi tutta. E si voltò a guardare il marito. 

- Ma voi conoscete Anna Arkad'evna? È una donna molto interessante. 

- Sì - rispose a Veslovskij, arrossendo ancora di più; poi si alzò e si accostò al marito. 

- Allora domani vai a caccia? - ella disse. 

La gelosia di Levin in quei pochi momenti, in particolare a causa del rossore che aveva ricoperto le guance di Kitty mentre parlava con Veslovskij, era già arrivata lontano. Adesso, sentendo le parole di lei, le intendeva a modo suo. Per quanto in seguito gli fosse strano ricordare questo, ora gli appariva chiaro che, nel chiedergli se andava a caccia, il fatto interessava lei solo per sapere se egli avrebbe procurato questo piacere a Vasen'ka Veslovskij, del quale, secondo lui, ella era già innamorata. 

- Sì, andrò - rispose con voce non naturale, antipatica a lui stesso. 

- Ma è meglio che domani restiate qui per un giorno, altrimenti Dolly non lo vede proprio suo marito; e che andiate dopodomani - disse Kitty. 

Ma il senso delle parole di Kitty si era già trasformato così in Levin: ``Non mi separare da lui. Che tu parta per me è indifferente, ma lasciami godere la compagnia di questo delizioso giovane'' 

- Ah, se vuoi domani rimarremo - rispose Levin, con gentilezza accentuata. 

Vasen'ka intanto, non sospettando per nulla la sofferenza che causava la sua presenza, si alzò da tavola dopo Kitty e, seguendola con uno sguardo sorridente, affabile, le tenne dietro. 

Levin aveva visto quello sguardo. Impallidì e per un attimo non riuscì a respirare. ``Ma come si permette di guardare mia moglie così!'' gli ribolliva dentro. 

- Allora domani? Andiamo, per favore? - disse Vasen'ka, sedendosi su di una sedia e piegando di nuovo una gamba secondo la sua abitudine. 

La gelosia di Levin si spinse ancora oltre. Si vedeva già un marito ingannato, di cui la moglie e l'amante avevano bisogno solo perché procurasse loro gli agi e gli svaghi della vita. 

Tuttavia, malgrado questo, interrogava con cortese ospitalità Vasen'ka sulle sue cacce, sul fucile, sugli stivali e acconsentì a partire l'indomani. 

Per fortuna di Levin, la vecchia principessa pose fine alle sue sofferenze con l'alzarsi lei stessa e consigliare a Kitty di andare a letto. Ma anche qui la cosa non fu senza pena per Levin. Nel salutare la padrona di casa, Vasen'ka voleva di nuovo baciarle la mano, ma Kitty, arrossendo, con un'ingenua scortesia di cui la principessa la rimproverò, disse, sottraendo la mano: 

- Questo da noi non si usa. 

Agli occhi di Levin ella era colpevole di aver ammesso simili rapporti, e ancor più colpevole di aver mostrato, così goffamente, che non le piacevano. 

- Via, che gusto c'è a dormire! - disse Stepan Arkad'ic, che, dopo parecchi bicchieri di vino bevuti a tavola era adesso dell'umore più cordiale e poetico. - Guarda, Kitty - diceva, indicando la luna che si levava di là dai tigli - che incanto! Veslovskij, ecco, ora ci vorrebbe una serenata. Sai, ha una bella voce, io e lui abbiamo cantato insieme in viaggio. Ha portato con sé delle bellissime romanze, due nuove. Sarebbe bene cantare con Varvara Andreevna. 

Quando tutti si lasciarono, Stepan Arkad'ic passeggiò ancora a lungo con Veslovskij per il viale, e si sentivano le loro voci che intonavano la nuova romanza. 

Levin, accigliato, stava seduto, ascoltando queste voci, su di una poltrona nella camera della moglie, e taceva ostinatamente alle domande di lei su cosa gli fosse accaduto; ma quando alla fine lei stessa, sorridendo timida, gli chiese: ``Forse non t'è piaciuto qualcosa in Veslovskij?'' egli proruppe e disse tutto. Ma quello che diceva lo umiliava e perciò ancor più si irritava. Stava in piedi dinanzi a lei, con gli occhi paurosamente scintillanti di sotto alle ciglia aggrottate, e stringeva al petto le mani forti, come volesse tendere tutte le proprie forze per contenersi. L'espressione del viso sarebbe stata severa e perfino crudele, se non avesse espresso nello stesso tempo una sofferenza che commoveva lei. Gli zigomi gli tremavano e la voce si spezzava. 

- Tu devi capire che io non sono geloso: è una parola disprezzabile. Non posso esser geloso e credere che\ldots{} Non posso dire quello che sento, ma ciò è terribile\ldots{} Non sono geloso, ma sono umiliato, offeso dal fatto che qualcuno osi guardarti con occhi così\ldots{} 

- Ma quali occhi? - diceva Kitty, cercando di ricordare, il più scrupolosamente possibile, tutti i discorsi e i gesti di quella serata e tutte le sfumature. 

In fondo all'anima ella riteneva che ci fosse stato qualcosa proprio nel momento in cui Levin era andato a sedersi dietro di lei all'altra estremità della tavola, ma non osava confessarlo neanche a se stessa, e tanto meno dirlo a lui, rendendogli più forte la sofferenza. 

- E che cosa può esserci di attraente in me, così come sono? 

- Ah - gridò lui, mettendosi le mani nei capelli. - Sarebbe meglio che non parlassi!\ldots{} Vuol dire che se fossi attraente\ldots{} 

- Ma no, Kostja, aspetta, ascolta! - diceva lei, guardandolo con un'espressione tormentata e pietosa. - Su, cosa mai puoi pensare? Quando per me non c'è nessuno, nessuno, nessuno\ldots{} Su, vuoi che io non veda nessuno? 

Nel primo momento la gelosia di lui le era parsa offensiva; era irritata che la più piccola distrazione, la più innocente, le venisse preclusa; ma ora, in quel momento, avrebbe sacrificato volentieri non solo sciocchezze simili, ma qualsiasi cosa per la tranquillità di lui, per liberarlo dalla sofferenza che provava. 

- Tu devi capire l'orrore e la comicità della mia situazione - continuava lui con un sussurro disperato - che lui è in casa mia, che in particolare non ha fatto nulla di sconveniente, oltre quella sua disinvoltura e quell'accavallare le gambe\ldots{} Lui ritiene questo del miglior tono, e perciò devo essere cortese con lui. 

- Ma, Kostja, tu esageri! - diceva Kitty, rallegrandosi in fondo all'animo di quella forza d'amore verso di lei che adesso si esprimeva nella gelosia. 

- La cosa più orribile è che tu sei come sempre, e adesso, quando tu sei una cosa tanto sacra per me, e noi siamo così felici, a un tratto un simile fango\ldots{} Ma non fango, perché lo insulto? Lui non mi riguarda. Ma perché la mia, la tua felicità\ldots{} 

- Sai, capisco come ciò sia avvenuto - cominciò Kitty. 

- Perché, perché? 

- Ho visto come guardavi quando parlavamo a cena. 

- Sì, sì - disse Levin spaventato. 

Ella gli raccontò di che cosa avevano parlato. E, raccontando questo, le veniva meno il respiro per l'agitazione. Levin stette un po' in silenzio, poi le esaminò il viso pallido, spaventato, e a un tratto si prese la testa fra le mani. 

- Katja, ti ho tormentata! Amore mio, perdonami! È una follia! Katja, sono proprio colpevole. E si poteva tormentarsi tanto per una sciocchezza simile? 

- No, mi fai pena tu. 

- Io? Io? Cosa sono io, un pazzo!\ldots{} Ma tu perché? È orribile pensare che una qualsiasi persona estranea possa turbare la nostra felicità. 

- Certo, questo è proprio offensivo. 

- No, allora io, al contrario, lo lascerò stare apposta da noi tutta l'estate e lo colmerò di gentilezze - diceva Levin, baciandole le mani. - Ecco, vedrai. Domani\ldots{} Sì, è vero, domani andremo via. 

\capitolo{VIII}Il giorno dopo, le signore non s'erano ancora alzate, e gli equipaggi per la caccia, un calesse e un carro, erano fermi dinanzi all'ingresso. Laska, che fin dal mattino aveva capito che si sarebbe usciti a caccia, dopo aver mugolato e saltato a sazietà, sedeva a cassetta accanto al cocchiere, guardando, agitata e scontenta del ritardo, la porta dalla quale dovevano ancora uscire i cacciatori. Per primo uscì Vasen'ka Veslovskij, con grandi stivali nuovi che gli arrivavano fino a metà delle cosce grasse, un camiciotto verde, la cartuccera nuova che odorava di pelle, il berrettino coi nastri e un fucile inglese nuovo fiammante, senza ganci e senza cinghia. Laska gli andò incontro, lo salutò, saltando, gli chiese a modo suo se sarebbero usciti presto gli altri, ma, non avendone ricevuta risposta, s'accucciò al suo posto d'attesa e di nuovo trattenne il respiro con la testa girata da un lato e un orecchio teso. Alla fine la porta si aprì con fracasso, ne corse fuori, girando e voltando il muso all'aria, Krak, il pointer pezzato di giallo di Stepan Arkad'ic, e uscì lo stesso Stepan Arkad'ic col fucile in mano e un sigaro in bocca. ``Tout beau, tout beau Krak!'' egli gridava, carezzando il cane che gli poneva le zampe sul ventre e sul petto, impigliandosi con esse nel carniere. Stepan Arkad'ic aveva calzature di cuoio d'un sol pezzo, fasce, pantaloni sdruciti e cappotto corto. In testa aveva l'avanzo di un cappello, ma il fucile di nuovo tipo era un gioiello e il carniere e la cartuccera, benché consunti, erano di ottima qualità. 

Vasen'ka Veslovskij non aveva conosciuto, fino a quel momento, una simile vera eleganza venatoria, essere, cioè, rivestito di cenci, ma possedere gli attrezzi di caccia della qualità migliore. Lo capì ora, guardando Stepan Arkad'ic che, fra quei cenci, splendeva nella figura elegante, signorile, ben curata e allegra, e decise che per la prossima caccia si sarebbe assolutamente equipaggiato in una maniera simile. 

- Be', e il nostro padron di casa che fa? - gli domandò. 

- La moglie giovane - disse sorridendo, Stepan Arkad'ic. 

- Già, e così deliziosa. 

- Era già vestito. Probabilmente sarà corso di nuovo da lei. 

Stepan Arkad'ic aveva indovinato. Levin era corso di nuovo dalla moglie a domandarle ancora una volta se lo aveva perdonato per la sciocchezza del giorno prima, e ancora per pregarla d'essere prudente, in nome di Dio. Soprattutto che stesse lontana dai bambini: loro potevano sempre darle un urtone. Poi bisognò ancora una volta ricevere da lei conferma che non era arrabbiata con lui perché se ne andava via per due gironi, e ancora pregarla di mandargli assolutamente un biglietto l'indomani mattina a mezzo di un uomo a cavallo, di scriver magari solo due parole tanto perché egli potesse sapere che stava bene. 

Per Kitty, come sempre, era doloroso separarsi dal marito per due giorni, ma, vista la figura di lui animata, che sembrava ancor più grande e forte con gli stivaloni e il camiciotto bianco, e una certa luce negli occhi, per lei incomprensibile, dovuta alla eccitazione della caccia, per questa sua gioia dimenticò il proprio cruccio e lo congedò allegra. 

- Perdonate, signori! - egli disse, venendo di corsa sulla scala. - La colazione l'hanno messa dentro? Perché il sauro a destra? Be', fa lo stesso. Laska, lascia, a cuccia! 

- Lasciali andare nell'armento giovane - disse rivolto al bovaro che lo aveva aspettato accanto alla scalinata per chiedergli dei torelli. - Perdonate, ecco che arriva un altro sciagurato. 

Levin saltò giù dal calesse, dove già aveva preso posto, incontro al legnaiuolo imprenditore, che veniva verso la scala di ingresso con una sazen' in mano. 

- Ecco, ieri non sei venuto in amministrazione e ora mi fai perder tempo. Che c'è? 

- Ordinate di fare ancora un giro. Si devono aggiungere tre scalini. E ce li metteremo proprio bene. Sarà molto più comoda. 

- Avresti dovuto ascoltarmi - riprese Levin con stizza. - Ti avevo detto: metti a posto le pareti di sostegno e poi incastra gli scalini. Adesso non potrai più correggere. Fa' come t'ho detto; tagliane una nuova. 

Il fatto era che, nell'ala del fabbricato che si stava costruendo, l'imprenditore aveva sciupato la scala, tagliandola a parte e senza calcolare la pendenza, in modo che tutti gli scalini erano risultati inclinati, quando l'avevano messa a posto. Adesso, l'imprenditore voleva lasciar la stessa scala e aggiungervi tre scalini. 

- Sarà molto meglio. 

- Ma dove ti uscirà mai coi tre scalini? 

- Di grazia, signore - disse l'imprenditore con un sorriso sprezzante\ldots{} - Uscirà proprio a misura. Così, cioè, verrà fuori di giù - disse con un gesto convincente - e andrà, andrà su e arriverà. 

- Ma tre scalini sposteranno la lunghezza\ldots{} E dove arriverà? 

- Ma, come andrà giù, così pure arriverà, s'intende - diceva con insistenza e persuasione l'imprenditore. 

- Già, sotto al soffitto arriverà, e nel muro. 

- Vi prego. Ecco, comincerà di giù. Andrà, andrà su e arriverà. 

Levin tirò fuori la bacchetta del fucile e cominciò a disegnare una scala sulla polvere. 

- Su, vedi? 

- Come ordinate voi - disse il falegname mentre a un tratto gli si rischiaravano gli occhi perché, evidentemente, aveva capito. - Si vede che bisogna tagliarne un'altra. 

- E già, fa' proprio come t'è stato ordinato - gridò Levin, sedendosi in calesse. - Via! Tieni i cani, Filipp. 

Levin adesso, lasciate dietro di sé tutte le preoccupazioni familiari e amministrative, sentiva un così forte senso di gioia di vivere e di attesa, che non aveva voglia di parlare. Inoltre provava quel senso di agitazione che prova ogni cacciatore, avvicinandosi al luogo della caccia Se qualcosa ancora lo occupava, adesso, erano solo dei particolari: se avrebbero trovato qualcosa nella palude di Kolpen, come si sarebbe comportata Laska in confronto di Krak, e come lui stesso quel giorno sarebbe riuscito a tirare. E se avesse fatto una brutta figura dinanzi a una persona nuova? E se Oblonskij l'avesse superato nel tiro? pure questo gli veniva in mente. 

Oblonskij provava una sensazione simile, ed era poco loquace anche lui. Solo Vasen'ka Veslovskij parlava allegramente, senza interruzione. Adesso, ascoltandolo, Levin si vergognava di ricordare come fosse stato ingiusto con lui il giorno innanzi. Vasen'ka era veramente un buon ragazzo, semplice, cordiale e allegro. Se Levin l'avesse conosciuto da scapolo, si sarebbe certo legato a lui. A Levin spiaceva solo un po' quel suo modo ozioso di veder la vita e quella sua certa disinvolta eleganza. Come se si riconoscesse un alto, indiscusso valore, perché aveva le unghie lunghe e il berrettino e tutto il resto bene assortito; ma questo gli si poteva perdonare in cambio della cordialità e della finezza. Piaceva a Levin per la sua buona educazione, per l'ottima pronuncia del suo francese e del suo inglese e perché era un uomo del suo mondo. 

A Vasen'ka piaceva straordinariamente il cavallo di steppa del Don che era al bilancino sinistro. Continuamente se ne estasiava. 

- Com'è bello galoppare per la steppa su di un cavallo da steppa. Eh? Non è vero? - egli diceva. 

Nel montare un cavallo da steppa, s'immaginava qualcosa di selvaggio, di poetico, da cui non veniva fuori nulla; ma la sua ingenuità, unita alla sua bellezza, al sorriso cordiale e alla grazia dei suoi movimenti, era molto attraente. Forse perché la sua natura era simpatica a Levin, o perché Levin cercava di espiare la colpa del giorno prima giudicando tutto buono in lui, certo è che gli faceva piacere stare con lui. 

Allontanatisi di tre verste, Veslovskij a un tratto s'accorse che non aveva i sigari e il portafoglio, e non sapeva se li avesse perduti o lasciati sul tavolo. Nel portafoglio c'erano trecentosettanta rubli e perciò non si poteva lasciar perdere. 

- Sapete cosa, Levin? Faccio una galoppata fino a casa su questo cavallo del Don del bilancino. Sarà un'ottima cosa. Eh? - egli diceva, preparandosi già a montare. 

- Ma no, perché - rispose Levin che calcolava che Vasen'ka non doveva pesar certo meno di sei pudy. - Manderò il cocchiere. 

Il cocchiere montò sul cavallo del bilancino e Levin si mise a guidar lui stesso la pariglia. 

\capitolo{IX}-Be', qual è il nostro itinerario? Diccelo ben bene - disse Stepan Arkad'ic. \\
- Il programma è il seguente: per ora andiamo fino a Gvozdëv. A Gvozdëv c'è una palude da beccacce bottaie, da questa parte, e dietro a Gvozdëv ci sono delle meravigliose paludi da beccacce, e ci sono anche le bottaie. Adesso fa caldo, e noi verso sera (sono venti verste) arriveremo e faremo la caccia della sera: poi pernotteremo e domani andremo alle paludi grandi. 

- E per strada non c'è nulla? 

- C'è qualcosa; ma perderemmo del tempo, e poi fa caldo. Ci sono due bei piccoli passi ma è difficile che ci sia qualcosa. 

A Levin stesso era venuta voglia di fermarsi ai due passi; ma questi erano vicini a casa e ci poteva sempre andare, ed erano tanto stretti che per tre non c'era spazio per tirare. E perciò mancava di sincerità nel dire che era difficile che ci fosse qualcosa. Giunti all'altezza di una piccola palude, Levin voleva passare oltre, senza fermarsi, ma l'occhio esperto di Stepan Arkad'ic vide subito l'acquitrino che si scorgeva dalla strada. 

- Non ci passiamo? - disse, indicando la piccola palude. 

- Levin, per favore, ma è una cosa magnifica! - cominciò a pregare Vasen'ka Veslovskij, e Levin non poté non acconsentire. 

Non fecero in tempo a fermarsi che i cani, sorpassandosi l'un l'altro, volavano già verso la palude. 

- Krak! Laska! 

I cani tornarono. 

- In tre staremo stretti. Io rimarrò qui - disse Levin, sperando che non avrebbero trovato niente all'infuori delle pavoncelle che, levatesi a causa dei cani e dondolandosi in volo, piangevano lamentosamente sopra la palude, 

- No, andiamo; Levin, andiamo insieme! - chiamò Veslovskij. 

- Davvero staremo stretti. Laska, indietro! Laska! Non avete bisogno di un altro cane? 

Levin rimase accanto al calesse e guardava con invidia i cacciatori che attraversavano tutta la piccola palude. All'infuori di una gallinella e delle pavoncelle, di cui Vasen'ka ne uccise una, non c'era altro. 

- Su, ecco, come vedete, non ne valeva la pena - disse Levin - si perde solo tempo. 

- Tuttavia è piacevole. Avete visto? - diceva Vasen'ka Veslovskij, montando poco destramente sul calesse col fucile e la pavoncella in mano. - Come l'ho ammazzata bene questa! Non è vero? Via, arriveremo presto al passo vero? 

D'un tratto i cavalli si lanciarono in avanti, Levin batté col capo contro la canna del fucile di qualcuno ed echeggiò uno sparo. Lo sparo era echeggiato prima, almeno così parve a Levin. Il fatto era che Vasen'ka Veslovskij, nell'abbassare i cani, aveva premuto un grilletto e trattenuto l'altro. La cartuccia si conficcò nel terreno, senza far male a nessuno. Stepan Arkad'ic scosse il capo e rise con aria di rimprovero verso Veslovskij, ma Levin non aveva il coraggio di sgridarlo. In primo luogo, qualsiasi rimprovero sarebbe sembrato provocato dallo scampato pericolo e dal bernoccolo che gli era spuntato sulla fronte; in secondo luogo, Veslovskij fu prima così ingenuamente addolorato e poi così cordialmente e piacevolmente esilarato dalla loro comune confusione, che non poté non riderne lo stesso Levin. 

Quando si avvicinarono alla seconda palude, che era abbastanza grande e doveva prender molto tempo, Levin li esortò a non scendere, ma Veslovskij di nuovo lo pregò. Di nuovo da padrone ospitale, giacché la palude era stretta, Levin rimase accanto agli equipaggi. 

Appena arrivati, Krak, si lanciò verso le montagnole. Vasen'ka Veslovskij per primo corse dietro al cane. E Stepan Arkad'ic non fece in tempo ad avvicinarsi che era già volata fuori una beccaccia bottaia. Veslovskij fece padella, e la bottaia si posò su di un prato non falciato. La beccaccia fu lasciata a Veslovskij. Krak la scovò di nuovo, la puntò e Veslovskij l'uccise e tornò verso gli equipaggi. 

- Adesso andate voi e io resterò coi cavalli - egli disse. 

Levin cominciava ad essere tormentato dalla gelosia venatoria. Consegnò le redini a Veslovskij e si avviò nella palude. Laska, che già da tempo guaiva, protestando lamentosamente per l'ingiustizia, si lanciò in avanti, verso un gruppo di montagnole, che era noto a Levin e pieno di possibilità, perché Krak non vi era andato. 

- Come mai non la fermi? - gridò Stepan Arkad'ic . 

- Non la spaventerà - rispose Levin, compiacendosi del cane e affrettandosi a tenergli dietro. 

Nella ricerca, Laska, quanto più si avvicinava alle note montagnole, tanto più diventava seria. Un piccolo uccello di palude non la distrasse che per un attimo. Aveva fatto un giro intorno alle montagnole, ne cominciò un altro e a un tratto tremò tutta e s'irrigidì. 

- Va', va', Stiva! - gridò Levin, sentendo che il cuore gli cominciava a battere forte e che a un tratto, come se un chiavistello avesse aperto il suo udito teso, tutti i suoni, perduta la misura della distanza, cominciavano a colpirlo disordinatamente, ma con chiarezza. Sentiva i passi di Stepan Arkad'ic e gli parevano un lontano calpestio di cavalli; sentiva il suono dell'estremità della montagnola su cui era salito, che cedeva con le radici, e questo suono gli pareva il volo della beccaccia. Udiva pure dietro, non lontano, un certo ciangottio su per l'acqua di cui non riusciva a rendersi conto. 

Scegliendo il posto per poggiare il piede, si avvicinò al cane. 

- Pille! 

Non una beccaccia, ma una bottaia era sfuggita di sotto al cane. Levin imbracciò il fucile, ma nel momento in cui mirava, quello stesso ciangottio su per l'acqua si fece più forte, più vicino e si unì alla voce di Veslovskij che gridava in maniera strana e forte qualche cosa. Levin si accorse di mirar la beccaccia di dietro, e tuttavia tirò. Convinto d'aver fatto padella, Levin si voltò e vide che i cavalli col calesse non erano più sulla strada, ma nella palude. 

Veslovskij, per osservare il tiro, era entrato nella palude e aveva fatto impantanare i cavalli. 

``Che il diavolo se lo pigli!'' sbottò Levin fra sé, tornando verso il veicolo impantanato. 

- Perché vi siete mosso? - disse secco e, chiamato il cocchiere, si diede a liberare i cavalli. 

Levin era irritato che gli avessero disturbato il tiro, che gli avessero fatto impantanare i cavalli e, soprattutto, che, per liberare i cavalli, per staccarli, né Stepan Arkad'ic, né Veslovskij dessero una mano a lui e al cocchiere, poiché né l'uno né l'altro avevano la più piccola idea di cosa fosse il tiro a due. Senza rispondere neppure una parola a Vasen'ka che lo rassicurava che là era completamente asciutto, Levin lavorava in silenzio col cocchiere per liberare i cavalli. Ma poi, riscaldatosi nel lavoro e avendo visto con quanta buona volontà Veslovskij tirasse il calesse per un parafango, fin quasi a staccarlo, Levin si rimproverò di essere stato troppo freddo verso di lui, sotto l'influenza del sentimento del giorno innanzi, e cercò di riparare alla propria freddezza con una particolare affabilità. Quanto tutto fu messo in ordine e gli equipaggi furono ricondotti sulla strada, Levin ordinò di tirar fuori la colazione. 

- Bon appétit, bonne conscience! Ce poulet va tomber jusqu'au fond de mes bottes! - diceva con un motto francese Vasen'ka che, di nuovo allegro, mangiava un secondo pollastrino. - Via, adesso i nostri guai sono finiti; ora tutto andrà bene. Solo che io, in sconto del mio peccato, ho il dovere di sedere a cassetta. Non è vero? Eh no, no, io sono l'automedonte. Vedrete come vi porterò! - rispondeva, senza lasciare le redini, a Levin che lo pregava di lasciare fare al cocchiere. - no, io devo espiare la mia colpa, e sto benissimo a cassetta. - E partì. 

Levin temeva ch'egli avrebbe stancato i cavalli, specialmente quello di sinistra, il sauro, che non riusciva a tener bene; ma senza volere si sottometteva all'allegria di lui, ascoltava le romanze che Veslovskij, seduto a cassetta, cantò per tutta la strada, e i racconti e le rappresentazioni dialogate, e ancora su come bisognava guidare all'inglese, four in hand; e tutti, dopo colazione, nella più amena disposizione d'animo, giunsero alla palude di Gvozdëv. 

\capitolo{X}Vasen'ka guidò i cavalli così in fretta che giunsero alla palude troppo presto, sì che faceva ancora caldo. 

Avvicinandosi alla palude più grande, mèta prima del viaggio, Levin pensò involontariamente come liberarsi di Vasen'ka e procedere senza impedimenti. Stepan Arkad'ic, evidentemente desiderava anche lui questo, e sul suo viso Levin scorse l'espressione ansiosa che ha il cacciatore prima di cominciare la caccia, mista a una certa bonaria furberia che gli era propria. 

- E come andremo? La palude è magnifica, vedo, e ci sono gli avvoltoi - disse Stepan Arkad'ic, mostrando due grandi uccelli che volteggiavano sopra la càrice palustre. - Dove ci sono gli avvoltoi, c'è anche la selvaggina. 

- Su, ecco, vedete, signori - disse Levin con un'espressione alquanto torva, stringendosi gli stivali ed esaminando i pistoni del fucile. - Vedete questa càrice? - Egli indicò un isolotto, che sembrava scuro per le erbe nereggianti in un enorme prato bagnato, falciato a metà, che si stendeva sul lato destro del fiume. - La palude comincia qui, dritto dinanzi a voi, guardate, là dove è più verde. Di qua volta a destra, dove camminano i cavalli; là ci sono delle montagnole, di solito ci sono le bottaie, intorno a quella càrice fino a quegli ontani e fin proprio al mulino. Ecco, là, vedi, dove c'è l'ansa. È il luogo migliore. Là una volta ho ucciso diciassette beccacce. Ci divideremo coi due cani in direzione opposta e là accanto al mulino ci riuniremo. 

- Be', chi va a destra e chi a sinistra? - domandò Stepan Arkad'ic. - A destra è più largo, andate voi due, mentre io andrò a sinistra - disse con noncuranza. 

- Bene! Lo vinceremo nel tiro! Su, andiamo, andiamo, andiamo! - rincalzò Vasen'ka. 

Levin non poteva non acconsentire, ed essi si divisero. 

Erano appena entrati nella palude, che tutti e due i cani cominciarono a cercare insieme e si spinsero verso l'acqua rugginosa. Levin conosceva questo braccare di Laska, cauto e vago; conosceva anche il luogo e aspettava un piccolo stormo di beccacce. 

- Veslovskij, venitemi a fianco, a fianco! - disse, con voce smorzata, al compagno che sguazzava dietro nell'acqua e del quale gli interessava la direzione del fucile, dopo lo sparo casuale nella palude di Kolpen. 

- No, non voglio darvi fastidio, non vi curate di me. 

Ma Levin, senza volere, pensava e ricordava le parole di Kitty, quando lo aveva lasciato partire: ``Attenti a non ammazzarvi l'un l'altro''. I cani si avvicinavano sempre più evitandosi e seguendo ognuno la propria traccia; l'attesa di una beccaccia era così grande che il cigolar del proprio tacco, tirato fuori dall'acqua rugginosa, sembrava a Levin lo zirlio della beccaccia, ed egli afferrava e stringeva il calcio del fucile. 

- Pum, pum - gli echeggiò sopra l'orecchio. Vasen'ka aveva tirato a uno stormo di anatre che volteggiavano sopra la palude e che in quel momento erano volate fin troppo sopra ai cacciatori. Levin non fece in tempo a guardare, che una beccaccia zirlò, poi un'altra e una terza, e ancora otto se ne levarono l'una dietro l'altra. 

Stepan Arkad'ic ne uccise una proprio nel momento in cui si metteva a fare le sue volute e l'uccello cadde come una palla nel terreno paludoso. Oblonskij, senza perder tempo, ne mirò un'altra che volava ancora in basso verso la càrice e, contemporaneamente al suono dello sparo, anche questo uccello cadde, e lo si vide saltellar fuori dalla càrice falciata, sbattendo l'ala rimasta intatta, bianca di sotto. 

Levin non fu così fortunato: tirò alla prima beccaccia troppo dappresso e fece padella, la mirò quando aveva già cominciato a sollevarsi, ma in quel momento ne volò fuori un'altra di sotto ai suoi piedi e lo distrasse, ed egli fece padella un'altra volta. 

Mentre caricavano i fucili, ancora una beccaccia si alzò e Veslovskij, che aveva fatto in tempo a caricare un'altra volta, lasciò andare in acqua altre due cariche a pallini piccoli. Stepan Arkad'ic raccolse le sue beccacce e guardò Levin con occhi splendenti. 

- Su, ora ci separiamo - disse Stepan Arkad'ic e, zoppicando un po' con la gamba sinistra e tenendo il fucile pronto e fischiando al cane, andò da una parte. Levin e Veslovskij andarono dall'altra. 

A Levin capitava sempre che, se falliva i primi colpi, si accalorava, si irritava e sparava male per tutta la giornata. Così fu anche quel giorno. Di beccacce evidentemente ce n'erano molte. Di sotto al cane, di sotto ai piedi dei cacciatori ne volavano continuamente, e Levin avrebbe potuto riprendersi; ma quanto più sparava tanto più faceva una brutta figura dinanzi a Veslovskij, che tirava allegramente bene o male, senza colpir nulla e senza per nulla confondersi. Levin si affannava, non resisteva, si scalmanava sempre più ed era giunto ormai al punto che, tirando, non sperava quasi più di colpire. Anche Laska sembrava capire questo. Aveva cominciato a cercare più pigra e si voltava a guardare i cacciatori, perplessa e contrariata. Gli spari seguivano agli spari. Il fumo della polvere era intorno ai cacciatori e nel grande, spazioso carniere c'erano soltanto tre beccacce leggere, piccole. Di queste, una era stata uccisa da Veslovskij e un'altra da tutti e due insieme. Intanto, dall'altra parte della palude, si sentivano gli spari non frequenti ma, così pareva a Levin, sostanziosi di Stepan Arkad'ic e quasi dopo ogni sparo si sentiva: ``Krak, Krak porta qua!''. 

Questo agitava ancor più Levin. Le beccacce volavano senza posa nell'aria sopra la càrice. Lo zirlio raso terra e il gracidare alto si sentivano da ogni parte continuamente; le beccacce, fatte alzar prima e poi sospese nell'aria, si posavano davanti ai cacciatori. Invece di due avvoltoi, adesso, ne volavano decine, stridendo al di sopra della palude. 

Oltrepassata la parte più grande della palude, Levin e Veslovskij si spinsero là dove un prato di contadini era diviso a lunghe strisce fiancheggiate dalla càrice, segnato qua da strisce calpestate, là da un piccolo tratto falciato. Una metà di queste strisce era già falciata. 

Sebbene per il tratto non falciato ci fosse poca speranza di trovare altrettanta caccia quanto per il tratto falciato, Levin aveva promesso a Stepan Arkad'ic di riunirsi con lui e andò innanzi col compagno, per le strisce falciate e per quelle non falciate. 

- Ehi, cacciatori - gridò loro uno dei contadini seduto presso un carro staccato - venite a far la siesta con noi! c'è da bere il vino! 

Levin si voltò a guardare. 

- Venite, via! - gridò un allegro contadino con la barba, dal viso simpatico, mostrando i denti bianchi e sollevando una bottiglia quadrata, verdastra, che splendeva al sole. 

- Qu'est ce qu'ils disent? - domandò Veslovskij. 

- Invitano a bere la vodka. Probabilmente hanno diviso i campi. Io andrei a bere - disse Levin non senza malizia, sperando che Veslovskij, sedotto dalla vodka, andasse accanto a loro. 

- E perché offrono? 

- Così, se ne rallegrano. Davvero, avvicinatevi a loro. Per voi ciò sarà interessante. 

- Allons, c'est curieux. 

- Andate, andate, troverete la strada per il mulino! - gridò Levin e, voltandosi, vide con piacere che Veslovskij, curvo e incespicante, con le gambe stanche e il fucile nella mano tesa, si tirava fuori dalla palude verso i contadini. 

- Vieni anche tu! - gridava il contadino a Levin. - Non aver paura! Mangerai il pirog! 

Levin aveva una gran voglia di bere la vodka e di mangiare un pezzo di pane. Era fiacco e sentiva che tirava a stento fuori dal terreno melmoso le gambe che s'impigliavano, e per un attimo fu in dubbio. Ma il cane si fermò. E subito tutta la stanchezza scomparve, ed egli andò spedito attraverso il terreno melmoso verso il cane. Di sotto alle sue gambe volò fuori una beccaccia, egli sparò e la colpì; il cane continuava a star fermo. ``Pille!''. Di sotto al cane se ne sollevò un'altra. Levin tirò. Ma la giornata era cattiva; fece padella e, quando andò a cercare l'uccello ucciso, non riuscì a trovarlo. Si trascinò per tutta la càrice, ma Laska non credeva ch'egli avesse colpito, e quando egli la mandava a cercare, fingeva di cercare e non cercava. 

Anche senza Vasen'ka, al quale Levin attribuiva la propria sfortuna, le cose non andarono meglio. Anche qua di beccacce ce n'erano molte, ma Levin faceva una padella dopo l'altra. 

I raggi obliqui del sole erano ancora caldi; il vestito, passato da parte a parte dal sudore, si appiccicava al corpo; lo stivale sinistro, pieno d'acqua, era pesante e ciangottava; il sudore gli colava a gocce per il viso sporco del sedimento della polvere; in bocca aveva un certo sapore amaro, nel naso odor di polvere e d'acqua rugginosa, negli orecchi l'incessante zirlio delle beccacce; le canne del fucile non si potevano toccare tanto erano roventi, il cuore aveva battiti forti e rapidi, le mani gli tremavano per l'agitazione, e le gambe stanche inciampavano e si intersecavano nelle anfrattuosità del terreno melmoso; ma egli continuava a camminare e a sparare. Finalmente, fatta una padella vergognosa, buttò a terra il fucile e il cappello. 

``No, bisogna rientrare in sé'' si disse. Riprese il fucile e il cappello, chiamò ai suoi piedi Laska e uscì dalla palude. Uscito all'asciutto, sedette su di un monticello, si tolse gli stivali, ne versò fuori l'acqua; poi si avvicinò alla palude, bevve un po' d'acqua dal sapor di ruggine, bagnò le mani infocate e si lavò il viso e le mani. Rinfrescatosi, si avviò di nuovo verso il luogo dove era andata la beccaccia, con la ferma intenzione di non scalmanarsi. 

Voleva essere tranquillo, ma fu sempre lo stesso. Il suo dito premeva sul grilletto prima ch'egli prendesse la mira dell'uccello. Tutto andava di male in peggio. 

Aveva soltanto cinque pezzi nel carniere quando uscì dalla palude, dirigendosi verso gli ontani dove si doveva incontrare con Stepan Arkad'ic. 

Prima di vedere Stepan Arkad'ic, ne scorse il cane. Di sotto alla radice capovolta di un ontano, saltò fuori Krak, tutto nero di fango maleodorante della palude e, con aria di vincitore, scambiò un'annusata con Laska. Dietro a Krak si fece vedere, all'ombra degli ontani, anche la figura ben fatta di Stepan Arkad'ic. Veniva incontro rosso, sudato, col colletto sbottonato, zoppicando sempre un po' alla stessa maniera. 

- Be', avete sparato molto! - disse, sorridendo allegramente. 

- E tu? - domandò Levin. Ma era inutile domandare, perché aveva già visto il carniere pieno. 

- Be', non c'è male. 

Aveva quattordici pezzi. 

- Una bella palude! A te, probabilmente, ha dato fastidio Veslovskij. In due con un cane solo si va male - disse Stepan Arkad'ic, per attenuare il proprio trionfo. 

\capitolo{XI}Quando Levin e Stepan Arkad'ic giunsero all'izba del contadino dal quale si fermava sempre Levin, Veslovskij era già là. Era seduto nel centro della capanna e, tenendosi con tutte e due le mani alla panca dalla quale lo tirava via un soldato fratello della padrona, per cavargli gli stivali spruzzati di melma, rideva del suo riso contagiosamente allegro. 

- Sono arrivato or ora. Ils ont été charmants. Figuratevi, mi han dato da bere, da mangiare. Che pane! Una meraviglia! Délicieux! E la vodka, io non ne ho mai bevuta di più gustosa! E a nessun costo hanno voluto prendere del denaro. E non facevano che dire: ``non discutere'', così press'a poco. 

- E perché pigliar denari? Si vede che ve l'hanno offerta. Che forse hanno la vodka per venderla? - disse il soldato, dopo aver tirato via, finalmente, insieme con la calza annerita, lo stivale bagnato. 

Malgrado la sporcizia dell'izba, insudiciata dagli stivali dei cacciatori e dai cani sporchi che si leccavano, malgrado l'odore di polvere e di palude di cui s'era impregnata e l'assenza di coltelli e forchette, i cacciatori bevvero il tè, e cenarono con un gusto tale, quale solo a caccia si prova. Lavati e ripuliti, andarono in un fienile che avevano notato e dove i cocchieri avevano preparato i giacigli per i signori. 

Sebbene cominciasse già ad annottare, nessuno dei cacciatori aveva voglia di dormire. Dopo aver ondeggiato fra i ricordi e i racconti sul tiro, sui cani, sulle cacce precedenti, il discorso cadde su un tema che interessò tutti. Pigliando spunto dalle espressioni di entusiasmo, più volte ripetute da Vasen'ka, sul fascino di quel ricovero notturno, dell'odore del fieno e del carro rotto (a lui sembrava rotto perché era stato staccato dalla parte anteriore), sulla cordialità dei contadini che gli avevano dato da bere la vodka, sui cani che giacevano ciascuno ai piedi del proprio padrone, Oblonskij parlò dell'incanto della caccia da Malthus, alla quale aveva partecipato l'anno prima. Stepan Arkad'ic descriveva quali paludi avesse comprato questo Malthus, nel governatorato di Tver', e come fossero mantenute, e quali equipaggi avessero trasportato i cacciatori, quali mute di cani e quale tenda con colazione fosse stata piantata accanto alla palude. 

- Non capisco - disse Levin, sollevandosi sul fieno - come non ti siano antipatiche quelle persone. Capisco che una colazione con del Lafite possa esser piacevole, ma possibile che non ti sia odioso tutto quello sfarzo? Tutte queste persone, come un tempo i nostri appaltatori, guadagnano il denaro in un modo tale che, mentre lo guadagnano, meritano il disprezzo della gente, ed essi non si curano di questo disprezzo e dopo, con il loro profitto disonesto, si riscattano dal precedente disprezzo. 

- Perfettamente giusto! - rispose Vasen'ka Veslovskij. - Perfettamente! S'intende, Oblonskij lo fa per bonhomie, ma gli altri dicono: ``Oblonskij ci va\ldots{}''. 

- Per nulla - e Levin sentiva che Oblonskij sorrideva dicendo questo; - io non lo stimo più disonesto di qualunque altro mercante e nobile ricco. E quelli e questi hanno guadagnato egualmente col lavoro e con l'ingegno. 

- Già, ma con quale lavoro? È forse lavoro ottenere una concessione e rivenderla? 

- Certo che è un lavoro. Un lavoro nel senso che se non ci fosse lui o altri simili a lui, non ci sarebbero neanche le strade. 

- Ma non un lavoro come il lavoro del contadino o dello scienziato. 

- Ammettiamolo, ma è lavoro nel senso che la sua attività dà dei risultati: le strade. Ma già, tu pensi che le strade ferrate siano inutili. 

- No, questa è un'altra questione; sono pronto a riconoscere che sono utili. Ma qualsiasi acquisto non corrispondente al lavoro impiegatovi non è onesto. 

- Ma chi determinerà la corrispondenza? 

- L'acquisto per via disonesta, per mezzo dell'astuzia - disse Levin, sentendo di non saper definire chiaramente la linea di separazione fra l'onesto e il disonesto - così come l'accaparramento degli uffici bancari - continuò. - Questo male, l'acquisto di enormi sostanze senza lavoro, esiste come al tempo degli appalti, solo che ha cambiato forma. Le roi est mort, vive le roi! Hanno appena fatto in tempo a distruggere gli appalti, che sono apparse le strade ferrate, le banche: è lo stesso: un lucro senza lavoro. 

- Sì, tutto questo è forse giusto e acuto\ldots{} A cuccia, Krak! - gridò al cane Stepan Arkad'ic, evidentemente sicuro della esattezza del proprio tema e perciò calmo e posato. - Ma tu non hai definito la linea di separazione fra il lavoro onesto e quello disonesto. Il fatto che io riceva uno stipendio maggiore del mio capo-ufficio, anche s'egli conosce il lavoro meglio di me, è disonesto? 

- Non lo so. 

- Su, allora ti dirò: il fatto che tu per il tuo lavoro nell'azienda ricavi, ammettiamo, cinquemila rubli, mentre il contadino che ci ospita, per quanto si affatichi non ricavi più di cinquanta rubli, è disonesto nello stesso preciso modo che io riceva più del capo-ufficio e che Malthus riceva più di un ispettore delle ferrovie. Ma, al contrario, c'è un certo atteggiamento ostile, basato su nulla, della società verso queste persone, e mi sembra che ci sia dell'invidia\ldots{} 

- No, questo è ingiusto - disse Veslovskij - invidia non può esserci, piuttosto qualcosa di poco pulito in questo lavoro. 

- No, permetti - proseguì Levin. - Tu dici che è ingiusto che io ricavi cinquemila rubli, e il contadino cinquanta: è vero. È ingiusto, lo sento, ma\ldots{} 

- È proprio così. Come mai noi mangiamo, beviamo, andiamo a caccia, non facciamo nulla, e lui è eternamente, eternamente al lavoro? - disse Vasen'ka Veslovskij, dopo aver pensato, evidentemente per la prima volta in vita sua, in modo chiaro e, perciò con piena sincerità. 

- Sì, tu lo senti, ma non gli daresti il tuo podere - disse Stepan Arkad'ic, che pareva proprio voler stuzzicare Levin. 

Negli ultimi tempi fra i due cognati si erano stabiliti dei segreti rapporti ostili; come se, da quando avevano sposato le sorelle, fosse sorta fra di loro una certa rivalità in chi avesse assestato meglio la propria vita, e quella ostilità, ora, si esprimeva nella conversazione che cominciava ad assumere un tono personale. 

- Non lo do, perché nessuno lo pretende da me, e se volessi, non potrei darlo - rispondeva Levin - e non ci sarebbe a chi darlo. 

- Dàllo pure a questo contadino; non rifiuterà. 

- Sì, come glielo darò? Andrò da lui e concluderò un contratto d'acquisto? 

- Non so, ma se sei convinto di non averne diritto\ldots{} 

- Non sono del tutto convinto. Al contrario sento di non avere il diritto di alienare, sento di avere dei doveri verso la terra e verso la famiglia. 

- No, permetti; ma se tu consideri questa disuguaglianza ingiusta, allora perché non agisci in maniera da\ldots{} 

- Ma io agisco appunto, soltanto negativamente, nel senso che non cerco di aumentare quella differenza di condizione che esiste fra me e lui. 

- No, perdonami, questo è un paradosso. 

- Sì, è una spiegazione in un certo modo sofistica - confermò Veslovskij. - Oh, padrone - disse al contadino che, facendo scricchiolare la porta, era entrato nel fienile. - Non dormi ancora? 

- Macché dormire! Pensavo che lor signori dormissero e invece sento che discutono. Devo prendere un forcone qua. Non morde? - soggiunse, camminando cauto a piedi nudi. 

- E tu dove dormi? 

- Noi facciamo la guardia. 

- Ah, che notte - disse Veslovskij, guardando il limitare dell'izba e la vettura staccata che si intravedeva alla debole luce del crepuscolo, nella grande cornice del portone spalancato. - Ascoltate, sono voci di donne che cantano, e davvero, mica male. Chi è che canta, padrone? 

- Sono le ragazze di casa, qui accanto. 

- Andiamo a divertirci! Tanto non dormiremo. Oblonskij, andiamo! 

- Come sarebbe bello starsene sdraiati e camminare nello stesso tempo! - rispose Oblonskij, stiracchiandosi. - Sdraiàti si sta benissimo. 

- Be', andrò da solo - disse Veslovskij, alzandosi con vivacità e calzandosi. - Arrivederci, signori. Se si sta allegri, vi vengo a chiamare. Mi avete offerto della caccia e io non mi scorderò di voi. 

- Non è vero che è un bravo ragazzo? - disse Oblonskij, quando Veslovskij fu uscito e il contadino ebbe chiuso il portone dietro di lui. 

- Sì, bravo - rispose Levin, continuando a pensare all'argomento della conversazione che c'era stata or ora. Gli pareva d'aver chiaramente espresso, per quanto gli era stato possibile, i suoi pensieri e i suoi sentimenti, e invece tutti e due loro, persone intelligenti e sincere, avevano detto ch'egli si consolava con dei sofismi. Questo lo tormentava. 

- È proprio così, amico mio. Ci vuole una delle due: o confessare che la presente organizzazione della società è giusta, e allora difendere i propri diritti; o confessare che si usufruisce di privilegi ingiusti e, come faccio io, usufruirne con piacere. 

- No, se questo fosse ingiusto, tu non potresti usufruire con piacere di questi beni, almeno io non potrei. Io, soprattutto, ho bisogno di non sentirmi in colpa. 

- Ma davvero non dobbiamo andarci? - disse Stepan Arkad'ic, evidentemente stanco per la tensione del pensiero. - Tanto non dormiremo. Via, andiamo! 

Levin non rispondeva. La frase da lui detta nella conversazione, che egli agiva giustamente solo in senso negativo, lo teneva preoccupato. ``Possibile che solo in senso negativo si possa essere giusti?'' si domandava. 

- Ma come odora forte il fieno fresco! - disse Stepan Arkad'ic, cercando il berretto nel buio. 

- Non per principio, ma perché dovrei venirci? 

- Ma, credimi, ti creerai dei guai - disse Stepan Arkad'ic, che, trovato il berretto, si stava alzando. 

- Perché? 

- E forse non vedo in quali rapporti ti sei messo con tua moglie? Ho sentito che tra voi è questione di primaria importanza che tu vada o no a caccia per due giorni. Tutto questo va bene come idillio, ma per tutta la vita non soddisfa. L'uomo deve essere indipendente, egli ha i suoi interessi virili. L'uomo deve essere uomo - disse Oblonskij, aprendo il portone. 

- Cioè? Andare a far la corte alle ragazze di casa? - domandò Levin. 

- E perché non andarci, se è una cosa allegra? Ça ne tire pas à conséquence. Mia moglie non per questo starà peggio, e io me la spasserò. La cosa principale consiste nel conservare il sacrario della casa. Che in casa non ci sia nulla. Ma le mani non te le legare. 

- Può darsi - disse asciutto Levin, e si voltò su di un fianco. - Domani bisogna andar via presto, e io non sveglio nessuno, e vado via all'alba. 

- Messieurs, venez vite! - si sentì la voce di Veslovskij ch'era tornato. - Charmante! L'ho scoperta proprio io. Charmante, proprio una Gretchen, e abbiamo già fatto amicizia io e lei. Davvero carina assai! - egli raccontava con aria d'approvazione, come se fosse stata fatta carina proprio per lui, ed egli si mostrasse contento di chi gliela aveva preparata. 

Levin finse di dormire, e Oblonskij, infilate le pantofole e acceso un sigaro, uscì dal fienile e subito le loro voci si spensero. 

Levin a lungo non poté prender sonno. Sentiva i cavalli che masticavano il fieno, poi il padrone che col ragazzo più grande si preparava e andava via per far la guardia; sentiva poi che il soldato si metteva a letto, dall'altra parte del fienile, col nipote, un figlioletto del padrone; e il bimbo, con una vocina sottile, comunicava allo zio la propria impressione sui cani che gli erano sembrati terribili ed enormi; il ragazzo poi domandava chi avrebbe acchiappato quei cani, e il soldato con voce sorda e assonnata gli diceva che domani i cacciatori sarebbero andati nella palude e avrebbero sparato coi fucili, e infine, per liberarsi dalle domande del bambino, diceva: ``Dormi, Vas'ka, dormi, se no guarda''; e presto si mise a russare lui stesso, e tutto si chetò; si sentiva solo il nitrito di un cavallo e lo zirlio di una beccaccia. ``Possibile che io sia soltanto negativo? - egli si ripeteva. - Be', e allora? non sono colpevole!''. E si mise a pensare all'indomani. 

``Domani andrò di buon mattino e m'impegno a non scalmanarmi. Di beccacce ce n'è un'infinità. E anche di bottaie. E tornerò all'alloggio e troverò un biglietto di Kitty. E che Stiva abbia ragione, magari: non sono virile con lei, divento una femminuccia. Ma che fare! Sono di nuovo negativo!''. Nel sonno sentì il riso e l'allegro parlottare di Veslovskij e di Stepan Arkad'ic. Per un attimo aprì gli occhi: la luna era spuntata, ed essi stavano discorrendo nel vano del portone aperto, illuminato in pieno dalla luce lunare. Stepan Arkad'ic diceva qualcosa sulla freschezza della ragazza, paragonandola a una fresca nocciuola appena schiusa, e Veslovskij, ridendo del suo riso comunicativo, ripeteva le parole dettegli probabilmente da un contadino: ``Sbrigatela come puoi con una moglie tua''. Levin disse nel sonno: 

- Signori, a domani, appena si fa giorno! - e s'addormentò. 

\capitolo{XII}Svegliatosi sul far dell'alba, Levin provò a svegliare i compagni. Vasen'ka, sdraiato sul ventre e con una gamba allungata avvolta nella calza, dormiva così profondamente che non si poteva ottenere risposta da lui. Oblonskij, nel sonno, si rifiutava di andare via così presto. Perfino Laska, che dormiva acciambellata sull'orlo del fieno, si alzò di malavoglia, allungando e raddrizzando pigramente, una dopo l'altra, le zampe posteriori. Calzatosi, preso il fucile e aperta cautamente la porta del fienile che scricchiolava, Levin uscì sulla strada. I cocchieri dormivano accanto alle vetture, i cavalli sonnecchiavano. Uno di essi mangiava l'avena, spargendola con il muso per il trogolo. Nel cortile era ancora grigio 

- Com'è che ti sei alzato così presto, giaggiolo mio? - gli si rivolse con cordialità, come a un buon vecchio amico, la padrona di casa uscitagli incontro dalla izba. 

- Ma per andare a caccia, zia. Si va di qua alla palude? 

- Là di dietro, diritto; per le aie e per la canapa, uomo caro, c'è un viottolo. 

Camminando cauta coi piedi nudi abbronzati, la vecchia accompagnò Levin e levò la chiusura presso l'aia. 

- Diritto così e arriverai alla palude. I ragazzi iersera hanno spinto là le bestie. 

Laska correva avanti allegramente per il sentiero; Levin le teneva dietro con passo veloce, leggero, guardando continuamente il cielo. Voleva che il sole non sorgesse prima ch'egli fosse giunto alla palude, ma il sole non indugiava. La luna, che splendeva ancora quand'egli era uscito, adesso brillava opaca come un pezzo di mercurio; il lampeggiare mattutino, che prima non si poteva non distinguere, ora bisognava cercarlo; le macchie, prima indefinite nella campagna lontana, erano adesso già chiaramente visibili. Erano mucchi di segala. La rugiada, non ancora visibile senza la luce del sole sulla canapa alta, profumata, dalla quale era stata tolta quella secca, bagnava le gambe e il camiciotto di Levin più su della cintola. Nel silenzio trasparente del mattino si udivano i suoni più sottili. Un'ape piccola, col sibilo di una palla, volò accanto all'orecchio di Levin. Egli la guardò attento e ne scorse un'altra, poi una terza. Volavano tutte fuori dal graticcio di un'arnia e scomparivano sopra la canapa in direzione della palude. Il viottolo lo portò diritto alla palude, che si poteva riconoscere per i vapori che ne esalavano dove più densi, dove più radi, così che la càrice e i cespugli di citiso, come isolette, si cullavano su quel vapore. Al limite della palude e della strada i ragazzi e i contadini, che avevano fatto la guardia, erano sdraiati, e, innanzi l'alba, dormivano tutti sotto i gabbani. Non lontano da loro andavano tre cavalli impastoiati. Uno di essi faceva rumore coi ferri. Laska camminava accanto al padrone, chiedendo di andare avanti e guardandosi in giro. Oltrepassati i contadini che dormivano e giunto all'altezza del primo tratto paludoso, Levin esaminò i percussori e lasciò andare il cane. Uno dei cavalli, quello bruno, ben pasciuto, di un tre anni, visto il cane, fece uno scarto e, sollevata la coda, sbuffò. Gli altri cavalli si spaventarono anch'essi e, sguazzando per l'acquitrino con le zampe impastoiate e producendo con gli zoccoli tirati su dall'argilla spessa un suono simile a uno schiocco, si misero a saltar fuori della palude. Laska si fermò, guardando con irrisione i cavalli e interrogativamente Levin. Levin l'accarezzò e fischiò in segno che si poteva cominciare. 

Laska si mise a correre allegra e intenta per la melma che tremolava sotto di lei. Entrata di corsa nella palude, distinse subito, fra gli odori a lei noti della palude, della ruggine e l'odore estraneo dello sterco dei cavalli, l'odore degli uccelli, sparso per tutto il luogo, di quegli stessi uccelli odorosi che più degli altri l'agitavano. Qua e là, tra il muschio e le bardane della palude, quest'odore era molto forte, ma non si poteva stabilire, da quale parte venisse più forte e da quale più debole. Per trovare la direzione bisognava andare più lontano, sottovento. Senza avvertire il movimento delle proprie zampe, Laska a galoppo trattenuto, in modo da potersi fermare a ogni salto, se ve ne fosse stata la necessità, corse a destra, lontano dal venticello antelucano che spirava da oriente e si voltò verso il vento. Aspirata l'aria con le narici dilatate, sentì subito che non solo c'erano le orme, ma che essi erano là, davanti a lei, e non uno, ma molti. Laska diminuì la velocità della corsa. Essi erano là, ma dove esattamente, essa non poteva ancora precisarlo. Per trovare proprio il posto, aveva cominciato un giro, quando improvvisamente la voce del padrone la distrasse. ``Laska, qua!'' egli disse, indicandole l'altra parte. Essa si fermò un po', quasi a chiedergli se non fosse meglio fare così come aveva cominciato. Ma egli ripeté l'ordine con voce irritata, indicando un ammasso di montagnole ricoperto di acqua, dove non poteva esserci nulla. Essa obbedì, fingendo di cercare, per fargli piacere, rovistò le montagnole e tornò al posto di prima, e subito sentì di nuovo. Adesso, quando egli non la disturbava, essa sapeva che cosa fare e, senza guardare sotto le zampe, impigliandosi con stizza nelle montagnole ripide e cadendo nell'acqua, ma raddrizzandosi con le zampe agili, forti, cominciò un giro che doveva spiegarle tutto. L'odore degli uccelli la colpiva in maniera sempre più forte, sempre più precisa, e improvvisamente tutto le si fece chiaro, che, cioè, uno di quelli era là, dietro a un monticello, a cinque passi davanti a lei. Si fermò e si irrigidì in tutto il corpo. Sulle zampe basse non poteva veder nulla dinanzi a sé, ma dall'odore sapeva che esso si era posato non più lontano di cinque passi. Stava ritta, percependo sempre di più e pregustando l'attesa. La coda diritta era allungata, e tremava soltanto alla punta. La bocca era leggermente aperta, gli orecchi sollevati. Un orecchio s'era voltato ancora durante la corsa, ed essa respirava faticosamente, ma con cautela, e con cautela ancora maggiore si voltava a guardare, più con gli occhi che con la testa, il padrone. Costui, col suo solito viso, ma sempre con gli occhi minacciosi, camminava, inciampando, su per le montagnole e straordinariamente lento, come a lei sembrava. Le sembrava che camminasse lentamente, mentre egli correva. Notato questo speciale braccare di Laska (essa si stringeva tutta al terreno, come se avanzasse a grandi passi raccogliendo con le zampe posteriori, e apriva leggermente la bocca), Levin capì che fiutava le beccacce e, pregando Iddio per il successo, in particolare per il primo uccello, corse verso di essa. Accostatosi, cominciò a guardare dietro di sé dalla propria altezza, e vide con gli occhi quello ch'essa vedeva con il naso. Nel viottolo, fra le montagnole, sopra una di queste si vedeva una beccaccia bottaia. Col capo voltato stava in ascolto. Poi, assestate e richiuse le ali, dimenata goffamente la coda, scomparve dietro un angolo. 

- Pille, pille! - gridò Levin, spingendo Laska nel sedere. 

``Ma io non posso andare - pensava Laska. - Dove vado? Di qua li sento, ma se mi muovo in avanti non sentirò più nulla, né dove sono né chi sono''. Ma ecco ch'egli la spinse col ginocchio e con un mormorio agitato proferì: ``Pille, Lasocka, pille!''. 

``Ebbene, se lui lo vuole, lo farò, ma non rispondo più di me'' essa pensò e si lanciò in avanti fra le montagnole a gambe levate. Adesso non fiutava più e vedeva e sentiva soltanto, senza capir nulla. 

A dieci passi dal posto di prima, con lo zirlio chioccio e il rumore particolare delle beccacce, se ne alzò una. E subito, dopo lo sparo, cadde pesantemente, battendo il petto bianco contro la melma bagnata. Un'altra non aspettò e si levò dietro a Levin, senza bisogno del cane. 

Quando Levin si voltò, era già lontana, ma il colpo la raggiunse. Dopo aver volato per una ventina di passi, la seconda beccaccia si levò in alto come una palla lanciata e precipitò pesante sull'asciutto. 

``Ecco, ce ne saranno molte!'' pensava Levin, riponendo nel carniere le beccacce tiepide e grasse. 

- Ehi, Lasocka, ce ne saranno molte? 

Quando Levin, caricato il fucile, si mosse per andare avanti, il sole, non ancora visibile dietro le piccole nuvole, s'era già levato. La luna, perduto tutto il suo splendore, sbiancava nel cielo come una piccola nube; di stelle non se ne vedeva neanche più una. I tratti acquitrinosi, che prima si inargentavano di rugiada, ora si facevano d'oro. La ruggine era tutta ambrata. L'azzurro delle erbe tramutava in un verde giallastro. Gli uccelli di palude brulicavano sui piccoli cespugli che, presso il ruscello, brillavano di rugiada e proiettavano un'ombra lunga. Un avvoltoio s'era svegliato e s'era andato a posare sopra una bica, voltando il capo da un lato all'altro guardando scontento la palude. Le cornacchie volavano sui campi e un ragazzetto scalzo spingeva già innanzi a sé i cavalli verso un vecchio, che s'era levato di sotto al gabbano e si grattava. Il fumo degli spari biancheggiava come latte su per il verde dell'erba. Uno dei ragazzi venne, correndo, da Levin. 

- Zio, iersera qua c'erano le anitre! - gli gridò, e gli tenne dietro a distanza. 

E a Levin fece molto piacere colpire ancora, proprio là, una dietro l'altra, tre beccacce, alla presenza di quel ragazzetto che esprimeva la sua approvazione. 

\capitolo{XIII}Il presagio venatorio: se non ti lasci scappare la prima bestia o il primo uccello, la caccia sarà buona, si mostrò veritiero. 

Stanco, affamato, felice, Levin, verso le dieci del mattino, dopo aver camminato per trenta verste, con diciannove capi di selvaggina reale e un'anatra, che aveva legata alla cintura, giacché non entrava più nel carniere, tornò al rifugio. I compagni s'erano svegliati da un pezzo e avevano avuto il tempo di farsi venir fame e di far colazione. 

- Aspettate, aspettate, lo so che sono diciannove - diceva Levin, ricontando per la seconda volta le beccacce e le beccacce bottaie, che non avevano più quell'aspetto importante che avevano quando volavano, così ritorte e rinsecchite com'erano, col sangue coagulato e i capini voltati da un lato. 

Il conto era giusto, e l'invidia di Stepan Arkad'ic fece piacere a Levin. Gli fece anche piacere trovare, appena giunto all'alloggio, l'inserviente inviato da Kitty, già là con un biglietto. 

``Sto perfettamente bene e di ottimo umore. E se temi per me, puoi star più tranquillo di prima. Ho una nuova guardia del corpo, Mar'ja Vlas'evna - era la levatrice, un personaggio nuovo, importante nella vita familiare di Levin. - È venuta a farmi visita. M'ha trovato in ottima salute e noi l'abbiamo trattenuta fino al tuo arrivo. Stiamo tutti di ottimo umore, bene in salute, e tu, per favore, non affrettarti, e se la caccia è buona, rimani ancora un giorno''. 

Queste due gioie, la caccia fortunata e il biglietto della moglie, erano tanto grandi che i due piccoli dispiaceri, seguìti alla caccia, passarono con facilità per Levin. Uno consisteva nel fatto che il bilancino, il sauro che il giorno prima evidentemente aveva faticato troppo, non mangiava la biada e s'era fatto triste. Il cocchiere diceva che era sfiancato. 

- Ieri l'avete fiaccato, Konstantin Dmitric - diceva. - E come l'avete spinto per dieci verste fuori di strada! 

L'altro disappunto che distrusse sulle prime la sua buona disposizione d'animo, ma che poi lo fece rider molto, consistette nel fatto che di tutte le provviste, date da Kitty con un'abbondanza tale che sembrava non si potessero finire in una settimana, non era rimasto più nulla. Ritornando stanco ed affamato dalla caccia, Levin sognava in modo così preciso gli sfogliantini che, accostandosi all'alloggio, ne pregustava l'odore e il sapore in bocca, così come Laska fiutava la selvaggina, e ordinò subito a Filipp di portarglieli. Risultò che non solo di sfogliantini, ma neanche di pollastri ce n'erano più. 

- Uhm, che appetito! - disse Stepan Arkad'ic, indicando Vasen'ka Veslovskij. - Io non soffro di inappetenza, ma questo qui è sorprendente\ldots{} 

- E che fare? - disse Levin, guardando torvo Veslovskij. - Filipp, dammi del manzo. 

- Il manzo l'hanno mangiato e l'osso l'hanno dato ai cani - disse Filipp. 

Levin provò un così gran dispetto che disse con stizza: 

- E mi avessero almeno lasciato qualcosa! - e gli venne voglia di piangere. - Allora, sventra un po' di caccia - disse con voce tremante a Filipp, cercando di non guardare Vasen'ka - e mettici sopra l'ortica. E per me, chiedi almeno del latte. 

Dopo, quando si fu saziato, si vergognò di aver mostrato la propria furia a una persona estranea, e prese a ridere del proprio affamato risentimento. 

La sera fecero ancora una battuta, nella quale anche Veslovskij ammazzò alcuni capi, e nella notte tornarono a casa. 

La via del ritorno fu altrettanto amena come era stata quella dell'andata. Veslovskij ora cantava, ora ricordava con piacere le proprie avventure con i contadini che gli avevano offerto la vodka, e gli avevano detto: ``non discutere'', ora le proprie avventure notturne con le nocciuole e la ragazza di casa e col contadino che gli aveva chiesto se era ammogliato e, avendo sentito che non era ammogliato, gli aveva detto: ``E tu non desiderare le mogli degli altri, ma piuttosto sbrigati a procurartene una per te''. Queste parole in particolare facevano ridere Veslovskij. 

- In generale, sono straordinariamente soddisfatto della nostra gita. E voi Levin? 

- Contentissimo - disse con sincerità Levin che era in particolar modo felice, non soltanto di non sentire più quell'ostilità che a casa aveva provato verso Vasen'ka, ma di sentire, al contrario, verso di lui la più cordiale disposizione d'animo. 

\capitolo{XIV}Il giorno dopo alle dieci, Levin, dopo aver fatto il giro dell'azienda, bussò alla camera dove aveva dormito Vasen'ka. 

- Entrez! - gli gridò Vasen'ka. - Perdonatemi, non ho ancora finito le mie ablutions - disse sorridendo, ritto dinanzi a lui, con la sola biancheria indosso. 

- Non vi preoccupate, vi prego. - Levin sedette accanto alla finestra. - Avete dormito bene? 

- Come un morto. E oggi che giornata è per la caccia? 

- Cosa bevete: tè o caffè? 

- Né l'uno né l'altro. Faccio colazione. Mi vergogno proprio. Le signore, penso, si saranno alzate. Adesso è bellissimo passeggiare. Fatemi vedere i cavalli. 

Dopo aver passeggiato per il giardino, dopo esser stati nella scuderia, e aver perfino fatto insieme la ginnastica alle sbarre, Levin tornò a casa con l'ospite ed entrò nel salotto con lui. 

- Abbiamo cacciato magnificamente, e quante impressioni! - disse Veslovskij avvicinandosi a Kitty, che sedeva presso il samovar. - Che peccato che le signore siano private di questi piaceri! 

``Be', che c'è di strano, bisogna pure ch'egli parli in qualche modo con la padrona di casa'' si diceva Levin. Gli era parso di veder di nuovo qualcosa nel sorriso, in quell'espressione trionfante con cui l'ospite si era rivolto a Kitty\ldots{} 

La principessa, seduta dall'altra parte della tavola con Mar'ja Vlas'evna e Stepan Arkad'ic, chiamò a sé Levin e avviò una conversazione con lui sul trasferimento a Mosca per il parto di Kitty e sulla preparazione della casa. Per Levin, come al tempo del matrimonio, era spiacevole qualsiasi preparativo che offendesse, con la sua piccineria, la grandiosità di quello che si compiva; così ancora più offensivi gli sembravano i preparativi per il prossimo parto, il cui tempo veniva calcolato in un certo modo, sulle dita. Aveva cercato sempre di non ascoltare questi discorsi sul metodo di fasciare il futuro bambino, aveva cercato di voltarsi dall'altra parte e di non vedere certe misteriose infinite strisce a maglia, certi triangoli di tela, ai quali Dolly attribuiva una particolare importanza, e cose simili. L'avvenimento della nascita di un figlio (era sicuro che sarebbe stato un figlio), che gli avevano promesso, ma al quale tuttavia non poteva credere, tanto gli sembrava straordinario, gli appariva da un lato una felicità così immensa e perciò impossibile, dall'altro lato un avvenimento così misterioso, che quella pretesa conoscenza di ciò che sarebbe stato, e di conseguenza la preparazione come a qualcosa di ordinario, opera degli uomini stessi, gli appariva disgustosa e umiliante. 

Ma la principessa non capiva i suoi sentimenti e interpretava la sua pigrizia a pensare e a parlare di questo per leggerezza e indifferenza; perciò non gli dava tregua. Aveva dato l'incarico a Stepan Arkad'ic di andare a vedere un appartamento e ora aveva chiamato a sé Levin. 

- Io non so nulla, principessa. Fate come volete - egli diceva. 

- Bisogna decidere a quando il trasferimento. 

- Io, davvero, non lo so. So che nascono milioni di bambini senza Mosca e senza medici\ldots{} perché mai\ldots{} 

- Ma se è così\ldots{} 

- Ma no, come vuole Kitty. 

- Con Kitty non se ne può parlare! Che vuoi, che la spaventi? Ecco, questa primavera Natalie Golicyna è morta per colpa d'un cattivo ostetrico. 

- Come direte voi, così farò - disse lui cupo. 

La principessa cominciò a parlargli, ma egli non l'ascoltava. Sebbene questa conversazione lo sconvolgesse, s'era fatto cupo non per la conversazione, ma per quello che vedeva presso il samovar. 

``No, non è possibile'' pensava guardando Vasen'ka che s'era chinato verso Kitty, dicendole qualcosa col suo bonario sorriso e lei che arrossiva e si agitava\ldots{} 

C'era qualcosa d'impuro nell'atteggiamento di Vasen'ka, nel suo sguardo, nel sorriso. Levin vedeva perfino qualcosa d'impuro nell'atteggiamento e nello sguardo di Kitty. E di nuovo la luce si oscurò dinanzi ai suoi occhi. Di nuovo, come il giorno prima, a un tratto, senza il più piccolo passaggio, si sentì gettato giù dall'altezza della propria felicità, della propria calma e dignità in un abisso di disperazione, di rancore, di umiliazione. Di nuovo tutto e tutti gli divennero disgustosi. 

- Allora, principessa, fate come volete - disse, voltandosi di nuovo a guardare. 

- ``Tiara del Monomaco, sei pesante!'' gli disse, scherzando, Stepan Arkad'ic, alludendo, evidentemente, non alla sola conversazione con la principessa, ma all'agitazione di Levin che egli aveva notata. - Come hai fatto tardi, oggi Dolly! 

Tutti si alzarono per accogliere Dar'ja Aleksandrovna. Vasen'ka si alzò per un attimo solo e, con quella mancanza di cortesia verso le signore, propria dei giovanotti moderni, s'inchinò appena e riprese il discorso, mettendosi a ridere per qualcosa. 

- Maša mi ha tormentato. Ha dormito male e oggi è terribilmente capricciosa - disse Dolly. 

La conversazione avviata da Vasen'ka con Kitty riguardava di nuovo l'argomento del giorno precedente, cioè Anna e la questione se l'amore possa sovrapporsi alle convenienze sociali. Per Kitty questa conversazione non era gradita, l'agitava e per la sostanza stessa e per il tono col quale era condotta, e soprattutto perché sapeva già come avrebbe agito su suo marito. Ma era troppo semplice e innocente per saper porre termine alla conversazione e finanche per nascondere il piacere esteriore che le procurava l'evidente premura di quel giovane. Voleva porre termine alla conversazione, ma non sapeva cosa dovesse fare. Qualunque cosa avesse fatto, sarebbe stata notata da suo marito e, lo sapeva, interpretata in senso cattivo. E realmente quando domandò a Dolly cosa avesse Maša mentre Vasen'ka, aspettando la fine di questa interruzione per lui noiosa, guardava con indifferenza Dolly, questa domanda sembrò a Levin un'astuzia poco naturale, odiosa. 

- Ebbene, andiamo a cercar funghi, oggi? - domandò Dolly. 

- Andiamo, vi prego, verrò anch'io - disse Kitty e arrossì. Voleva domandare a Vasen'ka per cortesia se sarebbe andato, e non lo domandò. - Tu dove vai, Kostja? - ella domandò al marito con aria colpevole, mentre egli, con fare deciso, le passava accanto. Quest'espressione colpevole confermò tutti i dubbi di lui. 

- Quando non c'ero è arrivato il meccanico, non l'ho ancora visto - disse, senza guardarla. 

Egli scese giù, ma non aveva ancora fatto in tempo a uscire dallo studio che sentì i noti passi della moglie, che veniva verso di lui cauta e frettolosa. 

- Che vuoi? - le disse asciutto. - Siamo occupati. 

- Scusatemi - ella disse rivolta al meccanico tedesco: - devo dire due parole a mio marito. 

Il tedesco voleva andarsene, ma Levin gli disse: 

- Non vi incomodate. 

- Il treno è alle tre? - domandò il tedesco - che non abbia a giungere in ritardo. 

Levin non gli rispose e uscì lui stesso con la moglie. 

- Be', cosa avete da dirmi? - egli disse in francese. 

Egli non la guardava e non voleva accorgersi che lei, nelle sue condizioni, tremava tutta in viso e aveva un'aria pietosa, annientata. 

- Io\ldots{} io voglio dire che così non si può vivere, che è un tormento\ldots{} - ella pronunciò. 

- Qui nella dispensa c'è la servitù - egli disse irritato: - non fate scene. 

- Su, andiamo di qua! 

Essi stavano in piedi, in una stanza di passaggio. Kitty voleva entrare in quella accanto, ma là l'inglese dava lezione a Tanja. 

- Su, andiamo in giardino! 

In giardino s'imbatterono in un contadino che ripuliva un viottolo. E, senza pensare che il contadino vedeva il viso lacrimoso di lei e quello agitato di lui, senza pensare che avevano l'aspetto di persone che venissero fuori da una qualche sventura, andavano avanti a passi svelti, sentendo che dovevano dirsi tutto e disingannarsi a vicenda, stare un po' soli insieme e con questo liberarsi del tormento che provavano tutti e due. 

- Così non si può vivere! È un tormento! Io soffro, tu soffri. Per cosa! - disse lei quando finalmente raggiunsero una panca solitaria all'angolo del viale dei tigli. 

- Ma tu dimmi una cosa sola: c'era nel suo tono qualcosa di sconveniente, di impuro, di umiliante e pauroso? - egli diceva, ponendosi dinanzi a lei di nuovo in quella stessa posa, coi pugni davanti al petto, così come s'era messo dinanzi a lei quella notte. 

- C'era - diceva lei con voce tremante. - Ma, Kostja, non vedi forse che io non sono colpevole? Io fin dalla mattina volevo prendere un tono diverso, ma queste persone\ldots{} Perché è venuto? Come eravamo felici! - ella diceva, soffocando per i singhiozzi che sollevavano tutto il suo corpo ingrossato. 

Il giardiniere si accorse con stupore, che, sebbene nulla li inseguisse e non ci fosse motivo alcuno di fuggire e sebbene nulla di particolarmente gioioso avessero potuto trovare sulla panchina - il giardiniere si accorse che tornavano a casa, passandogli accanto, con il viso rasserenato, luminoso. 

\capitolo{XV}Accompagnata la moglie di sopra, Levin andò nell'appartamento di Dolly. Dar'ja Aleksandrovna, per conto suo, era molto amareggiata quel giorno. Camminava per la stanza e diceva arrabbiata alla bambina che stava in piedi in un angolo e piangeva: 

- E starai nell'angolo tutto il giorno, e pranzerai sola, e non vedrai neanche una bambola, e il vestito nuovo non te lo farò - diceva, non sapendo più come punirla. - No, è una bambina cattiva! - disse rivolta a Levin. - Di dove vengono queste inclinazioni disgustose? 

- Ma che ha fatto mai? - chiese Levin alquanto indifferente, e, desideroso di trovar consiglio per la cosa sua, s'irritò d'esser capitato fuor di proposito. 

- Lei e Griša sono andati dove ci sono i lamponi e là\ldots{} non posso neppur dire quello che lei ha fatto. Rimpiango mille volte miss Elliot. Questa qui non bada a nulla. Figurez vous, que la petite\ldots{} - E Dar'ja Aleksandrovna raccontò il delitto di Maša. 

- Ma questo non dimostra nulla, queste sono tutt'altro che inclinazioni cattive, questa è semplice birichineria - la tranquillizzava Levin. 

- Ma tu sei un po' sconvolto! Perché sei venuto? - chiese Dolly. - Che succede là? 

E dal tono di questa domanda, Levin sentì che gli sarebbe stato facile metter fuori quello che aveva in mente di dire. 

- Non sono stato là, sono stato in giardino con Kitty. Abbiamo litigato per la seconda volta da che\ldots{} Stiva è arrivato. 

Dolly lo guardava con gli occhi intelligenti che intendevano. 

- Su, di', con una mano sul cuore, c'era\ldots{} non in Kitty, ma in quel signore, un tono tale da essere spiacevole, non solo spiacevole, ma spaventoso, offensivo per un marito? 

- Be', come dirti\ldots{} Resta, resta nell'angolo! - si rivolse a Maša che, visto un sorriso appena percettibile sul viso della madre, stava per voltarsi. - L'opinione del mondo sarebbe che egli si comporta come si comportano tutti i giovanotti. Il fait la cour à une jeune et jolie femme. E un marito mondano deve esserne semplicemente lusingato. 

- Sì, sì - disse cupo Levin - ma tu l'hai notato? 

- Non solo io, ma Stiva l'ha notato. M'ha proprio detto dopo il tè: ``Je crois que Veslovskij fait un petit brin de cour à Kitty''. 

- E allora benissimo, ora sono tranquillo. Lo caccerò fuori - disse Levin. 

- Ma sei impazzito? - gridò Dolly con orrore. - Che hai, Kostja, ritorna in te - disse ella, ridendo. - Su, adesso puoi andare da Fanny - ella disse a Maša. - No, se vuoi proprio, allora lo dirò a Stiva. Lui lo porterà via. Si può dire che tu aspetti ospiti. In fondo, è di troppo in casa nostra. 

- No, no, faccio da me. 

- Ma ti bisticcerai?\ldots{} 

- Niente affatto. Per me sarà molto divertente, realmente divertente - disse Levin con gli occhi che gli brillavano. - Su, perdonala, Dolly! Non lo farà più - disse egli della piccola delinquente, che non andava da Fanny e stava lì in piedi, indecisa dinanzi alla madre, aspettando e cercando il suo sguardo di sotto in su. 

La madre la guardò. La bambina scoppiò in singhiozzi, si sprofondò col viso nelle ginocchia della mamma, e Dolly le mise sul capo la sua mano magra, tenera. 

``E che c'è di comune fra noi e lui?'' pensò Levin e andò in cerca di Veslovskij. 

Passando per l'anticamera, ordinò di far attaccare il calesse, per andare alla stazione. 

- Ieri s'è rotta una molla - rispose il servitore. 

- Su, allora la carretta, ma presto. Dov'è l'ospite? 

- Il signore è andato in camera sua. 

Levin trovò Vasen'ka che, tolta dalla valigia tutta la roba, e sparpagliate le nuove romanze, provava delle ghette per andare a cavallo. 

O che ci fosse qualcosa di particolare nel viso di Levin o che lo stesso Vasen'ka avesse sentito che ce petit brin de cour, ch'egli aveva vagheggiato, era fuor di posto in quella famiglia, certo egli fu un poco (quanto può esserlo un uomo di mondo) confuso dall'entrata di Levin. 

- Montate a cavallo con le ghette? 

- Sì, è molto più igienico - disse Vasen'ka, ponendo la gamba grassa su di una sedia, agganciando la fibbia di sotto e sorridendo allegramente, di cuore. 

Era senza dubbio un buon ragazzo, e Levin provò pena per lui e vergogna per sé, padrone di casa, quando notò la timidezza nello sguardo di Vasen'ka. 

Sulla tavola giaceva un pezzo di bastone che la mattina avevano spezzato insieme nel far ginnastica, provando a sollevare le sbarre rigonfie per l'umidità. Levin prese in mano questo pezzo di legno e cominciò a romperne l'estremità già spaccata, senza sapere da che parte cominciare a parlare. 

- Volevo\ldots{} - Stava già per tacere, ma ad un tratto, ricordatosi di Kitty e di tutto quello che c'era stato, disse guardandolo risoluto negli occhi: - Ho ordinato che vi attaccassero i cavalli. 

- Sarebbe a dire, come? - cominciò Vasen'ka con sorpresa. - E dove si deve andare? 

- Per voi, alla stazione - disse torvo Levin, tormentando il bastone. 

- Forse voi partite, o è accaduto qualcosa? 

- È accaduto che aspetto ospiti - disse Levin, rompendo sempre più in fretta con le dita forti le estremità del bastone che s'era spaccato. - Anzi, non aspetto ospiti e non è accaduto nulla, ma vi prego di partire. Potete spiegare come volete la mia sgarberia. 

Vasen'ka si raddrizzò. 

- Io prego voi di spiegarmi\ldots{} - disse con dignità, avendo finalmente capito. 

- Non posso spiegarvelo - cominciò a dire piano e lento Levin, cercando di nascondere il tremito dei propri zigomi. - Ed è meglio che non domandiate. 

E siccome le estremità scheggiate erano state già tolte, Levin s'attaccò con le dita alle estremità grosse, spaccò il bastone e raccolse con cura la punta che era caduta. 

Probabilmente la vista di quelle mani tese, di quegli stessi muscoli che la mattina aveva palpato facendo ginnastica, e degli occhi scintillanti, la voce piana e gli zigomi tremanti convinsero Vasen'ka più delle parole. Dopo aver alzato le spalle e sorriso con sprezzo, egli s'inchinò. 

- Non posso vedere Oblonskij? 

L'alzata di spalle e il sorriso non irritarono Levin. ``Che gli rimane da fare?'' egli pensò. 

- Ve lo mando subito. 

- Che cosa insensata! - diceva Stepan Arkad'ic, dopo aver saputo dall'amico che lo si cacciava di casa e avendo trovato Levin in giardino, dove passeggiava, in attesa della partenza dell'ospite. - Mais c'est ridicule! Ma che mosca t'ha mai pizzicato? Mais c'est du dernier ridicule! Cosa mai ti è sembrato, se un giovanotto\ldots{} 

Ma il punto in cui la mosca aveva pizzicato Levin doleva ancora, si vedeva, perché egli di nuovo impallidì, quando Stepan Arkad'ic cercò di chieder ragione, e lo interruppe in fretta. 

- Per favore, non chiedermi la ragione! non posso fare altrimenti. Me ne vergogno molto dinanzi a te e dinanzi a lui. Ma per lui, penso, non sarà un gran dolore partire e per me e per mia moglie la sua presenza è spiacevole. 

- Ma per lui è offensivo! Et puis c'est ridicule! 

- E per me è offensivo e tormentoso! E io non sono colpevole di nulla e non ho ragione di soffrire. 

- Ma questo non me l'aspettavo da te! On peut être jaloux, mais à ce point c'est du dernier ridicule! 

Levin si voltò in fretta e andò via in fondo al viale, continuando a camminare solo, avanti e indietro. Presto sentì il rumore della carretta e vide di là dagli alberi Vasen'ka col berrettino alla scozzese seduto sul fieno (per colmo di sventura sulla carretta non c'era il sedile), mentre passava per il viale, sobbalzando alle scosse. 

``Che altro c'è?'' pensò Levin quando un servitore, uscito di corsa dalla casa, fermò la carretta. Era il meccanico, di cui Levin si era del tutto dimenticato. Il meccanico, salutando, disse qualcosa a Veslovskij, poi montò sulla carretta e partirono insieme. 

Stepan Arkad'ic e la principessa erano indignati dell'azione di Levin. E lui stesso si sentiva non solo ridicule al sommo grado, ma anche completamente colpevole e coperto di vergogna; ma, ricordando quello che lui e sua moglie avevano sofferto, e domandandosi come avrebbe agito una seconda volta, si rispondeva che avrebbe agito proprio allo stesso modo. 

Malgrado tutto questo, alla fine di quel giorno, tutti, tranne la principessa, che non perdonava quell'azione a Levin, divennero straordinariamente allegri e animati, come bambini dopo una punizione o come grandi dopo un faticoso ricevimento ufficiale. Così la sera, in assenza della principessa, si parlò della cacciata di Vasen'ka, come di un avvenimento lontano. E Dolly, che aveva ricevuto dal padre l'arte di raccontare le cose con spirito, faceva morir dal ridere Varen'ka, quando per la terza o quarta volta, sempre con nuove aggiunte umoristiche, raccontava come, proprio mentre lei si accingeva a mettersi dei nuovi nastri in omaggio all'ospite, e stava per uscire in salotto, ecco che, all'improvviso, aveva sentito il rumore della carretta. E chi c'era mai nella carretta? Vasen'ka in persona, col berrettino scozzese, le romanze e le ghette, era seduto sul fieno. 

- Avesse almeno fatto attaccar la carrozza! No, e poi sento: ``Aspettate!''. Be', penso, avranno avuto compassione. Guardo: gli hanno messo a sedere accanto quel grassone di tedesco e l'hanno portato via\ldots{} E i miei nastri sono andati perduti. 

\capitolo{XVI}Dar'ja Aleksandrovna mise in atto la sua intenzione e andò da Anna. Le spiaceva molto addolorare la sorella e far cosa sgradita al marito di lei; capiva come avessero ragione i Levin a non desiderare d'avere nessun rapporto con i Vronskij; ma riteneva suo dovere stare un po' da Anna e dimostrarle che i suoi sentimenti non potevano essere cambiati, malgrado la mutata situazione di lei. 

Per non dipendere dai Levin per quel viaggio, Dar'ja Aleksandrovna mandò in paese a noleggiare i cavalli; ma Levin, saputolo, venne da lei a fare le sue rimostranze. 

- Ma perché pensi che mi spiaccia il tuo viaggio? E se anche ciò fosse mi spiace ancor più se tu non prendi i miei cavalli - egli diceva. - Non m'hai detto neppure una volta che eri decisa ad andare. E poi, prenderli in affitto al paese, in primo luogo mi rincresce, e poi anche se prenderanno l'incarico, non ti porteranno fin là. Io li ho i cavalli. E se non vuoi darmi un dispiacere, prendi i miei. 

Dar'ja Aleksandrovna aveva dovuto acconsentire, e, il giorno fissato, Levin preparò per la cognata un tiro a quattro e dei cavalli di cambio, scegliendoli fra quelli da lavoro e da sella, molto brutti, ma che potevano portare a destinazione Dar'ja Aleksandrovna in un solo giorno. In quel momento, in cui i cavalli servivano e per la principessa che partiva e per la levatrice, la cosa era stata difficile per Levin, ma per dovere di ospitalità egli non poteva permettere a Dar'ja Aleksandrovna di noleggiare i cavalli, inoltre, sapeva che i venti rubli che le avrebbero chiesto per quel viaggio avevano grande peso per lei, e le faccende finanziarie di Dar'ja Aleksandrovna, così dissestate, erano sentite dai Levin come loro proprie. 

Dar'ja Aleksandrovna, per consiglio di Levin, partì prima dell'alba. La strada era buona, la vettura ottima, i cavalli correvano allegramente, e a cassetta, oltre il cocchiere, sedeva, invece del servitore, lo scrivano, mandato da Levin come scorta. Dar'ja Aleksandrovna sonnecchiava e si svegliò solo all'avvicinarsi della locanda, dove bisognava dare il cambio ai cavalli. 

Dopo aver bevuto il tè da quello stesso ricco contadino-proprietario dal quale s'era fermato Levin nel recarsi da Svijazskij, e dopo aver chiacchierato con le donne dei bambini e col vecchio del conte Vronskij, ch'egli lodava molto, Dar'ja Aleksandrovna, alle dieci, proseguì. A casa, per le preoccupazioni dei bambini, non aveva mai il tempo di riflettere. In compenso adesso, in quel lungo spazio di quattro ore, tutti i pensieri prima contenuti, le si affollarono a un tratto nella mente. Ed ella ripensò a tutta la sua vita come non mai prima, e dai lati più diversi. Erano strani anche per lei i suoi pensieri. Pensò dapprima ai bambini, per i quali, benché la principessa e soprattutto Kitty (in lei aveva più fiducia) avessero promesso di sorvegliarli, era tuttavia inquieta. ``Che Maša non incominci di nuovo a far birichinate, che il cavallo non faccia male a Griša, e che lo stomaco di Lily non si disturbi di più''. Ma poi le questioni del presente cominciarono a tramutarsi nelle questioni dell'immediato futuro. Si mise a pensare che a Mosca, per quell'inverno, bisognava prendere in affitto un appartamento nuovo; cambiare la mobilia in salotto e fare una pelliccia nuova alla figlia maggiore. Poi cominciarono ad apparire le questioni d'un futuro più lontano: come avrebbe sistemato in società i figliuoli. ``Per le bambine, ancora non è nulla - pensava - ma i ragazzi?''. 

``Va bene, io ora mi occupo di Griša, ma è solo perché ora sono libera e non devo partorire. Su Stiva, naturalmente, non c'è da fare nessun assegnamento. E io con l'aiuto di persone buone, li sistemerò nel mondo; ma se c'è di nuovo un parto\ldots{}''. E le venne l'idea che era stato detto ingiustamente che una maledizione gravava sulla donna, perché generasse i figli fra i tormenti: ``Partorire non è niente, ma essere incinta, ecco quello ch'è tormento'' ella pensò ricordandosi della sua ultima gravidanza e della morte dell'ultimo bambino. E le tornò in mente la conversazione con una giovane là alla locanda. Alla domanda se avesse bambini, la bella sposa aveva risposto allegra: 

- Ho avuto una bambina, ma Dio mi ha liberata, l'ho sotterrata a quaresima. 

- N'hai sofferto molto, vero? - chiese Dar'ja Aleksandrovna. 

- E perché soffrire? Il vecchio, anche così, di nipoti ne ha tanti. È soltanto una preoccupazione. Non puoi lavorare, né fare altro. Non è che un legame. 

Questa risposta era parsa ripugnante a Dar'ja Aleksandrovna, malgrado l'aspetto buono della giovane donna; ma ora ella si ricordò involontariamente di quelle parole. In quelle ciniche parole c'era anche una parte di verità. 

``Già, così sempre - pensava Dar'ja Aleksandrovna, dopo aver dato uno sguardo a tutta la sua vita di quei quindici anni di matrimonio: - gravidanze, nausee, confusione mentale, indifferenza e, soprattutto, la deformità. Kitty, anche Kitty così giovane e carina, è diventata brutta e io, incinta, divento mostruosa, lo so. Il parto, le sofferenze mostruose, quegli ultimi momenti\ldots{} poi l'allattamento, quelle notti insonni, quei dolori terribili\ldots{}''. 

Dar'ja Aleksandrovna rabbrividì al solo ricordo del dolore ai capezzoli screpolati, che provava quasi ad ogni bambino. ``Poi le malattie dei bambini, quest'eterno terrore; poi l'educazione, le inclinazioni cattive - ricordò il delitto della piccola Maša fra i lamponi - lo studio, il latino, tutto questo è così incomprensibile e difficile. E al di sopra di tutto, la morte di questi stessi bambini''. E di nuovo nella sua mente si sollevò il ricordo crudele che sempre opprimeva il suo cuore di mamma, la morte dell'ultima bambina lattante, morta di crup, il funerale, l'indifferenza di tutti per la piccola bara rosa e il proprio dolore, solo, che le spezzava il cuore, dinanzi a quella minuta e pallida fronte con le tempie dai capelli ondulati, e la piccola bocca aperta e stupita, che appariva dalla bara nel momento in cui la coprivano d'un coperchio rosa con una croce di gallone. 
\enlargethispage*{1\baselineskip}

``E tutto questo perché? Che cosa risulterà mai da tutto questo? Che io passerò la mia vita senza un momento di pace, ora incinta, ora in periodo di allattamento, eternamente irritata, scontenta, tormentata io stessa e tormentatrice degli altri, invisa a mio marito, e che cresceranno dei figli infelici, male educati e poveri. E ora, se non avessimo trascorso questa estate dai Levin, non so come ce la saremmo passata. S'intende, Kostja e Kitty sono così delicati da non farcene accorgere; ma questo non può durare. Cominceranno ad avere anche loro dei bambini, non potranno più aiutarci e anche adesso sono a disagio. Che forse, ci aiuterà papà, cui non è rimasto quasi nulla? E, per dare una sistemazione ai miei figliuoli, non posso farlo da sola, ma soltanto con l'aiuto degli altri, attraverso umiliazioni. Ma via, pensiamo a qualcosa di più felice: di bambini non ne moriranno più e io in qualche modo li educherò. Nel caso migliore non saranno dei furfanti, ecco. Ecco tutto quello che posso desiderare. Per tutto questo, quanti tormenti, quante fatiche\ldots{} Tutta una vita rovinata!''. Le tornò di nuovo in mente quello che aveva detto la giovane donna, e di nuovo provò ribrezzo a rammentarsene; ma non poteva non convenire che in quelle parole c'era anche una parte di verità brutale. 
\enlargethispage{\baselineskip}

- È lontano, Michajla? - domandò Dar'ja Aleksandrovna allo scrivano, per distrarsi dai pensieri che la spaventavano. 

- Da questo villaggio, dicono, sette verste. 

La vettura, per la strada del villaggio, scendeva verso un ponticello. Per il ponte, discorrendo ad alta voce e allegramente, una folla di donne andava colle funi dei covoni legate dietro alle spalle. Le donne si fermarono sul ponte esaminando con curiosità la vettura. Tutti i visi rivolti verso di lei sembrarono a Dar'ja Aleksandrovna sani, allegri, quasi eccitanti in lei la gioia di vivere: ``Tutti vivono, tutti godono la vita - seguitò a pensare Dar'ja Aleksandrovna, oltrepassando le donne, e uscendo su di un poggio, cullata di nuovo piacevolmente sulle morbide molle del vecchio carrozzino al trotto - e io, liberata, come da una prigione, da un mondo che mi uccide di preoccupazioni, soltanto adesso, per un attimo, sono tornata in me. Tutti vivono: e queste donne e Natalie mia sorella, e Varen'ka e Anna dalla quale vado, soltanto io no''. 

``E si scagliano contro Anna. Perché mai? Io forse sono migliore? Io almeno ho un marito che amo. Non così come vorrei amarlo, ma lo amo e Anna non amava il suo. In che cosa è mai colpevole? Vuol vivere. Iddio, questo, ce l'ha messo dentro l'anima. Può darsi benissimo che anch'io avrei fatto lo stesso. E finora non so se ho fatto bene ad ascoltarla in quel momento terribile quando venne a Mosca. Allora io dovevo abbandonare mio marito e cominciare a vivere di nuovo. Potevo amare ed essere amata davvero. E adesso, è forse meglio? Non lo stimo. Mi è necessario - ella pensava del marito - e lo sopporto. È forse meglio? Allora potevo ancora piacere, mi rimaneva la bellezza'' seguitò a pensare Dar'ja Aleksandrovna, e le venne voglia di guardarsi nello specchio. Aveva uno specchietto da viaggio nella borsa e avrebbe voluto tirarlo fuori; ma guardando la schiena del cocchiere e quella dello scrivano che si dondolava, sentì che si sarebbe vergognata se uno di loro si fosse voltato, e non tirò fuori lo specchio. 

Ma pur senza guardarsi nello specchio, pensava che, adesso, era troppo tardi; e ricordò Sergej Ivanovic, particolarmente gentile con lei, un amico di Stiva, il buon Turovcyn che insieme con lei aveva curato i bambini durante la scarlattina e s'era innamorato di lei. E c'era stato anche un uomo proprio giovane che, come aveva detto scherzando il marito, riteneva ch'ella fosse la più bella delle sorelle. E i romanzi più appassionati e impossibili si presentavano alla mente di Dar'ja Aleksandrovna. ``Anna ha agito benissimo, e io, poi, non starò certo a rimproverarla. È felice, fa la felicità d'un'altra persona e non è avvilita come me, ma probabilmente fresca, sveglia, aperta a tutto, sempre allo stesso modo'' pensava Dar'ja Aleksandrovna, e un sorriso malizioso le increspava le labbra, proprio perché, pensando al romanzo di Anna, parallelamente ad esso, Dar'ja Aleksandrovna si figurava un suo romanzo, quasi simile, con un personaggio anonimo e collettivo che fosse innamorato di lei. Come Anna, ella avrebbe confessato tutto al marito. E lo stupore e la confusione di Stepan Arkad'ic, a questa notizia, la facevano sorridere. 

In queste fantasticherie giunse alla svolta che, dalla strada maestra, portava a Vozdvizenskoe. 

\capitolo{XVII}Il cocchiere fermò il tiro a quattro e si voltò a destra verso un campo di segala, dove stavano seduti, accanto a un carro, dei contadini. Lo scrivano parve voler scendere, ma poi cambiò idea e chiamò imperiosamente un contadino, facendogli segno di accostarsi. Quel po' di vento che c'era stato durante il cammino, era cessato quando si fermarono; i tafani si attaccavano ai cavalli sudati che se ne liberavano con rabbia. Il suono metallico di un martello sulla falce, che veniva da un carro, cessò. Uno dei contadini, alzatosi, si avviò verso la carrozza. 

- Guarda, non si regge in piedi! - gridò irritato lo scrivano al contadino che camminava piano coi piedi scalzi sulla strada asciutta, non battuta. - Vieni, via! 

Il vecchio riccioluto, coi capelli legati da uno stelo di tiglio, la schiena ricurva e scurita dal sudore, affrettato il passo, s'accostò alla vettura e afferrò con la mano abbronzata un parafango della carrozza. 

- Vozdvizenskoe? alla casa dei signori? dal conte? - ripeté. - Ecco, appena sbocca il viottolo. A sinistra c'è una svolta. Dritto per il viale e ci vai a finir dentro. Ma voi chi volete, proprio lui? 

- Sono in casa, amico mio? - disse senza precisare Dar'ja Aleksandrovna, non sapendo come domandare di Anna nemmeno a un contadino. 

- Devono essere in casa - disse il contadino, muovendo lentamente i piedi scalzi e lasciando nella polvere l'impronta chiara delle cinque dita del piede. - Devono essere in casa - egli ripeté, desiderando evidentemente di attaccar discorso. - Anche ieri sono arrivati degli ospiti. Di ospiti ce n'è sempre tanti\ldots{} Che vuoi? - disse, rivolto a un giovanotto che gli gridava qualcosa dal carro. - Anche quello! Son passati or ora tutti a cavallo per vedere la mietitrice. Adesso devono essere a casa. E voi di dove sareste? 

- Noi siamo di lontano - disse il cocchiere, salendo a cassetta. - Allora è vicino? 

- Vi dico che è proprio qui. Appena vai in là - egli disse, toccando con la mano il parafango della vettura. 

Un giovanotto sano, tarchiato, si avvicinò pure. 

- Be', non c'è lavoro per il raccolto? - domandò. 

- Non lo so, amico. 

- Allora, prendi a sinistra, troverai subito - diceva il contadino, lasciando evidentemente andar via malvolentieri i passanti, perché desiderava chiacchierare un po'. 

Il cocchiere si mosse, ma avevano appena voltato che il contadino si mise a gridare: 

- Ferma, ohi, amico! Fermati! - gridavano due voci. Il cocchiere si fermò. 

- Vengono! Eccoli! - gridò il contadino. - E guarda come montano bene! - disse, indicando le quattro persone a cavallo e le altre due in uno char à bancs che venivano per la strada. 

Erano Vronskij con un fantino, Veslovskij e Anna a cavallo e la principessa Varvara con Svijazskij nello char à bancs. Erano andati a fare una passeggiata e a vedere funzionare le mietitrici meccaniche fatte venire da poco. 

Quando la carrozza si fermò, le persone a cavallo si misero al passo. Avanti andava Anna, accanto a Veslovskij. Anna andava a passo tranquillo su di un piccolo cavallo inglese, basso e tozzo, con la criniera tagliata e la coda corta. La sua bella testa, con i capelli neri sfuggenti di sotto il cappello alto, le spalle piene, la vita sottile nell'amazzone nera, e la calma, aggraziata posizione in sella colpirono Dolly. 

Nel primo momento le parve sconveniente che Anna andasse a cavallo. Alla figura di una signora che cavalcava si univa, nella concezione di Dar'ja Aleksandrovna, l'immagine di una giovanile, leggera civetteria, che, secondo lei, non si addiceva alla situazione di Anna; ma quando l'ebbe esaminata da vicino, si riconciliò subito con la sua equitazione. Malgrado l'eleganza, tutto era così semplice, tranquillo e dignitoso, e nell'atteggiamento e nell'abito e nei movimenti di Anna, che non ci poteva essere nulla di più spontaneo. 

Accanto ad Anna, su di un accaldato cavallo grigio di cavalleria, allungando in avanti le gambe grasse ed evidentemente compiaciuto, cavalcava Vasen'ka Veslovskij col berrettino scozzese dai nastri svolazzanti. Dar'ja Aleksandrovna non poté trattenere un sorriso d'ilarità, dopo averlo riconosciuto. Dietro di loro veniva Vronskij. Cavalcava un cavallo baio scuro, purosangue, evidentemente accaldatosi nel galoppo. Egli, nel trattenerlo, lavorava di redini. 

Dietro di lui cavalcava un ometto in costume di fantino. Svijazskij e la principessa, in uno char à bancs nuovo fiammante, tirato da un grosso morello trottatore, tenevano dietro ai cavalieri. 

Il viso di Anna, nel momento in cui, nella piccola figura che si stringeva contro l'angolo della vecchia vettura, riconobbe Dolly, s'illuminò d'un tratto di un sorriso di gioia. Ella diede un grido, ebbe un sussulto sulla sella e mise il cavallo al galoppo. Avvicinatasi alla vettura, saltò giù senza aiuto e, sollevando l'amazzone, corse verso Dolly. 

- Lo pensavo e non osavo pensarlo. Che gioia! Non puoi immaginare la mia gioia!- diceva, ora stringendosi col viso a Dolly e baciandola, ora allontanandosi ed esaminandola con un sorriso. - Ecco, questa è una gioia, Aleksej! - ella disse, volgendosi a Vronskij che era sceso da cavallo e si avvicinava a loro. 

Vronskij, levatosi il cappello grigio alto, si accostò a Dolly. 

- Non potete credere come siamo contenti del vostro arrivo - disse, dando un significato particolare alle parole pronunciate e scoprendo nel sorriso i suoi forti denti bianchi. 

Vasen'ka Veslovskij, senza smontar da cavallo, si levò il berrettino e salutò l'ospite, agitando gioiosamente i nastri sopra il capo. 

- È la principessa Varvara - rispose Anna a uno sguardo interrogativo di Dolly, quando lo char à bancs si fu accostato. 

- Ah! - disse Dar'ja Aleksandrovna, e il suo viso espresse lo scontento. 

La principessa Varvara era una zia di suo marito, e lei da tempo la conosceva e non la stimava. Sapeva che la principessa Varvara aveva passato tutta la sua vita da parassita presso parenti ricchi; ma che adesso vivesse da Vronskij, persona per lei estranea, la offendeva per la parentela del marito. Anna notò l'espressione del viso di Dolly e si confuse, arrossì, lasciò sfuggire dalle mani l'amazzone e vi inciampò. 

Dar'ja Aleksandrovna si avvicinò allo char à bancs che s'era fermato e salutò freddamente la principessa Varvara. Anche Svijazskij era a lei noto. Egli domandò come stava quell'originale del suo amico con la giovane moglie e, esaminati con uno sguardo fuggevole i cavalli non appaiati e la vettura dai parafanghi rappezzati, offrì alle signore di montare nello char à bancs. 

- E io andrò in codesto veicolo - egli disse. - Il cavallo è tranquillo e la principessa guida benissimo. 

- No, restate così come siete - disse Anna che s'era avvicinata - e noi andremo nella vettura - e, presa Dolly sotto braccio, la condusse via. 

Dar'ja Aleksandrovna era abbagliata da quella vettura elegante non mai vista da lei, da quei cavalli bellissimi, da quelle persone mondane, raffinate che la circondavano. Ma più di tutto la stupiva il cambiamento avvenuto nell'Anna che conosceva e amava. Un'altra donna, meno attenta, che non avesse conosciuto Anna prima e che, soprattutto, non avesse avuto quei pensieri che Dar'ja Aleksandrovna aveva avuto durante il viaggio, non avrebbe neppure notato nulla di particolare in Anna. Ma adesso, Dolly era colpita da quella bellezza momentanea che le donne sogliono avere quando amano e che lei scopriva nel viso di Anna. Tutto, infatti, era tale nel viso di Anna: le fossette ben definite delle guance e del mento, la piega delle labbra, il riso, ch'era come se le errasse sul volto, lo scintillio degli occhi, la grazia e la rapidità dei movimenti, la pienezza dei suoni della voce, perfino il modo carezzevolmente irato col quale ella rispondeva a Veslovskij che le chiedeva il permesso di montare il suo cavallo inglese per insegnargli l'ambio della zampa destra, tutto era particolarmente attraente; e sembrava che ella ne fosse consapevole e ne gioisse. 

Quando le due donne furono sedute nella vettura, furono prese a un tratto da un certo disagio. Anna, per quello sguardo attentamente interrogativo con cui la guardava Dolly, si era confusa; Dolly, per le parole di Svijazskij sul veicolo, aveva involontariamente cominciato a vergognarsi della vecchia vettura sdrucita nella quale Anna si era seduta con lei. Il vecchio Filipp e lo scrivano provavano lo stesso sentimento. Lo scrivano, per nascondere il proprio disagio, si dava da fare, mettendo a sedere le signore, ma Filipp il cocchiere si era fatto scuro e s'era preparato fin d'ora a non sottostare a quella superiorità esteriore. Sorrise ironicamente, dopo aver guardato il trottatore morello e dopo aver già stabilito in cuor suo che il morello dello char à bancs era buono solo per la prominaz e che non avrebbe fatto quaranta verste, col caldo, in una sola tirata. 

I contadini s'erano alzati tutti dal carro e guardavano curiosi e allegri l'incontro degli ospiti, facendo le loro osservazioni. 

- Anche loro sono contente, è un pezzo che non si son viste - diceva il vecchio riccioluto, quello con il legaccio di tiglio. 

- Ecco, zio Gerasim, lo stallone morello sì che andrebbe alla svelta a portare i covoni! 

- Guarda un po'! Questa qui in pantaloni è una donna? - disse uno di loro indicando Vasen'ka Veslovskij che sedeva sulla sella da signora. 

- No, è un uomo. Guarda come è saltato su svelto! 

- Be', figliuoli, non dormiremo, vero? 

- Ma che dormire oggi! - disse il vecchio, e guardò il sole di traverso. - Mezzogiorno è passato, guarda! Prendi il forcone, comincia! 

\capitolo{XVIII}Anna guardava il viso magro, sfinito, cosparso di polvere nelle piccole rughe, di Dolly e voleva dire quello che pensava, che cioè Dolly era sciupata; ma, ricordando che lei era invece divenuta più bella e che lo sguardo di Dolly glielo diceva, sospirò e cominciò a parlare di sé. 

- Tu mi guardi - ella disse - e pensi come mai io possa essere felice nella mia situazione. Ma, non so! È vergognoso confessarlo; ma io\ldots{} sono imperdonabilmente felice. M'è accaduto qualcosa di magico, come quando in un sogno si prova spavento, impressione e a un tratto ci si sveglia e si sente che tutti quegli spaventi non esistono. Mi sono svegliata. Ho vissuto il tormento e la paura, e ora, già da tempo, ma in particolare da che siamo qui, sono felice!\ldots{} - ella disse, guardando Dolly con un timido sorriso di domanda. 

- Come sono contenta! - disse, sorridendo, Dolly, involontariamente più fredda di quanto volesse. - Sono molto contenta per te. Perché non m'hai scritto? 

- Perché?\ldots{} perché non osavo\ldots{} tu dimentichi la mia situazione\ldots{} 

- A me? non osavi? Se sapessi come io\ldots{} Io credo\ldots{} 

Dar'ja Aleksandrovna voleva dire i suoi pensieri della mattina, ma, chissà perché, questo le parve fuor di posto. 

- Del resto di questo parleremo poi. Cosa sono tutte queste costruzioni? - domandò, desiderando cambiar discorso e indicando i tetti rossi e verdi che si scorgevano di là dal verde vivo delle siepi di acacia e di serenelle. - Pare una piccola città. 

Ma Anna non rispondeva. 

- No! no! Cosa pensi mai della mia situazione, cosa pensi, cosa? - ella domandò. 

- Io suppongo\ldots{} - cominciò a dire Dar'ja Aleksandrovna, ma in quel momento Vasen'ka Veslovskij, messo al galoppo sul piede destro il piccolo cavallo inglese e battendo con la sua giacchetta corta contro la sella scamosciata da signora, passò loro accanto di galoppo. 

- Va bene, Anna Arkad'evna! - gridò. 

Anna non lo guardò neppure; ma di nuovo a Dar'ja Aleksandrovna parve che là nella vettura non fosse opportuno incominciare quel lungo discorso, e perciò abbreviò il suo pensiero. 

- Io non penso nulla - ella disse - ma t'ho sempre voluto bene, e quando si vuol bene, si vuol bene a tutta la persona così com'è, e non come si vuole che sia. 

Anna, allontanando gli occhi dal viso dell'amica e socchiudendoli (era questo un nuovo tratto che Dolly non le conosceva), si fece pensierosa, desiderando di afferrare in pieno il senso di queste parole. E avendole evidentemente capite così come voleva lei, guardò Dolly. 

- Se tu avessi delle colpe - ella disse - ti sarebbero tutte perdonate per la tua venuta e per codeste tue parole. 

E Dolly vide che le eran venute le lacrime agli occhi. Ella strinse in silenzio la mano ad Anna. 

- E allora che cosa sono queste costruzioni? Quante! - ella ripeté la sua domanda dopo un attimo di silenzio. 

- Sono le case degli impiegati, l'allevamento, le scuderie - rispose Anna. - E qui comincia il parco. Tutto ciò era abbandonato, ma Aleksej ha rinnovato tutto. Egli ama molto questa tenuta e, cosa che non m'aspettavo in nessun modo, è stato preso dalla passione per l'organizzazione dell'azienda. Del resto è una natura così ricca! Qualunque cosa intraprenda, fa tutto così bene. Non solo non si annoia, ma si occupa con passione. È diventato, così come lo conosco io, un padrone calcolatore, perfetto, perfino avaro nell'azienda. Ma solo nell'azienda. Là dove si tratta di decine di migliaia di rubli, non sta a calcolare - ella diceva con quel sorriso gioiosamente sagace con cui spesso le donne parlano delle particolarità misteriose, a loro sole rivelate, della persona amata. - Ecco, vedi questa grande costruzione? È un nuovo ospedale. Io penso che costerà più di centomila rubli. È il suo dada adesso. E sai perché è venuto fuori, questo? I contadini gli avevano chiesto di ceder loro più a buon mercato i prati, mi pare, ma lui aveva rifiutato, e io gli ho rimproverato la sua avarizia. Certo, non solo per questo, ma per tutto un insieme di cose, ha cominciato questo ospedale per mostrare, capisci, come non sia avaro. Se vuoi, c'est une petitesse; ma io l'amo ancora di più per questo. Ma ecco che vedrai subito la casa. È ancora la casa del nonno e non è affatto cambiata all'esterno. 

- Com'è bella! - disse Dolly, guardando con involontario stupore la bella casa con le colonne che sovrastava il verde variegato dei vecchi alberi del giardino. 

- Non è vero che è bella? E dalla casa, di sopra, c'è una veduta meravigliosa. 

Esse entrarono in un cortile cosparso di ghiaia e accomodato a giardino, in cui due operai circondavano di pietre porose, grezze, un'aiuola di fiori rimossa, e si fermarono nell'ingresso coperto. 

- Ah, sono già arrivati! - disse Anna, guardando i cavalli da sella che stavano portando via proprio allora dall'ingresso. - Vero che è bello questo cavallo? È un cob. È il mio preferito. Portatelo qua e datemi dello zucchero. Dov'è il conte? - ella chiese ai due servitori in livrea che erano sbucati fuori. - Ah, ecco anche lui! - ella disse, vedendo Vronskij che, con Veslovskij, le veniva incontro. 

- Dove metterete la principessa? - disse Vronskij in francese, rivolto ad Anna, e, senza aspettare la risposta, salutò ancora una volta Dar'ja Aleksandrovna e adesso le baciò la mano. - Io penso, nella camera grande col balcone. 

- Oh, no, è lontano! È meglio in quella d'angolo, ci vedremo di più. Su, andiamo - disse Anna che dava lo zucchero, portatole dal servitore, al cavallo preferito. 

- Et vous oubliez votre devoir - ella disse a Veslovskij che era uscito anche lui sulla scalinata. 

- Pardon, j'en ai tout plein les poches - rispose lui sorridendo, sprofondando le dita nelle tasche del panciotto. 

- Mais vous venez trop tard - ella disse, asciugando col fazzoletto la mano che le aveva bagnato il cavallo nel prender lo zucchero. Anna si voltò a Dolly. 

- Sei venuta per restare a lungo? Un giorno solo? È impossibile! 

- Così ho promesso, e i bambini\ldots{} - disse Dolly impacciata e perché doveva prendere la borsa dalla vettura e perché sapeva di avere il viso molto impolverato. 

- No, Dolly, amica mia\ldots{} Su, vedremo. Andiamo, andiamo! - e Anna condusse Dolly nella sua camera. 

Questa stanza non era quella di lusso, che aveva proposto Vronskij, ma una stanza per cui Anna disse a Dolly di scusarla. Ma anche questa stanza, della quale bisognava scusarsi, era di uno sfarzo nel quale Dolly non aveva mai vissuto e che le ricordò i migliori alberghi all'estero. 

- Ma come sono contenta, anima mia! - disse Anna, sedendosi per un attimo accanto a Dolly nella sua amazzone. - Raccontami dunque dei tuoi. Stiva l'ho visto di sfuggita. Ma lui non può raccontare dei bambini. Come sta Tanja, la mia preferita? È una bambina grande, penso. 

- Sì, molto grande - rispose breve Dar'ja Aleksandrovna, meravigliandosi lei stessa di rispondere così freddamente sui suoi figliuoli. - Stiamo benissimo dai Levin - ella aggiunse. 

- Ecco, se avessi saputo - disse Anna - che non mi disprezzi\ldots{} Sareste venuti tutti da noi. Perché Stiva è un vecchio e grande amico di Aleksej - ella aggiunse, e a un tratto arrossì. 

- Sì, ma ci stiamo così bene\ldots{} - rispose Dolly, confondendosi. 

- Ma del resto, è per la gioia che dico delle sciocchezze. Una cosa sola, amica mia, come sono contenta di vederti! - disse Anna, baciandola di nuovo. - Tu non mi hai ancora detto come e cosa pensi di me e io voglio sapere tutto. Ma sono contenta che tu mi vedrai così come sono. Io, soprattutto, non voglio che pensino che io tenda a dimostrare qualcosa. Io non voglio dimostrare nulla, voglio semplicemente vivere, non far del male a nessuno, tranne che a me. Di questo ho il diritto, vero? Ma questo è un discorso lungo e noi parleremo ancora di tutto. Ora andrò a vestirmi e ti manderò la donna. 

\capitolo{XIX}Rimasta sola, Dar'ja Aleksandrovna esaminò la camera con lo sguardo della padrona di casa. Tutto quello che aveva visto, avvicinandosi alla villa e attraversandola, e quello che ora vedeva nella propria camera, tutto produceva in lei un'impressione di ricchezza ed eleganza, e di lusso moderno, di cui aveva letto soltanto nei romanzi inglesi, ma che non aveva ancora visto in Russia e per di più in campagna. Tutto era moderno, dalle tappezzerie francesi fino al tappeto che ora era disteso per tutta la stanza. Il letto era a molle con un piccolo materasso e un capezzale speciale, le federe di seta cruda sopra i piccoli guanciali. Il lavabo di marmo, la specchiera, la sedia a sdraio, le tavole, l'orologio di bronzo sul camino, le tende e i tendaggi, tutto questo era costoso e nuovo. 

L'elegante cameriera ch'era venuta a offrire i suoi servigi, con un'acconciatura e un vestito più alla moda di quelli di Dolly, era egualmente costosa e alla moda quanto la stanza. A Dar'ja Aleksandrovna piaceva quella sua cortesia, quel lindore e ossequio, ma si sentiva a disagio con lei; si vergognava dinanzi a lei della propria camicetta rammendata che le era stata messa dentro per sbaglio, come per disgrazia. Si vergognava di quegli stessi rammendi e di quelle stesse toppe di cui a casa andava orgogliosa. A casa era chiaro che per sei camicette occorrevano ventiquattro aršiny di batista a sessantacinque copeche, il che formava più di quindici rubli, all'infuori delle guarnizioni e della manifattura, e che questi quindici rubli erano stati risparmiati. Ma dinanzi alla cameriera non già che si vergognasse, ma non era a suo agio. 

Dar'ja Aleksandrovna si sentì molto sollevata quando nella camera entrò Annuška, che ella conosceva da tempo. La cameriera elegante era richiesta dalla signora e Annuška rimase con Dar'ja Aleksandrovna. 

Annuška era evidentemente molto contenta dell'arrivo della signora e discorreva senza posa. Dolly notò che aveva voglia di esprimere la propria opinione sulla posizione della padrona, soprattutto sull'amore e sulla devozione del conte per Anna Arkad'evna, ma Dolly la trattenne con garbo, non appena cominciò a parlare. 

- Io sono cresciuta con Anna Arkad'evna, m'è più cara di tutto il mondo. E poi non dobbiamo essere noi a giudicare. E poi l'ama tanto, mi pare. 

- Allora, per favore, date a lavare, se è possibile - la interruppe Dar'ja Aleksandrovna. 

- Sissignora. Da noi, due donne sono addette alla lavatura, ma la biancheria si fa tutta a macchina. Il conte pensa a tutto. Altro che marito\ldots{} 

Dolly fu contenta quando Anna entrò da lei e con la sua presenza fece cessare il chiacchierio di Annuška. 

Anna aveva cambiato abito, e aveva indossato un abito di batista molto semplice. Dolly esaminò attentamente questo vestito semplice. Sapeva cosa significasse e a qual prezzo si acquistasse una simile semplicità. 

- Una vecchia amica - disse Anna di Annuška. 

Anna adesso non si vergognava più. Era del tutto libera e calma. Dolly vedeva che adesso s'era già riavuta in pieno dall'impressione che aveva prodotto in lei il suo arrivo e aveva già preso quel tono superficiale, indifferente, per cui pareva che la porta di quel reparto, dove si trovavano i suoi sentimenti e i suoi pensieri intimi, fosse preclusa. 

- E la tua bambina, Anna? - domandò Dolly. 

- Annie? - così ella chiamava sua figlia Anna. - Sta bene. Si è molto rimessa. Vuoi vederla? Andiamo, te la farò vedere. C'è stato un gran trambusto con le bambinaie - ella cominciò a raccontare. - Abbiamo avuto un'italiana per balia. Buona, ma così sciocca! La volevamo mandar via, ma la bambina s'era tanto abituata a lei, che la teniamo ancora. 

- E come vi siete accomodati?\ldots{} - cominciò Dolly e voleva chiedere che nome avrebbe portato la bambina; ma avendo notato il viso di Anna, divenuto improvvisamente cupo, cambiò il senso della domanda. - E come vi siete accomodati? L'avete già svezzata? 

Ma Anna aveva capito. 

- Non è questo che mi volevi chiedere. Non volevi chiedermi del suo nome? Vero? Questo tormenta Aleksej. Non ha nome. Cioè è una Karenina - disse Anna, socchiudendo gli occhi così che le si vedevano soltanto le ciglia unite. - Del resto - disse, rischiarandosi d'un tratto in viso - di questo diremo tutto dopo. Andiamo, te la farò vedere. Elle est très gentille. Si trascina già per terra. 

Nella camera della bambina, lo sfarzo, che in tutta la casa stupiva Dar'ja Aleksandrovna, la colpì ancora di più. C'erano carrettini, fatti venire dall'Inghilterra, strumenti per imparare a camminare, un divano fatto apposta a guisa di biliardo per trascinarsi, e sedie a dondolo e vasche speciali, moderne. Tutto era inglese, solido e di buona qualità e, evidentemente, molto costoso. La camera era grande, molto alta e luminosa. 

Quando esse entrarono, la bambina, con la sola camicina, stava seduta in una piccola poltrona accanto al tavolo e sorbiva un brodo che s'era versato addosso su tutto il piccolo petto. Dava da mangiare alla bambina, e visibilmente mangiava lei stessa insieme alla piccola, una ragazza russa che faceva il servizio nella camera della bambina. Non c'erano né la balia, né la njanja; erano nella camera attigua, e di là si sentiva il loro chiacchierio in uno strano francese, unica lingua nella quale riuscivano a intendersi. 

Udita la voce di Anna, l'inglese tutta adorna, alta, con un viso antipatico e una espressione disonesta, entrò per la porta, agitando i riccioli biondi in fretta, e cominciò subito a giustificarsi, benché Anna non l'accusasse di nulla. A ogni parola di Anna, l'inglese aggiungeva in fretta, varie volte: ``yes, my lady''. 

La bambina, dalle sopracciglia e i capelli neri, colorita, con un corpicino forte e colorito e dalla pelle aggricciata, malgrado l'espressione severa con cui guardò la persona nuova, piacque molto a Dar'ja Aleksandrovna; ella invidiò perfino il suo aspetto sano. E le piacque pure come si trascinava, quella piccola. Neanche uno dei suoi bambini s'era trascinato così. Quando la misero a sedere sul tappeto e le ficcarono da dietro il vestitino, fu straordinariamente graziosa. Voltandosi a guardare come una piccola belva, con i grandi occhi neri scintillanti, evidentemente rallegrata dall'ammirazione, sorridendo e tenendo le gambette di traverso, s'appoggiava con forza sulle mani e traeva a sé tutto il sederino e poi di nuovo s'aggrappava avanti con le manine. 

Ma tutto l'insieme della camera della bambina e in particolare l'inglese, non piacquero per nulla a Dar'ja Aleksandrovna. Solo col fatto che, in una famiglia irregolare come quella di Anna, una buona donna non ci sarebbe andata, Dar'ja Aleksandrovna si spiegò come proprio Anna, con la sua conoscenza delle persone, avesse potuto assumere per la propria creatura una inglese così poco simpatica, così poco rispettabile. Inoltre, da alcune parole, Dar'ja Aleksandrovna capì che Anna, la balia, la njanja e la bambina non erano affiatate per nulla e che una visita da parte della madre era una cosa insolita. Anna voleva tirar fuori alla bambina un giocattolo e non riuscì a trovarlo. 

La cosa, poi, più sorprendente di tutto fu che, alla domanda su quanti denti avesse, Anna si sbagliasse e non sapesse affatto degli ultimi due. 

- A volte è penoso per me essere come superflua qui - disse Anna, uscendo dalla camera della bambina e sollevando il proprio strascico per evitare i giocattoli che si trovavano accanto alla porta. - Non era così col primo. 
\enlargethispage{\baselineskip}

- Io pensavo al contrario - disse timida Dar'ja Aleksandrovna. 

- Oh, no! Perché lo sai, ho visto Serëza - disse Anna, socchiudendo gli occhi, come se osservasse qualcosa lontano. - Del resto questo lo diremo dopo. Tu non ci crederai, ma io sono proprio come un'affamata alla quale, d'un tratto, abbiano messo dinanzi un intero pranzo, e che non sa da che cosa cominciare. Il pranzo completo sei tu e i miei discorsi con te, che non potevo avere con altri; e non so da quale di questi incominciare. Mais je ne vous ferai grâce de rien. Già, devo farti un quadro della compagnia che troverai da noi - ella cominciò. - Comincio dalle signore. La principessa Varvara. Tu la conosci e io conosco l'opinione tua e di Stiva su di lei. Stiva dice che tutto lo scopo della sua vita consiste nel dimostrare la propria superiorità sulla zia Katerina Pavlovna; questo è vero, ma è buona, e io le sono riconoscente. A Pietroburgo c'è stato un momento in cui mi era indispensabile avere un chaperon. In quel momento è capitata lei. Ma davvero è buona. M'ha alleviato molto la situazione. M'accorgo che non ti rendi conto di tutta la difficoltà della mia posizione\ldots{} là a Pietroburgo - aggiunse. - Qui sono completamente tranquilla e felice. Ma di questo, dopo. Facciamo l'elenco. Dopo c'è Svijazskij, maresciallo della nobiltà e persona molto a modo, ma ha bisogno di qualcosa da Aleksej. Capirai, adesso che ci siamo stabiliti in campagna, Aleksej, con il suo patrimonio, può avere una grande influenza. Poi Tuškevic, l'hai visto, era addetto a Betsy. Ora l'hanno licenziato, e lui è venuto da noi. Come dice Aleksej, è una di quelle persone che vanno prese per quel che vogliono parere, et puis, il est comme il faut, come dice la principessa Varvara. Dopo c'è Veslovskij\ldots{} questo lo conosci. Un ragazzo molto simpatico - disse, e un sorriso malizioso le increspò le labbra. - Cos'è mai quella selvaggia storia con Levin? Veslovskij l'ha raccontata ad Aleksej e noi non ci crediamo. Il est très gentil et näif - disse, di nuovo con lo stesso sorriso. - Gli uomini hanno bisogno di distrazione, ed Aleksej ha bisogno di pubblico, perciò io mi procuro tutta questa compagnia. Bisogna che da noi ci sia animazione e allegria, e che Aleksej non desideri nulla di nuovo. Poi vedrai l'amministratore. Un tedesco molto bravo che sa il suo mestiere. Aleksej lo apprezza molto. Dopo, il dottore, un giovanotto, non è proprio un nichilista, ma sai, mangia col coltello\ldots{} eppure è un bravo medico. Dopo c'è l'architetto\ldots{} Une petit cour. 

\capitolo{XX}-Eccovi anche Dolly, principessa, volevate tanto vederla - disse Anna, uscendo insieme con Dar'ja Aleksandrovna su di una grande terrazza in muratura, sulla quale, all'ombra, ricamando una poltrona per il conte Aleksej Kirillovic, sedeva al telaio la principessa Varvara. - Lei dice di non voler prendere nulla prima di pranzo, ma voi fate portar la colazione, e io andrò a cercare Aleksej e li condurrò tutti qua. 

La principessa Varvara accolse Dolly affabilmente e con aria di protezione e subito cominciò a spiegarle che s'era stabilita da Anna perché le aveva voluto sempre bene, più di sua sorella Katerina Pavlovna, quella stessa che aveva allevato Anna, e che adesso, quando tutti l'avevano abbandonata, lei aveva ritenuto suo dovere aiutarla, in questo periodo transitorio che era il più difficile. 

- Il marito le darà il divorzio e allora io rientrerò di nuovo nella mia solitudine, ma ora posso essere utile e compirò il mio dovere, per quanto mi sia pesante, e non farò come gli altri. E come sei cara, come hai fatto bene a venire! Essi vivono proprio come i migliori coniugi; li giudicherà Iddio, non noi. E che forse Birjuzovskij e la Aveneva\ldots{} E lo stesso Nikandrov, e Vasil'ev con la Mamonova, e Liza Neptunova\ldots{} Del resto chi diceva nulla? E andò a finire che tutti li ricevevano. E dopo, c'est un interieur si joli, si comme il faut. Tout-à-fait à l'anglaise. On se réunit le matin au breakfast et puis on se sépare. Ognuno fa quello che vuole fino a pranzo. Il pranzo è alle sette. Stiva ha fatto bene a mandarti. Egli deve tenersi unito a loro. Sai, lui, per mezzo di sua madre e del fratello, può ottenere tutto. Poi fanno molta beneficenza. Non t'ha parlato del suo ospedale? Ce sera admirable, viene tutto da Parigi. 

La loro conversazione fu interrotta da Anna, che aveva trovato la compagnia degli uomini nella stanza del biliardo e ritornava insieme con loro sulla terrazza. Per il pranzo ci voleva ancora molto; il tempo era splendido e perciò furono proposti vari modi di passare quelle due ore che rimanevano. Modi di passare il tempo ce ne erano molti a Vozdvizenskoe, ed erano del tutto diversi da quelli che erano in uso a Pokrovskoe. 

- Une partie de lawn tennis - propose Veslovskij, sorridendo col suo bel sorriso. - Di nuovo con voi, Anna Arkad'evna. 

- No, fa caldo; meglio passare per il giardino e andare a fare una passeggiata in barca, far vedere a Dar'ja Aleksandrovna le rive - propose Vronskij. 

- Io sono d'accordo su tutto - disse Svijazskij. 

- Io penso che per Dolly la cosa più piacevole sia passeggiare un po', non è vero? E poi in barca - disse Anna. 

Fu stabilito proprio così. Veslovskij e Tuškevic andarono al bagno e promisero di preparare là una barca e di aspettare. 

Si misero in cammino per un sentiero in due coppie: Anna con Svijazskij e Dolly con Vronskij. Dolly era un po' turbata e preoccupata dall'ambiente per lei del tutto nuovo in cui era venuta a trovarsi. In astratto, in teoria, non solo giustificava, ma approvava perfino l'atto di Anna. Come in generale le donne moralmente irreprensibili, non di rado stanche dell'uniformità della loro vita morale, così ella, da lontano, non solo scusava l'amore colpevole, ma l'invidiava perfino. Inoltre, voleva bene di cuore ad Anna. Ma in realtà, vistala in quell'ambiente di persone per lei estranee, di gran tono, inusitato per Dar'ja Aleksandrovna, si sentiva a disagio. Spiacevole soprattutto era stato per lei trovarvi la principessa Varvara, che perdonava tutto per quelle comodità di cui profittava. 

In generale, Dolly approvava in astratto il gesto di Anna, ma veder l'uomo per cui era stato compiuto quel gesto le era spiacevole. Inoltre, Vronskij non le era mai piaciuto. Lo riteneva molto orgoglioso e non vedeva in lui nulla di cui potesse inorgoglirsi, all'infuori della ricchezza. Ma, contro la sua volontà, là, a casa sua, egli le incuteva ancor più soggezione, e non riusciva a sentirsi disinvolta con lui. Provava una sensazione simile a quella che aveva provato con la cameriera per la camicetta. Come dinanzi alla cameriera, non che si vergognasse, ma non si sentiva a suo agio per i rammendi, così di continuo, anche con lui, non che si vergognasse, ma si sentiva a disagio per se stessa. 

Dolly s'era confusa, e cercava un argomento di conversazione. Pur ritenendo che, col suo orgoglio, gli dovessero spiacere le lodi della casa e del giardino, ella, non trovando altro argomento di conversazione, gli disse tuttavia che le era piaciuta molto la casa. 

- Sì, è una costruzione molto bella e in buono stile antico - egli disse. 

- M'è piaciuto molto il cortile davanti alla scalinata. Era così? 

- Oh, no! - egli disse, e il viso gli s'illuminò di piacere. - Se aveste veduto quel cortile, questa primavera! 

Ed egli cominciò, prima cauto, poi accalorandosi sempre più, ad attrarre l'attenzione di lei sui vari particolari della casa e del giardino. Si vedeva che, avendo dedicato molto lavoro al miglioramento e all'abbellimento della sua casa signorile, Vronskij sentiva la necessità di vantarsene dinanzi a una persona nuova, e si rallegrava con tutta l'anima delle lodi di Dar'ja Aleksandrovna. 

- Se volete dare un'occhiata all'ospedale e non siete stanca, non è lontano. Andiamo - egli disse, dopo averla guardata in viso per convincersi che ella proprio non s'annoiasse. - Tu vieni, Anna? - si rivolse a lei. 

- Andiamo, non è vero? - ella disse rivolta a Svijazskij - Mais il ne faut pas laisser le pauvre Veslovskij et Tuškevic se morfondre là dans le bateau. Bisogna mandare qualcuno ad avvisarli. 

- Sì, questo è un monumento ch'egli pone qui - disse Anna, rivolgendosi a Dolly con quello stesso sorriso malizioso, consapevole, col quale prima ella aveva parlato dell'ospedale. 

- Oh, una cosa monumentale! - disse Svijazskij. Ma poi per non mostrarsi ammiratore incondizionato di Vronskij, aggiunse subito un'osservazione lievemente critica. - Ma mi sorprendo, conte - disse - come voi, che fate tanto per la salute del popolo, siate così indifferente alle scuole. 

- C'est devenu tellement commun, les écoles - disse Vronskij. - Voi capite, non è certo per questo, ma così, mi ci sono appassionato. Allora bisogna passare di qua per andare all'ospedale - disse rivolto a Dar'ja Aleksandrovna, indicando un'uscita laterale del viale. 

Le signore aprirono gli ombrelli e uscirono sul sentiero laterale. Dopo aver passato alcune svolte e dopo essere entrate attraverso una porticina, Dar'ja Aleksandrovna vide dinanzi a sé, su di un luogo alto, una costruzione grande, rossa, di forma ingegnosa, già quasi finita. Il tetto di ferro, non ancora verniciato, scintillava in modo accecante al sole vivo. Accanto alla costruzione finita, ne spuntava un'altra circondata di impalcature, e sopra i ponti gli operai in grembiule ponevano i mattoni, versavano dai secchi la calcina e la spianavano con le cazzuole. 

- Come va in fretta da voi il lavoro! - disse Svijazskij. - Quando ci sono stato l'ultima volta non c'era ancora il tetto. 

- In autunno sarà tutto pronto. Di dentro è già quasi tutto rifinito - disse Anna. 

- E questo nuovo edificio cos'è mai? 

- È il locale per il dottore e la farmacia - rispose Vronskij, avendo visto l'architetto in cappotto corto che si avvicinava loro, e, dopo aver chiesto scusa alle signore, gli andò incontro. 

Fatto il giro della buca dalla quale gli operai tiravano fuori la calce ammonticchiandola, si fermò con l'architetto e prese a parlare di qualcosa con calore. 

- Il frontone riesce sempre più basso - rispose ad Anna che aveva domandato di che cosa si trattava. 

- Lo dicevo che bisognava sollevare le fondamenta - disse Anna. 

- Già, s'intende, sarebbe stato meglio, Anna Arkad'evna - disse l'architetto - ma ormai è stato trascurato. 

- Sì, me ne interesso molto - rispose Anna a Svijazskij, che aveva espresso meraviglia per le sue nozioni di architettura. - Bisogna che la nuova costruzione corrisponda all'ospedale. Ma è stata concepita dopo e cominciata senza progetto. 

Terminata la conversazione con l'architetto, Vronskij si unì alle signore e le condusse nell'interno dell'ospedale. 

Mentre, all'esterno, finivano ancora di fare i cornicioni e al piano di sotto verniciavano, di sopra quasi tutto era già rifinito. Dopo essere passati per una scala larga di ghisa, su di un pianerottolo, entrarono nella prima stanza grande. I muri erano intonacati di gesso, uso marmo, le finestre enormi interne erano già incastrate, soltanto il pavimento di legno non era ancora finito e i falegnami che piallavano un riquadro sollevato sospesero il lavoro per salutare i signori, togliendosi i legacci che trattenevano loro i capelli. 

- Questa è la sala da ricevimento - disse Vronskij; - qui ci saranno un banco, un tavolo, un armadio e nient'altro. 

- Di qua, andiamo di qua. Non accostarti alla finestra - disse Anna, provando se la vernice s'era asciugata. - Aleksej, la vernice è già asciutta - aggiunse. 

Dalla sala di ricevimento passarono in un corridoio. Qui Vronskij mostrò loro come funzionava la ventilazione, con un sistema nuovo. Poi mostrò le vasche da bagno di marmo, i letti con delle molle straordinarie. Dopo mostrò uno dopo l'altro i reparti, la dispensa, la stanza per la biancheria, poi le stufe di nuova foggia, le carriole fatte in modo da non produrre rumore nel trasportare la roba per il corridoio, e molte altre cose. Svijazskij apprezzava tutto come persona che conosce tutti i nuovi perfezionamenti. Dolly si meravigliava semplicemente di quello che finora non aveva mai veduto e, desiderando di capir tutto, chiedeva spiegazioni, il che faceva evidente piacere a Vronskij. 

- Sì, io penso che sarà in Russia l'unico ospedale attrezzato in modo pienamente adeguato - disse Svijazskij. 

- E non avrete un reparto di maternità? - domandò Dolly. - È così necessario in campagna. Io spesso\ldots{} 

Malgrado la sua cortesia, Vronskij la interruppe. 

- Questa non è una clinica ostetrica, ma un ospedale, ed è destinato a tutte le malattie, tranne le contagiose - egli disse. - E questo, guardate un po'\ldots{} - ed egli fece rotolare verso Dar'ja Aleksandrovna una poltrona per convalescenti fatta venire da poco. - Guardate. - Egli si sedette sulla poltrona e cominciò a muoverla. - Uno non può camminare, è ancora debole o ha una malattia alle gambe, ma ha bisogno d'aria, e va, passeggia\ldots{} 

Dar'ja Aleksandrovna si interessava di tutto, tutto le piaceva molto, ma ancor più le piaceva lo stesso Vronskij con quella spontanea, ingenua esaltazione. ``Sì, è una persona molto simpatica, buona'' ella pensava a volte, senza ascoltarlo, e guardandolo e penetrandone l'espressione, si trasportava col pensiero in Anna. Egli le piaceva talmente, adesso, nella sua animazione, che capiva come Anna avesse potuto innamorarsene. 

\capitolo{XXI}-No, la principessa è stanca, penso, e i cavalli non la interessano - disse Vronskij ad Anna che aveva proposto di andare fino all'allevamento, dove Svijazskij voleva vedere uno stallone nuovo. - Voi andate, e io accompagnerò a casa la principessa, e discorreremo un po' - egli disse - se vi fa piacere - aggiunse rivolto a lei. 

- Di cavalli non capisco nulla, e sono molto contenta - disse Dar'ja Aleksandrovna alquanto sorpresa. 

Vedeva dal viso di Vronskij ch'egli aveva bisogno di qualcosa da lei. Non s'era sbagliata. Non appena entrarono, per la porticina, di nuovo nel giardino, egli guardò dalla parte dove era Anna, e, rassicuratosi ch'ella non potesse né sentirli né vederli, cominciò: 

- Avete indovinato che desideravo parlare un po' con voi - disse, guardandola con gli occhi ridenti. - Non mi sbaglio ritenendovi amica di Anna. - Si tolse il cappello e, tirato fuori il fazzoletto, s'asciugò la testa, che andava diventando calva. 

Dar'ja Aleksandrovna non rispose nulla e lo guardò soltanto come spaurita. Rimasta sola con lui, aveva avuto paura tutto ad un tratto: gli occhi ridenti e l'espressione severa del viso la spaventavano. 

Le più varie supposizioni su quello ch'egli stava per dirle, le balenarono in mente; prenderà a dirmi di voler venire a trovar noi e i bambini, e io dovrò dirgli di no; oppure di predisporre un certo ambiente per Anna a Mosca\ldots{} O che non sia Vasen'ka Veslovskij e dei suoi rapporti con Anna? o forse di Kitty, del fatto ch'egli si senta colpevole? Non prevedeva se non cose spiacevoli, ma non indovinò quello di cui egli voleva parlarle. 

- Voi avete un tale potere su di Anna, vi vuole tanto bene - disse - aiutatemi. 

Dar'ja Aleksandrovna guardava intimidita e interrogativa il viso energico di lui che, tutto o in parte, ora usciva in una striscia di sole fra l'ombra dei tigli, ora si oscurava di nuovo di ombra: e aspettava quello ch'egli avrebbe detto ancora, ma lui, urtando con un bastone nei ciottoli, camminava in silenzio accanto a lei. 

- Se siete venuta da noi, unica donna fra le vecchie amicizie di Anna (non conto la principessa Varvara) capisco che l'avete fatto non perché consideriate regolare la nostra posizione, ma perché, comprendendo tutta la gravità di questa posizione, le volete bene egualmente e desiderate aiutarla. È così, vi ho capita? - chiese voltandosi a guardarla. 

- Oh, sì - rispose Dar'ja Aleksandrovna, chiudendo l'ombrellino, - ma\ldots{} 

- No - egli l'interruppe e, senza volere, dimenticando di porre la propria interlocutrice in una posizione disagevole, si fermò così che anche lei dovette fermarsi. - Nessuno sente più fortemente di me tutta la gravità della posizione di Anna. E questo è comprensibile, se voi mi fate l'onore di considerarmi un uomo di cuore. Io sono la causa di questa situazione e perciò la sento. 

- Capisco - disse Dar'ja Aleksandrovna, involontariamente ammirandolo per la sua sincerità e la fermezza con cui aveva pronunciato ciò. - Ma proprio perché sentite di esserne la causa, temo che, esageriate. La sua posizione in società è penosa, capisco. 

- Nella società è un inferno! - egli pronunciò in fretta, accigliandosi. - Non si può immaginare tormento morale peggiore di quello che lei ha sofferto a Pietroburgo in due settimane\ldots{} e vi prego di crederlo. 

- Sì, ma qui, fino a che né Anna\ldots{} né voi sentiate il bisogno della società\ldots{} 

- La società! - egli disse con disprezzo - che bisogno posso avere io della società? 

- Fino ad allora, e potrebbe essere sempre, voi sarete felici e tranquilli. Io vedo che Anna è felice, completamente felice, ha già avuto il tempo di comunicarmelo - disse Dar'ja Aleksandrovna, sorridendo; e senza volere, proprio nel dire questo, ella dubitò che Anna fosse totalmente felice. 

Ma Vronskij sembrava non dubitarne. 

- Sì, sì - egli disse. - Lo so che s'è ripresa dopo tutte le sue sofferenze, è felice. È felice del presente. Ma io?\ldots{} io ho paura di quello che ci attende\ldots{} Perdonate, volete camminare? 

- No, è lo stesso. 

- Su, allora sediamoci qui. 

Dar'ja Aleksandrovna sedette su una panchina del giardino, in un angolo del viale. Egli rimase in piedi dinanzi a lei. 

- Vedo che lei è felice - egli ripeté, e il dubbio ch'ella fosse felice colpì ancor di più Dar'ja Aleksandrovna. - Ma può mai durare così? Che noi abbiamo agito bene o male è un'altra questione; ma il dado è tratto - egli disse, passando dal russo al francese - e noi siamo legati per tutta la vita. Siamo uniti con il vincolo per noi più santo dell'amore. Abbiamo una creatura, possiamo averne ancora. Ma la legge e tutte le circostanze della nostra situazione sono tali che si presentano migliaia di complicazioni, che lei adesso, distendendo l'animo dopo tante prove e sofferenze, non vede e non vuol vedere. E questo è comprensibile. Ma io non voglio non vedere. Mia figlia, secondo la legge, non è figlia mia, ma è una Karenina. Io non voglio questo inganno, no! - egli disse con un gesto energico di rifiuto e guardò con cupa interrogazione Dar'ja Aleksandrovna. 

Ella non rispondeva nulla e lo guardava soltanto. Egli continuò. 

- Domani nascerà un figlio, il figlio mio, e secondo la legge è un Karenin, non è l'erede del mio nome, né del mio patrimonio, e per quanto felici noi potremo essere in famiglia, e per quanti figli potremo avere, fra me e loro non vi sarà legame. Sarebbero dei Karenin. Voi vi rendete conto della gravità e dell'orrore di questa situazione? Ho provato a parlarne ad Anna. Ciò la irrita. Non capisce e io non posso dir tutto a lei. Adesso guardate la cosa da un altro lato. Io sono felice del suo amore, ma devo avere un'occupazione. Ho trovato questa occupazione, e ne sono orgoglioso, e la considero più nobile delle occupazioni dei miei colleghi di un tempo a corte e in servizio. E senza dubbio, ormai, non cambierò questo lavoro con il loro. Io lavoro qui, sempre allo stesso posto, e sono felice, soddisfatto e non ci occorre altro per essere felici. Io amo questa attività. Cela n'est pas un pis-aller, al contrario\ldots{} 

Dar'ja Aleksandrovna notò che in questo punto della spiegazione egli si confondeva, e a lei non era possibile capire bene la digressione, ma sentiva che, messosi una volta a parlare dei suoi rapporti intimi, di cui non poteva parlare con Anna, egli adesso rivelava tutto, e la questione della sua attività in campagna rientrava nello stesso reparto di pensieri intimi, così come la questione dei suoi rapporti con Anna. 

- E così, continuo - egli disse, dopo essersi ripreso. - La cosa principale poi è che, lavorando, mi è indispensabile avere la certezza che ciò che si fa non morirà con me, che avrò degli eredi; e questa certezza io non l'ho. Figuratevi la situazione di un uomo che sa in precedenza che i figli suoi e della donna che ama non saranno suoi, ma di un essere che li odia e non vuol saperne di loro. Questo è tremendo! 

Tacque visibilmente agitato. 

- Sì, s'intende, capisco. Ma cosa può mai Anna? - domandò Dar'ja Aleksandrovna. 

- Ecco, questo mi porta allo scopo del mio discorso - disse lui, dominandosi con sforzo. - Anna può, ciò dipende da lei\ldots{} Perfino per chiedere la legittimazione allo zar, è necessario il divorzio. E ciò dipende da Anna. Suo marito in qualunque momento è stato consenziente al divorzio, e vostro marito stava proprio per predisporlo. E ora, lo so, egli non si rifiuterebbe. Basterebbe solo scrivergli. Allora aveva risposto apertamente che, se lei ne avesse espresso il desiderio, egli non avrebbe opposto un rifiuto. S'intende - disse cupo - è una di quelle crudeltà farisaiche di cui sono capaci solo uomini senza cuore. Egli sa quale tormento le costi ogni ricordo suo e, conoscendola, esige una lettera da lei. Capisco che per lei ciò sia tormentoso. Ma le ragioni sono così gravi che bisogna passer pardessus toutes ces finesses de sentiment. Il y va du bonheur et de l'existence d'Anne et de ses enfants. Io non parlo di me, pur essendo in uno stato penoso, molto penoso - disse con una espressione di minaccia verso qualcuno per il fatto che era in uno stato penoso. - E così, principessa, io mi aggrappo senza riguardi a voi, come all'ancora di salvezza. Aiutatemi a convincerla a scrivergli e ad esigere il divorzio! 

\begin{itemize} \itemsep1pt\parskip0pt\parsep0pt \item Sì, s'intende - disse pensierosa Dar'ja Aleksandrovna, ricordando chiaramente l'ultimo suo incontro con Aleksej Aleksandrovic. - Sì, s'intende - ella ripeté decisa, ricordandosi di Anna. \end{itemize} 

- Adoperate tutto il vostro ascendente su di lei, fate in modo che ella scriva. Io non voglio e non posso quasi parlare di questo con lei. 

- Va bene, parlerò. Ma come, non ci pensa lei stessa? - disse Dar'ja Aleksandrovna, ricordando a un tratto la nuova strana abitudine di Anna di socchiudere gli occhi. E ricordò che Anna socchiudeva gli occhi, proprio quando si trattava delle questioni intime della sua vita. ``Proprio come se li socchiudesse dinanzi alla sua vita, per non vedere tutto'' pensò Dolly. - Assolutamente, e per me e per lei le parlerò - rispose Dar'ja Aleksandrovna, all'espressione grata di lui. 

Si alzarono e andarono verso casa. 

\capitolo{XXII}Trovata Dolly già di ritorno, Anna la guardò attenta negli occhi, come a interrogarla di quella conversazione ch'ella aveva avuto con Vronskij, ma non chiese nulla. 

- Mi pare che sia già tempo di andare a pranzo - disse. - Non ci siamo ancora viste per nulla. Ma io conto sulla serata. Adesso bisogna andare a cambiarsi. Anche tu, penso. Ci siamo tutti inzaccherati là, sulla costruzione. 

Dolly andò in camera sua, e le venne da ridere. Per cambiarsi non aveva nulla, perché aveva già indossato il vestito migliore; ma per segnalare in qualche modo la propria acconciatura per il pranzo, pregò la cameriera di pulirle il vestito, cambiò i manichini e il nastro e mise dei pizzi in testa. 

- Ecco tutto quello che ho potuto fare - disse, sorridendo ad Anna che venne da lei in un terzo abito, di nuovo straordinariamente semplice. 

- Sì, noi qui siamo molto ricercati - ella disse, quasi a scusarsi della propria eleganza. - Aleksej è veramente contento del tuo arrivo come raramente gli accade di esserlo di qualcosa. È proprio innamorato di te - soggiunse. - E tu non sei stanca? 

Prima di pranzo, non c'era tempo di parlare di qualcosa. Entrate nel salotto, vi trovarono già la principessa Varvara e gli uomini in finanziera nera. L'architetto era in frac. Vronskij presentò all'ospite il dottore e l'amministratore. L'architetto era stato già presentato all'ospedale. 

Un maggiordomo grasso che risplendeva in un viso tondo e rasato, con un nodo inamidato alla cravatta bianca, riferì che le pietanze erano pronte, e le signore si alzarono. Vronskij pregò Svijazskij di dare il braccio ad Anna Arkad'evna, e lui stesso si avvicinò a Dolly. Veslovskij diede il braccio alla principessa Varvara prima di Tuškevic, così che Tuškevic con l'amministratore e il dottore si avviarono soli. 

Il pranzo e la sala da pranzo, le stoviglie, la servitù, i vini e i cibi non solo corrispondevano al tono generale di rinnovato sfarzo della casa, ma sembrava che fossero ancora più eleganti e moderni. Dar'ja Aleksandrovna osservava questo sfarzo per lei inusitato e, come padrona che mandava avanti una casa, senza speranza alcuna di poter mai applicare alla propria nulla di tutto quello che vedeva, tanto tutto questo era lontano per lusso dal suo tenore di vita, ne osservava senza volere tutti i particolari e si chiedeva chi avesse predisposto tutto ciò e come. Vasen'ka Veslovskij, suo marito e perfino Svijazskij e molte persone che lei conosceva, non ci pensavano proprio e credevano realmente quello che ogni bravo padrone di casa aspira a far sentire ai propri ospiti, che cioè tutto quello che da lui è così bene organizzato non sia costato alcun lavoro a lui, padrone di casa, ma si sia compiuto da sé. Dar'ja Aleksandrovna invece sapeva che da sé non si fa neanche la farinata per la colazione dei bambini e che, perciò, per un'organizzazione così complessa e perfetta, doveva essere stata impiegata la cura intensa di qualcuno. E dallo sguardo di Aleksej Kirillovic, da come egli esaminò la tavola e fece un cenno col capo al maggiordomo, da come offrì a Dar'ja Aleksandrovna la scelta tra la botvin'ja e la soupe, ella capì che tutto era stato disposto e diretto dalle cure dello stesso padrone di casa. Era evidente che tutto ciò non dipendeva da Anna più che da Veslovskij. Ella, Svijazskij, la principessa e Veslovskij erano egualmente ospiti, che profittavano allegramente di quello che veniva loro preparato. 

Anna si rivelava padrona di casa soltanto nel dirigere la conversazione. E questa conversazione, molto difficile per una padrona di casa con una tavola non grande, con persone come l'amministratore e l'architetto, di tutt'altro mondo, che cercavano di non intimidirsi dello sfarzo inusitato, ma che non potevano partecipare a lungo alla conversazione generale, questa difficile conversazione, ella la dirigeva con il suo tatto abituale, con naturalezza e quasi con soddisfazione, come notava Dar'ja Aleksandrovna. 

La conversazione si aggirò su come Tuškevic e Veslovskij fossero andati soli in barca, e Tuškevic allora prese a raccontare delle ultime regate allo Yacht-club a Pietroburgo. Ma Anna, aspettata un'interruzione, si rivolse subito all'architetto per trarlo fuori dal suo silenzio. 

- Nikolaj Ivanyc è rimasto sorpreso - disse a proposito di Svijazskij - come sia andata avanti la nuova costruzione da che egli è stato qui l'ultima volta; ma io stessa che ci vado ogni giorno, ogni giorno mi sorprendo di come proceda alla svelta. 

- Col signor conte si lavora bene - disse con un sorriso l'architetto (era un uomo cosciente del proprio merito, rispettoso e calmo). - Non è come avere a che fare con le autorità del governatorato. Là dove si riempirebbe una risma di carta, io riferisco al conte, si discute e tutto è bell'e fatto in tre parole. 

- All'americana - disse Svijazskij sorridendo. 

- Già, là gli edifici si elevano razionalmente\ldots{} 

La conversazione passò agli abusi dei poteri negli Stati Uniti, ma Anna la portò subito su di un altro tema per tirar fuori dal silenzio l'amministratore. 

- Hai mai visto una mietitrice americana? - disse rivolta a Dar'ja Aleksandrovna. - Eravamo andati a vederla quando t'abbiamo incontrato. Io stessa la vedevo per la prima volta. 

- E come funzionano? - chiese Dolly. 

- Proprio come forbici. Una tavola e molte piccole forbici. Ecco, così. 

Anna prese con le sue belle mani coperte di anelli un coltello e una forchetta e cominciò a far vedere. Ella, evidentemente, vedeva che dalla sua spiegazione non si sarebbe capito nulla; ma sapendo di parlare piacevolmente e di avere le mani belle, continuò la spiegazione. 

- Piuttosto come dei temperini - disse provocante Veslovskij, che non staccava gli occhi da lei. 

Anna sorrise in modo appena percettibile, ma non gli rispose. 

- Non è vero, Karl Fëdorovic, che sono come delle forbicine? - disse rivolta all'amministratore. 

- Oh, ja - rispose il tedesco. - Es ist ein ganz einfaches Ding - e cominciò a spiegare il congegno della macchina. 

- Peccato che non leghi. Ne ho vista una all'esposizione di Vienna che lega col filo di ferro - disse Svijazskij: - quelle sarebbero più convenienti. 

- Es kommt drauf an\ldots{} Der Preis vom Draht muss ausgerechnet werden. - E il tedesco, tratto fuori dal suo silenzio, si rivolse a Vronskij. - Dass lasst sich ausrechnen, Erlaucht. - Il tedesco stava già per metter la mano in tasca, dove aveva una matita infilata in un libretto in cui calcolava tutto, ma, ricordatosi ch'era a tavola, e notato lo sguardo freddo di Vronskij, si trattenne: - Zu compliziert, macht zu viel Klopot - concluse. 

- Wünscht man Dochots, so hat man auch Klopots - disse Vasen'ka Veslovskij, prendendo in giro il tedesco. - J'adore l'allemand - disse rivolto ad Anna con lo stesso sorriso. 

- Cessez - ella disse con scherzosa severità. 

- E noi credevamo di trovarvi nei campi, Vasilij Semënyc - disse rivolta al dottore, persona alquanto malaticcia - eravate là? 

- Ero là, ma mi sono volatilizzato - rispose, cupamente scherzoso, il dottore. 

- Perciò avete fatto un bel po' di moto. 

- Magnifico! 

- Su, e come va la salute della vecchietta? Spero che non sia tifo. 

- Per essere tifo non è tifo. Ma con questo non si trova in vantaggio. 

- Peccato! - disse Anna e, dato così un tributo di cortesia alle persone di casa, si rivolse ai suoi. 

- E tuttavia costruire una macchina secondo il vostro racconto sarebbe difficile, Anna Arkad'evna - disse scherzando Svijazskij. 

- No, e perché? - disse Anna con un sorriso che diceva com'ella sapesse che nella sua spiegazione del congegno della macchina c'era qualcosa di grazioso, notato anche da Svijazskij. Questa nuova civetteria giovanile colpì spiacevolmente Dolly. 

- Ma in compenso, in architettura, le conoscenze di Anna sono sorprendenti - disse Tuškevic. 

- E come, ieri ho sentito che Anna Arkad'evna diceva: nella corrente e i plinti - disse Veslovskij. - Ho detto giusto? 

- Non c'è nulla di sorprendente, quando si vedono e si sentono tante cose - disse Anna. - E voi, probabilmente, non sapete neppure con che cosa si fanno le case. 

Dar'ja Aleksandrovna vedeva che Anna era scontenta di quel tono di scherzosità che c'era fra lei e Veslovskij, ma lei stessa vi cadeva involontariamente. 

Vronskij, in questo caso, agiva in modo del tutto opposto a Levin. Evidentemente, egli non attribuiva alcuna importanza al chiacchierio di Veslovskij e, al contrario, incoraggiava questi scherzi. 

- Sì, dite un po', Veslovskij, con che cosa si uniscono le pietre? 

- S'intende, col cemento. 

- Bravo! E che cos'è il cemento? 

- Ma così, una specie di polenta d'orzo\ldots{} no, del mastice - disse Veslovskij, suscitando il riso generale. 

La conversazione fra quelli che pranzavano, a eccezione del dottore, l'architetto e l'amministratore immersi in un cupo silenzio, non languiva mai, ora scivolando, ora impigliandosi, ora toccando nel vivo qualcuno. Una volta Dar'ja Aleksandrovna fu toccata nel vivo e si accalorò tanto da arrossire perfino, e, dopo, cercò di ricordarsi se non avesse fatto qualcosa di superfluo o di spiacevole. Svijazskij si era messo a parlare di Levin, raccontando i suoi strani giudizi sulle macchine che, secondo lui, erano dannose in un'azienda russa. 

- Io non ho il piacere di conoscere codesto signor Levin - disse, sorridendo, Vronskij - ma probabilmente egli non ha mai visto quelle macchine che condanna. E se ne avrà vista e sperimentata qualcuna, sarà stata una qualunque, e non straniera ma una qualche macchina russa. E che idee ci possono mai essere qui? 

- In genere idee turche - disse Veslovskij con un sorriso, rivolto ad Anna. 

- Io non posso difendere le sue idee - disse, accalorandosi, Dar'ja Aleksandrovna - ma posso dire che è un uomo colto e che se fosse qui, saprebbe cosa rispondervi; io invece non so. 

- Io gli voglio molto bene, e io e lui siamo grandi amici - disse Svijazskij, sorridendo cordialmente. - Mais, pardon, il est un petit toqué; ecco ad esempio, egli sostiene che il consiglio distrettuale e i giudici di pace, tutto questo, insomma, non sia necessario, e non vuol far parte di nulla. 

- È la nostra apatia russa - disse Vronskij, versando l'acqua da una caraffa ghiacciata in un bicchiere sottile a calice - non sentire gli obblighi che i nostri diritti ci impongono, e perciò negare questi obblighi. 

- Io non conosco una persona più austera nel compimento dei propri doveri - disse Dar'ja Aleksandrovna, irritata da questo tono di superiorità di Vronskij. 

- Io, al contrario - continuò Vronskij evidentemente, chissà perché, toccato nel vivo da questa conversazione - io al contrario, come vedete, sono molto riconoscente dell'onore che mi hanno fatto, ecco, grazie a Nikolaj Ivanyic - ed egli indicò Svijazskij - eleggendomi giudice di pace onorario. Io considero che per me l'obbligo di andare a un congresso, di giudicare la causa di un contadino per un cavallo sia egualmente importante come tutto quello che riesco a fare. E lo riterrò un onore se mi eleggeranno consigliere distrettuale. Soltanto con questo posso ripagare tutti quei vantaggi di cui usufruisco come proprietario di terre. Per nostra disgrazia, i grossi proprietari di terre non capiscono l'importanza che devono avere nello stato. 

Per Dar'ja Aleksandrovna era strano ascoltare come egli fosse sicuro d'aver ragione alla propria tavola. Ricordò come Levin, che pensava il contrario, fosse altrettanto risoluto nei suoi giudizi a casa sua, a tavola. Ma ella voleva bene a Levin e perciò era dalla parte sua. 

- Allora possiamo contare su di voi, conte, per la prossima assemblea? - disse Svijazskij. - Ma occorre andare prima, per essere là il giorno 8. Mi farete l'onore di venire da me? 

- E io sono un po' d'accordo col tuo beau-frère - disse Anna. - Soltanto non così come lui - ella aggiunse con un sorriso. - Io temo che, nell'ultimo tempo, noi abbiamo troppi di questi obblighi pubblici. Prima c'erano tanti funzionari e per ogni affare era necessario un funzionario; adesso non ci sono che uomini pubblici. Aleksej è qui da sei mesi, ed è già membro di cinque o sei, mi pare, istituzioni diverse: è curatore, giudice, consigliere distrettuale, giurato, qualcosa di ippico. Du train que cela va, tutto il tempo andrà via per questo. E io temo che, con una tal massa d'affari, tutto si riduca a formalità. Voi di quanti uffici siete membro, Nikolaj Ivanyc? - si rivolse a Svijazskij - sembra, più di venti. 

Anna parlava scherzosamente, ma nel suo tono si sentiva l'irritazione. Dar'ja Aleksandrovna, che osservava attenta Anna e Vronskij, notò questo immediatamente. Notò anche come il viso di Vronskij, a questo discorso, prendesse un'espressione dura e ostinata. Avendo notato ciò e il fatto che la principessa Varvara, per cambiar discorso, s'era affrettata a parlare di conoscenti di Pietroburgo, Dolly ricordò quello che Vronskij in giardino aveva detto inconsideratamente sulla propria attività, e capì che alla questione della pubblica attività era collegata una certa questione intima fra Anna e Vronskij. 

Il pranzo, i vini, il servizio, tutto era stato molto bello, ma era tutto come Dar'ja Aleksandrovna aveva veduto in pranzi e balli di gala, che ormai non frequentava più e che avevano quello stesso carattere impersonale e poco disteso; e perciò, in un giorno qualunque e in una piccola compagnia, tutto questo le fece un'impressione spiacevole. 

Dopo pranzo rimasero un po' a sedere sulla terrazza. Poi, si misero a giocare al lawn-tennis. I giocatori, dopo essersi divisi in due campi, si disposero su un croquet-ground accuratamente spianato e battuto, dalle due parti di una rete tesa con colonnine dorate. Dar'ja Aleksandrovna tentò, ma a lungo non le riuscì di afferrare il giuoco, e quando lo capì era così stanca che sedette accanto alla principessa Varvara e rimase solo a guardare i giocatori. Il suo compagno, Tuškevic, s'era ritirato anch'esso, ma gli altri continuarono a lungo il giuoco. Svijazskij e Vronskij giocavano tutti e due molto bene e con impegno. Sorvegliavano con occhio vigile la palla lanciata loro, senza affrettarsi e senza indugiare si avvicinavano ad essa, ne aspettavano il rimbalzo e, colpendola di sotto in modo preciso e giusto con la racchetta, la rilanciavano al di là della rete. Veslovskij giocava peggio degli altri. Si accalorava troppo, ma in compenso con la sua allegria animava i giocatori. Il suo riso e le sue grida non cessavano. S'era tolto come gli altri, col permesso delle signore, la giacca, e la sua robusta e bella figura in maniche di camicia bianche, il viso sudato e rosso, e i suoi movimenti a scatti s'imprimevano proprio nella memoria. 

Quando Dar'ja Aleksandrovna andò a dormire, quella notte, appena chiudeva gli occhi vedeva Vasen'ka Veslovskij che si agitava su per il croquet-ground. 

Durante il giuoco Dar'ja Aleksandrovna non si divertì. Non le piacevano i rapporti che intercorrevano fra Vasen'ka Veslovskij e Anna e quella generale innaturalezza dei grandi quando, da soli, senza bambini, fanno un giuoco infantile. Ma per non turbare gli altri e per passare in qualche modo il tempo, dopo essersi riposata, si unì di nuovo al giuoco e finse di divertirsi. Tutto quel giorno le sembrò sempre di recitare con attori migliori di lei e di sciupare tutto con la sua cattiva recitazione. 

Era arrivata con l'intenzione di restare due giorni, se ci si fosse trovata bene. Ma la sera, durante il giuoco, decise di partire l'indomani. Le tormentose preoccupazioni materne che durante il viaggio le erano apparse così detestabili, ora, dopo un giorno passato senza di esse, le si presentavano già in altra luce, e l'attiravano di nuovo. 

Quando, dopo il tè della sera e una passeggiata notturna in barca, Dar'ja Aleksandrovna entrò sola nella sua stanza da letto e, toltosi il vestito, sedette ad accomodare per la notte i suoi capelli radi, sentì un gran sollievo. 

Le spiaceva quasi pensare che Anna sarebbe venuta subito da lei. Voleva restare un po' sola con i suoi pensieri. 

\capitolo{XXIII}Dolly si disponeva già a coricarsi, quando Anna, in veste da notte, entrò da lei. \\
Durante il giorno, diverse volte, Anna aveva avviato il discorso su cose intime, e ogni volta, dette alcune parole, si era fermata. ``Dopo, a quattr'occhi diremo tutto. Ho tante cose da dirti'' aveva detto. 

Ora erano sole, e Anna non sapeva di che cosa parlare. Stava seduta accanto alla finestra, guardava Dolly e, passando in rassegna nella mente tutte quelle riserve che le erano parse inesauribili di discorsi intimi, non trovava nulla. Le pareva in quel momento che tutto fosse stato detto. 

- Be', come sta Kitty? - ella disse dopo aver sospirato profondamente e guardato Dolly con aria colpevole. - Dimmi la verità, Dolly, non è arrabbiata con me? 

- Arrabbiata? No - disse, sorridendo, Dar'ja Aleksandrovna. 

- Ma mi odia, mi disprezza? 

- Oh no! ma tu lo sai, questo non si perdona. 

- Sì, sì - disse Anna, voltandosi a guardare dalla finestra aperta. - Ma io non ne ho colpa. E di chi è la colpa? E quale colpa? Poteva forse essere diversamente? Dimmi, come credi tu? Era possibile che tu non fossi la moglie di Stiva? 

- Davvero non so. Ma, ecco, cosa mi devi dire\ldots{} 

- Sì, sì, ma non abbiamo finito di Kitty. È felice? Egli è un'ottima persona, dicono. 

- È poco dire ottima. Non conosco uomo migliore. 

- Ah, come sono contenta! Sono molto contenta! È poco dire ottima - ella ripeté. 

Dolly sorrise. 

- Ma tu dimmi di te. Io con te devo fare un lungo discorso. Abbiamo parlato con\ldots{} - Dolly non sapeva come chiamarlo. Era imbarazzata a chiamarlo conte e a chiamarlo Aleksej Kirillovic. 

- Con Aleksej - disse Anna - lo so che avete parlato. Ma io volevo chiederti apertamente, che cosa pensi di me, della mia vita. 

- Come dire così, a un tratto? Io davvero non so. 

- Ma dimmi comunque\ldots{} Tu vedi la mia vita. Ma non dimenticare che ci vedi ora d'estate, ora che sei arrivata tu e non siamo soli\ldots{} Ma siamo arrivati al principio della primavera, abbiamo vissuto completamente soli e vivremo soli, e io non desidero nulla di meglio. Ma figurati quando io dovrò viver sola, senza di lui, sola, e ciò avverrà\ldots{} Io, da tutto, vedo che questo si ripeterà spesso, che la metà del tempo egli sarà fuori casa - ella disse, alzandosi e avvicinandosi a Dolly. - S'intende - ella riprese, interrompendo Dolly che voleva obiettare - s'intende, con la forza non lo tratterrò. Non ci riuscirei nemmeno. Oggi sono le corse, i suoi cavalli corrono, egli va via. Sono molto contenta. Ma tu pensa a me, figurati la mia situazione\ldots{} Ma perché parlare di questo? - Sorrise. - Allora di che cosa ha parlato con te? 

- Ha parlato di quello di cui io stessa voglio parlare e mi è facile essere il suo avvocato: se non ci sia la possibilità e non si possa\ldots{} - Dar'ja Aleksandrovna esitò - accomodare, migliorare la tua situazione\ldots{} Tu sai come io consideri\ldots{} Tuttavia, se è possibile, occorre sposarsi\ldots{} 

- Cioè, il divorzio? - disse Anna. - Tu sai, l'unica donna che sia venuta da me a Pietroburgo è stata Betsy Tverskaja. La conosci, è vero? Au fond c'est la femme la plus dépravée qui existe. Aveva una relazione con Tuškevic, ingannando il marito nel modo più disgustoso. Eppure mi disse che non voleva saperne di me, finché la mia posizione non fosse regolare. Non pensare che io faccia confronti\ldots{} Ti conosco, anima mia. Ma io mi son ricordata senza volere\ldots{} Allora, cosa mai ti ha detto? - ripeté. 

- Ha detto che soffre per te e per sé. Forse tu dirai che è egoismo, ma è un egoismo così legittimo e nobile! Egli vuole, in primo luogo, legittimare sua figlia ed essere tuo marito, aver diritto su di te. 

- Quale moglie, quale schiava, può essere a tal punto schiava come me, nella mia situazione? - ella interruppe torva. 

- La cosa principale poi, che egli vuole, è che tu non soffra. 

- Questo è impossibile! E poi? 

- E poi, la cosa più legittima: vuole che i vostri figli abbiano un nome. 

- E quali figli? - disse Anna, senza guardare Dolly e socchiudendo gli occhi. 

- Annie e quelli che potranno venire. 

- Per questo può star tranquillo: io non avrò più figli. 

- E come puoi dire che non ne avrai? 

- Non ne avrò, perché non ne voglio. 

E, malgrado tutta la sua agitazione, Anna sorrise avendo notato un'espressione ingenua di curiosità, di stupore e di orrore sul viso di Dolly. 

- Il dottore m'ha detto, dopo la mia malattia\ldots{} 

- Non può essere! - disse Dolly, spalancando gli occhi. Per lei questa era una di quelle rivelazioni, le cui conseguenze e deduzioni sono così enormi che, nel primo momento, si sente soltanto che considerar tutto non si può, che bisogna pensarci molto e poi molto. 

Questa rivelazione, che le spiegava a un tratto tutte quelle famiglie per lei prima incomprensibili in cui vi erano soltanto uno o due bambini, suscitò in lei tanti pensieri, considerazioni e sentimenti contraddittori, ch'ella non aveva più nulla da dire e guardava soltanto Anna, stupita, con gli occhi spalancati. Era quella stessa cosa ch'ella aveva sognato durante il viaggio, ma ora, venuta a sapere che era possibile, inorridiva. Sentiva che era la soluzione troppo semplicistica di una troppo complessa questione. 

- N'est-ce-pas immoral? - ella disse soltanto dopo un po' di silenzio. 

- Perché? Pensa, io ho la scelta fra le due: o essere incinta, cioè malata, o essere l'amica, la compagna di mio marito - disse Anna con un tono cosciente di superficialità e leggerezza. 

- Eh già, eh già - diceva Dar'ja Aleksandrovna, ascoltando quegli identici argomenti ch'ella pure portava a se stessa, e senza più trovarvi la forza di persuasione di una volta. 

- Per te, per gli altri - diceva Anna, quasi indovinando i suoi pensieri - ci può ancora essere un dubbio, ma per me\ldots{} Devi capire, io non sono sua moglie; egli mi ama finché mi ama. E con che posso trattenere il suo amore? Con questo? 

E allungò le braccia bianche dinanzi al ventre. 

Con una rapidità straordinaria, come accade in un momento di agitazione, i pensieri e i ricordi si affollarono nella mente di Dar'ja Aleksandrovna. ``Io - pensava - non ho attirato a me Stiva; s'è allontanato da me verso altre, e quella prima per la quale mi ha tradito, non l'ha trattenuto con l'esser sempre bella e allegra. Egli l'ha buttata e ne ha presa un'altra. È possibile che Anna con questo attiri e trattenga il conte Vronskij? S'egli cercherà questo, troverà toilettes e maniere ancora più attraenti di queste. E per quanto siano bianche e per quanto siano splendide le sue braccia nude, per quanto sia bella la sua figura morbida, il suo viso acceso, e questi suoi capelli neri, egli troverà ancora di meglio, come cerca e trova il mio disgustoso, compassionevole e caro marito''. 

Dolly non rispose nulla e sospirò soltanto. Anna notò questo sospiro che esprimeva dissenso, e continuò. In riserva aveva argomenti ancora così forti, ai quali non si poteva rispondere nulla. 

- Tu dici che questo non è bene? Ma bisogna ragionare - ella proseguì. - Tu dimentichi la mia situazione. Come posso desiderare dei figli? Io non parlo delle sofferenze, non le temo. Pensa, cosa saranno i miei figli? disgraziati che dovranno portare un cognome estraneo. Per la stessa nascita dovranno vergognarsi della madre, del padre, della propria venuta al mondo. 

- Ma dunque, proprio per questo è indispensabile il divorzio 

Ma Anna non l'ascoltava. Voleva esporre fino in fondo quegli argomenti con cui si era persuasa tante volte. 

- E perché mi è data la ragione, se non l'adopero per non mettere al mondo dei disgraziati? 

Guardò per un attimo Dolly, ma senza aspettare risposta, continuò. 

- Io mi sentirei sempre colpevole dinanzi a questi disgraziati figli - disse. - Se non ci sono, almeno non sono disgraziati; invece se sono disgraziati, la colpa è mia soltanto. 

Erano quegli identici argomenti che Dar'ja Aleksandrovna aveva portato a se stessa; ma adesso li ascoltava e non li intendeva: ``Come essere colpevole dinanzi a esseri non esistenti?'' pensava. E a un tratto le venne un pensiero, se avrebbe potuto, in un qualche caso, esser meglio per il suo Griša, il beniamino, di non esistere. E questo le parve così strano, così bizzarro, che scosse un po' la testa, quasi a cacciarne via quella confusione di pensieri che vi turbinava. 

- No, non so, non è bene - ella disse soltanto, con una espressione di ripugnanza in viso. 

- Sì, ma tu non devi dimenticare cosa sei tu e cosa sono io\ldots{} E poi - soggiunse Anna che, malgrado la ricchezza dei propri argomenti dinanzi alla povertà di quelli di Dolly, aveva tuttavia l'aria di confessare che ciò non era bene - tu non dimenticare la cosa principale, che adesso io mi trovo in una posizione non simile alla tua. Per te la questione è se devi o no desiderare di non avere più figli, e per me se devo o no desiderare di averne. C'è una grande differenza. Capirai come io non possa desiderarne nella mia posizione. 

Dar'ja Aleksandrovna non obiettava nulla. Aveva sentito a un tratto di essere già così lontana da Anna, che fra di loro esistevano questioni sulle quali non si sarebbero mai intese e delle quali era meglio non parlare. 

\capitolo{XXIV}-Allora tanto più devi accomodare la tua situazione, se è possibile - disse Dolly. \\
- Sì, se è possibile - disse Anna, a un tratto, con voce completamente diversa, piana e triste. 

- Non è forse possibile il divorzio? M'hanno detto che tuo marito acconsente. 

- Dolly! Non voglio parlare di questo. 

- Via, non ne parleremo - si affrettò a dire Dar'ja Aleksandrovna dopo aver notato l'espressione di sofferenza nel viso di Anna. - Io vedo soltanto che tu guardi la cosa troppo cupamente. 

- Io? per nulla. Io sono allegra e soddisfatta. Hai visto. Je fais des passions. Veslovskij\ldots{} 

- Sì, a dire la verità, non mi è piaciuto il tono di Veslovskij - disse Dar'ja Aleksandrovna, desiderando cambiare discorso. 

- Ah! per nulla! Ciò eccita Aleksej e niente altro; ma lui è un ragazzo, ed è tutto nelle mie mani; capisci, lo dirigo come voglio. È proprio come il tuo Griša\ldots{} Dolly! - ella cambiò discorso a un tratto - tu dici che io guardo cupamente. Tu non puoi capire. È troppo orribile. Io cerco di non guardare proprio. 

- Ma bisogna, mi pare. Bisogna fare quello che è possibile. 

- Ma cosa mai è possibile? Nulla. Tu dici che devo sposare Aleksej e che non ci penso. Io non ci penso! - ripeté e il rossore le apparve sul viso. Si alzò, raddrizzò il busto, sospirò penosamente, e si mise a camminare col suo passo leggero avanti e indietro per la stanza, fermandosi di tanto in tanto. - Non ci penso? Non c'è giorno, non c'è ora in cui non pensi e non mi rimproveri perché penso\ldots{} perché questi pensieri possono fare impazzire. Fare impazzire - ella ripeté. - Quando ci penso, non mi addormento mai senza morfina. Ma via. Parliamo con calma. Mi dicono: il divorzio. In primo luogo, lui non me lo darà. Lui adesso è sotto l'influenza della contessa Lidija Ivanovna. 

Dar'ja Aleksandrovna, allungata sulla sedia, volgendo il capo, seguiva, con viso di sofferente compassione, Anna che camminava. 

- Bisogna tentare - ella disse piano. 

- Ammettiamolo, tentare. Che cosa significa? - disse, esprimendo evidentemente un pensiero mille volte ripensato e imparato a memoria. - Significa che io, che lo detesto, eppure mi riconosco colpevole dinanzi a lui, e lo stimo magnanimo, io mi umilio a scrivergli\ldots{} Ma ammettiamo pure; farò uno sforzo, lo farò. O riceverò una risposta offensiva o il consenso. Va bene, ricevo il consenso\ldots{} - Anna in quel momento era in un angolo lontano della stanza e si fermò là, facendo qualcosa con la tenda d'una finestra. - Riceverò il consenso, e mio fi\ldots{} figlio? Perché non me lo renderanno. Perché lui cresce disprezzandomi, accanto a un padre che io ho abbandonato. Tu devi capire che io amo, mi pare, allo stesso modo, ma tutti e due più di me, due esseri: Serëza e Aleksej. 

Ella venne in mezzo alla stanza e si fermò dinanzi a Dolly, premendosi il petto con le mani. Nell'accappatoio bianco la sua figura sembrava particolarmente grande e larga. Aveva chinato il capo e guardava di sotto in su con gli occhi scintillanti e umidi la figura magra, piccola e meschina di Dolly, con la sua camicetta rammendata e la cuffia da notte, tutta tremante per l'agitazione. 

- Soltanto questi due esseri io amo, e l'uno esclude l'altro. Io non posso unirli, eppure questa sola cosa mi è necessaria. E se questo non è, allora è tutto lo stesso. Tutto, tutto è lo stesso. E in qualche modo finirà, e perciò non posso, non amo parlare di questo. Così tu non giudicarmi, non rimproverarmi nulla. Tu, con la tua purezza, non puoi capire tutto quello che io soffro. 

Si avvicinò, sedette accanto a Dolly e, esaminando il suo viso con un'espressione di colpa, la prese per mano. 

- Cosa pensi? cosa pensi di me? Non mi disprezzare. Io non merito il disprezzo. Sono proprio disgraziata. Se c'è un essere disgraziato, questo sono io - disse, e, voltando le spalle, si mise a piangere. 

Rimasta sola, Dolly pregò Iddio e si coricò. Aveva provato pena di Anna, mentre aveva parlato con lei; ma ora non poteva costringersi a pensare a lei. I ricordi della casa e dei bambini sorgevano nella sua immaginazione con un fascino particolare, nuovo per lei, in una luce nuova. Questo suo mondo adesso le era apparso così caro e gentile, che non voleva per nessuna ragione trascorrere una giornata di più fuori di esso e decise di partire assolutamente l'indomani. 

Anna, intanto, tornata nel suo studio, prese un bicchierino e vi versò alcune gocce di un medicinale, in cui predominava la morfina, e, dopo aver bevuto ed essere rimasta a sedere immobile qualche tempo, andò in camera sua con lo spirito rasserenato e gaio. 

Quando entrò in camera Vronskij la guardò con attenzione. Cercava i segni di quel colloquio che, per essere rimasta così a lungo nella stanza di Dolly, egli sapeva doveva aver avuto con lei. Ma nella sua espressione eccitata e contenta, ma che nascondeva qualcosa, egli non trovò nulla, oltre la bellezza, per lui abituale, e che sempre ancora lo seduceva, la consapevolezza di essa e il desiderio che agisse su di lui. Egli non voleva chiederle cosa avessero detto, ma sperava ch'ella stessa ne riferisse qualcosa. Ma ella disse solo: 

- Sono contenta che ti sia piaciuta Dolly. Non è vero? 

- Ma io la conosco da tempo. È molto buona, mi pare, mais excessivement terre-à-terre. Ma sono stato molto contento di vederla. 

Egli prese le mani di Anna e la guardò negli occhi interrogativamente. 

Ella, capito in altro modo quello sguardo, gli sorrise. 

La mattina dopo, malgrado le insistenze dei padroni di casa, Dar'ja Aleksandrovna si preparò a partire. Il cocchiere di Levin, col gabbano non più nuovo e il cappello quasi a foggia di postiglione, coi cavalli di colore diverso, la vettura dai parafanghi rappezzati, entrò torvo e risoluto nell'androne coperto, cosparso di sabbia. 

Il congedo dalla principessa Varvara e dagli uomini era poco piacevole per Dar'ja Aleksandrovna. Dopo essere rimasta un giorno, sia lei che i padroni avevano sentito che non si confacevano reciprocamente, e che era meglio non trovarsi insieme. Solo Anna era triste. Sentiva che con la partenza di Dolly nessuno più avrebbe agitato nell'animo suo quei sentimenti che s'erano sollevati in lei in quell'incontro. Agitare questi sentimenti era per lei doloroso; tuttavia sapeva che era la parte migliore dell'animo suo e che questa parte dell'animo suo si sarebbe cicatrizzata rapidamente nella vita che conduceva. 

Quando furono usciti nei campi, Dar'ja Aleksandrovna provò un piacevole senso di sollievo e voleva domandare agli uomini come s'erano trovati da Vronskij quando subito il cocchiere Filipp prese a parlare: 

- Ricchi, ricconi, ma d'avena ne hanno date tre misure in tutto. Prima che cantassero i galli, gli animali avevano già tutto ripulito. Che cosa sono tre misure? solo per l'antipasto. Al giorno d'oggi l'avena, dai portieri, sta a quarantacinque copeche. Da noi, grazie al cielo, a quelli che arrivano, quanta ne mangiano tanta ne danno. 

- Un signore avaro - confermò lo scrivano. 

- Be', e i cavalli ti son piaciuti? - domandò Dolly. 

- Son cavalli, in una parola. E il mangiare è buono. Ma a me è parso un po' d'annoiarmi, Dar'ja Aleksandrovna, non so, se a voi - disse, volgendo verso di lei il suo viso bello e buono. 

- Ma anche a me. Che dici, arriveremo verso sera? 

- Bisogna arrivare. 

Tornata a casa e trovati tutti completamente felici e particolarmente simpatici, Dar'ja Aleksandrovna raccontò con grande animazione del suo viaggio, di come l'avevano accolta bene, dello sfarzo e del buon gusto della vita dei Vronskij, degli svaghi, e non permise a nessuno di dire una parola contro di loro. 

- Bisogna conoscere Anna e Vronskij; io adesso li ho conosciuti meglio, per capire come siano simpatici e commoventi - ella diceva, con piena sincerità, dimenticando quel vago senso di scontento e di disagio che aveva provato in casa loro. 

\capitolo{XXV}Vronskij e Anna, sempre nelle stesse condizioni, sempre senza prendere alcun provvedimento per il divorzio, passarono tutta l'estate e parte dell'autunno in campagna. Era stabilito tra di loro che non sarebbero andati in nessun posto; ma tutti e due sentivano che, quanto più avrebbero vissuto soli, specie di autunno e senza ospiti, tanto più non avrebbero sopportato quella vita e sarebbe stato necessario cambiarla. 

La vita pareva tale, quale migliore non si poteva desiderare: c'era ogni agio, c'era la salute, c'era la bambina, e tutti e due avevano delle occupazioni. Anna, senza ospiti, si occupava di sé sempre allo stesso modo e leggeva moltissimo: romanzi e libri seri, alla moda. Ordinava tutti i libri che erano recensiti favorevolmente nei giornali e nelle riviste che riceveva dall'estero, e li leggeva con quell'attenzione che si pone nella lettura soltanto nella solitudine. Inoltre, tutte le materie di cui si occupava Vronskij, lei le studiava sui libri e sulle riviste specializzate, così che spesso egli si rivolgeva direttamente a lei con domande di agricoltura, di architettura, perfino di allevamento equino e di sport. Egli si sorprendeva delle conoscenza, della memoria di lei, e in principio dubbioso, desiderava una conferma; e lei trovava nei libri quello di cui egli l'aveva richiesta e glielo mostrava. 

Anche la costruzione dell'ospedale la occupava. Non soltanto aiutava, ma molte cose le organizzava e le pensava lei. Ma la sua preoccupazione maggiore era tuttavia la propria persona, lei stessa, in quanto era cara a Vronskij, in quanto poteva sostituire per lui tutto quello ch'egli aveva abbandonato. Vronskij apprezzava questo desiderio, che era divenuto l'unico scopo della vita di lei, non solo di piacergli, ma di essergli utile; tuttavia nello stesso tempo, sentiva anche il peso di quelle reti amorose in cui ella cercava di avvolgerlo. Quanto più il tempo passava, quanto più spesso egli si vedeva avvolto in queste reti, tanto più gli veniva il desiderio non di uscirne, ma di provare se queste non intralciassero la sua libertà. Se non fosse stato questo desiderio sempre più forte di sentirsi libero, di non avere scenate ogni volta che doveva andare in città per un'assemblea, alle corse, Vronskij sarebbe stato del tutto soddisfatto della propria vita. La parte ch'egli aveva scelto, la parte di uno di quei ricchi proprietari di terre, che dovevano costituire il nerbo dell'aristocrazia russa, non solo lo soddisfaceva in pieno, ma adesso, dopo sei mesi di questa vita, gli procurava una soddisfazione sempre maggiore. E il suo lavoro, occupandolo e assorbendolo sempre di più procedeva benissimo. Malgrado le enormi spese sostenute per l'ospedale, le macchine, le mucche fatte venire dalla Svizzera e molte altre cose, era sicuro di non dissestare, ma di valorizzare la propria sostanza. Là dove si trattava di entrate, di vendita di boschi, di grano, di lana, del fitto delle terre, Vronskij era duro come la pietra, e sapeva mantenere il prezzo. Negli affari economici in grande, in questo e negli altri possedimenti, si atteneva ai sistemi più semplici e meno arrischiati ed era estremamente economo e calcolatore nelle piccole cose dell'azienda. Malgrado tutta la furberia e l'abilità del tedesco, che lo spingeva agli acquisti ed esponeva ogni calcolo in modo che in principio ci voleva molto di più, ma dopo aver riflettuto, si poteva far la stessa cosa più a buon mercato e ricavarne subito un utile, Vronskij non gli si sottometteva mai. Ascoltava l'amministratore, lo interrogava e consentiva con lui soltanto quando quello che si ordinava o si organizzava era la cosa più nuova, ancora sconosciuta in Russia, che poteva suscitare la meraviglia. Inoltre si decideva a una grande spesa solo quando c'era denaro superfluo e, facendo tale spesa, entrava in tutti i dettagli e insisteva per avere la cosa migliore in corrispettivo dei suoi denari. Così che, dal modo come egli conduceva i propri affari era chiaro che non dissestava, ma aumentava il patrimonio. 

Nel mese di ottobre c'erano le elezioni nobiliari nel governatorato di Kašin, dove erano i possedimenti di Vronskij, di Svijazskij, di Koznyšev, di Oblonskij e una piccola parte di quelli di Levin. 

Queste elezioni, per molte circostanze e per le persone che vi partecipavano, attiravano l'attenzione generale. Se ne parlava molto e ci si preparava. Da Mosca, da Pietroburgo e dall'estero, persone che non andavano mai alle elezioni, convenivano per quelle. 

Vronskij aveva promesso a Svijazskij di andarci, già da tempo. 

Prima delle elezioni, Svijazskij, che visitava spesso Vozdvizenskoe, venne a prendere Vronskij. 

Fin dalla vigilia di questo giorno, fra Vronskij e Anna c'era stata una discussione per il viaggio progettato. Era il periodo autunnale più noioso, più penoso in campagna, e perciò Vronskij, preparandosi alla lotta, con un'espressione severa e fredda, come non aveva mai parlato ad Anna, le annunciò la sua partenza. Ma, con suo stupore, Anna accolse questa notizia con molta calma e domandò solo quando sarebbe tornato. Egli la guardò attento, senza spiegarsi questa calma. Lei sorrise al suo sguardo. Egli conosceva in lei la facoltà di ritrarsi in se stessa, e sapeva che ciò accadeva solo quando aveva deciso qualcosa fra di sé, senza comunicargli i suoi piani. Egli ne aveva paura; ma aveva un così grande desiderio di evitare una scenata, che finse di credere, e in parte sinceramente credette, a quello che voleva credere, alla ragionevolezza di lei. 

- Spero che non ti annoierai. 

- Spero - disse Anna. - Ieri ho ricevuto una cassa di libri da Gauthier. No, non mi annoierò. 

``Lei vuol prender questo tono, e tanto meglio - egli pensò - e se no, fa lo stesso''. 

E così, senza averla spinta a una spiegazione sincera, partì per le elezioni. Era la prima volta dal principio della loro relazione, ch'egli si separava da lei senza che si fossero spiegati fino in fondo. Da una parte ciò lo agitava, dall'altra giudicava che fosse meglio. ``Da principio ci sarà, come ora, qualcosa di poco chiaro, di nascosto, ma dopo ci si abituerà. In ogni caso io posso darle tutto, ma non la mia indipendenza di uomo'' pensò. 

\capitolo{XXVI}A settembre Levin si trasferì a Mosca per il parto di Kitty. Viveva già da un mese a Mosca senza far nulla, quando Sergej Ivanovic, che aveva un podere nel governatorato di Kašin e prendeva parte attiva alle imminenti elezioni, si preparò a recarvisi. Egli invitava ad andare con lui anche il fratello, che aveva un voto per il distretto di Seleznevsk. Oltre a ciò, Levin aveva a Kašin un affare di estrema urgenza, per sua sorella che viveva all'estero, circa una tutela e l'esazione del denaro di un riscatto. 

Levin era ancora incerto, ma Kitty, vedendo ch'egli si annoiava a Mosca, gli consigliò di andare, e gli ordinò, a sua insaputa, una divisa nobiliare che costava ottanta rubli. E questi ottanta rubli, pagati per la divisa, furono la ragione principale che spinse Levin ad andare. Andò a Kašin. 

Levin era a Kašin già da sei giorni, frequentando tutti i giorni l'assemblea e affannandosi per l'affare della sorella che non riusciva a sistemare. I marescialli della nobiltà erano tutti presi dalle elezioni, e non si riusciva a portare a termine quel semplicissimo affare che dipendeva dalla tutela. L'altro affare, poi, la riscossione del denaro, incontrava gli stessi ostacoli. Dopo lunghe sollecitazioni perché fosse tolto il divieto, il denaro era pronto per il pagamento; ma il notaio, uomo servizievolissimo, non poteva consegnare il mandato, perché era necessaria la firma del presidente, e il presidente, senza aver fatto la consegna dell'ufficio, era alla sessione. Tutte queste preoccupazioni, l'andare da una parte all'altra, le conversazioni con quelle buone, brave persone, che si rendevano pienamente conto della spiacevolezza della situazione del sollecitatore, ma non potevano aiutarlo, tutta quella tensione, che non portava a nulla, producevano in Levin un senso di molestia, simile a quella stizzosa impotenza che si prova in sogno quando si vuole usare della forza fisica. Egli provava spesso ciò, discorrendo con quel brav'uomo del suo procuratore. Questo procuratore sembrava far tutto il possibile e tendere tutte le sue forze intellettive per togliere Levin dalle difficoltà. ``Ecco cosa dovete provare - diceva più di una volta - andate prima là e poi là'' e il procuratore faceva tutto un piano, per superare quel fatale principio che intralciava tutto. Ma subito aggiungeva: ``Tuttavia non lo rilasceranno; provate, comunque''. E Levin provava, andava a piedi, andava in carrozza. Tutti erano buoni e cortesi, ma veniva in chiaro che ciò che si era evitato spuntava di nuovo alla fine e di nuovo intralciava lo svolgimento della pratica. Era particolarmente increscioso che Levin non potesse in nessun modo capire contro chi lottava, né chi traesse vantaggio dal fatto che il suo affare non giungesse a termine. Questo, nessuno sembrava saperlo; neppure il procuratore lo sapeva. Se Levin avesse potuto capire, così come capiva perché alla stazione non ci si può avvicinare alla biglietteria senza mettersi in fila, non avrebbe provato stizza e vergogna; ma degli ostacoli che incontrava nella sua faccenda nessuno gli poteva dire perché esistessero. 

Ma Levin era molto cambiato dal tempo del suo matrimonio; era paziente, e, se non capiva perché tutto questo fosse disposto così, si diceva che, non sapendo tutto, non poteva giudicare, che, probabilmente, doveva essere così e cercava di non indignarsene. 

Ora, assistendo alle elezioni e partecipandovi, cercava egualmente di non giudicare, di non discutere, ma per quanto era possibile cercava di capire quella faccenda di cui si occupavano con tanta serietà e passione persone oneste e brave, da lui stimate. Da quando s'era sposato, gli si erano rivelati tanti lati nuovi e profondi della vita che prima, per la leggerezza con cui li trattava, gli sembravano insignificanti; così anche nell'affare delle elezioni egli supponeva e ricercava un senso profondo. 

Sergej Ivanovic gli spiegò il senso e il significato del cambiamento di rotta che si prevedeva nelle elezioni. Il maresciallo della nobiltà del governatorato, nelle cui mani, secondo la legge, si trovavano tanti pubblici affari (le tutele, quelle stesse per cui Levin adesso soffriva, e somme enormi di denaro nobiliare, e i ginnasi, quello femminile, quello maschile e quello militare, e l'istruzione popolare, secondo il nuovo regolamento, e infine il consiglio distrettuale), il maresciallo della nobiltà del governatorato, Snetkov, era un uomo d'antico stampo nobiliare, che aveva sperperato un enorme patrimonio, un uomo buono, onesto nel suo genere, ma che non capiva affatto le esigenze dei tempi nuovi. Sosteneva sempre, in tutto, le parti della nobiltà, contrastava apertamente la diffusione dell'istruzione popolare, e dava al consiglio distrettuale, che doveva avere una così vasta importanza, un carattere di classe. Bisognava mettere al suo posto un uomo nuovo, moderno, attivo, completamente diverso, e bisognava condurre la cosa in modo da ricavare da tutti i diritti concessi alla nobiltà, non come nobiltà, ma come elemento del consiglio distrettuale, tutti quei vantaggi d'autonomia che potevano essere ricavati. Nel ricco governatorato di Kašin, che era sempre in testa, s'erano raccolte adesso tali forze che l'azione condotta là, come andava condotta, poteva servire di modello per gli altri governatorati, per tutta la Russia. E per questo tutta la faccenda aveva un grande significato. Come maresciallo al posto di Snetkov, si pensava di mettere o Svijazskij o, ancora meglio, Nevedovskij, che era docente universitario, uomo di straordinaria intelligenza e grande amico di Sergej Ivanovic. 

L'assemblea l'aprì il governatore, che fece un discorso ai nobili, perché eleggessero funzionari non secondo le simpatie personali, ma secondo i meriti e il bene della patria, e disse di sperare che la degna nobiltà di Kašin, come nelle elezioni precedenti, avrebbe compiuto santamente il proprio dovere e avrebbe giustificato l'alta fiducia del monarca. 

Finito il discorso, il governatore andò via dalla sala, e i nobili lo seguirono con rumorosa animazione, mentre egli infilava la pelliccia e discorreva cordialmente con il maresciallo del governatorato. Levin, desiderando di penetrare il senso di tutto e di non lasciarsi sfuggire nulla, stava dritto proprio lì nella folla e sentiva come il governatore diceva: ``Per favore dite a Mar'ja Ivanovna che a mia moglie spiace molto di non poter andare all'asilo''. E dopo di questo, i nobili presero allegramente le pellicce e andarono tutti alla cattedrale. 

Nella cattedrale Levin, sollevando il braccio e ripetendo le parole dell'arciprete insieme con gli altri, giurò con i giuramenti più terribili di compiere tutto quello che sperava il governatore. Il servizio divino aveva sempre un influsso su Levin, e quando egli pronunciò le parole: ``bacio la croce'' e si voltò a guardare quella folla di persone giovani e vecchie, che ripetevano la stessa cosa, si sentì commosso. 

Il secondo e il terzo giorno si discussero gli affari delle somme di denaro nobiliare e del ginnasio femminile, che non avevano, come spiegò Sergej Ivanovic, alcuna importanza, e Levin, preso dal corso dei suoi affari, non li seguì. Il quarto giorno, intorno al tavolo del governatorato, si svolse la verifica delle somme di denaro del governatorato stesso. E qui, per la prima volta, avvenne un urto fra il partito nuovo e il vecchio. La commissione, incaricata di verificare le somme, riferì che le somme erano intatte. Il maresciallo del governatorato si alzò, ringraziando la nobiltà per la fiducia, e sparse qualche lacrima. I nobili lo acclamarono a gran voce e gli strinsero la mano. Ma in quel punto, un nobile del partito di Sergej Ivanovic disse di aver sentito che la commissione non aveva verificato le somme, considerando la verifica un'offesa per il maresciallo del governatorato. Uno dei membri della commissione, incautamente, confermò la cosa. Allora un signore molto piccolo, molto giovane all'aspetto, ma velenoso, prese a dire che al maresciallo del governatorato avrebbe, probabilmente, fatto piacere dare un rendiconto delle somme, e che la eccessiva delicatezza dei membri della commissione lo privava di questa soddisfazione morale. Allora i membri della commissione rinunciarono alla propria dichiarazione, e Sergej Ivanovic cominciò a dimostrare, a fil di logica, che bisognava o riconoscere che le somme erano state da loro verificate o che non lo erano state, e svolse minuziosamente questo dilemma. A Sergej Ivanovic ribatté l'oratore del partito opposto. Poi parlo Svijazskij e di nuovo il signore velenoso. Le discussioni durarono a lungo e finirono senza concludere nulla. Levin era sorpreso che si discutesse così a lungo di questo, soprattutto perché quando aveva chiesto a Sergej Ivanovic se egli supponeva che le somme fossero state malversate, Sergej Ivanovic aveva risposto: 

- Oh, no! È una persona onesta. Ma questo sistema antiquato di amministrazione familiare, paterna, degli affari nobiliari bisogna scrollarlo. 

Il quinto giorno ci furono le elezioni dei marescialli distrettuali. Questa giornata fu abbastanza burrascosa in alcuni distretti. Nel distretto di Seleznevsk, Svijazskij fu eletto, all'unanimità senza ballottaggio, e quel giorno ci fu un pranzo per lui. 

\capitolo{XXVII}Per il sesto giorno erano fissate le elezioni del governatorato. Le sale grandi e le piccole erano piene di nobili in varie divise. Molti erano arrivati solo quel giorno. Amici che non si vedevano da tempo, chi dalla Crimea chi da Pietroburgo, chi dall'estero, s'incontravano nelle sale. Alla tavola del governatorato, sotto il ritratto dello zar, si svolgevano i dibattiti. 

I nobili, nella sala grande e nella piccola, si raggruppavano in campi opposti, e dall'ostilità e dalla diffidenza, dal discorso che taceva all'avvicinarsi di gente estranea, dal fatto che alcuni, parlando sottovoce, si spingevano perfino in un corridoio lontano, si vedeva che ciascuna parte aveva dei misteri da nascondere all'altra. Dall'aspetto esteriore i nobili si dividevano decisamente in due specie: nei vecchi e nei nuovi. I vecchi erano per la maggior parte o in vecchie divise nobiliari abbottonate, con spade e cappelli, o nelle loro particolari divise, che s'erano conquistate, di marina, di cavalleria, di fanteria. Le divise dei vecchi nobili erano cucite all'antica, con gli sbuffi sulle spalle; erano, evidentemente, piccole, corte di vita e strette come se coloro che le portavano ne fossero cresciuti fuori. I giovani erano, invece, con le divise nobiliari sbottonate, con la vita bassa e le spalle larghe, coi panciotti bianchi, o con le divise dal colletto nero e i fregi a foglie di lauro, emblema del ministero della giustizia. Pure ai giovani appartenevano alcune divise di corte, che abbellivano qua e là la massa. 

Ma la divisione in giovani e vecchi non corrispondeva alla divisione dei partiti. Alcuni fra i giovani, come aveva osservato Levin, appartenevano al partito vecchio, e, al contrario, alcuni dei nobili più vecchi parlavano sottovoce con Svijazskij, ed erano, evidentemente, caldi sostenitori del partito nuovo. 

Levin stava in piedi nella sala piccola, dove si fumava e si mangiava, accanto a un gruppo dei suoi, prestando orecchio a quello che dicevano e tendendo invano le forze del suo ingegno per capire cosa dicessero. Sergej Ivanovic era il centro intorno al quale si raggruppavano gli altri. Egli ora ascoltava Svijazskij e Chljustov, maresciallo di un altro distretto, che apparteneva al loro partito. Chljustov non consentiva ad andare da Snetkov col suo distretto e pregarlo di mettersi in ballottaggio, ma Svijazskij lo esortava a farlo e Sergej Ivanovic approvava questo piano. Levin non capiva perché il partito contrario dovesse pregare di mettere in ballottaggio quel maresciallo ch'essi non volevano eleggere. 

Stepan Arkad'ic, che proprio allora aveva finito di mangiare e di bere, asciugandosi la bocca con un fazzoletto di batista, ricamato e profumato, si accostò a loro nella sua divisa di ciambellano. 

- Occupiamo la posizione - disse, lisciandosi tutte e due le fedine - Sergej Ivanyc! 

E prestato ascolto alla conversazione, confermò l'opinione di Svijazskij. 

- Basta un distretto, e Svijazskij significa già, evidentemente, l'opposizione - disse con parole comprensibili a tutti fuor che a Levin. 

- Be', Kostja, anche tu, a quanto pare, ci hai preso gusto? - aggiunse rivolto a Levin, e lo prese sotto braccio. Levin sarebbe stato contento di prenderci gusto, ma non riusciva a capire di che cosa si trattava e, allontanandosi di alcuni passi da quelli che parlavano, espresse a Stepan Arkad'ic la propria meraviglia, per il fatto che bisognasse pregare il maresciallo del governatorato. 

- O Sancta simplicitas! - disse Stepan Arkad'ic e, in breve e con chiarezza, spiegò a Levin di che si trattava. 

Se tutti i distretti, come nelle elezioni passate, avessero pregato il maresciallo del governatorato, l'avrebbero eletto con tutte palle bianche. Questo non doveva accadere. Ora, otto distretti consentivano a pregarlo; se invece due si rifiutavano di pregarlo, Snetkov avrebbe potuto rinunciare al ballottaggio. E allora il vecchio partito poteva eleggere un altro dei suoi, poiché tutto il calcolo sarebbe andato perduto. Ma se il solo distretto di Svijazskij non lo avesse pregato, Snetkov sarebbe stato messo in ballottaggio. Lo avrebbero perfino eletto, riservando dei voti per lui, così che il partito contrario si sarebbe confuso nei calcoli e quando avessero proposto un candidato dei nostri, avrebbero portato i voti su di lui. 

Levin capì, ma non perfettamente, e voleva ancora fare delle domande, quando a un tratto tutti presero a parlare, a rumoreggiare e a muoversi verso la sala grande. 

- Che c'è? cosa? chi? - La procura? a chi? perché? - La ricusano? - Non la procura. - Non ammettono Flerov. - Ma che cosa, se è sotto giudizio? - Così non ammetteranno nessuno. Ciò è vile. - La legge! - sentiva Levin da varie parti e, insieme con gli altri che si affrettavano chi sa dove, e temevano di perdere qualcosa, si diresse nella sala grande e, stretto dai nobili, si accostò al tavolo del governatorato, presso al quale discutevano con calore il maresciallo del governatorato, Svijazskij e altri rappresentanti. 

\capitolo{XXVIII}Levin era in piedi, abbastanza lontano. Un nobile che accanto a lui respirava greve, con l'affanno, e un altro che scricchiolava con le suole doppie, gli impedivano di ascoltare chiaramente. Sentiva soltanto la voce morbida, suadente del maresciallo, poi quella stridula del nobile velenoso e poi la voce di Svijazskij. Discutevano, a quanto egli poteva capire, sul significato degli articoli di legge o sul significato delle parole: ``trovantesi sotto inchiesta''. 

La folla si divise per lasciar passare Sergej Ivanovic che si accostava al tavolo. Sergej Ivanovic, dopo aver ascoltato la fine del discorso del nobile velenoso disse che gli sembrava che la cosa più sicura da fare sarebbe stata la consultazione dell'articolo di legge, e pregò il segretario di trovare l'articolo. Nell'articolo era detto che, in caso di dissenso, bisognava passare alla votazione. 

Sergej Ivanovic lesse l'articolo e cominciò a spiegare il senso, ma a questo punto, un proprietario alto e grasso, un po' curvo, con i baffi tinti, in una divisa stretta, con un bavero che gli sosteneva il collo da dietro, lo interruppe. Si accostò al tavolo e, picchiandovi sopra con un anello, cominciò a gridare a gran voce: 

- Bisogna votare! Alle urne! Senza chiacchiere! Alle urne! 

Allora, improvvisamente, presero a levarsi parecchie voci, e il nobile alto con l'anello, irritandosi sempre più, gridava sempre più forte. Non si poteva distinguere quello che diceva. 

Diceva la stessa cosa che aveva proposto Sergej Ivanovic, ma evidentemente odiava lui e tutto il suo partito, e questo senso di odio si comunicava a tutto il partito e suscitava un'eguale animosità in risposta, anche se più corretta, dall'altra parte. Si levarono grida e per un momento tutto si fece confuso, tanto che il maresciallo del governatorato dovette invocare l'ordine. 

- Bisogna votare, votare! Chi è nobile capisce\ldots{} Noi versiamo il sangue\ldots{} La fiducia del monarca\ldots{} Non fare i conti col maresciallo\ldots{} non è mica un amministratore\ldots{} Ma non si tratta di questo\ldots{} Permettete che si vada alla votazione!\ldots{} - si udiva gridare animosamente, con violenza, da tutte le parti. Gli sguardi e le espressioni erano ancora più irritati e più violenti delle grida. Esprimevano un odio irriconciliabile. Levin non riusciva in nessun modo a capire di che si trattasse e si stupiva della passionalità con cui si esaminava la questione se mettere o no ai voti la opinione su Flerov. Dimenticava, come poi gli spiegò Sergej Ivanovic, questo sillogismo: che per il bene comune bisognava far cadere il maresciallo del governatorato; per far cadere il maresciallo era necessaria la maggioranza dei voti, per la maggioranza dei voti bisognava dare a Flerov il diritto di voto; per riconoscere Flerov idoneo, bisognava spiegare l'articolo della legge. 

- E un voto può decidere tutto l'affare, bisogna essere seri e coerenti, se si vuol servire la causa pubblica - concluse Sergej Ivanovic. 

Ma Levin se l'era dimenticato, e gli era penoso vedere quelle brave persone, da lui stimate, in un'eccitazione così spiacevole e perversa. Per liberarsi da questa sensazione penosa, senz'attendere la fine del dibattito, se ne andò in una sala dove non c'era nessuno, tranne i servitori vicino a una credenza. Nel vedere i camerieri, affaccendati a rasciugar stoviglie e a disporre piatti e bicchieri, nel vedere i loro visi calmi, animati, Levin provò un inaspettato senso di sollievo, come se da una stanza maleodorante fosse uscito all'aria aperta. Si mise a camminare avanti e indietro, guardando con soddisfazione i camerieri. Gli piaceva molto che un cameriere con le fedine grige, disprezzando gli altri giovani che lo prendevano in giro, insegnasse loro come piegare i tovaglioli. Levin stava proprio per intavolare una conversazione col vecchio cameriere, quando il segretario della tutela nobiliare, un vecchietto che aveva la specialità di conoscere tutti i nobili del governatorato per nome e patronimico, lo distrasse. 

- Prego, Konstantin Dmitric - gli disse - vostro fratello vi cerca. Si vota. 

Levin entrò nella sala, ricevette una pallina bianca e dietro a suo fratello Sergej Ivanovic si avviò al tavolo, presso al quale stava in piedi, con un viso espressivo e ironico, Svijazskij, raccogliendo nel pugno la barba e annusandola. Sergej Ivanovic mise la mano nella cassetta, mise chi sa dove la pallina e, fatto posto a Levin, si fermò proprio lì. Levin si avvicinò ma, avendo completamente dimenticato di che cosa si trattava, ed essendosi confuso, si voltò verso Sergej Ivanovic a chiedergli: ``dove la metto?''. Egli aveva domandato sottovoce, mentre là vicino si parlava, sperando che la sua domanda non fosse udita. Ma quelli che parlavano tacquero, e la sua domanda sconveniente fu udita. Sergej Ivanovic si accigliò: 

- È una questione che riguarda la convinzione del singolo - disse severo. 

Alcuni sorrisero. Levin arrossì, ficcò in fretta la mano sotto il panno e la mise a destra, poiché la palla era nella mano destra. Dopo averla messa ricordò che bisognava metter dentro anche la sinistra, e la mise, ma era tropo tardi, e, confusosi ancor più, si ritirò subito nelle ultime file. 

- Centoventisei favorevoli! Novantotto sfavorevoli! - risonò la voce del segretario che non pronunciava l'erre. Poi si sentirono delle risate: si eran trovati nell'urna un bottone e due noci. Il nobile era ammesso e il partito nuovo aveva vinto. 

Ma il partito vecchio non si considerava vinto. Levin sentì che pregavano Snetkov di presentarsi al ballottaggio, e vide che una folla di nobili circondava il maresciallo del governatorato che diceva qualcosa. Levin si fece dappresso. Rispondendo ai nobili, Snetkov parlava della fiducia della nobiltà, dell'amore per lui, che non meritava, poiché tutto il suo merito consisteva nella dedizione alla nobiltà, alla quale egli aveva dedicato dodici anni di servizio. Parecchie volte egli ripeté le parole: ``ho servito con quante forze avevo, con fede e verità, apprezzo e ringrazio'' e improvvisamente si fermò per le lacrime che lo soffocavano, e uscì dalla sala. Derivassero queste lacrime dalla coscienza d'una ingiustizia fattagli, o dall'amore per la nobiltà, o dalla tensione dovuta allo stato in cui si trovava nel sentirsi circondato da nemici, fatto sta che l'agitazione si comunicò, e la maggior parte dei nobili erano commossi, e Levin provò tenerezza per Snetkov. 

Sulla porta il maresciallo del governatorato si scontrò con Levin. 

- Chiedo scusa, perdonatemi, vi prego - disse egli come a un estraneo; ma, riconosciuto Levin, sorrise timidamente. A Levin sembrò ch'egli volesse dire qualcosa, ma che non potesse per l'agitazione. La espressione del viso e di tutta la sua figura in divisa, con le croci e i pantaloni bianchi gallonati, mentre camminava in fretta, ricordò a Levin una bestia inseguita che vede la propria situazione farsi cattiva. Questa espressione sul viso del maresciallo era in particolar modo emozionante per Levin, perché, soltanto il giorno prima, era stato a casa sua per l'affare della tutela e l'aveva visto in tutta la sua grandezza di uomo buono e casalingo. Una grande casa con vecchi mobili di famiglia; vecchi camerieri senza eleganza, un po' trasandati, ma rispettosi, che evidentemente provenivano ancora da servi della gleba che non avevano mai cambiato padrone; una moglie grassa e cordiale, in cuffia con pizzi e scialle turco che carezzava una graziosa nipotina, figlia della figlia; un bel figliuolo, allievo della sesta classe del ginnasio, che tornava dalla scuola e che, salutando il padre, gli aveva baciato la mano grossa; le parole e i gesti solenni, affabili del padrone di casa, tutto questo il giorno prima aveva destato in Levin un involontario rispetto e una forte simpatia. Per Levin adesso quel vecchio era commovente e pietoso, e voleva dirgli qualcosa di cordiale. 

- Siete dunque di nuovo il nostro maresciallo - disse. 

- È difficile - disse il maresciallo, voltandosi a guardare spaventato. - Io sono stanco, già vecchio. Ce n'è di più degni e di più giovani di me; che questi prestino servizio. 

E il maresciallo sparve per una porta laterale. 

Venne il momento più solenne. Si doveva procedere alle elezioni. I capi dell'uno e dell'altro partito calcolavano sulle dita le palline bianche e le nere. 

Le discussioni su Flerov avevano dato al partito nuovo non solo il voto di Flerov, ma anche il tempo per portare a votare tre nobili che erano stati privati, dagli intrighi del partito vecchio, della possibilità di partecipare alle elezioni. Due di essi, che avevano un debole per il vino, erano stati ubriacati dai fautori di Snetkov, e a un terzo era stata tolta la divisa. 

Saputo ciò, il partito nuovo, durante le discussioni su Flerov, aveva avuto il tempo di mandare con una vettura qualcuno dei suoi a procurare una divisa al nobile e a portare uno dei due ubriachi all'assemblea. 

- Uno l'ho portato, gli ho versato dell'acqua addosso - proferì il proprietario ch'era andato a prelevarlo, avvicinandosi a Svijazskij. - Non c'è male, può andare. 

- Non è ubriaco fradicio, non cascherà? - stava chiedendo Svijazskij. 

- No, sta su da bravo, purché qui non gli diano da bere\ldots{} Ho detto al cameriere che non gliene dia per nessuna ragione al mondo. 

\capitolo{XXIX}La sala stretta nella quale si fumava e si mangiucchiava, era piena di nobili. L'agitazione aumentava sempre più e su tutti i visi si notava l'inquietudine. Erano agitati specialmente i capi che conoscevano tutti i particolari e lo scrutinio di tutti i voti. Erano gli organizzatori del combattimento imminente. Gli altri invece, come i soldati prima della battaglia, anche se pronti a battersi, cercavano di distrarsi. Alcuni mangiavano qualcosa in piedi o seduti al tavolo; altri camminavano, fumando sigarette, avanti e indietro per la stanza lunga e discorrevano con amici che non vedevano da tempo. 

Levin non aveva voglia di mangiare e non fumava; unirsi ai suoi, cioè a Sergej Ivanovic, Stepan Arkad'ic, Svijazskij e gli altri, non voleva, perché stava con loro in animato colloquio Vronskij, in divisa da scudiere. Anche il giorno prima Levin l'aveva visto alle elezioni e lo aveva evitato con cura, non desiderando di incontrarsi con lui. Si avvicinò alla finestra e sedette, osservando i gruppi e prestando orecchio a quel che si diceva intorno a lui. Era triste in particolar modo perché tutti, come vedeva, erano animati, preoccupati, e soltanto lui con un vecchietto sdentato, vecchio decrepito, in divisa di marina, che biascicava con le labbra e s'era venuto a sedere accanto a lui, non prendeva interesse e parte a nulla. 

- È un volpone quello! io glielo dicevo e lui, no. Come! In tre anni non poteva prepararsi - diceva energicamente un proprietario un po' curvo, non alto, coi capelli impomatati sopra il colletto ricamato della divisa, battendo con forza i tacchi degli stivali nuovi, messi per le elezioni. E il proprietario, dopo aver lanciato uno sguardo di insoddisfazione su Levin, si voltò bruscamente. 

- Sì, è un affare poco pulito, non c'è che dire - pronunciò con voce sottile un proprietario di piccola statura. 

Dietro a questi, tutta una folla di proprietari, che circondava un generale grasso, si avvicinò frettolosa a Levin. I proprietari cercavano evidentemente un luogo dove poter parlare senza essere uditi. 

- Come osa dire che io gli ho fatto rubare i calzoni! Se li è bevuti, io penso. Io gli posso sputare addosso, a lui e al suo principato! Che non osi dirlo, roba da porci! 

- Ma permettete dunque! Essi si basano sull'articolo - dicevano in un altro gruppo - la moglie deve essere iscritta come nobile. 

- E al diavolo l'articolo! Io parlo sinceramente. Per questo siamo nobili galantuomini. Abbi fiducia. 

- Eccellenza, andiamo a bere, fine champagne! 

Un'altra folla andava dietro a un nobile che gridava forte qualcosa; era uno dei tre ubriachi. 

- Io ho sempre consigliato a Mar'ja Semënovna di dare in affitto, perché così ne ricaverà dell'utile - diceva con voce piacevole un proprietario con i baffi grigi, in divisa di colonnello del vecchio Stato Maggiore generale. Era quello stesso proprietario che Levin aveva incontrato da Svijazskij. Lo riconobbe subito. Anche il proprietario guardò Levin con attenzione e si salutarono. 

- Con molto piacere. E come! Ricordo benissimo. L'anno scorso, da Nikolaj Ivanovic, il maresciallo. 

- E come va la vostra azienda? - chiese Levin. 

- Sempre allo stesso modo, in perdita - rispose il proprietario fermandoglisi accanto con un sorriso rassegnato, ma con un'espressione di calma e di convinzione che la cosa così dovesse andare. 

- E voi come mai siete venuto nel nostro governatorato? - domandò. - Siete venuto per prendere parte al nostro coup d'état? - disse, pronunciando con fermezza, anche se male, le parole francesi. 

- Tutta la Russia vi è convenuta: i ciambellani e, per poco, anche i ministri. - Egli indicò la figura rappresentativa di Stepan Arkad'ic in pantaloni bianchi e divisa da ciambellano, che camminava con un generale. 

- Vi devo confessare che capisco molto male il significato delle elezioni nobiliari - disse Levin. 

Il proprietario lo guardò. 

- Ma che cosa c'è da capire qui? Non c'è nessun significato. Un'istituzione tramontata che continua il suo movimento per forza d'inerzia soltanto. Guardate le divise, anche quelle vi parlano: questa è un'assemblea di giudici di pace, di consiglieri effettivi e via di seguito, ma non di nobili. 

- Allora perché ci venite? - chiese Levin. 

- Per un'abitudine, in primo luogo. Poi, bisogna mantenere le relazioni. È un dovere morale, in un certo modo. E poi, a dire il vero, c'è l'interesse personale. Mio cognato vuol presentarsi al ballottaggio dei membri effettivi; sono persone modeste, bisogna farlo passare. Ecco, questi signori perché ci vengono? - egli disse, indicando quel signore velenoso che aveva parlato al tavolo del governatorato. 

- È la nuova generazione di nobili. 

- Nuova per modo di dire. Ma non è nobiltà. Essi sono possessori di terre e noi siamo proprietari. Loro, come nobili, vanno contro i loro interessi. 

- Allora voi dite che è un'istituzione che ha fatto il suo tempo? 

- Per fare il suo tempo l'ha fatto, tuttavia bisognerebbe considerarla con maggior rispetto. Almeno Snetkov\ldots{} Che siamo buoni o no, siamo pur cresciuti su per mille anni. Sapete, anche se dobbiamo fare un giardinetto davanti alla casa, farne il disegno, e se là cresce un albero di cent'anni\ldots{} Anche se è contorto e vecchio, voi tuttavia per delle aiuole di fiori non taglierete un albero, ma disporrete le aiuole in modo da profittarne. In un anno non lo cresci - egli disse cauto e cambiò immediatamente discorso. - Be', e la vostra azienda come va? 

- Ma, non bene. Un cinque per cento. 

- Sì, ma voi non vi contate. Perché anche voi valete qualcosa. Ecco, vi dirò di me. Finché non ho preso a dirigere l'azienda, in servizio guadagnavo tremila rubli. Ora lavoro più di quanto lavoravo in servizio, e proprio come voi, ne ricavo il cinque per cento, e ancora quello che Dio mi dà. E il mio lavoro è inutile. 

- E allora perché lo fate, se c'è una perdita diretta? 

- Ma ecco, si fa! Che volete? L'abitudine, e poi si sa che così bisogna fare. Vi dirò di più - continuò il proprietario, poggiandosi con i gomiti alla finestra e parlando senza interruzione; - mio figlio non ha nessuna passione per l'azienda. Evidentemente sarà uno studioso. Così che non ci sarà nessuno a continuarla. Eppure si fa sempre, ancora. Ecco, ora ho piantato un giardino. 

- Sì, sì - disse Levin - è proprio vero. Io sento sempre che non c'è un vero tornaconto nella mia azienda, eppure si fa\ldots{} Senti un certo dovere verso la terra. 

- Ma ecco, vi dirò - continuò il proprietario. - È stato da me un vicino che fa il mercante. Abbiamo passeggiato per il podere, per il giardino. ``No, dice, Stepan Vasil'evic, tutto da voi è in ordine, ma il giardino è in abbandono. Da me è in ordine. A criterio mio, codesto bosco di tigli, lo taglierei. Basta farlo quando è in succhio. Perché son mille tigli, da ognuno ne vengon fuori due buone tavole. E oggi giorno le tavole di tiglio sono pregiate, e io ci taglierei tante intelaiature''. 

- E con questi denari lui comprerebbe del bestiame e un po' di terra per quattro soldi e la darebbe in affitto ai contadini - concluse con un sorriso Levin, che evidentemente più di una volta si era imbattuto in calcoli simili. - E si formerà un patrimonio. Invece, voi ed io\ldots{} basta che Dio ci conceda di conservare il nostro e di lasciarlo ai figli. 

- Vi siete ammogliato, ho sentito - disse il proprietario. 

- Sì - rispose Levin con soddisfazione orgogliosa. - Già, è un po' strano - continuò. - Noi viviamo proprio così, senza ricavar nulla, proprio come se fossimo addetti, come le antiche vestali, a custodire una certa fiamma. 

Il proprietario sorrise sotto i baffi bianchi. 

- Ce ne sono anche fra i nostri, ecco, magari il nostro amico Nikolaj Ivanovic o adesso il conte Vronskij che è venuto qui a stabilirsi, i quali vogliono condurre un'azienda agricola; ma finora tutto questo, fuor che distruggere il capitale, non ha portato a niente. 

- Ma perché non facciamo come i mercanti? Perché non tagliamo il giardino per farne tavole? - disse Levin, tornando a un pensiero che l'aveva colpito. 

- Ma ecco, come avete detto voi, per custodire la fiamma. Altrimenti non è lavoro da nobili. E il nostro lavoro da nobili non è qui alle elezioni, ma là, nel nostro angolo. C'è pur sempre il nostro istinto di classe, per quel che si deve e quello che non si deve fare. Ecco, anche i contadini io li osservo bene: appena c'è un bravo contadino, prende in affitto quanta più terra può. Per quanto scadente possa essere la terra, lui la ara sempre. Anche senza tornaconto. Proprio in perdita. 

- Così anche noi - disse Levin. - Sono stato molto, molto contento d'avervi incontrato - aggiunse, avendo visto Svijazskij che si avvicinava. 

- È la prima volta che ci siamo incontrati dopo esserci visti da voi - disse il proprietario - e ci siamo messi a discorrere. 

- Be', avrete detto male degli ordinamenti nuovi? - disse con un sorriso Svijazskij. 

- Non se ne poteva fare a meno. 

- Ci siamo alleviato l'animo. 

\capitolo{XXX}Svijazskij prese Levin sotto braccio e andò con lui verso i suoi. \\
Ormai non si poteva evitare Vronskij. Egli stava con Stepan Arkad'ic e Sergej Ivanovic e guardava Levin che si avvicinava. 

- Molto lieto. Mi pare d'aver avuto il piacere di incontrarvi\ldots{} dalla principessa Šcerbackaja - disse, dando la mano a Levin. 

- Sì, mi ricordo bene del nostro incontro - disse Levin e, fattosi rosso di fuoco, si voltò subito dall'altra parte e si mise a parlare con suo fratello. 

Sorridendo lievemente, Vronskij riprese a parlare con Svijazskij, non avendo, evidentemente, nessun desiderio di intavolare una nuova conversazione con Levin; ma Levin, parlando con il fratello, si voltava di continuo a guardare Vronskij, pensando cosa potergli dire, per cancellare la propria scontrosità. 

- Di che si tratta adesso? - domandò Levin, volgendosi a guardare Svijazskij e Vronskij. 

- Di Snetkov. Bisogna che rifiuti o accetti - rispose Svijazskij. 

- Ma lui che ha fatto, ha consentito o no? 

- Qui sta la faccenda, che non ha fatto né l'una né l'altra cosa - disse Vronskij. 

- E se rifiuterà, chi entrerà in ballottaggio? - domandò Levin, guardando Vronskij. 

- Chi vorrà - disse Svijazskij. 

- Voi lo farete? - domandò Levin. 

- Io no di certo - disse Svijazskij, confondendosi e gettando uno sguardo spaventato al signore velenoso che stava dritto lì accanto, insieme a Sergej Ivanovic. 

- E allora chi? Nevedovskij? - disse Levin, sentendo di confondersi. 

Ma era ancora peggio, Nevedovskij e Svijazskij erano due candidati. 

- Io, poi, in nessun caso - disse il signore velenoso. 

Era Nevedovskij in persona. Svijazskij gli presentò Levin. 

- Be', anche tu ci sei entrato in pieno! - disse Stepan Arkad'ic, strizzando l'occhio a Vronskij. - È un po' come alle corse. Ci si può scommettere. 

- Già, prende in pieno - disse Vronskij. - E, una volta che si è dentro, si vuol arrivare fino in fondo. La lotta! - disse, accigliandosi e stringendo gli zigomi forti. 

- Che uomo d'affari quel Svijazskij! Tutto è chiaro per lui. 

- Oh, sì - disse distratto Vronskij. 

Seguì un silenzio durante il quale Vronskij, tanto per guardare qualcosa, osservò Levin, le sue gambe, la divisa, poi il viso, e notando gli occhi cupi rivolti su di lui, per dir qualcosa, chiese: 

- Come mai voi che siete un abitatore fisso della campagna, non siete giudice di pace? Non ne portate la divisa. 

- Perché ritengo che il tribunale di pace sia un'istituzione sciocca - rispose Levin, che aveva sempre aspettato l'occasione per mettersi a parlare con Vronskij e dissipare così la scontrosità del primo incontro. 

- Io non lo credo, al contrario - disse Vronskij con calma sorpresa. 

- È un giuoco - l'interruppe Levin. - I giudici di pace non sono necessari. Io in otto anni non ho avuto neanche una causa. E quelle che ho avuto, sono state giudicate alla rovescia. Il giudice di pace è a quaranta verste da me. Io, per una causa che vale due rubli, devo mandare un procuratore che ne costa quindici. 

E raccontò come un contadino avesse rubato la farina al mugnaio, e come, quando il mugnaio glielo aveva detto, il contadino l'avesse citato per calunnia. Tutto ciò era fuor di proposito e sciocco, e Levin, mentre parlava, lo sentiva lui stesso. 

- Oh, è un tal originale! - disse Stepan Arkad'ic col suo sorriso più dolce. - Andiamo però, mi pare che si voti\ldots{} 

E si separarono. 

- Io non capisco - disse Sergej Ivanovic, che aveva notato l'uscita spiacevole del fratello - io non capisco come si possa esser privi fino a tal punto di qualsiasi tatto politico. Ecco quello che noi russi non abbiamo. Il maresciallo del governatorato è nostro avversario, tu sei ami cochon con lui e lo preghi di entrare in ballottaggio. E il conte Vronskij\ldots{} io non me ne farò un amico; m'ha invitato a pranzo, io non andrò da lui, ma è uno dei nostri, perché mai farsene un nemico? Poi tu domandi a Nevedovskij se entrerà in ballottaggio. Questo non si fa. 

- Ah, io non capisco nulla, e poi tutte queste cose sono sciocchezze - rispose torvo Levin. 

- Ecco, tu mi dici che tutte queste cose sono sciocchezze, ma se ti ci metti dentro, allora imbrogli tutto. 

Levin tacque, ed entrarono insieme nella sala grande. 

Il maresciallo del governatorato, malgrado sentisse nell'aria l'inganno preparatogli, e malgrado non tutti l'avessero pregato, decise tuttavia di entrare in ballottaggio. Si fece silenzio nella sala, il segretario annunciò a voce alta che veniva messo ai voti come maresciallo del governatorato il capitano di cavalleria della Guardia Michail Stepanovic Snetkov. 

I marescialli distrettuali passavano, con dei vassoi in cui erano le palle, dalle proprie tavole a quella del governatorato e le elezioni cominciarono. 

- Metti a destra - sussurrò Stepan Arkad'ic a Levin, quando egli, insieme col fratello, si accostò alla tavola dietro al maresciallo. Ma Levin aveva dimenticato il calcolo che gli avevano spiegato, temeva che Stepan Arkad'ic si fosse sbagliato, dicendo: ``a destra''. Snetkov infatti era un nemico. Avvicinandosi alla cassetta egli teneva la palla nella destra, ma pensando d'essersi sbagliato, proprio dinanzi all'urna, passò la palla nella mano sinistra e, in maniera evidente, la mise successivamente a sinistra. Un conoscitore della faccenda, che stava ritto presso l'urna, e che dal solo movimento del gomito capiva dove ciascuno avrebbe messo la palla, fece una smorfia di scontento. Non aveva su che cosa esercitare la propria penetrazione. 

Si fece silenzio e si sentì il conto delle palle. Dopo, una voce isolata, proclamò il numero delle favorevoli e delle sfavorevoli. 

Il maresciallo era eletto con una maggioranza considerevole. Tutti presero a far rumore e si lanciarono impetuosamente verso la porta. Snetkov entrò e la nobiltà lo circondò, congratulandosi. 

- Be', adesso è finita? - domandò Levin a Sergej Ivanovic. 

- Comincia solo adesso - disse, sorridendo, Svijazskij per Sergej Ivanovic. - Il candidato a maresciallo può ricevere più voti. 

Levin l'aveva di nuovo dimenticato. Si ricordò soltanto adesso che lì c'era una certa sottigliezza, ma per lui era noioso ricordarsi in che cosa consistesse. Si sentiva abbattuto, e voleva uscire da quella folla. 

Poiché nessuno faceva attenzione a lui, e lui, a quel che sembrava, non serviva a nessuno, si diresse pian piano nella sala piccola dove si mangiava e provò un gran sollievo nel veder di nuovo i camerieri. Il vecchio gli offrì da mangiare e Levin acconsentì. Dopo aver mangiato una costoletta con fagioli e dopo aver parlato con i camerieri dei signori di una volta, Levin, che non desiderava rientrare nella sala dove provava un'impressione così spiacevole, andò a passeggiare sulla tribuna. 

La tribuna era piena di signore eleganti, che si curvavano sulla balaustrata e cercavano di non perdere neanche una parola di quello che veniva detto giù. Vicino alle signore stavano a sedere avvocati eleganti, professori di ginnasio con gli occhiali e ufficiali. Dovunque si parlava delle elezioni e di come erano state belle le discussioni; in un gruppo Levin sentì una lode rivolta a suo fratello. Una signora diceva a un avvocato: 

- Come sono contenta d'aver sentito Koznyšev! Vale la pena di soffrire un po' la fame. Un incanto! Com'è chiaro e come si sente tutto! Ecco, da voi, in tribunale nessuno parla così. Non c'è che Meidel, e anche quello è lontano dall'essere così eloquente. 

Trovato un posto libero sulla balaustrata, Levin si appoggiò e cominciò a guardare e ad ascoltare. 

Tutti i nobili erano seduti dietro a tramezzi, divisi per distretti. In mezzo alla sala stava in piedi un uomo in divisa e con voce stridula e alta proclamava: 

- È proposto come candidato alla carica di maresciallo del governatorato il capitano in seconda di cavalleria Evgenij Ivanovic Apuchtin! 

Seguì un silenzio di morte e si sentì una debole voce di vecchio: 

- Rinuncia! 

- È proposto il consigliere di corte Pëtr Petrovic Bol' - riprese di nuovo la voce. 

- Rinuncia! - risonò una voce giovane e stridula. 

Di nuovo cominciò la stessa cosa e di nuovo ``rinuncia''. Così durò per quasi un'ora. Levin, appoggiandosi coi gomiti alla balaustrata, guardava e ascoltava. Da principio si sorprendeva e voleva capire che cosa significasse ciò; poi, convintosi di non poterlo capire, cominciò a sentir noia. Poi, ricordando l'agitazione e la cattiveria scorte nelle facce di tutti, gliene venne tristezza: decise di andar via e scese giù. Passando per il vestibolo della tribuna, si imbatté in uno studente, che aveva gli occhi gonfi, e camminava triste avanti e indietro. Sulla scala poi, gli venne incontro una coppia: una signora che correva veloce sui tacchetti, e un agile sostituto procuratore. 

- Ve lo dicevo che non sareste arrivata in ritardo - disse il procuratore, mentre Levin si faceva da parte per cedere il passo alla signora. 

Levin era già sulla scala d'uscita e tirava fuori dalla tasca del panciotto il numero della pelliccia quando il segretario lo afferrò: 

- Favorite, Konstantin Dmitric, si vota. 

Veniva messo in ballottaggio, come candidato, Nevedovskij che aveva rifiutato così recisamente. 

Levin si avvicinò alla porta della sala: era chiusa. Il segretario picchiò: la porta si aprì e incontro a Levin scivolarono via due proprietari, rossi in viso. 

- Non ne posso più - diceva uno dei proprietari tutto rosso in viso. 

Dietro il proprietario spuntò il viso del maresciallo del governatorato. Il suo viso era spaventoso per l'abbattimento e il terrore. 

- Io t'ho detto: non far uscire! - gridò al custode. 

- Ho fatto entrare, eccellenza! 

- Signore! - e, dopo aver sospirato penosamente, il maresciallo del governatorato, sgattaiolando stanco nei suoi pantaloni bianchi, con il capo chino, andò in mezzo alla sala, verso la tavola grande. 

Su Nevedovskij avevano portato i voti, come del resto era stato calcolato, ed egli era maresciallo del governatorato. Molti erano allegri, molti contenti, felici, entusiasti; molti scontenti e infelici. Il maresciallo del governatorato era in una disperazione che non riusciva a nascondere. Quando Nevedovskij uscì dalla sala, la folla lo circondò, e lo seguì entusiasticamente, proprio così come il primo giorno aveva seguito il governatore che aveva aperto le elezioni, e così come aveva seguito Snetkov quando era stato eletto. 

\capitolo{XXXI}Il maresciallo del governatorato appena eletto e molti del partito trionfante dei giovani erano, in quel giorno, a pranzo da Vronskij. 

Vronskij era venuto alle elezioni perché si annoiava in campagna e perché aveva bisogno di affermare i propri diritti di libertà di fronte ad Anna, e per disobbligarsi con Svijazskij, appoggiandolo nelle elezioni, di tutte le beghe che s'era prese per Vronskij alle elezioni del consiglio; ma più di tutto per adempiere rigorosamente tutti gli obblighi di quella posizione di nobile e di proprietario di terre che s'era scelta. Ma non s'aspettava in alcun modo che quella faccenda delle elezioni lo potesse interessare tanto e che potesse riuscirgli così bene. Era una persona completamente nuova nell'ambiente nobiliare, ma evidentemente aveva avuto successo e non si sbagliava, pensando d'aver già acquistato influenza fra i nobili. A questa influenza aveva contribuito la sua ricchezza e notorietà, il bellissimo appartamento in città cedutogli dal vecchio amico Širkov, che si occupava di affari finanziari e che aveva organizzato una fiorente banca a Kašin, l'ottimo cuoco di Vronskij, portato dalla campagna, l'amicizia col governatore che era stato suo compagno, ed era ancora un compagno protetto da Vronskij, e, più di tutto, i suoi modi semplici, eguali verso tutti, che avevano costretto molto presto i nobili a mutar giudizio sulla sua presunta superbia. Egli stesso sentiva che, tranne quello stravagante signore che aveva sposato Kitty Šcerbackaja, che, à propos des bottes, gli aveva detto con furiosa irritazione un mucchio di sciocchezze che non c'entravano per niente, ogni nobile di cui aveva fatto la conoscenza era divenuto suo partigiano. Vedeva chiaramente, e gli altri lo riconoscevano, che al successo di Nevedovskij aveva cooperato molto lui. E adesso, alla propria tavola, nel festeggiare l'elezione di Nevedovskij, egli provava una piacevole sensazione di trionfo per il proprio eletto. Le stesse elezioni lo avevano talmente preso che, se fosse riuscito a regolare la sua posizione di marito, pensava, per il futuro triennio, di entrare lui stesso in ballottaggio, quasi come, dopo aver vinto un premio per mezzo di un fantino, gli fosse venuta la voglia di correre lui stesso. 

Adesso, invece, si festeggiava la vittoria del fantino. Vronskij sedeva a capotavola, alla sua destra sedeva il governatore, generale di corte. Per tutti gli altri questi era il padrone del governatorato, colui che aveva aperto solennemente le elezioni, pronunciato un discorso e suscitato in molti, come aveva visto Vronskij, stima e servilità; per Vronskij, invece, era Maslov Kat'ka, così era soprannominato al corpo dei paggi, che dinanzi a lui si confondeva e che egli cercava di mettre à son aise. A sinistra sedeva Nevedovskij col suo viso giovane, imperturbabile e velenoso. Con lui Vronskij era semplice e rispettoso. 

Svijazskij sopportava allegramente il proprio smacco. Non era neppure uno smacco per lui, come egli stesso disse, rivolgendosi con la coppa a Nevedovskij: non si poteva trovare un rappresentante migliore di quella nuova tendenza che la nobiltà doveva seguire. E perciò tutti gli onesti, com'egli disse, erano dalla parte del successo di oggi e lo celebravano solennemente. 

Stepan Arkad'ic pure era felice e perché aveva passato allegramente il tempo e perché tutti erano di buon umore. Durante l'ottimo pranzo si rievocarono gli episodi delle elezioni. Svijazskij riferì comicamente il lacrimoso discorso del maresciallo e notò, rivolto a Nevedovskij, che sua eccellenza avrebbe dovuto scegliere un controllo diverso e più complesso delle somme che non le lacrime. Un altro nobile scherzoso raccontò come fossero stati fatti venire i camerieri con le calze lunghe per il ballo del maresciallo del governatorato e come adesso si sarebbe dovuti rimandarli indietro, se il nuovo maresciallo del governatorato non avesse dato il ballo con i camerieri in calze lunghe. 

Durante il pranzo, quando le persone si rivolgevano a Nevedovskij, dicevano continuamente: ``il nostro maresciallo del governatorato'' e ``vostra eccellenza''. 

Questo era pronunciato col medesimo piacere con cui si chiama una giovane donna ``madame'' e con il cognome del marito. Nevedovskij fingeva non solo d'essere indifferente, ma anche di spregiare quel titolo; ma era evidente che ne era felice e che si conteneva per non mostrare un entusiasmo poco adatto a quell'ambiente nuovo, liberale, in cui si trovavano tutti. 

Durante il pranzo, furono spediti alcuni telegrammi a persone che si interessavano dell'andamento delle elezioni. E Stepan Arkad'ic, che era molto allegro, mandò a Dar'ja Aleksandrovna un telegramma così concepito: ``Nevedovskij eletto con venti voti. Sono felice. Comunica la notizia''. Lo dettò ad alta voce, notando: ``bisogna farli contenti''. Dar'ja Aleksandrovna, invece, ricevuto il telegramma, sospirò soltanto per il rublo del telegramma e capì che la cosa era avvenuta alla fine d'un pranzo. Ella sapeva che Stiva aveva la debolezza, alla fine dei pranzi, di faire jouer le télégraphe. 

Tutto, compreso l'ottimo pranzo e i vini, acquistati non certo da rivenditori di vino russo, ma direttamente da produttori esteri, fu molto nobile, semplice, allegro. Quel gruppetto di venti persone era stato scelto da Svijazskij fra uomini pubblici delle stesse idee liberali, giovani e nello stesso tempo intelligenti e selezionati. Si fecero dei brindisi, anch'essi semiseri, alla salute del nuovo maresciallo del governatorato, e del governatore, e del direttore della banca e del ``nostro gentile padron di casa''. 

Vronskij era soddisfatto. Non si aspettava per nulla un tono così simpatico in provincia. 

Alla fine del pranzo ci fu ancor più allegria. Il governatore pregò Vronskij di andare a un concerto di beneficenza a favore dei ``fratelli'', che aveva organizzato sua moglie, la quale desiderava conoscerlo. 

- Ci sarà un ballo, e vedrai la nostra bellezza. Davvero, è una cosa straordinaria. 

- Not in my line - rispose Vronskij, al quale piaceva questa espressione, ma sorrise e promise di andare. 

Ancora prima che si alzassero da tavola, quando tutti avevano cominciato a fumare, il cameriere di Vronskij gli si avvicinò con una lettera su di un vassoio. 

- Da Vozdvizenskoe, con un corriere espresso - disse con aria significativa. 

- È sorprendente come assomigli al sostituto procuratore Sventickij - disse in francese, del cameriere, uno degli ospiti, mentre Vronskij leggeva la lettera, accigliato. 

La lettera era di Anna. Ancora prima di leggerla, egli ne sapeva bene il contenuto. Supponendo che le elezioni sarebbero finite entro cinque giorni, aveva promesso di tornare venerdì. Si era al sabato, ed egli sapeva che il contenuto della lettera consisteva nei rimproveri per il ritardo. La lettera da lui spedita il giorno innanzi probabilmente non era ancora giunta. 

Il contenuto era quello che si aspettava, ma la forma era inaspettata e particolarmente spiacevole per lui. ``Annie è molto malata. Il dottore dice che può trattarsi di infiammazione. Io sola perdo la testa. La principessa Varvara non è un aiuto, ma un intralcio. T'ho aspettato ieri l'altro, ieri, e adesso mando a domandare dove sei e che fai. Volevo venire io stessa, ma ho cambiato idea, pensando che ti sarebbe spiaciuto. Dammi una risposta qualsiasi, purché sappia che cosa fare''. 

La bimba ammalata e lei stessa che voleva venire. La figlia ammalata e questo tono ostile. 

L'innocente allegria delle elezioni e quel cupo, pesante amore a cui egli doveva tornare colpirono Vronskij col loro contrasto. Ma bisognava andare, e, col primo treno della notte, egli partì per tornare a casa. 

\capitolo{XXXII}Prima della partenza di Vronskij per le elezioni, considerando che quelle scenate che si ripetevano fra di loro a ogni partenza potevano soltanto raffreddarlo e non legarlo, Anna si era decisa a fare su di sé tutti gli sforzi possibili per sopportare con calma la separazione da lui. Ma quello sguardo freddo, severo, con cui egli l'aveva guardata quando era venuto ad annunciare la sua partenza, l'aveva offesa, ed egli non era ancora partito, che la calma di lei era già distrutta. 

Ripensando poi in solitudine a quello sguardo, che aveva espresso il diritto alla libertà, ella venne, come sempre a un'unica conclusione: alla coscienza della propria umiliazione. ``Lui ha tutti i diritti, io non ne ho nessuno. Eppure, sapendolo, non doveva far ciò. Ma cosa mai ha fatto? Mi ha guardato con un'espressione fredda, severa. S'intende, è una cosa indefinita, impalpabile, ma prima non c'era, e questo sguardo significa molte cose - ella pensava. - Questo sguardo dimostra che comincia il raffreddamento''. 

E, pur convinta che cominciava il raffreddamento, non poteva far nulla, non poteva cambiar in nulla i suoi rapporti con lui. E sempre, come prima, poteva trattenerlo solo con l'amore e il proprio fascino. E sempre allo stesso modo di prima, con le occupazioni di giorno e con la morfina di notte riusciva a soffocare i pensieri tremendi su quello che sarebbe stato s'egli si fosse disincantato di lei. In verità, c'era ancora un mezzo: non trattenerlo, poiché ella non voleva altro che l'amore di lui, ma legarglisi, essere in una situazione tale ch'egli non la lasciasse. Questo mezzo era il divorzio e il matrimonio. Ed ella cominciò a desiderare questo e si decise a consentire, la prima volta che lui o Stiva avessero preso a parlargliene. 

In tali pensieri ella passò senza di lui i cinque giorni, quegli stessi in cui egli doveva essere assente. 

Le passeggiate, le conversazioni con la principessa Varvara, le visite all'ospedale e soprattutto la lettura, la lettura di un libro dietro l'altro, occuparono il suo tempo. Ma il sesto giorno, quando il cocchiere ritornò senza di lui, ella sentì che non aveva più la forza di soffocare, in nessun modo, il pensiero di lui e di quel ch'egli facesse là. In quello stesso giorno sua figlia s'ammalò. Anna cominciò a curarla, ma anche questo non la distrasse, tanto più che la malattia non era grave. Per quanto si sforzasse, ella non amava quella bambina, e fingere di volerle bene non poteva. Verso la sera di quel giorno, rimasta sola, sentì un tale terrore per lui, che fu sul punto di andare in città; ma dopo aver esitato un po' scrisse quella lettera contraddittoria che Vronskij aveva ricevuta, e, senza rileggerla, la mandò per espresso. La mattina dopo ricevette la lettera di lui e si pentì della propria. Aspettava con terrore il ripetersi di quello sguardo severo ch'egli le aveva lanciato nel partire, specie quando sarebbe venuto a sapere che la bambina non era stata gravemente malata. Tuttavia era contenta d'avergli scritto. Adesso Anna riconosceva già ch'egli sentiva il peso di lei, che lasciava con rammarico la propria libertà per tornare da lei, e, malgrado questo, era contenta che egli sarebbe tornato. Che ne sentisse pure il peso, ma fosse là con lei. In ogni modo ch'ella lo vedesse, sapesse ogni suo movimento. 

Era seduta in salotto, sotto la lampada, con un nuovo libro del Taine e leggeva, prestando orecchio al suono del vento fuori e aspettando da un momento all'altro l'arrivo della carrozza. Parecchie volte le era parso di sentire il suono delle ruote, ma si era sbagliata; finalmente si udì non soltanto il suono delle ruote, ma anche il vociare del cocchiere e un suono sordo nell'ingresso coperto. Perfino la principessa Varvara, che faceva un solitario, lo confermò, e Anna si alzò, rossa di fiamma, ma invece di andare giù, come aveva fatto due volte prima, si fermò. A un tratto provò vergogna del proprio inganno, ma ancor più terrore di come egli l'avrebbe accolta. Il sentimento d'offesa era già passato; ella aveva soltanto paura dell'espressione del suo scontento. Si ricordò che la figlia, già da due giorni, stava perfettamente bene. Le venne perfino rabbia contro di lei che si era rimessa proprio quando aveva mandato la lettera. Poi si ricordò che lui era là. Ne sentì la voce. E, dimentica di tutto, gli corse incontro. 

- Be', come sta Annie? - disse lui, con ansia, di sotto, guardando Anna che gli era corsa incontro. 

Era seduto su di una sedia, e un cameriere gli tirava via uno stivale caldo. 

- Non c'è male, sta meglio. 

- E tu? - egli chiese scotendosi la roba addosso. 

Ella con tutte e due le mani prese un suo braccio e se lo passò intorno alla vita, senza levargli gli occhi di dosso. 

- Be', sono molto contento - egli disse, guardando freddo la pettinatura, il vestito di lei, ch'egli sapeva preparati per lui. 

Tutto ciò gli piaceva, ma quante volte gli era piaciuto! E quell'espressione severamente impietrita ch'ella paventava, si fissò sul viso di lui. 

- Sono molto contento. E tu stai bene? - disse lui, dopo aver asciugato col fazzoletto la barba bagnata e baciandole la mano. 

``È sempre lo stesso - ella pensava - basta ch'egli sia qui e quando è qui, non può, non può, non oserà non amarmi''. 

La serata passò serena e allegra in presenza della principessa Varvara che si lamentava con lui che Anna in sua assenza prendeva la morfina. 

- E che fare? Non potevo dormire\ldots{} I pensieri me lo impedivano. Quando c'è lui non la prendo mai. Quasi mai. 

Egli raccontò delle elezioni, e Anna seppe eccitarlo con domande a parlare proprio di quello che lo rallegrava: del suo successo. Ella gli raccontò tutto quello che gli interessava della casa. E tutte le informazioni di lei erano allegre. 

Ma la sera tardi, quando rimasero soli, Anna, vedendo che era di nuovo pienamente padrona di lui, volle cancellare quella penosa impressione dello sguardo, a causa della lettera. Disse: 

- Di' la verità, ti sei arrabbiato quando hai ricevuto la lettera, e non m'hai creduto? 

Appena detto ciò, capì che, per quanto egli fosse pieno d'amore verso di lei, quella non gliel'aveva perdonata. 

- Sì - disse lui. - La lettera era così strana: un momento Annie era malata, un momento tu stessa volevi venire. 

- Tutto questo era vero. 

- Ma io non ne dubito neppure. 

- No, ne dubiti. Sei scontento, lo vedo. 

- Neppure un attimo. Soltanto sono scontento, è vero, che tu non voglia ammettere che ci siano degli obblighi\ldots{} 

- Gli obblighi di andare al concerto\ldots{} 

- Ma non ne parliamo - egli disse. 

- E perché non parlarne? - disse lei. 

- Voglio dire soltanto che possono capitare degli affari inderogabili. Ecco, adesso, dovrò andare a Mosca per l'affare della casa\ldots{} Ah, Anna, perché sei così suscettibile? Ma non sai ch'io non posso vivere senza di te? 

- Ma se è così - disse Anna, cambiata a un tratto nella voce - allora tu senti il peso di questa vita\ldots{} Sì, verrai per un giorno e partirai, come fanno\ldots{} 

- Anna, questo è crudele. Io sono pronto a dar tutta la vita\ldots{} 

Ma ella non l'ascoltava. 

- Se tu andrai a Mosca, ci verrò anch'io. Non rimarrò qui. O ci dobbiamo separare, o vivere insieme. 

- Lo sai che è l'unico mio desiderio. Ma per questo\ldots{} 

- Ci vuole il divorzio? Gli scriverò. Vedo che non posso vivere così\ldots{} Ma verrò con te a Mosca. 

- Come se minacciassi. Ma io non desidero altro che non separarmi mai da te - disse Vronskij, sorridendo. 

Ma non solo uno sguardo freddo, anche uno sguardo cattivo di uomo perseguitato e accanito balenò nei suoi occhi, mentr'egli diceva queste tenere parole. 

Ella aveva visto questo sguardo e ne aveva intuito con precisione il senso. 

``Se è così è una disgrazia!'' diceva lo sguardo di lui. Fu l'impressione di un attimo ma ella non lo dimenticò più. 

Anna scrisse una lettera al marito, chiedendogli il divorzio, e alla fine di novembre, separatasi dalla principessa Varvara che aveva bisogno di recarsi a Pietroburgo, andò a stabilirsi a Mosca con Vronskij. Aspettando da un giorno all'altro la risposta di Aleksej Aleksandrovic e poi il divorzio, essi si stabilirono, come coniugi, insieme. 
