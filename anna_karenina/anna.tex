% arara: lualatex
% arara: lualatex: { synctex: yes }
\documentclass{scrbook}

\usepackage{fontspec}
\usepackage{polyglossia}
\setmainlanguage{italian}
\setmainfont[Ligatures=TeX]{Adobe Caslon Pro}
\setsansfont[Ligatures=TeX]{Gill Sans MT}

\usepackage[all]{nowidow}
\usepackage[rivers,hyphenation,parindent,lastparline,nosingleletter,homeoarchy]{impnattypo}
\usepackage{microtype}
\usepackage{scrpage2}

\title{Anna Karenina}
\author{Lev Tolstoj}
\date{}

\newcommand{\oldparte}[1]{%
\null\cleardoubleevenemptypage\cleardoubleoddemptypage%
\def\nomeparte{#1}\part*{#1}}
\newcommand{\oldcapitolo}[1]{%
\def\nomecapitolo{#1}\chapter*{#1}%
\addcontentsline{toc}{section}{\protect\numberline{}#1}}

\newcommand{\parte}[1]{%
\null\cleardoubleevenemptypage\cleardoubleoddemptypage%
\def\nomeparte{#1}{\vspace*{\fill}\begin{center}%
\color{rosso}\sffamily\Huge\bfseries#1%
\end{center}\vspace*{\fill}}%
\addcontentsline{toc}{chapter}{\protect\numberline{}#1}\cleardoubleoddemptypage}

\newcommand{\capitolo}[1]{\cleardoubleoddemptypage%
\def\nomecapitolo{#1}{\vspace*{3em}\color{rosso}\sffamily\Huge\bfseries#1\vspace*{5em}}%
\addcontentsline{toc}{section}{\protect\numberline{}#1}}

\usepackage{scrpage2}
\defpagestyle{pagina}{%
{\nomeparte\hfill}%
{\hfill\nomecapitolo}%
{}}%
{%
{\pagemark\hfill}%
{\hfill\pagemark}%
{}%
}
\usepackage{hologo}
\usepackage{geometry}
%\usepackage{showframe}
\usepackage{emptypage}
\usepackage{graphicx}
\usepackage{afterpage}
\usepackage{xcolor}
% colori da
% http://rottencupcakes.com/wp-content/uploads/2010/02/classicdrama.jpg
\definecolor{rosso}{cmyk}{0.11,1,0.75,0.2}
\definecolor{panna}{cmyk}{0,0.05,0.17,0.02}
\definecolor{giallo}{RGB}{255,251,64}

\usepackage{titlesec}
\titleformat{\chapter}{\thispagestyle{empty}\filcenter\color{rosso}\sffamily\Huge\bfseries}{}{}{}
\titlespacing*{\chapter}{0pt}{*10}{*15}
\titleformat{\part}{\thispagestyle{empty}\filcenter\color{rosso}\sffamily\Huge\bfseries}{}{}{}
\titlespacing*{\part}{0pt}{\fill}{\fill}

\begin{document}
\newgeometry{margin=2cm}
\begin{titlepage}
\begin{center}
\pagecolor{rosso}\afterpage{\nopagecolor}
\color{panna}
{\sffamily
\vspace*{\fill}
\resizebox{\textwidth}{!}{ANNA}\newline
\vspace*{5pt}
\resizebox{\textwidth}{!}{KARENINA}
}
\color{giallo}
\vspace*{5pt}
\resizebox{\textwidth}{!}{Lev Tolstoj}
\end{center}
\end{titlepage}
\restoregeometry

\cleardoublepage

\pagestyle{empty}
\vspace*{\fill}
\begin{flushright}\emph{A me la vendetta, io farò ragione}\end{flushright}
\vspace*{\fill}
\cleardoublepage
\thispagestyle{empty}
\tableofcontents
\cleardoublepage

\parte{PARTE PRIMA}\label{parte-prima}
\pagestyle{pagina}

\capitolo{I}\label{i}

Tutte le famiglie felici sono simili le une alle altre; ogni famiglia infelice è infelice a modo suo. 

Tutto era sottosopra in casa Oblonskij. La moglie era venuta a sapere che il marito aveva una relazione con la governante francese che era stata presso di loro, e aveva dichiarato al marito di non poter più vivere con lui nella stessa casa. Questa situazione durava già da tre giorni ed era sentita tormentosamente dagli stessi coniugi e da tutti i membri della famiglia e dai domestici. Tutti i membri della famiglia e i domestici sentivano che non c'era senso nella loro convivenza, e che della gente incontratasi per caso in una qualsiasi locanda sarebbe stata più legata fra di sé che non loro, membri della famiglia e domestici degli Oblonskij. La moglie non usciva dalle sue stanze; il marito era già il terzo giorno che non rincasava. I bambini correvano per la casa abbandonati a loro stessi; la governante inglese si era bisticciata con la dispensiera e aveva scritto un biglietto ad un'amica chiedendo che le cercasse un posto; il cuoco se n'era già andato via il giorno prima durante il pranzo; sguattera e cocchiere avevano chiesto di essere liquidati. 

Tre giorni dopo il litigio, il principe Stepan Arkad'ic Oblonskij - Stiva, com'era chiamato in società - all'ora solita, cioè alle otto del mattino, si svegliò non nella camera della moglie, ma nello studio, sul divano marocchino. Rigirò il corpo pienotto e ben curato sulle molle del divano, come se volesse riaddormentarsi di nuovo a lungo, rivoltò il cuscino, lo abbracciò forte e vi appoggiò la guancia; ma a un tratto fece un balzo, sedette sul divano e aprì gli occhi. 

``Già già, com'è andata? - pensava riandando al sogno. - Già, com'è andata? Ecco\ldots{} Alabin aveva dato un pranzo a Darmstadt; no, non Darmstadt, ma qualcosa d'America. Già, ma là, Darmstadt era in America. Sì, sì, Alabin aveva dato un pranzo su tavoli di vetro, già, e i tavoli cantavano `Il mio tesoro', eh no, non `Il mio tesoro', ma qualcosa di meglio; e c'erano poi certe piccole caraffe, ed anche queste erano donne'' ricordava. 

Gli occhi di Stepan Arkad'ic presero a brillare allegramente ed egli ricominciò a pensare sorridendo: ``Eh già, si stava bene, tanto bene. Ottime cose là; ma prova un po' a parlarne e a pensarne; da sveglio neanche arrivi a dirle''. E, notata una striscia di luce che filtrava da un lato della cortina di panno, sporse allegramente i piedi fuori dal divano, cercò con essi le pantofole di marocchino dorato ricamategli dalla moglie (dono per l'ultimo suo compleanno), e per vecchia abitudine, ormai di nove anni, senza alzarsi, allungò il braccio verso il posto dove, nella camera matrimoniale, era appesa la vestaglia. E in quel momento, a un tratto, ricordò come e perché non dormiva nella camera della moglie, ma nello studio, il sorriso gli sparve dal volto; corrugò la fronte. 

- Ahi, ahi, ahi! - mugolò, ricordando quanto era accaduto, e gli si presentarono di nuovo alla mente tutti i particolari del litigio, la situazione senza via di uscita e, più tormentosa di tutto, la propria colpa. 

``Già, lei non perdonerà, non può perdonare. E quel ch'è peggio è che la colpa di tutto è mia\ldots{} la colpa è mia, eppure non sono colpevole! Proprio in questo sta il dramma'' pensava. ``Ahi, ahi, ahi!'' ripeteva con disperazione, ricordando le impressioni più penose per lui di quella rottura. 

Più spiacevole di tutto il primo momento, quando, tornato da teatro, allegro e soddisfatto, con un'enorme pera in mano per la moglie, non l'aveva trovata nel salotto; con sorpresa non l'aveva trovata neanche nello studio, e infine l'aveva scorta in camera con in mano il malaugurato biglietto che aveva rivelato ogni cosa. 

Lei, quella Dolly eternamente preoccupata e inquieta, e non profonda, come egli la giudicava, sedeva immobile, con il biglietto in mano, e lo guardava con un'espressione di orrore, d'esasperazione e di rabbia. 

- Cos'è questo biglietto, cos'è? - chiedeva mostrando il biglietto. 

E a quel ricordo, come talvolta accade, ciò che tormentava Stepan Arkad'ic non era tanto il fatto in se stesso, quanto il modo col quale egli aveva risposto alle parole della moglie. 

Gli era accaduto in quel momento quello che accade alle persone che vengono inaspettatamente accusate di qualcosa di troppo vergognoso. Non aveva saputo adattare il viso alla situazione in cui era venuto a trovarsi di fronte alla moglie dopo la scoperta della propria colpa. Invece di offendersi, negare, giustificarsi, chiedere perdono, rimanere magari indifferente - tutto sarebbe stato meglio di quel che aveva fatto - il suo viso, in modo del tutto involontario (azione riflessa del cervello, pensò Stepan Arkad'ic, che amava la fisiologia), in modo del tutto involontario, aveva improvvisamente sorriso del suo usuale, buono e perciò stupido sorriso. 

Questo stupido sorriso non riusciva a perdonarselo. Visto quel sorriso, Dolly aveva rabbrividito come per un dolore fisico; era scoppiata, con l'impeto che le era proprio, in un diluvio di parole dure, ed era corsa via di camera. Da quel momento non aveva più voluto vedere il marito. 

``Tutta colpa di quello stupido sorriso - pensava Stepan Arkad'ic. - Ma che fare, che fare?'' si chiedeva con disperazione, e non trovava risposta 

\capitolo{II}\label{ii} 

Stepan Arkad'ic era un uomo leale con se stesso. Non poteva ingannare se stesso e convincersi d'essere pentito del suo modo di agire. Non poteva, in questo momento, pentirsi di non essere più innamorato - lui, bell'uomo trentaquattrenne, facile all'amore - di sua moglie, di un anno solo più giovane, madre di cinque bambini vivi e di tre morti. Era pentito solo di non averlo saputo nascondere più abilmente alla moglie. Ma sentiva tutto il peso di questa situazione e commiserava la moglie, i figli e se stesso. Forse avrebbe cercato di nascondere più accortamente le proprie colpe alla moglie, se avesse previsto che questa scoperta avrebbe agito tanto su di lei. A questo non aveva riflettuto mai con chiarezza; tuttavia, vagamente, si figurava che sua moglie, da tempo, indovinasse che egli non le era fedele e chiudesse un occhio. Gli sembrava inoltre che lei, donna esaurita, invecchiata, non più bella e per nulla affatto interessante, semplice, buona madre di famiglia soltanto, dovesse, per un senso di giustizia, essere indulgente. Era avvenuto il contrario. 

``Ah, è terribile! Ahi, ahi, ahi, ahi! Terribile! - si ripeteva Stepan Arkad'ic e non riusciva a trovare una via d'uscita. - E come andava tutto bene prima d'ora! Come vivevamo bene! Lei era contenta, felice dei bambini; io non l'ostacolavo in nulla, la lasciavo libera di regolarsi come voleva, coi bambini, con la casa. È vero, non è bello che quella sia stata governante in casa nostra! Non è bello! C'è qualcosa di triviale, di volgare nel far la corte alla propria governante. Ma che governante! - e ricordò con vivezza il riso e gli occhi neri assassini di m.lle Rolland. - Del resto finché è stata in casa nostra, io non mi sono permesso nulla. E il peggio di tutto è che già\ldots{} Ci voleva proprio tutto questo, neanche a farlo apposta! Ah, ahi, ahi! Ma che fare, che fare?'' 

Una risposta che non c'era all'infuori della risposta comune che dà la vita a tutte le più complicate e insolubili questioni, e la risposta è questa: bisogna vivere delle piccole necessità del giorno, smemorarsi. Nel sogno non è più possibile; almeno fino a stanotte, non si può tornare alla musica che cantavano le donne-caraffe; ci si deve dunque smemorare con il sonno della vita. 

``Staremo a vedere'' si disse Stepan Arkad'ic e, alzatosi, indossò la veste da camera grigia dalla fodera di seta azzurra, fermò i due lacci con un nodo, e introdotta aria a sazietà nella vasta cavità toracica, coll'usuale passo deciso dei suoi piedi all'infuori che così leggermente sostenevano il corpo pienotto, si avviò alla finestra, sollevò la tenda e sonò forte. Entrò subito il suo vecchio amico, Matvej il maggiordomo, che portava il vestito, le scarpe e un telegramma. Dietro a Matvej entrò anche il barbiere con l'occorrente per la barba. 

- Ci sono carte d'ufficio? - chiese Stepan Arkad'ic dopo aver preso il telegramma, sedendosi di fronte allo specchio. 

- Sulla tavola - rispose Matvej. Guardò interrogativamente, con interesse, il padrone, e, dopo aver atteso un poco, aggiunse con un sorriso ammaliziato: - Sono venuti da parte del signor cocchiere. 

Stepan Arkad'ic non rispose nulla e guardò soltanto Matvej nello specchio: nello sguardo che incrociarono era evidente come si intendessero l'un l'altro. Lo sguardo di Stepan Arkad'ic sembrava chiedere: ``Perché dici questo? che forse non sai?''. Matvej ficcò le mani nelle tasche del giubbetto, tirò indietro una gamba in silenzio, bonariamente, sorridendo appena, guardò il padrone. 

- Ho detto loro di venire la prossima domenica, e che fino allora non si disturbino e non disturbino voi inutilmente - disse con una frase evidentemente già preparata. 

Stepan Arkad'ic capì che Matvej voleva scherzare e attirare su di sé l'attenzione. Aperto il telegramma, lo lesse, correggendo per intuito le parole, come sempre alterate, e il viso gli si illuminò. 

- Matvej, mia sorella Anna Arkad'evna viene domani - disse, arrestando per un attimo la mano lustra e grassoccia del barbiere che andava tracciando una via rosea tra le lunghe fedine ricciute. 

- Sia lodato Iddio - disse Matvej, mostrando con la risposta di capire, allo stesso modo del padrone, il significato di questo arrivo, e che cioè Anna Arkad'evna, sorella carissima di Stepan Arkad'ic, poteva contribuire alla riconciliazione tra marito e moglie. 

- Sola o col consorte? - chiese Matvej. 

Stepan Arkad'ic, che non poteva parlare perché il barbiere era alle prese col labbro superiore, alzò un dito solo. Matvej fece cenno col capo nello specchio. 

- Sola. C'é da preparare di sopra? 

- Chiedilo a Dar'ja Aleksandrovna; dove dirà lei. 

- A Dar'ja Aleksandrovna? - ripeté con aria dubbiosa Matvej. 

- Sì, diglielo. Ecco, prendi il telegramma, riferiscimi poi. 

``Volete provare'' pensò Matvej, ma disse solo: 

- Sissignore. 

Stepan Arkad'ic era già lavato e pettinato e si preparava a vestirsi quando Matvej, camminando lentamente con le scarpe che scricchiolavano, rientrò nella stanza col telegramma in mano. Il barbiere era già andato via. 

- Dar'ja Aleksandrovna ha ordinato di dirvi che parte. Che faccia pure come piace a lui, cioè a voi - disse, ridendo solo con gli occhi e, cacciate le mani in tasca e chinato il capo da un lato, fissò il padrone. 

Stepan Arkad'ic tacque. Poi un sorriso buono e un po' pietoso apparve sul suo bel viso. 

- Eh, Matvej - disse, scotendo il capo. 

- Non è nulla, signore; tutto si appianerà - disse Matvej. 

- Si appianerà? 

- Proprio così. 

- Credi? Chi c'è di là? - chiese Stepan Arkad'ic sentendo dietro la porta un fruscio di abito femminile. 

- Sono io, signore - disse una voce di donna, e di dietro la porta si sporse il viso severo e butterato di Matrëna Filimonovna, la njanja. 

- E allora, Matrëna? - domandò Stepan Arkad'ic andandole incontro sulla porta. Sebbene Stepan Arkad'ic fosse per ogni verso colpevole di fronte alla moglie, ed egli stesso lo sentisse, quasi tutti in casa, persino la njanja, la più grande amica di Dar'ja Aleksandrovna, erano dalla parte sua. 

- E allora? - disse con aria afflitta. 

- Andate da lei, signore, dichiaratevi ancora colpevole. Forse Iddio lo concederà. Si tormenta molto ed è una pena guardarla, e poi tutto in casa va alla malora. Ci si deve preoccupare dei bambini, signore. Accusatevi, signore. Che fare? Fatto il male\ldots{} 

- Eh già, non mi riceverà\ldots{} 

- E voi fate il dover vostro. Dio è misericordioso, pregate Iddio, signore, pregate Iddio. 

- E va bene; va'\ldots{} - disse Stepan Arkad'ic, arrossendo improvvisamente. - Su vestiamoci - disse rivolto a Matvej, e con fare deciso si tolse la veste da camera. 

Matvej teneva in mano, soffiandovi sopra come a togliere qualcosa di invisibile, la camicia disposta a collare, e con evidente soddisfazione ne circondò il corpo ben curato del padrone. 

\capitolo{III}\label{iii} 

Vestitosi, Stepan Arkad'ic si spruzzò di profumo, assestò le maniche della camicia, distribuì per le tasche con gesti abituali le sigarette, il portafoglio, i fiammiferi, l'orologio con la catena doppia e i ciondoli e, spiegazzato il fazzoletto, sentendosi pulito, profumato, sano e, malgrado il suo guaio, fisicamente allegro, si avviò, tentennando leggermente su ciascuna gamba, verso la sala da pranzo dove già l'aspettavano il caffè e, accanto al caffè, le lettere e le carte del tribunale. 

Lesse le lettere. Una era molto spiacevole: era del compratore del bosco di sua moglie. Il bosco doveva essere necessariamente venduto; ma ora, fino alla riconciliazione, non se ne poteva parlare. Più increscioso di tutto era il fatto che si veniva in tal modo a frammischiare una questione di denaro al prossimo avvenimento della riconciliazione. E il pensiero ch'egli potesse lasciarsi guidare da una questione di denaro, che per la vendita del bosco cercasse di far pace con la moglie, questo pensiero l'offendeva. 

Letta la posta, Stepan Arkad'ic tirò a sé le carte d'ufficio: sfogliò in fretta due pratiche, segnò con un grosso lapis qualche annotazione e, allontanate le carte, cominciò a sorbire il caffè e nello stesso tempo, aperto il giornale della mattina, ancora umido, prese a leggerlo. 

Stepan Arkad'ic riceveva e leggeva un giornale liberale, non estremista, ma della tendenza che la maggioranza sosteneva. Benché non lo interessassero in modo particolare né scienza, né arte, né politica, egli si atteneva strettamente alle opinioni alle quali, in tutte queste materie, si attenevano la maggioranza e il suo giornale, e le cambiava soltanto quando le cambiava la maggioranza, o per meglio dire non lui le cambiava, ma esse stesse, inavvertitamente, si cambiavano in lui. 

Stepan Arkad'ic non sceglieva né le tendenze né le opinioni, ma queste stesse tendenze e opinioni giungevano a lui da sole, proprio allo stesso modo come non lui sceglieva la foggia del cappello o del soprabito, ma adottava quella che era di moda. E per lui, che viveva nella società più in vista, avere delle opinioni, oltre al bisogno di una certa attività di pensiero che normalmente si sviluppa negli anni della maturità, era così indispensabile come avere un cappello. E anche se c'era una ragione per preferire la tendenza liberale a quella conservatrice, cui si atteneva la maggioranza del suo ambiente, questa consisteva non solo nel fatto che egli trovava la tendenza liberale più ragionevole, ma anche perché questa era in realtà più conforme al suo modo di vivere. Il partito liberale diceva che in Russia tutto andava male, ed in effetti Stepan Arkad'ic aveva molti debiti e il denaro non gli bastava proprio. Il partito liberale diceva che il matrimonio era un'istituzione superata ed era necessario riformarlo, e in realtà la vita familiare dava scarse soddisfazioni a Stepan Arkad'ic e lo costringeva a mentire e a fingere, il che era affatto avverso alla sua natura. Il partito liberale diceva, o meglio faceva intendere, che la religione era soltanto un freno per la parte incolta della popolazione, e in realtà Stepan Arkad'ic non poteva sopportare, senza che gli dolessero le gambe, neppure il più piccolo Te Deum, e non poteva capire che senso avessero tutte quelle tremende altisonanti parole sull'altro mondo, quando anche in questo era così piacevole vivere. Inoltre a Stepan Arkad'ic, che amava gli scherzi ameni, faceva piacere turbare talvolta qualche pacifico essere col dire, che se ci si vuole inorgoglire della razza, non conviene fermarsi a Rjurik e rinnegare il progenitore, la scimmia. Dunque le opinioni liberali erano divenute un'abitudine per Stepan Arkad'ic e gli piaceva il suo giornale, così come il sigaro dopo il pranzo, per quella leggera nebbia che gli generava in testa. Lesse l'articolo di fondo, nel quale si spiegava che ``al tempo nostro del tutto invano si levan querele contro il radicalismo, il quale minaccia di inghiottire tutti gli elementi conservatori, e che il governo non si decide a prendere delle misure per soffocare l'idra rivoluzionaria; che al contrario, secondo la nostra opinione, il pericolo risiede non già nella presunta idra rivoluzionaria, ma nel tradizionalismo ostinato che rallenta il progresso'' e così di seguito. Lesse anche un altro articolo, finanziario, nel quale si parlava del Bentham e dello Stuart Mill e si lanciavano frecciate al ministero. Con la prontezza di spirito che gli era propria egli afferrava il senso di ogni frecciata: da chi veniva e contro chi era diretta e in quale occasione, e questo, come sempre, gli procurava un certo piacere. Ma oggi questo piacere era avvelenato dal ricordo dei consigli di Matrëna Filimonovna e dal fatto che in casa tutto andava tanto male. Lesse pure che il conte Beist, come correva voce, era partito per Wiesbaden, e che si vendeva una carrozza leggera, e che una persona giovane faceva una proposta; ma queste notizie non gli davano più il solito tranquillo, ironico compiacimento di una volta. 

Finito il giornale, la seconda tazza di caffè e la ciambellina al burro, s'alzò scrollando le briciole dal panciotto e, allargando il petto ampio, sorrise di piacere: non perché avesse in animo qualcosa di particolarmente lieto, ma solo perché la buona digestione gli procurava quel sorriso di gioia. 

Ma quel sorriso di gioia gli fece tornare subito tutto in mente ed egli si fece pensieroso. 

Due voci infantili (Stepan Arkad'ic riconobbe le voci di Griša, il più piccolo, e di Tanja, la maggiore) si udirono dietro la porta. Avevano trascinato e lasciato cadere qualcosa. 

- Lo dicevo io che non si possono lasciar sedere i passeggeri sull'imperiale - gridava in inglese la bimba - ora, su, raccatta. 

``È tutto sottosopra - pensò Stepan Arkad'ic - ecco, i bambini scorrazzano da soli''. E fattosi sulla porta, li chiamò. Essi lasciarono la scatola che rappresentava il treno ed entrarono dal padre. 

La bimba, beniamina del padre, corse franca ad abbracciarlo e ridendo gli si appese al collo, rallegrandosi come sempre del noto profumo che si spandeva dalle sue fedine. Baciatolo infine sul volto arrossato per la posizione inclinata e raggiante di tenerezza, la bimba sciolse le braccia per scappar via, ma il padre la trattenne. 

- E la mamma? - chiese passando la mano sul collo liscio e morbido della figlia. - Buongiorno - disse poi sorridendo al piccolo che salutava. 

Aveva coscienza di amare meno il bambino e si sforzava di essere imparziale, ma il bambino lo sentiva e non sorrise al sorriso freddo del padre. 

- La mamma? S'è alzata - rispose la bimba. 

Stepan Arkad'ic sospirò. ``Già; non avrà dormito tutta la notte'' pensò. 

- Ma è di buon umore? 

La bambina sapeva che fra padre e madre c'era stata una certa questione e che la madre non poteva essere di buon umore; e il padre doveva saperlo, mentre ora fingeva, chiedendone con tanta disinvoltura. Arrossì per il padre. Egli capì subito e arrossì anche lui. 

- Non so - disse. - Non ha detto di studiare, ha detto di andare a spasso con miss Hull dalla nonna. 

- Su, va', Tancurocka mia. Ah, già, aspetta - disse trattenendola ancora e guardandole la manina morbida. 

Prese dal camino, là dove l'aveva messa il giorno prima, una scatola di dolci e gliene diede due, scegliendole i preferiti, uno di cioccolato e uno fondente. 

- A Griša? - disse la bambina indicando quello di cioccolato. 

- Sì, sì. - E accarezzando ancora una volta le piccole spalle, la baciò alla radice dei capelli e sul collo e la lasciò andare. 

- La carrozza è pronta - disse Matvej. - C'è poi una persona che chiede di voi - aggiunse. 

- È molto che è qui? - chiese Stepan Arkad'ic. 

- Da una mezz'ora. 

- Ma quante volte ti ho detto di annunziare subito! 

- Bisogna pur darvi il tempo di prendere almeno il caffè - disse Matvej con quel tono fra il confidenziale e lo screanzato che non dava la possibilità di arrabbiarsi. 

- Su, fa' passare subito - disse Oblonskij aggrottando le sopracciglia dalla stizza. 

La signora, moglie del capitano in seconda Kalinin, chiedeva una cosa assurda e sciocca; ma Stepan Arkad'ic, secondo la sua abitudine, la fece sedere, l'ascoltò con attenzione, senza interromperla, le consigliò dettagliatamente a chi e come dovesse rivolgersi, e le scrisse perfino alla svelta e bene, con la sua grossa, larga e bella scrittura chiara, un biglietto per la persona che avrebbe potuto aiutarla. Congedata la moglie del capitano in seconda, Stepan Arkad'ic prese il cappello e si fermò, cercando di ricordare se non avesse dimenticato qualcosa. Gli parve di non aver dimenticato nulla, fuorché quello che voleva dimenticare, la moglie. 

``Ah, sì''. Abbassò il capo e il suo bel viso prese un'aria afflitta. ``Andare o non andare?'' si diceva. E una voce interna gli diceva di non andare, che oltre a falsità non poteva esserci altro, che riparare, accomodare le loro relazioni non era più possibile, perché non era possibile rendere lei di nuovo attraente e capace di suscitare l'amore, e lui vecchio e incapace di amare. Dunque, oltre a falsità e menzogna, non ne poteva uscir fuori nulla, e la falsità e la menzogna erano avverse alla sua natura. 

``Eppure prima o poi bisogna farlo; non si può restar così'' disse, cercando di farsi coraggio. Raddrizzò il petto, tirò fuori una sigaretta, l'accese, ne aspirò due boccate, la gettò in un portacenere di madreperla a conchiglia, attraversò il salotto oscuro a passi svelti, e aprì l'altra porta che dava nella camera della moglie. 

\capitolo{IV}\label{iv} 

Dar'ja Aleksandrovna, in veste da notte, con le trecce ormai rade, un tempo folte e belle, appuntate alla nuca, col viso asciutto, affilato, e i grandi occhi spauriti che risaltavano nella magrezza del viso, stava in piedi in mezzo alle cose gettate alla rinfusa per la stanza, dinanzi a un armadio aperto dal quale sceglieva qualcosa. Udito il passo del marito, si fermò guardando la porta e cercando inutilmente di dare al viso un'espressione severa e sprezzante. Sentiva di aver paura di lui, paura dell'incontro imminente. Aveva tentato proprio allora di fare quello che aveva tentato già dieci volte in quei tre giorni: preparare la roba sua e dei bambini per trasportarla dalla madre, ma poi, di nuovo, non aveva saputo decidersi: eppure anche ora, come le altre volte, diceva a se stessa che così non poteva durare, che doveva fare qualcosa, punirlo, svergognarlo, vendicarsi almeno in minima parte del male che le aveva fatto. Si diceva ogni volta che lo avrebbe lasciato, ma sentiva che questo era impossibile; era impossibile perché non poteva disabituarsi a considerarlo suo marito e ad amarlo. Sentiva, inoltre, che se qui, in casa sua, riusciva appena ad aver cura dei suoi cinque bambini, la cosa sarebbe stata ancora più difficile là, dove sarebbe andata a stare con tutti loro. E proprio in quei tre giorni, il più piccolo si era ammalato perché gli avevano dato del brodo guasto, mentre il giorno innanzi gli altri erano quasi rimasti senza mangiare. Sentiva che non era possibile andar via; ma, ingannando se stessa, preparava la roba e si fingeva di partire. 

Visto il marito, tuffò la mano in un cassetto dell'armadio, come se cercasse qualcosa, e girò lo sguardo su di lui solo quando le fu proprio accanto. Ma il viso al quale aveva voluto dare un'espressione severa e decisa, esprimeva smarrimento e pena. 

- Dolly! - disse lui con voce timida e sommessa. Aveva ritirato la testa nelle spalle e voleva avere un'aria afflitta e contrita, ma suo malgrado, raggiava freschezza e salute. 

Con un'occhiata rapida dalla testa ai piedi ella notò la figura di lui raggiante freschezza e salute. ``Già, lui è felice e soddisfatto - pensò - e io? E anche questa bontà disgustosa, che lo fa amare e lodare da tutti, io la detesto questa sua bontà'' pensò. La bocca le si contrasse, il muscolo della guancia prese a tremare dalla parte destra del viso pallido e nervoso. 

- Che vi occorre? - disse con voce affrettata, sorda, non sua. 

- Dolly! - ripeté lui con un fremito nella voce. - Anna arriva oggi. 

- Ebbene, a me che importa? Io non posso riceverla! - gridò lei. 

- Eppure, Dolly\ldots{} 

- Andate via, andate via - gridò senza guardarlo, come se questo grido fosse provocato da un male fisico. 

Stepan Arkad'ic aveva potuto rimaner tranquillo quando aveva pensato a sua moglie, aveva potuto sperare che tutto si sarebbe ``appianato'', così come diceva Matvej, aveva potuto leggere tranquillamente il giornale e bere il caffè; ma quando vide il viso tormentato e dolente di lei, quando udì quel tono di voce rassegnato e affranto, il respiro gli si mozzò, qualcosa gli venne alla gola e gli occhi gli brillarono di lacrime. 

- Dio mio, che ho fatto! Dolly! Per amor di Dio\ldots{} Del resto\ldots{} - ma non poté continuare: un singhiozzo gli si era fermato in gola. Ella sbatté l'armadio e si voltò a guardarlo. - Dolly, cosa posso dire? Solo una cosa: perdona, perdona\ldots{} Ricorda\ldots{} nove anni di vita non possono forse far perdonare un minuto, un minuto\ldots{} 

Ella aveva abbassato gli occhi e ascoltava quello ch'egli stava per pronunciare, quasi supplicandolo di dire qualcosa che potesse dissuaderla. 

- Un minuto di esaltazione - riprese a dire lui, e voleva continuare, ma a questa parola, come per un male fisico, a lei si strinsero i denti e di nuovo il muscolo della guancia prese a tremare dalla parte destra del viso. 

- Andate via, andate via! - gridò con voce ancora più tagliente - e non mi venite a parlare delle vostre esaltazioni e delle vostre sconcezze! 

Voleva andar via, ma vacillò e si aggrappò alla spalliera della sedia per sorreggersi. Il viso di lui si dilatò, le labbra si gonfiarono, gli occhi si riempirono di lacrime. 

- Dolly! - pronunziò ormai singhiozzando. - In nome di Dio, pensa ai bambini, loro non sono colpevoli. Sono io il colpevole, e tu puniscimi, ordinami di scontare la mia pena. In quello che posso, sono pronto a tutto! Sono colpevole, non ci sono parole, come sono colpevole! Ma, Dolly, perdona! 

Ella si mise a sedere. Egli sentiva il respiro grave di lei e gliene veniva una pena indicibile. Più volte ella si provò a parlare, ma non poté. Egli aspettava. 

- Tu ti ricordi dei bambini per giocare con loro, mentre io sì che me ne ricordo, e lo so oramai che sono rovinati - disse lei, usando evidentemente una delle frasi che in quei tre giorni s'era ripetuta più d'una volta. 

Gli aveva parlato col ``tu'', ed egli la guardò riconoscente, e si mosse per prenderle una mano, ma lei si scostò con avversione. 

- Io mi ricordo dei bambini e farei di tutto al mondo per salvarli, ma non so io stessa come salvarli: se sottrarli al padre o abbandonarli a un padre depravato. Sì, depravato\ldots{} Eh sì, ditemi voi, dopo quello\ldots{} che c'è stato, è forse possibile vivere insieme? È possibile forse? Dite voi, è possibile? - ripeté alzando la voce. 

- Dopo che mio marito, il padre dei miei figli ha una relazione con la governante dei suoi bambini\ldots{} 

- Ma che fare, che fare? - diceva lui con voce pietosa, non sapendo egli stesso che dire e abbassando sempre più il capo. 

- Mi fate ribrezzo, disgusto! - gridò lei, riscaldandosi ancora di più. - Le vostre lacrime cosa sono? acqua! Non mi avete mai amata, non avete cuore, non siete generoso! Siete vile, abietto, mi siete estraneo, sì, del tutto estraneo - e pronunziò la parola ``estraneo'', per lei terribile, con pena e rancore. 

Egli la guardò e l'odio che appariva sul viso di lei lo sgomentò e sorprese. Non capiva che quella sua pietà verso di lei la irritava, perché vedeva in lui la compassione, ma non l'amore. ``Mi odia - pensò. - Non perdonerà''. 

- È terribile, è terribile - disse. 

Nel frattempo, nella stanza accanto, probabilmente perché caduto, un bimbo si mise a gridare: Dar'ja Aleksandrovna tese l'orecchio, e il viso d'un tratto le si raddolcì. 

Parve rientrare in sé per qualche istante e, come se non sapesse dov'era e cosa stesse facendo, si alzò in fretta e si avviò alla porta. 

``Ma allora vuol sempre bene al mio bambino - pensò lui, avendo notato il mutar del viso al grido del piccolo - al mio bambino; e come può odiare tanto me?''. 

- Dolly, ancora una parola - disse seguendola. 

- Se mi seguite, chiamerò gente, i bambini! Che tutti sappiano che siete un mascalzone! Me ne vado oggi stesso e voi restate pure qua a vivere con la vostra amante! 

E uscì, sbattendo la porta. 

Stepan Arkad'ic sospirò, si asciugò il viso e a passi lenti si avviò per uscire. ``Matvej dice che si appianerà; ma come? Io non ne vedo neppure la possibilità. Ahi, ahi, che orrore! E come gridava, e in che modo triviale! - diceva a se stesso ricordando le grida e le parole `mascalzone' e `amante'. - E forse le ragazze hanno sentito! Terribilmente triviale, terribilmente''. Stepan Arkad'ic si fermò per qualche istante, si asciugò gli occhi, sospirò e, raddrizzato il busto, uscì dalla camera. 

Era venerdì, e nella sala da pranzo l'orologiaio tedesco dava corda all'orologio. Stepan Arkad'ic si ricordò della sua battuta di spirito su quell'orologiaio calvo e preciso: ``Il tedesco è stato caricato per tutta la vita per caricare orologi'' e sorrise. A Stepan Arkad'ic piaceva una bella battuta. ``Ma forse davvero tutto `si appianerà'! Bella frase: `si appianerà' - pensò. - Bisogna farla circolare''. 

- Matvej! - chiamò. - Prepara tutto con Mar'ja per Anna Arkad'evna, di là nel salotto - disse a Matvej che era apparso. 

- Sissignore. 

Stepan Arkad'ic infilò la pelliccia e uscì fuori. 

- Non pranzerete a casa? - chiese Matvej, accompagnandolo. 

- Non so, come capiterà. Ecco, prendi per la spesa - disse dandogli dieci rubli dal portafoglio. - Basta? 

- Basti o non basti, ci si deve rigirare - rispose Matvej, sbattendo lo sportello e indietreggiando verso l'ingresso. 

Dar'ja Aleksandrovna intanto, acquietato il bambino e capito, dal rumore della carrozza, ch'egli se n'era andato, tornò di nuovo in camera. Era l'unico suo rifugio dalle cure familiari che la opprimevano non appena ne usciva fuori. E anche ora, in quei pochi momenti che aveva passato nella camera dei bambini, la governante inglese e Matrëna Filimonovna si erano affrettate a farle alcune domande che non ammettevano indugio e alle quali solo lei poteva rispondere: cosa mettere indosso ai bambini per andare a spasso, dare o no il latte, mandare a chiamare oppure no un altro cuoco. 

- Ah, lasciatemi, lasciatemi! - aveva detto e, tornata in camera, si era seduta di nuovo là dove aveva parlato col marito, stringendo le mani smagrite con gli anelli che scivolavano dalle dita ossute, e aveva cominciato a ripensare a tutto il colloquio avvenuto. ``È andato via. Ma l'ha finita poi con quella? Possibile che la veda ancora? Perché non gliel'ho chiesto? No, no, non ci si può riunire. E anche se dovessimo restare nella stessa casa, saremmo estranei. Per sempre estranei! - ripeté di nuovo, e con particolare significato, questa parola per lei terribile. - E come l'ho amato, Dio mio, come l'ho amato! E ora, non l'amo forse? Non l'amo forse più di prima? È terribile, soprattutto il fatto che\ldots{}'' cominciò, ma non finì il pensiero, che già Matrëna Filimonovna si era affacciata alla porta. 

- Su via, mandate a chiamare mio fratello - disse - almeno preparerà il pranzo; se no, come ieri, fino alle sei i bambini non avran mangiato. 

- Va bene, vengo, vengo a dare gli ordini. Non hanno mandato a prendere il latte fresco? 

E Dar'ja Aleksandrovna s'ingolfò nelle cure del giorno, e per un po' sommerse in esse la sua pena. 

\capitolo{V}\label{v} 

Stepan Arkad'ic a scuola aveva studiato bene, grazie alle sue buone capacità, ma, pigro e svagato, aveva finito gli studi tra gli ultimi. Tuttavia, pur conducendo una vita sempre scapestrata, in età ancor giovane, con un titolo modesto, aveva ottenuto il posto ragguardevole e ben retribuito di capo di uno degli uffici amministrativi di Mosca. Aveva avuto questo posto per mezzo del marito di Anna, Aleksej Aleksandrovic Karenin, il quale occupava uno dei più alti gradi nel ministero a cui apparteneva l'ufficio; ma se Karenin non avesse designato suo cognato a quel posto, Stiva Oblonskij, per mezzo di un centinaio di alti personaggi, fratelli, sorelle, prozii, zii, zie, avrebbe avuto quel posto o altro equivalente con quei seimila rubli di stipendio che gli erano necessari, perché i suoi affari, malgrado la considerevole proprietà della moglie, andavano male. 

Una buona metà della società di Mosca e Pietroburgo era in relazioni di parentela o di amicizia con Stepan Arkad'ic. Egli era nato nella cerchia di coloro che erano o erano in seguito diventati i potenti di quel mondo. Un terzo dei funzionari di stato, i vecchi, erano amici di suo padre e lo avevano visto nascere; un altro terzo gli davano del ``tu'' e un terzo ancora erano suoi buoni conoscenti. Pertanto, i dispensatori di beni terreni sotto forma di posti, appalti, concessioni e cose simili, erano tutti amici suoi e non avrebbero mai lasciato fuori uno dei loro. Così Oblonskij non aveva dovuto brigare per ottenere un posto vantaggioso; gli era bastato non rifiutare, non avere invidie, non leticare, non offendersi, cose tutte ch'egli neppure faceva per quella bonarietà che gli era propria. Gli sarebbe parso ridicolo se gli avessero detto che non avrebbe ottenuto un posto retribuito con lo stipendio che gli era necessario, dal momento che non pretendeva niente di eccezionale, ma voleva solo quello che avevano gli altri suoi coetanei quando, non peggio di chiunque altro, egli era in grado di adempiere una funzione di tal genere. 

A Stepan Arkad'ic volevano bene tutti quelli che lo conoscevano non solo per quel suo carattere buono e gioviale e per la sua indubbia onestà, ma perché in quel suo bell'aspetto luminoso, negli occhi splendenti, nelle sopracciglia e nei capelli neri, nel colorito bianco e rosso del viso vi era qualcosa che agiva in modo cordiale e festoso sul fisico delle persone che lo incontravano. ``Oh, Stiva! Oblonskij! Eccolo!'' dicevano quasi sempre con un sorriso di gioia, incontrandolo. E anche se talvolta ci si rendeva conto che, dopo una conversazione con lui, non succedeva nulla di particolarmente gioioso, l'indomani, due giorni dopo, tutti di nuovo si rallegravano nell'incontrarlo, proprio allo stesso modo. 

Occupando già da tre anni il posto di capo di uno degli uffici amministrativi di Mosca, Stepan Arkad'ic aveva conquistato, oltre la simpatia, la stima dei colleghi, dei dipendenti, dei superiori, e di tutti coloro che avevano a che fare con lui. Le principali qualità che gli procuravano la stima generale in ufficio consistevano, in primo luogo, in una straordinaria indulgenza verso gli altri, basata sulla coscienza dei propri difetti; in secondo luogo, in un grande liberalismo, non quello di cui leggeva nei giornali, ma quello ch'egli aveva nel sangue e che gli faceva trattare perfettamente allo stesso modo tutte le persone, di qualunque classe o condizione fossero; e in terzo luogo, e questa era la cosa più importante, in un'assoluta indifferenza verso gli affari che trattava, per cui non se ne appassionava mai e non commetteva errori. 

Arrivato in ufficio, Stepan Arkad'ic, accompagnato da un usciere ossequioso che gli portava la cartella, passò nel suo gabinetto particolare, indossò la divisa ed entrò in aula. Gli scrivani e gli impiegati si alzarono tutti, salutandolo con rispetto e giovialità. Stepan Arkad'ic, in fretta come sempre, andò al proprio posto, strinse la mano ai colleghi e sedette. Scherzò e discorse proprio quel tanto che era conveniente, e cominciò il lavoro. Nessuno più di Stepan Arkad'ic sapeva con maggiore precisione il limite tra la cordialità confidenziale e il tono ufficiale, così necessario al piacevole disbrigo degli affari. Il segretario, con giovialità e rispetto, come del resto tutti nell'ufficio di Stepan Arkad'ic, gli si accostò con alcune carte e riferì con quel tono familiarmente libero che era stato introdotto da Stepan Arkad'ic. 

- Siamo riusciti così ad avere notizie dell'amministrazione provinciale di Penza. Ecco, non vi piacerebbe\ldots{} 

- Le avete avute finalmente - prese a dire Stepan Arkad'ic, fermando col dito la carta. - Allora, signori\ldots{} - E la seduta cominciò. 

``Se sapessero - pensava chinando la testa con aria d'importanza nell'ascoltare il rapporto - che ragazzo colpevole era mezz'ora fa il loro capo!''. E gli occhi gli ridevano alla lettura del rapporto. La seduta doveva durare fino alle due, senza interruzione; alle due, intervallo e colazione. 

Non erano ancora le due quando la grande porta a vetri dell'aula si aprì improvvisamente e qualcuno entrò. Tutti i membri ritratti sotto il ritratto dell'imperatore e al di là dello specchio a tre facce, lieti della distrazione, si voltarono a guardare verso la porta; ma l'usciere che stava all'ingresso respinse subito colui che s'era infilato e richiuse la porta a vetri. 

Quando tutto il rapporto fu letto, Stepan Arkad'ic si alzò stiracchiandosi e, pagando il proprio tributo al liberalismo dell'epoca, tirò fuori, ancora nell'aula, una sigaretta, e si avviò nel suo ufficio. Due colleghi, il vecchio funzionario Nikitin e il gentiluomo di camera Grinevic, uscirono con lui. 

- Dopo colazione arriveremo a finire - disse Stepan Arkad'ic. 

- Altro che arriveremo! - disse Nikitin. 

- Ma deve essere un furbo matricolato quel Fomin - disse Grinevic accennando a un personaggio implicato nell'affare di cui si discuteva. 

Alle parole di Grinevic Stepan Arkad'ic si accigliò, facendo capire con questo che non era corretto dare un giudizio prima del tempo, e non rispose nulla. 

- Chi è entrato? - chiese all'usciere. 

- Un tale, eccellenza, senza chiedere permesso, s'è fissato dentro appena mi sono girato. Domandava di voi. Io dico: quando usciranno i membri, allora\ldots{} 

- Dov'è? 

- È forse uscito nell'ingresso, non faceva che camminare. Eccolo - disse l'usciere, indicando un uomo di costituzione forte, largo di spalle, con la barba ricciuta, il quale, senza togliersi il berretto di montone, saliva lesto e leggero i gradini consumati della scala di pietra. Uno di quelli che scendevano, un impiegato magrolino con una cartella sotto il braccio, fermatosi, guardò con riprovazione le gambe di colui che correva e fissò interrogativamente Oblonskij. 

Stepan Arkad'ic era dritto in cima alla scala. Il suo viso bonario, che splendeva emergendo dal bavero ricamato dell'uniforme, s'illuminò ancor più quando riconobbe chi correva. 

- Ma è proprio lui! Levin, finalmente! - esclamò con un sorriso cordialmente canzonatorio, guardando Levin che gli si avvicinava. - Com'è che non hai disdegnato di venirmi a pescare in quest'antro? - disse Stepan Arkad'ic baciando l'amico, non contento di una stretta di mano. - Sei qui da un pezzo? 

- Sono arrivato or ora, e avevo una gran voglia di vederti - rispose Levin, guardandosi attorno timido e, nello stesso tempo, inquieto e contrariato. 

- Su, andiamo nel mio gabinetto - disse Stepan Arkad'ic, conoscendo la timidezza ombrosa e scontrosa dell'amico; e, presolo per un braccio, lo trascinò dietro di sé come per guidarlo in mezzo ai pericoli. 

Stepan Arkad'ic si dava del ``tu'' con quasi tutti i suoi conoscenti: coi vecchi di sessant'anni, coi ragazzi di venti; con gli attori, coi ministri, coi negozianti e con gli aiutanti generali; così che molti di quelli che gli davano del ``tu'' si trovavano ai due punti estremi della scala sociale, e molti si sarebbero stupiti nel constatare di avere qualcosa di comune per mezzo di Oblonskij. Egli dava del ``tu'' a tutti quelli con i quali beveva lo champagne, e di champagne ne beveva con tutti; perciò, incontrandosi in presenza dei suoi dipendenti con i suoi ``tu'' vergognosi, come chiamava scherzando molti amici, sapeva diminuire, con quel tatto che gli era proprio, la spiacevolezza dell'impressione che potevano riportarne i dipendenti. Levin non era un ``tu'' vergognoso, ma Oblonskij intuì che Levin pensava ch'egli potesse non desiderare di mostrare la propria intimità con lui dinanzi ai propri dipendenti, e perciò si affrettò a condurlo nel proprio gabinetto. 

Levin era quasi della stessa età di Oblonskij e si davano del ``tu'' non solo per lo champagne. Levin gli era compagno e amico di prima giovinezza. Si volevano bene, malgrado la diversità dei caratteri e dei gusti, così come si vogliono bene gli amici incontratisi nella prima giovinezza. Malgrado ciò, come capita spesso fra persone che hanno scelto generi diversi di attività, ciascuno di loro, pur giustificando col ragionamento l'attività dell'altro, finiva col disprezzarla dentro di sé. A ciascuno sembrava che la vita che egli stesso conduceva fosse la vera vita, mentre l'altra, quella che conduceva l'amico, non ne fosse che la parvenza. Oblonskij non poteva trattenere un lieve riso canzonatorio alla vista di Levin. L'aveva visto già varie volte arrivare a Mosca dalla campagna dove faceva qualcosa; che cosa facesse precisamente, Stepan Arkad'ic non aveva mai potuto capir bene e non se ne curava. Levin veniva a Mosca sempre agitato, frettoloso, un po' timido e urtato da questa timidezza, e quasi sempre con delle vedute nuove e inaspettate su tutte le cose. Stepan Arkad'ic ne rideva e se ne compiaceva. Nello stesso preciso modo Levin disprezzava dentro di sé il modo di vivere cittadino dell'amico e quel suo impiego che considerava sciocco e vuoto, e ci rideva su. Ma la differenza consisteva in questo: Oblonskij, facendo quello che fanno tutti, rideva con sicurezza e bonarietà, Levin, invece, senza sicurezza e, a volte, con dispetto. 

- Ti aspettavamo da tempo - disse Stepan Arkad'ic entrando nello studio e lasciando il braccio di Levin come a dire che là non c'erano più pericoli. - Sono molto contento di rivederti. Ebbene, come va? Quando sei arrivato? 

Levin taceva, sbirciando le due facce dei colleghi di Oblonskij che non conosceva, e in particolar modo dell'elegante Grinevic dalle dita affilate e bianche, e dalle unghie così lunghe, gialle e ricurve in punta, e dai gemelli della camicia così grossi e luccicanti che queste mani, evidentemente, avevano assorbito tutta la sua attenzione e non gli davano libertà di pensiero. Oblonskij lo notò subito, e sorrise. 

- Ah, già, permettete che vi presenti - disse. - I miei colleghi Filipp Ivanovic Nikitin e Michail Stanislavic Grinevic - e, rivolto verso Levin: - Il fautore del consiglio distrettuale, l'uomo nuovo del consiglio, il ginnasta che solleva con una mano sola cinque pudy, l'allevatore di bestiame, il cacciatore, nonché amico mio, Konstantin Levin, fratello di Sergej Ivanyc Koznyšev. 

- Molto piacere - disse il vecchietto. 

- Ho l'onore di conoscere vostro fratello Sergej Ivanyc - disse Grinevic porgendo la mano affilata dalle unghie lunghe. 

Levin si accigliò, strinse la mano e si rivolse subito a Oblonskij. Pur avendo una grande considerazione per il fratellastro, scrittore noto in tutta la Russia, non poteva sopportare che ci si rivolgesse a lui, non come Konstantin Levin ma come al fratello del famoso Koznyšev. 

- No, non sono più consigliere distrettuale. Ho litigato con tutti, e non vado più alle riunioni - disse a Oblonskij. 

- Hai fatto presto, però!- disse Oblonskij con un sorriso. - Ma come, perché? 

- È una storia lunga. Una volta o l'altra te la racconterò - disse Levin prendendo però subito a raccontarla. - Ecco, per dirla in breve, mi sono convinto che non c'è e non può esserci alcuna attività distrettuale; - cominciò come se qualcuno l'avesse offeso allora allora: - da una parte è un giuoco; si giuoca al parlamento, ed io non sono abbastanza giovane, né abbastanza vecchio per divertirmi coi balocchi; dall'altra - e qui balbettò - è un mezzo per guadagnare denaro per la coterie del distretto. Prima c'erano le tutele, i tribunali, ora invece c'è il consiglio distrettuale; non è una forma di subordinazione, ma una forma di stipendio non meritato - disse con tanto calore come se qualcuno dei presenti avversasse la sua opinione. 

- Eh! Ma tu, a quanto vedo, sei ancora in una nuova fase, in quella conservatrice - disse Stepan Arkad'ic. - Ma, del resto, di questo parleremo dopo. 

- Sì, dopo. Ma io avevo bisogno di vederti - disse Levin, fissando con antipatia la mano di Grinevic. 

Stepan Arkad'ic sorrise appena percettibilmente. 

- Be', dicevi che mai più ti saresti messo un vestito all'europea? - disse guardandogli il vestito nuovo, fatto evidentemente da un sarto francese. - Eh, già, vedo, siamo in una fase nuova. 

Levin arrossì improvvisamente, ma non come arrossiscono le persone adulte, leggermente, senza avvertirlo, ma come arrossiscono i ragazzi quando sentono d'essere ridicoli con la loro timidezza e, vergognandosene, arrossiscono ancora di più fin quasi alle lacrime. Ed era così strano vedere quel viso intelligente, maschio diventare così infantile, che Oblonskij smise di guardarlo. 

- E allora, dove ci vediamo? Ho molto bisogno di parlarti - disse Levin. 

Oblonskij si mise a riflettere. 

- Ecco, andiamo a far colazione da Gurin e parleremo là. Fino alle tre son libero. 

- No - rispose Levin dopo aver pensato un po'; - devo ancora andare in giro. 

- Su via, andiamo a pranzare insieme. 

- Pranzare? Ma io non ho bisogno di niente di straordinario, solo due parole; devo farti una domanda, e a chiacchierare ci penseremo poi. 

- E allora, dille subito queste due parole, così a pranzo chiacchiereremo. 

- Eccole, le due parole; - disse Levin - del resto, niente di straordinario. 

E la sua faccia prese a un tratto un'espressione cattiva, dovuta allo sforzo fatto per vincere la propria timidezza. 

- Che fanno gli Šcerbackij? Tutto come prima? - disse. 

- Tu hai detto due parole, ma io non posso rispondere con due parole, perché\ldots{} Scusami un momento\ldots{} 

Era entrato il segretario che, con la deferenza familiare e la modesta consapevolezza, comune a tutti i segretari, della propria superiorità, rispetto al capo, nella conoscenza degli affari, si era avvicinato con le carte a Oblonskij e, con aria interrogativa, aveva cominciato a esporre una certa difficoltà. Stepan Arkad'ic, senza finir di ascoltare, pose affabilmente una mano sulla manica del segretario. 

- No, fate come già vi ho detto - disse addolcendo con un sorriso l'osservazione e, spiegato in breve come intendeva l'affare, allontanò le carte e disse: - Fate così, vi prego, così, Zachar Nikitic. 

Il segretario, confuso, si allontanò. Levin, che durante il colloquio con il segretario aveva avuto modo di rimettersi completamente, stava in piedi, poggiando tutte e due le mani ad una sedia, e sul suo viso vi era un'attenzione ilare. 

- Non capisco, non capisco - diceva. 

- Cosa non capisci? - chiese Oblonskij sorridendo anche lui allegramente e tirando fuori una sigaretta. Si aspettava da Levin qualche uscita strana. 

- Non capisco quello che fate - disse Levin alzando le spalle. - Come puoi prendere tutto questo sul serio? 

- Perché? 

- Ma perché qui non avete nulla da fare. 

- Tu credi così, ma noi siamo sovraccarichi di lavoro. 

- Scartoffie! Già, ma tu ci sei tagliato per questo - aggiunse Levin. 

- Allora tu credi che io manchi di qualcosa? 

- Forse sì - disse Levin. - Tuttavia ammiro la tua importanza e sono orgoglioso di avere un così grand'uomo per amico. Ma tu non hai risposto alla mia domanda - aggiunse guardando Oblonskij con uno sforzo disperato, diritto negli occhi. 

- E va bene, e va bene. Aspetta un po' e arriverai a questo anche tu. Finché hai tremila desjatiny nel distretto di Karazin e questi muscoli e la freschezza di una ragazzina di dodici anni, va tutto bene, ma poi ci arriverai anche tu. Già, ecco, a proposito di quello che mi chiedevi; nessun cambiamento, ma peccato che tu sia stato lontano tanto tempo. 

- Perché? che c'è? - chiese Levin spaventato. 

- No, nulla - rispose Oblonskij. - Ne riparleremo. Ma tu per quale particolare motivo sei venuto? 

- Ah, di questo parleremo poi - disse Levin, arrossendo di nuovo fino alle orecchie. 

- Su, va bene, ho capito - disse Stepan Arkad'ic. - Vedi: ti avrei invitato a casa, ma mia moglie non sta bene. Ecco, però; se le vuoi vedere, oggi sono certamente al giardino zoologico, dalle quattro alle cinque. Kitty va a pattinare. Tu va' là; io passerò, e andremo a pranzare insieme in qualche posto. 

- Benissimo, arrivederci, allora. 

- Guarda, io ti conosco; tu sei capace di scordartene o di partirtene subito per la campagna! - rise Stepan Arkad'ic. 

- No, certamente. 

E dopo essersi ricordato di non aver salutato i colleghi di Oblonskij soltanto quand'era già sulla porta, Levin uscì dall'ufficio. 

- Deve essere un proprietario pieno di energia - disse Grinevic, quando Levin fu uscito. 

- Sì, amico mio - disse Stepan Arkad'ic annuendo col capo - ecco un uomo felice! Tremila desjatiny nel distretto di Karazin, tutto davanti a sé e quanta vitalità! Non così noi! 

- Perché vi lamentate, Stepan Arkad'ic? 

- Va male, proprio male - disse Stepan Arkad'ic sospirando pesantemente. 

\capitolo{VI}\label{vi} 

Quando Oblonskij aveva chiesto a Levin per quale motivo particolare fosse venuto, Levin s'era fatto rosso e s'era irritato con se stesso d'essersi fatto rosso, perché non gli aveva saputo rispondere: ``Son venuto a chiedere la mano di tua cognata'' pur essendo venuto proprio per questo. 

Le famiglie dei Levin e degli Šcerbackij erano vecchie casate di nobili moscoviti ed erano sempre state fra loro in rapporti di intima amicizia. Questi rapporti si erano fatti più stretti durante lo studentato di Levin. Levin si era presentato ed era entrato all'università insieme al giovane principe Šcerbackij, fratello di Dolly e di Kitty. In quel tempo Levin andava spesso in casa Šcerbackij ed era innamorato di casa Scerbackij. Per quanto ciò possa sembrare strano, Konstantin Levin era proprio innamorato della casa, della famiglia, in particolar modo della parte femminile degli Šcerbackij. Levin non ricordava sua madre, e l'unica sua sorella era più grande di lui, di modo che per la prima volta in casa Šcerbackij aveva conosciuto quell'ambiente di vecchia famiglia nobile, colta e onesta, del quale era stato privato per la morte della madre e del padre. Tutti i membri di questa famiglia, ed in particolare la parte femminile, gli apparivano avvolti in un certo misterioso velo di poesia; ed egli non solo non vedeva in loro alcun difetto, ma sotto questo poetico velo che li avvolgeva, immaginava i sentimenti più elevati e ogni possibile perfezione. Per qual motivo le tre signorine dovessero parlare un giorno in francese e un giorno in inglese; per qual motivo, in determinate ore, sonassero alternativamente il pianoforte i cui suoni giungevano su in camera del fratello dove gli amici studiavano; perché venissero insegnanti di letteratura francese, di musica, di disegno e di ballo; per qual motivo, a una data ora, tutte e tre le signorine con m.lle Linon giungessero in carrozza al boulevard Tverskoj avvolte nelle pelliccette rasate: Dolly in una lunga, Natalie in una meno lunga e Kitty in una del tutto corta, così che apparissero le sue gambette ben fatte nelle calze rosse attillate; per qual motivo dovessero passeggiare sul boulevard Tverskoj, accompagnate da un servitore con la coccarda dorata sul cappello; tutto questo e molto altro ancora di quel che si faceva nel loro mondo misterioso, egli non riusciva a capire; sapeva però che tutto quello che si faceva là era bello, ed era innamorato della misteriosità di quello che vi si compiva. 

Durante il suo studentato, era stato lì lì per innamorarsi della maggiore, Dolly; ma ben presto l'avevano data in sposa a Oblonskij. Aveva preso ad innamorarsi della seconda. Sentiva che avrebbe dovuto innamorarsi di una delle sorelle, ma non sapeva di quale precisamente. Ma anche Natalie, appena apparsa in società, andò sposa al diplomatico L'vov. Kitty era ancora ragazzina quando Levin finì l'università. Il giovane Šcerbackij, entrato in marina, morì nel mar Baltico e i rapporti di Levin con gli Šcerbackij, malgrado la sua amicizia con Oblonskij, divennero più radi. Ma quando, al principio dell'inverno, Levin giunse a Mosca dopo un anno di campagna e rivide gli Šcerbackij, capì di quale delle tre sorelle la sorte aveva destinato che egli si innamorasse. Nulla di più semplice doveva sembrare che lui, giovane di buona famiglia, benestante, trentaduenne, chiedesse la mano della principessina Šcerbackaja; con tutta probabilità sarebbe stato subito giudicato un buon partito. Ma Levin era innamorato, e gli sembrava che Kitty fosse, sotto ogni aspetto, una tale perfezione, un essere così superiore ad ogni altro sulla terra, e lui invece così umile e basso, da non poter neppure formulare il pensiero che gli altri ed ella stessa lo giudicassero degno di lei. 

Dopo aver passato due mesi a Mosca, come avvolto in una nebbia, vedendo Kitty ogni giorno in società dove aveva preso ad andare per incontrarla, Levin aveva improvvisamente deciso che la cosa non era possibile, ed era ripartito per la campagna. 

La convinzione di Levin che la cosa non andasse si basava sull'idea che agli occhi dei familiari egli dovesse sembrare un partito poco convincente, non degno della deliziosa Kitty, e che la stessa Kitty non potesse amarlo. Agli occhi dei parenti egli non aveva nessuna attività stabile e definita e nessuna posizione in società; a trentadue anni, alla sua stessa età, i suoi coetanei erano già chi colonnello e aiutante di campo, chi professore di università, chi direttore di banca o delle ferrovie, chi capufficio come Oblonskij; e lui invece (lo sapeva bene come appariva agli altri) era un proprietario di terre, che si occupava dell'allevamento delle vacche, del tiro alle beccacce e di costruzioni; era cioè un giovane senza talento, dal quale non era uscito fuori nulla, e che faceva, secondo il giudizio della gente di mondo, proprio quello che fanno gli uomini che non sono buoni a nulla. 

La stessa misteriosa e deliziosa Kitty non poteva amare un uomo così brutto, come egli stesso si considerava, e, quel ch'era peggio, così semplice, che non brillava in nulla. Oltre a ciò i suoi primi rapporti con Kitty, rapporti di un giovane verso una bambina sorti per l'amicizia col fratello, gli sembravano un altro ostacolo all'amore. A un brav'uomo brutto, come si considerava lui, si poteva voler bene come a un amico, ma per innamorarsene, com'era innamorato lui di Kitty, avrebbe dovuto essere un bell'uomo, e soprattutto un uomo interessante. 

Aveva sentito dire che spesso le donne amano uomini brutti e rudi; ma non ci credeva, perché giudicava da se stesso, che non avrebbe potuto amare se non donne belle, affascinanti, eccezionali. 

Ma, trascorsi due mesi in campagna, in solitudine, si era convinto che questo non era uno di quegli innamoramenti che aveva provato nella prima giovinezza; che questo sentimento non gli dava un attimo di tregua, che non poteva vivere senza risolvere la questione se ella sarebbe stata o no sua moglie; che la sua disperazione derivava solo dalla sua fantasia e che non aveva prova alcuna per credere di dover essere respinto. E adesso era arrivato a Mosca con la ferma decisione di chiedere la mano di Kitty e di sposarsi, se fosse stato accolto. Se no\ldots{} se l'avessero respinto, non sapeva neppure immaginare cosa sarebbe successo di lui. 

\capitolo{VII}\label{vii} 

Giunto a Mosca col treno della mattina, Levin si era fermato dal fratellastro maggiore Koznyšev, e, mutato d'abito, era entrato nello studio col proposito di dirgli subito per quale motivo era venuto a chiedergli consiglio; ma il fratello non era solo. C'era da lui un noto professore di filosofia che era venuto da Char'kov proprio per chiarire una divergenza sorta fra di loro a proposito di una questione importante. Il professore conduceva un'accesa polemica contro i materialisti e Sergej Koznyšev seguiva con interesse tale polemica e, dopo aver letto l'ultimo articolo del professore, gli aveva scritto in una lettera le proprie obiezioni, rimproverandogli le troppo larghe concessioni fatte ai materialisti. E il professore era venuto subito per discutere la cosa. Il discorso era avviato sulla questione di moda; esiste o no un limite fra i fenomeni psichici e quelli fisiologici, e dove esso si trova? 

Sergej Ivanovic andò incontro al fratello con l'usuale sorriso cortesemente freddo che aveva per tutti e, presentandolo al professore, continuò il discorso. 

L'ometto giallognolo, occhialuto, dalla fronte bassa, si distolse un attimo dalla conversazione per salutare, e riprese il discorso senza fare attenzione a Levin. Levin sedette in attesa che il professore se ne andasse, quando improvvisamente prese interesse all'argomento. 

Levin si era spesso imbattuto negli articoli di cui si parlava e li aveva letti in riviste, per completare le sue cognizioni di laureato in scienze naturali; ma non aveva mai collegato quelle deduzioni scientifiche sull'origine zoologica dell'uomo, sui riflessi, sulla biologia o sulla sociologia, ai problemi sul significato della vita e della morte che negli ultimi tempi pur gli venivano in mente sempre e sempre più spesso. 

Nell'ascoltare la conversazione del fratello col professore, notava che essi collegavano le questioni scientifiche a quelle dello spirito; alcune volte si avvicinavano a quest'ultime, ma ogni volta che si avvicinavano al punto che a lui sembrava essenziale, se ne ritraevano immediatamente e si ingolfavano nel campo delle disquisizioni sottili, delle riserve, delle citazioni, delle allusioni, dei rinvii a nomi autorevoli, ed egli stentava a capire di che cosa parlassero. 

- Io non posso ammettere - diceva Sergej Ivanovic con la sua abituale chiarezza e precisione di pensiero ed eleganza di eloquio, - io non posso in nessun modo essere d'accordo col Keiss nell'ammettere che tutta la mia visione del mondo esteriore derivi dalle sensazioni. Il concetto fondamentale dell'essere non ci viene dalla sensazione, giacché non abbiamo neanche un organo speciale che ci trasmetta questo concetto. 

- Sì, ma loro, Wurst e Knaust e Pripasov, vi risponderanno che il vostro concetto dell'essere deriva dall'insieme di tutte le sensazioni, che questo concetto dell'essere è il risultato delle sensazioni. Wurst dice addirittura che non appena viene a mancare la sensazione cessa anche la nozione dell'essere. 

- Io dico al contrario\ldots{} - cominciò Sergej Ivanovic. 

Ma a questo punto parve di nuovo a Levin che essi, avvicinatisi al punto essenziale, se ne ritraessero e decise di rivolgere una domanda al professore. 

- Allora, dunque, se i miei sensi sono annientati, se il mio corpo muore, non vi è più esistenza alcuna? - chiese. 

Il professore, contrariato, e come colto, per l'interruzione, da un dolore intellettuale, si voltò verso lo strano interlocutore che aveva più l'aria di un facchino che di un filosofo, e portò gli occhi su Sergej Ivanovic come a dirgli: ``Che cosa rispondere qui?''. Ma Sergej Ivanovic, che era lontano dal parlare con lo sforzo e la unilateralità con cui parlava il professore, e che aveva nella mente abbastanza spazio per rispondere al professore e per cogliere nello stesso tempo il semplice spontaneo punto di vista con cui era stata formulata la domanda, sorrise e disse: 

- Non abbiamo ancora il diritto di risolvere una questione simile. 

- Non abbiamo dati - asserì il professore e continuò le sue argomentazioni. 

- No - diceva - io fo notare che se, come dice precisamente il Pripasov, la percezione ha come base la sensazione, noi dobbiamo allora distinguere rigorosamente questi due concetti. 

Levin non ascoltava già più e aspettava solo che il professore se ne andasse. 

\capitolo{VIII}\label{viii} 

Quando il professore se ne fu andato, Sergej Ivanovic si rivolse al fratello: 

- Sono molto contento che tu sia venuto. Per molto? Come vanno le nostre cose? 

Levin sapeva che le cose di casa interessavano molto poco il fratello maggiore e che solo per compiacenza gliene chiedeva; rispose perciò soltanto circa la vendita del frumento e il ricavato. 

Avrebbe voluto dire al fratello della sua intenzione di sposarsi e chiedergli consiglio, ed era fermamente deciso a questo; ma dopo aver visto il fratello, dopo aver ascoltato la conversazione con il professore, e udito quel tono involontario di protezione col quale il fratello gli chiedeva delle faccende amministrative (il fondo materno era indiviso e Levin si occupava di entrambe le parti), Levin sentì che c'era qualcosa che gli impediva di parlare al fratello della sua decisione di sposarsi. Sentiva che il fratello non avrebbe visto la cosa così come egli avrebbe voluto. 

- Ebbene, come va da voi il consiglio distrettuale? - domandò Sergej Ivanovic che si interessava molto dell'istituzione del consiglio cui attribuiva grande importanza. 

- Ma, davvero, non so\ldots{} 

- Ma come? Ma tu non sei membro del consiglio distrettuale? 

- No, non lo sono più; me ne sono uscito e non vado più alle riunioni. 

- Peccato! - esclamò Sergej Ivanovic, accigliandosi. 

Levin prese a dire a sua discolpa quello che si faceva nelle riunioni del distretto. 

- Ecco, sempre così - lo interruppe Sergej Ivanovic. - Noi russi siamo fatti così. Forse è anche una nostra buona qualità\ldots{} la facoltà di vedere sempre i nostri difetti; ma noi esageriamo, e ci consoliamo con l'ironia che abbiamo sempre pronta sulle labbra. Io ti dico solo questo: metti in mano a un altro popolo d'Europa un'istituzione come il nostro consiglio\ldots{} i tedeschi e gli inglesi ne caverebbero la libertà; noi invece, ci ridiamo su. 

- Ma che fare? - disse Levin mortificato. - Era il mio ultimo esperimento e l'ho fatto con tutta l'anima. Non posso. Non sono adatto. 

- Non è che non sei adatto - disse Sergej Ivanovic - tu non guardi la cosa così come va guardata. 

- Forse - disse Levin scoraggiato. 

- Sai, Nikolaj è di nuovo qui. 

Il fratello Nikolaj, germano e maggiore di Konstantin Levin e fratello per parte di madre di Sergej Ivanovic, era un uomo rovinato che aveva sperperato la maggior parte del suo patrimonio, frequentava l'ambiente più strano e più guasto, ed era in lite coi fratelli. 

- Cosa dici? - gridò Levin. - Come lo sai? 

- Prokofij l'ha visto per istrada. 

- Qui, a Mosca? e dov'è? lo sai? - Levin s'alzò dalla sedia, come se volesse andar via subito. 

- Mi dispiace d'averti detto questo - disse Sergej Ivanovic, scrollando il capo all'agitazione del fratello minore. - Ho cercato di sapere dove vive e gli ho mandato la sua cambiale intestata a Trubin che ho pagato io. Ecco quello che mi ha risposto. 

E Sergej Ivanovic prese un biglietto di sotto a un fermacarte e lo porse al fratello. 

Levin lesse quello che vi era stato tracciato con una scrittura strana, a lui familiare: ``Chiedo umilmente di essere lasciato in pace. Questa è l'unica cosa che pretendo dai miei cari fratelli. Nikolaj Levin''. 

Levin lesse e, senza alzar la testa, rimase in piedi davanti a Sergej Ivanovic col biglietto in mano. 

Nell'animo suo lottavano in quel momento il desiderio di ignorare il fratello disgraziato e la coscienza che ciò sarebbe stato male. 

- È evidente che vuole offendermi - continuò Sergej Ivanovic. - Ma non può offendermi; e io vorrei aiutarlo con tutta l'anima, ma so che non è possibile. 

- Eh, sì - ripeté Levin. - Capisco e apprezzo il tuo atteggiamento verso di lui; ma io andrò da lui. 

- Se ne hai voglia, vacci, ma non te lo consiglio - disse Sergej Ivanovic. - Non già per me, non temo certo che egli ti metta in urto con me; ma è per te, che ti consiglio di non andare. Aiutarlo non si può. Comunque fa' come vuoi. 

- È probabile che non si possa neanche aiutarlo, ma io sento\ldots{} proprio in questo particolare momento\ldots{} già, ma questa è un'altra cosa\ldots{} Insomma, sento che non posso restarmene tranquillo. 

- Io questo non lo capisco - disse Sergej Ivanovic. - Una cosa capisco invece - aggiunse - che questa è una lezione di umiltà. Da che nostro fratello Nikolaj è diventato quello che è, io ho preso a considerare in modo diverso e con maggiore indulgenza ciò che si chiama abiezione\ldots{} Lo sai cosa ha fatto\ldots{} 

- Ah, tremendo, tremendo! - ripeté Levin. 

Dopo aver avuto dal domestico di Sergej Ivanovic l'indirizzo del fratello, Levin avrebbe voluto andare immediatamente da lui; ma, riflettendo, aveva deciso di rinviare la visita alla sera. Prima di tutto, per avere serenità di spirito, doveva decidere la faccenda per la quale era venuto a Mosca. Così dalla casa del fratellastro, Levin era passato all'ufficio di Oblonskij e, informatosi degli Šcerbackij, era andato dove gli era stato detto che avrebbe potuto trovare Kitty. 

\capitolo{IX}\label{ix} 

Alle quattro precise, col cuore che gli batteva, Levin scese dalla vettura al giardino zoologico e si incamminò per un viottolo verso le montagne di ghiaccio e verso il campo di pattinaggio dove era sicuro di trovare lei, perché all'ingresso aveva visto la carrozza degli Šcerbackij. 

La giornata era chiara, gelida. All'ingresso c'erano file e file di carrozze, slitte, vetturini e gendarmi. Una folla ben vestita, coi cappelli che luccicavano al sole forte, brulicava all'ingresso e per i viali ben spazzati, fra le casette russe dalle travi scolpite, mentre le vecchie betulle frondose del giardino, con tutti i rami curvi per la neve, sembravano adorne di nuove pianete di gala. 

Levin andava per un viottolo verso il campo di pattinaggio, e diceva a se stesso: ``Non bisogna agitarsi\ldots{} bisogna star tranquilli. Perché? Come mai? Taci, sciocco'' diceva rivolto al cuore. Ma quanto più cercava di calmarsi, tanto più gli si mozzava il respiro. Un amico lo incontrò e lo chiamò, ma Levin non riconobbe chi era. Si accostò alle montagne di ghiaccio sulle quali stridevano le catene delle piccole slitte rotolanti e risonavano voci allegre. Fece ancora alcuni passi, e davanti a lui si aprì il campo di pattinaggio e, subito, in mezzo a tutti quelli che pattinavano riconobbe lei. 

Riconobbe che era là per la gioia e l'ansia che gli afferrarono il cuore. Lei stava in piedi, parlando con una signora, all'estremo opposto del campo. Non c'era nulla di particolare, almeno così sembrava, nell'abito suo e nella sua posa; ma per Levin era così facile riconoscere lei tra tanta gente, così come una pianta di rose fra le ortiche. Tutto prendeva luce da lei: era lei il sorriso che illuminava tutto, d'ogni intorno. ``Ma potrò davvero scendere là sul ghiaccio, accostarmi?'' pensò. Il luogo dove lei era gli sembrò un impenetrabile luogo sacro, e per un attimo fu sul punto di andarsene, tanta agitazione lo aveva preso. Dovette fare uno sforzo su se stesso e considerare che accanto a lei camminava gente di ogni specie e che anche lui poteva andare là a pattinare. Scese, evitando di guardarla a lungo, come si fa col sole, ma vedeva lei, come si vede il sole, anche senza guardare. 

In quel giorno della settimana e a quell'ora si riunivano sul ghiaccio persone di uno stesso gruppo che si conoscevano fra di loro. C'erano i campioni del pattinaggio, che si esibivano con arte, e c'erano quelli che imparavano reggendosi alle sedie, con movimenti timidi e impacciati, e c'erano ragazzi e persone anziane che pattinavano per ragioni igieniche: tutti parvero a Levin persone elette e felici perché erano là, vicino a lei. I pattinatori, invece, sembravano sorpassarla con assoluta indifferenza, raggiungerla, parlare persino e divertirsi in modo del tutto indipendente da lei, profittando del ghiaccio ottimo e del buon tempo. 

Nikolaj Šcerbackij, cugino di Kitty, in giacchetta corta e pantaloni stretti, sedeva su di una panchina, coi pattini ai piedi e, visto Levin, gli gridò: 

- Olà, il primo pattinatore di Russia! Da quanto tempo siete qui? Ottimo ghiaccio, mettetevi i pattini. 

- Non li ho neppure - rispose Levin, sorpreso di quell'ardire e di quella disinvoltura alla presenza di lei, che egli, anche senza guardare, non perdeva mai di vista. Sentiva che il sole si avvicinava. Ella era in un canto e, strette ad angolo ottuso le gambe sottili negli stivaletti, visibilmente incerta, gli pattinò incontro. Un ragazzo in costume russo, che gesticolava in maniera disperata e si piegava verso terra, la sorpassò. Ella non pattinava ancora del tutto sicura; e, cacciate le mani fuori del piccolo manicotto, sospeso a un cordone, le teneva pronte; guardando Levin che aveva riconosciuto, sorrideva a lui e alla propria timidezza. Superata la curva, si dette una leggera spinta con la gamba agile e pattinò diritta verso Šcerbackij; afferratasi a lui con la mano, fece, sorridendo, un cenno col capo a Levin. Era più bella di quanto non immaginasse. 

Quando Levin la pensava poteva rappresentarsi al vivo tutta lei, e in particolare l'incanto di quella testina bionda dall'espressione infantile, limpida e buona, così graziosamente posata sulle spalle ben fatte di fanciulla. L'infantile espressione del viso congiunta alla bellezza sottile della figura formavano il suo incanto particolare, che egli aveva ben presente; ma quello che in lei lo colpiva sempre come cosa inattesa, era l'espressione degli occhi miti, tranquilli e schietti, e quel sorriso che trasportava Levin in un mondo magico nel quale si sentiva intenerito e placato, così come ricordava di essere stato nei pochi giorni felici della sua prima infanzia. 

- Da molto qui? - disse lei, dandogli la mano. - Grazie - aggiunse quando egli raccattò il fazzoletto cadutole dal manicotto. 

- Io? io, da poco, ieri\ldots{} quest'oggi, cioè\ldots{} sono arrivato - rispose Levin non avendo capito subito la domanda per l'agitazione. - Volevo venire da voi - aggiunse, ma, ricordatosi subito con quale intenzione la cercava, si turbò e arrossì. - Non sapevo che pattinaste, ed anche bene. 

Lei lo guardò con attenzione come se desiderasse capire la causa di quel turbamento. 

- Bisogna tenerla in conto la vostra lode. Qui corre voce che siate il miglior pattinatore - disse lei, scotendo con la piccola mano inguantata gli aghi di brina che si erano posati sul manicotto. 

- Già, una volta pattinavo con passione; volevo raggiungere la perfezione. 

- Voi fate tutto con passione, a quanto pare - disse lei sorridendo. - Ho tanta voglia di vedere come pattinate. Mettetevi i pattini e andiamo a pattinare insieme. 

``Pattinare insieme? È mai possibile?'' pensò Levin guardandola. 

- Me li infilo subito - disse. 

E andò a mettersi i pattini. 

- Da un pezzo non vi si vedeva, signore - disse l'uomo dei pattini alzandogli un piede e avvitando il tacco.- Dopo di voi, di signori, non ce n'è stato più nessuno in gamba. Va bene così? - diceva, stringendo le cinghie. 

- Bene, bene, presto per favore - rispondeva Levin, trattenendo a stento il sorriso di felicità che gli appariva involontariamente sul viso. ``Sì: ecco la vita - pensò - ecco la felicità. Insieme, ha detto lei, andiamo a pattinare insieme. Dirglielo ora? Ma, ecco perché ho paura di parlare, perché sono felice, felice sia pure di speranza\ldots{} E allora? Ma si deve, si deve! Bando alla paura!''. 

Levin si alzò, si tolse il soprabito e, correndo sul ghiaccio non levigato intorno al casotto, si lanciò di corsa sulla superficie liscia e pattinò senza sforzo, rallentando e dirigendo la corsa, come spinto solo dalla propria volontà. Le si accostò timido, ma di nuovo il sorriso di lei lo placò e rasserenò. 

Gli dette la mano e si avviarono insieme aumentando l'andatura, e quanto più questa diveniva veloce tanto più forte ella stringeva la mano di lui. 

- Con voi avrei imparato più presto; non so perché, mi sento sicura con voi - gli disse. 

- Ed anch'io mi sento sicuro quando voi vi appoggiate a me - disse lui; ma spaventato di quello che aveva detto, arrossì. E infatti, appena pronunziate quelle parole, fu come se il sole si fosse nascosto dietro le nuvole: il viso di lei perse la sua tenerezza, e Levin riconobbe il giuoco a lui noto del viso che rivelava lo sforzo del pensiero: sulla fronte spianata era apparsa una piccola ruga. 

- C'è qualcosa che vi spiace? Ma già io non ho il diritto di chiedere - aggiunse in fretta. 

- E perché no? No, non c'è nulla che mi spiaccia - rispose, fredda, lei, e aggiunse subito: - Non avete visto m.lle Linon? 

- Non ancora. 

- Andate da lei, vi vuole tanto bene. 

``Cos'è questo? L'ho contrariata. Signore, aiutami!'' pensò Levin e corse verso la vecchia francese dai riccioli grigi, seduta sulla panchina. Costei l'accolse come un vecchio amico, mostrando nel sorriso i suoi denti finti. 

- Ecco, si cresce, non è vero? - gli disse indicando con gli occhi Kitty - e noi si invecchia. Tiny bear è già grande! - continuò la francese ridendo e ricordando lo scherzo sulle tre signorine ch'egli chiamava col nome dei tre orsacchiotti della fiaba inglese. - Ricordate, voi la chiamavate così? 

Egli non ricordava proprio nulla, ma lei rideva e si compiaceva di questo scherzo da più di dieci anni. 

- Su, su andate a pattinare. La nostra Kitty ha cominciato a pattinare bene, non è vero? 

Quando Levin corse di nuovo verso Kitty, il viso di lei non era più severo, gli occhi guardavano sinceri e dolci, eppure a Levin parve che nella sua dolcezza ci fosse una particolare intonazione di calma voluta. E gliene venne tristezza. Dopo aver parlato un po' della vecchia governante, delle sue originalità, ella gli chiese della sua vita. 

- Non vi vien noia d'inverno, in campagna? - disse. 

- No, niente noia, sono tanto occupato - disse lui sentendo d'essere soggiogato da quel tono calmo al quale non avrebbe avuto la forza di sottrarsi, proprio così com'era successo al principio dell'inverno. 

- Siete venuto per molto tempo? - gli chiese Kitty. 

- Non lo so - rispose lui, senza pensare a quel che diceva. Gli era venuto in mente il pensiero che se si fosse avvezzato a quel tranquillo tono di amicizia, sarebbe di nuovo partito senza aver risolto nulla, e decise di opporvisi. 

- Come, non lo sapete? 

- Non so, dipende da voi - disse, ed ebbe subito paura delle proprie parole. 

O ch'ella non avesse sentito quelle parole, o che non avesse voluto sentirle, certo sembrò inciampicare, batté due volte col piedino a terra e pattinò in fretta via da lui. Si avvicinò pattinando a m.lle Linon, le disse qualcosa e si diresse verso il casotto dove le signore toglievano i pattini. 

``Dio mio, che ho fatto! Signore Iddio! Aiutami, guidami tu!'' diceva Levin pregando e, sentendo nello stesso tempo un bisogno di moto violento, prendeva la rincorsa e disegnava giri esterni e interni. 

In quel momento uno dei giovani, il più abile dei nuovi pattinatori, con la sigaretta in bocca, uscì dal caffè sui pattini e, presa la rincorsa, si slanciò giù per gli scalini, strepitando e saltellando. Era volato giù, e, senza cambiar neppure la libera posizione delle braccia, aveva ripreso a pattinar sul ghiaccio. 

- Ah, ecco un esercizio nuovo - disse Levin, e corse subito a tentarlo. 

- Volete ammazzarvi! Ci vuol allenamento! - gli gridò Nikolaj Šcerbackij. 

Levin salì i gradini, prese la rincorsa quanto più poté dall'alto e si lanciò giù, mantenendosi in equilibrio con le braccia nel movimento insolito. Sull'ultimo scalino inciampò, ma, sfiorato appena il ghiaccio con la mano, fece un movimento brusco, si raddrizzò e, ridendo e pattinando, volò via. 

``Bravo, simpatico - pensò Kitty, uscendo in quel momento dal casotto con m.lle Linon e guardandolo con un sereno sorriso carezzevole, come un fratello al quale si vuol bene. - Possibile che io sia colpevole, possibile che abbia fatto qualcosa di male? Dicono: civetteria. Lo so che non amo lui, ma intanto con lui ci sto volentieri e lui è così bravo. Ma perché ha detto quella cosa?\ldots{}'' pensava. 

Nel veder Kitty che andava via e la madre che la raggiungeva sulle scale, Levin, tutto rosso ancora per il movimento brusco che aveva fatto, si fermò a riflettere. Si tolse i pattini e raggiunse all'uscita madre e figlia. 

- Molto lieta di vedervi - disse la principessa. - Il giovedì, come sempre, riceviamo. 

- Allora, oggi? 

- Saremo molto lieti di vedervi - rispose asciutta la principessa. 

Questo tono secco amareggiò Kitty, ed ella non poté trattenersi dall'attenuare la freddezza della madre. Girò la testa e con un sorriso disse: 

- A rivederci. 

In quel momento Stepan Arkad'ic, col cappello di traverso, il viso e gli occhi luccicanti, entrava nel giardino come un trionfatore. Ma, avvicinatosi alla suocera, rispose con viso contrito e confuso alle domande di lei sulla salute di Dolly. Dopo aver parlato a voce bassa e sommessa con la suocera, raddrizzò il torace e prese Levin sottobraccio. 

- E allora, andiamo? - chiese. - Non ho fatto che pensare a te e sono molto contento che tu sia venuto - disse, guardandolo negli occhi con aria significativa. 

- Andiamo, andiamo - rispose felice Levin che non cessava di ascoltare il tono della voce che aveva detto ``a rivederci'' e di vedere il sorriso col quale le parole erano state dette. 

- All'``Inghilterra'' o all'``Ermitage''? 

- Per me è lo stesso. 

- Su, all'``Inghilterra'' - disse Stepan Arkad'ic e scelse l'``Inghilterra'' perché all'``Inghilterra'' aveva un debito più grosso che non all'``Ermitage'', e riteneva suo dovere farsi vedere in quel ristorante. - Hai una vettura? Benissimo, perché ho rimandato indietro la mia. 

Per tutta la strada gli amici tacquero. Levin pensava cosa potesse significare quel mutamento di espressione nel viso di Kitty, e ora rassicurava se stesso col dirsi che una speranza c'era, ora si abbandonava alla disperazione sembrandogli chiaro che la sua speranza fosse completamente insensata; intanto si sentiva tutto un altro uomo, non più simile a quello che era stato fino al sorriso di lei e fino alle parole ``a rivederci''. 

Stepan Arkad'ic durante il percorso componeva la lista del pranzo. 

- Ti piace il rombo? - disse a Levin quando furono giunti. 

- Che cosa? - domandò Levin. - Il rombo? Ah, sì, mi piace straordinariamente il rombo. 

\capitolo{X}\label{x} 

Quando Levin entrò nel locale con Oblonskij, non poté fare a meno di notare una certa particolare espressione, come una vivacità contenuta nel viso e nella figura tutta di Stepan Arkad'ic. Oblonskij si tolse il cappotto e, col cappello calato da un lato, passò nella sala da pranzo, dando gli ordini ai tartari che gli si erano messi dietro, in frac e col tovagliolo sul braccio. Salutando a destra e a sinistra gli amici che si trovavano là e che lo salutavano dovunque con gioia, si avvicinò al banco, prese come antipasto vodka e pesce salato e disse qualcosa alla francese che sedeva alla cassa tutta pitturata e ricoperta di nastri, pizzi e ghirigori, in modo che anche questa si mise a ridere schiettamente. Levin non bevve la vodka solo perché gli dava fastidio quella francese che sembrava fatta di capelli finti, poudre de riz e vinaigre de toilette. Si allontanò in fretta da lei come da un luogo sudicio. L'animo suo era tutto pieno del ricordo di Kitty e nei suoi occhi splendeva un sorriso di trionfo e di felicità. 

- Di qua, eccellenza, prego, qua nessuno disturberà vostra eccellenza - diceva un vecchio tartaro biancastro, che più degli altri gli si era appiccicato, con un vasto ventre che sporgeva tra le falde del frac aperte. - Prego, eccellenza, - diceva a Levin, mostrando di occuparsi, in segno di deferenza verso Stepan Arkad'ic, anche dell'ospite. 

Dopo aver steso, in un batter d'occhio, una tovaglia di bucato su di un tavolo tondo già ricoperto di un'altra tovaglia, proprio sotto a un doppiere di bronzo, accostò le sedie di velluto e si piantò davanti a Stepan Arkad'ic col tovagliolo e la lista in mano, aspettando ordini. 

- Se vostra eccellenza ordina un salottino separato, subito se ne farà uno libero: il principe Golycin con una signora. Sono arrivate le ostriche fresche. 

- Ah, le ostriche! 

Stepan Arkad'ic si mise a pensare. 

- Dobbiamo cambiare piano, Levin? - disse fermando un dito sulla carta. Il suo viso esprimeva seria perplessità. - Son buone le ostriche? Bada, ve'! 

- Di Flensburg, eccellenza, non di Ostenda. 

- Flensburg o non Flensburg, sono poi fresche? 

- Le abbiamo avute ieri, eccellenza. 

- E va bene, non potremmo forse incominciare dalle ostriche e poi cambiare tutto il piano? Eh? 

- Per me è lo stesso. Per me meglio di tutto\ldots{} zuppa di cavoli e polenta. Ma qui non c'è di questa roba. 

- Kaša a la rjùss? desidera il signore? - disse il tartaro chinandosi su Levin come una balia sul bambino. 

- No, scherzi a parte, per me va bene quello che sceglierai tu. Ho pattinato un po' e ora ho voglia di mangiare. E non credere - aggiunse notando sul viso di Oblonskij un'aria di disappunto - che non apprezzi la tua scelta. Mangerò e con gusto. 

- Altro che! Di' quello che vuoi, ma questo è uno dei piaceri della vita - disse Stepan Arkad'ic. - Su, allora, amico mio, dacci due dozzine, ma forse è poco, tre dozzine di ostriche, una minestra di radiche\ldots{} 

- Prentanjèr - riprese il tartaro. Ma Stepan Arkad'ic evidentemente non voleva concedergli la soddisfazione di chiamare le pietanze in francese. 

- Di radiche, sai. Poi del rombo con una salsa densa, poi del rosbif: ma guarda che sia buono. Un cappone, e che so, via, della macedonia di frutta. 

Il tartaro, ricordatosi che Stepan Arkad'ic aveva l'abitudine di non nominare mai le portate in francese, non gli tenne dietro a ripetere, ma si concesse infine la soddisfazione di elencare tutta l'ordinazione secondo la carte: ``Sup prentanjèr, tjurbò sos Bomaršé, pulàrd alestragón, maseduàn de frjuì''; - e subito, come una molla, riposta la lista rilegata e presane un'altra, quella dei vini, la sottopose a Stepan Arkad'ic. 

- E cosa berremo? 

- Per me quello che vuoi tu; pur che non sia molto\ldots{} Dello champagne. 

- Come? In principio? Ma sì, hai ragione. Ti piace quello di marca bianca? 

- Kašé blan - riprese il tartaro. 

- Su, via, dacci marca bianca sulle ostriche, e poi vedremo. 

- Sissignore. E quale vino da pasto? 

- Del nuits; ma no, allora è meglio il classico chablis. 

- Sissignore, il solito formaggio? 

- Ma sì; del parmigiano. O te ne piace un altro? 

- No, per me è lo stesso - disse Levin trattenendo a stento un sorriso. 

E il tartaro con le falde svolazzanti, corse via e dopo cinque minuti entrò volando con un vassoio di ostriche aperte sui gusci di madreperla e una bottiglia fra le dita. 

Stepan Arkad'ic spiegazzò il tovagliolo inamidato, se lo ficcò nel panciotto e, posate tranquillamente le braccia sulla tavola, prese a occuparsi delle ostriche. 

- Non sono cattive - diceva, strappando con la forchetta d'argento le ostriche in guazzo dal guscio di madreperla e inghiottendone una dietro l'altra. 

- Non sono cattive - ripeteva, alzando gli occhi umidi e lustri ora su Levi, ora sul tartaro. 

Levin mangiava anche lui le ostriche, sebbene il pane bianco col formaggio gli piacesse di più. Ma si beava a guardare Oblonskij. Perfino il tartaro che aveva stappato lo champagne e lo versava nelle larghe coppe sottili guardava Stepan Arkad'ic con un evidente sorriso di compiacimento, aggiustandosi la cravatta bianca. 

- Ma non ti piacciono le ostriche? - disse Stepan Arkad'ic vuotando la coppa - o forse sei preoccupato? Eh? 

Voleva che Levin stesse di buon umore. Non che Levin non fosse di buon umore, ma era piuttosto impacciato. Con quello che aveva nell'animo provava sgomento e disagio in quel ristorante, in mezzo a salottini riservati dove si pranzava con donne, fra un andirivieni di gente e in mezzo a tutta quella mostra, a quello sfoggio di bronzi, specchi, becchi a gas e tartari. Tutto questo lo offendeva. Aveva paura di contaminare quel che gli riempiva l'anima. 

- Io? Sì, sono preoccupato; ma poi tutto questo mi dà soggezione - disse. - Tu non puoi immaginare come per me, abitante della campagna, tutto questo sia strano, così come le unghie di quel signore che ho visto da te\ldots{} 

- Già, ho visto che le unghie del povero Grinevic ti interessavano molto - disse, ridendo, Stepan Arkad'ic. 

- Non riesco a capire - rispose Levin. - Ma tu cerca di metterti nei panni miei, mettiti dal punto di vista dell'abitante di campagna. Noi in campagna cerchiamo di avere le mani fatte in modo che sia comodo lavorarci, perciò le unghie le tagliamo, e qualche volta ci rimbocchiamo le maniche. E qui invece c'è chi lascia crescere le unghie finché reggono e si aggancia ai polsi bottoni che paion piattini, in modo da non poter far più nulla con le mani. 

Stepan Arkad'ic sorrideva allegro. 

- Eh, già. Questo vuol dire che per lui il lavoro manuale non è più necessario. È il cervello che lavora\ldots{} 

- Sarà. Ma per me ciò è strano; così come, per me, è strano che, mentre noi abitanti di campagna cerchiamo di saziarci al più presto per metterci in condizione di compiere il nostro lavoro, noi due, in questo momento, stiamo facendo di tutto per non saziarci; e per questo mangiamo le ostriche\ldots{} 

- Su via, ma s'intende - riprese Stepan Arkad'ic. - Ma è proprio in questo lo scopo dell'evoluzione: nel fare di tutto un godimento. 

- Se questo è lo scopo, aspirerei a essere un selvaggio. 

- Sei un selvaggio anche così. Voi Levin siete tutti selvaggi. 

Levin sospirò. Si ricordò del fratello Nikolaj, provò vergogna e pena e si accigliò, ma Oblonskij prese a parlare di un argomento che lo distrasse subito. 

- E allora, ci vai stasera dai nostri, dagli Šcerbackij? - disse, allontanando i gusci vuoti e scabri, avvicinando a sé il formaggio e ammiccando significativamente con gli occhi. 

- Sì, ci vado senz'altro - rispose Levin. - Benché sia convinto che la principessa mi abbia invitato controvoglia. 

- Ma che dici! Sciocchezze! È il suo modo di fare\ldots{} Su, via, amico, dacci la minestra!\ldots{} È il suo modo di fare, grande dame - disse Stepan Arkad'ic. - Anch'io verrò ma prima devo andare alla prova di canto della contessa Bonina. Eh già, come si fa a dire che non sei un selvaggio? Come spiegare che sul più bello sei scomparso da Mosca? Gli Šcerbackij mi chiedevano di te continuamente, come se io dovessi sapere. E io so una sola cosa: che fai sempre quello che nessuno fa. 

- Già - disse Levin lentamente e con emozione. - Tu hai ragione, sono un selvaggio. Ma questa mia selvatichezza non consiste nel fatto che me ne sono andato, ma che son venuto. Ora io son venuto\ldots{} 

- Oh che uomo felice! - esclamò Stepan Arkad'ic guardando Levin negli occhi. 

- E perché? 

- ``Conosco i cavalli ardenti da certi loro segni; conosco i giovani innamorati dagli occhi'' - declamò Stepan Arkad'ic. - Tu hai tutto l'avvenire davanti a te. 

- E che forse tu hai già tutto nel passato? 

- No, non avrò solo il passato, ma tu hai l'avvenire, mentre io ho il presente, e anche quello a sbalzi. 

- Ma che c'è? 

- Non va bene, non va bene. Ma io di me non voglio parlare, e poi, dopo tutto, non si può neanche spiegare - disse Stepan Arkad'ic. - Ma tu perché mai sei venuto a Mosca?\ldots{} Ehi, piglia su! - gridò al tartaro. 

- Non l'indovini? - rispose Levin senza staccare da Stepan Arkad'ic i suoi occhi luminosi. 

- L'indovino, ma non posso cominciare io a parlarne. Già da questo puoi vedere se colgo o no nel segno - disse Stepan Arkad'ic, guardando Levin con un sorriso sottile. 

- E allora che ne dici? - disse Levin con voce tremante e sentendo vibrare tutti i muscoli del viso. - Come la vedi tu la cosa? 

Stepan Arkad'ic bevve lentamente il suo bicchiere di chablis, senza staccare gli occhi da Levin. 

- Io? - disse Stepan Arkad'ic - io non desidero niente più di questo. È la cosa migliore che possa accadere. 

- Ma tu non ti sbagli? Sai bene di che parliamo? - ripeté Levin, ficcando gli occhi nel suo interlocutore. - Credi che sia possibile? 

- Credo che sia possibile. E perché mai impossibile? 

- Ma pensi proprio che sia possibile? No, dimmi tutto quello che pensi! E se mi aspetta un rifiuto? E io anzi ne sono certo\ldots{} 

- Perché pensi questo? - disse Stepan Arkad'ic sorridendo a quell'agitazione. 

- A volte così mi sembra. Certo questo sarebbe terribile per me e per lei. 

- Be', veramente, in ogni caso, per una ragazza non c'è nulla di terribile. Ogni ragazza è lusingata di essere chiesta in matrimonio. 

- Già, ogni ragazza, ma non lei. 

Stepan Arkad'ic sorrise. Conosceva bene il sentimento di Levin; sapeva che per lui tutte le ragazze del mondo si dividevano in due categorie: nella prima c'erano tutte le ragazze di questo mondo tranne lei, e queste ragazze avevano tutte le debolezze umane ed erano esseri molto comuni; nella seconda, c'era lei sola e non aveva nessuna debolezza, ed era superiore ad ogni cosa umana. 

- Aspetta, prendi la salsa - disse trattenendo il braccio di Levin che allontanava da sé la salsa. 

Levin si servì docilmente, ma non permise a Stepan Arkad'ic di mangiare. 

- No, aspetta, aspetta - diceva. - Tu devi capire che questo per me è questione di vita o di morte. Io non ne ho mai parlato con nessuno. E con nessun altro posso parlare di questo se non con te. Perché, ecco, io e te siamo estranei l'uno all'altro: gusti diversi, opinioni, tutto. Ma io so che tu mi vuoi bene e mi capisci e per questo ti voglio un gran bene. Ma in nome di Dio sii sincero con me. 

- Io ti dico quello che penso - disse Stepan Arkad'ic, sorridendo. - Ma io ti dirò di più: mia moglie, una donna straordinaria\ldots{} - Stepan Arkad'ic sospirò, ricordando i suoi rapporti con la moglie, e, sostando un attimo, continuò: - ha il dono dell'introspezione. Vede da una parte all'altra; ma questo è poco, sa quello che accadrà, specie in materia di matrimoni. Per esempio, ha predetto che la Šachovskaja avrebbe sposato Brentel'n. Nessuno ci voleva credere, ed è stato così. Ebbene, lei è dalla parte tua. 

- Come? 

- Così: non solo ti vuol bene, ma dice che Kitty sarà certamente tua moglie. 

A queste parole il viso di Levin s'illuminò d'un tratto di quel sorriso ch'è vicino alle lacrime della commozione. 

- Lei dice questo! - gridò Levin. - Ho sempre detto che tua moglie è un tesoro! E ora basta, basta, non ne parliamo più! - disse, alzandosi. 

- Sì, va bene, mettiti a sedere. 

Levin non poteva stare seduto. Andò su e giù due volte con passo deciso per la stanza che sembrava una piccola gabbia. Sbatté le palpebre per non mostrare le lacrime e solo allora sedette di nuovo a tavola. 

- Tu comprendi - disse - che questo non è un innamoramento. Sono stato innamorato ma non è questo. Questo non è un sentimento mio, ma è una forza esterna che si è impossessata di me. Ero andato via perché avevo concluso che ciò non poteva essere, cioè, intendimi, come una felicità che non poteva esistere sulla terra; ma ho lottato con me stesso e ora vedo che senza di questo non c'è vita. E bisogna dunque decidere\ldots{} 

- E per questo sei andato via? 

- Ah, lascia stare! Quanti pensieri! Quante cose ti devo chiedere! Ascolta. Tu già non puoi immaginare che cosa hai fatto ora per me nel dirmi ciò. Sono così felice da diventare quasi disgustoso; ho dimenticato tutto. Ho saputo oggi che mio fratello Nikolaj\ldots{} Anche di lui mi sono scordato. Mi sembra che anche lui debba essere felice. Questa è una specie di pazzia. Ma c'è una cosa che è terribile\ldots{} Ecco, tu ti sei sposato, tu certamente lo conosci questo sentimento\ldots{} Ed è terribile questo, che noi\ldots{} non più giovani, già con un passato\ldots{} non di amore, ma di peccato\ldots{} ci avviciniamo a un tratto a un essere puro, ignaro. È ripugnante, e non si può non sentirsene indegni. 

- Su, via, tu di peccati ne hai pochi. 

- Eppure, eppure - disse Levin - ``considerando con disgusto la mia vita, fremo e maledico e amaramente mi dolgo''. Proprio così. 

- Che fare? Così è fatto il mondo - disse Stepan Arkad'ic. 

- L'unica mia consolazione è in quella preghiera che ho sempre amata: ``Non secondo i miei meriti, ma secondo la tua misericordia, perdonami''. Soltanto così anche lei può perdonare. 

\capitolo{XI}\label{xi} 

Levin bevve la sua coppa e i due rimasero in silenzio. 

- Una cosa nuova devo dirti. Conosci Vronskij? - chiese Stepan Arkad'ic a Levin. 

- No, non lo conosco. Perché me lo chiedi? 

- Versane un'altra - disse Stepan Arkad'ic al tartaro che aveva cessato di riempire le coppe e che gironzolava intorno a loro proprio quando non era necessario. 

- Perché dovrei conoscere Vronskij? 

- Dovresti conoscere Vronskij perché è uno dei tuoi rivali. 

- E che tipo è questo Vronskij? - chiese Levin e il viso suo tramutò l'espressione d'infantile entusiasmo che proprio allora aveva incantato Oblonskij in un'espressione torva e spiacevole. 

- Vronskij è uno dei figli del conte Kirill Ivanovic Vronskij ed è uno dei più bei campioni della gioventù dorata di Pietroburgo. L'ho conosciuto a Tver' quando prestavo servizio là e lui ci veniva per l'arruolamento delle reclute. Ricco sfondato, bello, grandi relazioni, aiutante di campo dello zar e, nello stesso tempo, molto simpatico, un buon ragazzo. Ma oltre che un buon ragazzo, come ho potuto poi conoscerlo qui, è anche colto e intelligente; un giovane che si farà strada. 

Levin si faceva scuro in viso e taceva. 

- Dunque costui è comparso qua dopo di te e, a quanto mi pare di aver capito, è innamorato pazzo di Kitty, e tu capirai che la madre\ldots{} 

- Scusami, ma non capisco nulla - disse Levin cupo e accigliato. E subito si ricordò di suo fratello Nikolaj e come fosse stato perfido l'averlo dimenticato. 

- Aspetta, aspetta - disse Stepan Arkad'ic, sorridendogli e toccandogli il braccio. - Io ti ho detto quello che so, e ti ripeto che per quanto si possa indovinare in cose tanto sottili e delicate, mi sembra che le probabilità siano dalla parte tua. 

Levin si abbandonò all'indietro sulla sedia; il suo viso era pallido. 

- Ma io ti consiglio di decidere la questione al più presto - continuò Oblonskij, riempiendogli la coppa. 

- No, grazie, non posso bere più - disse Levin, allontanando la coppa. - Mi ubriacherei\ldots{} E tu, come te la passi? - continuò, volendo cambiare discorso. 

- Ancora una parola: in ogni caso ti consiglio di risolvere la cosa al più presto. Non ti consiglio di parlare oggi - disse Stepan Arkad'ic. - Va' domattina a far la tua domanda secondo l'uso classico, e che Dio ti benedica\ldots{} 

- Be', non dicevi sempre di voler venire a caccia da me? Ecco, vieni a primavera - disse Levin. 

Ora egli si pentiva con tutta l'anima di aver cominciato quel discorso con Stepan Arkad'ic. Il sentimento tutto suo era contaminato dal discorso su quel tale ufficiale di Pietroburgo suo rivale e dalle supposizioni e dai consigli di Stepan Arkad'ic. 

Stepan Arkad'ic sorrideva. Capiva quel che avveniva nell'animo di Levin. 

- Verrò un giorno o l'altro - disse. 

- Eh, già, amico mio, le donne\ldots{} ecco l'elica intorno alla quale tutto gira. Ecco, anche le mie cose vanno male. E tutto per colpa delle donne. Dimmi tu sinceramente - continuò - dopo aver tirato fuori un sigaro e tenendo la coppa con una mano - dammi un consiglio. 

- A che proposito? 

- Ecco qua. Mettiamo che tu sia ammogliato, che ami tua moglie, ma che tu abbia perso la testa per un'altra donna. 

- Scusa, ma io questo non lo capisco; come se, ecco, proprio così, io ora, dopo essermi saziato, passando accanto a quel negozio di ciambelle ne rubassi una. 

Gli occhi di Stepan Arkad'ic brillavano più del solito. 

- E perché? La ciambella a volte è così profumata che non puoi resistere. 

\begin{quote}
Himmlisch ist's wenn ich bezwungen

Meine irdische Begier; 

Aber noch wenn's nicht gelungen, 

Hatt'ich auch recht hubsch Plaisir!
\end{quote} 

Dicendo questo Stepan Arkad'ic sorrideva finemente. Anche Levin non poté non sorridere. 

- Sì, ma scherzi a parte - continuò Oblonskij - immagina una donna graziosa, un essere mite, affettuoso, povero, solo che abbia sacrificato ogni cosa. Ora, quando tutto è già avvenuto\ldots{} tu m'intendi, si può forse buttarla via? Ammettiamo pure: troncare per non distruggere la propria vita familiare; ma non si può forse avere pena di lei, provvedere, mitigare? 

- Eh, già, scusami. Tu sai, per me le donne si dividono in due categorie\ldots{} cioè, no, più esattamente: vi sono le donne e vi sono\ldots{} Io di magnifiche creature cadute non ne ho viste e non ne vedrò mai, e le donne come quella francese lì al banco, coi ricci, quelle per me sono vermi, e tutte quelle cadute sono tali. 

- E quella del Vangelo? 

- Ah, lascia stare! Cristo non avrebbe mai detto quelle parole, se avesse preveduto quanto se ne sarebbe abusato. Di tutto il Vangelo non si ricordano che quelle parole. Del resto io non dico ciò che penso, ma ciò che sento. Ho avversione per le donne cadute. Tu hai paura dei ragni e io di quei vermi. E tu certamente non hai studiato i ragni e non conosci le loro abitudini: e neanche io. 

- Va bene a parlare così, per te: mi sembri quel tal signore del Dickens che gettava con la mano sinistra dietro la spalla destra tutte le questioni spinose. Ma la negazione di un fatto non ne è la soluzione. Che fare mai, dimmi, che fare? Tua moglie invecchia e tu sei pieno di vita. Non fai in tempo a girarti che già senti di non potere più amare di amore tua moglie, per quanto la stimi. E qui a un tratto ti capita l'amore e sei perduto, sei perduto! - esclamò con sommessa disperazione Stepan Arkad'ic. 

Levin sorrise. 

- Già, e sei perduto - continuò Oblonskij. - Ma che fare? 

- Non rubare le ciambelle. 

Stepan Arkad'ic scoppiò a ridere. 

- Oh, il moralista! Ma tu devi capire che qui ci sono due donne: una insiste solo sui suoi diritti, e questi diritti sono l'amore che tu non puoi più darle; l'altra invece ti sacrifica tutto e non ti chiede nulla. Che devi fare? Come regolarti? Qui sta il dramma pauroso. 

- Se vuoi sapere il mio punto di vista, ti dirò che in questo non ci scorgo dramma. Ed ecco perché? Per me l'amore\ldots{} tutti e due gli amori che, ricordi, Platone definisce nel suo Convito, tutti e due questi amori servono di pietra di paragone degli uomini. Alcuni comprendono l'uno, altri l'altro. E quelli che comprendono solo l'amore non platonico parlano a vuoto di dramma. In un amore simile non può esservi dramma. ``Vi ringrazio umilmente per il piacere, i miei rispetti'' ed ecco tutto il dramma. E per l'amore platonico neppure può esservi dramma perché in un amore simile tutto è chiaro, puro, perché\ldots{} 

In quel momento Levin si ricordò delle sue colpe e della lotta interiore che aveva vissuto e inaspettatamente aggiunse: 

- Ma forse hai ragione, in fin dei conti, anche tu. Anzi, molto probabilmente\ldots{} Ma io non so, non so proprio. 

- Ecco, vedi - disse Stepan Arkad'ic - tu sei un uomo tutto d'un pezzo. Questo è il tuo pregio e il tuo difetto. Tu sei tutto d'un pezzo e vorresti che la vita fosse fatta di avvenimenti integrali, e questo non succede. Ecco, tu disprezzi l'attività del pubblico impiego, perché vorresti che essa corrispondesse sempre allo scopo, e questo non succede. Vorresti che l'attività di un uomo avesse sempre uno scopo, che l'amore e la vita familiare fossero tutt'uno. E questo non succede. Tutta la varietà, la delizia, la bellezza della vita son fatte d'ombre e di luci. 

Levin sospirò e non rispose nulla. Pensava alle cose sue e non ascoltava già più Oblonskij. 

E a un tratto tutti e due sentirono che, pur essendo amici, pur avendo pranzato insieme e bevuto il vino, cosa che ancor più avrebbe dovuto avvicinarli, tuttavia ognuno di loro pensava solo alle proprie cose, e a ciascuno non importava nulla dell'altro. Oblonskij conosceva già questo estremo distacco che avviene, in luogo della fusione, dopo un pranzo, e sapeva bene cosa si dovesse fare in casi simili. 

- Il conto! - gridò, e uscì nella sala accanto dove subito incontrò un aiutante di campo e si mise a parlare con lui di un'attrice e di chi la manteneva. E subito, parlando con l'aiutante di campo, Oblonskij provò sollievo e respirò dopo il colloquio con Levin che lo aveva sempre più sottoposto a uno sforzo intellettuale e spirituale troppo intenso. 

Quando il tartaro comparve col conto di 26 rubli e alcune copeche con l'aggiunta per la vodka, Levin che in altro momento, da buon provinciale, sarebbe inorridito per la propria quota di 14 rubli, non ci fece caso; pagò e si diresse verso casa per cambiar d'abito e andar dagli Šcerbackij dove si sarebbe decisa la sua sorte. 

\capitolo{XII}\label{xii} 

La principessina Šcerbackaja aveva diciotto anni. Era il primo inverno che faceva il suo ingresso nel gran mondo. I suoi successi erano superiori a quelli delle sorelle e superiori anche a quelli che la principessa si aspettava. Non solo i giovani che frequentavano i balli moscoviti erano tutti più o meno innamorati di Kitty, ma fin dal principio dell'inverno si erano presentati due partiti seri: Levin e, subito dopo la partenza di lui, il conte Vronskij. 

L'apparizione di Levin al principio dell'inverno, le visite frequenti e il suo evidente amore per Kitty erano stato l'oggetto dei primi discorsi seri fra i genitori di Kitty sul suo avvenire, e di litigi fra il principe e la principessa. Il principe era dalla parte di Levin; diceva che non desiderava nulla di meglio per Kitty. La principessa invece, con l'abitudine propria delle donne di girar la questione, diceva che Kitty era troppo giovane, che Levin non mostrava in nessun modo di aver intenzioni serie, che Kitty non mostrava affetto per lui e altre cose: ma non diceva la ragione principale, che s'aspettava, cioè, un partito migliore per sua figlia, e che Levin non le era simpatico, che non lo capiva. Quando Levin partì all'improvviso, la principessa ne fu contenta e diceva trionfante al marito: ``Vedi, avevo ragione io''. 

Quando poi apparve Vronskij, ella fu ancora più contenta, riconfermandosi nella propria idea, che cioè Kitty non doveva trovare un partito semplicemente buono, ma brillante. 

Per la madre non c'era paragone tra Levin e Vronskij. In Levin non le piacevano quegli strani e taglienti giudizi e quella sua mancanza di disinvoltura dovuta a orgoglio, come ella supponeva, e a quella sua vita di campagna, selvaggia a suo parere, fra bestie e contadini. Non le era piaciuto neanche troppo il fatto che, innamoratosi della figlia, avesse frequentato la casa per un mese e mezzo, quasi aspettando qualcosa e osservando, come se temesse di offenderla nel chiederla in isposa, e senza capire che, frequentando una casa dove c'era una ragazza da marito, fosse necessario dichiararsi. E poi a un tratto, senza aver parlato, era andato via. ``Meno male che è così poco attraente che Kitty non si è innamorata di lui'' pensava la madre. 

Vronskij invece soddisfaceva in pieno tutte le aspirazioni della madre. Molto ricco, intelligente, di famiglia nota, sulla via di una brillante carriera militare a corte, era un uomo affascinante. Non si poteva desiderare nulla di meglio. 

Ai balli Vronskij faceva apertamente la corte a Kitty, ballava con lei e ne frequentava la casa; non si poteva dubitare, dunque, della serietà delle sue intenzioni. Nonostante ciò la madre aveva passato tutto l'inverno in grande inquietudine e turbamento. 

La principessa si era sposata trent'anni prima, pronuba una zia. Il fidanzato, del quale si erano già prese informazioni, era venuto, aveva visto la sposa, si era fatto vedere lui stesso; la zia aveva saputo e riferito l'effetto prodotto. L'impressione era stata favorevole; così nel giorno stabilito era stata fatta ai genitori ed era stata da essi accolta la domanda di matrimonio. Tutto era andato in modo facile e semplice, almeno così sembrava alla principessa. Per le figliuole, invece, aveva provato come non fosse facile né semplice la faccenda, pur così comune, di dar marito alle figliuole. Quante ansie, quanti mutamenti di pensiero, quanto denaro speso, quanti urti col marito per i matrimoni delle prime due figlie, Dar'ja e Natal'ja! Adesso, nel presentare in società la più piccola, provava le stesse ansie, gli stessi dubbi; ed erano ancora più gravi che per le figlie maggiori le discussioni col marito. Il vecchio principe, come del resto tutti i padri, era particolarmente severo per l'onore e la virtù delle figliuole e ne era irragionevolmente geloso, specie di Kitty che era la beniamina; ogni momento faceva una scenata alla principessa perché comprometteva la figlia. La principessa si era abituata a questo già quando si era trattato delle altre due, ma ora sentiva che la suscettibilità del principe aveva maggior fondamento. Vedeva che negli ultimi tempi molte cose erano cambiate nelle usanze mondane; che i doveri di una madre erano diventati ancora più difficili. Vedeva che le coetanee di Kitty formavano certi gruppi, frequentavano certi corsi, trattavano con disinvoltura gli uomini, andavano sole in carrozza per le strade, molte di esse non facevano già più l'inchino, e, quel ch'era peggio, erano tutte fermamente convinte che la scelta del marito fosse affar loro e non dei genitori. ``Oggigiorno non ci si marita più come prima'' pensavano e dicevano tutte queste ragazze e anche tutte le persone anziane. Ma come si facesse ora a maritar le figlie, la principessa non riusciva a saperlo da nessuno. L'uso francese - ai genitori la decisione della sorte dei figli - non era accolto, era criticato. L'uso inglese - piena libertà alla ragazza - non era accolto ugualmente ed era impossibile nella società russa. L'uso russo della mediazione era considerato come cosa sconveniente sulla quale tutti ridevano, compresa la principessa. Ma come ci si dovesse maritare e come si dovesse dar marito, nessuno lo sapeva. Tutte le persone con le quali capitava alla principessa di parlarne, le dicevano una cosa sola: ``Su via, di grazia, oggigiorno è tempo di abbandonare tutto questo vecchiume. Sono i giovani che debbono sposarsi e non i genitori; bisogna lasciare ai giovani la facoltà di decidere come vogliono loro''. Ma era un bel dire per chi non aveva figliuole; e la principessa temeva che, facendo conoscenze, la figlia avrebbe potuto innamorarsi, e innamorarsi di chi non aveva nessuna intenzione matrimoniale o di chi non era adatto come marito. E per quanto tutti dicessero alla principessa che al giorno d'oggi i giovani devono da soli costruire il proprio avvenire, non riusciva ad ammetterlo, così come non avrebbe potuto ammettere che, in una qualsiasi epoca, i giocattoli migliori per i bambini di cinque anni potessero essere le pistole cariche. E perciò la principessa era ancora più inquieta per Kitty di quanto non lo fosse stata per le figliuole più grandi. 

Attualmente temeva che Vronskij si limitasse solo a far la corte alla figlia. Si accorgeva che la ragazza era già innamorata di lui, ma si rassicurava pensando che egli era un uomo d'onore e che perciò non avrebbe fatto questo. Ma sapeva pure come, con l'attuale libertà di costumi, fosse facile far perdere la testa ad una ragazza, e come, in genere, gli uomini guardassero con leggerezza a una colpa di questo genere. La settimana prima Kitty aveva raccontato alla madre la sua conversazione con Vronskij durante una mazurca. Questa conversazione aveva tranquillizzato in parte la principessa, ma del tutto serena ella non poteva sentirsi. Vronskij aveva detto a Kitty che, tanto lui che suo fratello, erano così abituati a sottostare in tutto alla madre, che non decidevano mai nulla di importante senza essersi prima consigliati con lei. ``E ora aspetto come una fortuna particolare l'arrivo della mamma da Pietroburgo'' aveva detto lui. 

Kitty aveva raccontato la cosa senza dare alcun peso a queste parole. La madre invece le aveva interpretate diversamente. Sapeva che si aspettava la vecchia signora da un giorno all'altro; sapeva che la vecchia signora sarebbe stata contenta della scelta del figlio, e le pareva strano ch'egli, solo per timore di offendere la madre, non facesse ancora la sua proposta di matrimonio; tuttavia desiderava tanto il matrimonio, e soprattutto la quiete ai propri affanni, che credeva a questo. Per quanto fosse amaro constatare la sfortuna della prima figlia, Dolly, che stava per separarsi dal marito, l'agitazione per la sorte della minore soffocava in lei ogni altro sentimento. Quel giorno con l'apparire di Levin le si era aggiunta una nuova inquietudine. Temeva che la figlia, che pur un tempo - così le era parso - aveva avuto della simpatia per Levin, rifiutasse per troppa onestà Vronskij, e temeva che, per un insieme di cose, l'arrivo di Levin non avesse a confondere e a ostacolare un affare già così prossimo alla conclusione. 

- Ma è molto che è arrivato? - disse, accennando a Levin, la principessa nel tornare a casa. 

- Oggi, maman. 

- Io voglio dire una cosa sola\ldots{} - cominciò la principessa e dal suo viso serio e animato, Kitty indovinò su quale argomento sarebbe scivolato il discorso. 

- Mamma - disse, avvampando in viso e volgendosi con vivacità. - Vi prego, vi prego, non mi parlate di questo. Io so, so tutto. 

Desiderava la stessa cosa che desiderava la madre; ma i motivi del desiderio materno la offendevano. 

- Io voglio dire solo che, dopo aver incoraggiato uno\ldots{} 

- Mamma, amore mio, in nome di Dio, non parlate. Fa così paura parlare di questo. 

- Non ne parlerò - disse la madre vedendo le lacrime negli occhi della figlia. - Ma una cosa sola, figliuola mia: tu mi hai promesso che non avrai segreti per me. Vero? 

- Mai, mamma, nessuno, - rispose Kitty, arrossendo e guardando dritto in faccia alla madre. - Ma ora non ho nulla da dire. Io\ldots{}io se volessi, non so, cosa dire e come\ldots{}non so\ldots{} 

``No, non può mentire con questi occhi'' pensò la madre, sorridendo di quell'agitazione e di quella felicità. La principessa sorrideva perché capiva come appariva grande e importante a lei, poverina, quello che accadeva nell'animo suo. 

\capitolo{XIII}\label{xiii} 

Kitty, dopo pranzo e fino al principio della serata, provò una sensazione simile a quella che prova un giovane prima del combattimento. Il cuore le batteva forte e il pensiero non riusciva a fermarsi su nulla. 

Sentiva che quella sera, quando i due uomini si sarebbero incontrati per la prima volta, si sarebbe decisa la sua sorte. E se li raffigurava continuamente, ora distinti, ora tutti e due insieme. Quando pensava al passato, con gioia e tenerezza si fermava sui ricordi dei suoi rapporti con Levin. I ricordi d'infanzia e l'amicizia di Levin col suo fratello morto davano un particolare poetico incanto ai suoi rapporti con lui. Il suo amore per lei, di cui era sicura, la lusingava e rallegrava. E le era naturale pensare a Levin. Al pensiero di Vronskij invece si frammischiava un certo impaccio, pur essendo egli un perfetto e sereno uomo di mondo; sembrava esserci una certa falsità, non in lui - era molto semplice e cortese - ma piuttosto in lei; mentre con Levin si sentiva completamente spontanea e serena. Ma intanto, quando pensava all'avvenire con Vronskij, le si presentava un luminoso sfondo di felicità; mentre con Levin l'avvenire si presentava nebbioso. 

Salita in camera per indossare l'abito da sera, gettò un'occhiata allo specchio, e si accorse con gioia che era in una delle sue giornate migliori, nel pieno possesso di tutte le sue attrattive, e questo le era tanto necessario per quello che stava per avvenire. Sentiva in sé la calma esteriore e la libera grazia dei movimenti. 

Alle sette e mezzo, appena discesa in salotto, il cameriere annunciò: ``Konstantin Dmitric Levin''. La principessa era ancora in camera sua e il principe non era uscito fuori. ``Ci siamo'' pensò Kitty, e tutto il sangue le affluì al cuore. Nel guardarsi allo specchio ebbe paura del proprio pallore. 

Ormai sapeva con certezza che egli era venuto prima proprio per trovarla sola e farle la sua proposta di matrimonio. E allora soltanto, per la prima volta, la cosa le apparve sotto un aspetto completamente nuovo, diverso. Ora soltanto lei capiva che la questione non riguardava lei sola: con chi sarebbe stata felice e chi amava, ma che in quel momento lei avrebbe dovuto offendere un uomo a cui voleva bene. E offenderlo crudamente\ldots{} Perché? Perché lui, povero caro, l'amava, era innamorato di lei. Ma non c'era nulla da fare; così doveva andare. 

``Dio mio, e dovrò dirglielo proprio io? - pensò. - E che cosa gli dirò? Gli dirò forse che non gli voglio bene? Ma questo non è vero! Che gli dirò allora? Dirò che amo un altro. No, non è possibile. Allora me ne vado via\ldots{}''. 

Si era già accostata alla porta, quando udì il passo di lui. ``No, non è onesto. Ma perché ho paura? Non ho fatto nulla di male. Sarà quel che sarà. Dirò la verità. E poi con lui non ci si può sentire impacciati. Eccolo'' si disse vedendo tutta la sua forte e timida figura con gli occhi scintillanti, rivolti verso di lei. Ella lo guardò diritto nel viso, quasi supplicandolo di farle grazia, e gli porse la mano. 

- Son venuto prima del tempo, mi pare, troppo presto - disse lui guardando il salotto vuoto. E accortosi che le sue previsioni si erano avverate, che cioè nulla gli impediva di dichiararsi, si rabbuiò in viso. 

- Oh, no - disse Kitty e sedette al tavolo. 

- Ma io volevo proprio questo, trovarvi sola - cominciò senza sedersi e senza guardarla per non perder coraggio. 

- La mamma viene subito. Ieri s'è stancata molto. Ieri\ldots{} 

Parlava senza saper lei stessa quello che pronunciavano le sue labbra e senza staccare da lui il suo sguardo supplice e carezzevole. 

Egli la guardò; ella arrossì e tacque. 

- Vi ho detto che non sono venuto per restar molto\ldots{} che questo dipende da voi\ldots{} 

Ella chinava sempre più la testa, non sapendo ella stessa che cosa avrebbe risposto a quello che stava per avverarsi. 

- Che ciò dipende da voi - ripeté lui. - Io volevo dirvi\ldots{} Per questo son venuto\ldots{} che voi\ldots{} siate mia moglie! - esclamò non sapendo egli stesso cosa diceva, ma sentiva che il peggio era stato detto; si fermò e la guardò. 

Lei respirava con affanno, senza guardarlo. Provava un certo incantamento. L'anima sua era come gonfia di felicità. Non credeva che in nessun modo il rivelarsi dell'amore di lui potesse produrle un'impressione così intensa. Ma questo durò un attimo solo. Si ricordò di Vronskij. Alzò su Levin i suoi cari occhi sinceri e, vedendo il viso disperato di lui, rispose in fretta: 

- Questo non può essere, perdonatemi. 

Come gli era stata vicina un minuto prima, tanto importante per la sua vita! E come ora gli si faceva estranea e lontana! 

- Non poteva essere altrimenti - disse lui, senza guardarla. 

S'inchinò e fece per andarsene. 

\capitolo{XIV}\label{xiv} 

Ma proprio in quel momento entrò la principessa. Quando li vide soli e sconvolti, il terrore le si espresse in viso. Levin si inchinò e non disse nulla. Kitty taceva senza alzar gli occhi. ``Grazie a Dio, ha detto di no'' pensò la madre, e il viso le si schiarì nel consueto sorriso col quale accoglieva gli ospiti il giovedì. Sedette e incominciò a interrogare Levin sulla sua vita in campagna. Egli sedette di nuovo in attesa degli ospiti per andarsene inavvertito. 

Dopo cinque minuti entrò un'amica di Kitty che si era sposata l'inverno prima, la contessa Nordston. 

Era una donna secca e giallognola, con occhi neri scintillanti, malaticcia e nervosa. Voleva bene a Kitty e il suo affetto per lei, come accade sempre alle donne maritate che vogliono bene a una ragazza, si esprimeva nel desiderio di trovarle marito secondo il proprio ideale di felicità; desiderava perciò darla a Vronskij. Levin, che aveva incontrato da loro al principio dell'inverno, le era sempre riuscito antipatico. Ogni volta che lo vedeva, la sua occupazione favorita consisteva nel prenderlo in giro. 

``Mi piace quando mi guarda dall'alto della sua superiorità, o interrompe la sua saggia conversazione con me, perché sono una sciocca o quando ancora si benigna di scendere fino a me. Questo mi piace: che discenda. Sono molto contenta che non mi possa sopportare'' diceva di lui. 

Aveva ragione, perché realmente Levin non la poteva sopportare e la disprezzava per tutto quello di cui lei andava orgogliosa e vaga: per quel suo nervosismo, per quel suo sottile spregio e per quella sua indifferenza verso tutto ciò che è comune e quotidiano. 

Fra la Nordston e Levin si erano perciò venuti a stabilire quei rapporti, frequenti nel gran mondo, per cui due persone, pur rimanendo esteriormente in rapporti di cortesia, si disprezzano reciprocamente a tal punto da non riuscire non solo a trattarsi con serietà, ma da non sentirsi neppure offese l'una dall'altra\ldots{} 

La contessa Nordston investì subito Levin. 

- Ah, Konstantin Dmitric. Siete venuto di nuovo in questa nostra depravata Babilonia - disse dandogli la sua piccola mano giallognola e ripetendo le parole dette da lui in una certa occasione al principio dell'inverno, che cioè Mosca era una Babilonia. - Che forse Babilonia si è messa sulla giusta via, o siete voi ad esservi pervertito? - soggiunse, guardando Kitty con un sorriso. 

- Sono molto lusingato, contessa, che ricordiate le mie parole - rispose Levin che si era affrettato a rimettersi, entrando subito, per abitudine, nei suoi rapporti di scherzosa inimicizia con la contessa Nordston. - Evidentemente, esse hanno fatto molto effetto su di voi. 

- Oh, e come! Io prendo nota di tutto. Ebbene, Kitty, hai pattinato di nuovo? 

E cominciò a parlare con Kitty. Per quanto poco conveniente fosse ora per Levin andarsene, tuttavia gli era più facile compiere questa sgarberia che rimaner tutta la serata a osservare Kitty che ogni tanto lo guardava di sfuggita ed evitava di incontrare il suo sguardo. Stava per alzarsi, quando la principessa, avendo notato il suo silenzio, gli rivolse la parola: 

- Vi trattenete a lungo a Mosca? Perché voi, mi pare, vi occupate degli arbitrati del consiglio distrettuale e non potete assentarvi a lungo. 

- No, principessa, non mi occupo più del consiglio distrettuale - disse. - Sono venuto per pochi giorni. 

``Ha qualcosa di speciale - pensò la Nordston, osservando il viso di lui serio e severo. - Chi sa perché non si ingolfa nei suoi ragionamenti. Ma io ce lo porterò. Mi piace immensamente di fargli fare la figura dello sciocco davanti a Kitty, e ci riuscirò''. 

- Konstantin Dmitric - gli disse - spiegatemi, vi prego, voi sapete tutto ciò, che cosa mai significa che da noi, nel villaggio di Kaluga, i contadini e perfino le donne si son mangiati tutto quello che avevano e a noi non hanno dato proprio un bel nulla. Che significa? Voi non fate che cantar le lodi dei contadini. 

In quel momento entrò nella stanza una signora e Levin si alzò. 

- Perdonatemi, contessa, ma io davvero non so nulla di questo e non posso dirvene nulla - disse, e si mise a guardare un ufficiale che era entrato dopo la signora. 

``Deve essere Vronskij'' pensò Levin e, per convincersene, guardò Kitty. Ella aveva fatto appena in tempo a guardare Vronskij e s'era poi girata verso Levin. E da questo solo sguardo dei suoi occhi, involontariamente illuminati, Levin capì che ella amava quell'uomo, e lo capì così fermamente come se glielo avesse detto lei a parole. Ma che uomo era mai? 

Adesso - fosse bene o fosse male - Levin non poteva non rimanere: doveva sapere che uomo era mai quello che lei amava. 

Ci sono delle persone che, incontrando un rivale fortunato in una qualsiasi cosa, sono subito pronte a distogliere lo sguardo da tutto ciò che c'è di buono in lui e a vederne solo le manchevolezze; vi sono persone, invece, che desiderano trovare nel rivale fortunato proprio quelle qualità con le quali costui ha vinto loro, e vedono in lui, con una punta di dolore al cuore, solo le buone qualità. Levin apparteneva a queste ultime persone. Ma a lui non fu difficile trovare il lato buono e attraente di Vronskij; questo anzi gli saltò subito agli occhi. Vronskij era di statura media, ma di costituzione forte, bruno, con un viso simpatico e bello, straordinariamente calmo e deciso. Nel viso e nella persona di lui, dai capelli neri dal taglio corto e dal mento rasato di fresco fino all'uniforme ampia e nuova fiammante, tutto era semplice e nello stesso tempo elegante. Cedendo il passo alla signora che entrava, Vronskij si avvicinò alla principessa e poi a Kitty. 

Nel momento in cui si avvicinò a lei, i suoi begli occhi brillarono di una particolare tenerezza e con un impercettibile sorriso felice di trionfo discreto (così parve a Levin), chinandosi con rispetto verso di lei, le tese la mano non grande, ma larga. 

Dopo aver salutato tutti e dopo aver detto qualche parola, sedette senza guardare neppure una volta Levin che non staccava gli occhi da lui. 

- Permettete che vi presenti - disse la principessa indicando Levin. - Konstantin Dmitric Levin. Il conte Aleksej Kirillovic Vronskij. 

Vronskij si alzò e, guardando cordialmente Levin negli occhi, gli strinse la mano. 

- Questo inverno dovevo pranzare con voi, mi pare - disse, sorridendo del suo semplice e aperto sorriso. - Ma voi partiste all'improvviso per la campagna. 

- Konstantin Dmitric disprezza e odia la città e tutti noi cittadini - disse la contessa Nordston. 

- Si vede proprio che le mie parole vi hanno fatto effetto, per ricordarle tanto - disse Levin; ma pensando di averlo già detto prima, arrossì. 

Vronskij guardò Levin e la contessa Nordston e sorrise. 

- E voi siete sempre in campagna? - chiese. - Ci si annoia, penso, d'inverno. 

- No, non ci si annoia, quando si ha da fare; e poi anche a star da soli con se stessi non ci si annoia - rispose aspro Levin. 

- A me piace la campagna - disse Vronskij, avendo notato, ma fingendo di non rilevare, il tono di Levin. 

- Ma spero, conte, che non acconsentiate a vivere sempre in campagna - disse la contessa Nordston. 

- Non so, non ho mai provato a lungo. Ho provato un sentimento strano però - soggiunse. - Non ho mai provato tanta nostalgia per la campagna, per la campagna russa con i lapti e con i muziki, come dopo aver vissuto un inverno intero a Nizza con mia madre. Nizza di per sé è noiosa e anche Napoli, Sorrento, sono belle solo per poco tempo. E proprio là ci si ricorda intensamente della Russia e in particolare della campagna russa. Esse sono quasi come\ldots{} 

Egli parlava rivolto a Kitty e a Levin, passando dall'una all'altro il suo tranquillo sguardo cordiale; diceva, evidentemente, quel che gli veniva in testa. 

Avendo notato che la contessa Nordston voleva dire qualcosa, non finì ciò che aveva cominciato e prese ad ascoltarla attentamente. 

La conversazione non venne meno neppure un attimo, così che la vecchia principessa che aveva sempre di riserva, in caso fossero venuti a mancare gli argomenti, i due pezzi forti dell'istruzione classica o tecnica e del servizio militare obbligatorio, non ebbe bisogno di tirarli fuori, e la Nordston non ebbe modo di stuzzicare Levin. 

Levin avrebbe voluto entrare nella conversazione generale, ma non gli riusciva; e dicendo a se stesso ogni minuto: ``è ora d'andar via'' non se ne andava, come aspettando qualcosa. 

La conversazione si era orientata intanto verso i tavoli che girano e gli spiriti, e la Nordston, che credeva allo spiritismo, cominciò a raccontare i prodigi che aveva visto. 

- Ah, contessa, portatemi ad ogni costo, per amor di Dio, portatemi da loro! Io non ho mai visto nulla di straordinario, pur cercandolo dappertutto - disse sorridendo Vronskij. 

- Va bene, per sabato prossimo - rispose la contessa Nordston. - Ma voi, Konstantin Dmitric, ci credete? - chiese a Levin. 

- Perché me lo chiedete? Sapete già la mia risposta. 

- No, io voglio sentire la vostra opinione. 

- La mia opinione è semplicemente questa - rispose Levin - che questi tavolini che girano dimostrano che la cosiddetta società colta non è al di sopra dei contadini. Questi credono al malocchio, alla fattura e ai sortilegi, e noi\ldots{} 

- Dunque, non ci credete? 

- Non posso crederci, contessa. 

- Ma se ho visto con i miei occhi? 

- Anche le contadine raccontano di aver veduto con i loro occhi gli spiriti. 

- Così voi pensate che io non dica il vero. 

E rise senza allegria. 

- Ma no, Maša, Konstantin Dmitric dice che non ci può credere - disse Kitty, arrossendo per Levin, e questi lo capì e, irritatosi ancor più, voleva rispondere; ma Vronskij col suo sorriso aperto, cordiale, venne subito in aiuto della conversazione che minacciava di farsi spiacevole. 

- Voi non ne ammettete per nulla la possibilità? - chiese. - Perché mai? Noi ammettiamo l'esistenza dell'elettricità che pure non conosciamo; perché allora non potrebbe esistere una nuova forza ancora sconosciuta, e noi che\ldots{} 

- Quando fu scoperta l'elettricità - interruppe pronto Levin - fu scoperto soltanto un fenomeno, e non si sapeva da che cosa derivasse e che cosa producesse; e passarono secoli prima che si pensasse alla sua applicazione. Gli spiritisti, invece, hanno cominciato dalla constatazione che i tavolini scrivono e che gli spiriti vanno loro a far visita, e solo dopo si son messi a parlare di una certa forza sconosciuta. 

Vronskij ascoltava, come del resto ascoltava tutti sempre, attentamente Levin, interessandosi alle sue parole. 

- Sì, ma gli spiritisti dicono: noi non sappiamo qual forza sia questa, ma è una forza, ed ecco in quali condizioni agisce. E che gli scienziati scoprano in che cosa consiste questa forza. No, io non vedo perché questa non possa essere una nuova forza, se essa\ldots{} 

- E perché - interruppe Levin - nel campo dell'elettricità, ogniqualvolta sfregate della resina contro della lana, si manifesta un determinato fenomeno; mentre qui non sempre si manifesta, dunque non si tratta di un fenomeno naturale. 

Vronskij, accorgendosi che la conversazione stava per prendere un tono troppo serio per un salotto, non replicò e, cercando di mutar argomento, sorrise allegramente e si rivolse alle signore. 

- Su, proviamo subito, contessa - cominciò; ma Levin voleva finire di esporre quello che pensava. 

- Io penso - continuò - che questo sistema degli spiritisti di spiegare i loro prodigi con la trovata della forza nuova, sia quanto mai infelice. Essi parlano arditamente di forza spirituale e vogliono poi sottoporre questa a un'esperienza materiale. 

Tutti aspettavano che egli smettesse di parlare e lui lo sentiva. 

- Ma io penso che sareste un ottimo medium - disse la contessa Nordston - in voi c'è un certo che di esaltato. 

Levin aprì la bocca, volle dire qualcosa, arrossì e tacque. 

- Su, principessina, proviamo i tavoli, per favore - disse Vronskij. - Voi permettete, principessa? 

E Vronskij cominciò a cercar con gli occhi un tavolino. 

Kitty si alzò dal tavolo e, nel passare accanto a Levin, i suoi occhi si incontrarono con quelli di lui. Con tutta l'anima ne aveva pena, tanto più che era lei la causa della sua infelicità. ``Se mi si può perdonare, perdonatemi - diceva il suo sguardo. - Sono così felice''. 

``Odio tutti, e voi e me stesso'' rispondeva lo sguardo di Levin; e proprio in quel momento egli afferrò il cappello. Ma non era destino che dovesse andar via. Mentre gli altri volevano disporsi attorno al tavolino e Levin cercava di andarsene, entrò il principe e, salutate le signore, si rivolse a Levin. 

- Ah - cominciò festoso - è un pezzo che sei qua? Non lo sapevo neppure. Son contento di vedervi, molto. 

Il vecchio principe parlava a Levin a volte col tu, a volte col voi. Lo abbracciò e, parlando con lui, non si accorse di Vronskij che s'era alzato e aspettava tranquillamente che il principe gli rivolgesse la parola. 

Kitty sentiva che, dopo quello che era successo, l'espansione del padre doveva riuscire penosa a Levin. Ma notò pure con quanta freddezza suo padre rispondesse finalmente all'inchino di Vronskij, e come Vronskij guardasse il principe con affettuosa perplessità, cercando di capire come e perché si potesse essere maldisposti verso di lui. E arrossì. 

- Principe, lasciateci Konstantin Dmitric, - disse la contessa Nordston. - Vogliamo fare una prova. 

- Quale prova? Quella di far girare i tavolini? Su, scusatemi, signore e signori, ma per me è più divertente giocare all'anellino - disse il vecchio principe, guardando Vronskij e indovinando che era stato lui a organizzare la cosa. - Nell'anellino, ancora ancora, c'è un certo senso. 

Vronskij guardò con sorpresa il principe coi suoi occhi fermi e, sorridendo appena, cominciò subito a parlare con la Nordston del grande ballo della settimana seguente. 

- Spero che ci verrete - disse rivolto a Kitty. 

Non appena il vecchio principe si allontanò da lui, Levin uscì inosservato, riportando, quale ultima impressione della serata, il viso sorridente e felice di Kitty che rispondeva alla domanda di Vronskij a proposito del ballo. 

\capitolo{XV}\label{xv} 

Quando la serata fu finita, Kitty raccontò alla madre il suo colloquio con Levin, e, malgrado la pena che provava per lui, la rallegrava l'idea di aver avuto una ``domanda di matrimonio''. Non aveva nessun dubbio di non essersi regolata così come conveniva. Ma a letto, per molto tempo, non poté prendere sonno. Un'unica immagine la perseguitava ostinata. Era il viso di Levin con le sopracciglia aggrottate e gli occhi buoni che guardavano di sotto in su, scoraggiati e tristi, mentre, in piedi, ascoltava suo padre e guardava lei e Vronskij. E provò tanta pena per lui che le vennero le lacrime agli occhi. Ma allora pensò subito a quegli col quale lo aveva cambiato. Ricordò con vivezza il viso maschio di lui, la calma dignitosa e la benevolenza che emanavano in ogni suo gesto verso tutti; ricordò l'amore per lei dell'uomo che amava e la gioia le tornò nell'animo e con un sorriso di felicità poggiò la testa sul guanciale. ``Che pena, che pena, ma che farci? La colpa non è mia'' si andava dicendo; eppure una voce interiore le diceva il contrario. Di che cosa provasse rimorso - d'aver attratto a sé Levin o di averlo respinto - non sapeva. Ma la felicità sua era avvelenata dal dubbio. ``Signore abbi pietà, Signore abbi pietà'' diceva fra sé e sé finché si addormentò. 

Intanto giù, nello studio del principe, si svolgeva una di quelle scenate frequenti fra i genitori, a proposito della figlia preferita. 

- Ecco, ecco cosa c'è - gridava il principe agitando le braccia e incrociando subito i risvolti della vestaglia di vaio. - C'è che voi non avete né orgoglio né dignità, c'è che disonorate, rovinate la figliuola con questo stupido e indegno modo di cercarle marito. 

- Ma abbiate pazienza, per amor di Dio, principe, che ho fatto mai? - diceva la principessa, quasi piangendo. 

Dopo la conversazione con la figlia, era venuta dal principe a salutarlo, felice e soddisfatta e, pur non avendo intenzione di parlargli della proposta di Levin e del rifiuto di Kitty, aveva accennato al marito la faccenda di Vronskij che le sembrava del tutto definita, non appena fosse arrivata la madre di lui. E proprio a questo punto il principe aveva preso fuoco, e si era messo a gridare parole sconvenienti. 

- Che cosa avete fatto? Ecco cosa: in primo luogo avete adescato un giovanotto e tutta Mosca ne parlerà; e a ragione. Se volete dare una serata, invitate pure chi volete, ma non questi fidanzatelli prescelti. Invitateli pure tutti questi moscardini - così il principe chiamava i giovani brillanti di Mosca - chiamate pure uno strimpellatore e fate pure ballare, ma non mi mettete insieme, come avete fatto questa sera, tutti questi fidanzatelli. A me veder questo, fa schifo, schifo, e ci siete riuscita voi a far girar la testa alla ragazza. Levin è mille volte migliore. Questo invece è un cascamorto di Pietroburgo; li fanno a macchina questi elegantoni, son tutti d'uno stampo, e son tutti\ldots{} brodaglia. E fosse anche un principe di sangue, mia figlia non ha bisogno di nessuno! 

- Ma che cosa ho mai fatto io? 

- Questo, questo\ldots{} - gridò con rabbia il principe. 

- Lo so che a dar retta a te - interruppe la principessa - noi non dovremmo mai dar marito a nostra figlia. Ma se è così, meglio allora ritirarsi in campagna. 

- Eh sì che è meglio là. 

- Ma dimmi, che forse sono io che li adesco? Io non li attiro per nulla. Ma se un giovane, un giovane che ha tutte le qualità, s'innamora, e lei mi pare\ldots{} 

- Sì, ecco, vi pare! E se lei per l'appunto si innamorasse e lui pensasse a sposarsi tanto quanto me? Oh, che non lo vedano i miei occhi!\ldots{} ``Ah, lo spiritismo, ah, Nizza, ah, il ballo!''. - E il principe, immaginando di rifare il verso a sua moglie, faceva una riverenza ad ogni parola. 

- Ecco, quando avremo fatta l'infelicità di Katen'ka, quando si sarà davvero messa in testa\ldots{} 

- Ma perché lo pensi? 

- Io non lo penso, lo so; per questo noi uomini abbiamo gli occhi per vedere e non così le donnicciuole. Io vedo, da una parte, un uomo che ha intenzioni serie, Levin; e dall'altro un gallinaccio fanfarone come questo qua, che vuole soltanto divertirsi. 

- Eh già, ormai ti sei messo in testa certe cose\ldots{} 

- Ecco, te lo ricorderai, ma tardi, come è stato per Dašen'ka. 

- Su, va bene, non ne parliamo più - lo fermò la principessa, ricordandosi di Dolly infelice. 

- E va bene, addio! 

E fattisi scambievolmente la croce e baciatisi, i coniugi si separarono, sentendo, però, che ognuno era rimasto nella propria convinzione. 

La principessa, che prima era fermamente convinta che quella serata avrebbe deciso la sorte di Kitty e che non si dovevano avere più dubbi sulle intenzioni di Vronskij, era in questo momento turbata dalle parole del marito. E tornata in camera sua, proprio alla stessa maniera di Kitty, col terrore di un avvenire così incerto, ripeté parecchie volte in cuor suo:``Signore abbi pietà, Signore abbi pietà, Signore abbi pietà!''. 

\capitolo{XVI}\label{xvi} 

Vronskij non aveva conosciuto mai la vita di famiglia. Sua madre in gioventù era stata una brillante donna di mondo, e aveva avuto, durante la sua vita coniugale, e specialmente dopo, molte avventure note a tutta la società. Di suo padre quasi non si ricordava, ed egli stesso era stato educato al corpo dei paggi. 

Uscito giovanissimo dalla scuola, brillante ufficiale, si era trovato subito nella carreggiata comune a tutti i facoltosi ufficiali di Pietroburgo. Sebbene frequentasse di tanto in tanto la società pietroburghese, i suoi interessi amorosi ne erano tutti al di fuori. Dopo la vita di Pietroburgo, lussuosa e dissoluta, a Mosca aveva provato per la prima volta l'incanto di avvicinarsi ad una graziosa ed ignara fanciulla della società, la quale aveva preso ad amarlo. Non gli era venuto neppure in mente che potesse esserci qualcosa di poco onesto nei suoi rapporti con Kitty. Nelle feste ballava soprattutto con lei, ne frequentava la casa. Le diceva quello che comunemente si dice in società: una sciocchezza qualsiasi, alla quale, senza volere, dava un significato particolare per lei. Tuttavia, pur non dicendo nulla che non fosse conveniente dire in presenza di tutti, avvertiva ch'ella sempre più subiva il suo fascino, e più egli s'accorgeva di questo più se ne compiaceva, e il suo affetto per lei diveniva sempre più tenero. Non sapeva che questo suo modo di agire nei riguardi di Kitty avrebbe potuto chiaramente essere definito un tentativo di adescare una ragazza senza avere alcuna intenzione di sposarla, e che questo adescamento era una delle cattive azioni dei giovani mondani come lui. Gli sembrava d'essere stato il primo a scoprire una simile soddisfazione e godeva della propria scoperta. 

S'egli avesse potuto ascoltare ciò che dicevano i genitori di Kitty quella sera, se egli avesse potuto mettersi dal punto di vista della famiglia e pensare che Kitty sarebbe stata infelice se egli non l'avesse sposata, si sarebbe molto sorpreso e non ci avrebbe creduto. Non avrebbe potuto credere che quello che procurava un piacere così grande e buono a lui e specialmente a lei, potesse essere un male. Ancor meno avrebbe pensato di doversi sposare. 

Il matrimonio non gli si era presentato mai come una possibilità. Non solo non amava la vita di famiglia, ma nella famiglia, e particolarmente nella figura del marito, egli vedeva, secondo l'opinione dell'ambiente di scapoli in cui viveva, qualcosa di estraneo, di ostile, e soprattutto di ridicolo. Ma pur senza sospettare la conversazione dei genitori di Kitty, Vronskij, uscendo quella sera da casa Šcerbackij, sentì che il segreto legame sentimentale che esisteva tra lui e Kitty si era così saldamente rafforzato, ch'egli doveva prendere una decisione. Ma quale precisamente non sapeva immaginare. 

``Anche questo è delizioso - pensava tornando da casa Šcerbackij, riportandone, come sempre, un senso di piacevole purità e freschezza dovuto forse, in parte, al fatto di non aver fumato per tutta la sera; ed insieme a questo un nuovo senso di tenerezza dinanzi all'amore di Kitty. - Anche questo è delizioso, che niente sia stato detto fra me e lei; ma ci siamo talmente intesi in quella invisibile conversazione fatta di sguardi e di toni di voce che oggi, in maniera più chiara che mai, ella mi ha detto che mi ama. E così teneramente, con tanta semplicità e soprattutto con fiducia. Io stesso mi sento migliore, più puro. Sento di avere un cuore e che c'è molto di buono in me. Quei cari occhi innamorati! Quando ha detto: e molto\ldots{}E allora? E allora nulla. Io sto bene e lei pure sta bene''. E si mise a pensare dove finir la serata. 

Passò in rassegna tutti i luoghi dove sarebbe potuto andare. ``Al club? Una partita a bazzica, lo champagne con Ignatov? No, non ci vado. Allo Château des fleurs e trovarci Oblonskij, le canzonette e il can can? No, m'è venuto a noia. Ecco, proprio perché mi piacciono gli Šcerbackij è segno che divento migliore. Andrò a casa''. Andò direttamente all'albergo Dussau nella sua camera, si fece servir la cena e, spogliatosi, fece appena in tempo a posar la testa sul guanciale che s'addormentò d'un sonno pesante e tranquillo come sempre. 

\capitolo{XVII}\label{xvii} 

Il giorno dopo, alle undici del mattino, Vronskij andò alla stazione della ferrovia di Pietroburgo a rilevare la madre; e il primo viso in cui si imbatté sui gradini della scalinata principale fu Oblonskij che aspettava la sorella con quello stesso treno. 

- Oh, eccellenza! - gridò Oblonskij - tu qua? a prendere chi? 

- Io? a prendere la mamma - rispose Vronskij, sorridendo come tutti quelli che incontravano Oblonskij, e stringendogli la mano salì con lui la scalinata. - Deve arrivare oggi da Pietroburgo. 

- E io ti ho aspettato fino alle due! Dove sei andato dopo gli Šcerbackij? 

- A casa - rispose Vronskij. - A dir la verità, stavo così bene ieri sera dopo casa Šcerbackij che non ho avuto voglia di andare in nessun altro posto. 

- ``Conosco i cavalli focosi da certi loro segni, conosco i giovani innamorati dagli occhi'' - declamò Stepan Arkad'ic, proprio come aveva detto il giorno prima a Levin. 

Vronskij sorrise con l'aria di non negare, ma subito cambiò discorso. 

- E tu chi aspetti? - domandò. 

- Io? Una bella donna - disse Oblonskij. 

- Bene! 

- Honny soit qui mal y pense! Mia sorella Anna. 

- Ah, la Karenina. 

- La conosci, vero? 

- Mi pare di conoscerla. Forse no. A dire il vero, non ricordo - rispondeva distrattamente Vronskij, immaginandosi al nome di Karenina qualcosa di borioso e noioso. 

- Ma Aleksej Aleksandrovic, il mio famoso cognato, lo conosci probabilmente. Tutti lo conoscono. 

- Lo conosco infatti di fama e di vista. So che è molto intelligente, uno scienziato, qualcosa di superno\ldots{} Ma tu lo sai, questo non rientra nella mia\ldots{} not in my line - disse Vronskij. 

- Già, è un uomo molto interessante; un po' conservatore, ma una brava persona. 

- Be', tanto meglio per lui - disse Vronskij sorridendo. - Ah, tu sei qui - disse rivolto al servitore della madre, un vecchio di alta statura, che stava accanto alla porta. - Entra qua. 

Vronskij, oltre la simpatia che aveva, come tutti avevano, per Stepan Arkad'ic, si sentiva legato a lui in quell'ultimo tempo per il fatto che in mente sua lo associava a Kitty. 

- Ebbene, domenica, facciamo il pranzo per la diva? - gli disse prendendolo sotto braccio con un sorriso. - Io raccoglierò le quote. Ah, ieri hai conosciuto il mio amico Levin? - chiese Stepan Arkad'ic. 

- E come! Ma è andato via un po' presto. 

- È un caro ragazzo - continuò Oblonskij - non è vero? 

- Io non capisco - rispose Vronskij - perché in tutti i moscoviti, esclusi naturalmente quelli con cui parlo - intercalò scherzosamente - vi sia qualcosa di duro. Non so perché si inalberano sempre, si arrabbiano come se volessero far sempre sentire qualcosa\ldots{} 

- È così, è vero, è\ldots{} - disse ridendo allegramente Stepan Arkad'ic. 

- Arriva presto? - chiese Vronskij a un ferroviere. 

- È già partito dall'ultima stazione - rispose il ferroviere. 

L'avvicinarsi del treno si notava sempre più per il movimento dei preparativi nella stazione, per il correre dei facchini, per l'apparire dei gendarmi e dei ferrovieri e per l'arrivo di coloro che aspettavano. Attraverso la nebbia gelida si vedevano gli operai con le giubbe corte di pelliccia, le scarpe morbide di feltro, che passavano attraverso gli scambi delle curve delle linee. Si udiva il fischio di una locomotiva su rotaie lontane, e l'incedere di qualcosa di pesante. 

- No - disse Stepan Arkad'ic il quale aveva una gran voglia di raccontare a Vronskij le intenzioni di Levin nei riguardi di Kitty. - No, tu non hai apprezzato al giusto punto il mio Levin. È un uomo molto nervoso e a volte antipatico, è vero, ma in compenso è molto caro. È una natura, così onesta, così leale, e ha un cuore d'oro. Ma ieri vi erano delle ragioni particolari - continuò Stepan Arkad'ic con un sorriso d'intesa, dimenticando completamente la sincera simpatia che aveva provato il giorno prima per il suo amico e sentendone ora una simile, solo che per Vronskij. - Sì, vi era una ragione per la quale egli poteva diventare particolarmente felice o particolarmente infelice. 

Vronskij si fermò e chiese franco: 

- Cos'è, cos'è mai? Che forse ieri ha fatto domanda di matrimonio alla tua belle-soeur? 

- Può darsi - disse Stepan Arkad'ic. - M'è parso di capire qualcosa di simile, ieri. Già, se n'è andato via presto ed era anche di cattivo umore, deve essere stato così. È innamorato da tanto tempo e mi fa tanta pena. 

- Eh, già! Io penso, del resto, che lei può aspirare a un partito migliore - disse Vronskij e, raddrizzando il busto, si mise di nuovo a camminare. - Del resto, non lo conosco - soggiunse. - Già, deve essere una situazione penosa. Proprio per questo la maggioranza degli uomini preferisce far conoscenza con le donnine allegre. In questo caso un insuccesso dimostra solo che non hai avuto abbastanza quattrini, nell'altro, invece, è messo in giuoco il tuo onore. Ma ecco il treno. 

Infatti la locomotiva fischiava già. Dopo qualche secondo la piattaforma tremò e, sbuffando del vapore appesantito dal gelo, la locomotiva avanzò con lo stantuffo che si piegava e si distendeva lentamente e ritmicamente, e con il macchinista tutto imbacuccato e ricoperto di brina che salutava; e poi dietro al tender, scotendo sempre più lentamente e sempre più forte la banchina, passò il bagagliaio con un cane che guaiva; ed infine, traballando prima di fermarsi, avanzarono le carrozze dei passeggeri. 

Un capotreno aitante, fischiando, saltò giù mentre il treno era ancora in corsa, e dietro di lui cominciarono a scendere, uno ad uno, i viaggiatori impazienti: un ufficiale della guardia che si teneva dritto e guardava severamente attorno a sé, un piccolo mercante inquieto che sorrideva allegramente tenendo in mano una borsa, un contadino con un sacco sulle spalle. 

Vronskij, dritto accanto a Oblonskij, guardava le vetture e quelli che ne venivano fuori, e s'era completamente scordato di sua madre. Quello che aveva saputo proprio allora di Kitty lo eccitava e rallegrava. Il suo petto involontariamente si raddrizzava e gli occhi gli brillavano. Si sentiva vincitore. 

- La contessa Vronskaja è in questo scompartimento - disse il capotreno aitante, accostandosi a Vronskij. 

Le parole del capotreno lo scossero e lo costrinsero a ricordarsi della madre e dell'imminente incontro con lei. Egli, in fondo, non stimava sua madre e, senza rendersene conto, non l'amava neppure, sebbene, per l'ambiente in cui viveva e per la propria educazione, non sapeva immaginare altri rapporti verso di lei che quelli propriamente sottomessi e rispettosi, anzi tanto più sottomessi e rispettosi quanto meno intimamente la stimava ed amava. 

\capitolo{XVIII}\label{xviii} 

Vronskij entrò nella vettura dietro al capotreno e all'ingresso dello scompartimento si fermò per cedere il passo a una signora che ne usciva. Con l'intuito abituale dell'uomo di mondo, Vronskij ne rilevò l'appartenenza al gran mondo. Si scusò e stava per entrare, quando provò il bisogno di guardarla ancora una volta non perché era molto bella, non per quella eleganza e quella grazia modesta che apparivano da tutta la sua figura, ma perché nell'espressione piacente del viso, quando gli era passata accanto, c'era qualcosa di affettuoso e di dolce. Nel momento in cui si era voltato a guardarla, ella pure aveva girato il capo. I suoi occhi grigi, luminosi, che sembravano scuri per le sopracciglia folte, si fermarono attenti con un'espressione amichevole sul viso di lui, come se lo riconoscessero, e subito si portarono sulla folla che si avvicinava, cercando qualcuno. In questo breve sguardo Vronskij riuscì a notare una vivacità contenuta che le errava sul viso e balenava tra gli occhi lucenti e un riso appena percettibile che increspava le labbra vermiglie. Come se qualcosa di esuberante colmasse tanto il suo essere da esprimersi contro il suo volere, ora nella luce degli occhi, ora nel riso. Ella aveva deliberatamente attutito la luce degli occhi, ma questa luce, contro il suo volere, si era illuminata nel riso appena percettibile. 

Vronskij entrò nello scompartimento. Sua madre, una vecchietta asciutta dai riccioli e dagli occhi neri, socchiudeva le palpebre guardando il figlio e sorrideva lieve con le labbra sottili. Alzatasi dal sedile e porgendo la borsetta alla cameriera, tese la piccola mano asciutta al figlio e, sollevandogli la testa dalla mano, lo baciò. 

- Hai ricevuto il telegramma? Stai bene? Sia lodato Iddio. 

- Avete fatto buon viaggio? - disse il figlio sedendosi accanto a lei e prestando involontariamente ascolto alla voce femminile che gli giungeva da dietro la porta. Egli sapeva che era la voce della signora che aveva incontrato nell'entrare. 

- Io non sono d'accordo con voi - diceva la voce della signora. 

- È il punto di vista pietroburghese, signora. 

- Non pietroburghese, ma semplicemente femminile - rispondeva lei. 

- Permettetemi di baciare la vostra piccola mano. 

- A rivederci, Ivan Petrovic. E guardate se mio fratello è qui, e mandatemelo - disse la signora proprio sulla porta, ed entrò di nuovo nello scompartimento. 

- Ebbene, avete trovato vostro fratello? - disse la Vronskaja rivolgendosi alla signora. 

Vronskij allora si ricordò che era la Karenina. 

- Vostro fratello è qui - disse alzandosi in piedi. - Perdonatemi, non vi ho riconosciuto; ma già, la nostra conoscenza è stata così breve - disse Vronskij inchinandosi - che probabilmente voi non vi ricordate di me. 

- Oh, no - disse lei - vi avrei riconosciuto, perché con vostra madre, per tutto il viaggio, mi pare, abbiamo parlato soltanto di voi - disse, permettendo infine a quella vivacità che le urgeva di esprimersi nel riso. - Ma com'è che mio fratello non viene? 

- Va' a chiamarlo, Alëša - disse la vecchia contessa. 

Vronskij uscì sulla piattaforma e gridò: 

- Oblonskij, qui! 

Ma la Karenina non aspettò che il fratello si avvicinasse e, non appena lo vide, col suo passo leggero e deciso scese subito dalla vettura. E non appena il fratello le fu dappresso, con un movimento che stupì Vronskij per la grazia e la prontezza, circondò con il braccio sinistro il collo di Oblonskij, l'attirò a sé con mossa rapida e lo baciò forte. 

Vronskij, senza staccare gli occhi da lei, l'osservava e, senza saper lui stesso perché, sorrideva. Ma, ricordatosi che la madre aspettava, montò in vettura. 

- Non è vero che è molto carina? - disse la contessa. - Il marito l'ha fatta sedere qui accanto a me e io ne sono stata molto contenta. Abbiamo parlato tutto il viaggio. E ora, su, a te, mi si dice\ldots{}vous filez le parfait amour. Tant mieux, mon cher, tant mieux. 

- Io non so a che cosa alludiate, maman - rispose freddo il figlio. - Dunque, maman, andiamo. 

La Karenina entrò di nuovo nello scompartimento per salutare la contessa. 

- Ed eccoci qua, contessa; voi avete trovato vostro figlio e io mio fratello - disse gaia. - E così tutte le mie storie si sono esaurite; forse più avanti non ci sarebbe stato più nulla da raccontare. 

- Eh, no - disse la contessa prendendole una mano - io con voi farei il giro del mondo e non mi annoierei mai. Voi siete una di quelle donne gentili con le quali è piacevole parlare e tacere. E non vi preoccupate di vostro figlio, vi prego; è impossibile non separarsene mai. 

La Karenina stava immobile, mantenendosi ben dritta e i suoi occhi ridevano. 

- Anna Arkad'evna - disse la contessa spiegando al figlio - ha un bimbo di otto anni, mi pare, e non s'è mai staccata da lui, e ora si tormenta d'averlo lasciato. 

- Già, con la contessa abbiamo parlato tutto il tempo io del mio e lei del suo figliuolo\ldots{} - disse la Karenina e di nuovo il riso le illuminò il volto, un riso carezzevole che riguardava lui. 

- Probabilmente questo vi avrà annoiato - disse lui afferrando al volo la pallina di civetteria ch'ella gli aveva lanciato. Ma ella evidentemente non voleva proseguire la conversazione su questo tono e si rivolse alla vecchia contessa. 

- Vi ringrazio molto. Non mi sono neppure accorta come ho passato la giornata di ieri. A rivederci, contessa. 

- Addio, mia piccola amica - rispose la contessa. - Fatemi baciare il vostro bel visino. Vi dico così, semplicemente, da vecchia, che sono innamorata di voi. 

Per quanto usuale fosse questa frase la Karenina evidentemente ci credette di cuore, e se ne rallegrò. Arrossì, si chinò leggermente porgendo il viso alle labbra della contessa, si raddrizzò, e sempre con quel riso che le balenava fra le labbra e gli occhi, dette la mano a Vronskij. Egli strinse la piccola mano offertagli e si rallegrò come di una cosa particolare per quella stretta energica con la quale ella scosse ardita e forte la sua mano. Ella uscì col passo svelto che portava con così strana leggerezza il corpo assai pieno. 

- È molto carina - disse la vecchia signora. 

La stessa cosa pensava il figlio. Egli accompagnò con lo sguardo la graziosa figura finché non sparve e il sorriso gli rimase sul volto. Dal finestrino la vide accostarsi al fratello, mettergli una mano sul braccio e cominciare a parlargli con animazione di qualcosa che evidentemente non aveva nulla in comune con lui, Vronskij, e questo gli dette fastidio. 

- E allora, maman, state proprio bene? - ripeté lui volgendosi alla madre. 

- Bene, benissimo. Alexandre è stato molto gentile. E Marie è diventata bella. È molto interessante. 

E prese a raccontare quello che più di tutto le interessava: il battesimo del nipote per cui era andata a Pietroburgo, e la particolare benevolenza dello zar verso il figlio maggiore. 

- Ecco anche Lavrentij - disse Vronskij guardando dal finestrino. - Ora andiamo, se non vi spiace. 

Il vecchio maggiordomo che aveva viaggiato con la contessa venne a dire che tutto era pronto e la contessa si alzò per andare. 

- Andiamo, ora c'è poca gente - disse Vronskij. 

La cameriera afferrò una sacca e il cagnolino, il maggiordomo e un facchino presero le valigie. Vronskij offrì il braccio alla madre; ma mentre uscivano dalla vettura, a un tratto alcune persone dal viso spaventato passarono vicino correndo. Passò anche il capostazione col berretto dal colore vivace. Doveva essere successo qualcosa d'eccezionale. La gente del treno correva in senso inverso. 

- Cos'è? Cos'è? S'è gettato sotto! L'ha schiacciato!\ldots{} - si sentiva dire fra quelli che passavano. 

Stepan Arkad'ic e la sorella ch'egli aveva al braccio, anche loro coi visi spaventati, tornarono indietro e si fermarono accanto alla vettura. 

Le signore vi salirono, mentre Vronskij e Stepan Arkad'ic seguirono la folla per informarsi dei particolari della disgrazia. 

Un guardiano, forse ubriaco o forse troppo imbacuccato per il gran gelo, non aveva sentito il treno che retrocedeva ed era rimasto schiacciato. 

Ancor prima che Vronskij e Oblonskij fossero tornati, le signore avevano saputo tutti i particolari dal maggiordomo. 

Oblonskij e Vronskij avevano tutti e due visto il corpo deformato. Oblonskij soffriva visibilmente. Corrugava la fronte e sembrava stesse per piangere. 

- Ah, che orrore! Oh, Anna, se avessi visto! Ah, che orrore! - esclamava. 

Vronskij taceva e il suo bel viso era serio, ma perfettamente tranquillo. 

- Ah, se aveste visto, contessa - diceva Stepan Arkad'ic. - E la moglie è qui\ldots{} È uno strazio a vederla. S'è gettata sul corpo. Dicono che era lui solo a dar da mangiare a una famiglia enorme. Che orrore! 

- Non si può fare qualcosa per lei? - disse la Karenina con un bisbiglio agitato. 

Vronskij la guardò e uscì dallo scompartimento. 

- Vengo subito, maman - aggiunse, voltandosi indietro sulla porta. 

Quando rientrò, dopo pochi minuti, Stepan Arkad'ic parlava già con la contessa di una nuova cantante, ma la contessa guardava impaziente verso la porta in attesa del figlio. 

- Ora andiamo - disse Vronskij entrando. 

Uscirono insieme. Vronskij andava avanti con la madre. Dietro venivano la Karenina e il fratello. All'uscita, raggiuntolo, il capostazione si avvicinò a Vronskij. 

- Voi avete consegnato duecento rubli al mio aiutante. Vogliate precisare a chi li destinate. 

- Alla vedova - disse Vronskij alzando le spalle. - Non capisco che bisogno ci sia di chiederlo. 

- Li avete dati voi? - gridò da dietro Oblonskij e, stretto il braccio alla sorella, aggiunse: - Che caro, che caro! Non è vero che è un gran bravo ragazzo? I miei rispetti contessa. 

E lui e la sorella si fermarono alla ricerca della cameriera. 

Quando uscirono, la carrozza dei Vronskij era già andata via. Le persone che entravano parlavano ancora fra di loro di quello che era accaduto. 

- Ecco una morte terribile! - diceva un signore passando accanto. - Dicono che sia stato fatto in due pezzi. 

- Io penso invece che sia la migliore: in un attimo - osservò un altro. 

- Ma come, non prendono delle misure di sicurezza? - diceva un terzo. 

La Karenina sedette nella carrozza e Stepan Arkad'ic si accorse con sorpresa che le labbra le tremavano e che a stento tratteneva le lacrime. 

- Che c'è, Anna? - chiese quando si furono allontanati di un centinaio di sazeni. 

- Un cattivo presagio - disse lei. 

- Sciocchezze! - disse Stepan Arkad'ic. - Tu sei arrivata, questo è l'importante. Tu non puoi immaginare come io speri in te. 

- È molto che conosci Vronskij? - chiese lei. 

- Sì, forse lo sai, noi speriamo che sposi Kitty. 

- Sì? - disse piano Anna. - Suvvia, dimmi ora di te - aggiunse, scotendo la testa come per scacciar via materialmente qualcosa di superfluo e di fastidioso. - Dimmi delle tue cose. Ho avuto la lettera ed eccomi qua. 

- Sì, ogni speranza è in te - disse Stepan Arkad'ic. 

- Su, raccontami tutto. 

E Stepan Arkad'ic prese a raccontare. 

Giunti a casa, Oblonskij fece scendere la sorella, sospirò, le dette la mano e si diresse in ufficio. 

\capitolo{XIX}\label{xix} 

Quando Anna entrò nella stanza, Dolly stava nel salottino con un bimbo biondo e paffuto che fin d'ora assomigliava al padre, e gli risentiva la lezione di lettura francese. Il bambino leggeva, rigirandosi in mano e cercando di strappare al giubbotto un bottone che appena appena si reggeva. La madre aveva varie volte allontanato quella mano, ma la manina grassoccia tornava di nuovo al bottone. La madre alla fine staccò il bottone e se lo mise in tasca. 

- Fermo con le mai, Griša - disse, e si mise di nuovo alla coperta, suo vecchio lavoro al quale attendeva sempre nei momenti penosi e che ora eseguiva nervosamente, intrecciando il filo con le dita e contando le maglie. Benché avesse fatto dire al marito, il giorno prima, che l'arrivo della sorella non la riguardava, aveva preparato tutto per riceverla e aspettava con ansia la cognata. 

Dolly era schiantata dal dolore, ne era tutta divorata. Ma ricordava che Anna era la moglie di uno dei personaggi più importanti di Pietroburgo e una grande dame pietroburghese. E per questo, contrariamente a quello che aveva fatto dire al marito, non aveva dimenticato che sarebbe arrivata la cognata. ``Poi, in fondo, Anna non ha nessuna colpa - pensava. - Io non so altro di lei se non quanto si può dir di meglio, e nei miei riguardi ne ho sempre ricevuto affetto ed amicizia''. Però, per quanto ricordava, l'impressione da lei riportata a Pietroburgo dei Karenin, non era stata favorevole: non le era piaciuta la loro casa; c'era qualcosa di falso in quell'ambiente di vita familiare. ``Ma perché mai non riceverla? Che non le venga in mente di consolarmi, però! - pensava Dolly. - Tutte le consolazioni, le esortazioni e i perdoni, tutto questo l'ho già pensato e ripensato mille volte, e non serve a nulla''. 

Tutti quei giorni Dolly era stata sola coi bambini. Parlare della sua pena non voleva, e con quel dolore nel cuore parlare di cose indifferenti non le riusciva. Sapeva che in un modo o nell'altro avrebbe detto tutto ad Anna; e ora la rallegrava il pensiero di come l'avrebbe detto, ora l'irritava quel bisogno di raccontare la propria umiliazione a lei, sorella del marito, e sentirne frasi fatte di esortazione e di conforto. 

L'aspettava guardando l'orologio ogni momento, ma, come spesso accade, le sfuggì proprio quello in cui l'ospite giunse, così che non sentì il campanello. 

Udito il fruscio di vesti e di passi lievi già sulla porta, si voltò e sul viso tormentato si espresse involontariamente non la gioia, ma la sorpresa. Si alzò e abbracciò la cognata. 

- Come, già qui? - disse baciandola. 

- Dolly, come sono contenta di vederti! 

- Anch'io sono contenta - disse Dolly, sorridendo debolmente e cercando di indovinare dall'espressione del viso di Anna se sapeva o no. ``Probabilmente sa'' pensò, notando una certa compassione sul viso di Anna. - Su, andiamo, ti accompagno in camera tua - continuò, cercando di allontanare, per quanto possibile, il momento della spiegazione. 

- Questo è Griša? Dio, com'è cresciuto! - disse Anna e, baciatolo, senza staccare gli occhi da Dolly, si fermò e arrossì. - No, permettimi di restare qui. 

Si tolse lo scialle, il cappello e, avendovi impigliato una ciocca di capelli neri inanellati, scotendo la testa, liberò la capigliatura. 

- Come splendi di felicità e di salute! - disse Dolly quasi con invidia. 

- Io? Sì - disse Anna. - Dio mio, Tanja! La coetanea del mio Serëza - aggiunse rivolta alla bambina che era entrata di corsa. La prese in collo e la baciò. - Una bimba deliziosa! un amore! Fammeli vedere tutti. 

Nominava e ricordava non soltanto i nomi, ma gli anni, i mesi, i caratteri, le malattie di tutti loro, e Dolly non poteva non apprezzare tutto questo. 

- Su, allora, andiamo da loro - disse lei. - Vasja dorme ora, peccato! 

Dopo aver veduto i bambini, sedettero davanti al caffè, ormai sole, nel salotto. 

Anna prese il vassoio, ma poi lo scostò. 

- Dolly - disse - lui mi ha parlato. 

Dolly guardò fredda Anna. Si aspettava ora delle frasi convenzionali di simpatia, ma Anna non disse nulla di simile. 

- Dolly, cara - disse - io non voglio parlarti in suo favore, né consolarti; non si può. Ma ho pena di te, cara, ne ho pena con tutta l'anima! 

Dietro alle ciglia dei suoi occhi comparvero le lacrime. Venne a sedersi più vicina alla cognata e le prese una mano con la sua piccola mano energica. Dolly non si ritrasse, ma il suo viso non mutò l'espressione arida. Disse: 

- Non è possibile consolarmi. Dopo quello che è avvenuto, tutto è perduto, tutto è finito! 

E non appena ebbe detto questo, il viso le si addolcì d'un tratto. Anna sollevò la mano magra di Dolly, la baciò e le disse: 

- Ma, Dolly, che fare, che fare? Quale la via migliore in questa terribile situazione? ecco quello a cui bisogna pensare. 

- Tutto è finito e non c'è più nulla da fare - disse Dolly. - E il peggio è, tu mi capisci, che io non posso abbandonarlo: ci sono i bambini, sono legata. E con lui non posso vivere, è un tormento per me vederlo. 

- Dolly, cara, lui mi ha parlato, ma io voglio sentire da te, dimmi tutto. 

Dolly la guardò interrogativamente. 

Una compassione un affetto sinceri apparivano chiaramente sul viso di Anna. 

- E sia - disse improvvisamente lei. - Ma voglio cominciare dal principio. Tu sai come mi sono sposata. Io, con l'educazione di maman, ero non solo ingenua, ma sciocca. Non sapevo nulla, io. Dicono, lo so, che i mariti raccontino alle mogli la loro vita di prima, ma Stiva\ldots{} - si corresse - Stepan Arkad'ic non mi aveva detto nulla. Tu non ci crederai, ma io fino ad ora credevo di essere la sola donna che egli avesse conosciuto. Così ho vissuto per otto anni. Tu capisci, io non solo non sospettavo un'infedeltà, ma la consideravo impossibile; e allora, figurati, con delle idee simili, venire a sapere improvvisamente tutto l'orrore, tutto il ribrezzo\ldots{} Comprendimi. Essere sicura in pieno della propria felicità e d'un tratto\ldots{} - continuò Dolly, trattenendo i singhiozzi - avere in mano la lettera, la sua lettera per l'amante, la mia governante. No, è troppo terribile! - Trasse fuori in fretta il fazzoletto e si coprì il viso. - Capirei anche un momento di capriccio - continuò, dopo una pausa. - Ma ingannarmi così meditatamente, con tanta astuzia\ldots{} E con chi? Continuare ad essere mio marito e nello stesso tempo con lei\ldots{} questo è orribile! Tu non puoi capire\ldots{} 

- Oh, no, capisco. Capisco, cara Dolly, capisco\ldots{} - diceva Anna, stringendole la mano. 

- E tu pensi ch'egli senta tutto l'orrore della mia posizione? - proseguì Dolly. - Per nulla! Lui è felice e soddisfatto. 

- Oh, no - interruppe in fretta Anna. - Fa pena, è distrutto dal rimorso. 

- E che forse è capace di rimorso? - interruppe Dolly, guardando attenta il viso della cognata. 

- Sì, io lo conosco. Non potevo guardarlo senza provarne pena. Noi lo conosciamo tutte e due. È buono, ma è orgoglioso, e ora è così umiliato. E poi quello che soprattutto mi ha commosso\ldots{} - e qui Anna indovinò quello che poteva commuovere Dolly - è che lo tormentano due cose: si vergogna dei bambini, e amandoti\ldots{} sì, sì, amandoti più di tutto al mondo - disse, interrompendo in fretta Dolly che voleva ribattere - ti ha fatto del male, ti ha uccisa. ``No, no, non mi perdonerà'' dice continuamente. 

Dolly guardava pensosa al di là della cognata, ascoltando le sue parole. 

- Sì, capisco come la sua situazione sia orribile; peggio per il colpevole che per l'innocente - disse - se sente che dalla colpa sua deriva tutto il male. Ma come perdonare, come posso essere di nuovo sua moglie dopo di lei? Per me vivere con lui sarebbe un tormento, proprio perché mi è così caro l'amore che ho avuto per lui. 

E i singhiozzi spezzarono le sue parole. 

Ma poi, come apposta, ogni volta che si raddolciva, riprendeva a parlare di ciò che la irritava. 

- Quella lì è giovane, è bella - continuò. - Ma tu capisci, Anna, da chi sono state prese la mia gioventù, la mia bellezza? da lui e dai suoi figli. Ora ho finito di servirgli, e in questo servizio ho dato tutta me stessa; ora, s'intende, gli è più gradita una persona fresca e volgare. Probabilmente, parlavano di me fra di loro, o peggio ancora, non ne parlavano proprio, capisci? - I suoi occhi si accesero di nuovo di rancore. - E poi, dopo tutto questo, mi dirà\ldots{} Come potergli credere? Mai. No; ormai è finito tutto quello che formava la consolazione, la ricompensa a tanto lavoro, al tormento\ldots{} Lo crederesti? Stavo facendo or ora lezione a Griša: prima questa era per me una gioia, ora è un tormento. Perché mi affanno, perché mi affatico? Perché i bambini? È terribile come ad un tratto l'anima mia si sia sconvolta e come invece di tenerezza io non senta per lui altro che rancore, sì, rancore. Lo ucciderei, e\ldots{} 

- Ma tesoro mio, Dolly, ti capisco, ma non tormentarti. Sei tanto offesa, tanto eccitata che molte cose le vedi come non sono. 

Dolly si calmò ed entrambe tacquero per alcuni minuti. 

- Che fare? Anna, pensaci tu, aiutami tu. Io ho riflettuto senza posa e non ho trovato niente. 

Neppure Anna sapeva trovar nulla, ma il suo cuore vibrava ad ogni parola, ad ogni espressione del viso della cognata. 

- Io dico una cosa sola - cominciò Anna - io sono sua sorella, e conosco il suo carattere, quella sua facilità a dimenticarsi di tutto, di tutto - ella fece un gesto sulla fronte - quella sua disposizione all'abbandono completo; ma, in compenso, anche al pentimento completo. Egli in questo momento non crede a quanto è accaduto, non capisce come abbia potuto fare quello che ha fatto. 

- No, lo capisce, lo ha capito - interruppe Dolly. - Ma io\ldots{} tu ti dimentichi di me\ldots{} sto forse meglio, io? 

- Lasciami dire. Quando egli ne parlava, ti confesso, non avevo ancora capito tutto lo sgomento della tua posizione. Vedevo soltanto lui e il fatto che un'intera famiglia fosse sconvolta; mi faceva pena lui; ma ora, dopo aver parlato con te, io, come donna, vedo un'altra cosa: vedo la tua sofferenza e non so dirti quanta pena ne abbia. Ma Dolly, anima mia, io capisco in pieno la tua sofferenza, ma una cosa non so. Io non so\ldots{} non so quanto amore c'è ancora nell'anima tua per lui. Sai solo tu se ve n'è tanto che sia possibile perdonare. Se ve n'è, e tu perdona! 

- No - cominciò Dolly, ma Anna la interruppe, baciandole ancora una volta la mano. 

- Io conosco il mondo più di te - disse lei. - Conosco questi uomini come Stiva, so come considerano queste cose. Tu dici che egli con quella avrà parlato di te. Questo no, non è accaduto. Questi uomini commettono delle infedeltà, ma il loro focolare domestico e la moglie, queste, per loro, sono cose sacre. Per loro, in un certo modo, quelle donne restano spregevoli, e non le confondono con la famiglia. Essi tracciano come una linea insormontabile tra la famiglia e quelle donne. Non lo capisco bene, ma so che è così. 

- Sì, ma lui la baciava\ldots{} 

- Dolly, ascolta, anima mia. Ho visto Stiva quando era innamorato di te. Mi ricordo il tempo in cui veniva a casa mia e parlando di te si commuoveva; e a quale poetica altezza ti trovavi tu per lui; e io so che più egli viveva con te e più in alto tu salivi per lui. Noi a volte ridevamo di lui che ad ogni parola ripeteva: ``Dolly è una donna sorprendente''. Tu sei sempre stata e sei rimasta per lui una cosa celeste, mentre questa è un'attrazione non certo dell'anima sua\ldots{} 

- Ma se questa attrazione si ripeterà? 

- Non è possibile, così per quanto possa intendere io\ldots{} 

- Già, ma tu perdoneresti? 

- Non so, non posso giudicare\ldots{} Sì, posso - disse Anna, dopo aver pensato un po'; e poi, abbracciata col pensiero la situazione e soppesatala sulla bilancia sua intima, aggiunse: - Sì, posso, posso, posso. Sì, lo perdonerei. Non sarei la stessa, ma perdonerei, come se non fosse accaduto affatto\ldots{} 

- Eh, s'intende - interruppe in fretta Dolly, come se stesse per dire quello che aveva pensato più di una volta. - Altrimenti non sarebbe un perdono. Su, andiamo ti accompagno in camera tua - disse, alzandosi, e durante il cammino abbracciò Anna. - Mia cara, come sono contenta che tu sia venuta! Mi sento meglio, molto meglio. 

\capitolo{XX}\label{xx} 

Tutto quel giorno Anna lo passò in casa degli Oblonskij e non volle ricevere nessuno, mentre già alcuni amici, informati del suo arrivo, erano venuti quel giorno stesso. Passò tutta la mattinata con Dolly e i bambini. Mandò soltanto un biglietto al fratello perché venisse senz'altro a pranzare a casa. ``Vieni, Dio è misericordioso'' aveva scritto. 

Oblonskij pranzò a casa; la conversazione fu generale e la moglie parlò con lui dandogli del tu, cosa che ultimamente non accadeva. Fra marito e moglie permaneva lo stesso distacco di rapporti, ma già non si parlava più di separazione e Stepan Arkad'ic vedeva già la possibilità di spiegarsi e far pace. 

Subito dopo pranzo venne Kitty. Conosceva già Anna Arkad'evna, ma molto poco, ed era venuta ora dalla sorella non senza temere come l'avrebbe accolta questa signora del gran mondo pietroburghese che tutti decantavano. Ma piacque ad Anna Arkad'evna; se ne accorse subito. Anna, evidentemente ne ammirava la grazia e la giovinezza e Kitty non fece in tempo a rassicurarsi che già si sentì non solo sotto il fascino di lei, ma addirittura innamorata di lei, così come le ragazze sono capaci di innamorarsi delle signore sposate più grandi di loro. Anna non aveva nulla di simile a una dama di mondo o a una mamma di un bimbo di otto anni; sarebbe piuttosto somigliata a una ragazza di vent'anni per l'agilità delle movenze, per la vivacità che le balenava ora nel riso ora nello sguardo, se non avesse avuto quell'espressione degli occhi seria, a volte triste, che aveva colpito e attirato a sé Kitty. Kitty sentiva che Anna era affatto spontanea e che non nascondeva nulla, ma che portava in sé un mondo di interessi più alti, inaccessibili a lei, complessi e poetici. 

Dopo pranzo, quando Dolly uscì per andare in camera sua, Anna si alzò in fretta e si accostò al fratello che aveva acceso un sigaro. 

- Stiva - disse, ammiccandogli con vivacità, accennandogli alla porta e facendogli il segno della croce: - va', e che il Signore ti aiuti. 

Egli capì, gettò via il sigaro e scomparve dietro la porta. 

Appena Stepan Arkad'ic fu uscito, Anna ritornò sul divano dove sedeva circondata dai bambini. O che i bambini avessero notato come la mamma voleva bene a questa zia, o che essi stessi si sentissero attratti verso di lei, certo è che i due più grandi, e dietro di questi i più piccoli, come spesso fanno i bambini, ancor prima del pranzo si erano attaccati alla nuova zia e non la lasciavano più. E fra di loro si era venuto a formare come una specie di giuoco che consisteva nello star seduti il più vicino possibile a lei, nel toccarla, nel tenere tra le proprie la sua piccola mano, nel baciarla, nel giocar con l'anello suo, o nel toccare almeno la gala del suo vestito. 

- Su, su, così come eravamo seduti prima - disse Anna Arkad'evna riprendendo il proprio posto. 

E di nuovo Griša ficcò la testa sotto il braccio di lei e poggiò la testina sull'abito, splendendo di gioia e trionfo. 

- E così ora, a quando un ballo? - ella disse rivolta a Kitty. 

- La settimana prossima, e un ballo bellissimo. Uno di quei balli in cui ci si diverte sempre. 

- E ce n'è di quelli in cui ci si diverte? - chiese con tenera ironia Anna. 

- È strano, ma ce n'è. Dai Bobrišcev ci si diverte sempre, dai Nikitin anche, ma dai Mezkovyj ci si annoia sempre. Non l'avete notato, forse? 

- No, cara, per me ormai non ci sono balli in cui ci si diverta - disse Anna, e Kitty vide negli occhi di lei quel suo mondo particolare a lei precluso. - Per me ci sono di quelli dove è meno noioso ed increscioso\ldots{} 

- Ma come potete annoiarvi voi a un ballo? 

- E perché non potrei annoiarmi, io, a un ballo? 

Kitty notò che Anna sapeva già quale risposta sarebbe seguita. 

- Ma perché voi siete dovunque la più bella. 

Anna sapeva ancora arrossire. Arrossì e disse: 

- In primo luogo, non è così; e in secondo luogo, anche se questo fosse vero, a che mi servirebbe? 

- Verrete a questo ballo? - chiese Kitty. 

- Credo che non potrò non venire. Ecco, prendi questo - disse a Tanja che tirava un anello che scivolava facilmente dal dito bianco affusolato. 

- Sarò molto contenta se verrete. Vorrei tanto vedervi a un ballo. 

- Almeno così, se sarà proprio necessario andare, mi consolerò al pensiero di farvi cosa gradita\ldots{} Griša, non tirare, ti prego, sono già tutta spettinata - disse, aggiustandosi una ciocca di capelli fuori di posto con la quale Griša aveva giocato. 

- Vi immagino al ballo in lilla. 

- E perché proprio in lilla? - chiese sorridendo Anna. - Su ragazzi, andate, andate. Sentite? Miss Hull chiama per il tè - disse, staccando da sé i bambini e avviandoli in sala da pranzo. 

- Ma io so perché mi invitate a venire al ballo. Voi vi aspettate molto da questo ballo e volete che tutti siano là, che tutti vi prendano parte. 

- Come lo sapete? È così. 

- Com'è bella la vostra età! - continuò Anna. - Ricordo e conosco anch'io quella nebbia azzurra simile a quella che è sulle montagne svizzere. Quella nebbia che vela tutto, in quel tempo beato in cui è appena appena finita l'infanzia, e da quel cerchio immenso, fortunato e gaio, il cammino si fa sempre e sempre più angusto; e ne vien gioia e sgomento a entrare in quella galleria, ancor che appaia e bella e chiara. Chi non è passato attraverso questo? 

Kitty sorrideva in silenzio. ``Ma come mai ella era passata attraverso questo? Come vorrei conoscere tutta la sua storia!'' pensava Kitty ricordando l'aspetto poco poetico del marito Aleksej Aleksandrovic. 

- Io so qualcosa. Stiva mi ha detto, e io mi compiaccio con voi; mi piace molto Vronskij - continuò Anna - l'ho incontrato alla stazione. 

- Ah, era là? - domandò Kitty arrossendo. - Ma che vi ha detto Stiva? 

- Stiva mi ha rivelato tutto. E io sono stata molto contenta. Ho viaggiato con la madre di Vronskij - continuò - ed essa non ha smesso un momento di parlare di lui; è il figlio preferito; io so come siano parziali le mamme, ma\ldots{} 

- E che cosa vi ha detto di lui sua madre? 

- Ah, un mondo di cose! Lo so che è il suo preferito, però, si vede che è un perfetto cavaliere\ldots{} Ecco, per esempio, mi ha raccontato che ha voluto dare tutto il suo patrimonio al fratello e che, fanciullo ancora, ha salvato una donna che annegava. Insomma, un eroe - disse Anna, sorridendo e ricordandosi di quei duecento rubli che egli aveva dato alla stazione. 

Ma nulla disse di quei duecento rubli. Chi sa perché non le piaceva rammentarsene. Sentiva che in quel gesto c'era qualcosa che riguardava lei, e così come non avrebbe dovuto essere. 

- Mi ha pregato tanto di andare da lei - continuò Anna - e io sono contenta di vedere quella vecchietta, e domani ci andrò. Però, grazie a Dio, Stiva rimane a lungo nello studio da Dolly - aggiunse Anna, cambiando discorso e alzandosi, come contrariata da qualcosa, così almeno parve a Kitty. 

- No, prima io, no, io - gridavano i bambini, dopo aver preso il tè, correndo verso la zia. 

- Tutti insieme - disse Anna e, ridendo, corse loro incontro e li abbracciò facendo cadere tutto quel mucchio di bambini brulicanti che mandavano strida di entusiasmo. 

\capitolo{XXI}\label{xxi} 

Per il tè dei grandi Dolly uscì dalla sua camera: Stepan Arkad'ic non si faceva ancora vedere. Forse era uscito dalla camera della moglie per il passaggio di dietro. 

- Temo che avrai freddo di sopra - osservò Dolly rivolta ad Anna - vorrei farti venire giù, così staremo più vicine. 

- Oh, non ti preoccupare per me - rispondeva Anna, guardando il viso di Dolly e cercando di capire se v'era stata o no la riconciliazione. 

- Però qui avrai troppa luce - rispose la cognata. 

- Ti assicuro che dormo dovunque e sempre come un ghiro. 

- Che c'è - chiese Stepan Arkad'ic, venendo fuori dallo studio e rivolgendosi alla moglie. 

Dal suo tono di voce Kitty e Anna capirono che la pace era avvenuta. 

- Vorrei far passare Anna giù, ma bisogna cambiare le tende. Nessuno lo saprà fare, devo farlo da me - rispose Dolly rivolta a lui. 

``Dio lo sa se han fatto pace in pieno'' pensò Anna, sentendo il tono di lei freddo e calmo. 

- Ah, basta, Dolly, a far sempre difficoltà - disse il marito. - Su, se vuoi, faccio io tutto. 

``Sì, sì, devono aver fatto pace'' pensò Anna. 

- Sì, sì, lo so come farai tutto - rispondeva Dolly: - dirai a Matvej di fare proprio quello che è impossibile fare e te ne andrai e lui ingarbuglierà ogni cosa - e nel dir questo l'abituale sorriso canzonatorio increspò le estremità delle labbra di Dolly. 

``La pace è fatta, in pieno - pensò Anna. - Sia lodato Iddio!'' e, rallegrandosi d'essere stata la fautrice, si avvicinò a Dolly e la baciò. 

- Ma niente affatto; perché ci disprezzi tanto, me e Matvej? - disse Stepan Arkad'ic, sorridendo impercettibilmente, rivolto alla moglie. 

Tutta la serata Dolly fu, come al solito, leggermente canzonatoria col marito, e Stepan Arkad'ic contento e allegro, ma non tanto da apparire, dopo il perdono, dimentico della propria colpa. 

Alle nove e mezzo la conversazione serale in casa Oblonskij, particolarmente lieta e piacevole intorno al tavolo da tè, fu turbata da un avvenimento all'apparenza quanto mai naturale, ma che sembrò strano a tutti. Parlando di conoscenti comuni di Pietroburgo, Anna si era alzata, svelta. 

- Ce l'ho nel mio album - disse - sì, anzi, così vi mostrerò il mio Serëza - aggiunse con un materno sorriso d'orgoglio. 

Avvicinandosi le dieci, l'ora in cui era solita salutare il figlio o metterlo a letto lei stessa, prima di andare a un ballo, si era immalinconita per esserne tanto lontana; e di qualunque cosa si parlasse, non riusciva più a interessarsi, e tornava sempre col pensiero al suo Serëza riccioluto. Le era anzi venuta gran voglia di guardarne la fotografia e di parlare di lui. Approfittando del primo pretesto, si era alzata col suo passo leggero, deciso ed era andata a prendere l'album. La scala per salire in camera sua dava sul pianerottolo della grande scalinata dell'ingresso riscaldato. 

Nel momento in cui ella usciva dal salotto in anticamera il campanello squillò. 

- Chi può essere? - disse Dolly. 

- Per venire a riprendermi è presto, per una visita è tardi - osservò Kitty. 

- Forse sarà qualcuno con le carte d'ufficio - aggiunse Stepan Arkad'ic e mentre Anna passava accanto alla scala, un servo corse su per annunciare chi era venuto, mentre l'ospite era in piedi sotto la lampada. Anna, guardando giù, riconobbe subito Vronskij e una sensazione strana di piacere e insieme di paura le agitò il cuore. Egli stava lì dritto, senza togliersi il cappello, e tirava fuori qualcosa dalla tasca. Nel momento in cui ella fu a metà scala, egli alzò gli occhi, la vide e nell'espressione del suo viso ci fu qualcosa come tra la confusione ed il timore. Ella, chinato lievemente il capo, passò e, dietro di lei, si sentì la voce forte di Stepan Arkad'ic che invitava a entrare e la voce bassa, morbida e pacata di Vronskij che rifiutava. 

Quando Anna rientrò con l'album, Vronskij già non c'era più e Stepan Arkad'ic diceva che egli era venuto per informarsi del pranzo dell'indomani in onore di una celebrità straniera. 

- E per nessuna ragione è voluto entrare. È piuttosto strano. 

Kitty arrossì. Credeva di essere l'unica a capire perché egli fosse passato di là e perché non avesse voluto entrare. 

``È stato da noi - si diceva - e non mi ha trovata; ha pensato che fossi qui; ma non è entrato perché è tardi e perché sa che anche Anna è qui''. 

Tutti si scambiarono un'occhiata, senza dir nulla, e presero a guardare l'album di Anna. 

Niente di straordinario, o di strano che una persona passasse da casa di un amico a chiedere i particolari di un pranzo da offrire e che non entrasse; ma strana a tutti parve la cosa. Più che agli altri parve strana e inopportuna ad Anna. 

\capitolo{XXII}\label{xxii} 

Il ballo era appena cominciato quando Kitty, accompagnata dalla madre, faceva il suo ingresso sulla scala grande inondata di luce e piena di fiori e di servitori incipriati e in giacca rossa. Dalle sale giungeva un brusio prodotto da un movimento uniforme, come di alveare; e mentre esse sul ripiano, fra le piante, si andavano acconciando allo specchio le pettinature e gli abiti, dalla sala si udirono i suoni accorti e precisi dei violini dell'orchestra che aveva attaccato il primo valzer. Un vecchietto in borghese, che esalava profumo di acqua di Colonia e che ravviava ad un altro specchio le piccole tempie grige, si imbatté in loro sulla scala e, facendosi da parte, ammirò visibilmente Kitty che non conosceva. Un giovanotto imberbe, uno di quelli che il vecchio principe Šcerbackij definiva ``moscardini'', con un panciotto esageratamente aperto e una cravatta bianca che s'andava aggiustando nel camminare, la salutò, passò oltre e tornò indietro per invitare Kitty per la quadriglia. La prima quadriglia era già stata concessa a Vronskij, fu quindi concessa al giovanotto la seconda. Un ufficiale che si abbottonava un guanto, si scansò presso la porta e, accarezzandosi i baffi, ammirò la rosea Kitty. 

Sebbene l'abito, l'acconciatura e i preparativi tutti del ballo fossero costati a Kitty grandi fatiche e riflessioni, in questo momento ella entrava nel ballo così disinvolta e naturale nel suo complicato vestito di tulle con trasparente rosa, come se tutte quelle roselline e quelle trine e i particolari dell'abbigliamento non fossero costati a lei e a quelli di casa neppure un attimo di attenzione; come se ella fosse venuta al mondo in quel tulle, in quelle trine, con quell'acconciatura alta con una rosa e due foglioline in cima. 

Quando la vecchia principessa, prima di entrare in sala, volle aggiustarle un nastro della cintura che si era spostato, Kitty si tirò leggermente indietro: sentiva che tutto andava bene e si aggraziava addosso a lei e che non c'era più nulla da ritoccare. 

Kitty era in una delle sue giornate felici. L'abito non tirava da nessuna parte, da nessuna parte pendeva la berta di pizzo, le roselline non s'erano sgualcite né staccate; le piccole scarpe rosa sui tacchi ricurvi non premevano, ma rallegravano il piedino. Le folte bande di posticci biondi si mantenevano come naturali sulla piccola testa. Tutti e tre i bottoni si erano chiusi senza staccarsi sul guanto lungo che avvolgeva il braccio rilevandone la forma. Il vellutino nero del medaglione cingeva il collo, proprio con tenerezza. Questo vellutino era un incanto e a casa, guardandosi allo specchio il collo, Kitty aveva sentito che quel nastrino parlava. Per tutto il resto avrebbe potuto sussistere ancora qualche dubbio, ma il vellutino era un incanto. Anche qui, al ballo, Kitty sorrise nel guardarlo allo specchio. Su per le spalle e le braccia nude Kitty sentiva freddo come di marmo, sensazione che amava in modo particolare. Gli occhi le scintillavano e le labbra vermiglie non potevano non sorridere della consapevolezza del proprio incanto. Non fece in tempo a entrare in sala e a giungere fino alla folla variegata, tutta tulle nastri pizzi e fiori delle signore in attesa di essere invitate (Kitty non si trovava mai fra queste), che già fu invitata al valzer, e dal migliore, dal primo cavaliere nella gerarchia dei balli, da un noto direttore di danze, gran cerimoniere, ammogliato, piacente e ben fatto, Egoruška Korsunskij. Lasciata allora allora la contessa Bonina con la quale aveva ballato un primo giro di valzer, questi aveva dato uno sguardo intorno alla sua corte di coppie danzanti, e avendo visto Kitty entrare, era corso verso di lei con quella particolare andatura disinvolta, propria dei direttori di danze, e, dopo essersi inchinato, senza neppure chiedere s'ella volesse o no, aveva alzato il braccio per cingerle la vita sottile. Kitty si voltò per consegnare a qualcuno il ventaglio e la padrona di casa glielo prese sorridendo. 

- Come avete fatto bene a venire per tempo - egli disse cingendole la vita; - che modo è quello di arrivare in ritardo! 

Piegato il braccio sinistro, ella lo poggiò sulla spalla di lui e i piccoli piedi si mossero nelle scarpette rosa, veloci e leggeri, a tempo di musica, sul pavimento levigato. 

- È un riposo ballare il valzer con voi - disse lui lanciandosi nei primi passi lenti del valzer. - Un incanto! una piuma! che précision! - diceva, ripetendo a lei quel che diceva a quasi tutte le sue brave dame. 

Ella sorrise della lode e continuò a osservare la sala al di sopra della spalla di lui. Non era entrata in società da così poco tempo che al ballo tutti i visi potessero fondersi in un'unica estatica visione; non ne era neppure un'assidua frequentatrice alla quale tutti i visi potessero essere così noti da poterne ricevere noia; era nel giusto mezzo: animata, ma nello stesso tempo padrona di sé tanto da poter osservare. Nell'angolo a sinistra vide che si era raccolto il fiore della società. Là, inverosimilmente scollata, stava la bella Lidie, moglie di Korsunskij; là c'era la padrona di casa, e là brillava con la sua calvizie Krivin, sempre presente nella cerchia migliore; là guardavano i giovanissimi, non osando accostarsi, e là ella trovò Stiva e subito dopo vide la testa e la figura di Anna, in abito di velluto nero. Anche lui era là. Kitty non l'aveva visto da quella sera in cui aveva detto di no a Levin. Con i suoi occhi presbiti lo riconobbe subito e notò che la guardava. 

- Ebbene, ancora un giro? Siete forse stanca? - disse Korsunskij, sentendola leggermente ansante. 

- No, grazie. 

- Dove volete che v'accompagni? 

- La Karenina è là, accompagnatemi da lei. 

- Ai vostri ordini. 

E Korsunskij riprese a ballare il valzer, smorzando l'andatura e dirigendosi verso il gruppo che era nell'angolo a sinistra della sala, mormorando: ``Pardon, mesdames, pardon, pardon, mesdames''. Bordeggiando fra un mare di trine, di tulle, di nastri, senza impigliarvisi neppure per un pelo, girò brusco la dama così che le si scoprirono le gambe sottili nelle calze traforate e lo strascico si aprì a ventaglio e coprì le ginocchia di Krivin. Korsunskij s'inchinò, raddrizzò il petto aperto e le diede la mano, per accompagnarla da Anna Arkad'evna. Kitty, rossa in viso, liberò lo strascico dalle ginocchia di Krivin e, ancora stordita, si voltò a cercare Anna. Anna non era in lilla, come proprio avrebbe voluto Kitty, ma aveva un abito di velluto nero, molto scollato che le scopriva le spalle piene e tornite di avorio antico, il petto e le braccia tonde dal polso minuscolo. Tutto l'abito era ornato di merletto veneziano. In testa, sui capelli neri, tutti suoi, aveva una piccola corona di violette, e un'altra simile sul nastro nero della cintura fra le trine bianche. La pettinatura era semplice: spiccavano soltanto quelle brevi anella restie di capelli ricci che, aggraziandola, si sbizzarrivano continuamente sulla nuca e sulle tempie. Al collo tornito e forte aveva un filo di perle. 

Kitty vedeva Anna ogni giorno, era incantata di lei e se l'era figurata sempre in lilla. Ma ora, vedendola in nero, sentì che non ne aveva afferrato tutto il fascino. Le appariva completamente nuova e insospettata. Capì, ora, che Anna non avrebbe potuto essere vestita in lilla e che il fascino suo consisteva nell'emergere sempre dall'abbigliamento, così che l'abito indossato da lei non venisse notato. E il vestito nero con i merletti pregiati neppure si notava; era solamente una cornice, e ne balzava fuori lei, semplice, naturale, elegante e, nello stesso tempo, gaia e viva. 

Stava in piedi, tenendosi, come sempre, straordinariamente diritta e quando Kitty si avvicinò al gruppo, parlava col padrone di casa volgendo lieve il capo verso di lui. 

- No, io non scaglierò la prima pietra - rispondeva - benché non capisca - aggiunse, alzando le spalle, e subito si rivolse a Kitty con un tenero sorriso di protezione. Colto in un fuggevole sguardo femminile tutto l'abbigliamento di Kitty, le fece con la testa un appena percettibile, ma ben comprensibile cenno d'approvazione per l'abito e per la bellezza. - Voi entrate in sala ballando - disse. 

- È una delle più fedeli collaboratrici - disse Korsunskij, salutando Anna Arkad'evna che non aveva ancora visto. - La principessina ci aiuta a rendere bello e allegro il ballo. Anna Arkad'evna, un giro di valzer - disse inchinandosi. 

- Ah, vi conoscete? - disse la padrona di casa. 

- Chi non ci conosce? Mia moglie ed io siamo come i lupi bianchi, tutti ci conoscono - rispose Korsunskij. - Un giro di valzer, Anna Arkad'evna. 

- Io non ballo, quando è possibile farne a meno - disse lei. 

- Ma oggi non se ne può fare a meno - rispose Korsunskij. In quel momento si avvicinò Vronskij. 

- Ebbene, se oggi non si può farne a meno, allora andiamo - disse lei senza notare l'inchino di Vronskij e sollevando rapida la mano sulla spalla di Korsunskij. 

``Perché è scontenta di lui?'' pensò Kitty avendo notato che Anna determinatamente non aveva risposto all'inchino di Vronskij. Vronskij si accostò a Kitty, per ricordarle la prima quadriglia, rammaricandosi di non avere avuto il piacere di vederla in tutto quel tempo. Kitty guardava, ammirata, Anna che ballava il valzer e intanto ascoltava lui. Si aspettava di essere invitata al valzer; ma egli non lo fece e lei lo guardò con sorpresa. Vronskij arrossì e si precipitò a chiederle il ballo, ma non appena ebbe abbracciata la vita sottile di lei e mosso il primo passo, la musica cessò di colpo. Kitty guardò quel viso che era a così breve distanza da lei; e in seguito, per parecchi lunghi anni, quello sguardo pieno d'amore che ella gli aveva rivolto e a cui egli non aveva risposto, le angosciò il cuore di tormentosa vergogna. 

- Pardon, pardon, un valzer, un valzer - gridava dall'altra parte della sala Korsunskij e, presa a volo la prima signorina che gli capitò, ricominciò a ballare. 

\capitolo{XXIII}\label{xxiii} 

Vronskij fece qualche giro di valzer con Kitty. Dopo il valzer Kitty si avvicinò alla madre ed ebbe appena il tempo di scambiare qualche parola con la Nordston, che Vronskij era già venuta a riprenderla per la prima quadriglia. Durante la quadriglia non fu detto nulla di particolare. La conversazione, smozzicata, si aggirò ora sui Korsunskij, marito e moglie, che Vronskij descriveva, con molta amenità, come cari ragazzi quarantenni, ora sul futuro teatro pubblico, e solo una volta la toccò nel vivo, quando egli le chiese se c'era Levin e soggiunse che gli era piaciuto molto. Ma Kitty non si aspettava nulla di più dalla quadriglia. Aspettava invece con trepidazione la mazurca. Le sembrava che nella mazurca si dovesse decidere tutto. Il fatto che durante la quadriglia egli non l'avesse invitata per la mazurca, non l'inquietava. Era sicura di ballare la mazurca con lui, come nelle altre feste, e rifiutò cinque cavalieri dicendo d'essere già impegnata. Tutto il ballo, fino all'ultima quadriglia, fu per Kitty una magica visione di colori gioiosi, di suoni e di movimento. Tralasciava di ballare e chiedeva un po' di riposo solo quando si sentiva troppo stanca. Ma ballando l'ultima quadriglia con uno di quei giovanotti uggiosi al quale non aveva potuto dire di no, venne a trovarsi vis-à-vis con Vronskij e Anna. Dall'inizio del ballo non si era più ritrovata con Anna; ed ecco, a un tratto, la vide ancora del tutto nuova e insospettata. Riconobbe in lei i segni dell'eccitamento dovuto al successo ch'ella stessa conosceva. Vedeva che Anna era come inebriata dall'incanto da lei suscitato. Conosceva questa sensazione, ne conosceva i segni e li vedeva in Anna. Vedeva lo scintillio degli occhi, tremulo e avvampante, e il riso di felicità e di eccitamento che senza volere le increspava le labbra; vedeva la grazia misurata, la sicurezza e la levità dei movimenti. 

``Ma per chi? Per tutti o per uno solo?'' si chiese. E, senza venire in aiuto al disgraziato giovanotto col quale ballava e che s'era lasciato sfuggire il filo di una conversazione iniziata e non riusciva a riannodarlo, e prestando apparentemente orecchio alle forti grida allegre e imperiose di Korsunskij che ora lanciava tutti in un grand rond, ora in una chaîne, Kitty osservava, e il cuore le si stringeva sempre più. ``No, non è l'ammirazione di tutti che l'ha inebriata, ma l'esaltazione di uno solo. E chi è quest'unico? Possibile che sia lui?''. Ogni volta che Vronskij parlava con Anna, negli occhi di lei si accendeva uno scintillio gioioso e un riso di felicità increspava le sue labbra vermiglie. Era come se ella volesse contenersi per non fare apparire questi segni, ma questi salivano da soli sul viso. ``E lui?''. Kitty lo guardò ed ebbe paura. Ciò che con tanta chiarezza appariva nello specchio del viso di Anna, Kitty vide anche in lui. Dove erano più quell'atteggiamento calmo e deciso e quell'espressione del viso liberamente serena? No, ora, ogni volta che egli si volgeva a lei, piegava un po' il capo, quasi desideroso di caderle ai piedi, e nello sguardo suo non vi era che un'espressione di sottomissione e di paura. 

``Non voglio offendervi - diceva ogni volta il suo sguardo - ma voglio salvarmi e non so come''. Un'espressione quale non aveva mai vista nel viso di lui. 

Parlavano di amici comuni, facevano la più insignificante delle conversazioni, ma a Kitty pareva che ogni parola pronunziata decidesse il loro e il suo destino. E lo strano era che, in realtà, pur parlando di come fosse ridicolo Ivan Ivanovic col suo francese e del fatto che per la Elackaja si sarebbe potuto trovare un partito migliore, tuttavia queste parole avevano un senso speciale per loro ed essi lo sentivano così come lo sentiva Kitty. Tutto il ballo, il mondo intero, tutto si coprì di nebbia nel cuore di Kitty. Soltanto la severa educazione ricevuta la sosteneva e l'obbligava a fare quello che da lei si pretendeva, cioè ballare, rispondere alle domande, parlare, sorridere persino. Ma, prima che cominciasse la mazurca, quando già si allontanavano le sedie e alcune coppie s'erano mosse dai salotti verso la sala grande, Kitty fu presa da un attimo di disperazione e di sgomento. Aveva rifiutato cinque cavalieri e ora non ballava la mazurca. Non c'era neppure speranza che qualcuno l'invitasse; proprio perché ella aveva un così grande successo in società, a nessuno poteva venire in mente che non fosse impegnata fino a quel momento. Occorreva dire alla madre che non stava bene e voleva tornare a casa, ma non ne aveva la forza. Era stroncata. 

Si ritirò in fondo a un piccolo salotto e si lasciò cadere su di una poltrona. La gonna lieve come un soffio si sollevò come una nuvola intorno alla vita sottile; la mano nuda, magra e delicata di fanciulla, abbandonata e senza forza affondò nelle pieghe della gonna rosa; l'altra reggeva il ventaglio e con movimento rapido rinfrescava il viso accaldato. Ma a dispetto di questa sua parvenza di farfalla attaccata appena a un filo d'erba e pronta a volar via aprendo le ali iridate, un'angoscia paurosa le stringeva il cuore. 

``Ma forse mi sbaglio, forse questo non è accaduto'' e di nuovo le tornava in mente quello che aveva visto. 

- Kitty, cos'è mai? - disse la contessa Nordston, avvicinandosi senza far rumore sul tappeto. - Non capisco. 

A Kitty tremò il labbro inferiore; si alzò in fretta. 

- Kitty, non balli la mazurca? 

- No, no - disse Kitty con voce che tremava di lacrime. 

- Lui l'ha invitata davanti a me per la mazurca - disse la Nordston, sapendo che Kitty avrebbe capito chi era lui e chi era lei. - Lei ha detto: ``Non ballate forse con la principessina Šcerbackaja?''. 

- Ah, a me che importa! - rispose Kitty. 

Nessuno, all'infuori di se stessa, poteva capire la sua situazione, nessuno sapeva ch'ella aveva detto di no il giorno prima a un uomo che forse amava, e che gli aveva detto di no perché credeva in un altro. 

La contessa Nordston trovò Korsunskij col quale doveva ballare la mazurca e gli impose di andare a invitare Kitty. Kitty ballava nella prima fila e per sua fortuna non doveva parlare perché Korsunskij correva su e giù tutto il tempo dando ordini al suo stuolo di ballerini. Vronskij ed Anna erano situati quasi di fronte a lei. Li vide da lontano con i suoi occhi presbiti, li vide poi da vicino, quando si incontrarono fra le coppie, e più li vedeva più si convinceva che la rovina sua era compiuta. Vedeva che essi si sentivano soli in quella sala piena di gente. E sul viso di Vronskij, sempre così deciso e libero, vedeva quell'espressione di smarrimento e di sottomissione che l'aveva stupita; l'espressione di un cane intelligente che si senta colpevole. 

Anna rideva e il riso si trasmetteva a lui. Anna diveniva pensosa, ed egli si faceva serio. Una forza magica attirava gli occhi di Kitty sul viso di Anna. Ella era incantevole con quel semplice vestito nero, ed incantevoli erano le braccia tonde con i bracciali, ed il collo forte col filo di perle; incantevole la capigliatura inanellata e sciolta e incantevoli le movenze lievi dei piccoli piedi graziosi e delle mani, e il viso piacente pieno di vita; eppure c'era qualcosa di pauroso e di crudele in quell'incanto. 

Kitty l'ammirava ancor più di prima, e sempre di più soffriva. Si sentiva stroncata e il suo viso lo rivelava. Quando Vronskij, scontratosi con lei nella mazurca, la vide, non la riconobbe al primo momento, tant'era mutata. 

- Splendido ballo - le disse, tanto per dire qualcosa. 

- Sì - rispose lei. 

Durante la mazurca, ripetendo una figura complicata inventata da Korsunskij, Anna uscì nel mezzo del circolo, prese due cavalieri e chiamò a sé una signora e Kitty. Kitty la guardò come spaurita e le si accostò. Anna la guardava, socchiudendo gli occhi, e sorrideva stringendole una mano. Ma, visto che il viso di Kitty rispondeva al suo sorriso con disperata sorpresa, si allontanò da lei e si mise a parlare allegramente con l'altra signora. 

``Sì, c'è qualcosa di strano, di diabolico e di affascinante in lei'' si diceva Kitty. 

Anna non voleva restare a cena, ma il padrone di casa cominciò a pregarla. 

- Su via, Anna Arkad'evna - prese a dire Korsunskij, mettendo il braccio nudo di lei sotto la manica del suo frac. - Che idea mi è venuta per il cotillon! Un bijou! 

E si spostava a poco a poco, cercando di trascinarla. Il padrone di casa sorrideva approvando. 

- No, non resterò - rispondeva Anna sorridendo e, malgrado il sorriso, Korsunskij e il padrone di casa capirono, dal tono deciso di lei, che non sarebbe rimasta. - No, anche così ho ballato più a Mosca al vostro ballo che un intero inverno a Pietroburgo - disse Anna voltandosi a guardare Vronskij che stava in piedi davanti a lei. - Bisogna riposare prima d'intraprendere il viaggio. 

- E voi partite certamente domani? 

- Sì, credo - rispose Anna, sorpresa dell'audacia della domanda; e mentre diceva queste parole l'irrefrenabile tremulo scintillio degli occhi e del riso arse lui. 

Anna Arkad'evna non rimase a cena e andò via. 

\capitolo{XXIV}\label{xxiv} 

``Sì, c'è qualcosa di sgradevole e di scostante in me - pensava Levin, uscendo da casa Šcerbackij e dirigendosi a piedi dal fratello. - Non piaccio alla gente. Orgoglio, dicono. Ma non è orgoglio. Se fossi stato orgoglioso, non mi sarei messo in una posizione come questa''. E si figurava Vronskij felice, buono, intelligente e calmo che, probabilmente, non s'era mai trovato nella posizione orribile nella quale s'era venuto a trovare lui quella sera. 

``Sì, certamente ella doveva preferire lui. Così doveva andare; ed io non ho da lamentarmi di niente e di nessuno. La colpa è mia. Quale diritto avevo io di credere ch'ella avrebbe voluto legare la sua vita alla mia? Chi sono io? Che cosa sono? Un uomo da nulla, che non è necessario a niente e a nessuno. - E si ricordò del fratello Nikolaj, e fu contento di fermarsi su questo pensiero. - Non ha forse ragione lui che tutto al mondo è cattivo e sleale? Noi non abbiamo giudicato con giustizia il fratello Nikolaj. Certo dal punto di vista di Prokofij, che l'ha incontrato ubriaco e con la pelliccia stracciata, egli è un uomo spregevole, ma io lo conosco sotto un altro aspetto. Conosco l'anima sua; so che ci somigliamo io e lui. Eppure, invece di andarlo a cercare sono andato a pranzo e poi sono andato là''. Levin si accostò a un fanale, lesse l'indirizzo del fratello che aveva nel portafoglio e poi chiamò un vetturino. Durante il percorso, Levin riandò con la mente a tutti gli episodi a lui noti della vita del fratello Nikolaj. Ricordò che suo fratello durante gli anni universitari e ancora un anno dopo, malgrado le irrisioni dei colleghi, aveva condotto una vita da cenobita, adempiendo rigorosamente i riti della religione, il servizio divino, i digiuni e rifuggendo da qualsiasi piacere, soprattutto dalle donne; ma dopo, come se a un tratto si fosse sbandato, s'era accostato alle persone più indegne e s'era lasciato andare alla vita più sregolata. Ricordò la storia del ragazzo che egli aveva preso dalla campagna per educarlo e che in un accesso di cattiveria aveva battuto tanto da farsi intentare un processo per lesioni. Ricordò la storia del baro col quale aveva perso i denari e al quale aveva richiesto una cambiale e sporto poi egli stesso querela, dimostrando d'essere stato ingannato (era questo il denaro che aveva sborsato Sergej Ivanyc). Ricordò ch'egli aveva passato una notte in guardina per atti di violenza. Ricordò l'ignobile processo che aveva imbastito contro il fratello Sergej Ivanyc per accusarlo di non aver pagato la quota del fondo materno; e la sua ultima impresa, quando cioè, inviato come impiegato nella regione occidentale, era stato messo sotto processo per aver percosso un collega anziano\ldots{} Tutto questo era certamente molto abietto, eppure a Levin non appariva così abietto come a coloro che non conoscevano Nikolaj Levin, che non conoscevano tutta la sua storia, che non conoscevano il suo cuore. 

Levin ricordava come nel tempo in cui Nikolaj era nella fase della mania religiosa, dei digiuni, dei monaci, delle funzioni, nel periodo in cui egli cercava nella religione un aiuto, un freno alla sua natura sensuale, non solo nessuno l'aveva mai sorretto, ma tutti, ed egli stesso, l'avevano irriso. Lo punzecchiavano, lo chiamavano Noè, il monaco; e quando s'era traviato, nessuno gli aveva dato aiuto, e tutti, con orrore e disgusto, gli avevano voltato le spalle. 

Levin sentiva che suo fratello Nikolaj, in fondo all'anima, malgrado la sregolatezza della sua vita, non era più irragionevole delle persone che lo disprezzavano. Non era colpa sua l'essere nato con quel carattere ribelle e con la mente ottenebrata da qualcosa: al contrario aveva sempre cercato d'essere buono. ``Gli esporrò tutto, lo costringerò a dirmi tutto, e gli mostrerò di volergli bene e di capirlo'' decise Levin, giungendo dopo le dieci all'albergo indicato nell'indirizzo. 

- Di sopra, numero 12 e 13 - rispose il portiere alla richiesta di Levin. 

- Ma c'è? 

- Dovrebb'esserci. 

La porta del numero 12 era semiaperta e ne usciva, in un fascio di luce, un fumo denso di tabacco cattivo e fiacco, e il suono di una voce che Levin non conosceva; ma Levin capì subito che il fratello era là: aveva sentito il suo tossicchiare. 

Quando entrò nel vano della porta, la voce sconosciuta diceva: ``Tutto dipende da come sarà condotto l'affare, se ragionevolmente e con coscienza''. 

Konstantin Levin guardò attraverso la porta e vide che quegli che parlava era un giovane intabarrato, con un'enorme capigliatura, mentre una donna giovane butterata, con un abito di lana senza polsi e senza colletto, sedeva sul divano. Il fratello non lo si arrivava a scorgere. Ma a Konstantin si strinse il cuore dalla pena nel vedere in quale ambiente di strane persone viveva suo fratello. Nessuno lo aveva sentito; e Konstantin nel togliersi le soprascarpe ascoltava quello che diceva il signore intabarrato. Parlava di una certa impresa. 

- E che il diavolo le scortichi, quelle classi privilegiate - proruppe tossendo la voce del fratello. - Maša, procurati da cena e dacci del vino se ce n'è restato; se no, manda a prendere. 

La donna si alzò e, uscendo fuori di là dell'intelaiatura, vide Konstantin. 

- C'è un signore, Nikolaj Dmitric - disse. 

- Che vuole? - chiese rabbiosa la voce di Nikolaj. 

- Sono io - rispose Konstantin Levin venendo avanti nella luce. 

- Chi io? - ripeté ancora più rabbiosa la voce di Nikolaj. Si sentì che egli si era alzato di scatto, impigliandosi in qualcosa, e Levin vide dinanzi a sé, sulla porta, la figura enorme, magra e ricurva del fratello; figura a lui nota, ma tuttavia lo sorprese per la selvatichezza, per l'aria malandata, per i grandi occhi spaventati. 

Era ancora più magro che non tre anni prima, quando Konstantin Levin l'aveva visto l'ultima volta. Portava una finanziera: le mani e l'ampia ossatura sembravano ancora più enormi. I capelli s'erano diradati, ma gli stessi baffi spioventi coprivano le labbra, gli stessi occhi guardavano strani e ingenui lui che era entrato. 

- Ah, Kostja! - esclamò subito riconoscendo il fratello, e i suoi occhi s'illuminarono di gioia. Ma, nello stesso momento, si voltò a guardare il giovane e fece quel movimento convulso, così noto a Konstantin, con la testa e il collo, come se la cravatta lo soffocasse, e tutta un'altra espressione, selvaggia, martoriata e crudele, si fermò sul suo viso scarno. 

- Io ho scritto a voi e a Sergej Ivanyc che non vi conosco e non voglio conoscervi. Di che hai\ldots{} di che avete bisogno? 

Era affatto diverso da come se l'era immaginato Konstantin. Konstantin infatti, pensando a lui, aveva dimenticato tutto quello che rendeva tanto laboriosi i rapporti con lui; ma ora, nel vedere il suo viso, e in particolare quel volger convulso del capo, gli tornò in mente tutto questo. 

- Non ho bisogno di nulla per nessuna ragione - rispose timido. - Sono venuto semplicemente per vederti. 

La timidezza del fratello ammansì evidentemente Nikolaj. Egli storse le labbra. 

- Ah, sì? - disse. - Allora entra, siedi. Vuoi cenare? Maša, porta per tre. No, aspetta. Sai chi è? - disse rivolto al fratello, indicando il signore intabarrato. - Questo è il signor Krickij, amico mio sin dal tempo di Kiev, un uomo molto notevole. La polizia, naturalmente, lo perseguita perché non è un vigliacco. 

E secondo la sua abitudine, si voltò a guardare in giro tutti quelli ch'erano nella camera. Visto che la donna sulla porta stava per uscire, le gridò: ``Aspetta, ho detto''. E con quell'imprecisione e discontinuità di discorso che Konstantin conosceva bene, guardando di nuovo tutti, cominciò a raccontare al fratello la storia di Krickij: come l'avessero cacciato dall'università perché aveva organizzato una società di soccorso per gli studenti poveri e scuole domenicali, e come poi fosse entrato in una scuola elementare quale maestro, e come anche di là l'avessero cacciato e infine processato per qualche cosa. 

- Siete dell'università di Kiev? - chiese Konstantin Levin a Krickij per interrompere il silenzio imbarazzante che si era stabilito. 

- Sì, ero a Kiev - disse Krickij stizzito e accigliato. 

- E questa donna - lo interruppe Nikolaj Levin, indicandola - è la compagna della mia vita Mar'ja Nikolaevna. L'ho presa da una casa - e nel dire ciò contrasse il collo. - Ma le voglio bene e la rispetto, e quelli che vogliono avere rapporti con me - aggiunse, alzando la voce e accigliandosi - sono pregati di amarla e di rispettarla. È come se fosse mia moglie, proprio lo stesso. Ecco, così tu sai con chi hai a che fare. E se credi di abbassarti, ecco la porta, e vattene con Dio. 

E di nuovo i suoi occhi percorsero tutti interrogativamente. 

- Non capisco perché mai dovrei abbassarmi. 

- Su, allora ordina, Maša; fa' portare da cena: tre porzioni, vodka e vino\ldots{} No, non occorre. Va'. 

\capitolo{XXV}\label{xxv} 

- Allora guarda - continuò Nikolaj Levin, contraendosi e corrugando con sforzo la fronte. Evidentemente gli era difficile riflettere che cosa dire e che cosa fare. - Ecco, guarda - e mostrò nell'angolo della stanza vari spezzoni di ferro legati con funi. - Vedi questo? È il principio di una nuova impresa alla quale ci accingiamo. Quest'impresa è un'artel'. 

Konstantin non ascoltava quasi. Fissava quel viso malaticcio, tisico, e sempre più ne aveva pena, e non riusciva a seguire quello che suo fratello gli andava raccontando di quella sua artel'. Si rendeva conto che questa artel' era soltanto un espediente per salvarsi dal disgusto di se stesso. Nikolaj Levin continuò a dire: 

- Tu sai che il capitale schiaccia il lavoratore. Da noi gli operai, i contadini sostengono tutto il peso del lavoro e sono posti in una condizione tale che, per quanti sforzi facciano, non riescono ad uscire dalla loro situazione di bestie da soma. Tutto il margine del guadagno, col quale potrebbero migliorare la loro sorte, procurarsi un po' di tempo libero e con esso l'istruzione, tutto il soprappiù della paga è sottratto loro dai capitalisti. E la società è congegnata così che più quelli lavorano, più s'arricchiscono i mercanti, i proprietari di terre, mentre loro rimangono sempre bestie da soma. Quest'ordine di cose va mutato - e guardò fisso e interrogativamente il fratello. 

- Sì, s'intende - disse Konstantin, notando il rossore che era apparso sotto gli zigomi sporgenti del fratello. 

- E poi, ecco, organizziamo un'artel' di fabbriferrai, dove la produzione e il profitto, i principali attrezzi di produzione, tutto sarà in comune. 

- E dove avrà sede quest'artel'? - chiese Konstantin Levin. 

- Nel villaggio di Vozdrëm, nel governatorato di Kazan'. 

- E perché in un villaggio? Nei paesi, mi pare, c'è già tanto da fare. E perché un'artel' di fabbriferrai in un paese? 

- Ma perché anche ora i contadini sono gli stessi schiavi di prima; e appunto per questo, a te e a Sergej Ivanyc dispiace che si voglia farli uscire da questa schiavitù - disse Nikolaj Levin, irritato dall'obiezione. 

Konstantin Levin sospirò, e si mise a esaminare la camera tetra e sudicia. Questo sospiro parve irritare ancor più Nikolaj. 

- Conosco le opinioni aristocratiche tue e di Sergej Ivanyc. So che egli adopera tutte le forze dell'ingegno per giustificare il male esistente. 

- No, ma perché parli di Sergej Ivanyc? - proruppe Levin sorridendo. 

- Sergej Ivanyc? Ah, ecco perché! - gridò ad un tratto Nikolaj, sentendo pronunciare il nome di Sergej Ivanyc - ecco a che scopo\ldots{} Sì, ma a che scopo parlare? Dimmi una cosa\ldots{} Perché sei venuto da me? Tu disprezzi tutto ciò e va bene, e allora vattene con Dio, vattene! - gridò alzandosi dalla sedia - vattene, vattene! 

- Io non lo disprezzo affatto - disse timido Konstantin Levin. - Non discuto neppure. 

Nel frattempo era tornata Mar'ja Nikolaevna. Nikolaj Levin si voltò rabbioso verso di lei. Ella gli si accostò e gli mormorò qualcosa. 

- Non so bene, sto diventando irascibile - disse Nikolaj, calmandosi e respirando faticosamente - e poi tu mi parli di Sergej Ivanyc e del suo articolo. È una tale assurdità, una tale menzogna, un tale autoinganno. Che cosa mai può scrivere sulla giustizia un uomo che non la conosce nemmeno? Avete letto il suo articolo? - disse rivolto a Krickij, sedendosi di nuovo accanto al tavolo e spostando fino alla metà di esso le sigarette sparse, per far posto. 

- Non l'ho letto - disse cupo Krickij, non volendo evidentemente entrare in conversazione. 

- Perché? - si voltò ora a Krickij con irritazione Nikolaj Levin. 

- Perché non ritengo utile perdere il tempo in questo. 

- Ma, scusate, come fate a sapere che verreste a perdere il tempo? Per molti quell'articolo è inaccessibile, troppo alto. Ma per me è un'altra cosa, io vedo da parte a parte le sue idee e so perché tutto questo è debole. 

Tutti tacquero. Krickij si alzò lentamente e prese il berretto. 

- Non volete cenare? Allora, addio. Domani venite col fabbroferraio. 

Appena Krickij fu uscito, Nikolaj Levin sorrise e strizzò l'occhio. 

- Anche lui è cattivo - disse. - Perché io vedo\ldots{} 

Ma in quel momento Krickij sulla porta lo chiamò. 

- Che occorre ancora? - disse Nikolaj e uscì nel corridoio verso di lui. Rimasto solo con Mar'ja Nikolaevna, Levin si rivolse a lei. 

- E voi, è molto che vivete con mio fratello? - le chiese. 

- Ecco, è già più di un anno. La sua salute è molto peggiorata. Beve tanto. 

- E che cosa beve? 

- La vodka beve, e gli fa male! 

- Molta forse? - mormorò Levin. 

- Sì - disse lei, guardando timida la porta sulla quale era apparso Nikolaj Levin. 

- Di che stavate parlando? - domandò, aggrottando le sopracciglia e facendo passare dall'uno all'altra i suoi occhi spauriti. - Di che cosa? 

- Di nulla - rispose Konstantin confondendosi. 

- E se non volete dirlo, fate pure. Solo non c'è bisogno che tu parli con lei. Lei è una di quelle ragazze\ldots{} e tu sei un signore - disse contraendo il collo. - Tu, io lo vedo, hai capito tutto, l'hai apprezzata, e consideri con pietà i miei traviamenti - cominciò di nuovo, alzando la voce. 

- Nikolaj Dmitric, Nikolaj Dmitric - mormorò di nuovo Mar'ja Nikolaevna, accostandosi a lui. 

- Su, va bene, va bene! Già, e che ne è della cena? Ah, eccola - esclamò, vedendo un cameriere col vassoio. - Qua, metti qua - disse irritato e, presa la vodka, ne versò un bicchierino e bevve avidamente. - Bevi? ne vuoi? - disse, fattosi allegro d'un tratto, al fratello. - Su, via, basta di Sergej Ivanyc. Eppure son contento di vederti. Checché si dica, non siamo estranei tuttavia. Su, bevi, dunque. Racconta su, che cosa fai? - continuò, masticando avidamente un pezzo di pane e versando un altro bicchierino. - Come te la passi? 

- Vivo solo in campagna, così come vivevo prima, mi occupo dell'amministrazione - rispose Konstantin, guardando con terrore l'avidità con la quale il fratello beveva e mangiava e sforzandosi di nascondere la propria attenzione. 

- Perché non prendi moglie? 

- Non m'è capitato - rispose arrossendo Konstantin. 

- Come mai? Per me è finita. Me la sono sciupata la mia vita. L'ho detto e lo dirò ancora: se mi avessero dato la mia parte quando ne avevo bisogno, tutta la mia vita sarebbe stata un'altra. 

Konstantin Dmitrevic si affrettò a cambiare discorso. 

- Lo sai che il tuo Vaniuška è da me a Pokrovskoe come inserviente? - disse. 

Nikolaj contrasse il collo e divenne pensoso. 

- Su, raccontami che si fa a Pokrovskoe? La casa è sempre in piedi? E le betulle, e la nostra stanza di studio? E Filipp il giardiniere è possibile che sia vivo ancora? Come ricordo la pergola e il sedile! Bada, però, a non cambiar nulla in casa; ma prendi moglie al più presto, e assesta tutto così com'era prima. Io allora verrò da te, se tua moglie sarà una brava donna. 

- Ma vieni adesso da me - disse Levin. - Come ci sistemeremmo bene! 

- Verrei da te se sapessi di non trovare Sergej Ivanyc. 

- Ma non lo troverai. Io vivo del tutto indipendente da lui. 

- Sì; ma qualunque cosa tu dica, devi scegliere fra me e lui - disse, guardando timido il fratello negli occhi. Questa timidezza commosse Konstantin. 

- Se vuoi conoscere tutta la mia opinione a questo riguardo, ti dirò che nella questione tua con Sergej Ivanyc, io non prendo le parti né dell'uno né dell'altro. Avete torto tutti e due. Tu hai torto in un modo più formale, lui in un modo più sostanziale. 

- Ah, ah, tu hai capito questo, l'hai capito davvero? - gridò con gioia Nikolaj. 

\begin{itemize} \itemsep1pt\parskip0pt\parsep0pt \item Ma io, personalmente, tengo più alla tua amicizia, perché\ldots{} \end{itemize} 

- Perché, perché? 

Konstantin non poteva dire che ci teneva perché Nikolaj era un disgraziato e aveva bisogno di affetto. Ma Nikolaj capì ch'egli voleva dire proprio questo e, accigliandosi, allungò di nuovo la mano verso la vodka. 

- Basta, Nikolaj Dmitric - disse Mar'ja Nikolaevna, stendendo la mano grassoccia verso la caraffa. 

- Lascia! Non seccare! Ti picchio! - gridò. 

Mar'ja Nikolaevna sorrise d'un sorriso mansueto e buono che si comunicò anche a Nikolaj e allontanò la vodka. 

- Tu credi che lei non capisca nulla? - disse Nikolaj. - Capisce tutto meglio di noi. Non è vero che in lei c'è qualcosa di buono e di caro? 

- Non siete stata mai prima a Mosca? - le disse Konstantin, tanto per dire qualche cosa. 

- Ma non darle del voi. Ne può avere soggezione. Nessuno, tranne il giudice di pace, quando l'hanno giudicata perché voleva andarsene dalla casa di corruzione, le ha mai dato del voi. Dio mio, che razza di insensatezze al mondo! - gridò improvvisamente. - Queste nuove istituzioni, questi giudici di pace, il consiglio distrettuale, che assurdità. 

E prese a raccontare i suoi contrasti con le nuove istituzioni. 

Konstantin Levin lo ascoltava, ma ora quel negare il valore di tutte le pubbliche istituzioni, cosa che egli stesso condivideva e che spesso aveva espresso, gli spiaceva sulle labbra del fratello. 

- In quell'altro mondo capiremo tutto questo - disse scherzando. 

- In quell'altro mondo? Oh, io non amo l'altro mondo! Non l'amo - disse, fermando i suoi selvaggi occhi spauriti in faccia al fratello. - Perché ora, ecco, ci sembra bello andarcene via da tutta questa turpitudine, da tutta questa confusione degli altri e nostra, ma io ho paura della morte, ho paura, tremenda paura della morte. - Rabbrividì. - Ma bevi qualcosa? Vuoi dello champagne? Oppure, andiamo in qualche posto. Andiamo dagli zigani! Sai, mi piacciono gli zigani e anche le canzoni russe. 

La sua lingua cominciò ad imbrogliarsi ed egli prese a saltare da un argomento all'altro. Konstantin, con l'aiuto di Maša, lo convinse a non muoversi di casa e lo mise a letto completamente ubriaco. 

Maša promise di scrivere a Konstantin in caso di necessità e di convincere Nikolaj Levin ad andare a vivere presso il fratello. 

\capitolo{XXVI}\label{xxvi} 

La mattina Konstantin Levin partì da Mosca e verso sera giunse a casa. In treno parlò con i compagni di viaggio di politica, delle nuove strade ferrate, e durante il percorso, così come durante il soggiorno a Mosca, fu sopraffatto da una certa confusione di idee, da uno scontento di sé, come da una vergogna di fronte a qualcosa. Ma quando uscì dalla stazione e riconobbe Ignat il cocchiere, orbo di un occhio, col bavero del gabbano rialzato; quando, nella luce incerta che filtrava dalle finestre della stazione, vide la slitta coi tappeti, i suoi cavalli con le code legate, le bardature ad anelli e i fiocchi, e quando Ignat il cocchiere, prima ancora di finire di sistemare i bagagli, prese a raccontargli le novità della campagna: l'arrivo dell'imprenditore, lo sgravo della Pava, egli sentiva che a poco a poco la confusione si diradava, che la vergogna e lo scontento scomparivano. Al solo vedere Ignat e i cavalli aveva provato questo; ma quando infilò il pellicciotto di montone che gli avevan portato e, sedutosi tutto imbacuccato nella slitta, partì, pensando alle imminenti disposizioni da dare in campagna e guardando il bilancino sgroppato eppur focoso, un tempo cavallo da sella del Don, cominciò a considerare in modo del tutto diverso quello che gli era successo. Sentiva di essere di nuovo se stesso e di non voler essere altri. Voleva soltanto essere migliore di come era prima. In primo luogo, da quel giorno decise di non sperare più in quella felicità straordinaria che gli doveva essere data dal matrimonio e, in conseguenza, di non disdegnare tanto il presente. In secondo luogo non avrebbe permesso a se stesso di lasciarsi trascinare dal vizio carnale il cui ricordo lo aveva tanto tormentato al momento di fare la sua proposta. 

Dopo, ricordando il fratello Nikolaj, decise con se stesso di non dimenticarlo mai più, di aiutarlo invece, di non allontanarlo mai più dalla sua mente e di essere pronto a venirgli in aiuto quando si fosse trovato in cattive condizioni. E questo sarebbe accaduto presto, lo sentiva. Poi, anche il discorso del fratello sul consumismo, che egli aveva lì per lì abbandonato con tanta leggerezza, ora lo faceva meditare. Riteneva un'assurdità il cambiamento delle condizioni economiche esistenti, ma sentiva sempre l'ingiustizia del proprio superfluo di fronte alla miseria del popolo. E decise che d'ora in poi, per sentirsi pienamente nel giusto, pur avendo sempre lavorato e vissuto senza sperpero, avrebbe lavorato ancora di più e ancora di meno si sarebbe consentito del lusso. E tutto questo gli sembrava così facile a ottenersi, che passò tutto il tempo del viaggio nei sogni più lusinghieri. Con un vigoroso senso di fiducia in una vita migliore, giunse a casa alle nove di sera. 

Dalle finestre di Agaf'ja Michajlovna, la vecchia njanja che in casa occupava il posto di governante, veniva giù la luce sulla neve del piazzale davanti alla casa. Ella non dormiva ancora. Kuz'ma, svegliato da lei, corse fuori sulla scala, assonnato e scalzo. La cagna da caccia Laska, che per poco non buttò a terra Kuz'ma, saltò fuori anche lei a guaire e a strofinarsi contro le ginocchia di Levin; si sollevava sulle zampe, desiderando, senza peraltro arrischiarvisi, mettergli le zampe anteriori sul petto. 

- Siete tornato presto, batjuška - diceva Agaf'ja Michajlovna. 

- M'è venuta addosso la noia, Agaf'ja Michajlovna. In albergo si sta bene, ma a casa è meglio - le rispose, e passò nello studio. 

Lo studio fu illuminato a poco a poco da una candela che vi portarono. Cominciarono a comparire i noti particolari; le corna di cervo, gli scaffali coi libri, lo specchio, la stufa con la bocca di calore che da tempo doveva essere riaccomodata, il divano del padre, il grande scrittoio, sullo scrittoio un libro aperto, un portacenere rotto, un quaderno con la propria scrittura. Quando egli vide tutto questo, per un attimo fu preso dal dubbio di poter costruire quella nuova vita di cui aveva sognato durante il viaggio. Era come se tutte queste impronte di vita lo afferrassero e gli dicessero: ``No, non ti libererai di noi e non sarai un altro; ma sarai così come sei sempre stato, con tutti i tuoi dubbi e con quell'eterno scontento di te, con gli inutili tentativi di ripresa e con le ricadute, con quell'eterna ansia di felicità che non ti è data e che per te è impossibile''. 

Ma questo lo dicevano le sue cose, mentre un'altra voce nell'animo suo diceva che non ci si doveva sottomettere al passato e che di se stessi si poteva fare tutto. E obbedendo a questa voce, si accostò a un angolo dove si trovavano due pesi da un pud ciascuno e cominciò a sollevarli da ginnasta qual era, cercando di mettersi in uno stato di vigore. Di là dalla porta scricchiolarono dei passi. Egli abbassò in fretta i pesi. 

Entrò il fattore e disse che tutto, grazie a Dio, andava bene; ma comunicò che il grano saraceno s'era bruciacchiato nel nuovo essiccatoio. Questa notizia esasperò Levin. Il nuovo essiccatoio era stato costruito e in parte ideato da Levin. Il fattore era sempre stato contrario al nuovo essiccatoio e ora, con celata soddisfazione, dichiarava che il grano saraceno s'era bruciato. Levin invece era fermamente convinto che s'era bruciato solo perché non erano state prese quelle misure che egli aveva cento volte disposto. Si indispettì, fece una solenne risciacquata al fattore. Ma c'era stato un avvenimento importante e lieto; s'era sgravata la Pava, la vacca più bella, più costosa, comprata a una esposizione. 

- Kuz'ma, dammi il pellicciotto. E voi, andate a prendere un po' la lanterna, voglio dare un'occhiata - disse al fattore. 

La stalla per le mucche pregiate si trovava subito dietro alla casa. Attraversando il cortile, vicino al mucchio di neve che era accanto alle serenelle, Levin raggiunse la stalla. Quando si aprì la porta coperta di gelo, si sentì una zaffata di letame caldo, fumante e le mucche, sorprese dalla luce insolita della lanterna, si agitarono sulla paglia fresca. Baluginò la groppa vasta, liscia, a macchie nere e bianche dell'olandese. Berkut, il toro, disteso con l'anello al labbro, avrebbe voluto alzarsi, ma cambiò idea, soffiò due volte quando gli passarono accanto. La bella Pava, rossa, enorme come un ippopotamo, con la schiena voltata, nascondeva a quelli che entravano la vitellina e se l'andava annusando. 

Levin entrò nel recinto, guardò la Pava e fece alzare sulle lunghe zampe traballanti la vitella bianca e rossa. La Pava, agitata, stava per mugghiare, ma quando Levin le accostò la vitellina, si acquietò e, dopo aver soffiato pesantemente, prese a leccarla con la lingua scabra. 

La vitella intanto dava dei colpi col muso, annaspando sotto l'anguinaia della madre e movendo in giro la piccola coda. 

- Su, fa' luce qua, Fëdor, qua la lanterna - diceva Levin osservando la vitella. - Tale e quale la madre! Benché per il colore somigli al padre. Bella, molto bella. Lunga e lattaiola. Vasilij Fëdorovic, è bella, eh? - si voltò al fattore, completamente in pace con lui per il grano saraceno, tanto era contento della vitella. 

- E a chi dovrebbe somigliare per essere brutta? Il giorno dopo la vostra partenza è venuto Semën l'imprenditore. Bisognerà mettersi d'accordo con lui, Konstantin Dmitric - disse il fattore. - Vi ho già parlato della macchina. 

Questa sola questione immise Levin in tutti i particolari dell'azienda, che era vasta e complessa, ed egli dalla stalla passò in ufficio, e, dopo aver parlato col fattore e con Semën l'imprenditore, rientrò in casa e andò difilato di sopra, in salotto. 

\capitolo{XXVII}\label{xxvii} 

La casa era grande, all'antica, e Levin, pur vivendo solo, la occupava e la riscaldava tutta. Sapeva che questo era sciocco, sapeva che era perfino malfatto e contrario ai suoi attuali nuovi propositi, ma questa casa era tutto un mondo per Levin. Era il mondo nel quale avevano vissuto ed erano morti suo padre e sua madre. Essi avevano vissuto quella vita che per Levin rappresentava l'ideale di ogni perfezione e che egli sognava di rinnovare con la propria moglie e con la propria famiglia. 

Levin ricordava appena sua madre. L'immagine di lei era sempre stata un ricordo sacro, e nella sua mente la futura sposa avrebbe dovuto essere una riproduzione di quell'ideale delicato e santo di donna che era stata sua madre. 

Egli non solo non poteva immaginare l'amore per la donna al di fuori del matrimonio, ma immaginava prima la famiglia e poi la donna che gliel'avrebbe data. Perciò le sue idee sul matrimonio non erano simili a quelle della maggioranza degli uomini che conosceva, per i quali il matrimonio era uno dei molti affari della vita sociale. Per Levin era il più grande avvenimento della vita, dal quale dipendeva tutta la felicità. E ora bisognava rinunciarvi. 

Quando entrò nel salottino dove era solito prendere il tè e si mise a sedere nella sua poltrona con un libro, e quando Agaf'ja Michajlovna gli portò il tè e, col suo solito ``e mi metto a sedere anch'io, batjuška'', si accomodò sulla sedia accanto alla finestra, Levin sentì che, per quanto ciò fosse strano, egli non aveva abbandonato il suo sogno e non poteva vivere senza esso. O con lei o con un'altra, ma questo sarebbe avvenuto. Leggeva il libro, pensava a quello che leggeva, soffermandosi a sentire Agaf'ja Michajlovna che parlottava senza posa; e intanto vari quadri della sua azienda agricola e della futura vita familiare si presentavano senza alcun legame alla sua immagine. Sentiva che in fondo all'anima qualcosa si fissava, si dimensionava, si assestava. 

Ascoltava il parlottare di Agaf'ja Michajlovna, di come Prochor avesse dimenticato Dio e con i denari che gli aveva regalato Levin per comprare il cavallo cioncasse tutto il giorno e picchiasse a morte la moglie; ascoltava e leggeva il libro, e ricordava tutto il procedimento delle sue idee risvegliato dalla lettura. Era un libro di Tyndall sul calore. Ricordava le sue critiche mosse al Tyndall per quella sua disinvoltura nel condurre gli esperimenti e per quella sua mancanza di visione filosofica. Ma d'un tratto gli affiorò alla mente un pensiero piacevole: ``Fra due anni avrò nella mia mandria due mucche olandesi, la stessa Pava potrà essere ancora viva, e alle dodici giovenche di Berkut aggiungici a far bella mostra queste tre, che meraviglia!''. E prese di nuovo il libro. 

``E va bene, l'elettricità e il calore sono una cosa sola; ma è possibile, per risolvere un problema, porre in un'equazione una grandezza in luogo di un'altra? E allora? Il collegamento di tutte le forze della natura anche così si sente per istinto\ldots{} Proprio bello quando la figlia di Pava sarà già una mucca pezzata di rosso e quando ci sarà già una mandria cui aggiungere queste tre!\ldots{} Perfetto!\ldots{} Uscire con la moglie e con gli ospiti a incontrar la mandria\ldots{} Mia moglie dirà: `Kostja ed io abbiamo tirato su questa vitella come fosse un bambino'. `Come vi può interessare tutto questo?' dirà l'ospite. `Tutto quello che interessa lui, interessa me'. Chi sarà mai lei? - Ed egli ricordava quello che era successo a Mosca\ldots{} - Ma che fare? Io non ne ho colpa. Ora tutto andrà in modo nuovo. È assurdo che la vita, che il passato non lascino raggiungere lo scopo. Bisogna lottare per vivere meglio\ldots{}''. Sollevò il capo e si fece pensoso. La vecchia Laska, che non aveva ancora completamente smaltito la gioia dell'arrivo del padrone, e che era corsa ad abbaiare in cortile, ritornò scodinzolando e portando con sé odor d'aria fresca; si accostò a Levin, gli ficcò il muso sotto il braccio, guaendo flebile e chiedendo d'essere carezzata. 

- Solo la parola non ha - disse Agaf'ja Michajlovna. - È un cane, eppure capisce che il padrone è tornato ed è di umor nero. 

- Perché d'umor nero? 

- E che forse non lo vedo, batjuška? Basta aver la salute e la coscienza pulita. 

Levin la guardava fisso, meravigliandosi che ella avesse intuito i suoi pensieri. 

- Be', devo portare dell'altro tè? - disse lei e, presa la tazza, uscì. 

Laska continuava a ficcargli il muso sotto il braccio. Egli la lisciò. Allora essa si acciambellò ai suoi piedi, poggiando il muso sulla zampa posteriore che sporgeva. E a mostrar che ora stava bene, che era contenta, aprì leggermente la bocca, schioccò un po' con le labbra e, accostate ai vecchi denti le labbra bavose, s'acquietò in una calma beata. Levin seguì attentamente quest'ultimo movimento. 

``Anch'io così - si disse - anch'io così. Non fa niente\ldots{} Tutto va bene''. 

\capitolo{XXVIII}\label{xxviii} 

La mattina dopo il ballo, Anna Arkad'evna inviò di buon'ora un telegramma al marito, annunziandogli la sua partenza da Mosca per quel giorno stesso. 

- No, devo partire, devo partire - diceva alla cognata, spiegando il cambiamento di programma con un tono tale che pareva si fosse ricordata di tante faccende da non poterle nemmeno elencare. 

- No, è meglio oggi stesso! 

Stepan Arkad'ic non pranzò a casa, ma promise di venire alle sette per accompagnare la sorella. Nemmeno Kitty venne: mandò un biglietto in cui diceva di aver mal di capo. 

Dolly e Anna pranzarono soltanto con i bambini e la signorina inglese. Ma i bambini, o perché incostanti e ipersensibili, o perché avvertivano che Anna quel giorno era tutt'altra da quella che essi avevano preso ad amare e che già non s'occupava più di loro, certo è che avevano smesso improvvisamente il loro giuoco con la zia, quell'attaccarsi a lei; e il fatto che lei partisse non li interessava per nulla. Tutta la mattina Anna fu presa dai preparativi per la partenza. Scrisse alcuni biglietti ad amici moscoviti, annotò alcuni conti, e preparò il bagaglio. A Dolly pareva che nell'insieme ella non fosse in tranquillità di spirito, ma in un certo stato di inquietudine che ella conosceva bene e che sorge non senza ragione e per lo più nasconde lo scontento di sé. Quando, dopo pranzo, Anna andò in camera sua a cambiarsi, Dolly la seguì. 

- Come sei strana oggi - le disse. 

- Io? Trovi? Non sono strana, sono cattiva. Mi accade talvolta. Avrei voglia di piangere. È molto sciocco, ma passa - disse svelta Anna, abbassando il viso divenuto rosso su un minuscolo sacchetto dove andava riponendo la cuffia da notte e i fazzoletti di batista. I suoi occhi brillavano in modo particolare e si riempivano continuamente di lacrime. - Mi è tanto dispiaciuto lasciare Pietroburgo, e ora invece non me ne andrei più via di qua. 

- Sei venuta ed hai fatto un'opera di bene - disse Dolly, osservandola attentamente. 

Anna la guardò con gli occhi pieni di lacrime. 

- Non dir questo, Dolly. Io non ho fatto nulla, e non potevo far nulla. Mi meraviglio, a volte, a veder come la gente sembri d'accordo nel guastarmi. Che ho fatto e che potevo fare? In te, nel tuo cuore s'è trovato tanto amore da perdonare\ldots{} 

- Senza di te, Dio lo sa cosa sarebbe stato! Come sei felice, Anna! - disse Dolly. - Nell'anima tua tutto è limpido e bello. 

- Ognuno ha nell'anima i propri skeletons, come dicono gli inglesi. 

- E tu quali skeletons hai mai? In te tutto è così chiaro. 

- Eppure ci sono - disse Anna a un tratto e, inaspettatamente, dopo le lacrime, un riso sottilmente ironico le increspò le labbra. 

- Ma certamente sono gai, i tuoi skeletons, non tenebrosi - disse sorridendo Dolly. 

- No, sono tenebrosi. Lo sai perché vado via oggi e non domani? Questa confessione che mi pesa te la voglio fare - disse Anna decisa, riversandosi sulla poltrona e guardando dritto negli occhi Dolly. 

E con sorpresa, Dolly vide che Anna era diventata rossa fino alle orecchie, fino a quelle brevi anella di capelli neri che le si sbizzarrivano sul collo. 

- Sì - continuò Anna. - Sai perché Kitty non è venuta a pranzo? È gelosa di me. Io le ho sciupato tutto\ldots{} sono stata io a renderle quel ballo un tormento e non una gioia. Ma davvero, davvero non ne ho colpa, oppure solo un poco - disse, indugiando con voce sottile sulla parola ``poco''. 

- Oh, come l'hai detto alla stessa maniera di Stiva! - disse ridendo Dolly. 

Anna si urtò. 

- Oh, no, no! io non sono Stiva - disse, accigliandosi. - Te lo racconto perché neppure un attimo mi permetto di dubitare di me stessa - disse Anna. 

Ma proprio nel momento in cui pronunciava queste parole, sentì che non erano vere: non solo dubitava di se stessa ma provava un'agitazione al pensiero di Vronskij, e partiva prima di quello che avrebbe voluto solo per non incontrarsi più con lui. 

- Già, Stiva ci diceva che hai ballato la mazurca con lui e che lui\ldots{} 

- Non puoi immaginare come ciò sia stato ridicolo. Io non pensavo che a combinare il matrimonio e a un tratto, ecco tutt'altra cosa. Forse senza volere io\ldots{} 

Arrossì e si fermò. 

- Oh, loro lo sentono subito! - disse Dolly. 

- Ma io sarei desolata se ci fosse qualcosa di serio da parte sua - l'interruppe Anna. - E sono sicura che tutto questo sarà dimenticato, e che Kitty cesserà di odiarmi. 

- Del resto, Anna, a dirti la verità, io non desidero molto questo matrimonio per Kitty. Ed è meglio che vada a monte se lui, Vronskij, ha potuto innamorarsi di te in un giorno. 

- Oh, Dio mio, questo sarebbe così sciocco! - disse Anna, e di nuovo un rossore denso di soddisfazione le apparve sul viso, nel sentire espresso in parole il pensiero che l'occupava tutta. - E così, ecco, me ne vado dopo essermi fatta una nemica di Kitty, che avevo preso ad amare. Ah, com'è cara! Ma tu appianerai tutto questo, Dolly, vero? 

Dolly poteva trattenere a stento un sorriso. Voleva bene ad Anna, ma non le spiaceva scorgere anche in lei qualche debolezza. 

- Una nemica! Non può essere. 

- Vorrei tanto che tutti voi mi voleste bene come ve ne voglio io; e ora ho preso a volervene ancora di più - disse Anna con le lacrime agli occhi. - Ah, come sono sciocca, oggi! 

Si passò il fazzoletto sul viso e cominciò a vestirsi. 

Stepan Arkad'ic giunse proprio al momento della partenza, in ritardo, col viso accaldato, allegro, fragrante di vino e di sigaro. 

L'emotività di Anna si era comunicata a Dolly e, quando abbracciò per l'ultima volta la cognata, le mormorò: 

- Sappi, Anna, che quello che hai fatto per me non lo dimenticherò mai. E ricordati che ti ho voluto bene e te ne vorrò sempre come all'amica migliore. 

- Non capisco perché - disse Anna, baciandola e nascondendo le lacrime. 

- Tu mi hai capita e mi capisci. Addio, cara! 

\capitolo{XXIX}\label{xxix} 

``Finalmente tutto è finito, sia lodato Iddio!''. Fu questo il primo pensiero che venne ad Anna quando salutò per l'ultima volta il fratello che fino al terzo segnale aveva ostruito con la propria persona l'ingresso della vettura. Sedette nel piccolo sedile accanto ad Annuška e diede un'occhiata in giro nella penombra della vettura letto. ``Grazie a Dio, domani vedrò Serëza ed Aleksej Aleksandrovic, e la mia buona vita d'ogni giorno scorrerà come prima''. 

Ancora in quello stato di inquietudine che l'aveva posseduta tutto il giorno, Anna si preparò con cura e piacere per il viaggio; con le piccole mani agili aprì e richiuse il sacchetto rosso, tirò fuori un piccolo guanciale, se lo pose sulle ginocchia e, avvoltesi accuratamente le gambe, sedette tranquilla. Una signora malata si disponeva già a dormire. Altre due signore presero a parlare con lei, mentre una vecchia obesa si ravvolgeva le gambe e faceva delle osservazioni sul riscaldamento. Anna rispose qualche parola alle signore, ma non prevedendo alcun interesse dalla conversazione, chiese ad Annuška di tirar fuori la lanterna da viaggio, l'appese al bracciolo della poltrona e prese dalla borsetta un tagliacarte e un romanzo inglese. In un primo tempo non le fu possibile di leggere. Le davano noia innanzi tutto il chiasso e l'andirivieni della gente; poi, quando il treno si mise in moto, non poté non prestare orecchio ai rumori, e la neve che picchiava sul finestrino di sinistra e si attaccava al vetro, la vista di un capotreno tutto imbacuccato che passava tutto ricoperto di neve da un lato solo, i discorsi sulla tormenta che infuriava distrassero la sua attenzione. Poi tutto divenne uniforme, il traballio interrotto da scosse, la neve al finestrino, gli improvvisi passaggi da un caldo di vaporazione al freddo e poi di nuovo al caldo, il baluginare di quegli stessi volti nella penombra e il suono delle stesse voci; e Anna prese a leggere e a capire quello che leggeva. Annuška già sonnecchiava, tenendo la sacca rossa sulle ginocchia con le mani larghe nei guanti, uno dei quali era bucato. Anna Arkad'evna leggeva e capiva, ma non provava piacere a leggere e a seguire il riflesso della vita degli altri. Aveva troppa voglia di viverla lei, la vita. Leggeva che l'eroina del romanzo vegliava un malato e le veniva voglia di camminare in punta di piedi per la camera del malato; leggeva che un membro del parlamento faceva un discorso e le veniva voglia di pronunciare lei quel discorso; leggeva che lady Mary inseguiva a cavallo un branco di bestie, provocando la cognata e facendo meravigliare tutti del suo ardire, e le veniva voglia di far lei tutto questo. Non c'era nulla da fare, invece, e rigirando il coltellino liscio tra le piccole mani, si sforzava di leggere. 

L'eroe del romanzo aveva già cominciato a raggiungere la sua felicità inglese, il titolo di baronetto e una tenuta, e Anna stava per desiderare di andare con lui in questa tenuta, quando improvvisamente sentì ch'egli avrebbe dovuto vergognarsi e che anche lei avrebbe dovuto vergognarsi di quella medesima cosa. Ma di che cosa mai egli doveva vergognarsi? ``Di che cosa mai mi vergogno io?'' si domandò con meraviglia offesa. Lasciò il libro e si abbandonò sulla spalliera della poltrona, stringendo forte il tagliacarte con tutte e due le mani. Nulla da aver vergogna. Esaminò tutti i suoi ricordi di Mosca. Erano tutti buoni, piacevoli. Ricordò il ballo, ricordò Vronskij e il suo viso innamorato, sottomesso, ricordò tutti i suoi rapporti con lui; non c'era nulla di cui vergognarsi. Ma intanto proprio a questo punto il senso di vergogna diveniva più forte, come se una certa voce interiore, proprio lì, nel punto in cui si ricordava di Vronskij, le dicesse: ``Caldo, caldo, scottante''. ``Ebbene - si disse decisa, cambiando posizione nella poltrona. - Che vuol dire ciò? Possibile che fra me e quel giovane ufficiale, quel ragazzo, esistano o possano esistere altri rapporti fuorché quelli che esistono con ogni altro conoscente?''. Sorrise con sprezzo e riprese di nuovo il libro; ma ormai davvero non poteva più afferrare quello che leggeva. Passò il tagliacarte sul vetro del finestrino, ne accostò la superficie liscia e fredda alla guancia e si mise quasi a ridere di un'allegrezza che si era impossessata di lei senza ragione. Sentiva che i nervi, come corde, si tendevano sempre di più come su cavicchi avvitantisi. Sentiva che gli occhi le si dilatavano, e che le dita delle mani e dei piedi le si muovevano nervosamente, che qualche cosa dentro le soffocava il respiro e che tutte le immagini e i suoni in quella penombra vacillante la colpivano con una impressionante chiarità. Ad ogni momento era assalita da attimi di dubbio: ``La vettura va avanti o indietro, o sta del tutto ferma? È Annuška vicino a me o una donna estranea? Che cosa c'è lì, sul bracciuolo, una pelliccia o una bestia? E che cosa sono io? Sono proprio io, o un'altra?''. Aveva paura di lasciarsi andare a questo vaneggiamento. Ma qualcosa ve l'attirava e a volontà ella poteva abbandonarvisi o trattenersene. Si alzò per rientrare in sé, gettò indietro lo scialle da viaggio e tolse la pellegrina dal vestito pesante. Per un attimo si riebbe, e capì che l'uomo allampanato che era entrato col cappotto lungo di nanchino al quale mancava un bottone, era un fochista che era venuto a guardare il termometro, e capì che la neve e il vento avevano fatto irruzione dietro di lui attraverso la porta; ma poi di nuovo tutto si confuse. L'uomo allampanato si metteva a rosicchiare qualcosa appoggiato alla parete, la vecchia cominciava ad allungare le gambe per tutta la lunghezza della vettura e la riempiva di un vapore nero; poi qualcosa di pauroso strideva e picchiava come se sbranassero qualcuno, infine un fuoco rosso accecava gli occhi e tutto veniva chiuso da un muro. Ad Anna parve di sprofondare. Eppure tutto questo non era terribile, ma esilarante. La voce di un uomo imbacuccato e ricoperto di neve le gridò qualcosa all'orecchio. Si alzò e ritornò in sé; capì che erano arrivati ad una stazione e che questi era il controllore. Pregò Annuška di darle la pellegrina che s'era tolta e lo scialle, se li mise e si diresse verso lo sportello. 

- Volete scendere? - chiese Annuška. 

- Sì, voglio prendere una boccata d'aria. Qui fa troppo caldo. 

E aprì lo sportello. La tormenta e il vento le si scagliarono addosso contrastandole lo sportello. La cosa la divertì. Aprì lo sportello e venne fuori. Fu come se il vento avesse atteso proprio lei; prese a fischiare con gioia e voleva afferrarla e portarla via, ma lei si aggrappò con una mano ad una colonnina gelata e, trattenendo l'abito, scese sulla banchina e passò dietro la vettura. Il vento era forte sulla scaletta, ma sulla banchina, dietro la vettura, c'era calma. Respirò con gioia, a pieni polmoni, l'aria di neve, gelida, e, in piedi accanto alla vettura, si mise a guardare tutt'intorno la banchina e la stazione illuminata. 

\capitolo{XXX}\label{xxx} 

Una tormenta paurosa s'era scatenata e fischiava fra le ruote della vettura, lungo le colonne, al di là dell'angolo della stazione. Vetture, colonne, uomini; tutto quello che si poteva scorgere veniva ricoperto da un sol lato di neve e sempre di più se ne ricopriva. Per un attimo la tormenta parve calmarsi, ma poi di nuovo si sferrò con raffiche tali che sembrava non si potesse resisterle. Nel frattempo alcune persone corsero e, scambiando allegramente qualche parola, fecero scricchiolare le assi della banchina aprendo e richiudendo continuamente la porta grande. L'ombra contorta di un uomo scivolò sotto i piedi di lei e si udì il rumore di un martello sul ferro\ldots{} ``Telegrafa!'' echeggiò una voce irritata dall'altra parte nel buio della tormenta. ``Favorite qua, n. 28!'' gridarono ancora altre voci e delle persone imbacuccate corsero, ricoperte di neve. Due signori, con le sigarette accese in bocca, le passarono accanto. Ella respirò ancora una volta per prendere aria a sazietà e aveva già tirato fuori la mano dal manicotto per afferrarsi alla colonnina e rientrare in vettura, quando accanto a lei un individuo dal cappotto militare le intercettò la luce vacillante del fanale. Si voltò e in quell'attimo riconobbe il viso di Vronskij. Portando la mano alla visiera, egli s'inchinò e domandò se avesse bisogno di qualcosa e se potesse esserle utile. Anna lo fissò a lungo senza rispondere nulla e, malgrado l'ombra in cui era, vedeva, o le sembrava di vedere, anche l'espressione del viso e degli occhi di lui. Ancora quell'espressione di reverente ammirazione che la sera prima l'aveva tanto impressionata. Più di una volta in quei giorni, e fino a pochi momenti prima, era andata ripetendo a se stessa che Vronskij era per lei uno dei cento giovanotti eternamente identici che s'incontrano dovunque, e che ella mai avrebbe concesso a se stessa di pensare a lui; ma ora, in quel primo attimo dell'incontro, fu presa da un senso di orgoglio gioioso. Non c'era bisogno di chiedere perché fosse là. Lo sapeva così sicuramente come s'egli avesse detto che si trovava là perché voleva essere dov'era lei. 

- Non sapevo che foste in viaggio. Perché viaggiate? - disse, abbassando la mano con la quale stava aggrappata alla colonnina. E un'irrefrenabile gioia e animazione le illuminarono il viso. 

- Perché viaggio? - ripeté lui, guardandola dritto negli occhi. - Voi sapete che io viaggio per essere dove siete voi - disse - e non posso fare altrimenti. 

Nello stesso tempo, come se avesse superato degli ostacoli, il vento spazzò via la neve dai tetti delle vetture, strascinò una lamiera di ferro ch'era riuscito a strappare, e il fischio della locomotiva ruggì, lugubre e cupo. A lei ora tutto l'orrore della tormenta pareva ancora più bello. Egli aveva detto proprio quello che l'anima sua desiderava, ma che la sua ragione temeva. Ella non rispondeva nulla, e sul viso di lei egli scorgeva la lotta interiore. 

- Perdonatemi se vi spiace quello che ho detto - disse umilmente. 

Parlava con cortesia, con rispetto, ma con tanta fermezza e ostinazione che per molto tempo ella non poté rispondere nulla. 

- È male quello che dite, e vi prego, se siete un gentiluomo, dimenticate quello che avete detto; anch'io dimenticherò - disse infine. 

- Non una vostra parola, non un vostro gesto dimenticherò mai, e non posso\ldots{} 

- Basta, basta! - gridò lei, cercando invano di dare un'espressione severa al viso che egli andava scrutando avidamente. E afferratasi con la mano alla colonnina gelida, montò sul predellino ed entrò in fretta nel corridoio della vettura. Ma nel piccolo ingresso si fermò per riflettere a quello che era accaduto. Non ricordava né le parole proprie, né quelle di lui, ma ebbe la sensazione che quella conversazione di pochi istanti li avesse terribilmente avvicinati e ne era spaventata e felice. Dopo esser rimasta in piedi per qualche secondo, entrò nello scompartimento e sedette al proprio posto. Quello stato di tensione che l'aveva tormentata poco prima non solo si rinnovò, ma aumentò sino a farle temere che da un momento all'altro si spezzasse in lei qualcosa di troppo teso. Non dormì tutta la notte. Ma in quella tensione e in quel vaneggiamento che le riempivano la mente, non c'era nulla di spiacevole e di tetro, al contrario, c'era qualcosa di gioioso e di eccitante. All'alba si assopì nella poltrona, e quando si svegliò era giorno chiaro e il treno si avvicinava a Pietroburgo. Il pensiero della casa, del marito, del figlio, le faccende di quel giorno e dei seguenti s'impossessarono subito di lei. 

A Pietroburgo, non appena il treno si fu fermato ed ella uscì, il primo viso che attirò la sua attenzione fu quello del marito: ``Ah, Dio mio, perché ha le orecchie fatte a quel modo?'' pensò guardando la figura di lui fredda e rappresentativa e in particolare le cartilagini delle orecchie che sostenevano le falde del cappello tondo e che in quel momento la colpivano. Egli, appena la vide, le andò incontro, atteggiando le labbra al sorriso canzonatorio che gli era abituale e guardandola diritto con i suoi grandi occhi stanchi. Una sensazione sgradevole le strinse il cuore quando incontrò lo sguardo di lui ostinato e stanco: come se avesse voluto vederlo diverso. La colpì soprattutto quello scontento di sé che aveva provato nell'incontrarlo. Era, questa, una sensazione da tempo provata, simile a quella sua mancanza di lealtà nei rapporti col marito; ma prima questa sensazione ella non la notava, ora invece la percepiva con chiarezza e con pena. 

- Be', lo vedi, hai un marito tenero, tenero come al primo anno di matrimonio: bruciava dal desiderio di vederti - disse lui con la sua voce sottile e strascicata e con quel tono che quasi sempre usava con lei, un tono di canzonatura verso chi avesse parlato così per davvero. 

- Serëza sta bene? - domandò lei. 

- E questa è tutta la tua ricompensa per il mio ardore? - disse lui. - Sta bene, sta bene\ldots{} 

\capitolo{XXXI}\label{xxxi} 

Per tutta quella notte Vronskij non tentò neppure d'addormentarsi. Sedeva sulla sua poltrona, ora con gli occhi fissi davanti a sé, ora osservando quelli che entravano e uscivano; e se anche prima egli colpiva e disorientava le persone che non lo conoscevano, per quella sua aria di imperturbabile indifferenza, ora sembrava ancor più pieno e soddisfatto di sé. Guardava agli uomini come a cose. Un impiegato del tribunale del distretto, un giovanotto nervoso seduto di fronte a lui, prese a detestarlo per quella sua aria. Il giovanotto accendeva la sigaretta a quella di Vronskij, cominciava a parlare con lui, lo urtava perfino per fargli sentire che non era una cosa, ma Vronskij lo guardava così come si guarda un fanale, e il giovanotto si contorceva, sentendo di perdere il dominio di se stesso, sottoposto alla pressione di quel mancato riconoscimento umano. 

Vronskij non vedeva nulla e nessuno. Si sentiva un dominatore, non perché credesse d'aver fatto colpo su Anna (a questo egli non credeva ancora), ma perché l'impressione che ella aveva prodotto su di lui lo rendeva felice e orgoglioso. 

Che cosa sarebbe venuto fuori da tutto questo, non lo sapeva, e non lo immaginava neppure. Sentiva che tutte le sue forze, fino ad ora rilasciate e disperse, si erano fuse e orientate con spaventosa energia verso un unico fine beato. E ne era felice. Sapeva solo di averle detto la verità, dicendole che andava là dov'era lei; sapeva che tutta la felicità della sua vita, l'unico senso della vita lo trovava adesso nel veder lei, nell'ascoltar lei. E quando era uscito dalla vettura a Bologovo per bere dell'acqua di seltz, e aveva visto Anna, involontariamente la prima cosa che aveva detto era stata proprio ciò che pensava. Ed era felice di averglielo detto, era felice ch'ella lo sapesse e ci pensasse. Non dormì tutta la notte. Da quando era rientrato in vettura, senza mutar posto, non aveva cessato di riandare con la mente a tutti gli atteggiamenti in cui l'aveva vista, a tutte le sue parole, mentre nell'immaginazione volteggiavano le figurazioni di un possibile futuro che lo facevano venir meno. 

Quando, a Pietroburgo, uscì dalla vettura, si sentiva, dopo la notte insonne, vivido e fresco come dopo un bagno freddo. Si fermò presso la vettura, ad aspettarla. ``La vedrò ancora una volta - si disse, sorridendo involontariamente - vedrò la sua andatura, il suo viso: dirà qualcosa, volgerà il capo, guarderà, riderà, forse''. Ma prima ancora di veder lei, vide il marito accompagnato con deferenza dal capostazione tra la folla. ``Ah, già, il marito!''. Solo in quel momento Vronskij capì per la prima volta con chiarezza che il marito era una persona legata a lei. Sapeva ch'ella aveva un marito, ma non credeva alla sua esistenza, e ci credette in pieno solo quando lo vide, con quella sua testa, con quelle sue spalle, con le gambe nei pantaloni neri; specialmente quando vide con quale senso di proprietà egli prendeva tranquillamente il braccio di lei. 

Nel vedere Aleksej Aleksandrovic, col viso rasato di fresco alla pietroburghese e col suo aspetto rigidamente sicuro di sé, col cappello a falde larghe, la schiena un po' curva, ci credette, e provò una sensazione sgradevole, simile a quella di un uomo che, tormentato dalla sete, e pervenuto a una fonte vi trovi un cane, una pecora o un maiale che, bevendo, ne abbia intorbidita l'acqua. L'andatura di Aleksej Aleksandrovic, che dimenava tutto il bacino movendo le gambe ad angolo ottuso, dava fastidio in modo particolare a Vronskij. Riconosceva solo a se stesso l'indubitabile diritto di amare lei. Ma lei era sempre la stessa, e la sua apparizione agì su di lui, come sempre, animandolo fisicamente, eccitandolo ed empiendogli l'anima di gioia. Ordinò al servitore tedesco, che gli veniva incontro correndo dalla seconda classe, di prender la roba e di andar via, e intanto si avvicinò a lei. Vide il primo incontro fra marito e moglie e osservò, con la penetrazione di chi ama, la leggera ombra di soggezione con la quale ella parlava col marito. ``No, non lo ama e non può amarlo'' decise fra sé e sé. Mentre si accostava alle spalle di Anna Arkad'evna, notò con gioia ch'ella aveva sentito il suo avvicinarsi e che stava per girarsi, ma che poi, riconosciutolo, si era rivolta nuovamente al marito. 

- Avete passata bene la notte? - chiese, inchinandosi dinanzi a lei e al marito insieme, lasciando ad Aleksej Aleksandrovic la facoltà di prender per sé quell'inchino e d'accettarlo oppur no a suo piacimento. 

- Grazie, molto bene - rispose lei. 

Il suo viso sembrava stanco e non v'era quel giuoco d'animazione che urgeva ora nel riso ora negli occhi; solo per un attimo, mentre lo guardava, qualcosa balenò nei suoi occhi, e sebbene questo fuoco si spegnesse subito, egli fu felice di quell'attimo. Ella guardò il marito per vedere se conosceva o no Vronskij. Aleksej Aleksandrovic guardava Vronskij con disappunto, cercando distrattamente di ricordarsi chi fosse. La sicurezza e la tranquillità di Vronskij in quel momento urtarono, come la falce nella selce, contro la fredda sicurezza di Aleksej Aleksandrovic. 

- Il conte Vronskij - disse Anna. 

- Ah, ci conosciamo, mi pare - disse con indifferenza Aleksej Aleksandrovic, dandogli la mano. - All'andata hai viaggiato con la madre, al ritorno col figlio - disse, pronunciando con precisione, come se elargisse un rublo ad ogni parola. - Probabilmente, tornate da una licenza? - disse e, senza aspettar risposta, si voltò alla moglie in tono scherzoso. - Dunque, molte lacrime sono state versate a Mosca al momento della separazione? 

Col rivolgersi alla moglie aveva fatto intendere a Vronskij che desiderava restar solo e, giratosi verso di lui, si toccò il cappello; ma Vronskij disse ancora ad Anna Arkad'evna: 

- Spero di aver l'onore di venire da voi - disse. 

Aleksej Aleksandrovic lo guardò con occhi stanchi. 

- Molto lieto - disse freddo; - riceviamo il lunedì. - Poi, dopo aver lasciato andar via definitivamente Vronskij, disse alla moglie: - È stato proprio bene che io abbia avuto mezz'ora di tempo per venirti a prendere e dimostrarti la mia tenerezza - egli continuò nello stesso tono scherzoso. 

- Tu insisti troppo su questa tua tenerezza, perché io possa apprezzarla - disse lei con lo stesso tono scherzoso, prestando involontariamente orecchio al suono dei passi di Vronskij che camminava dietro di loro. ``Ma che me ne importa?'' pensò, e cominciò a chiedere al marito come era stato senza di lei Serëza . 

- Oh, benissimo! Mariette dice che è molto caro e\ldots{} devo darti un dispiacere\ldots{} non ha sentito nostalgia di te, non certo come tuo marito. Ma ancora una volta merci, amica mia, di avermi regalato una giornata. Il nostro caro samovar sarà entusiasta - egli chiamava ``samovar'' la famosa contessa Lidija Ivanovna, perché sempre e per tutto si agitava e accalorava. - Ha domandato di te. E sai, ti posso dare un consiglio? Dovresti andare da lei oggi stesso. Perché le duole il cuore per ogni cosa. Ora poi, oltre tutti i suoi affanni, è preoccupata della riconciliazione degli Oblonskij. 

La contessa Lidija Ivanovna era un'amica di suo marito e il centro di uno di quei circoli della società di Pietroburgo al quale Anna era legata più intimamente che non a tutti gli altri per mezzo di suo marito. 

- Ma se le ho scritto. 

- Ma a lei occorre saper tutto per filo e per segno. Cerca di andarci, se non sei stanca, amica mia. Ora la carrozza te la farà venire Kondratij, e io vado al comitato. Finalmente non mangerò più solo - continuò Aleksej Aleksandrovic non più in tono scherzoso. - Tu non puoi credere come sia abituato\ldots{} 

E stringendole a lungo la mano, la fece salire in carrozza con un sorriso particolare. 

\capitolo{XXXII}\label{xxxii} 

La prima persona che venne incontro ad Anna in casa fu il figlio. Si lanciò verso di lei giù per la scala, malgrado le grida della governante, chiamando con un entusiasmo disperato: ``Mamma, mamma!''. Giunto di corsa fino a lei, le si appese al collo. 

- Ve lo dicevo io che era la mamma! - gridava alla governante. - Lo sapevo! 

Anche il figlio, così come il marito, produsse in Anna un senso di delusione. Se lo immaginava più bello di quanto non fosse in realtà. E dovette discendere fino alla realtà per compiacersi di come era. Ma in fondo anche così era delizioso, con i riccioli biondi, gli occhi azzurri e le gambette piene e ben fatte nelle calze attillate. Anna provava una gioia quasi fisica nel sentirsi vicino a lui e una tenerezza e una calma morale quando incontrava lo sguardo suo leale, fiducioso e tenero e ne ascoltava le domande ingenue. Tirò fuori i regali che avevano mandato i bambini di Dolly e raccontò al figlio come a Mosca ci fosse una bimba, una certa Tanja, la quale sapeva già leggere e insegnare perfino agli altri bambini. 

- Ma forse io sono meno bravo di lei? - chiese Serëza. 

- Per me tu sei il più bravo di tutti al mondo. 

- Lo so - disse Serëza, sorridendo. 

Anna aveva appena fatto in tempo a prendere il caffè che le annunciarono la contessa Lidija Ivanovna. La contessa Lidija Ivanovna era una donna alta e grossa, dal colorito giallastro e malato e dagli occhi neri, belli e pensosi. Anna le voleva bene, ma quel giorno era come se la vedesse per la prima volta con tutti i suoi difetti. 

- Dunque, amica mia, avete portato il ramoscello d'olivo? - chiese la contessa Lidija Ivanovna entrando nella stanza. 

- Già, tutto si è concluso; ma la cosa non era poi così grave come credevamo - rispose Anna. - In generale, la mia belle soeur è troppo impulsiva. 

Ma la contessa Lidija Ivanovna, che si interessava di tutto quello che non la riguardava, e aveva l'abitudine di non ascoltare mai quello che avrebbe potuto interessarla, interruppe Anna. 

- Ah, c'è molto dolore e molta cattiveria nel mondo e io oggi sono così sfinita. 

- Che c'è? - domandò Anna, cercando di trattenere un sorriso. 

- Comincio a stancarmi di spezzare inutilmente lance in favore della verità, e a volte mi sento proprio snervata. L'affare delle Piccole Suore - era questa un'istituzione religioso-patriottica - andava già a meraviglia; ma con quei signori non si può far nulla - aggiunse in tono di ironica rassegnazione. - Si sono afferrati a un'idea, l'hanno travisata e, dopo tutto, ragionano con molta meschinità e piccineria. Due o tre persone sole, fra le quali vostro marito, comprendono il significato di quest'opera; ma gli altri la lasciano cadere. Ieri mi ha scritto Pravdin\ldots{} 

Pravdin era un noto panslavista all'estero e la contessa Lidija Ivanovna riferì il contenuto della sua lettera. 

Dopo di che la contessa parlò anche delle contrarietà e delle insidie contro la questione dell'unione delle chiese, e se ne andò in fretta, perché in quel giorno doveva andare alla seduta di un'associazione e al comitato slavo di beneficenza. 

``Certo questo suo modo di fare è lo stesso di prima, ma perché prima non lo notavo? - si diceva Anna. - O forse oggi è molto eccitata? Comunque, è ridicolo: il suo scopo è la virtù, è una cristiana, ma è sempre in collera e vede sempre nemici in nome della cristianità e della virtù''. 

Dopo la contessa Lidija Ivanovna venne un'amica, la moglie di un direttore, la quale raccontò tutte le novità della città. Alle tre se ne andò anche lei, promettendo di venire a pranzo. Aleksej Aleksandrovic era al ministero. Rimasta sola, Anna occupò il tempo, fino all'ora del pranzo, nell'assistere al pasto del figlio (egli pranzava separatamente), nel mettere in ordine le sue cose, nel leggere e rispondere ai biglietti e alle lettere che le si erano ammonticchiati sullo scrittoio. 

Il senso di inspiegabile vergogna che aveva provato in viaggio e l'agitazione erano completamente scomparsi. Nelle condizioni abituali di vita si sentiva di nuovo sicura e irreprensibile. 

Ricordava con stupore il suo stato del giorno prima. ``Ma cos'era mai? Nulla. Vronskij ha detto una sciocchezza alla quale è stato facile porre fine, e io ho risposto proprio come conveniva. Parlare a mio marito non si deve e non si può. Parlarne, sarebbe come dare importanza a ciò che non ne ha''. Ricordò d'aver raccontato al marito di un accenno di dichiarazione che le aveva fatto a Pietroburgo un giovane dipendente di lui. Aleksej Aleksandrovic le aveva risposto che ogni donna, vivendo nel gran mondo, poteva essere esposta a cose simili, ma ch'egli fidava completamente nel suo tatto e che mai si sarebbe permesso di abbassare lei e se stesso alla gelosia. ``Dunque, non c'è ragione di parlarne. Ma, grazie a Dio, non c'è neanche nulla da dire'' si disse. 

\capitolo{XXXIII}\label{xxxiii} 

Aleksej Aleksandrovic tornò dal ministero alle quattro, ma, come spesso gli accadeva, non fece in tempo a passare da lei. Entrò nello studio a ricevere i sollecitatori che aspettavano e a firmare alcune carte portate dal capogabinetto. A pranzo (dai Karenin erano invitati a pranzo sempre un tre persone) vennero: una vecchia cugina di Aleksej Aleksandrovic, il direttore del dipartimento con la moglie e un giovanotto raccomandato ad Aleksej Aleksandrovic per un posto. Anna entrò in salotto per intrattenerli. Alle cinque in punto (l'orologio di bronzo in stile Pietro I non aveva finito di battere il quinto tocco) entrò Aleksej Aleksandrovic in cravatta bianca e in frac con due decorazioni, perché subito dopo pranzo doveva andar via. Ogni minuto della vita di Aleksej Aleksandrovic era impegnato e ripartito. E per riuscire a sbrigare quello che doveva fare ogni giorno, si atteneva alla più stretta puntualità. ``Senza fretta, ma senza tregua'' era il suo motto. Entrò frettoloso in sala, salutò tutti e, sorridendo alla moglie, sedette. 

- Sì, è finita la mia solitudine. Non puoi credere come sia spiacevole - egli marcò la parola ``spiacevole'' - pranzare da solo. 

A pranzo parlò con la moglie delle faccende di Mosca; con un sorriso canzonatorio chiese di Stepan Arkad'ic; ma la conversazione, prevalentemente generale, si aggirò sulle questioni amministrative e sociali di Pietroburgo. Dopo pranzo egli passò una mezz'ora con gli ospiti e, stretta di nuovo la mano alla moglie con un sorriso, uscì e andò al consiglio. Anna non andò questa volta né dalla principessa Betsy Tverskaja che, saputo del suo ritorno, l'aveva invitata per la serata, né a teatro dove quella sera aveva un palco. Non andò soprattutto perché il vestito sul quale contava non era pronto. Quando gli ospiti se ne andarono, dato uno sguardo generale al guardaroba, Anna s'indispettì molto. Prima della sua partenza per Mosca ella, che in genere era abilissima nel vestirsi senza spendere eccessivamente, aveva dato a rimodernare tre abiti alla sarta. Bisognava rifare i vestiti in modo da non farli riconoscere e dovevano essere pronti già da tre giorni. Invece due vestiti non lo erano affatto ed il terzo non era riuscito come avrebbe voluto lei. La sarta era venuta a giustificarsi e aveva sostenuto che in quel modo andava bene, e Anna si era indispettita tanto da provarne rimorso, dopo, al ricordo. Per rasserenarsi completamente era andata nella camera del bambino e aveva passato tutta la serata col figlio; lo aveva messo lei stessa a letto, gli aveva fatto il segno della croce e gli aveva rimboccato le coperte. Era felice di non essere andata in nessun posto e di aver passato così bene la serata. Si sentiva leggera e tranquilla e vedeva chiaramente che tutto quello che in viaggio le era parso così importante non era che uno degli insignificanti, comuni casi della vita mondana di cui non aveva da vergognarsi, né dinanzi a sé stessa, né dinanzi ad altri. Sedette presso il camino con in mano il romanzo inglese e aspettò il marito. Alle nove e mezzo in punto si udì la sua scampanellata ed egli entrò nella stanza. 

- Finalmente sei tu! - disse lei, tendendogli la mano. Egli le baciò la mano e le sedette accanto. 

- Vedo che il tuo viaggio è andato bene, nel complesso - disse. 

- Sì, molto - rispose lei, e cominciò a raccontargli tutto dal principio: il viaggio con la Vronskaja, l'arrivo, la disgrazia alla stazione. Dopo disse della sua impressione di pena provata prima per il fratello e poi per Dolly. 

- Io non credo che si possa scusare un uomo simile, anche se è tuo fratello - disse Aleksej Aleksandrovic severo. 

Anna sorrise. Capì che egli aveva detto ciò proprio per mostrare che le considerazioni di parentela non potevano trattenerlo dall'esprimere con franchezza la propria opinione. Conosceva questo tratto in suo marito e le piaceva. 

- Sono contento che tutto sia finito felicemente e che tu sia tornata - continuò. - Ebbene, che cosa dicono della nuova tesi che ho fatto passare al consiglio? 

Anna non aveva sentito dir nulla di questa tesi e si pentì d'aver dimenticato con tanta leggerezza quello che per lui era così importante. 

- Qui, al contrario, ha fatto molto scalpore - disse lui con un sorriso di compiacimento. 

Ella vedeva che Aleksej Aleksandrovic voleva comunicarle qualcosa che lo lusingava a proposito di quella questione, e, interrogandolo, lo portò a raccontare. 

Con lo stesso sorriso di compiacimento egli parlò delle ovazioni che gli erano state fatte in seguito all'approvazione della tesi. 

- Ne sono stato molto contento. Questo dimostra che finalmente da noi comincia a consolidarsi un'opinione ragionevole e decisiva su questa faccenda. 

Dopo aver preso il suo secondo bicchiere di tè con panna e pane, Aleksej Aleksandrovic si alzò e si diresse nello studio. 

- E tu non sei andata in nessun posto? Ti sarai annoiata, probabilmente - disse. 

- Oh, no! - rispose lei, alzandosi dietro di lui per accompagnarlo nello studio. - Cosa mai leggi ora? - domandò. 

- Sto leggendo la Poésie des enfers del Duc de Lille - rispose lui. - È un libro molto interessante. 

Anna sorrise, come si sorride alla debolezza delle persone care, e, posto il braccio sotto quello di lui, lo accompagnò fino alla soglia dello studio. Conosceva la sua abitudine, che era ormai una necessità, di leggere la sera. Sapeva che, malgrado i doveri d'ufficio che assorbivano quasi tutto il suo tempo, egli considerava doveroso seguire quanto di più notevole appariva nel mondo della cultura. Sapeva pure che in realtà lo interessavano solo i libri di politica, di filosofia e di teologia; che l'arte era del tutto estranea alla sua natura, ma che nonostante questo, o meglio per questo, Aleksej Aleksandrovic non trascurava nulla che avesse successo in questo campo e considerava suo dovere leggere tutto. Sapeva che nel campo della politica, della filosofia, della teologia Aleksej Aleksandrovic aveva dei dubbi o faceva delle ricerche; ma che nelle questioni di arte e di poesia, in particolare nella musica, del cui senso era completamente sprovvisto, aveva le più ristrette e tenaci convinzioni. Gli piaceva parlare di Shakespeare, di Raffaello, di Beethoven, del valore delle nuove correnti poetiche e musicali che venivano tutte classificate da lui con una logica molto chiara. 

- E Dio sia con te - disse lei presso la porta dello studio dove già gli erano stati preparati un paralume sulla candela e una caraffa d'acqua accanto alla poltrona. - Io intanto scriverò a Mosca. 

Egli le strinse la mano e la baciò di nuovo. 

``Però è un brav'uomo, leale, di buon cuore e notevole nel suo campo - si andava dicendo Anna, tornata in camera sua; quasi a difenderlo di fronte a qualcuno che lo accusasse e che dicesse a lei che non lo si poteva amare. - Ma come mai ha le orecchie che gli sporgono così stranamente in fuori? Forse si è tagliato i capelli''. 

A mezzanotte in punto, quando Anna era ancora seduta allo scrittoio terminando una lettera a Dolly, si udirono dei passi eguali e Aleksej Aleksandrovic, in pantofole, lavato e pettinato, col libro sotto al braccio, si accostò a lei. 

- È ora, è ora - disse, sorridendo in modo particolare, e si diresse in camera. 

``E quale diritto aveva di guardarlo così?'' pensò Anna, ricordando lo sguardo di Vronskij su di Aleksej Aleksandrovic. 

Spogliatasi, Anna entrò in camera, ma sul suo volto non solo non c'era più quell'animazione che durante il soggiorno a Mosca le balenava tra gli occhi e il riso, ma al contrario il fuoco sembrava ormai spento in lei, oppure nascosto in qualche parte, lontano. 

\capitolo{XXXIV}\label{xxxiv} 

Partendo da Pietroburgo, Vronskij aveva lasciato il suo grande appartamento nella Morskaja all'amico e compagno carissimo Petrickij. 

Petrickij era un giovane tenente non di alto lignaggio, e non solo non ricco, ma affogato nei debiti, sempre brillo verso sera e spesso agli arresti per varie scabrose e ridicole storie, ma amato dai compagni e dai superiori. Verso le undici, tornando a casa dalla stazione, Vronskij vide dinanzi al portone una vettura da nolo a lui nota. Alla sua scampanellata, attraverso la porta, sentì un riso di uomini, il balbettio di una voce femminile e il grido di Petrickij: ``Se è qualche manigoldo, che non entri!''. Vronskij ordinò all'attendente di annunciarlo, e pian piano entrò nella prima stanza. La baronessa Shilton, l'amica di Petrickij, col viso roseo e chiaro e tutta luccicante nel raso lilla del vestito, sedeva alla tavola rotonda, intenta a far bollire il caffè, e come un canarino riempiva tutta la stanza della sua parlata parigina. Petrickij in cappotto e il capitano di cavalleria Kamerovskij in uniforme completa, reduci probabilmente dal servizio, sedevano vicino a lei. 

- Bravo! Vronskij! - gridò Petrickij, alzandosi e facendo rumore con la sedia. - Il padrone in persona! Baronessa, del caffè dalla caffettiera nuova! Ecco, non ti si aspettava proprio! Spero che tu sia contento del nuovo ornamento del tuo studio - disse indicando la baronessa. - Vi conoscete, vero? 

- Altro che - disse Vronskij sorridendo allegramente e stringendo la piccola mano della baronessa. - E come! Una vecchia amica! 

- Be', tornate a casa da un viaggio - disse la baronessa. - E allora io me ne vado via di corsa. Ah, me ne vado via in questo momento, se do fastidio. 

- Siete a casa vostra, baronessa - disse Vronskij. - Salve, Kamerovskij - soggiunse, stringendo freddamente la mano di Kamerovskij. 

- Ecco, voi non sapete mai dirmi delle cose così gentili - disse la baronessa rivolta a Petrickij. 

- No, perché? Dopo pranzo vedrete che non ne dirò di peggiori. 

- Già, dopo pranzo non c'è merito! Su, allora, vi darò del caffè; andate intanto a lavarvi e a mettervi in ordine - disse la baronessa sedendosi di nuovo e girando con premura una vite nella caffettiera nuova. 

- Pierre, datemi il caffè - disse a Petrickij che chiamava così dal cognome Petrickij, senza nascondere i suoi rapporti con lui. - Ne aggiungo dell'altro. 

- Ma lo sciupate! 

- No, che non lo sciupo. Su, e la vostra sposa? - disse subito la baronessa interrompendo la conversazione di Vronskij col compagno. - Noi qui vi abbiamo ammogliato. Avete portato vostra moglie? 

- No, baronessa. Zingaro son nato e zingaro morirò. 

- Tanto meglio, tanto meglio. Qua la mano. 

E la baronessa, senza lasciare andare Vronskij, prese a raccontargli i suoi ultimi progetti di vita, infiorandoli di scherzi, e chiedendogli consigli. 

- Lui non vuole ancora consentire al divorzio. E allora che debbo fare? - ``Lui'' era suo marito. - Voglio iniziare il processo, perché ho bisogno di un patrimonio mio. Cosa mi consigliate? Kamerovskij, badate al caffè\ldots{} esce fuori; vedete, io sto parlando d'affari. Capite forse quest'assurdità? io gli sarei infedele - diceva lei con sprezzo - e lui per questo vuole usufruire della mia proprietà. 

Vronskij ascoltava volentieri l'allegro cinguettio di quella donna carina; le diceva di sì, le dava consigli scherzando e, in complesso, andava riprendendo rapidamente il suo tono abituale con le donne di questa specie. Nel suo mondo pietroburghese tutte le persone si dividevano in due categorie perfettamente opposte. Una, la categoria inferiore, si componeva di persone comuni, sciocche e soprattutto ridicole, le quali credevano che il marito dovesse vivere soltanto con la donna con la quale s'era sposato, che una ragazza dovesse essere innocente, la donna pudica, l'uomo virile, temperato e forte, che bisognasse educare i propri figli, provvedere al proprio pane, pagare i debiti e altre sciocchezze simili. Questa era la categoria delle persone fuori moda e ridicole. Ma c'era un'altra categoria, quella delle persone alla moda, alla quale tutti loro appartenevano, e nella quale bisognava essere soprattutto belli, eleganti, spenderecci, arditi, allegri e capaci di abbandonarsi a qualsiasi passione senza arrossire e ridendosi di tutto. 

Vronskij era rimasto stordito solo il primo momento, dopo le impressioni che aveva riportato da Mosca di un mondo del tutto diverso; ma poi, subito, come se avesse infilato i piedi in un vecchio paio di pantofole, entrò nell'allegro e piacevole suo mondo di prima. 

Il caffè infatti non arrivò neanche a bollire, che schizzò tutti e andò di fuori, producendo proprio quello che occorreva: versandosi su di un tappeto di valore e sul vestito della baronessa, offrì il pretesto al chiasso e al riso. 

- Su, allora, addio, altrimenti non vi laverete mai e sulla mia coscienza graverà il più grosso delitto d'un uomo per bene: la sporcizia. Dunque, voi mi consigliate di mettergli il coltello alla gola? 

- Proprio così, e in modo tale che la vostra manina si trovi vicina alle sue labbra. Egli bacerà la vostra mano e tutto andrà nel modo migliore - disse Vronskij. 

- Allora a stasera, al Teatro Francese! - E frusciando col vestito, scomparve. 

Kamerovskij si alzò anche lui, e Vronskij, senza aspettare che fosse uscito, gli diede la mano e si diresse nel bagno. Mentre si lavava, Petrickij gli descrisse in breve la propria situazione, tanto mutata dopo la partenza di Vronskij. Denaro niente. Il padre aveva detto che non ne avrebbe dato e che non avrebbe pagato debiti. Il sarto lo voleva fare arrestare e anche un altro lo minacciava decisamente di farlo schiaffar dentro. Il comandante del reggimento aveva dichiarato che, se tutti questi scandali non fossero finiti, egli avrebbe dovuto dare le dimissioni. La baronessa gli era venuta a noia come una radica amara, e soprattutto perché voleva continuamente dargli del denaro; ma ce n'era una, che egli poi avrebbe mostrata a Vronskij, una meraviglia, un amore, di perfetto stile orientale, ``genre schiava Rebecca, capisci''. Anche con Berkošëv aveva litigato e gli voleva mandare i padrini, ma, naturalmente, non ne sarebbe venuto fuori nulla. In complesso tutto era eccellente, e straordinariamente allegro. E senza dare all'amico la possibilità di approfondire i particolari di questa situazione, Petrickij si diede a raccontargli tutte le novità interessanti. Ascoltando i racconti così noti di Petrickij, in quell'atmosfera ancor più nota dell'appartamento che occupava da tre anni, Vronskij provava un piacevole senso di ritorno all'abituale spensierata vita di Pietroburgo. 

- Non può essere! - gridò, lasciando il pedale del lavabo, nel quale bagnava il collo rosso e sano. - Non può essere! - gridò alla notizia che Lora s'era unita con Mileev e aveva piantato Fertigov. - E lui, sempre così balordo e soddisfatto? Su, e di Buzulukov che ne è? 

- Ah, con Buzulukov c'è stata una storia, una delizia! - gridò Petrickij. - Dunque, la passione sua sono i balli, e non se ne perde nemmeno uno di quelli a corte. Era andato al gran ballo con l'elmo nuovo. Hai visto gli elmi nuovi? Molto belli, leggeri. Eccolo, è lì in piedi\ldots{} Su, ascolta. 

- Sì, che ascolto - rispose Vronskij, fregandosi con un asciugamano a spugna. 

- Passa una granduchessa con un ambasciatore, e, per disgrazia sua, il discorso cade sugli elmi nuovi. La granduchessa vuole mostrare l'elmo nuovo. Guardano, e il nostro giovincello sta lì impalato - Petrickij lo rifaceva così come stava, lì ritto con l'elmo sotto al braccio. - La granduchessa gli chiede di darle l'elmo. Lui, niente. Che succede? Non fanno che ammiccargli, fargli gesti, aggrottar le sopracciglia. Dàglielo. Non lo dà. Pare un morto. Ti puoi figurare\ldots{} Ma quello\ldots{} come si chiama\ldots{} vuol prendere l'elmo\ldots{} lui niente, non lo dà! Quello glielo strappa, lo dà alla granduchessa: ``Ecco l'elmo nuovo'' dice la granduchessa. Volta l'elmo, e figurati, dall'elmo, giù una pera, dei confetti, due libbre di confetti. Li aveva raccolti, poverino! 

Vronskij scoppiò a ridere. E a lungo dopo, quando già parlavano d'altro, se gli tornava in mente l'elmo, scoppiava a ridere del suo riso sano che metteva in mostra i denti regolari e forti. 

Sapute tutte le novità, Vronskij, con l'aiuto del servitore, si mise in uniforme per andare a presentarsi. Voleva poi, dopo essersi presentato, passare dal fratello, da Betsy e fare alcune visite per cominciare a entrare in quella società nella quale avrebbe potuto incontrare la Karenina. Come sempre a Pietroburgo, uscì di casa con l'intenzione di rientrarvi a notte alta. 

\parte{PARTE SECONDA}\label{parte-seconda} 

\capitolo{I}\label{i-1} 

Alla fine dell'inverno si tenne un consulto in casa Šcerbackij per accertare quali fossero le condizioni di salute di Kitty e decidere cosa fare per ristabilirne le forze sempre più deboli. Il medico curante le aveva prescritto l'olio di fegato di merluzzo, poi il ferro, poi il nitrato di argento; ma poiché né questo, né quello, né l'altro avevano giovato ed egli consigliava di condurla all'estero, nella primavera fu fatto venire un medico di grido. Costui, bell'uomo ancor giovane, volle visitare l'ammalata. Insisteva con particolare compiacimento sul fatto che il pudore verginale è solo un residuo di barbarie, e che non vi è nulla di sconveniente che un medico, se pur non del tutto vecchio, visiti una ragazza tastandone il corpo svestito. Gli pareva del tutto naturale, gli capitava ogni giorno, e non sentiva e non pensava che potesse esservi nulla di male: e perciò considerava il pudore di una fanciulla non solo un residuo di barbarie, ma un'offesa alla propria persona. 

Era necessario piegarvisi, perché, sebbene anche gli altri medici avessero frequentato la stessa scuola e studiato sugli stessi libri e tutti fossero in possesso di una stessa scienza, e pur avendo costui presso alcuni fama di medico inetto, tuttavia nella casa della principessa e nella sua cerchia, chi sa perché, si riteneva che solo questo medico famoso sapesse qualcosa di speciale e solo lui potesse salvare Kitty. Dopo aver visitato e tastato attentamente l'ammalata, smarrita e stordita per la vergogna, il medico famoso, lavatesi accuratamente le mani, rimase in piedi nel salotto a parlare col principe. Il principe aggrottava le sopracciglia e tossiva nell'ascoltarlo. Come uomo vissuto, non sciocco e di sana costituzione, non credeva alla medicina e nell'animo suo si irritava contro tutta quella commedia, tanto più che egli era forse il solo a capire in pieno la causa del malanno di Kitty. ``Eccolo, lo spadellatore!'' pensava, adattando nel pensiero il termine venatorio al medico di grido e ascoltandone le dissertazioni sui sintomi della malattia della figlia. Il medico, da parte sua, tratteneva a stento un'espressione di dispregio verso il vecchio gentiluomo e si abbassava con degnazione al livello dell'intelligenza di lui. Capiva che col vecchio non c'era nulla da fare e che in quella casa il capo era la madre. Dinanzi a costei si proponeva quindi di profondere le sue millanterie. In quel momento la principessa entrò in salotto col medico curante. Il principe si allontanò, cercando di non far notare quanto per lui fosse ridicola tutta quella commedia. La principessa era smarrita e non sapeva cosa fare. Si sentiva colpevole di fronte a Kitty. 

- Ebbene, dottore, decidete la nostra sorte - disse la principessa. - Ditemi tutto. - E voleva dire: ``C'è speranza?'', ma le labbra le tremarono, e non poté pronunciare la domanda. - Dunque, dottore? - 

- Subito, principessa; conferirò con il mio collega e poi avrò l'onore di dirvi la mia opinione. 

- Allora vi dobbiamo lasciare? 

- Come volete. 

La principessa, dopo aver sospirato, uscì. 

Quando i dottori rimasero soli, il medico curante cominciò timidamente a sottoporre la sua opinione che consisteva nell'ammettere un principio di processo tubercolare, ma\ldots{} e via di seguito. Il medico famoso lo ascoltava e, nel mezzo del discorso, guardò l'orologio d'oro massiccio. 

- Già, - disse - ma\ldots{} 

Il medico curante tacque rispettosamente, a metà discorso. 

- Come voi sapete, un principio di processo tubercolare noi non possiamo diagnosticarlo; fino alla comparsa delle caverne non vi è nulla di positivo. Possiamo fare solamente delle ipotesi. E sintomi ce ne sono: denutrizione, eccitamento nervoso, ecc. La questione si pone in questi termini: supposto un processo tubercolare, che cosa bisogna fare per sostenere la nutrizione? 

- Ma voi sapete, del resto, come in questi casi si nascondano sempre ragioni morali, spirituali - si permise di far presente, con un sorriso delicato, il medico curante. 

- Già, s'intende - rispose il medico famoso, dopo aver guardato di nuovo l'orologio. - Ditemi, vi prego, è stato rimesso il ponte Jauzskij o bisogna ancora fare il giro? - chiese. - Ah, è a posto. Allora potrò trovarmi là in venti minuti. Dunque, dicevamo, la questione si pone in questi termini: sostenere la nutrizione e sistemare i nervi. L'una cosa è legata all'altra; bisogna battere sulle due parti del cerchio. 

- E il viaggio all'estero? - chiese il medico curante. 

- Io son nemico dei viaggi all'estero. E guardate un po': se c'è un principio di processo tubercolare, cosa che non possiamo sapere, il viaggio all'estero non aiuta. È indispensabile un mezzo che sostenga la nutrizione senza far danno. 

Il medico curante ascoltava attento e deferente. 

- Ma in favore del viaggio all'estero io farei notare il cambiamento di abitudini, l'allontanamento da quanto può suscitare ricordi. E poi la madre lo desidera - disse. 

- Ah, allora, in tal caso, che vadano pure; badino, però, che quei ciarlatani di tedeschi non abbiano a nuocere loro. Che si attengano\ldots{} Ma che vadano pure. 

E guardò di nuovo l'orologio. 

- Oh, è già ora - e andò verso la porta. 

Il medico famoso annunciò alla principessa (un senso di convenienza glielo suggeriva) che aveva bisogno di visitare ancora una volta l'ammalata. 

- Come, osservarla ancora una volta? - esclamò la madre spaventata. 

- Oh, no, mi occorrono alcuni particolari principessa. 

- Prego, favorisca. 

E la madre, seguita dal dottore, entrò nel salottino di Kitty. 

Smagrita e arrossata, con un particolare luccichio negli occhi pel suo pudore violato, Kitty stava al centro della stanza. Quando il dottore entrò, avvampò tutta e gli occhi le si empirono di lacrime. La malattia e le cure le sembravano una così sciocca e risibile cosa! La cura poi le sembrava ridicola tanto quanto la ricomposizione di un vaso rotto. Il suo cuore era spezzato. Perché la volevano curare con polverine e pillole? Ma non si poteva dispiacere la mamma che del suo malessere si considerava colpevole. 

- Abbiate la compiacenza di sedervi, principessina - disse il medico famoso. 

Sedette di fronte a lei, con un sorriso le prese il polso e di nuovo cominciò a far domande oziose. Ella gli rispondeva, ma a un tratto, indispettita, si alzò. 

- Scusatemi, dottore, ma tutto questo è davvero inconcludente. Per tre volte mi avete chiesto la stessa cosa. 

Il medico famoso non si offese. 

- Irritazione morbosa - disse alla principessa quando Kitty fu uscita. - Del resto, ho finito\ldots{} 

E il dottore, come a una donna eccezionalmente intelligente, definì alla madre in termini scientifici lo stato della principessina, e concluse col prescrivere quelle acque che non erano necessarie. Alla domanda se si dovesse andare o no all'estero, si sprofondò in meditazioni, come se dovesse decidere una questione difficile. La decisione infine venne fuori: andare e non prestar fede ai ciarlatani, e rivolgersi per tutto a lui. 

Andato via il dottore, si ebbe la sensazione che fosse successo qualcosa di piacevole. La madre si mise di buon umore nel rientrare nella stanza della figlia, e Kitty finse di essere allegra. Le accadeva ormai spesso, anzi quasi sempre, di fingere. 

- Sto bene, maman, davvero. Ma se voi volete andare, andiamo - disse e, cercando di prendere interesse al prossimo viaggio, cominciò a parlare dei preparativi per la partenza. 

\capitolo{II}\label{ii-1} 

Dopo il dottore giunse Dolly. Sapeva che in quel giorno si sarebbe tenuto il consulto e, pur avendo di recente lasciato il letto (aveva dato alla luce una bambina alla fine dell'inverno), pur avendo molte pene e affanni da parte sua, lasciata la neonata e una bambina che si era ammalata, era venuta per sapere della sorte di Kitty che in quel giorno si decideva. 

- E allora? - chiese, entrando nel salotto e togliendosi il cappello. - Siete tutti di buon umore. Probabilmente, va bene? 

Provarono a riferirle quello che aveva detto il dottore, ma si accorsero che, sebbene il dottore avesse parlato diffusamente e a lungo, in nessun modo si riusciva a ripetere quello che aveva detto. Risultava chiaro solo il fatto che era stato deciso di andare all'estero. 

Dolly sospirò involontariamente. La sua amica migliore, la sorella, partiva. E la sua vita non era allegra. I rapporti con Stepan Arkad'ic, dopo la riconciliazione, erano divenuti umilianti. La saldatura fatta da Anna era risultata precaria, e l'accordo familiare si era spezzato di nuovo nello stesso preciso punto. Non v'era nulla di concreto, ma Stepan Arkad'ic non era mai in casa; e quasi mai c'era denaro, e i sospetti delle infedeltà tormentavano continuamente Dolly, ed ella li allontanava, temendo le pene già provate della gelosia. Il primo accesso di gelosia, una volta superato, non poteva più ripetersi, e anche la scoperta di un'altra infedeltà non avrebbe prodotto su di lei lo stesso effetto della prima. Una scoperta di questo genere avrebbe soltanto sconvolto le sue abitudini familiari, e perciò si lasciava ingannare, disprezzando lui e più di tutto se stessa per la propria debolezza. Oltre a questo, le cure di una famiglia numerosa la tormentavano incessantemente: ora l'allattamento della neonata non andava bene, ora la balia si licenziava, ora infine, come in quel momento, s'ammalava uno dei bambini. 

- Be', come stanno i tuoi? - chiese la madre. 

- Ah, maman, di pena da noi ce n'è sempre tanta. Lily s'è ammalata e io temo che sia scarlattina. Sono venuta solo ad informarmi, ma poi mi chiuderò in casa senza più uscire se, Dio ne liberi, dovesse essere scarlattina. 

Il vecchio principe, dopo che il dottore se ne era andato, era uscito anche lui dal suo studio e, dopo aver offerta la guancia a Dolly e aver parlato con lei, si era rivolto alla moglie: 

- Cosa è stato deciso, allora, andate? Be', e di me che ne volete fare? 

- Io ritengo che tu debba restare, Aleksandr - disse la moglie. 

- Maman, e perché papà non può venire con noi? - disse Kitty. - E per lui e per noi sarà più piacevole. 

Il vecchio principe si alzò e carezzò con la mano i capelli di Kitty. Ella aveva sollevato il viso e, sorridendo forzatamente, lo guardava. Le sembrava sempre ch'egli la capisse meglio degli altri in famiglia, benché parlasse poco con lei. Come ultima figlia era la preferita del padre, e a lei sembrava che quel grande affetto lo rendesse perspicace. E quando il suo sguardo incontrò quei suoi buoni occhi azzurri che la guardavano fissi, le sembrò ch'egli la vedesse da parte a parte e che capisse tutto il tormento che avveniva in lei. Arrossendo si protese verso di lui, aspettando un bacio, ma egli le batté soltanto sui capelli e disse: 

- Questi stupidi chignons! Non carezzi i capelli di tua figlia, ma quelli di femmine già morte. Be', Dolin'ka, che fa il tuo bel tomo? 

- Nulla, papà - rispose Dolly, comprendendo che l'allusione si riferiva al marito. - È sempre fuori e non lo vedo quasi - aggiunse con un sorriso ironico. 

- E che, non è ancora partito per la campagna a vendere il legname? 

- No, si prepara sempre. 

- Ecco - disse il principe. - Così allora anch'io devo prepararmi? Ai vostri ordini, signora - disse alla moglie, sedendosi. - E tu, ecco cosa devi fare, Katja - aggiunse, rivolgendosi alla figlia minore: - un bel mattino, quando parrà a te, svegliati e di' a te stessa: ecco io sto perfettamente bene e sono di ottimo umore; andiamo di nuovo con papà a spasso sul ghiaccio di buon'ora. Eh? 

Sembrava molto semplice quello che diceva il padre, ma Kitty a queste parole si confuse e si smarrì, come un delinquente colto in fallo. ``Sì, egli sa tutto, capisce tutto e con queste parole mi dice che, per quanto sia vergognoso quello che mi è accaduto, tuttavia bisogna sopravvivere alla propria vergogna''. Non riuscì a riprendersi per rispondere qualcosa. Stava incominciando quando improvvisamente scoppiò a piangere e scappò via dalla stanza. 

- Ecco, tu con i tuoi scherzi - disse la principessa, investendo il marito. - Sei sempre\ldots{} - e cominciò a rimproverarlo. 

Il principe ascoltò a lungo le recriminazioni della principessa e tacque, ma il viso gli si faceva sempre più scuro. 

- Fa tanta pena, la poveretta, tanta pena, e tu non ti accorgi che le fa male ogni accenno a quello che ne è la causa. Ah, sbagliarsi così sul conto della gente! - disse la principessa e, dal cambiamento di tono, Dolly e il principe capirono che alludeva a Vronskij. - Non capisco come non vi siano delle leggi contro esseri così disgustosi e ignobili. 

- Ah, se aveste dato retta a me! - esclamò cupo il principe, alzandosi dalla poltrona e desiderando andarsene; ma, fermandosi poi sulla porta: - Le leggi! ci sono, matuška, e giacché tu mi stai provocando, ti dirò che la colpa di tutto questo è tua, tua, tua soltanto. Leggi contro questi bellimbusti ci sono sempre state e ci sono. Sissignora; se non ci fosse stato da parte vostra quello che non ci sarebbe dovuto essere, io, anche vecchio, l'avrei sfidato a duello, quel cascamorto. Sì: e adesso curate pure la ragazza, fate venire in casa questi ciarlatani. 

Il principe sembrava avesse da dire ancora molte cose, ma non appena la principessa sentì il tono irato di lui, si calmò e si pentì subito come accadeva sempre nelle questioni serie. 

- Alexandre, Alexandre - mormorava, agitandosi e scoppiando in pianto. 

Non appena cominciò a piangere, anche il principe si calmò e le si avvicinò. 

- Su, basta, basta! Anche per te è penoso, lo so. Che fare? Non è un grosso guaio. Dio è misericordioso\ldots{} grazie\ldots{} - disse, non sapendo già più neppur lui cosa dire e, rispondendo al bacio umido di lei sulla sua mano, uscì dalla stanza. 

Già da quando Kitty in lacrime era uscita dalla stanza, Dolly, con la sua esperienza materna, aveva sentito subito che c'era un'opera femminile da compiere, e si era accinta a compierla. Si levò il cappello e, rimboccate moralmente le maniche, si preparò ad agire. Durante l'aggressione materna contro il padre, aveva cercato di trattenere la madre per quanto lo consentiva il suo rispetto filiale. Durante lo scoppio d'ira del padre aveva taciuto, provando vergogna per la madre e tenerezza per il padre, per quella sua bontà immediatamente sopraggiunta; ma appena il padre fu uscito, si apprestò a fare la cosa più urgente: andare da Kitty a calmarla. 

- Ve lo volevo dire da tempo, maman. Sapete che Levin voleva far domanda di matrimonio a Kitty, quando è stato qui l'ultima volta? L'ha detto a Stiva. 

- E allora? Non capisco\ldots{} 

- Allora, forse, Kitty l'avrà respinto. Non ve ne ha parlato? 

- No, non ha detto niente né di questo né dell'altro: è troppo orgogliosa. Ma io lo so che tutto dipende dal fatto che\ldots{} 

- Certamente. Immaginate\ldots{} se ha detto di no a Levin\ldots{} e non l'avrebbe mai respinto se non ci fosse stato l'altro, lo so\ldots{} E invece poi l'altro l'ha ingannata così orribilmente. 

La principessa si sentiva sgomenta a pensare quanto ella fosse colpevole verso la figlia, e montò in collera. 

- Ah, non capisco più nulla! Oggigiorno vogliono fare di testa loro, non dicono nulla alla mamma, e poi, ecco\ldots{} 

- Maman, io vado da lei. 

- Va', te lo proibisco forse? - disse la madre. 

\capitolo{III}\label{iii-1} 

Entrando nello studiolo di Kitty, una graziosa stanza color rosa, giovanile, rosea e gaia come la stessa Kitty fino a due mesi addietro, disseminata di figurine vieux saxe, Dolly ricordò con quanta gioia e con quanto amore avevano arredata insieme, l'anno prima, quella stanzetta. Le si gelò il cuore quando vide Kitty seduta su di una seggiola bassa, la più vicina alla porta, con gli occhi fissi immobili su di un angolo del tappeto. 

Kitty guardò la sorella e l'espressione fredda, un po' dura del viso non mutò. 

- Adesso me ne vado e dovrò chiudermi in casa, neanche tu potrai venire da me - disse Dar'ja Aleksandrovna, sedendosi accanto a lei. - Volevo parlare un po' con te. 

- Di che? - domandò Kitty in fretta, alzando spaventata la testa. 

- Di che, se non della tua pena? 

- Ma io non ho nessuna pena. 

- Basta, Kitty. Davvero pensi che io possa non capire? Io so tutto! E credimi, questo non è nulla. Ci siamo passate tutte. 

Kitty taceva, e il suo viso aveva un'espressione dura. 

- Non merita che tu soffra per lui - continuò Dar'ja Aleksandrovna andando dritta allo scopo. 

- Già, perché mi ha disdegnata - disse Kitty con voce tremante. - Non me ne parlare, ti prego, non me ne parlare! 

- Ma chi ti ha detto questo? Nessuno ha detto questo. Sono sicura che lui era innamorato di te ed è rimasto innamorato ma\ldots{} 

- Ah, la cosa più tremenda per me sono questi compatimenti! - gridò Kitty, irritandosi a un tratto. Si girò sulla seggiola, arrossì e prese a muovere rapidamente le dita, stringendo ora con una mano, ora con l'altra la fibbia della cintura. Dolly conosceva quel tratto della sorella, di afferrar qualcosa con le mani quando si eccitava: sapeva Kitty capace, in un momento d'ira, di trascendere e di pronunciare molte cose inutili e spiacevoli, e voleva calmarla, ma era già troppo tardi. - Cosa, cosa mi vuoi far sentire? - diceva con furia. - Che io ero innamorata di un uomo che non voleva saperne di me, e che muoio di amore per lui? E questo me lo dice una sorella che crede così di\ldots{} compatirmi! Non ne voglio di questi compatimenti e di queste mistificazioni! 

- Kitty, sei ingiusta. 

- E tu perché mi tormenti? 

- Ma al contrario\ldots{} Vedo che soffri\ldots{} 

Ma Kitty nella sua collera non l'ascoltava. 

- Non ho nulla di cui debba affliggermi o consolarmi. Sono tanto orgogliosa da non permettermi mai di amare un uomo che non mi ama. 

- Sì, ma io non dico\ldots{} Solo\ldots{} dimmi la verità - disse prendendole la mano Dar'ja Aleksandrovna. - Dimmi, Levin ti ha parlato? 

L'accenno a Levin fece perdere del tutto a Kitty il dominio di sé; scattò su dalla seggiola e, gettata via la fibbia, e agitando rapida le mani, si mise a dire: 

- E che c'entra, ora, anche Levin? Non capisco che bisogno abbia tu di tormentarmi. Ti ho detto e ti ripeto che sono orgogliosa e che mai, mai farò quello che fai tu: di ritornare a un uomo che ti ha tradito; che si è innamorato di un'altra. Io questo non lo capisco. Tu puoi, e io non posso! 

Dette queste parole, guardò la sorella e, vedendo che Dolly taceva, abbassando tristemente il capo, invece di uscire dalla stanza come stava per fare, Kitty ristette presso la porta e chinò la testa, nascondendo il viso nel fazzoletto. 

Il silenzio durò circa due minuti. Dolly pensava a sé. L'umiliazione che sempre sentiva, risonava in maniera particolarmente dolorosa in lei, ora che gliela rinfacciava la sorella. Non si aspettava tanta crudeltà da lei e ne provò sdegno. Ma improvvisamente sentì il fruscio di un abito e insieme il suono di un singhiozzo trattenuto che prorompeva e due braccia che dal basso le circondavano il collo. Kitty era davanti a lei in ginocchio. 

- Dolin'ka, sono tanto, tanto infelice! - mormorò in tono colpevole. 

E il viso gentile, coperto di lacrime, si nascose nella gonna di Dar'ja Aleksandrovna. Come se le lacrime fossero state l'olio indispensabile senza il quale non poteva muoversi la macchina delle reciproche confidenze fra sorelle, dopo le lacrime esse non parlarono più di quello che loro stava a cuore, ma anche conversando di altro, si intesero scambievolmente. Kitty capì che le parole pronunziate nella furia sull'infedeltà del cognato e sulla posizione umiliante della sorella avevano sì, ferito la poveretta in fondo al cuore, ma ch'ella aveva perdonato. Dolly da parte sua seppe quello che voleva sapere: si convinse cioè che le sue supposizioni erano fondate, che il dolore, l'inguaribile dolore di Kitty, consisteva proprio in questo: che Levin aveva fatto la sua proposta di matrimonio, e Kitty gli aveva detto di no, mentre Vronskij l'aveva ingannata; e ch'ella avrebbe amato Levin e odiato Vronskij. Ma Kitty non disse neppure una parola di questo. Parlava solo delle sue condizioni di spirito. 

- Non ho nessun male - diceva, dopo essersi calmata; - ma non puoi credere come per me tutto sia diventato brutto, ripugnante, volgare e prima di tutto me stessa. Tu non puoi immaginare quali brutti pensieri io abbia su tutto. 

- Ma quali brutti pensieri puoi mai avere tu? - chiese Dolly, sorridendo. 

- I più disgustosi e volgari, non te li posso dire. Non è malinconia, né stanchezza, ma qualcosa di molto peggiore. È come se tutto quello che c'era di buono in me si fosse nascosto e fosse rimasta solo la parte più ignobile. Ma come dirti? - continuò vedendo la perplessità negli occhi della sorella. - Papà comincia a parlare\ldots{} e a me sembra ch'egli pensi soltanto che io debba prender marito. Mamma mi accompagna a un ballo: e a me pare che mi ci conduca soltanto per darmi un marito al più presto e liberarsi di me. Lo so che questo non è vero, ma non posso scacciar via questi pensieri. I cosiddetti pretendenti non li posso più vedere. Mi sembra che mi prendan le misure. Prima per me andare in qualche posto, in abito da ballo, era un vero godimento, mi compiacevo di me stessa; ora mi vergogno, sono impacciata. Ma che vuoi! Il dottore\ldots{} e poi\ldots{} 

Kitty s'ingarbugliò; voleva dire ancora che, da quando era avvenuto in lei questo cambiamento, Stepan Arkad'ic le era divenuto insopportabilmente odioso, e che non poteva guardarlo senza associargli le immagini più volgari e sconvenienti. 

- Già, tutto mi appare nell'aspetto più volgare e più disgustoso - continuò. - Questa è la malattia, forse passerà. 

- Ma cerca di non pensare. 

- Non posso. Soltanto coi bambini sto bene. Soltanto da te. 

- Peccato che non ci potrai più venire. 

- Sì che verrò. Ho avuto già la scarlattina, e convincerò maman. 

Kitty insistette nel suo proposito e andò a stare dalla sorella, e per tutto il tempo della scarlattina, che realmente si manifestò, curò i bambini. Tutte e due le sorelle portarono felicemente a guarigione i sei piccoli, ma la salute di Kitty non migliorò, e durante la quaresima gli Šcerbackij partirono per l'estero. 

\capitolo{IV}\label{iv-1} 

Una sola è la cerchia mondana di Pietroburgo; tutti si conoscono e si scambiano visite. Ma in questa vasta cerchia vi sono delle suddivisioni. Anna Arkad'evna Karenina aveva amici e relazioni in tre circoli diversi. Il primo era quello burocratico, cioè il circolo ufficiale del marito composto di colleghi e di dipendenti, legati e divisi tra di loro dalle varie condizioni sociali nel modo più strano e capriccioso. Anna, ora, stentava assai a ricordarsi di quel senso di considerazione quasi devota che nei primi tempi aveva provato per questi personaggi. Ora li conosceva tutti come ci si conosce in un capoluogo di provincia: conosceva le abitudini, le debolezze e le insofferenze di ognuno, conosceva i rapporti fra di loro e i rapporti di ciascuno col centro, sapeva a chi precisamente ciascuno fosse legato e per mezzo di che cosa si congiungesse e si distaccasse dagli altri; ma a questo circolo di interessi burocratici maschili non era riuscita mai a interessarsi e, malgrado i suggerimenti della contessa Lidija Ivanovna, ne rifuggiva. 

Un altro circolo molto vicino ad Anna era quello attraverso il quale Aleksej Aleksandrovic aveva fatto carriera. Centro ne era la contessa Lidija Ivanovna. Era un circolo di donne vecchie e brutte, virtuose e bigotte, di uomini intelligenti, colti e ambiziosi. Una persona intelligente che ne faceva parte lo aveva definito: ``la coscienza della società di Pietroburgo''. Aleksej Aleksandrovic amava molto questo circolo, e Anna che sapeva trattare tutti, nei primi tempi della sua vita a Pietroburgo, si era fatta degli amici anche qui. 

Il terzo circolo, infine, che Anna frequentava, era proprio il cosiddetto gran mondo, il gran mondo dei balli, dei pranzi, delle toilettes, il mondo che si appoggiava alla corte per non scendere al livello di quel mondo equivoco che i membri di questo circolo credevano di poter disprezzare, pur avendo con esso gusti, più che simili, identici. Anna era legata a questo circolo per mezzo della principessa Betsy Tverskaja, moglie di un suo cugino, che aveva centoventimila rubli di rendita e che, dal suo primo apparire nel gran mondo, aveva preso a volerle bene, a circuirla e attrarla nel suo ambiente deridendo quello della contessa Lidija Ivanovna. 

- Quando sarò vecchia e brutta diventerò anch'io come loro - diceva Betsy. - Ma per voi, per una donna giovane e bella come voi, è prematuro un simile ospizio di vecchi. 

Anna, nei primi tempi, evitava, per quanto poteva, questo circolo della principessa Tverskaja, e perché la vita che vi svolgeva esigeva delle spese superiori alle sue possibilità e perché poi, in fondo all'animo, preferiva l'altro; ma dopo il viaggio a Mosca era avvenuto il contrario. Sfuggiva i suoi amici morali e frequentava il gran mondo. Là incontrava Vronskij, e provava una gioia conturbante in questi incontri. Incontrava Vronskij soprattutto da Betsy che era nata Vronskaja e gli era cugina. Vronskij si trovava ovunque potesse incontrare Anna, e le parlava, appena poteva, del suo amore. Ella non gliene dava pretesto, ma ogni volta che si incontrava con lui, le si accendeva nell'animo quella stessa esaltazione che l'aveva presa quel giorno in treno, quando l'aveva visto per la prima volta. Sentiva che, nel vederlo, la gioia le luceva negli occhi e le labbra le si increspavano nel riso e non riusciva ad attutire le manifestazioni di questa gioia. 

Nei primi tempi, Anna credeva in buona fede d'essere contrariata da lui che si permetteva di perseguitarla; ma poco dopo il ritorno da Mosca, una sera, in un ricevimento in cui pensava d'incontrarlo ed egli non c'era, dalla tristezza che s'impossessò di lei, capì che ingannava se stessa e che questa persecuzione non solo non le era spiacevole, ma costituiva tutto l'interesse della sua vita. 

La cantante famosa cantava per la seconda volta e tutto il gran mondo era a teatro. Vista dalla sua poltrona la cugina in prima fila, Vronskij, senza aspettare l'intervallo, entrò nel palco. 

- Com'è che non siete venuto a pranzo? - ella chiese. - Resto meravigliata di fronte a questa chiaroveggenza da innamorati - aggiunse con un sorriso e in modo ch'egli solo potesse sentire: - lei non c'era. Ma venite dopo l'opera. 

Vronskij la guardò interrogativamente. Ella chinò il capo, ed egli la ringraziò con un sorriso, sedendo vicino a lei. 

- E come ricordo le vostre beffe! - continuò la principessa Betsy che trovava un particolare piacere nel seguire l'accendersi di questa passione. - Dov'è andato a finire tutto quello che dicevate? Siete preso al laccio, mio caro! 

- È quel che desidero, d'esser preso - disse Vronskij col suo tranquillo sorriso cordiale. - Se mi lamento, è perché son troppo poco ``preso'', a dir il vero. Comincio a perdere la speranza. 

- Che speranza potete mai avere? - disse Betsy offesa per l'amica - entendons nous. - Ma nei suoi occhi saltellava un focherello che diceva come ella capisse molto bene, e proprio alla stessa guisa di lui, quale fosse la sua speranza. 

- Nessuna - disse Vronskij, ridendo e mettendo in mostra la sua bella dentatura. - Scusate - disse, prendendo il binocolo dalle mani di lei e osservando, al di là della sua spalla nuda, l'ordine opposto dei palchi. - Temo di diventar ridicolo. 

Egli sapeva molto bene che, agli occhi di Betsy e di tutte le persone di mondo, non rischiava di diventar ridicolo. Sapeva molto bene che agli occhi di queste persone la parte dell'amante infelice di una ragazza e in generale di una donna libera poteva parer ridicola; ma la parte del corteggiatore di una donna maritata, che, qualunque cosa accada, pone la propria vita in giuoco per trascinarla all'adulterio, questa parte aveva qualcosa di bello e di grande e non poteva mai apparire ridicola; e perciò con un sorriso d'orgoglio e di felicità che gli errava sotto i baffi, abbassò il binocolo e guardò la cugina. 

- E perché non siete venuto a pranzo? - disse lei, compiaciuta. 

- Questo proprio ve lo devo raccontare. Sono stato occupato, in che cosa? Ve lo do a indovinare su cento\ldots{} su mille. Non l'indovinerete mai. Ho fatto rappacificare un marito con l'offensore della propria moglie. Sì, davvero! 

- Be', e han fatto pace? 

- Quasi. 

- Me lo dovete raccontare - disse lei, alzandosi. - Venite nell'altro intervallo. 

- Non posso, vado al Teatro Francese. 

- E non ascoltate la Nilsson? - chiese con orrore Betsy che non avrebbe saputo in nessun modo distinguere la Nilsson da una qualsiasi corista. 

- Che fare? Ho un appuntamento là, sempre per questa mia opera di pace. 

- Beati i pacificatori, essi si salveranno - disse Betsy, ricordando qualcosa di simile, sentito dire da qualcuno. - Su, allora sedetevi e raccontate, cos'è? 

E riprese il proprio posto. 

\capitolo{V}\label{v-1} 

- È un po' scabrosa, ma è così carina che ho una voglia matta di raccontarla - disse Vronskij, guardandola con gli occhi ridenti. - Non farò nomi. 

- Tanto meglio, indovinerò. 

- Allora ascoltate: due giovani allegri vanno in carrozza\ldots{} 

- S'intende, ufficiali del vostro reggimento. 

- Non ho detto ufficiali, semplicemente due giovani che hanno fatto colazione\ldots{} 

- Traducete: che hanno bevuto. 

- Forse. Vanno in carrozza a pranzo da un amico, nella più allegra disposizione di spirito. E vedono una bella signora che li sorpassa in vettura, si volta e, così almeno a loro sembra, fa cenno e ride. Quelli, naturalmente, subito dietro a lei. Galoppano a tutta forza. Con sorpresa la bella si ferma all'ingresso di quella stessa casa dove vanno loro. La bella corre al piano di sopra. Essi scorgono solo le labbruzze vermiglie di sotto al velo corto e i deliziosi piccoli piedi. 

- Raccontate con tale sentimento che par proprio che siate voi uno dei due. 

- Be', a che cosa avete accennato or ora?\ldots{} Dunque i giovani entrano in casa del compagno; c'è un pranzo di addio. Qui forse appunto bevono un po' più del necessario, come sempre avviene nei pranzi di addio. E a tavola chiedono chi abita su in quella casa. Nessuno lo sa, ma quando chiedono se al piano di sopra ci sono delle mamzel', il servo del padrone risponde che lì ce n'è tante. Dopo pranzo i giovani vanno nello studio del padrone di casa e scrivono una lettera alla sconosciuta, una dichiarazione, e portano loro stessi la lettera di sopra per spiegare quello che nella lettera non sarebbe apparso del tutto comprensibile. 

- Ma perché mi raccontate tutte queste sciocchezze? E poi? 

- Bussano. Vien fuori una cameriera. Consegnano la lettera e assicurano la cameriera che sono tutti e due così innamorati che stanno lì lì per morire sulla porta. La cameriera, perplessa, conduce delle trattative. Ed ecco, a un tratto compare il padrone di casa con le fedine a salsicciotto, rosso come un gambero, il quale spiega che in casa non c'è nessuno all'infuori di sua moglie, e li caccia via tutti e due. 

- E come fate a sapere che ha le fedine, così come avete detto, a salsicciotto? 

- Ecco, ascoltate. Non sono forse andato oggi a far da paciere? 

- E allora? 

- E qui viene il bello. Viene in chiaro che si tratta di una coppia felice: un consigliere titolare e una consiglieressa titolare. Il consigliere titolare sporge querela e io faccio da paciere; e quale paciere!\ldots{} Vi assicuro, Talleyrand non è nulla a petto mio. 

- Ma in che consiste la vostra abilità? 

- Ecco, ascoltate. Noi ci siamo scusati a questo modo: ``siamo desolati, chiediamo venga perdonato il disgraziato equivoco''. Il consigliere titolare dai salsicciotti comincia a rabbonirsi, ma vuole anche lui esprimere i suoi sentimenti, e, non appena comincia a esprimerli, ecco che prende fuoco, si riscalda e dice villanie, e io devo di nuovo mettere in moto tutto il mio talento diplomatico. ``Sono d'accordo che l'azione non è punto lodevole, ma vi prego prendere in considerazione l'equivoco, l'età giovanile e il fatto che i ragazzi avevano allora allora finito di mangiare. Voi comprenderete! Essi sono pentiti con tutta l'anima, chiedono il vostro perdono''. Il consigliere titolare si rabbonisce di nuovo: ``D'accordo, conte, sono pronto a perdonare, ma capirete che mia moglie, mia moglie, una donna onesta, è stata sottoposta a un inseguimento, alle villanie ed alle impertinenze di due ragazzacci qualsiasi, masc\ldots{}''. E pensate che intanto uno di quei ragazzacci sta lì, e io devo far fare la pace. Metto di nuovo in moto tutta la mia diplomazia, ma appena l'affare si avvia alla conclusione, il mio consigliere titolare si scalda ancora, si fa rosso, solleva i salsicciotti e allora, di nuovo, io mi effondo in sottigliezze diplomatiche. 

- Ah, questa bisogna raccontarvela! - disse Betsy alla signora che entrava nel palco. - Mi ha fatto tanto ridere. 

- Su, bonne chance! - aggiunse, dando a Vronskij un dito libero della mano che teneva il ventaglio e abbassando, con un movimento delle spalle, il corpetto del vestito che si era sollevato per apparire interamente scollata quando si sarebbe accostata, secondo l'uso, al parapetto del palco, alla luce del gas e agli sguardi di tutti. 

Vronskij andò al Teatro Francese, dove realmente doveva vedere il comandante del reggimento, che non perdeva neanche una rappresentazione, per parlargli della sua opera di pace che lo occupava e lo divertiva da due giorni. In questo affare era implicato Petrickij, cui egli voleva bene, e un altro, entrato da poco nel reggimento, un buon ragazzo, un ottimo compagno, il giovane principe Kedrov. Ma era l'onore del reggimento principalmente in giuoco. Tutti e due erano dello squadrone di Vronskij. Al comandante del reggimento si era presentato un impiegato, il consigliere titolare Venden, con una querela contro gli ufficiali che gli avevano offeso la moglie. La sua giovane moglie (come raccontava Venden che era ammogliato da sei mesi appena) stava in chiesa con la mamma, quando, avvertito a un tratto un certo malessere dovuto a un suo particolare stato, e non potendo più stare in piedi, era andata a casa con la prima vettura che le era capitata. A questo punto le si erano messi dietro gli ufficiali, lei s'era spaventata e, sentendosi sempre peggio, era corsa su per le scale a casa. Lo stesso Venden, tornato dal tribunale, aveva sentito la scampanellata e il vocio e, visti gli ufficiali ubriachi con la lettera in mano, era uscito e li aveva scaraventati fuori. 

- Ma, dite quel che volete - diceva il comandante del reggimento a Vronskij dopo averlo fatto accostare a sé - Petrickij diventa impossibile. Non passa una settimana senza una storia. Questo funzionario non farà passar liscia la cosa, la manderà avanti. 

Vronskij vedeva quanto fosse incresciosa la faccenda, come si dovesse evitare un duello e far di tutto per rabbonire il consigliere e mettere a tacere la cosa. Il comandante del reggimento si era rivolto a Vronskij proprio perché egli apparteneva all'aristocrazia ed era persona intelligente e soprattutto gelosa dell'onore del reggimento. Discussero un po' e decisero di fare andare Petrickij e Kedrov con Vronskij da questo consigliere titolare a chiedere scusa. Il comandante del reggimento e Vronskij capivano entrambi che il nome di Vronskij e la sua qualifica di aiutante di campo dovevano contribuire non poco a rabbonire il consigliere titolare. E in realtà queste due prerogative risultarono in parte efficienti, ma la conclusione era rimasta dubbia, come del resto stava raccontando lo stesso Vronskij. 

Giunto al Teatro Francese, Vronskij si era appartato insieme con il comandante del reggimento nel ridotto e gli andava raccontando il suo successo o insuccesso. Dopo aver riflettuto, il comandante del reggimento decise di lasciar cadere la faccenda; ma poi, per divertirsi, cominciò a interrogare Vronskij sui particolari dell'incontro, e a lungo non poté trattenersi dal ridere ascoltando quel che Vronskij diceva del consigliere titolare che, quando stava per calmarsi, si accendeva di nuovo al ricordo dei particolari dell'offesa, e sul fatto che Vronskij, alla prima mezza parola conciliante, aveva battuto in ritirata, spingendo avanti a sé Petrickij. 

- È un brutto affare, ma esilarante. Kedrov non può certo battersi con quel signore. Ma si scaldava proprio così furiosamente? - tornava a chiedere, ridendo, il comandante. - E come vi pare questa sera Claire? Una meraviglia! - disse, alludendo alla nuova attrice francese. - Per quanto la si veda, ogni volta è nuova. Solo i francesi sanno essere così. 

\capitolo{VI}\label{vi-1} 

La principessa Betsy, senza aspettare la fine dell'ultimo atto, uscì dal teatro. Aveva fatto appena in tempo ad entrare nello spogliatoio, cospargere il lungo viso pallido di cipria e spalmarvela, ricomporsi e ordinare il tè nel salotto grande, che già una dietro l'altra cominciarono ad arrivare le carrozze alla sua enorme casa nella Bol'šaja Morskaja. Gli invitati raggiungevano la grande scala e il portiere imponente, che la mattina leggeva i giornali dietro la porta di vetro a edificazione dei passanti, apriva in silenzio la grande porta e faceva passare quelli che arrivavano. 

Entrarono quindi, nello stesso tempo, la padrona di casa, da una porta, con la pettinatura racconciata e il viso rinfrescato, e gli ospiti dall'altra nel salotto grande, dalle pareti scure e i tappeti lanosi, con la tavola illuminata a giorno su cui risplendevano, alla luce delle candele, il bianco della tovaglia, l'argento del samovar e la porcellana trasparente del servizio da tè. 

La padrona di casa sedette al samovar e si tolse i guanti. Spostando le sedie con l'aiuto dei camerieri che non si facevano notare, la compagnia si distribuì in due gruppi, uno accanto al samovar intorno alla padrona di casa, l'altro all'estremo opposto del salotto, intorno alla bella moglie di un ambasciatore, dalle sopracciglia scure marcate, in abito di velluto nero. La conversazione nei due gruppi, come del resto avviene sempre sulle prime in un ricevimento, oscillava interrotta dagli incontri, dai saluti, dal tè, come se cercasse un argomento su cui fissarsi. 

- È straordinaria come attrice: evidentemente si è studiata Kaulbach - diceva un diplomatico nel gruppo dell'ambasciatrice - avete notato con che arte è caduta\ldots{} 

- Ah, vi prego, non parliamo più della Nilsson! Di lei ormai non si può dire nulla di nuovo - disse una signora grassa, rossa, senza sopracciglia e senza chignons, coi capelli bianchi e un vecchio vestito di seta. Era la principessa Mjagkaja, nota per la sua semplicità e ruvidezza di tratto, e soprannominata l'enfant terrible. La Mjagkaja sedeva tra i due gruppi e, tendendo l'orecchio, prendeva parte ora a questo ora a quello. - Oggi tre persone mi hanno detto questa stessa frase su Kaulbach, proprio come se si fossero messe d'accordo. E non so capire perché la frase fosse loro piaciuta tanto. 

La conversazione fu interrotta da questa osservazione, e bisognò trovare un altro tema. 

- Raccontateci qualcosa di divertente, ma non di maligno - disse la moglie dell'ambasciatore, grande maestra di quella conversazione elegante che gli inglesi chiamano small-talk, rivolta al diplomatico che in quel momento non sapeva neanche lui che cosa dire. 

- Sembra che non sia facile, perché solo quello che è maligno fa ridere - cominciò lui con un sorriso. - Ma mi ci proverò. Datemi un tema. Tutto sta nel tema. Quando è dato il tema è più facile ricamarci su. Spesso penso che i famosi parlatori del secolo scorso si troverebbero oggigiorno in difficoltà a conversare con intelligenza. Tutto quello che è intelligente è così noioso\ldots{} 

- Già detto da tempo - lo interruppe, ridendo, la moglie dell'ambasciatore. 

La conversazione, incominciata piacevolmente, proprio perché già troppo cordiale, si arrestò di nuovo. Era il caso di ricorrere al mezzo sicuro che non viene mai meno: la maldicenza. 

- Non trovate che in Tuškevic c'è qualcosa alla Louis XV? - disse il diplomatico indicando con gli occhi un bel giovane biondo che era in piedi accanto alla tavola. 

- Oh, sì! È nello stesso stile del salotto; proprio per questo ci viene così spesso. 

Questo tema di conversazione attecchì, proprio perché alludeva a quello di cui non si sarebbe dovuto parlare in quel salotto, dei rapporti, cioè, di Tuškevic con la padrona di casa. 

Intanto, anche intorno al samovar e alla padrona di casa, la conversazione, dopo aver oscillato allo stesso modo per un po' fra i tre temi inevitabili: l'ultima novità mondana, il teatro e la maldicenza, si era fatta stabile, appena toccato l'ultimo tema, quello della maldicenza. 

- Avete sentito, anche la Maltišceva, non la figlia, ma la madre, si fa un vestito diable rose. 

- È impossibile! No, questa è bella! 

- Mi meraviglio come con la sua intelligenza, non è mica sciocca, non s'accorga di quanto sia ridicola. 

Ognuno aveva qualcosa da dire per criticare e prendere in giro la povera Maltišceva, e la conversazione scoppiettò allegra come un fastello di legna che prenda fuoco. 

Il marito della principessa Betsy, un panciuto bonaccione, appassionato raccoglitore di stampe, saputo che la moglie aveva ospiti, era entrato in salotto prima di andare al circolo. Silenziosamente, sul tappeto soffice, si era accostato alla Mjagkaja. 

- V'è piaciuta la Nilsson? - disse. 

- Ah!\ldots{} ma è forse permesso avvicinarsi così? Come mi avete spaventata! - disse lei. - Con me, vi prego, non parlate dell'opera; voi non capite nulla di musica. Piuttosto discenderò io fino a voi a parlar delle vostre maioliche e delle vostre stampe. Dunque, qual'è l'ultimo tesoro che avete comprato dal rigattiere? 

- Volete che ve lo mostri? Ma voi non capite nulla. 

- Mostratemelo. Ho imparato da quei tali, come si chiamano\ldots{} da quei banchieri\ldots{} hanno delle stampe bellissime. Ce le han fatte vedere. 

- Come, siete stata dagli Schützburg? - domandò la padrona di casa di là dal samovar. 

- Ci siamo stati, ma chère. Ci hanno invitato, me e mio marito, a pranzo, e m'han detto che la salsa a quel pranzo era costata mille rubli - diceva a gran voce la Mjagkaja, sentendo che tutti l'ascoltavano - e per giunta una salsa pessima, una certa broda verdastra. Poi ho dovuto invitarli a casa mia, e io ho fatto preparare una salsa da ottantacinque copeche, e tutti sono rimasti molto soddisfatti. Io non posso far mica sempre salse da mille rubli! 

- È unica! - disse la padrona di casa. 

- Sorprendente! - disse qualcuno. 

L'effetto prodotto dai discorsi della principessa Mjagkaja era sempre lo stesso, e il segreto di questo effetto consisteva nel dire, anche se non del tutto a proposito, come adesso, delle cose semplici che avevano un certo senso. Nella società in cui viveva queste parole producevano l'effetto dello scherzo più spiritoso. La Mjagkaja non riusciva a capire perché ciò accadesse, ma sapeva che così era, e ne approfittava. 

Dal momento che durante il discorso della Mjagkaja tutti avevano ascoltato lei e la conversazione intorno alla moglie dell'ambasciatore era cessata, la padrona di casa volle riunire i due gruppi e si rivolse all'ambasciatrice. 

- Ma proprio non volete del tè? Dovreste passare dalla parte nostra. 

- No, stiamo tanto bene qui - rispose con un sorriso la moglie dell'ambasciatore, e riprese la conversazione di poco prima. 

La conversazione era molto piacevole. Si criticavano i Karenin, marito e moglie. 

- Anna è molto cambiata dopo il viaggio a Mosca. C'è in lei qualcosa di strano - diceva una sua amica. 

- Il cambiamento di maggior rilievo è che ha portato con sé l'ombra di Aleksej Vronskij - disse l'ambasciatrice. 

- E che c'è di strano? C'è una favola di Grimm: l'uomo senza ombra, l'uomo privato dell'ombra. E questo gli è dato in castigo di qualcosa. Non ho mai capito in che cosa consistesse il castigo. Ma per una donna, sì che deve essere triste non aver l'ombra. 

- Sì, ma le donne con l'ombra, di solito, vanno a finir male - disse l'amica di Anna. 

- Che vi si secchi la lingua! - disse di botto la principessa Mjagkaja a queste parole. - La Karenina è un'ottima donna. Il marito non mi piace, ma a lei voglio un gran bene. 

- Perché non vi piace il marito? È un uomo così notevole - disse l'ambasciatrice. - Mio marito dice che uomini di stato come lui ce ne sono pochi in Europa. 

- Anche mio marito dice questo, ma io non ci credo - disse la Mjagkaja. - Se i nostri mariti non avessero detto ciò, noi vedremmo quello che è; e Aleksej Aleksandrovic, secondo me, è semplicemente scemo. Io lo dico sottovoce\ldots{} Ma non è vero che così tutto diventa chiaro? Prima, quando m'imponevano di ritenerlo intelligente, non facevo che cercare, e trovavo che ero io la sciocca che non vedeva la sua intelligenza; non appena mi son detta: ``è scemo'', ma sottovoce, tutto è diventato così chiaro; non è vero, forse? 

- Come siete cattiva, oggi! 

- Per nulla affatto. Non c'è altra soluzione. Uno dei due è scemo. Certo, voi lo sapete, di se stessi non si arriva mai a dirlo. 

- Nessuno è contento del proprio stato e ciascuno è contento della propria intelligenza - disse il diplomatico con un verso francese. 

- Ecco, ecco, proprio così - si voltò a lui la Mjagkaja. - Ma il fatto è che io Anna non ve la do in pasto. È così simpatica, gentile. Che fare se tutti si innamorano di lei e le corrono dietro come ombre? 

- Ma io non penso affatto di criticarla - si andava giustificando l'amica di Anna. 

- Se a noi non c'è nessuno che ci vien dietro come l'ombra, questa non è una ragione per aver il diritto di condannare. 

E dopo aver conciato per le feste, così come si conveniva, l'amica di Anna, la principessa Mjagkaja s'alzò e, insieme con la moglie dell'ambasciatore, si unì a quelli della tavola dove era avviata una conversazione di ordine generale sul re di Prussia. 

- Di chi stavate parlando male? - chiese Betsy. 

- Dei Karenin. La principessa ci ha dipinto le caratteristiche di Aleksej Aleksandrovic - rispose con un sorriso l'ambasciatrice, sedendosi a tavola. 

- Peccato che non abbiamo sentito - disse la padrona di casa, guardando la porta d'ingresso. - Oh, eccovi, ci siete anche voi, finalmente! - disse rivolta con un sorriso a Vronskij che entrava. 

Vronskij non solo conosceva tutti, ma s'incontrava ogni giorno con tutti quelli ch'erano lì; entrò quindi con quel suo fare calmo, così come si entra nella stanza di persone che si sono allora allora lasciate. 

- Di dove vengo? - rispose ad una domanda dell'ambasciatrice. - Non c'è scampo, bisogna confessarlo: dai Bouffes. Per la centesima volta e sempre con piacere nuovo, a quanto pare. Un incanto! Lo so che è vergognoso, ma all'opera dormo, mentre ai Bouffes rimango a sedere fino all'ultimo momento e mi diverto. Oggi\ldots{} 

Nominò un'attrice francese e voleva raccontare qualcosa su di lei, ma l'ambasciatrice l'interruppe con scherzoso raccapriccio. 

- Vi prego, non parlate di quell'orrore. 

- E sia, ve ne dispenserò, tanto più che tutti conoscono questi orrori. 

- E tutti ci andrebbero, se questo fosse di moda come andare all'opera - aggiunse la Mjagkaja. 

\capitolo{VII}\label{vii-1} 

Si udirono dei passi alla porta e la principessa Betsy, sapendo che era la Karenina, guardò Vronskij. Egli guardava l'uscio e il suo viso aveva un'espressione strana, nuova. Guardava fisso, con gioia e insieme con timidezza, colei che entrava, e nello stesso tempo si alzava lentamente. Anna entrava nel salotto. Straordinariamente diritta come sempre, con quel suo passo agile, sicuro e leggero che la distingueva dall'andatura delle altre donne del suo mondo, fece i pochi passi che la separavano dalla padrona di casa e, senza cambiare direzione allo sguardo, le porse la mano, sorrise e con quello stesso sorriso si voltò a guardare Vronskij. Vronskij s'inchinò profondamente e le accostò una sedia. 

Ella rispose con un semplice chinar del capo, arrossì e aggrottò le sopracciglia. Ma poi, facendo subito un cenno della testa agli amici e stringendo le mani tese, si rivolse alla padrona di casa: 

- Sono stata dalla contessa Lidija ed avrei voluto venir via prima. Ma c'era da lei sir John. È molto interessante. 

- Ah, quel missionario? 

- Sì, ha raccontato delle cose molto interessanti sulla vita degli indiani. 

La conversazione, interrotta dall'arrivo, si animò come la fiamma di una lampada avvivata. 

- Sir John, già, sir John. L'ho visto. Parla bene. La Vlas'eva è innamorata pazza di lui. 

- È vero che la Vlas'eva più piccola sposa Topov? 

- Già, dicono che sia tutto deciso. 

- Mi meraviglio dei genitori. Dicono che sia un matrimonio d'amore. 

- D'amore? Che idee antidiluviane che avete! Chi mai al giorno d'oggi parla ancora d'amore? - disse l'ambasciatrice. 

- Che fare? Questa stupida vecchia moda non è ancora passata - disse Vronskij. 

- Tanto peggio per quelli che vi si attengono. Io di matrimoni felici non conosco che quelli d'interesse. 

- Già, ma in cambio, quante volte la felicità di questi matrimoni d'interesse si polverizza proprio perché insorge quella tale passione che non si è voluta ammettere! - disse Vronskij. 

- Ma noi per matrimoni d'interesse intendiamo quelli in cui tutt'e due le parti si siano già ammansite. L'amore è come la scarlattina, bisogna passarci. 

- Allora bisogna imparare a inocularlo artificialmente, l'amore, come il vaiolo. 

- Io in gioventù mi sono innamorata di un sacrestano - disse la principessa Mjagkaja; - non so se questo mi abbia aiutato. 

- No, io penso, a parte gli scherzi, che per conoscere l'amore sia necessario sbagliare e poi correggersi - disse la principessa Betsy. 

- Anche dopo il matrimonio? - disse scherzosa l'ambasciatrice. 

- Non è mai troppo tardi per pentirsi - disse il diplomatico con un proverbio inglese. 

- Davvero - replicò a volo Betsy: - bisogna sbagliarsi e correggersi. Cosa ne pensate? - chiese rivolta ad Anna che ascoltava in silenzio questo discorso con un sorriso fisso, appena percettibile sulle labbra. 

- Io penso - disse Anna, giocando con un guanto che si era tolto - io penso\ldots{} se è vero che ci sono tante sentenze quante teste, così pure tante specie d'amore quanti cuori. 

Vronskij guardava Anna e, col cuore che gli veniva meno, aspettava quello che avrebbe detto. Respirò come dopo un pericolo, quando ella ebbe pronunciato queste parole. 

Anna a un tratto si voltò verso di lui. 

- Ho ricevuto una lettera da Mosca. Mi dicono che Kitty Šcerbackaja stia molto male. 

- Davvero? - disse Vronskij, aggrottando le sopracciglia. Anna lo guardò severa. 

- Non vi interessa questo? 

- Al contrario, molto. Cosa vi scrivono precisamente, se è lecito sapere? - chiese. 

Anna si alzò e si accostò a Betsy. 

- Datemi una tazza di tè - disse, fermandosi dietro la sedia di lei. 

Mentre Betsy le versava il tè, Vronskij si avvicinò ad Anna. 

- Cosa vi scrivono dunque? - ripeté. 

- Io penso molto spesso che gli uomini non capiscono quello che è ignobile, anche parlandone continuamente - disse Anna senza rispondergli. - Ve lo volevo dire da tempo - aggiunse, e, fatti alcuni passi, sedette a una tavola in angolo, sulla quale erano degli album. 

- Non capisco per nulla il senso delle vostre parole - disse lui, dandole la tazza. 

Ella accennò il divano vicino a sé ed egli sedette subito. 

- Sì, ve lo volevo dire - disse lei senza guardarlo. - Avete agito male, male, molto male. 

- Forse non lo so di aver agito male? Ma chi mi ha fatto agire male? 

- Perché mi dite questo? - disse lei, guardandolo severa. 

- Voi lo sapete perché - rispose lui franco e felice, incontrando lo sguardo di lei e senza staccarne gli occhi. 

Non lui, ma lei si turbò. 

- Questo dimostra soltanto che siete senza cuore - disse lei. Ma il suo sguardo diceva che sapeva bene come egli avesse un cuore e che per questo lo temeva. 

- Quello di cui parlavate poc'anzi è stato un abbaglio, non un amore. 

- Ricordatevi che vi ho proibito di pronunciare questa parola, questa parola disgustosa - disse Anna, rabbrividendo; ma in quello stesso attimo sentì che con la sola parola ``proibito'' dava prova di attribuirsi dei diritti su di lui, e che con questo lo eccitava a parlare d'amore. - Da tempo volevo dirvi questo - continuò guardandolo decisa negli occhi e tutta accesa dal rossore che le scottava il viso; - ma oggi sono venuta apposta, sapendo di incontrarvi. Sono venuta per dirvi che questo deve finire. Io non ho mai arrossito davanti a nessuno, e voi mi costringete a sentirmi colpevole di qualche cosa. 

Egli la guardava ed era colpito dalla nuova bellezza, tutta spirituale, del volto di lei. 

- Che volete da me? - disse semplice e serio. 

- Voglio che andiate a Mosca e chiediate perdono a Kitty - disse lei. 

- Voi questo non lo volete - disse lui. 

Egli sentiva che Anna diceva quello che s'era imposta di dire, non quello che avrebbe voluto dire. 

- Se mi amate come dite - ella mormorò - fate che io abbia pace. 

Il viso di lui s'illuminò. 

- Non sapete forse che siete per me tutta la vita? Questa pace io non conosco e non posso darvi. Tutto me stesso, l'amore\ldots{} sì. Non riesco a pensare a voi e a me separatamente. Per me, voi ed io siamo una cosa sola. E io non vedo davanti a me possibilità di pace, né per me, né per voi. Vedo una possibilità di disperazione, di infelicità\ldots{} o la possibilità di una gioia, quale gioia!\ldots{} È forse impossibile? - aggiunse a fior di labbra, ma lei sentì. 

Ella tese tutte le forze del suo spirito per dire quello che si sarebbe dovuto dire; ma, in luogo di questo, fermò il suo sguardo pieno d'amore su di lui, e tacque. 

``Ecco - pensò lui con esaltazione. - Mentre già mi disperavo e credevo dovesse venir la fine, ecco: mi ama. Lo confessa''. 

- Allora fate questo per me, non mi parlate mai più di queste cose e rimaniamo buoni amici - disse lei con le labbra, ma il suo sguardo diceva tutt'altra cosa. 

- Amici non saremo mai, questo lo sapete. Saremo gli esseri più felici o gli esseri più infelici della terra, questo dipende da voi. 

Ella voleva dire qualcosa, ma lui l'interruppe. 

- Perché io chiedo una cosa sola, chiedo il diritto di sperare, di tormentarmi come adesso; ma se anche questo non si può, ditemi allora di scomparire, e io scomparirò. Se la mia presenza vi è di peso, non mi vedrete più. 

- Io non voglio scacciarvi. 

- E allora non cambiate nulla. Lasciate tutto così com'è - disse lui con voce tremante. - Ecco vostro marito. - Infatti, proprio in quel momento, Aleksej Aleksandrovic con la sua andatura molle e sgraziata entrava nel salotto. 

Visti la moglie e Vronskij, si avvicinò alla padrona di casa e, sedutosi a bere una tazza di tè, prese a parlare con quella sua voce lenta e penetrante, con quel suo tono abitualmente scherzoso, come se prendesse in giro qualcuno. 

- Il vostro Rambouillet è al completo - disse, esaminando tutta la compagnia; - le Grazie e le Muse. 

Ma la principessa Betsy non tollerava questo suo tono, sneering come lo chiamava lei, e, da padrona di casa intelligente, lo avviò subito a una conversazione seria sul servizio militare obbligatorio. Aleksej Aleksandrovic fu subito preso dall'argomento e cominciò a difendere la nuova disposizione contro la principessa Betsy che la avversava. 

Vronskij e Anna continuavano a star seduti alla tavola piccola. 

- La cosa diventa scandalosa - mormorò una signora, indicando con gli occhi la Karenina, Vronskij e il marito di lei. 

- Cosa vi ho detto io? - rispondeva l'amica di Anna. 

Non solo queste signore, ma quasi tutti quelli che erano nel salotto, perfino la principessa Mjagkaja e la stessa Betsy, guardarono parecchie volte i due che si erano staccati dalla cerchia generale come se ne fossero infastiditi. Solo Aleksej Aleksandrovic non guardò neppure una volta da quella parte e non si distrasse dall'interesse della conversazione iniziata. 

Notando la cattiva impressione prodotta su tutti, la principessa Betsy mise al proprio posto un'altra persona ad ascoltare Karenin, e si accostò ad Anna. 

- Sono sempre sorpresa dalla chiarezza ed esattezza di esposizione di vostro marito - disse. - I concetti più trascendentali mi diventano accessibili quando parla lui. 

- Oh, sì - disse Anna, illuminandosi di un sorriso di felicità e senza capire una parola di quello che le andava dicendo Betsy. Si avvicinò alla tavola grande e prese parte alla conversazione generale. 

Aleksej Aleksandrovic, dopo essere rimasto mezz'ora, si avvicinò alla moglie e le propose di andare a casa; ma lei, senza guardarlo, rispose che rimaneva a cena. Aleksej Aleksandrovic salutò ed uscì. 

Il cocchiere della Karenina, un vecchio tartaro panciuto, con una giacca lustra di pelle, tratteneva a stento il cavallo grigio di sinistra che, intirizzito, s'impennava all'ingresso. Un servitore, diritto impalato, apriva lo sportello, mentre il portiere, in piedi, teneva la porta esterna. Anna Arkad'evna con la mano piccola e agile andava staccando i pizzi della manica da un gancio della pelliccia e, chinando la testa, ascoltava incantata quello che Vronskij le andava dicendo nell'accompagnarla. 

- Voi non avete detto nulla; va bene, neanche io pretendo nulla - diceva - ma voi sapete che non è l'amicizia di cui ho bisogno; per me è possibile una sola felicità nella vita, quella parola che tanto vi spiace\ldots{} sì, l'amore\ldots{} 

- L'amore\ldots{} - ripeté lentamente lei con una voce che proveniva dall'intimo del suo essere, e a un tratto, proprio nel momento in cui si staccava il pizzo, aggiunse: - Non mi piace questa parola anche perché significa qualcosa di troppo grande per me, molto più grande di quello che voi possiate immaginare - e lo guardò in viso. - A rivederci. 

Gli tese la mano, e col passo svelto ed elastico passò accanto al portiere e scomparve nella carrozza. 

Lo sguardo di lei, il contatto della sua mano, lo bruciarono. Baciò la palma nel punto in cui era stata toccata da lei, andò a casa felice, convinto d'essersi accostato al suo scopo, in quella sera, molto più che negli ultimi due mesi. 

\capitolo{VIII}\label{viii-1} 

Aleksej Aleksandrovic non aveva trovato nulla di singolare e di sconveniente nel fatto che sua moglie fosse rimasta insieme con Vronskij a una tavola separata, parlando animatamente di qualche cosa; ma aveva notato che a tutti nel salotto questo era parso singolare e sconveniente, perciò era parso sconveniente pure a lui. Decise di parlarne alla moglie. 

Tornato a casa, Aleksej Aleksandrovic, come al solito, andò nel suo studio e sedette in una poltrona, aprendo, nel punto segnato dal tagliacarte, un libro sul cattolicesimo e, come al solito, rimase a leggere fino all'una; soltanto, di quando in quando, si passava una mano sulla fronte alta e scoteva il capo come ad allontanare qualcosa. All'ora solita si alzò, e fece la sua toletta notturna. Anna Arkad'evna non c'era ancora. Con il libro sotto il braccio andò su; ma quella sera, invece dei soliti pensieri e delle solite considerazioni sugli affari di ufficio, la sua testa era piena della moglie e di qualcosa di spiacevole che la riguardava. Contrariamente alle proprie abitudini non si mise a letto ma, incrociate le mani dietro la schiena, cominciò ad andare su e giù per le stanze. Non poteva coricarsi, sentiva di dover prima riflettere su di una circostanza sorta di recente. 

Gli era sembrato facile e semplice decidere di parlare a sua moglie; ma ora che aveva preso a riflettere sulla circostanza sorta di recente, la cosa gli appariva complessa e difficile. 

Aleksej Aleksandrovic non era geloso. La gelosia, secondo lui, offendeva la moglie e nella moglie si doveva aver fiducia. Perché egli dovesse aver fiducia, perché, cioè, dovesse avere la sicurezza piena che la sua giovane moglie lo avrebbe sempre amato, non se lo chiedeva; ma non provava sfiducia perché aveva fiducia, e diceva a se stesso che si dovesse averne. Ora invece, benché la sua convinzione, che la gelosia è un sentimento riprovevole e che si doveva aver fiducia, non fosse stata distrutta, sentiva di trovarsi di fronte a qualcosa di illogico e di assurdo, e non sapeva cosa fare. Aleksej Aleksandrovic veniva a trovarsi di fronte alla vita, di fronte alla possibilità che sua moglie si innamorasse di qualcun altro che non fosse lui, e ciò gli sembrava assurdo e incomprensibile, proprio perché questo era la vita stessa. Aleksej Aleksandrovic aveva vissuto e lavorato tutta la vita negli ambienti burocratici che hanno a che fare con i riflessi della vita. E ogniqualvolta si era imbattuto nella vita vissuta, se ne era scostato. In questo momento provava una sensazione simile a quella di un uomo che, traversato tranquillamente un precipizio su di un ponte, si accorgesse improvvisamente che il ponte è crollato e che sotto c'era un abisso. L'abisso era la vita così come è; il ponte quella vita artificiale che aveva vissuta. Per la prima volta gli si affacciava alla mente l'ipotesi che sua moglie potesse amare un altro, ed egli inorridiva di fronte a questo. 

Senza essersi spogliato, andava avanti e indietro, con passo eguale, sul pavimento di legno scricchiolante della sala da pranzo illuminata da un'unica lampada, sul tappeto del salotto oscuro in cui la luce si rifletteva solo sul suo grande ritratto fatto da poco, appeso sopra il divano, e attraverso lo studiolo di lei, dove ardevano due candele che davan luce ai ritratti dei parenti e delle amiche e agli oggetti belli della scrivania a lui così noti da tempo. Attraversando lo studiolo giungeva alla porta della stanza da letto e voltava di nuovo indietro. 

A ogni giro del suo percorso, e soprattutto quando giungeva sul pavimento di legno della stanza da pranzo illuminata, si fermava e diceva a se stesso: ``Sì; è assolutamente necessario risolvere e far cessare tutto, esprimere la propria idea e la propria decisione''. E si voltava indietro. ``Ma esprimere che cosa? quale decisione?'' diceva a se stesso nel salotto, e non trovava risposta. ``Ma, dopo tutto - si chiedeva prima di voltare nello studiolo - che cosa è mai successo? Nulla. Ha parlato a lungo con lui, ebbene?\ldots{} Con chi non può parlare una donna in società? E, poi, essere geloso vuol dire umiliare se stesso e lei'' si diceva, entrando nello studiolo; ma questa convinzione che prima aveva tanto peso per lui, ora non ne aveva alcuno e non significava nulla. E dalla porta della camera tornava di nuovo verso la sala da pranzo; ma non appena rientrava nel salotto oscuro, una voce gli diceva che non era così e che se gli altri l'avevano notato, voleva dire che qualcosa c'era. E di nuovo, in sala da pranzo, si diceva: ``Sì, è assolutamente necessario risolvere e far cessare tutto ed esporre il proprio punto di vista\ldots{}''. E di nuovo, nel salotto, prima di voltare, si domandava: ``Ma in che modo decidere?''. E dopo, ancora: ``Che cosa è successo?''. E rispondeva: ``Nulla'' e tornava a ripetere a se stesso che la gelosia è un sentimento che avvilisce la moglie, mentre di nuovo, nel salotto, tornava a convincersi che qualcosa c'era stato. I suoi pensieri, così come la sua persona, compivano un intero giro, senza imbattersi in nulla di nuovo. Egli notò questo, si passò una mano sulla fronte e sedette nello studiolo di lei. 

Qui, guardando sullo scrittoio dove c'erano un tampone di malachite e un biglietto cominciato, i suoi pensieri cambiarono improvvisamente corso. Cominciò a pensare a lei, a quello che ella avrebbe potuto pensare e sentire. Per la prima volta si rappresentò con chiarezza la vita intima di lei, i suoi pensieri, i suoi desideri; e l'idea che ella potesse avere una vita tutta propria gli sembrò così spaventosa che s'affrettò a scacciarla. Era questo l'abisso nel quale era così pauroso guardare. Trasferirsi col pensiero e col sentimento in un altro essere era un'azione spirituale estranea ad Aleksej Aleksandrovic. Egli la considerava come dannosa e pericolosa fantasticheria. 

``E la cosa più terribile - pensava - è che ora, proprio quando la mia questione si approssima alla conclusione - alludeva al progetto che stava facendo passare - quando ho bisogno di tutta la serenità e di tutte le forze dello spirito, proprio ora mi si scaraventa addosso questa insensata inquietudine. Ma, che fare? Io non sono di quegli uomini che soffrono agitazioni e inquietudini senza aver la forza di affrontarle''. 

- Bisogna riflettere, decidere e sistemare tutto - disse ad alta voce. 

``La questione dei suoi sentimenti, di quello che avviene e può avvenire nell'anima sua non è affar mio; riguarda la sua coscienza e riguarda la religione'' si diceva, provando sollievo nel trovare il lato normativo al quale soggiaceva la circostanza che era sorta. 

``È così - si disse Aleksej Aleksandrovic - la questione dei suoi sentimenti e il resto sono questioni della sua coscienza con la quale io non ho nulla da spartire. Il mio dovere, d'altra parte, è chiaramente determinato. Come capo della famiglia, e come persona tenuta a guidarla e perciò in parte responsabile, devo prospettarle il pericolo che vedo, metterla in guardia e adoperare perfino la mia autorità. Devo parlarle''. 

E nella mente di Aleksej Aleksandrovic si andò formulando chiaramente tutto quello ch'egli avrebbe detto alla moglie. Riflettendo a quello che avrebbe detto, rimpiangeva di dover adoperare, a scopi domestici e in maniera così insignificante, il proprio tempo e le proprie facoltà intellettuali; nonostante ciò, nella testa gli si vennero a comporre, chiari e distinti, così come in una relazione ministeriale, la forma e lo svolgimento del discorso da fare. ``Devo esprimermi in questo ordine: in primo luogo, dimostrare l'importanza dell'opinione pubblica e delle convenienze sociali; in secondo luogo, precisare i valori religiosi del matrimonio; in terzo luogo, se necessario, indicare il danno che potrebbe derivare al figlio; in quarto luogo, prospettarle la sua stessa infelicità''. E incrociate le dita le une nelle altre, con le palme all'ingiù, Aleksej Aleksandrovic le stiracchiò e le dita scricchiolarono nelle giunture. 

Questo gesto, questa cattiva abitudine di riunire le mani e far scricchiolare le dita, lo tranquillizzava sempre, e gli dava quel senso di precisione che in questo momento gli era tanto necessario. Si sentì il rumore di una carrozza che giungeva all'ingresso. Aleksej Aleksandrovic si fermò in mezzo alla sala. 

Sulla scala risonarono dei passi femminili. Aleksej Aleksandrovic, pronto per il suo discorso, stava in piedi, stringendo le dita incrociate e provando se in qualche giuntura volessero ancora scricchiolare. Una giuntura scricchiolò. 

Dal suono dei passi leggeri su per la scala, egli sentì l'approssimarsi di lei; e, pur essendo soddisfatto del proprio discorso, ebbe paura della spiegazione imminente\ldots{} 

\capitolo{IX}\label{ix-1} 

Anna camminava a testa china, giocherellando con le nappine del cappuccio. Il suo viso emanava un bagliore vivo; ma questo bagliore non era gaio, ricordava il bagliore sinistro di un incendio in una notte oscura. Visto il marito, Anna alzò il capo e, come svegliandosi, sorrise. 

- Non sei a letto? Oh, ma questo è un miracolo! - disse, togliendosi il cappuccio e, senza fermarsi, proseguì verso lo spogliatoio. - È ora, Aleksej Aleksandrovic - disse di là dalla porta. 

- Anna, ho bisogno di parlare con te. 

- Con me? - disse lei sorpresa, uscendo dalla porta e guardandolo. - Cos'è mai? Di che si tratta? - chiese, sedendosi. - Parliamo pure, se è proprio tanto necessario. Sarebbe meglio dormire, però. 

Anna diceva quel che le veniva sulle labbra e, nell'ascoltarsi, stupiva della propria capacità di mentire. Come erano semplici e naturali le sue parole e come era verosimile il fatto ch'ella avesse proprio sonno! Si sentiva rivestita d'un'impenetrabile maglia d'inganno. Sentiva che una forza invisibile l'aiutava e la sosteneva. 

- Anna, devo metterti in guardia - egli disse. 

- Mettermi in guardia? - rispose lei. Ella appariva così schietta e allegra che chiunque non l'avesse conosciuta non avrebbe notato nulla di straordinario nel suono e nel senso delle sue parole. Ma per lui che la conosceva, che sapeva come ella notasse perfino se egli andava a letto cinque minuti più tardi e ne chiedeva la ragione; per lui che sapeva come ella gli confidasse ogni sua gioia, ogni allegrezza e ogni suo dispiacere, per lui vedere come in questo momento ella non volesse accorgersi dello stato suo e nulla volesse dire di sé, significava molto. Sentiva che il fondo dell'animo suo, che un tempo gli si offriva, gli veniva ora precluso. Non solo, ma dal suo tono sentiva che tutto questo non turbava minimamente lei, ed era come se gli dicesse sul viso: ``sì, è precluso, e così sarà d'ora in poi''. Provava una sensazione simile a quella di un uomo che nel tornare a casa trovi la propria casa chiusa. 

``Ma forse se ne troverà ancora la chiave'' pensava Aleksej Aleksandrovic. 

- Ti voglio mettere in guardia - disse a voce bassa - perché tu non dia, per incoscienza o leggerezza, motivo di far parlare di te in società. Il tuo colloquio di oggi troppo vivace con il conte Vronskij - pronunciò fermamente e dopo una tranquilla pausa questo nome - ha attirato su di te l'attenzione. 

Egli parlava e guardava gli occhi ridenti di lei, ormai paurosi per la loro impenetrabilità, e parlando sentiva tutta la vanità e l'oziosità delle proprie parole. 

- Tu sei sempre così - rispondeva lei, come se non riuscisse a capirlo in nessun modo e come se di tutto quello ch'egli aveva detto avesse afferrato solo l'ultima cosa. - Un momento ti spiace che io mi annoi, un momento che io sia allegra. Non mi sono annoiata. Questo forse ti offende? 

Aleksej Aleksandrovic ebbe un brivido, piegò le mani per farle scricchiolare. 

- Ah, ti prego, non le fare scricchiolare, non mi piace - disse lei. 

- Ma, Anna, sei proprio tu? - disse Aleksej Aleksandrovic piano, facendo uno sforzo su di sé per trattenersi dal gesto abituale delle mani. 

- Ma cos'è mai? - disse lei con uno stupore comicamente sincero. - Che vuoi da me? 

Aleksej Aleksandrovic tacque, si fregò la fronte e gli occhi con una mano. Si accorgeva che invece di quello che voleva fare, mettere cioè in guardia la moglie da quello che poteva apparire un errore agli occhi del mondo, si agitava involontariamente per quello che riguardava la coscienza di lei, e lottava contro un muro creato dalla sua stessa immaginazione. 

- Ecco quello che intendo dirti - continuò freddo e tranquillo - e ti chiedo di ascoltarmi. Come sai, io ritengo che la gelosia offenda e umilii, e non mi permetterò mai di lasciarmi andare a questo sentimento; ma ci sono certe leggi di convenienza che non possono essere impunemente trasgredite. Non sono stato io a notarlo quest'oggi, ma è l'impressione generale prodotta sulla compagnia; tutti hanno notato che il tuo contegno e il tuo comportamento non erano quali precisamente si potevano desiderare. 

- Non capisco proprio nulla - disse Anna, stringendosi nelle spalle. ``A lui personalmente non importa alcun che, ma la compagnia lo ha notato, e lui se ne inquieta''. - Tu stai poco bene, Aleksej Aleksandrovic - aggiunse, alzandosi per uscire dalla porta; ma egli le si parò innanzi, quasi a fermarla. 

Il suo viso era torvo e tetro come Anna non l'aveva mai veduto. Ella si fermò e, buttando il capo all'indietro, da un lato, prese a toglier via le forcine con la mano agile. 

- Ebbene, io ascolto quel che devi dirmi - disse con calma e irrisione. - E ascolto anzi con interesse, perché vorrei capire di che cosa si tratta. 

Parlava, e si stupiva del tono calmo e sincero che le veniva naturale e della scelta delle parole che adoperava. 

- Io non ho alcun diritto di entrare in fondo ai tuoi sentimenti, anzi in genere ritengo ciò inutile e perfino dannoso - cominciò Aleksej Aleksandrovic . - Tante volte, scavando nell'anima nostra, ne facciamo venir fuori qualcosa che sarebbe rimasto inosservato. I tuoi sentimenti riguardano la tua coscienza; ma io ho l'obbligo verso di te, verso di me e verso Dio di indicarti i tuoi doveri. La nostra vita è stata legata non dagli uomini, ma da Dio. Solo un delitto può infrangere questo legame, e un delitto di tal genere porta con sé una pena. 

- Non capisco nulla. Ah, Dio mio! e, per mia disgrazia, ho tanta voglia di dormire! - disse lei in fretta, toccando con la mano i capelli per cercarvi le forcine rimaste. 

- Anna, in nome di Dio, non parlare così - disse lui sommesso. - Può darsi che io mi sbagli, ma credimi, quello che dico lo dico tanto per me come per te. Io sono tuo marito e ti amo. 

Per un attimo la testa di lei si chinò e la luce ironica degli occhi si spense; ma la parola ``amo'' la irritò di nuovo. Pensò: ``Ama? Può forse amare lui? Se non avesse sentito dire che esiste l'amore, non avrebbe neanche mai usato questa parola. Ma lui non sa neppure cosa sia l'amore!''. 

- Aleksej Aleksandrovic, davvero, non capisco - disse. - Precisa quello che pensi\ldots{} 

- Lasciami parlare, ti prego. Io ti amo. Ma io non parlo di me; qui le persone principali siete voi, tu e nostro figlio. Può darsi benissimo, ripeto, che le parole ti sembrino del tutto inutili e fuori posto; forse sono provocate da un mio smarrimento. In questo caso ti prego di perdonarmi. Ma se tu stessa senti che c'è anche il più piccolo fondamento, allora, ti prego, pensaci, e, se il cuore te lo dice, confidati\ldots{} 

Aleksej Aleksandrovic, senza rendersene conto, diceva cose affatto diverse da quelle che aveva preparate. 

- Non ho nulla da dire. E poi\ldots{} - ella disse in fretta, trattenendo a stento un sorriso - davvero è ora di dormire. 

Aleksej Aleksandrovic sospirò e, senza dir più nulla, si diresse in camera. 

Quando ella entrò, egli era già a letto. Le sue labbra erano severamente strette e gli occhi non la guardavano. Anna si coricò nel suo letto, aspettando ch'egli da un momento all'altro riprendesse a parlare. Ne aveva insieme paura e desiderio. Ma egli taceva. Ella attese a lungo, immobile, ma già lo aveva dimenticato: pensava all'altro, vedeva l'altro e sentiva che il cuore a questo pensiero le si riempiva di ansia e di gioia colpevole. A un tratto sentì un ronfio nasale, eguale e calmo. Dapprima Aleksej Aleksandrovic si spaventò quasi del proprio russare e si fermò, ma, dopo due respiri, il ronfio si fece sentire calmo e cadenzato. 

- È tardi, è tardi ormai - mormorò lei con un sorriso. Rimase a lungo immobile con gli occhi aperti e le sembrava di vedere lei stessa, nel buio, il loro bagliore. 

\capitolo{X}\label{x-1} 

Da quella sera cominciò una nuova vita per Aleksej Aleksandrovic e sua moglie. Non accadde nulla di straordinario. Anna continuò a frequentare il gran mondo, andava spesso, più che altrove, dalla principessa Betsy, e s'incontrava con Vronskij dovunque. Aleksej Aleksandrovic rilevava tutto questo, ma non poteva farci nulla. A tutti i tentativi per portarla ad una spiegazione, ella opponeva il muro impenetrabile del suo allegro stupore. Esteriormente tutto era come prima, ma i loro rapporti intimi si erano completamente mutati. Aleksej Aleksandrovic, l'uomo così energico negli affari di stato, si sentiva impotente. Come un bue, aspettava, con il capo abbassato, la mazza che sentiva sospesa su di sé. Ogni qualvolta ci pensava, sentiva che era necessario tentare qualcosa, sentiva che, con la bontà, la tenerezza, la persuasione, c'era ancora la speranza di salvarla, di farla rientrare in sé, e ogni giorno si disponeva a parlare. Ma appena cominciava a parlare con lei, sentiva che lo spirito del male e dell'inganno che la possedeva s'impossessava anche di lui, ed egli parlava di cose del tutto diverse e con un tono contrario a quello che avrebbe voluto usare. Suo malgrado, parlava con lei con quell'abituale tono di canzonatura, come se proprio così volesse parlare. E con questo tono non si poteva dire ciò che era necessario dire. 

\capitolo{XI}\label{xi-1} 

Quello che per Vronskij era stato, per quasi un anno, l'unico, esclusivo desiderio che si era sostituito a tutti i desideri della sua vita, quello che per Anna era un impossibile, pauroso e così fascinoso sogno di felicità, quel desiderio era soddisfatto. Pallido, con la mascella inferiore che tremava, egli stava in piedi, chino su di lei, e la supplicava di calmarsi, non sapendo egli stesso di che, di che cosa. 

- Anna, Anna - diceva, con voce tremante - Anna, in nome di Dio! 

Ma quanto più forte egli parlava, tanto più bassa ella chinava la testa, un tempo orgogliosa e gaia, ora vergognosa; e si piegava tutta e scivolava dal divano sul quale era poggiata verso terra, ai piedi di lui; sarebbe caduta sul tappeto s'egli non l'avesse sorretta. 

- Dio mio, perdonami! - diceva, singhiozzando, stringendo al petto le mani di lui. 

Si sentiva così colpevole e peccatrice che non le restava che prostrarsi e chiedere perdono; ma adesso, nella sua vita, all'infuori di lui, non c'era più nessuno, e a lui volgeva la sua preghiera di perdono. Guardandolo, sentiva fisicamente la propria abiezione, e non poteva più parlare. Egli, invece, sentiva quello che deve sentire l'assassino quando vede il corpo da lui privato della vita. Questo corpo da lui privato della vita era il loro amore, il primo tempo del loro amore. C'era orrore e ripugnanza nel ricordare quello ch'era stato pagato a un così pauroso prezzo di vergogna. La vergogna dinanzi alla propria nudità spirituale soffocava lei e si comunicava a lui. Ma nonostante tutto l'orrore dell'assassino dinanzi al corpo assassinato, occorre fare a pezzi questo corpo, nasconderlo, valersi di ciò che l'assassino, uccidendo, ha conquistato. 

E con accanimento, con furore quasi, colui che ha ucciso si getta su questo corpo, e lo trascina e smembra: così anch'egli copriva di baci il viso e le spalle di lei. Ella gli teneva stretta una mano e non si moveva. Ecco, questi baci sono il prezzo di questa vergogna. Anche questa mano che sarà sempre mia, è la mano del mio complice. Sollevò la mano e la baciò. Egli si piegò sulle ginocchia e voleva scoprirle il viso, ma lei si nascondeva e non diceva nulla. Finalmente, facendo uno sforzo, si sollevò e lo respinse. Il suo viso era sempre bello, ma faceva tanta più pena. 

- Tutto è finito - disse. - Non ho nessuno all'infuori di te. Ricordalo. 

- Io non posso non ricordare quello che è la mia vita. Per me, un attimo di questa felicità\ldots{} 

- Quale felicità! - disse lei con ribrezzo e orrore; e l'orrore si comunicò a lui. - Per amor di Dio, non una parola, non una parola di più. 

Si alzò in fretta e si scostò. 

- Non una parola di più - ripeté e, con un'espressione strana, a lui sconosciuta, di fredda disperazione, andò via. Sentiva di non poter dire la vergogna, la gioia e l'orrore che provava nell'entrare in quella nuova vita, e non voleva parlarne e non voleva rendere volgare, con parole inadatte, quel che sentiva. Ma anche dopo, l'indomani, e il giorno seguente, non trovò le parole adatte a dire tutto il complesso delle sue sensazioni, e così neppure le idee adatte a mettere ordine nell'animo suo. 

``No, adesso non posso pensare - si diceva - dopo, quando sarò tranquilla''. Ma questa tranquillità per riflettere non veniva mai; ogni volta che le tornava in mente quello che aveva fatto, quello che sarebbe stato di lei e quello che doveva fare, era presa dallo sgomento e allontanava questi pensieri. 

``Dopo, dopo - diceva - quando sarò più tranquilla''. 

Nel sonno, invece, quando non aveva il dominio dei suoi pensieri, la situazione le appariva in tutta la sua informe nudità. Un unico identico sogno la visitava quasi ogni notte. Sognava che tutti e due erano nello stesso tempo suoi mariti, che tutti e due le prodigavano le loro carezze. Aleksej Aleksandrovic piangeva, baciandole le mani, e diceva: ``Come si sta bene, ora!''. E Aleksej Vronskij era là, e anche lui era suo marito. Ed ella stupiva come questo le fosse apparso prima impossibile, e spiegava loro, ridendo, che era molto più semplice, e che ora entrambi erano felici e contenti. Ma questo sogno la soffocava come un incubo. 

\capitolo{XII}\label{xii-1} 

Ancora nei primi tempi dopo il suo ritorno da Mosca, Levin, fremendo ed arrossendo ogni volta che ricordava l'offesa del rifiuto, finiva col dire a se stesso: ``Arrossivo e fremevo proprio così giudicando tutto perduto, quando presi uno in fisica e dovetti ripetere l'anno; così pure mi considerai fallito quando persi la causa affidatami da mia sorella. Ebbene?\ldots{} ora che gli anni sono passati, ricordo e stupisco come abbia potuto addolorarmene tanto. Sarà lo stesso anche per questo dispiacere. Passerà il tempo, e diverrò indifferente anche a questo''. 

Ma erano passati tre mesi e non diventava indifferente, e gli doleva, come nei primi giorni, questo ricordo. Non riusciva a rasserenarsene, perché, dopo aver sognato così a lungo una vita di famiglia, e sentendosi ormai maturo per essa, non s'era sposato, e s'era più che mai allontanato dal matrimonio. Sentiva, come lo sentivano tutti quelli che lo circondavano, che per un uomo della sua età rimaner celibe era un male. Ricordava che prima di partire per Mosca, aveva detto un giorno a Nikolaj il bovaro, un brav'uomo col quale amava parlare: ``Ehi, Nikolaj, voglio prender moglie'', e Nikolaj aveva risposto senza indugio, come di una cosa di cui non s'avesse a dubitare: ``È tempo da un pezzo, Konstantin Dmitric''. Ma il matrimonio s'era fatto più lontano che mai. Il posto nel suo cuore era occupato, e quando gli capitava di sostituirvi nell'immaginazione qualcuna delle ragazze di sua conoscenza, sentiva che tale sostituzione era assolutamente impossibile. Inoltre il ricordo del rifiuto e della parte che aveva recitato in quell'occasione, lo tormentava di vergogna. Per quanto si dicesse che non era per nulla colpevole, questo ricordo, al pari degli altri ricordi umilianti di tal genere, lo costringeva a rabbrividire e ad arrossire. Nel suo passato, come in quello di ogni uomo, c'erano delle cattive azioni da lui riconosciute come tali, per le quali la coscienza avrebbe dovuto rimordergli; ma il ricordo di queste cattive azioni era ben lungi dal tormentarlo allo stesso modo di questi inconsistenti, ma umilianti ricordi. Questa ferita non si rimarginava mai. E nel ricordo venivano a trovarsi adesso, sullo stesso piano, e il rifiuto e quella situazione penosa in cui era apparso agli altri in quella sera. Ma il tempo e le occupazioni facevano l'opera loro. I ricordi penosi venivano sempre più velati dagli impercettibili, ma significativi avvenimenti della vita di campagna. Di settimana in settimana ricordava sempre più di rado Kitty. Aspettava con ansia la notizia che si fosse sposata o stesse per sposarsi a giorni; sperava che una notizia simile, come l'estirpazione di un dente, finisse col guarirlo. 

Sopraggiunse intanto la primavera, splendida, improvvisa, senza le attese e gli inganni delle primavere; una di quelle primavere di cui si rallegrano insieme e piante e bestie e uomini. Questa primavera bellissima rianimò ancor più Levin e lo confermò nel suo proposito di rinunciare a tutti i suoi sogni precedenti per costruire, salda e indipendente, la sua vita di uomo solo. Pur non avendo mantenuto fede a molti propositi che aveva formulato nel viaggio di ritorno, tuttavia, l'aspirazione prima, la continenza di vita, egli l'aveva osservata. Non provava la vergogna che di solito lo tormentava dopo ogni caduta, e poteva coraggiosamente guardare in faccia agli uomini. Inoltre, in febbraio, aveva ricevuta da Mar'ja Nikolaevna una lettera in cui si diceva che le condizioni di salute del fratello erano peggiorate, e che egli non voleva curarsi; in seguito a questa lettera, Levin era andato a Mosca e aveva fatto in tempo a persuadere il fratello a consigliarsi con un medico e ad andare all'estero per la cura delle acque. Gli era riuscito così bene di convincere il fratello e di dargli in prestito, senza irritarlo, del denaro per il viaggio, che, sotto questo rapporto, era soddisfatto di sé. Oltre l'azienda che esigeva cure particolari in primavera, Levin aveva anche cominciato a scrivere, in quell'inverno, un libro di economia, la cui tesi consisteva nell'assumere in economia il temperamento del lavoratore come un dato assoluto, così come il suolo e il clima, e nel sostenere che tutte le tesi dell'economia dovessero essere di conseguenza dedotte non dai soli dati del suolo e del clima, ma da quelli del suolo, del clima e di un certo immutabile temperamento del lavoratore. Così che, malgrado la solitudine, e anzi proprio per la solitudine, la sua vita era straordinariamente ricca, e solo di rado sentiva il bisogno insoddisfatto di comunicare i pensieri che gli passavano per la testa a qualcuno che non fosse Agaf'ja Michajlovna, benché anche con lei gli accadesse di ragionar di fisica, di agraria e in particolare di filosofia; la filosofia, anzi, era l'argomento preferito da Agaf'ja Michajlovna. 

La primavera aveva tardato ad arrivare. Nelle ultime settimane della quaresima il tempo era stato sereno, gelido. Di giorno, al sole, sgelava; di notte la temperatura scendeva a sette gradi sotto lo zero. La neve era così indurita che i carri non seguivano più la strada. Per Pasqua c'era ancora la neve. Ma due giorni dopo la settimana santa, si levò a un tratto un vento tiepido, le nuvole si addensarono, e per tre notti cadde una pioggia burrascosa e calda. Il giovedì, il vento si calmò e, quasi a nascondere il mistero dei cambiamenti che si operavano nella natura, avanzò una nebbia fitta e grigia. Nella nebbia si sciolsero le acque, crepitarono e si smossero i ghiacci, più rapidi corsero i torrenti torbidi e schiumosi, e proprio per la domenica in Albis, la sera si squarciò la nebbia, le nuvole corsero via a pecorelle, si rasserenò, e si schiuse la primavera. Al mattino il sole, levatosi splendidamente, divorò in fretta il ghiaccio sottile che aveva coperto le acque, e l'aria trepidò dei vapori che si sprigionavano dalla terra rianimata, invadendola tutta. Verzicò l'erba vecchia e la novella che spuntava ad aghi; si gonfiarono le gemme del viburno, del ribes e della betulla viscosa e inebriante, e su di un ramo di salice, soffuso di fiori d'oro, prese a ronzare un'ape rimasta fuori che vagava all'intorno. Allodole invisibili presero a trillare sul velluto delle verzure e sulla stoppia gelata; piansero le pavoncelle sulle bassure e sulle paludi piene d'acqua nera non ancora riassorbita, e in alto, a volo, con un gridìo di primavera, passarono cicogne e oche. Gli armenti, che non avevano ancora del tutto mutato il pelo, presero a muggire nei pascoli, e gli agnelli dalla zampe ritorte ruzzarono intorno alle madri belanti che mutavano il vello, mentre i ragazzi dalle gambe agili presero a correre per i tratturi che, rasciugandosi, conservavano le impronte dei piedi scalzi; accanto allo stagno crepitarono le voci allegre delle comari intente a candeggiar le tele, e sulle aie risonarono le accette dei contadini che racconciavano aratri ed erpici. Era venuta la vera primavera. 

\capitolo{XIII}\label{xiii-1} 

Levin infilò gli stivali alti e, per la prima volta, indossò, invece della pelliccia, un giubbotto di panno, e s'avviò per il podere, saltando fra i rigagnoli che ferivano gli occhi luccicando al sole, e mettendo il piede ora su un ghiacciolo ora sul fango viscido. 

La primavera è il tempo dei progetti e dei propositi. Uscendo fuori, Levin, come un albero che non sa ancora, in primavera, dove e come spunteranno i germogli e i rami racchiusi nelle gemme turgide, non sapeva egli stesso bene a quali imprese si sarebbe particolarmente accinto ora, nella sua cara azienda; sentiva solo d'aver dentro di sé un mondo di pensieri e i migliori propositi. Per prima cosa andò a dare un'occhiata al bestiame. Le mucche erano state sospinte nel recinto e, luccicanti nel pelo liscio or ora mutato, riscaldatesi al sole, muggivano chiedendo di andare nei prati. Compiaciuto delle mucche che conosceva fin nei più piccoli particolari, Levin ordinò che venissero condotte al pascolo, e che nel recinto si lasciassero circolare i vitelli. Il mandriano corse allegro a prepararsi per andar nei campi. Le donne, sollevando le gonne e guazzando nel fango con i bianchi piedi nudi, non ancora abbronzati, correvano tenendo in mano frasche secche dietro i vitelli che muggivano e ruzzavano di gioia primaverile, e li sospingevano nel cortile. 

Soddisfatto dell'incremento del bestiame, che quell'anno era stato eccezionalmente fecondo, (i vitelli, precoci, erano come vacche da lavoro, la figlia di Pava, di tre mesi appena, sembrava già di un anno), Levin fece portar fuori la mangiatoia e dare il fieno fuori dalle greppie. Ma nel recinto chiuso, non adoperato nell'inverno, constatò che le greppie costruite nell'autunno erano rotte. Fece chiamare il falegname a cui era stato dato l'ordine di lavorare ad una trebbiatrice. Gli dissero che il falegname, invece, stava riparando gli erpici che avrebbero dovuti essere pronti fin da carnevale. Questo spiacque molto a Levin. Era infatti spiacevole che si ripetesse l'eterno disordine dell'azienda, contro il quale da tanti anni lottava con tutte le sue forze. Venne a sapere che le greppie, inutilizzabili d'inverno, erano state trasferite nella stalla dei cavalli da tiro, e là s'erano spezzate perché, costruite per i vitelli, erano risultate troppo leggere per i cavalli. Inoltre, era ormai chiaro che gli erpici e tutti gli strumenti agricoli che egli aveva ordinato di esaminare e di riparare durante l'inverno (lavoro pel quale erano stati assunti tre falegnami), non erano stati riparati, e che agli erpici si andava provvedendo ora che era già tempo di erpicare. Levin mandò a chiamare l'amministratore, e poi andò a cercarlo egli stesso. L'amministratore, risplendente, come ogni cosa in quel giorno, in un pellicciotto di montone guarnito d'agnina, veniva dall'aia, sminuzzando nelle mani una pagliuzza. 

- Perché il falegname non lavora alla trebbiatrice? 

- Eh, già, ve lo volevo dire ieri; era necessario accomodare gli erpici. Ecco che è già tempo d'arare. 

- E allora d'inverno che s'è fatto? 

- Ma perché vi occorre il falegname? 

- Dove sono le greppie del recinto dei vitelli? 

- Ho detto di portarle al posto loro. Che volete fare, con questa gente\ldots{} - disse l'amministratore, con un gesto della mano. 

- Altro che con questa gente! Con questo amministratore! - disse Levin, riscaldandosi. - Ma allora che vi tengo a fare? - gridò. Ma poi, ricordandosi che così non riparava a nulla, si fermò a mezzo il discorso e sospirò. - Su via, si può seminare? - domandò dopo essere rimasto per un po' in silenzio. 

- Al di là di Turkin sì, che si potrà, domani o domani l'altro. 

- E il trifoglio? 

- Ho mandato Vasilij e Miška a seminare. Ma non so se riusciranno a passare: c'è fango. 

- Su quante desjatiny? 

- Su sei. 

- E perché non su tutte? - urlò Levin. 

Che il trifoglio venisse seminato soltanto su sei e non su venti desjatiny, era ancora più increscioso. La seminagione del trifoglio, e teoricamente, e per sua personale esperienza, rendeva solo se fatta al più presto possibile e quasi sulla neve. E Levin non riusciva mai a ottenere che così si facesse. 

- Non ci sono gli operai; cosa mai volete che faccia con questa gente? Tre non sono venuti. Ma ecco Semën\ldots{} 

- Ma via, avreste dovuto toglierne dal lavoro della paglia. 

- Ma ne ho tolti anche di là. 

- Dove sono gli operai? 

- Cinque fanno lo sconcio - voleva dire ``il concio''. - Quattro trasportano l'avena\ldots{} ma anche quella, purché non prenda a ``sguigliare'', Konstantin Dmitric! 

Levin intendeva bene che ``purché non prenda a sguigliare'' significava che l'avena inglese da semenza l'avevano già fatta marcire; ancora una volta non era stato fatto quello che aveva ordinato. 

- Ma se l'ho detto che era ancora quaresima, trombone! - gridò. 

- Non v'inquietate, faremo tutto in tempo! 

Levin agitò con rabbia la mano, andò in granaio a dare un'occhiata all'avena, e tornò alla stalla. L'avena non era ancora andata a male; ma gli operai la rimovevano con le pale, quando sarebbe stato più facile farla scendere direttamente nella rimessa sottostante. Dati gli ordini in proposito, e tolti di lì due operai per la semina del trifoglio, Levin, rabbonito, si liberò della collera contro l'amministratore. Il tempo era così bello che non c'era modo di arrabbiarsi. 

- Ignat! - gridò al cocchiere che, con le maniche rimboccate, lavava una carrozza accanto al pozzo. - Metti la sella a\ldots{} 

- Chi volete? 

- Su, magari, vada per Kolpik. 

- Sissignore. 

Mentre sellavano il cavallo, Levin chiamò di nuovo l'amministratore che gli gironzolava intorno con l'evidente intenzione di far pace, e prese a parlargli dei lavori da farsi in primavera e dei suoi progetti agricoli. 

Bisognava cominciare al più presto il trasporto del concio, in modo da finire alla prima falciatura. E arare senza interruzione il campo più lontano per serbarlo come maggese nero. Il fieno bisognava falciarlo tutto, non a mezzadria, ma coi braccianti. 

L'amministratore ascoltava attento, ma era evidente che faceva uno sforzo per dare a intendere che approvava i progetti del padrone, e aveva, suo malgrado, quell'aria sfiduciata e rassegnata, ben nota a Levin, che sempre se ne irritava. Sembrava dire: ``tutto va bene, ma sarà come Dio vorrà''. 

Nulla amareggiava Levin più di questo atteggiamento. Ma era l'atteggiamento comune a tutti gli amministratori, quanti gliene erano passati per le mani. Tutti si comportavano allo stesso modo verso le sue nuove idee, perciò egli non se ne adirava più, ma se ne amareggiava e si sentiva ancor più spinto a lottare contro questa forza primordiale che gli si opponeva continuamente e che egli non sapeva definire altrimenti che ``come Dio vorrà''. 

- Se ce la faremo, Konstantin Dmitric - disse l'amministratore. 

- Perché non si dovrebbe farcela? 

- Bisogna ancora assumere almeno altri quindici operai. Ed ecco che non vengono. Oggi qualcuno è venuto, ma chiedono settanta rubli per l'estate. 

Levin tacque. Di nuovo gli si parava di fronte quella forza. Sapeva che, per quanto si cercasse, non si sarebbe potuto assumere più di quaranta, trentasette, trentotto operai al prezzo giusto: forse anche quaranta se ne potevano assumere, ma certamente non di più; tuttavia non poteva non lottare. 

- Mandateli a cercare a Sury, a cefirovka, se non vengono. Bisogna cercare. 

- Per cercare io cerco - disse sommessamente Vasilij Fëdorovic. - Ma poi, anche i cavalli si sono infiacchiti. 

- Ne compreremo degli altri. Perché io lo so - aggiunse, ridendo - quando fate voi, ne vien fuori sempre il meno e sempre il peggio; ma quest'anno non vi permetterò di fare a modo vostro. Farò tutto io. 

- Ma voi, del resto, anche ora, mi pare, non state dormendo. Del resto, noi viviamo più contenti sotto l'occhio del padrone. 

- Dunque, di là dal Berëzovyj Dol, si semina il trifoglio? Vado a vedere - disse, assestandosi sul piccolo Kolpik, il sauro che era stato condotto dal garzone. 

- Per il ruscello non passerete Konstantin Dmitric - gridò il garzone. 

- Su via, allora, per il bosco. 

E sull'arzilla andatura del buon cavallino che era rimasto a lungo a riposo, e che sbruffava sulle pozzanghere, chiedendo le briglie, Levin si avviò attraverso il fango del cortile, oltre il portone, verso i campi. 

Se Levin si rallegrava nel cortile del bestiame e in quello delle mucche, si rallegrava ancor più nei campi. Dondolandosi alla cadenza dell'ambio del buon cavallino, aspirando l'odore tiepido e fresco dell'aria e della neve, attraversava il bosco sul nevischio rimasto qua e là, sulla neve sfaldata sulla quale le impronte si andavano sciogliendo. Godeva di ogni pianta rigonfia di gemme, avvivata dal musco sulla corteccia. Quando uscì di là dal bosco, dinanzi a lui si distendevano, per uno spazio enorme, i prati verdi, come un liscio tappeto di velluto, senza piazzuole né pozzanghere, macchiati solo qua e là negli avvallamenti dai resti della neve che andava sciogliendosi. Levin non si turbò né alla vista di un cavallo da tiro e di uno stallone che calpestavano i suoi prati (ordinò a un contadino col quale s'era imbattuto di cacciarli via), né alla risposta canzonatoria e sciocca di Ipat, il contadino incontrato, il quale alla sua domanda: ``Ohi, Ipat, si semina presto?'' aveva risposto: ``S'ha prima da arare, Konstantin Dmitric!''. Quanto più andava avanti, tanto più gioiva, e i suoi piani di amministrazione gli sembravano l'uno migliore dell'altro: recingere di giunchi tutti i campi in linee meridiane, di modo che la neve non vi rimanesse a lungo; dividerli in sei campi da concio e in tre di riserva per la coltura delle erbe, costruire una stalla sull'estremo limite del campo e scavare una fossa per l'avena e per il concio, costruire dei recinti trasportabili per il bestiame al pascolo. E così avrebbe avuto trecento desjatiny di frumento, cento di patate, centocinquanta di trifoglio e neanche una desjatiny incolta. 

Con questi sogni, conducendo accorto il cavallo sui viottoli terminali per non calpestare i suoi prati, si avvicinò agli operai che seminavano il trifoglio. Il carro con la semenza era fermo, non sul limite, ma sul campo arato, e il frumento autunnale era solcato dalle ruote e scavato dalle zampe del cavallo. Tutti e due gli operai sedevano sulla proda, fumando la pipa, probabilmente a turno. La terra che era sul carro, frammischiata ai semi, non era impastata, ma tutta impiastricciata e a pallottole. Scorgendo il padrone, l'operaio Vasilij si mosse verso il carro e Miška si diede a seminare. Anche questo non andava bene, ma Levin si adirava di rado con gli operai. Quando Vasilij si avvicinò, Levin gli ordinò di portare il cavallo sulla proda. 

- Non fa nulla, padrone, si rimargina - rispose Vasilij. 

- Ti prego, non stare a discutere - disse Levin - ma fa' quello che ti vien detto. 

- Sissignore - rispose Vasilij e prese il cavallo per la cavezza. - Ma la semenza, Konstantin Dmitric - disse, adulando - è di prima qualità. Solo che camminare è un guaio! Tiri su un pud con un solo piede. 

- E perché non avete setacciato la terra? - disse Levin. 

- Ma la gramoliamo noi - rispose Vasilij, prendendo su della semenza e impastandovi un po' di terra nelle mani. 

Vasilij non aveva colpa lui, se gli avevano messo della terra non setacciata, tuttavia ciò era spiacevole. 

Ma Levin, avendo sperimentato più di una volta, con profitto, un mezzo sicuro per soffocare il proprio dispetto e per far tornare ad andar bene quel che sembrava andar male, lo provò anche in questo momento. Vide che Miška camminava a grandi passi, facendo rotolare enormi zolle di terreno che gli si appiccicavano ai piedi; scese da cavallo, tolse a Vasilij il sacco della semenza e andò a seminare. 

- Dove ti sei fermato? 

Vasilij fece un segno col piede, e Levin andò a seminare, così come sapeva far lui, il terreno misto alla semenza. Andare avanti era difficile, proprio come in un pantano; e Levin, seminato che ebbe un solco, cominciò a sudare e, fermatosi, restituì il sacco con la semenza. 

- Ohi, padrone, bada bene a non prendertela con me questa estate, per questo solco qua! - disse Vasilij. 

- E che c'è - disse allegro Levin, scorgendo già l'effetto del mezzo adoperato. 

- Sì, ecco, vedrete poi quest'estate. Si vedrà la differenza. Date un'occhiata dove ho seminato io la primavera scorsa. Come ho dato la semenza! Ecco, Konstantin Dmitric, io mi adopero, ecco, proprio come se foste il padre mio carnale. A me stesso non piace il lavoro fatto male, e non permetto che gli altri lo facciano male. Se va bene per il padrone, va bene anche per noi. Se dai un'occhiata laggiù - disse Vasilij, mostrando il campo - ti si rallegra il cuore. 

- Che bella primavera, Vasilij! 

- È una primavera che i vecchi non ricordano più bella. Io, ecco, sono stato a casa mia; anche là da noi il vecchietto ha seminato tre stai di frumento. Dice che non lo si distingue dalla segala. 

- E voi, è un pezzo che avete preso a seminare il frumento? 

- Ma se siete stato voi a insegnarcelo l'anno scorso! E me ne avete regalate pure due misure. Un quarto l'abbiamo venduto e tre stai l'abbiamo seminati. 

- Su, guarda, sfarina le pallottole - disse Levin, avvicinandosi al cavallo - e da' un occhio a Miška. E se verrà su bene, ti darò cinquanta copeche per desjatina. 

- Ringrazio umilmente! Noi, mi pare, anche così siamo molto contenti di voi. 

Levin montò a cavallo e andò nel campo dove c'era il trifoglio dell'anno precedente, e in quello arato, pronto per il grano marzuolo. 

Il trifoglio da stoppia veniva su magnificamente. S'era già tutto avvivato e verzicava dietro gli steli del frumento dell'anno prima. Il cavallo vi affondava fino al ginocchio e ogni sua zampata provocava uno scroscio quando si liberava dalla terra mezzo disgelata. Per i solchi arati non si poteva proprio passare; solo dove c'era un po' di ghiaccio il terreno sosteneva, ma nei solchi disgelati la zampa affondava fino a sopra il ginocchio. Ottima l'aratura; fra due giorni si sarebbe potuto erpicare e seminare. Tutto era bello, tutto era festoso. Levin decise di tornare indietro attraverso il ruscello, sperando che l'acqua vi fosse più bassa. E in effetti lo passò a guado, spaventando due anitre. ``Ci devono essere anche le beccacce'' pensò, e, proprio alla svolta per tornare a casa, incontrò il guardaboschi che lo confermò nella sua supposizione. 

Levin tornò a casa al trotto, per fare in tempo a mangiare e a preparare il fucile per la sera. 

\capitolo{XIV}\label{xiv-1} 

Mentre nella migliore disposizione d'animo si avvicinava a casa, Levin sentì un tinnir di sonagli dalla parte principale dell'ingresso della casa. 

``Ma è qualcuno che viene dalla stazione - pensò - è proprio l'ora del treno di Mosca\ldots{} Chi può essere? Che sia Nikolaj? L'ha detto del resto: `Può darsi che vada a fare la cura delle acque, ma chi sa che non venga da te'\,''. Sulle prime provò sgomento e rammarico al pensiero che la presenza del fratello Nikolaj non avesse a turbare quella sua felice disposizione d'animo. Ma poi si vergognò di questo suo sentimento, e subito gli aprì, per così dire, spiritualmente le braccia, e con gioia intenerita s'aspettò e desiderò con tutta l'anima che fosse il fratello. Stimolò il cavallo e, oltrepassata l'acacia, vide la trojka postale della stazione ferroviaria e un signore in pelliccia. Non era il fratello. ``Ah, se fosse qualche persona simpatica con la quale poter parlare!'' pensò. 

- Ah - gridò con gioia Levin, alzando tutte e due le braccia. - Ecco un ospite gradito! Ah, come sono felice di vederti! - gridò, riconoscendo Stepan Arkad'ic. 

``Così probabilmente saprò se si è sposata o quando si sposerà'' pensò. 

E in quella magnifica giornata di primavera, sentì che il ricordo di lei non gli faceva più alcun male. 

- Forse non m'aspettavi? - disse Stepan Arkad'ic, uscendo dalla slitta con vari schizzi di fango alla radice del naso, sulla guancia e sul sopracciglio, ma splendente di buonumore e di salute. - Sono venuto, prima di tutto, per vederti - disse, abbracciandolo e baciandolo; - poi, per fermarmi un po' per la caccia, ed infine anche per vendere il bosco di Ergušovo. 

- Benone! Ma che primavera! com'è che sei arrivato fin qui in slitta? 

- In carrozza è anche peggio, Konstantin Dmitric - rispose il postiglione che lo conosceva. 

- Be', sono molto contento di vederti - disse Levin, sorridendo sinceramente di un riso infantile e festoso. 

Levin guidò l'ospite nella camera dei forestieri, dove appunto erano state portate le cose di Stepan Arkad'ic: un sacco, un fucile nel fodero, una borsa per i sigari; e, lasciatolo a lavarsi e a cambiarsi, passò nel frattempo in amministrazione a dare gli ordini per l'aratura e per il trifoglio. Agaf'ja Michajlovna, sempre molto preoccupata del prestigio della casa, gli venne incontro in anticamera con alcune domande intorno al pranzo. 

- Fate come volete, purché al più presto - disse lui, e andò dall'amministratore. 

Quando tornò, Stepan Arkad'ic, lavato, pettinato e raggiante, usciva dalla sua camera, e insieme salirono. 

- Ma come son contento d'essere arrivato fin qui da te! Ora capirò in che cosa consistono i prodigi che tu compi qua! Ma, davvero, ti invidio. Che casa, come tutto è eccellente! - disse Stepan Arkad'ic, dimenticando che non sempre c'erano la primavera e le giornate chiare come quella. - E la tua governante che delizia! Forse sarebbe più desiderabile una graziosa cameriera in grembiulino, ma per il tuo cenobitismo e la tua austerità questo va proprio bene. 

Stepan Arkad'ic raccontò molte cose interessanti e gli diede la notizia, che riguardava in particolare Levin, che il fratello Sergej Ivanovic si preparava ad andare da lui in campagna per l'estate. 

Stepan Arkad'ic non disse neppure una parola di Kitty, né in generale degli Šcerbackij; riferì solo i saluti di sua moglie. Levin gli fu grato di questa delicatezza e fu molto contento dell'ospite. In genere, nel periodo del suo isolamento, gli si accumulavano un'infinità di pensieri e di sentimenti che non poteva comunicare a quelli che lo circondavano, e invece ora egli poteva riversare in Stepan Arkad'ic la gioia poetica della primavera, le vicende e i progetti per l'azienda, le idee e le osservazioni sui libri che aveva letto, e in particolare lo schema della sua opera che aveva a base, sebbene egli stesso non lo notasse, la critica di tutte le vecchie opere di economia. Stepan Arkad'ic, sempre simpatico, che afferrava tutto da un accenno, fu particolarmente cordiale in questo suo soggiorno, e Levin notò anche un nuovo tratto di considerazione e quasi di tenerezza verso di lui, che lo lusingò. 

Gli sforzi di Agaf'ja Michajlovna e del cuoco perché il pranzo fosse in tutto e per tutto ben fatto, produssero l'effetto che i due amici, affamati com'erano, seduti davanti all'antipasto, si rimpinzassero di pane e di burro, di uccelletti e di funghi sotto sale; inoltre, che Levin finisse con l'ordinare di servir la minestra senza gli sfogliantini con i quali il cuoco avrebbe voluto in particolar modo stupire l'ospite. Ma Stepan Arkad'ic, pur abituato a pranzi d'altro genere, trovava tutto eccellente; la salsa verde e il pane e il burro, gli uccelletti e i funghi, la minestra d'ortiche e la gallina in salsa bianca e il vino bianco di Crimea, tutto per lui era straordinario ed eccellente. 

- Ottimo, ottimo - diceva accendendo una grossa sigaretta dopo l'arrosto. - Sono arrivato da te proprio come chi, uscendo dal frastuono e dal rollio di un piroscafo, giunga ad una spiaggia silenziosa. Così, allora, tu dici che anche l'elemento ``lavoratore'' dev'essere preso in considerazione e deve guidare nella scelta dei sistemi economici. Io, già, in questo sono un profano; ma mi sembra che la tua teoria e la sua applicazione potranno incidere sul lavoratore. 

- Sì, ma aspetta: io non parlo di economia politica, parlo di scienza agraria. Questa, come le scienze naturali, deve prendere in esame i fenomeni dati e il lavoratore con le sue caratteristiche economiche, etnografiche\ldots{} 

In quel momento entrò Agaf'ja Michajlovna con la marmellata. 

- Ehi, Agaf'ja Michajlovna - le disse Stepan Arkad'ic baciandosi la punta delle dita grassocce - che uccelletti che avete!\ldots{} E che, non è ora Kostja? - aggiunse. 

Levin guardò dalla finestra il sole che scendeva dietro le cime del bosco che si riuscivano a scorgere. 

- È ora, è ora - disse. - Kuz'ma, fa' attaccare il calesse - e corse giù. 

Stepan Arkad'ic, disceso, tolse egli stesso con cura la fodera di tela dall'astuccio verniciato e, apertolo, cominciò a montare il suo costoso fucile di nuovo modello. Kuz'ma, che fiutava fin d'ora una grossa mancia, non si allontanava da Stepan Arkad'ic, e gli infilava calze e stivali, mentre Stepan Arkad'ic lasciava fare volentieri. 

- Ti prego, Kostja, se viene Rjabinin il compratore, gli ho detto di venire quest'oggi, fallo ricevere, e che mi aspetti\ldots{} 

- Ma forse vendi il bosco a Rjabinin? 

- Sì, lo conosci, per caso? 

- Altro se lo conosco. Ho avuto un affare con lui ``positivamente e definitivamente''. 

Stepan Arkad'ic rise. ``Definitivamente e positivamente'' erano gli intercalari del compratore. 

- Già, parla in modo proprio buffo. Ha capito dove va il padrone! - aggiunse, tastando con la mano Laska che gironzolava intorno a Levin mugolando e leccandogli ora una mano, ora gli stivali, ora il fucile. 

La vettura era già accanto alla scalinata, quando uscirono. 

- Ho ordinato di attaccare, sebbene non sia lontano: vogliamo andare a piedi? 

- No, meglio in carrozza - disse Stepan Arkad'ic, accostandosi alla vettura. Sedette, avvoltolò le gambe in uno scialle tigrato, e accese un sigaro. - Com'è che non fumi? Il sigaro non è proprio un godimento, ma il coronamento, il segno del godimento. Ecco, questa è vita! Come si sta bene! Ecco come vorrei vivere io! 

- E che cosa te lo impedisce? - disse Levin, sorridendo. 

- No, tu sei un uomo felice. Hai tutto quello che ti piace. Ti piacciono i cavalli\ldots{} ne hai, i cani\ldots{} ne hai, la caccia\ldots{} ce l'hai, l'azienda\ldots{} ce l'hai. 

- Forse perché mi contento di quello che ho, e non rimpiango quello che non ho - disse Levin, ricordandosi di Kitty. 

Stepan Arkad'ic capì, lo guardò, ma non disse nulla. 

Levin era grato a Oblonskij di aver notato, con il suo tatto abituale, ch'egli temeva il discorso sugli Šcerbackij, e di non avere detto nulla di loro; ora però Levin cominciava a desiderare di sapere quello che lo tormentava; ma non osava avviare il discorso. 

- Be', i tuoi affari come vanno? - disse Levin, dopo aver pensato che fosse poco gentile da parte sua pensare solo a se stesso. 

Gli occhi di Stepan Arkad'ic brillarono allegramente. 

- Tu, è vero, non ammetti che possano piacere le ciambelle, quando si ha la razione assegnata; questo per te è un delitto; ma io non so comprendere la vita senza amore - disse, interpretando a modo suo la domanda di Levin. - Che farci, son fatto così. E invero, con questo si fa tanto poco male a qualcuno e tanto piacere a se stesso. 

- Be', c'è forse qualcosa di nuovo? - disse Levin. 

- C'è, amico mio! Ecco, vedi: conosci il tipo delle donne ossianesche\ldots{} delle donne che vedi in sogno\ldots{} Queste donne vivono nella realtà\ldots{} e queste donne sono fatali. La donna, vedi, per quanto tu la studi, è un soggetto sempre nuovo. 

- Allora è meglio non studiarlo. 

- No, un matematico ha detto che la gioia non consiste nella scoperta della verità, ma nella ricerca di essa. 

Levin ascoltava in silenzio e, pur facendo tutti gli sforzi su se stesso, non riusciva in nessun modo a trasferirsi nell'animo dell'amico, non riusciva a capire i suoi sentimenti e il piacere ch'egli provava nello studio di donne siffatte. 

\capitolo{XV}\label{xv-1} 

Il passo non era lontano, al di sopra del fiume, in un boschetto di tremule. Giunti al bosco, Levin accompagnò Oblonskij all'angolo di una radura coperta di musco e di fango, già sgombra di neve. Egli stesso tornò indietro, all'altro estremo, verso una betulla doppia, e, appoggiato il fucile alla biforcazione del ramo inferiore secco, si tolse il pastrano, si mise la cintura e provò la scioltezza dei movimenti delle braccia. 

Lanka che gli andava dietro passo passo, grigia e vecchiotta, s'accucciò guardinga di fronte a lui, e tese le orecchie. Il sole scendeva dietro al bosco grande, e nella luce del tramonto le giovani betulle sparse fra le tremule si disegnavano nette coi loro rami pendenti dalle gemme gonfie, pronte a scoppiare. 

Dal bosco fitto, dove era rimasta ancora neve, scorreva appena percettibile l'acqua in rigagnoli stretti e tortuosi. Uccelli piccoli cinguettavano e di tanto in tanto frullavano da un albero all'altro. 

Negli intervalli di calma completa, si poteva udire il crepitar delle foglie dell'anno prima, smosse dallo sgelo della terra e dal germinare delle erbe. 

``Che meraviglia! Si sente e si vede come cresce l'erba!'' si disse Levin, notando una foglia bagnata di tremula color lavagna che si moveva accanto a un filo d'erba nuova. Egli stava in piedi, in ascolto, e guardava ora la terra umida muscosa, ora Laska tutt'orecchi, ora il mare delle cime spoglie degli alberi che si stendeva dinanzi a lui ai piedi della montagna, ora il cielo che scolorava velato da strati bianchi di nuvole. Un falco, battendo le ali lentamente, volò alto sul bosco lontano; un secondo, con moto eguale, volò nella stessa direzione e scomparve. Gli uccelli presero a cinguettare ancor più chiassosi e insistenti nel fitto del bosco. Non lontano urlò un gufo, e Laska, rabbrividendo, fece alcuni passi accorti e, piegata la testa da un lato, si mise in ascolto. Di là dal fiume si udì il cuculo. Per due volte lanciò il solito verso, poi s'arrochì, abborracciò, barbugliò. 

- Che bellezza! di già il cuculo! - disse Stepan Arkad'ic uscendo di dietro a un cespuglio. 

- Già, ho sentito - rispose Levin, rammaricandosi di rompere il silenzio del bosco con la propria voce, sgradita a lui stesso. - Ecco, arrivano! 

La figura di Stepan Arkad'ic passò di nuovo dietro al cespuglio e Levin vide solo la fiammella viva di un fiammifero seguìta subito dopo dal fuoco rosso della sigaretta e da un piccolo fumo turchino. 

Cik! cik!, scattarono i cani del fucile alzati da Stepan Arkad'ic. 

- Che cos'è che stride? - domandò Oblonskij, attirando l'attenzione di Levin su di uno stridio prolungato, come di un puledro che, ruzzando, nitrisse con voce acuta. 

- Ah, non sai? È una lepre, un maschio. Ma stiamo zitti! Senti?\ldots{} passano! - gridò quasi Levin, alzando i cani del fucile. 

Si udì un fischio lontano e, proprio all'intervallo regolare di due secondi così noto al cacciatore, un secondo, un terzo fischio e, dopo il terzo, lo zirlio era già percettibile. 

Levin girò gli occhi a destra e a sinistra, ed ecco, dinanzi a lui, nel cielo azzurro cupo, al di sopra dei germogli teneri e gonfi delle tremule, apparve l'uccello in volo. Volava diritto verso di lui: lo zirlio ormai vicino, simile allo squarciarsi a intervalli regolari di una grossa tela, gli risonò proprio sopra l'orecchio; si scorgeva già il becco lungo e il collo dell'uccello, ma nel momento in cui Levin prendeva la mira, di dietro al cespuglio dov'era Oblonskij, guizzò un lampo rosso; l'uccello, come una freccia, s'abbassò e salì di nuovo in alto. Guizzò un altro lampo e si udì un colpo, e sbattendo le ali, quasi cercando di reggersi nell'aria, l'uccello si fermò, rimase un attimo sospeso e precipitò pesantemente sul terreno fangoso. 

- Possibile che abbia fatto padella? - gridò Stepan Arkad'ic che non riusciva a vederci per il fumo. 

- Eccola! - disse Levin, indicando Laska che, con un orecchio alzato e agitando la punta della coda lanosa, a passi lenti, come se sorridesse e volesse prolungarsene il piacere, portava l'uccello ucciso al padrone. - Via, son contento che sia riuscito a te - disse Levin, pur provando un certo senso di invidia a non essere stato lui ad ammazzar la beccaccia. 

- Una brutta padella dalla canna destra - rispose Stepan Arkad'ic, ricaricando il fucile. - Sst\ldots{} passano\ldots{} 

Si udivano infatti fischi acuti susseguirsi l'uno all'altro, rapidi. Due beccacce, giocando a rincorrersi e fischiando solo, senza zirlare, volarono sopra le teste dei cacciatori. Risonarono quattro colpi, ma le beccacce, quasi rondini, compirono una voluta rapida e scomparvero dalla vista. 

\begin{center}\rule{3in}{0.4pt}\end{center} 

Il passo fu ottimo. Stepan Arkad'ic uccise due uccelli e Levin due, di cui uno non si trovò. Cominciava a imbrunire. In basso, al di là delle betulle, Venere con la sua luce tenue splendeva chiara d'argento; mentre in alto, a levante, il corrusco Arturo spandeva già la sua luce rossastra. Proprio sopra il suo capo, Levin ora scorgeva, ora smarriva le stelle dell'Orsa. Le beccacce avevano già cessato il volo; ma Levin decise di aspettare che Venere, ch'egli vedeva al di sotto di un piccolo ramo di betulla, passasse al di sopra, e che le stelle dell'Orsa apparissero chiare in ogni punto. Ma Venere aveva già oltrepassato il ramo, il carro dell'Orsa col suo timone era già tutto chiaro nel cielo azzurro fondo, e Levin aspettava ancora. 

- Non è ora? - chiese Stepan Arkad'ic. 

Nel bosco c'era già quiete e neppure il più piccolo uccello si moveva. 

- Restiamo ancora - rispose Levin. 

- Come vuoi. 

Adesso stavano in piedi, a quindici passi l'uno dall'altro. 

- Stiva! - disse a un tratto, inaspettatamente, Levin - come mai non mi dici se tua cognata s'è sposata o sta per sposarsi? 

Si sentiva così sicuro e sereno da ritenere che nessuna risposta potesse turbarlo. Ma proprio non si aspettava quello che rispose Stepan Arkad'ic. 

- Non ci ha pensato e neppure ci pensa a sposarsi; ma è molto malata, e i medici l'hanno mandata all'estero. Si teme persino per la sua vita. 

- Ma che dici? - gridò Levin. - Molto malata? E cosa mai le è accaduto? Come è\ldots{} 

Mentre dicevano questo, Laska, drizzando le orecchie, guardò in alto, verso il cielo, e poi verso di loro con aria di rampogna. ``Ecco, hanno scelto proprio il momento buono per chiacchierare\ldots{} e lei intanto se ne vola\ldots{} Eccola, è proprio così. Se la lasceranno scappare\ldots{}'' pensava Laska. 

Ma in quello stesso momento tutti e due sentirono a un tratto un fischio penetrante frustar loro l'orecchio, e tutti e due imbracciarono il fucile e due colpi risonarono nello stesso istante. La beccaccia, che volava in alto, piegò le ali e cadde nel fitto di un cespuglio curvandone i germogli sottili. 

- Ecco, perfetto! Insieme! - gridò Levin e corse con Laska nel cespuglio a cercare la beccaccia. ``Ah, sì, ma cos'è che m'ha fatto dispiacere? - andava ricordando. - Già, Kitty, che è malata. Ma non c'è nulla da fare; è un gran peccato'' pensava. 

- Ah, l'hai trovata. Ecco, l'intelligentona! - disse prendendo dalla bocca di Laska l'uccello ancora caldo e ponendolo nel carniere quasi pieno. - L'ho trovata, Stiva! - gridò. 

\capitolo{XVI}\label{xvi-1} 

Tornando a casa, Levin chiese tutti i particolari della malattia di Kitty e i progetti degli Šcerbackij, e in fondo (se ne vergognava persino nel confessarlo a se stesso) quello che aveva saputo gli faceva piacere. Gli faceva piacere e perché c'era ancora una speranza e ancor più perché soffriva chi aveva fatto soffrire tanto lui. Ma quando Stepan Arkad'ic cominciò a parlare della cause della malattia di Kitty e fece il nome di Vronskij, Levin lo interruppe. 

- Io non ho alcun diritto di sapere i particolari di famiglia, e, a dire il vero, neanche nessun interesse. 

Stepan Arkad'ic sorrise appena percettibilmente, cogliendo il mutamento subitaneo, e a lui così noto, del viso di Levin, divenuto tanto scuro quanto allegro era stato un momento prima. 

- Hai concluso del tutto il taglio del bosco con Rjabinin? - chiese Levin. 

- Sì, ho concluso. Il prezzo è ottimo, trentottomila rubli: otto anticipati e il resto in sei anni. Ho dovuto faticare per averlo. Nessuno mi offriva di più. 

- In conclusione, l'hai regalato il bosco - disse torvo Levin. 

- Come regalato? - disse Stepan Arkad'ic con un sorriso bonario, sapendo che ormai Levin avrebbe trovato tutto mal fatto. 

- Perché quel bosco vale almeno un cinquecento rubli a desjatina - rispose Levin. 

- Ah, eccoli questi proprietari di terre! - disse Stepan Arkad'ic scherzando. - Questo vostro tono di disprezzo verso noi cittadini!\ldots{} Intanto, quando c'è da concludere un affare, siamo noi a far meglio. Credimi, ho calcolato tutto - disse - e il bosco è stato venduto a condizioni molto vantaggiose: temo persino che egli rifiuti. Certo non è un bosco conveniente - disse Stepan Arkad'ic desiderando con la parola ``conveniente'' convincere Levin dell'infondatezza dei suoi dubbi - più che altro è legna da ardere. E ce ne saranno non più di trenta sazeni per desjatina e lui me ne dà duecento rubli. 

Levin sorrise sprezzante. ``Conosco - pensò - questo modo di fare, non solo suo, ma di tutti gli abitanti di città; vengono in campagna due volte in dieci anni, annotano due o tre termini campagnoli, e li usano a proposito e a sproposito, fermamente convinti di sapere tutto. `Conveniente, ce ne saranno trenta sazeni'. Ripete delle parole, ma non ne conosce il senso''. 

- Io non starò a insegnarti quel che scrivi là al tuo ufficio - disse - ma se fosse necessario chiederei di apprendere da te. E tu, invece, sei così sicuro di capire tutto in materia di legname. È difficile. Hai dato una contata agli alberi? 

- Come, la contata degli alberi? - disse, ridendo, Stepan Arkad'ic, desiderando sempre di far uscire Levin dal suo cattivo umore. - ``Contar le sabbie, i raggi dei pianeti, potrebbe ancora un alto ingegno\ldots{}''. 

- Eh, già, ma intanto l'alto ingegno di Rjabinin, sì, che lo può. E nessun compratore compra un taglio di bosco senza contare, a meno che non glielo regalino, così come hai fatto tu. Conosco il tuo bosco. Ci vado ogni anno a caccia e vale cinquecento rubli contanti a desjatina; mentre lui te ne dà duecento a rate. Il che significa che tu, a lui, ne regali trentamila. 

- Su, via, non esageriamo - disse con pena Stepan Arkad'ic - e allora perché nessuno me li offriva? 

- Perché lui è d'accordo con gli altri. Ho avuto a che fare con tutti loro, li conosco. Non sono dei compratori, ma degli accaparratori. Se non c'è da guadagnare il dieci, il quindici per cento egli non avvia neppure l'affare; aspetta a comprare un rublo con venti copeche. 

- Lascia andare! Tu vedi tutto nero. 

- Niente affatto - disse cupo Levin, mentre si avvicinavano a casa. 

All'ingresso c'era già una carretta tutta ricoperta di ferro e cuoio, con un cavallo ben pasciuto attaccato con corregge larghe ben tese. Nella carretta sedeva un inserviente che faceva da cocchiere a Rjabinin, fortemente stretto da una cintura, con una faccia turgida e iniettata di sangue. Lo stesso Rjabinin era già in casa, e venne incontro agli amici nell'anticamera. Era un uomo di mezza età, alto e rinsecchito, con i baffi, il mento raso sporgente e gli occhi torbidi all'infuori. Vestiva un soprabito turchino a lunghe falde, con i bottoni più in basso del dorso, e sopra agli alti stivali raggrinziti alle caviglie e tirati sui polpacci portava delle grosse calosce. Si asciugò tutto il viso in giro col fazzoletto e, allacciatosi il soprabito, che anche senza di questo chiudeva bene, salutò con un sorriso quelli che erano entrati, tendendo la mano a Stepan Arkad'ic come se volesse afferrare qualcosa. 

- Ah, siete arrivato - disse Stepan Arkad'ic, dandogli la mano. - Benone. 

- Non ho osato mancare a un ordine di vostra eccellenza, benché la strada fosse cattiva. Ho fatto tutta la strada positivamente a piedi, ma sono arrivato in tempo. Konstantin Dmitric, i miei rispetti - disse rivolto a Levin, cercando di afferrare anche a lui la mano. Ma Levin, accigliato, fingeva di non vedere e tirava fuori le beccacce. - I signori si sono divertiti a caccia? Ma che uccelli son codesti? - aggiunse Rjabinin guardando sprezzante le beccacce. - Be', un certo sapore lo avranno! - E scosse il capo, disapprovando e dubitando assai che il giuoco valesse la candela. 

- Vuoi andar nello studio? - disse Levin in francese, sempre scuro in viso, a Stepan Arkad'ic. - Accomodatevi nello studio, parlerete là. 

- Ma dove a voi piace, signore - disse con aria dignitosa e altera Rjabinin per far intendere che per gli altri potevano esserci difficoltà, sul come e con chi trattare l'affare, ma per lui mai e per nessuna cosa. 

Entrando nello studio, Rjabinin, per abitudine, guardò in giro a cercare l'icona, ma, trovatala, non si segnò. Esaminò gli armadi e gli scaffali coi libri, e con lo stesso atteggiamento di diffidenza assunto per le beccacce, sorrise sprezzante e scosse il capo disapprovando, deciso ormai a non ammettere che il giuoco valesse la candela. 

- Be', il denaro lo avete portato? - disse Oblonskij. - Sedete. 

- Noi non ci faremo certo attendere pel denaro. Sono venuto per vedervi, per discorrere un po'. 

- Discorrere di che? Ma sedetevi. 

- Questo sì - disse Rjabinin, sedendosi e appoggiandosi, nel modo più scomodo per lui, alla spalliera della poltrona. - Bisogna che abbassiate un po' il prezzo, principe. Sarebbe un peccato mandare a monte. E i denari son pronti, prontissimi. Fino all'ultima copeca. Pel pagamento non ci sarà ritardo. 

Levin, che, nel frattempo, aveva riposto il fucile nell'armadio e già stava per uscire, udite le parole del compratore, si fermò sulla porta. 

- E così, voi avete preso il bosco per niente - disse. - È giunto tardi da me, altrimenti il prezzo l'avrei fatto io. 

Rjabinin si alzò e guardò Levin in silenzio con un sorriso di sotto in su. 

- Siete molto attaccato al denaro, Konstantin Dmitric - disse, volgendosi a Stepan Arkad'ic con un sorriso. - Decisamente da lui non ci si può comprar nulla. Stavo trattando per il frumento, offrivo dei bei soldi, io. 

- Perché dovrei regalarvi il mio? Non l'ho mica trovato per terra, né rubato. 

- Vi prego. Al giorno d'oggi rubare è definitivamente impossibile. Tutto, al giorno d'oggi è definitivamente di dominio pubblico, oggi tutto è onesto; altro che rubare! Ma via, trattiamo onestamente. È caro questo legname, non ci esco neanche con le spese. Chiedo di cedere almeno di una piccolezza. 

- Ma voi l'affare lo avete concluso sì o no? Se è concluso non c'è più nulla da trattare, se non è concluso - disse Levin - il bosco lo compro io. 

Il sorriso scomparve a un tratto dal viso di Rjabinin. Una espressione di sparviero, rapace e dura, vi si fissò. Con le dita ossute e rapide sbottonò il soprabito, scoprendo la camicia che ne usciva fuori, i bottoni di rame del panciotto e la catena dell'orologio, e in fretta cavò fuori un grosso e vecchio portafoglio. 

- Vi prego, il bosco è mio - pronunciò, dopo essersi fatto in fretta il segno della croce e tendendo la mano. - Ecco il denaro, il bosco è mio. Ecco come fa gli affari Rjabinin, e non bada agli spiccioli - disse accigliato, agitando il portafoglio. 

- Io al posto tuo non avrei fretta - disse Levin. 

- Ti prego - disse sorpreso Oblonskij. - Ho dato la parola. 

Levin uscì dalla stanza, sbattendo la porta. Rjabinin, guardando verso questa, scosse la testa con un sorriso. 

- Gioventù, definitivamente; anzi fanciullaggine. Compro per tanto, credetemi sull'onore, solo per potermi vantare, ecco, che Rjabinin, e nessun altro, ha comprato il bosco da un Oblonskij. E anche, se Dio vuole, per guadagnarci su. Credete a Dio. Vi prego, signore. Scriviamo il contrattino. 

Un'ora dopo il compratore, incrociatasi accuratamente la veste e agganciati gli uncini del soprabito, col contratto in tasca, sedette nella sua carretta ben ferrata per tornarsene a casa. 

- Oh, questi signori! - disse all'inserviente - tutti a un modo! 

- Proprio così - rispose l'inserviente, dandogli le briglie e abbottonando il grembiule di cuoio. - E il vostro affaruccio com'è andato, Michail Ignat'ic? 

- Be', be'\ldots{} 

\capitolo{XVII}\label{xvii-1} 

Stepan Arkad'ic salì con la tasca piena di titoli che gli aveva dato il compratore come rata di tre mesi. L'affare del bosco era concluso, il denaro era in tasca, la caccia era stata magnifica, e Stepan Arkad'ic si trovava nella più amena disposizione d'animo; voleva perciò in particolar modo disperdere il cattivo umore che era piombato su Levin. Aveva voglia di chiudere piacevolmente con la cena la giornata che piacevolmente era cominciata. 

Levin era davvero di cattivo umore e, malgrado il desiderio suo di mostrarsi affettuoso, cordiale con l'ospite simpatico, non riusciva a dominarsi. Lo stordimento della notizia che Kitty non s'era sposata cominciava a prenderlo a poco a poco 

Kitty tuttora nubile e malata, malata d'amore per l'uomo che l'aveva disdegnata! Quest'offesa gli pareva ricadesse su di lui. Vronskij aveva disdegnato lei, e lei aveva respinto lui. Per conseguenza, Vronskij aveva il diritto di disprezzarlo, e perciò Levin sentiva per lui dell'avversione. Ma Levin non aveva chiaro nella mente tutto ciò. Sentiva solo in maniera confusa che in questo c'era qualcosa d'offensivo per lui e se ne irritava: non proprio per quello che l'aveva sconvolto, ma per ogni cosa che intanto gli accadeva. La vendita insensata del bosco, l'inganno in cui era caduto Oblonskij, e che si era perpetrato in casa sua, lo irritavano. 

- Be', hai finito? - disse, incontrando di sopra Stepan Arkad'ic. - Vuoi cenare? 

- Certo, non mi rifiuterò. Che appetito m'è venuto in campagna, un prodigio! Perché mai non hai offerto da mangiare a Rjabinin? 

- Che il diavolo se lo pigli! 

- Ma come lo maltratti! - disse Oblonskij. - Non gli hai dato neppure la mano. Perché non gli dai la mano? 

- Perché io non do la mano a un lacchè, e un lacchè è cento volte migliore di lui. 

- Ma che retrogrado che sei! E la fusione delle classi? - disse Oblonskij. 

- Buon pro' gli faccia a chi la vuole la fusione; a me non va. 

- Vedo che sei decisamente retrogrado. 

- Davvero non ho mai pensato a quel che sono. Io sono Konstantin Levin e nient'altro. 

- Un Konstantin Levin che è di pessimo umore! - disse, sorridendo, Stepan Arkad'ic. 

- Sì, sono di cattivo umore, e tu lo sai perché; per questa tua stupida, scusami, vendita. 

Stepan Arkad'ic corrugò bonariamente le sopracciglia come un uomo che viene offeso e mortificato senza ragione. 

- Su, basta - disse. - Quando mai s'è visto che un uomo vende qualcosa e non gli dicano subito, dopo la vendita: ``questo vale molto di più''? E finché la cosa è in vendita nessuno offre\ldots{} No, vedo che tu ce l'hai proprio con quel disgraziato di Rjabinin. 

- Può anche darsi. Ma sai perché? Tu dirai di nuovo che io sono un retrogrado o qualche altra strana cosa; tuttavia ti dico che mi spiace e mi offende assistere a questo impoverimento che d'ogni lato si va compiendo della nobiltà alla quale appartengo e alla quale, malgrado la fusione delle classi, sono molto lieto di appartenere\ldots{} E questo impoverimento non è oggetto di sperpero che sarebbe cosa di poco conto: lo sperpero da gran signore è affare da signori; solo i signori sanno sperperare così. Adesso i contadini, così come facciamo noi, si accaparrano le terre\ldots{} e nemmeno questo mi offende. Il signore non fa nulla, il contadino lavora e soppianta l'uomo ozioso. Così deve essere. Anzi sono molto contento per il contadino. Ma mi offende assistere a questo impoverimento dovuto a una certa tal quale, non so come chiamarla, dabbenaggine. Qua un affittuario polacco ha comprato a metà prezzo un magnifico podere della padrona che vive a Nizza, là danno in fitto a un mercante per un rublo una desjatina di terra che ne vale dieci. Qua tu, senza nessuna ragione al mondo, dài in regalo trentamila rubli a questo furfante. 

- E allora? Bisognava contare ogni albero? 

- Contare, sì, assolutamente. Ed ecco, tu non hai contato, ma Rjabinin ha contato. I figli di Rjabinin avranno i mezzi per vivere ed educarsi, e i tuoi, perdonami, non ne avranno. 

- Su, via, scusami, ma c'è qualcosa di meschino in questo contare. Noi abbiamo le nostre occupazioni, loro le loro, ed essi hanno bisogno di grossi profitti. Via, del resto l'affare è fatto, è concluso. Ma ecco le uova in tegame, le uova che più mi piacciono! E Agaf'ja Michajlovna ci darà quel suo meraviglioso sughetto di erbe\ldots{} 

Stepan Arkad'ic sedette a tavola e cominciò a scherzare con Agaf'ja Michajlovna, assicurando che un pranzo ed una cena simili da lungo tempo non li aveva mangiati. 

- Ecco, voi almeno mi fate un elogio - disse Agaf'ja Michajlovna - ma Konstantin Dmitric, qualunque cosa gli si dia, sia pure una crosta di pane, mangia e se ne va. 

Per quanto Levin cercasse di dominarsi, era cupo e silenzioso. Aveva gran voglia di fare una domanda a Stepan Arkad'ic, e non riusciva a decidersi e non trovava né il modo, né il momento di farla. 

Stepan Arkad'ic era già disceso in camera sua, s'era spogliato, lavato di nuovo, s'era messo in dosso una camicia da notte pieghettata, s'era coricato, e Levin era sempre con lui in camera, parlando di varie sciocchezze, senza avere il coraggio di chiedere quel che voleva sapere. 

- Ma in che modo meraviglioso fanno il sapone! - diceva, scartocciando e guardando un pezzo di sapone profumato che Agaf'ja Michajlovna aveva preparato per l'ospite, e che Oblonskij non aveva usato. - Guarda, è proprio un'opera d'arte. 

- Già, adesso si è raggiunto in tutto la perfezione - disse Stepan Arkad'ic, sbadigliando molle e beato. - I teatri per esempio e quei luoghi di divertimento\ldots{} ah ah ah! - sbadigliava. - La luce elettrica dovunque\ldots{} ah ah\ldots{} 

- Già, la luce elettrica - disse Levin. - Già\ldots{} e dov'è Vronskij adesso? - chiese dopo aver posato a un tratto il sapone. 

- Vronskij - disse Stepan Arkad'ic, arrestando uno sbadiglio - è a Pietroburgo. È andato via poco dopo di te, e poi non è venuto neppure una volta a Mosca. E sai, Kostja, ti dirò la verità - continuò, appoggiandosi col gomito sul tavolo e posando sulla mano il bel viso arrossato sul quale brillavano come stelle i languidi, buoni occhi assonnati. - La colpa è proprio tua. Tu hai avuto paura del rivale. E io, come ti avevo detto anche allora, so da parte di chi c'erano maggiori probabilità. Perché non ti sei fatto avanti? Io ti avevo detto allora che\ldots{} - sbadigliò con le sole mascelle senza aprire la bocca. 

``Lo sa o non lo sa che io ho fatto la mia domanda di matrimonio? - pensò Levin, guardandolo. - Già; c'è qualcosa di accorto, di diplomatico nel suo viso'' e, sentendo di arrossire, guardò in silenzio, diritto negli occhi, Stepan Arkad'ic. 

- Se allora da parte di lei c'era qualcosa, era un'attrazione per l'esteriorità - continuò Oblonskij. - Sai, quel perfetto aristocraticismo e la futura posizione nella società hanno agito, non su di lei, ma sulla madre. 

Levin si accigliò. L'affronto del rifiuto, attraverso il quale era passato, gli bruciava nel cuore come una ferita fresca, appena ricevuta. Ma era a casa sua e le mura della casa aiutano. 

- Aspetta, aspetta - prese a dire, interrompendo Oblonskij. - Tu dici: aristocraticismo. Ma permettimi di chiederti in che cosa consiste questo aristocraticismo di Vronskij, o di chiunque altro sia; questo tale aristocraticismo per il quale si possa disdegnare me. Tu consideri Vronskij un aristocratico, ma io no. Un uomo il cui padre è venuto su dal nulla con l'intrigo, la cui madre Dio sa con chi non ha avuto legami\ldots{} No, scusami, io considero aristocratico me stesso e le persone simili a me che possono vantare tre o quattro oneste generazioni di famiglie che hanno conseguito il più alto grado di cultura (il talento e l'ingegno sono un'altra cosa), che non si sono mai umiliate dinanzi a nessuno, che non hanno mai avuto bisogno di nessuno; così come hanno vissuto mio padre, mio nonno. E io ne conosco molti fatti così. A te sembra cosa meschina che io conti gli alberi nel bosco, mentre tu fai regalo di trentamila rubli a Rjabinin; ma tu riceverai un'indennità o non so cos'altro ancora, mentre io non la ricevo, e perciò mi tengo caro quello che m'han lasciato i miei e il frutto delle mie fatiche. Noi siamo i veri aristocratici, e non quelli che possono viver solo delle elargizioni dei potenti di questo mondo o quelli che si possono comprare con venti copeche. 

- Ma con chi te la prendi? Io sono d'accordo con te - disse Stepan Arkad'ic con sincerità e allegria, sebbene sentisse che Levin, accennando a quelli che si possono comprare con venti copeche, si riferisse anche a lui. L'eccitazione di Levin gli piaceva davvero. - Con chi ce l'hai? Benché gran parte di quel che dici di Vronskij non sia vero, io non mi riferivo a questo. Io ti dico francamente, al posto tuo\ldots{} insomma dovresti venire con me a Mosca e \ldots{} 

- No, io non so se tu sia al corrente o no, ma per me è lo stesso. E te lo dico subito: io ho fatto la mia domanda di matrimonio e ho ricevuto un rifiuto; e per me in questo momento, Katerina Aleksandrovna è un ricordo umiliante e molesto. 

- E perché? Ma guarda che sciocchezza! 

- Non ne parliamo più. Scusami, ti prego, se sono stato villano con te. - disse Levin. Ora, dopo aver messo fuori tutto, era diventato di nuovo quello che era stato la mattina. - Non sei mica in collera con me, Stiva? Ti prego, non ti arrabbiare - disse e, sorridendo, gli prese la mano. 

- Ma no, per nulla affatto, e non ce ne sarebbe ragione! Sono contento che ci siamo spiegati. E sai, la caccia del mattino di solito è fruttuosa. Non ci si potrebbe andare? Così non dormirei neppure, e dal posto di caccia andrei di filato alla stazione. 

- Benissimo! 

\capitolo{XVIII}\label{xviii-1} 

Sebbene la vita intima di Vronskij fosse tutta piena della sua passione, la sua vita esteriore si svolgeva immutata e immutabile sull'abituale carreggiata di prima, tra i rapporti e gli interessi del gran mondo e del reggimento. Gli interessi del reggimento occupavano un posto importante nella vita di Vronskij, e perché egli amava il reggimento e ancor più perché ne era amato. Al reggimento non solo gli volevano bene, ma lo stimavano ed erano orgogliosi di lui; erano orgogliosi che quest'uomo enormemente ricco, che possedeva una cultura ed aveva splendide attitudini, che aveva un avvenire aperto a ogni genere di successi, di ambizioni e di vanità, disdegnasse tutto questo e prendesse a cuore più di ogni altra cosa gli interessi del reggimento e dei compagni. Vronskij era cosciente di questa sua posizione presso i compagni, e oltre al fatto che amava quella vita, si sentiva impegnato a mantenere l'opinione che si aveva di lui. 

Naturalmente egli non parlava del suo amore con nessuno dei compagni; non si tradiva neppure nelle più grosse sbornie (del resto non era mai così ubriaco da perdere il controllo di se stesso), e tappava la bocca a quei compagni meno prudenti che tentavano di fare delle allusioni al suo legame. Malgrado ciò, il suo amore era noto a tutta la città: tutti indovinavano più o meno esattamente i suoi rapporti con la Karenina; la maggioranza dei giovani gli invidiava proprio quello che c'era di più penoso nel suo amore: l'alta posizione di Karenin, e perciò lo scalpore di quel legame nel gran mondo. 

La maggior parte delle giovani donne, gelose di Anna e che da tempo erano annoiate di sentirla definire ``irreprensibile'', si rallegravano di queste supposizioni sul suo conto, e aspettavano solo che l'opinione pubblica mutasse per piombarle addosso con tutto il peso del loro disprezzo. Preparavano già il fango da scagliare su di lei, non appena fosse giunto il momento. La maggior parte delle persone anziane di condizione sociale elevata erano spiacenti di questo scandalo che si preparava in società. 

La madre di Vronskij, conosciuta la relazione di lui, in principio ne era stata contenta perché nulla, secondo lei, dava l'ultima finitura a un giovane brillante quanto una relazione nel gran mondo, e perché la Karenina, che tanto le era piaciuta, che tanto aveva parlato del proprio figlio, aveva finito con l'essere così come, secondo la contessa Vronskaja, doveva essere ogni bella donna del gran mondo. Ma negli ultimi tempi aveva saputo che suo figlio aveva rifiutato l'offerta di un posto importante per la carriera, solo per voler rimanere al reggimento, ed aver così modo di vedere la Karenina; aveva saputo che i superiori erano scontenti di lui per questo rifiuto, e allora aveva cambiato idea. Le dispiaceva inoltre che la relazione, da quanto aveva saputo, non fosse la brillante, graziosa relazione mondana ch'ella aveva approvata, ma una certa passione alla Werther, disperata, come le riferivano, capace di trascinare lui a fare delle sciocchezze. Non vedeva il figlio dal tempo della sua partenza improvvisa da Mosca e, per mezzo del primogenito, aveva preteso che egli venisse da lei. 

Il fratello maggiore anch'egli era scontento del più giovane. Non capiva che specie di amore fosse quello: grande o piccolo, appassionato o non appassionato, vizioso o non vizioso (egli stesso, pur avendo dei figli, manteneva una ballerina ed era indulgente in tale materia), ma sapeva che quest'amore spiaceva a coloro ai quali era necessario piacere, perciò non approvava la condotta del fratello. 

Oltre le occupazioni del servizio e quelle mondane, Vronskij aveva un'altra occupazione: i cavalli, di cui era un appassionato. 

Proprio quell'anno erano state indette le corse a ostacoli per ufficiali. Vronskij vi si era iscritto, aveva comprato un purosangue inglese e, nonostante il suo amore, era tutto preso, anche se con riserbo, dalle corse imminenti. 

Queste due passioni non si contrastavano. Anzi egli aveva bisogno di trovare interesse e svago in qualcosa di diverso dal suo amore, in qualcosa in cui potersi rinnovare e riposare dalle impressioni che lo agitavano troppo. 

\capitolo{XIX}\label{xix-1} 

Il giorno delle corse di Krasnoe Selo, Vronskij andò prima del solito a mangiare una bistecca nella sala grande della mensa degli ufficiali. Non aveva bisogno di osservare una dieta rigorosa per mantenersi in forma; il suo peso era quello stabilito, di quattro pudy e mezzo; ma non doveva ingrassare, perciò evitava i farinacei e i dolciumi. Sedette col soprabito aperto sul panciotto bianco, appoggiando tutte e due le braccia sulla tavola, e, in attesa della bistecca ordinata, guardava in un romanzo francese poggiato sul piatto. Guardava nel libro solo per evitare di parlare con i colleghi che entravano ed uscivano, e pensava. 

Pensava che Anna gli aveva promesso un appuntamento per quel giorno, dopo le corse. Ma non la vedeva da tre giorni e, dopo il ritorno del marito dall'estero, non sapeva se ciò sarebbe stato possibile quel giorno o no, e non sapeva come fare per saperlo. S'era visto con lei l'ultima volta nella villa della cugina Betsy. Alla villa dei Karenin invece egli andava il meno possibile. Ora però voleva andarci e ne cercava il modo. 

``Dirò naturalmente che Betsy mi ha incaricato di chiederle se verrà o no alle corse. Certo che andrò'' decise fra sé, sollevando la testa dal libro. E, raffiguratasi la gioia nel vederla, si illuminò nel viso. 

- Manda a casa mia, perché attacchino al più presto la trojka - disse al cameriere che gli aveva servito la bistecca su un piatto d'argento caldo e, avvicinato il piatto, cominciò a mangiare. 

Dalla sala accanto, dei biliardi, si sentivano colpi di palle, vocìo e risa. Alla porta d'ingresso apparvero due ufficiali: uno giovanissimo con un viso delicato, magro, che da poco era entrato nel reggimento dal corpo dei paggi; l'altro grasso, anziano, con un braccialetto al polso e i piccoli occhi infossati. 

Vronskij li guardò, aggrottò le sopracciglia e, come se non li avesse notati, sbirciando il libro di traverso, prese a mangiare e a leggere insieme. 

- O che, ti metti in forza per il lavoro? - disse l'ufficiale grasso, sedendosi accanto a lui. 

- Lo vedi - rispose Vronskij, accigliandosi e asciugandosi le labbra senza guardarlo. 

- Ma non hai paura d'ingrassare? - disse quello, girando una sedia per l'ufficiale giovane. 

- Cosa? - disse Vronskij irritato, facendo una smorfia di disgusto che mostrò i suoi denti regolari. 

- Non hai paura d'ingrassare? 

- Cameriere, del Xeres - disse Vronskij senza rispondere e, posato il libro dall'altra parte, continuò a leggere. 

L'ufficiale grasso prese la carta dei vini e si voltò verso il giovane. 

- Scegli tu stesso quello che vuoi - disse, dandogli la carta e guardandolo. 

- Prego, del Reno - disse l'ufficiale giovane, guardando timido Vronskij e cercando di lisciare con le dita i baffi incipienti. Visto che Vronskij non gli dava retta, l'ufficiale giovane si alzò. 

- Andiamo al biliardo. 

Il grassone si alzò docile, e si diressero insieme verso la porta. 

In quel momento entrò nella sala il capitano Jašvin, alto e ben fatto, che, con un cenno sprezzante all'insù del capo verso i due ufficiali, s'accostò a Vronskij. 

- Ah, eccolo! - gridò, battendogli con forza sulla spallina con la mano grande. Vronskij si voltò a guardare con stizza, ma subito il viso gli si illuminò della cordialità calma e decisa che gli era propria. 

- Hai agito con intelligenza, Alëša - disse il capitano con una forte voce baritonale. - Adesso mangia e bevi un bicchierino. 

- Ma non ne ho voglia. 

- Ecco gl'inseparabili - aggiunse Jašvin, guardando con aria canzonatoria i due ufficiali che in quel momento uscivano dalla sala. E sedette accanto a Vronskij, piegando ad angolo acuto i suoi femori troppo lunghi per l'altezza delle sedie e le gambe strette nei pantaloni da cavallerizzo. - Come mai ieri sera non sei passato al teatro di Krasnoe Selo? La Numerova non andava mica male. Dove sei stato? 

- Ho fatto tardi dai Tverskij - rispose Vronskij. 

- Ah - ripeté Jašvin. 

Jašvin, giocatore, uomo sregolato, senza principi fuorché quelli immorali, era il miglior amico di Vronskij nel reggimento. Vronskij gli voleva bene e per la sua non comune forza fisica, della quale dava prova in particolar modo bevendo come un otre e facendo a meno di dormire pur rimanendo sempre presente a se stesso, e per la sua grande forza morale che dimostrava nei rapporti con i superiori e i compagni, suscitando timore e stima, e che nel giuoco (per il quale si impegnava per decine di migliaia di rubli e che sempre conduceva, malgrado il vino bevuto, con grande abilità e freddezza) lo faceva considerare il miglior giocatore del club inglese. Vronskij lo stimava e gli voleva bene soprattutto perché sentiva che Jašvin gliene voleva non per il suo nome o per la sua ricchezza, ma per la sua persona. E tra tutti gli uomini solo con lui Vronskij avrebbe voluto parlare del suo amore. Sentiva che solamente Jašvin, pur ostentando disprezzo verso qualsiasi sentimento, lui solo, così sembrava a Vronskij, poteva capire quella passione prepotente, che riempiva tutta la sua vita. Inoltre era sicuro che Jašvin non godeva del pettegolezzo e dello scandalo, ma intendeva quel sentimento così come andava inteso, e capiva e credeva cioè che quel suo amore non era un giuoco, né uno svago, ma qualcosa di molto più grave ed importante. 

Vronskij non aveva mai parlato con lui del suo amore, ma sentiva che egli sapeva tutto, che intendeva tutto così come andava inteso, e gli faceva piacere scorgere ciò dentro i suoi occhi. 

- Ah, sì - disse Jašvin, alludendo al fatto che Vronskij era stato dai Tverskij e, dopo aver lasciato sfuggire un guizzo dai suoi occhi neri, afferrò il baffo sinistro e cominciò a ficcarselo in bocca, secondo una cattiva abitudine. 

- Su, via, e tu ieri che hai fatto? Hai vinto? - chiese Vronskij. 

- Ottomila rubli. Ma tre non sono sicuri: difficile che li dia. 

- Su, così puoi anche perdere puntando su di me - disse Vronskij ridendo. (Jašvin aveva fatto una grossa scommessa su Vronskij). 

- Non perderò per nulla affatto. Solo Machotin è pericoloso. 

E la conversazione si aggirò sull'attesa delle corse di quel giorno alle quali solamente poteva pensare, ora, Vronskij. 

- Andiamo, ho finito - disse Vronskij e, alzatosi, si diresse verso la porta. Jašvin si alzò anche lui, allungando l'enormi gambe e la lunga schiena. 

- Per me è ancora presto per pranzare, ma non per bere. Vengo subito. Ehi, del vino! - gridò con la sua voce nota nel comando, tanto robusta da far tremare i vetri. - No, non occorre - gridò subito di nuovo. - Tu va' a casa, vengo anch'io con te. 

E andarono via insieme. 

\capitolo{XX}\label{xx-1} 

Vronskij alloggiava in un'izba finnica, spaziosa e pulita, divisa in due da un tramezzo. Petrickij viveva con lui. Petrickij dormiva quando Vronskij e Jašvin entrarono nell'izba. 

- Alzati, su, finiscila di dormire - disse Jašvin, entrando di là dal tramezzo e dando un colpo sulla spalla di quell'arruffone di Petrickij che s'era ficcato col naso nel guanciale. 

Petrickij saltò su a un tratto e si voltò a guardare. 

- È stato qui tuo fratello - disse a Vronskij. - Mi ha svegliato, che il diavolo se lo pigli; ha detto che verrà ancora. - E si gettò nuovamente sul guanciale, tirando su la coperta. - Ma smettila, Jašvin! - disse, arrabbiandosi con Jašvin che gli tirava via la coperta. - Basta! - Si girò e aprì gli occhi. - Di' piuttosto, cosa c'è da bere; ho una tale porcheria in bocca, che\ldots{} 

- Della vodka, è meglio di tutto - disse Jašvin con voce di basso. - Terešcenko! Vodka al signore e dei cetrioli - gridò, evidentemente compiaciuto d'ascoltare la propria voce. 

- Della vodka, pensi? Eh? - chiese Petrickij, facendo smorfie e fregandosi gli occhi. - E tu bevi? Beviamo insieme così. Vronskij, bevi anche tu? - disse Petrickij, alzandosi e avviluppandosi in una coperta tigrata. Uscì sulla porta del tramezzo, alzò le braccia e prese a cantare in francese: ``A Tule c'era un re''. - Vronskij, vuoi bere? 

- Fila via - disse Vronskij che metteva un soprabito tesogli dal servitore 

- Dove vai? - gli chiese Jašvin. - Ecco anche la trojka - aggiunse, dopo aver visto la vettura che si avvicinava. 

- Alla scuderia, e devo anche passare da Brjanskij per i cavalli - disse Vronskij. 

Vronskij aveva davvero promesso di andare da Brjanskij a dieci verste da Petergof, a portargli il denaro per i cavalli; voleva trovare il tempo di andare anche là. Ma i compagni capirono subito che non andava soltanto là. 

Petrickij, continuando a canterellare, ammiccò con un occhio e gonfiò le labbra come per dire: ``Lo sappiamo che Brjanskij è mai questo''. 

Jašvin disse soltanto: 

- Bada a non far tardi - e, per cambiar discorso: - Di', su, che forse il mio lupacchiotto fa il suo servizio tuttora? - chiese, guardando dalla finestra, a proposito di un cavallo da tiro che gli aveva venduto. 

- Fermati - disse Petrickij a Vronskij che stava già per uscire. - Tuo fratello ha lasciato per te una lettera e un biglietto. Aspetta un po', dove sono? 

Vronskij si fermò. 

- Su, dove sono? 

- Dove sono? Ecco, qui sta la questione! - disse solennemente Petrickij, facendo passare sul naso il dito indice. 

- Su parla ancora, non fare lo stupido - disse Vronskij, sorridendo. 

- Non ci ho mica acceso il camino. Devono essere qui in qualche parte. 

- Su, basta, amico. Dov'è la lettera? 

- No, davvero non ricordo. O che forse l'ho visto in sogno? Aspetta, aspetta! Ma perché arrabbiarsi? Se tu avessi bevuto quattro bottiglie, come me ieri, alla salute di tuo fratello, anche tu avresti dimenticato dove eri steso\ldots{} Aspetta, me lo ricordo subito! 

Petrickij andò di là dal tramezzo e si sdraiò sul letto. 

- Fermati! Ero sdraiato così io, così in piedi stava lui. Sì, sì, sì\ldots{} Eccola! - E Petrickij tirò fuori di sotto al materasso la lettera che aveva nascosta. 

Vronskij prese la lettera e il biglietto. Era proprio quel che si aspettava: una lettera della madre coi rimproveri perché non andava da lei e un biglietto del fratello che gli diceva di dovergli parlare. Vronskij sapeva che era sempre la stessa cosa. ``Che gliene importa a loro!'' pensò e, spiegazzate le lettere, se le ficcò tra i bottoni del soprabito per leggerle con calma per via. Nell'ingresso dell'izba incontrò due ufficiali, uno del proprio e l'altro di un altro reggimento. 

L'abitazione di Vronskij era sempre il ritrovo di tutti gli ufficiali. 

- Dove vai? 

- Devo andare a Petergof. 

- E il cavallo è venuto da Carskoe? 

- È arrivato, ma non l'ho visto ancora. 

- Dicono che Gladiator di Machotin si sia azzoppato. 

- Sciocchezze! Ma, come farete a saltare su questo fango? - disse l'altro. 

- Ecco i miei salvatori! - gridò Petrickij, vedendo quelli che erano entrati, mentre l'attendente gli stava davanti con la vodka e i cetrioli salati sopra un vassoio. - Ecco, è Jašvin che mi ordina di bere per rinfrescarmi. 

- Su, stanotte l'avete fatta bella - disse uno di quelli che erano entrati - tutta la notte non ci avete fatto dormire. 

- Già, ma sapete come è andata a finire? - raccontava Petrickij. - Volkov s'è arrampicato sul tetto e s'è messo a dire che si sentiva triste. Io dico: attacca la musica, una marcia funebre! E lui s'è addormentato proprio così sul tetto, al suono della marcia funebre! 

- Bevi, bevi assolutamente la vodka, e dopo l'acqua di seltz e molto limone - disse Jašvin, curvandosi sopra Petrickij come una madre che obblighi un bimbo a prendere la medicina - e dopo, anche un po' di champagne, così, una bottiglietta. 

- Ecco, questa è una cosa intelligente. Aspetta, Vronskij, beviamo. 

- No, addio, signori miei, adesso non bevo. 

- E che mai, diventi uggioso? Su, allora, da solo. Dammi dell'acqua di seltz e il limone. 

- Vronskij! - gridò qualcuno mentre egli usciva già nell'ingresso. 

- Che c'è? 

- Dovresti tagliarti i capelli, se no ti pesano, specie sulla zucca. 

Vronskij infatti cominciava a diventar calvo prima del tempo. Egli rise allegramente, mostrando i bei denti allineati e abbassando il berretto sulla calvizie; uscì e montò in carrozza. 

- Alla scuderia! - disse, e voleva tirar fuori le lettere per finire di leggerle, ma poi cambiò idea, per non distrarsi prima della visita al cavallo. ``Dopo!\ldots{}''. 

\capitolo{XXI}\label{xxi-1} 

La scuderia provvisoria era una baracca di assi costruita proprio accanto all'ippodromo, e là doveva essere stato condotto il giorno prima il suo cavallo. Non l'aveva ancora visto. In quegli ultimi giorni non l'aveva montato neppure per una breve passeggiata, ma l'aveva affidato all'allenatore, e ora non sapeva proprio in quale condizioni fosse giunto e si trovasse. Appena scese dalla carrozza, il garzone di scuderia (il groom, come era chiamato il ragazzo), che ne aveva già riconosciuto da lontano la vettura, chiamò l'allenatore. L'inglese magro, con gli stivali alti, la giacchetta corta e un ciuffo di peli lasciati crescere solo sotto il mento, gli venne incontro con il passo ondeggiante dei fantini, allargando i gomiti e dondolandosi. 

- Ebbene, che ne è di Frou-Frou? - disse Vronskij in inglese. 

- All right, sir\ldots{} tutto va bene, signore - pronunciò, chi sa in quale parte della gola, la voce dell'inglese. - È meglio che non andiate - aggiunse, togliendosi il berretto. - Le ho messo la musoliera e la cavalla è inquieta. È meglio che non andiate. 

- No, voglio entrare. Voglio dare un'occhiata. 

- Andiamo - disse l'inglese, sempre senza muover le labbra, e, tutto accigliato e dondolando i gomiti, andò avanti col suo passo dinoccolato. 

Entrarono nel piccolo cortile che era dinanzi alla baracca. Il mozzo di stalla, snello e robusto, con la giacchetta pulita, e una scopa in mano, si fece incontro a quelli che entravano, e li seguì. Nella baracca c'erano i cavalli nei loro recinti, e Vronskij sapeva che quel giorno doveva certo esservi stato condotto e trovarsi là il suo più grande antagonista, il sauro di Machotin, Gladiator, venti centimetri più lungo dell'ordinario. Più che il cavallo suo Vronskij avrebbe voluto vedere Gladiator che non aveva mai visto; ma sapeva che, secondo il codice dell'ippica, non era corretto volerlo vedere, e neppure chiederne. Mentre camminava per il corridoio, il mozzo aprì la porta del secondo recinto a sinistra e Vronskij intravide un sauro grosso dalle zampe bianche. Capì che era Gladiator, ma con l'istinto dell'uomo che distoglie lo sguardo da una lettera aperta che appartenga ad altri, distolse lo sguardo e si accostò al recinto di Frou-Frou. 

- Qui è il cavallo di Mak\ldots{} Mak\ldots{} non riesco mai a pronunciare questo nome - disse l'inglese al di sopra della spalla, indicando, col dito pollice dall'unghia sporca, il recinto di Gladiator. 

- Di Machotin? Già, questo è il mio temibile antagonista - disse Vronskij. 

- Se montaste voi questo - disse l'inglese - scommetterei su di voi. 

- Frou-Frou è più nervosa, questo qui è più forte - disse Vronskij, sorridendo per la lode alla sua abilità di cavallerizzo. 

- Nella corsa ad ostacoli tutto sta nell'arte di inforcare e nel pluck - disse l'inglese. 

Di pluck, di energia e di coraggio, cioè, Vronskij sentiva di averne ad usura, ma, quel che più contava, era fermamente convinto che nessuno al mondo di questo pluck potesse averne più di lui. 

- Siete sicuro che non occorra per la cavalla una grande sudata? 

- Non occorre - rispose l'inglese. - Per favore non parlate forte. Il cavallo si agita - aggiunse accennando col capo verso lo stallo davanti al quale stavano fermi e dal quale si udiva lo scalpitare della bestia sulla paglia. 

Aprì la porta e Vronskij entrò in un recinto debolmente illuminato da una grata. Nello stallo calpestava la paglia fresca una giumenta baia dal manto scuro con la musoliera. Abituato l'occhio alla penombra, Vronskij avvolse di nuovo con un solo sguardo le forme della sua cavalla favorita. Frou-Frou era una cavalla di media altezza, non senza difetti; era tutta stretta di ossatura ed anche il petto molto sporgente era stretto. Aveva la groppa un po' cascante e le zampe anteriori e soprattutto le posteriori alquanto ritorte. I muscoli delle posteriori e quelli delle anteriori non erano particolarmente grossi; ma in compenso, nel sottopancia, la cavalla era eccezionalmente larga, cosa che balzava agli occhi ora che, tenuta a regime, aveva la pancia incavata. Le ossa delle zampe, al di sotto dei ginocchi, viste di fronte, sembravano non più grosse di un dito, ma, in compenso, erano straordinariamente larghe viste di lato. Tutta la bestia, eccettuate le costole, era come se l'avessero schiacciata nei fianchi e tirata in lungo. Ma possedeva al massimo una qualità che faceva dimenticare tutti i suoi difetti; aveva il sangue, sangue ``che si fa sentire'', come dicono gli inglesi. I muscoli fortemente rilevati al di sotto della rete delle vene, distesi sotto la pelle sottile, mobile e liscia come raso, sembravano duri come ossa. La testa asciutta, con gli occhi in rilievo, luminosi e vivi, si allargava verso le froge prominenti dalle membrane iniettate di sangue all'interno. In tutta la linea della cavalla, e in particolare nella testa, c'era qualcosa di volitivo e nello stesso tempo di dolce. Era una di quelle bestie che sembra non parlino solo perché la conformazione della loro bocca non lo permette. 

A Vronskij, almeno, sembrò che essa capisse tutto quello che egli provava, ora, nel guardarla. Non appena Vronskij era entrato, essa aveva aspirato profondamente l'aria e, torcendo l'occhio sporgente tanto da iniettar di sangue la cornea, aveva guardato dalla parte opposta a quella da cui erano entrati, scotendo la musoliera e appoggiandosi elastica ora su una zampa ora su di un'altra. 

-Ecco, vedete come è agitata - disse l'inglese. 

- Oh, cara! - diceva Vronskij, accostandosi alla cavalla ed esortandola. 

Ma quanto più si accostava, tanto più essa si agitava. Solo quando egli si accostò alla testa, si calmò a un tratto, mentre i muscoli trasalivano sotto il pelame sottile, delicato. Vronskij le carezzò il collo forte, aggiustò sul garrese erto un ciuffo della criniera caduto dall'altra parte, ed accostò il viso alle narici dilatate e sottili come ala di pipistrello. Essa aspirò ed emise sonoramente l'aria dalle narici tese; rabbrividendo, drizzò l'occhio aguzzo e protese il labbro forte e nero verso Vronskij come se volesse afferrarlo per la manica. Ma, ricordatasi della musoliera, la scosse e cominciò di nuovo a far cambiare di posto, una dopo l'altra, le sue zampe tornite. 

- Calma, cara, calma - diceva Vronskij carezzandola ancora con la mano su per la groppa e, con la gioiosa convinzione che la cavalla fosse in forma perfetta, uscì dallo stallo. 

L'agitazione della cavalla si era comunicata a Vronskij; sentiva che il sangue gli affluiva al cuore e che anche lui, come l'animale, aveva voglia di muoversi, di mordere; c'erano in lui orgasmo e gioia. 

- Su, allora conto su di voi - disse all'inglese - alle sei e mezzo sul posto. 

- Tutto è in ordine - disse l'inglese. - E voi dove andate, milord? - chiese inaspettatamente, adoperando questa denominazione di my-Lord che non usava quasi mai. 

Vronskij sollevò il capo, sorpreso, e guardò così come sapeva guardare lui, non negli occhi, ma in fronte, l'inglese, meravigliandosi della temerarietà della sua domanda. Ma, capito che l'inglese, nel far la domanda, non l'aveva rivolta a lui come signore, ma come cavallerizzo, rispose: 

- Devo andare da Brjanskij, fra un'ora sarò di ritorno. 

``Quante volte mi fanno questa domanda, oggi!'' si disse, e arrossì, cosa che gli capitava di rado. L'inglese lo guardò attento, e, come se avesse saputo dove Vronskij andava, aggiunse: 

- Ciò che più conta è l'essere calmi prima di montare - disse. - Non vi contrariate per nessuna cosa, ed evitate ogni emozione. 

- All right - rispose sorridendo Vronskij e, saltato nella vettura, ordinò di andare a Petergof. 

Allontanatosi appena di qualche passo, una nuvola, che dalla mattina aveva minacciato la pioggia, si addensò e venne giù un acquazzone. 

``Male - pensò Vronskij, alzando il mantice della vettura. - C'era già fango, e ora sarà proprio un pantano''. Trovandosi solo nella vettura, tirò fuori la lettera della madre e il biglietto del fratello e ne terminò la lettura. 

Sì, sempre la stessa cosa. Tutti, sua madre, suo fratello, tutti ritenevano necessario immischiarsi nei suoi affari di cuore. Questa ingerenza suscitava in lui un rancore, sentimento che di rado provava. ``Cosa importa loro? Perché ognuno ritiene di doversi occupare di me? Ma perché non mi lasciano in pace? Perché è una cosa che non possono capire. Se fosse la solita volgare relazione mondana, mi avrebbero lasciato in pace. Sentono invece che è qualcosa di diverso, che non è uno svago, e che questa donna mi è più cara della vita. E proprio questo è incomprensibile per loro, e perciò inquietante. Qualunque sia e sarà il nostro destino, noi ce lo siamo fatto e non ce ne lamentiamo - diceva unendo nella parola `noi' se stesso e Anna. - E invece no, sentono il bisogno d'insegnarci a vivere. Non hanno neppure un'idea di che cosa sia la felicità, non sanno che senza questo amore per noi non c'è felicità, né infelicità, non c'è vita'' pensava. 

S'irritava contro tutti per questa intrusione nei fatti suoi, proprio perché sentiva in fondo all'anima che loro, tutti gli altri, avevano ragione. Sentiva che l'amore che lo legava ad Anna non era una distrazione momentanea che passa come passano tutte le relazioni mondane senza lasciare altra traccia nella vita dell'uno e dell'altra che un ricordo grato o increscioso. Sentiva tutto il tormento della posizione sua e di quella di lei, l'imbarazzo creato dalla necessità, esposti com'erano agli occhi del mondo, di dover nascondere il loro amore, e di mentire e di ingannare, di dover usare mille astuzie e doversi preoccupare continuamente degli altri, mentre la loro passione era così grande che per entrambi null'altro v'era al di fuori del loro amore. 

Ricordava con chiarezza le circostanze così frequenti nelle quali era stato necessario usare l'inganno e la falsità, così avversi alla natura sua; ricordava in modo particolarmente vivo il senso di vergogna più di una volta notato in lei per la necessità di mentire e di ingannare. Dal tempo della sua relazione con Anna egli provava una sensazione strana, che lo afferrava ogni tanto. Era come un senso di nausea per qualche cosa: per Aleksej Aleksandrovic, per se stesso o per il mondo intero, non sapeva bene. Ma allontanava sempre questa sensazione strana. E anche ora, dopo essersene liberato, continuava il corso dei suoi pensieri. 

``Già, lei prima era infelice, ma orgogliosa e tranquilla; ora, invece, non può essere orgogliosa e tranquilla, pur fingendo di esserlo. Sì, tutto questo deve finire'' decise da ultimo. 

Così per la prima volta gli apparve chiaro nella mente il pensiero che fosse indispensabile porre termine a quella menzogna, e che quanto prima ciò sarebbe accaduto tanto meglio sarebbe stato. 

``Lei ed io dobbiamo abbandonare tutto e andarci a nascondere in qualche luogo, noi due, con il nostro amore'' disse a se stesso. 

\capitolo{XXII}\label{xxii-1} 

L'acquazzone non durò a lungo, e mentre Vronskij si avvicinava a gran trotto col cavallo di centro che tirava e i due di lato che galoppavano liberi, senza redini, nel fango, il sole era già ricomparso, e i tetti delle ville e i vecchi tigli dei giardini, dall'una e dall'altra parte della strada maestra, scintillavano di un luccichio umido, mentre l'acqua gocciolava allegramente dai rami e grondava giù dai tetti. Non pensava ormai più che quell'acquazzone avrebbe potuto guastare l'ippodromo, si rallegrava invece che, grazie alla pioggia, avrebbe trovato certamente lei in casa e sola, poiché aveva saputo che Aleksej Aleksandrovic, rientrato da poco dalla cura delle acque, era rimasto a Pietroburgo. 

Con la speranza di trovarla sola, Vronskij, come del resto faceva sempre per non essere notato, smontò prima di arrivare al ponte, e andò a piedi. Evitò l'ingresso che dava sulla strada ed entrò per il cortile. 

- Il signore è arrivato? - chiese al giardiniere. 

- Nossignore. La signora è in casa. Ma vi prego, passate per la scala; là c'è gente, vi apriranno - rispose il giardiniere. 

- No, passerò dal giardino. 

Sicuro ormai di trovarla sola, e desideroso di coglierla di sorpresa (non aveva promesso di andare quel giorno, e probabilmente ella non sospettava di vederlo là prima delle corse), proseguì, trattenendo la sciabola e camminando cauto sulla ghiaia del viottolo fiancheggiato da fiori, verso la terrazza che dava sul giardino. Vronskij in quel momento non pensava più alla gravità ed alla difficoltà della situazione: pensava solo che l'avrebbe veduta, non già immagine, ma viva, tutta, così com'era nella realtà. Stava per entrare, poggiando per intero il piede per non far rumore, sugli scalini inclinati della terrazza, quando gli balenò nella mente il ricordo di quel che rappresentava il lato più tormentoso della sua relazione con Anna: il ricordo del figlio di lei, con quel suo sguardo indagatore, che gli sembrava ostile. 

Questo ragazzo, più di chiunque altro, rappresentava un intralcio alla loro relazione. Quando egli era presente, sia Vronskij che Anna non solo non si permettevano di parlare se non di cose da potersi dire dinanzi a tutti, ma non si concedevano neppure di fare allusioni a cose che il ragazzo non avrebbe potuto capire. E ciò non per averne parlato insieme, ma spontaneamente si era prodotto da sé. Sentivano come un'offesa a se stessi ingannare quel fanciullo. In sua presenza parlavano tra di loro come semplici conoscenti. Malgrado quest'accortezza, Vronskij scorgeva spesso fisso su di sé lo sguardo attento e perplesso del bambino e una strana timidezza, una discontinuità di atteggiamento, ora tenero, ora ritroso e riservato, nel modo di comportarsi del ragazzo nei suoi riguardi. Come se il ragazzo sentisse che tra sua madre e lui c'era un rapporto importante del quale non poteva penetrare la natura. 

Infatti il fanciullo sentiva di non poter intendere, per quanto ci si sforzasse, quel rapporto, e non sapeva rendersi conto di ciò che sentiva verso quell'uomo. Con la particolare sensibilità dei bambini, vedeva chiaramente che il padre, la governante, la njanja, tutti, non solo non amavano, ma pur senza parlarne, guardavano con avversione e timore Vronskij, che la madre invece considerava come il suo migliore amico. 

``Che cosa vuol dire questo? Chi è quell'uomo? Come debbo volergli bene? Se non lo capisco, la colpa è mia che sono un ragazzo sciocco e cattivo'' pensava il bambino; e da ciò derivavano la sua espressione indagatrice e quasi ostile, e quella discontinuità che tanto turbava Vronskij. La presenza di questo bambino suscitava sempre in Vronskij lo strano senso di nausea irragionevole che egli provava in quegli ultimi tempi. La presenza del bambino suscitava in Vronskij e in Anna una sensazione simile a quella del navigante che veda dalla bussola che la direzione, nella quale si muove rapido, si allontana da quella dovuta, ma che arrestare il moto non è più nelle sue forze, che ogni attimo lo allontana sempre più dalla giusta direzione e che confessare a se stesso la deviazione è lo stesso che confessare la propria rovina. 

Il bambino con la sua innocente visione della vita era la bussola che mostrava loro il grado di deviazione dalla rotta che conoscevano, ma che volevano ignorare. 

Questa volta Serëza non era in casa, e Anna, completamente sola, stava seduta sulla terrazza ad aspettare il ritorno del figlio uscito a spasso e sorpreso dalla pioggia. Aveva mandato un domestico e una cameriera a cercarlo e stava lì ad attenderlo. Vestiva un abito bianco con un largo ricamo; sedeva in un angolo della terrazza di là dai fiori e non aveva avvertito l'avvicinarsi di lui. Abbassata la testa nera e inanellata, premeva la fronte contro un freddo annaffiatoio che era sulla ringhiera, trattenendolo con entrambe le mani bellissime dagli anelli a lui noti. La bellezza di tutta la sua figura, della testa, del collo, delle braccia, colpiva ogni volta Vronskij come una cosa inattesa. Si fermò a guardarla incantato. Ma appena volle fare un passo per avvicinarsi, ella sentì subito l'appressarsi di lui, scostò l'annaffiatoio e voltò verso di lui il viso infiammato. 

- Che vi è accaduto? Non state bene? - disse egli in francese, avvicinandosi. Avrebbe voluto correre a lei, ma ricordando che potevano esserci estranei, si voltò a guardare verso la porta della terrazza e arrossì come arrossiva ogni volta che doveva temere ed essere guardingo. 

- No, sto bene - disse lei, alzandosi e stringendo con forza la mano ch'egli le tendeva. - Non ti aspettavo\ldots{} 

- Dio mio che mani fredde! - egli disse. 

- Mi hai spaventata. Sono sola e aspettavo Serëza che è andato a spasso: verranno di qui 

Ma, pur sforzandosi d'essere calma, le labbra le tremavano. 

- Perdonatemi se son venuto, ma non potevo far passare un altro giorno senza vedervi - continuò in quel francese che usava sempre per evitare il voi russo freddo fino all'impossibile fra di loro e il tu troppo pericoloso. 

- E perché perdonare? Sono così felice! 

- Ma voi non state bene, o siete rattristata - continuò senza lasciarle la mano e chinandosi su di lei. - A che pensavate? 

- Sempre alla stessa cosa - disse lei con un sorriso. Diceva la verità. Ogni volta, in qualunque momento le si fosse chiesto a cosa pensasse, poteva rispondere senza errore: a una cosa sola, alla sua felicità e alla sua infelicità. Pensava proprio questo nel momento in cui egli l'aveva sorpresa: pensava perché per gli altri, per Betsy, ad esempio (ella sapeva la sua relazione, tenuta segreta per il mondo, con Tuškevic), era facile ciò che per lei era tanto tormentoso. Quel giorno, questo pensiero, per varie ragioni, la tormentava in modo particolare. Gli domandò delle corse. Egli, vedendola agitata, prese a raccontare, per distrarla, con tono semplice, i particolari dei preparativi delle corse. 

``Dirlo o non dirlo? - pensava intanto lei, guardando negli occhi calmi e carezzevoli di lui. - È così felice, così preso dalle sue corse, che non gli darà il peso che si deve, non capirà tutta l'importanza per noi di questo avvenimento''. 

- Ma voi non avete detto a cosa pensavate quando sono entrato - egli disse, interrompendo il racconto - vi prego, ditemelo! 

Ella non rispondeva e, chinato un poco il capo, lo guardò interrogativamente di sotto in su con i suoi occhi luminosi, dietro le lunghe ciglia. La mano che giocava con una foglia strappata, tremò. Egli notò questo e il suo viso espresse quella sottomissione, quella dedizione da schiavo che tanto la seduceva. 

- Vedo che è accaduto qualcosa. Posso mai esser tranquillo, sapendo che avete una pena che io non divido? Parlate, per amor di Dio - ripeté supplichevole. 

``Non gli perdonerei se non capisse tutto il significato della cosa. Meglio non dirglielo: perché metterlo alla prova?'' pensava lei, continuando a guardarlo e sentendo che la mano che tratteneva la foglia tremava sempre di più. 

- Per amor di Dio - ripeté lui, prendendole la mano. 

- Devo dirlo? 

- Sì, sì, sì\ldots{} 

- Sono incinta - disse lei a voce bassa. La foglia tremò ancora di più nella mano, ma ella non distolse gli occhi da lui per scorgere come accogliesse la notizia. Egli impallidì, volle dire qualcosa, ma si fermò, lasciò cadere la mano di lei e chinò il capo. 

``Sì, ha capito tutta l'importanza di questo avvenimento'' ella pensò e gli strinse la mano con gratitudine. 

Ma si era sbagliata nel credere ch'egli intendesse il significato della notizia così come lei, donna, l'intendeva. A quella notizia egli aveva sentito dieci volte più intenso un attacco di quella strana sensazione che l'afferrava come una nausea di qualcosa; ma insieme a questo egli aveva sentito che la crisi desiderata era ormai giunta, che non si poteva più nascondere la cosa al marito e che era indispensabile rompere in un modo o nell'altro quella situazione equivoca. Oltre a ciò, l'agitazione di lei gli si era comunicata fisicamente. La guardò con uno sguardo intenerito, sottomesso, le baciò la mano, si alzò e si mise a camminare in silenzio per la terrazza. 

- Sì - disse poi, avvicinandosi a lei con decisione. - Né io né voi abbiamo considerato i nostri rapporti come un giuoco, e ora la nostra sorte è decisa. È indispensabile porre termine alla menzogna in cui viviamo - disse, guardandosi in giro. 

- Porre termine? E come, Aleksej? - ella disse piano. 

Era calma adesso, e il suo viso splendeva d'un sorriso tenero. 

- Lasciare vostro marito e unire la nostra vita. 

- È unita anche così - ella rispose in modo appena percettibile. 

- Sì, ma del tutto, del tutto. 

- Ma come, Aleksej, dimmi come? - disse con triste irrisione verso il suo caso senza via d'uscita. - Vi è forse una via d'uscita da una posizione come la nostra? Non sono forse la moglie di mio marito? 

- Da qualsiasi situazione c'è una via d'uscita. Bisogna decidersi - egli disse. - Qualunque cosa è migliore della posizione in cui vivi. Perché io vedo come ti tormenti per tutto, e per il mondo, e per tuo figlio e per tuo marito. 

- Ah, per mio marito, no - ella disse con un riso schietto. - Non so, non penso a lui, non esiste. 

- Tu non parli con sincerità. Ti conosco. Ti tormenti anche per lui. 

- Ma egli non lo sa neppure - ella disse e, a un tratto, un rossore vivo cominciò a salirle al viso; le guance, la fronte, il collo si arrossarono, e lacrime di vergogna le salirono agli occhi. - Ma non parliamo di lui. 

\capitolo{XXIII}\label{xxiii-1} 

Vronskij già altra volta aveva tentato, anche se non in maniera così decisa come ora, di indurla a esaminare la situazione, e ogni volta s'era imbattuto in quella superficialità e leggerezza di giudizio con la quale ella ora rispondeva al suo invito. Pareva esserci qualcosa ch'ella non potesse o non volesse chiarire a se stessa, pareva che non appena se ne cominciava a parlare, lei, la vera Anna, se ne andasse chi sa in qual parte di sé, e venisse fuori un'altra donna strana, estranea a lui, ch'egli non amava, anzi temeva, e che gli opponeva resistenza. Ma egli ora decise di chiarire tutto. 

- Ch'egli lo sappia o no - disse Vronskij col suo solito tono fermo e calmo - lo sappia o no, a noi questo non importa. Noi non possiamo, voi non potete continuare a stare così, specialmente ora. 

- E che fare, secondo voi? - chiese lei con la stessa sottile irrisione. A lei, che aveva tanto temuto ch'egli prendesse alla leggera la sua gravidanza, spiaceva, ora, che da questa egli traesse la necessità d'intraprendere qualcosa. 

- Rivelargli tutto, e lasciarlo. 

- Molto bene, ammettiamo che io lo faccia - ella disse. - Sapete cosa ne verrà fuori? Ve lo dico io fin d'ora - e una luce cattiva s'accese nei suoi occhi, un momento prima teneri. - ``Ah, voi amate un altro e avete contratto un legame peccaminoso con lui?'' - e, rifacendo il marito, metteva esattamente l'accento sulla parola ``peccaminoso'', così come faceva Aleksej Aleksandrovic. - ``Vi ho avvertito delle conseguenze dal punto di vista religioso, civile e familiare. Voi non m'avete dato ascolto. Adesso io non posso esporre al disonore il mio nome\ldots{}'' - ``e mio figlio'' ella avrebbe voluto dire, ma sul figlio non si poteva scherzare\ldots{} - ``al disonore il mio nome'' e ancora qualcosa del genere - aggiunse. - Si terrà sulle generali, parlerà col suo tono di uomo di stato e con chiarezza e precisione, dirà che non può lasciarmi andare, ma che prenderà le misure che dipendono da lui per arrestare lo scandalo. E farà tranquillamente, accuratamente tutto quello che avrà detto. Ecco quello che accadrà. Non è un uomo, ma una macchina, e una macchina cattiva, quando si arrabbia - ella aggiunse, ricordandosi intanto di Aleksej Aleksandrovic in tutti i particolari della sua figura, del suo modo di parlare e del suo carattere, e facendogli colpa di tutto quello che ella poteva trovare di cattivo in lui, senza perdonargli nulla, proprio per quella terribile colpa che essa aveva verso di lui. 

- Ma, Anna - disse Vronskij con voce suadente, dolce, cercando di calmarla - ma è indispensabile dirglielo e poi regolarsi secondo quello ch'egli farà. 

- E allora, fuggire? 

- E perché anche non fuggire? Non vedo la possibilità di continuare così\ldots{} E non per me\ldots{} vedo che voi ne soffrite. 

- Già, fuggire e diventare la vostra amante? - disse Anna con cattiveria. 

- Anna! - egli disse con rimprovero e tenerezza. 

- Già - continuò lei - diventare la vostra amante e perdere tutto\ldots{} 

Ella voleva di nuovo dire: ``mio figlio'', ma non poté pronunciarla, questa parola. 

Vronskij non riusciva a capire come lei, con la sua forte natura onesta, potesse sopportare quella situazione d'inganno e non desiderasse uscirne; ma egli non ne indovinava la ragione principale, che cioè era la parola ``figlio'' ch'ella non poteva pronunciare. Quando pensava al figlio e ai suoi futuri rapporti con la madre che avesse abbandonato il padre suo, era presa da un tale terrore di quello che aveva fatto da non ragionare più e, come ogni donna, si sforzava solo di calmarsi con ragionamenti mendaci e parole vane, desiderando che tutto rimanesse così com'era e che si potesse dimenticare la terribile questione: che cosa ne sarebbe stato del figlio. 

- Ti prego, ti supplico - diss'ella a un tratto con tono del tutto diverso, sincero e tenero, prendendolo per mano - non parlarmene mai più. 

- Ma Anna\ldots{} 

- Mai. Lascia a me tutto questo. Conosco tutta la bassezza, tutto l'orrore della mia posizione, ma la cosa non è così facile a decidersi come tu credi. E lascia fare a me e ascoltami. Non parlar mai più con me di questo. Me lo prometti? No, no, prometti! 

- Io prometto tutto, ma non posso esser tranquillo, specialmente dopo quello che mi hai detto. Non posso esser tranquillo quando non puoi esserlo tu\ldots{} 

- Io - ella ripeté - sì, io mi tormento a volte, ma passerà, se tu non parlerai mai più con me di questo. Quando tu ne parli, allora soltanto me ne tormento. 

- Non capisco - disse lui. 

- Io so - ella lo interruppe - quanto sia penoso per la tua natura onesta mentire, e mi fai pena. Spesso penso che tu, per me, hai rovinato la tua vita. 

- Anch'io, or ora, pensavo la stessa cosa - egli disse - come hai potuto sacrificare tutto a me? Io non posso perdonarmi d'averti resa infelice. 

- Io infelice? - ella disse, accostandosi a lui e guardandolo con un entusiastico riso d'amore - io sono come un essere affamato al quale abbiano dato da mangiare. Avrà forse freddo, e avrà il vestito lacero; avrà vergogna, forse, ma non è infelice. Io infelice? No, eccola, la mia felicità\ldots{} 

Aveva sentito la voce del figlio che era tornato e, data una rapida occhiata alla terrazza, si alzò di scatto. Il suo sguardo si accese del fuoco a lui ben noto, sollevò con un movimento rapido le belle mani coperte d'anelli, gli afferrò il capo, lo guardò con un lungo sguardo, e, avvicinando a lui il viso con le labbra aperte, ridenti, gli baciò in fretta la bocca e gli occhi, e lo respinse. Voleva andar via, ma egli la trattenne. 

- A quando? - bisbigliò in un sussurro, guardandola rapito. 

- Stanotte, all'una - ella mormorò e, con un sospiro profondo, andò incontro al figlio col suo passo svelto e leggero. 

La pioggia aveva sorpreso Serëza nel giardino grande, e lui e la njanja erano rimasti a sedere sotto una pergola. 

- Ebbene, arrivederci - diss'ella a Vronskij. - È necessario affrettarsi per le corse. Betsy ha promesso di passare a prendermi. 

Vronskij, guardato l'orologio, se ne andò in fretta. 

\capitolo{XXIV}\label{xxiv-1} 

Nel momento in cui Vronskij aveva guardato l'orologio sulla balconata dei Karenin era così turbato e preoccupato che aveva visto, sì, le lancette del quadrante, ma non aveva potuto capire che ora fosse. Uscì in strada e si diresse, camminando cauto nel fango, verso la vettura. Era dominato dal sentimento suo verso Anna, così che non pensava neppure più che ora fosse e se gli restasse il tempo per andare da Brjanskij. Gli rimaneva ora, come spesso accade, solo una certa memoria istintiva, quella che serve a indicare in quale ordine si è stabilito di fare le cose. Si accostò al cocchiere che s'era messo a dormire stando a cassetta, all'ombra già obliqua di un tiglio folto, e fu attratto per un attimo dai nugoli cangianti dei moscerini che volteggiavano sui cavalli sudati. Svegliato il cocchiere, saltò in vettura e ordinò di andare a Brjanskij. Solo dopo essersi allontanato di sette verste, tornò in sé, guardò l'ora, e questa volta capì che erano le cinque e mezzo e che era in ritardo. 

C'erano, quel giorno, varie corse: una per gli uomini di scorta, e una, su due verste, per gli ufficiali; un'altra su quattro verste e infine la corsa alla quale avrebbe partecipato lui. Per questa sarebbe giunto in tempo ma, passando prima da Brjanskij, sarebbe giunto dopo l'arrivo della corte. Non era ben fatto; ma aveva dato la sua parola a Brjanskij, e perciò decise di proseguire, dopo aver detto al cocchiere di non risparmiare la trojka. 

Arrivò da Brjanskij, rimase da lui cinque minuti e rifece la strada di galoppo. La corsa veloce lo calmò. Tutto quello che c'era di penoso nei suoi rapporti con Anna, tutta l'incertezza che era restata dopo la loro conversazione, tutto gli uscì di mente; ora pensava solo alla corsa con gioia e con orgasmo; pensava che sarebbe pure arrivato in tempo, e solo di tanto in tanto, l'attesa del convegno di quella notte si accendeva nella sua immaginazione di luce viva. 

La passione della corsa imminente lo prendeva sempre più a misura che si avvicinava all'ippodromo, nell'atmosfera delle corse, sorpassando le vetture di coloro che vi si recavano dai dintorni e da Pietroburgo. 

Nella sua abitazione non c'era più nessuno: tutti erano alle corse e il servo l'aspettava accanto al portone. Mentre egli si cambiava d'abito, il servo gli comunicò che era già cominciata la seconda gara e che molti signori erano venuti a chiedere di lui e che già due volte era venuto di corsa il garzone della scuderia. 

Cambiatosi senza fretta (egli non s'affrettava mai e non perdeva mai il dominio di sé), Vronskij ordinò di andare verso le baracche. Dalle baracche poteva vedere già quel mare di carrozze, di pedoni, di soldati che circondavano l'ippodromo, e le tribune piene di gente. Si correva, probabilmente, la seconda gara, perché entrando nella baracca, udì il suono della campana. Nell'avvicinarsi alla scuderia incontrò Gladatior, il sauro di Machotin dalle zampe bianche, condotto all'ippodromo con una groppiera arancione e azzurra, con le orecchie che sembravano enormi, anch'esse orlate di turchino. 

- Dov'è Kord? - domandò allo stalliere. 

- Nella scuderia, sta sellando. 

Nel recinto all'aperto, Frou-Frou era già sellata. Stava per essere portata fuori. 

- Non sono in ritardo? 

- All right! all right! Tutto, tutto bene - disse l'inglese - non vi agitate. 

Vronskij avvolse ancora una volta con lo sguardo le forme deliziose, a lui così care, della cavalla che vibrava per tutto il corpo e, staccatosene con rincrescimento, uscì dalla baracca. Si avvicinò alle tribune nel momento più opportuno per non attirare su di sé l'attenzione. Stava per terminare la corsa su due verste e tutti gli occhi erano fissi su di un cavalleggero della guardia che era in testa e su di un ussaro, a breve distanza da lui, che incitavano i cavalli all'ultimo sforzo nell'avvicinarsi al traguardo. Dal centro e dall'esterno dell'emiciclo tutti si affollavano verso il traguardo, e un gruppo di cavalleggeri, soldati e ufficiali, esprimeva, con rumorose acclamazioni, la gioia per l'atteso trionfo del loro ufficiale e compagno. Vronskij di soppiatto entrò nel mezzo della folla quasi nello stesso momento in cui sonava la campanella che annunciava la fine della corsa, e il cavalleggero della Guardia che era arrivato primo, alto, spruzzato di fango, abbandonatosi sulla sella, andava allentando le briglie allo stallone grigio, scurito dal sudore, ansante. 

Lo stallone, puntando le zampe con sforzo trattenne I'andatura veloce del corpo, e l'ufficiale dei cavalleggeri guardò intorno come un uomo risvegliato da un sonno pesante, e sorrise a stento. Una folla di amici e di estranei lo circondò. 

Vronskij evitava di proposito quella folla scelta del gran mondo che si moveva e discorreva con discrezione e disinvoltura dinanzi alle tribune. Sapeva che Ià c'erano la Karenina e Betsy e la moglie di suo fratello e, proprio per non distrarsi, non si avvicinava a loro. Ma gli amici che incontrava lo fermavano continuamente, gli raccontavano i particolari delle gare già corse, gli chiedevano perché fosse arrivato in ritardo. 

Mentre coloro che avevano già terminate le gare eran chiamati sulle tribune a ricevere i premi e tutti si volgevano verso quella parte, il fratello maggiore di Vronskij, Aleksandr, in alta uniforme da colonnello, non alto, robusto come Aleksej, ma più bello e colorito, col naso rosso e la faccia da ubriacone, aperta, si accostò a Iui. 

- Hai ricevuto il mio biglietto? - disse. - Non ti si trova mai. 

Aleksandr Vronskij, malgrado la sua vita dissipata e la sua fama di gran bevitore, era un perfetto gentiluomo di corte. 

Adesso, parlando col fratello di una cosa molto spiacevole per lui, sapendo che gli occhi di molti potevano esser fissi su di loro, assumeva un atteggiamento sorridente, come se scherzasse col fratello per cosa del tutto futile. 

- L'ho ricevuto e, davvero, non capisco di che mai tu voglia darti pensiero - disse Aleksej. 

- Mi preoccupo del fatto che proprio ora mi è stato fatto notare che non c'eri e che lunedì ti hanno incontrato a Petergof. 

- Ci sono delle cose che vanno giudicate solo da quelli che vi sono direttamente interessati, e proprio tale è la cosa di cui ti preoccupi tanto. 

- Sì, ma allora non si resta in servizio, non\ldots{} 

- Ti prego soltanto di non immischiarti. 

Il volto accigliato di Aleksej Vronskij si fece pallido, e la mascella inferiore sporgente tremò, il che accadeva di rado. Come uomo di cuore, di rado s'arrabbiava, ma quando si arrabbiava e gli tremava il labbro, allora, e Aleksandr Vronskij lo sapeva bene, era pericoloso. Aleksandr Vronskij sorrise allegro. 

- Io ti volevo unicamente consegnare la lettera della mamma. Rispondi a lei e non agitarti prima della corsa. Bonne chance - disse sorridendo, e si staccò da lui. 

Ma dopo di lui di nuovo un saluto amichevole lo fermò. 

- Non vuoi riconoscer gli amici! Buon giorno, mon cher! - cominciò a dire Stepan Arkad'ic brillando anche qui, fra lo splendore di Pietroburgo, non meno che a Mosca, col viso colorito e le fedine lucenti, ben ravviate. - Sono arrivato ieri, e sono molto contento di assistere al tuo trionfo. Quando ci vediamo? 

- Passa domani alla mensa - disse Vronskij e, strettagli, scusandosi, una manica del cappotto, si allontanò verso il centro dell'ippodromo dove facevano già entrare i cavalli per la grande corsa a ostacoli. 

I cavalli che avevano corso, sudati, sfiniti, accompagnati dagli stallieri, tornavano alla scuderia e, uno dopo l'altro, apparivano i cavalli per la corsa seguente, riposati, freschi, in gran parte inglesi, incappucciati, dal ventre asciutto, simili a strani enormi uccelli. Sulla destra conducevano Frou-Frou, magra e bella, che procedeva sulle giunture elastiche, piuttosto allungate, come su delle molle. Non lontano da lei toglievan la groppiera a Gladiator dalle orecchie lunghe. Le forme grandi, stupende, del tutto regolari dello stallone dal dorso magnifico e le giunture straordinariamente corte proprio al di sopra degli zoccoli, fermarono involontariamente l'attenzione di Vronskij. Voleva accostarsi alla sua cavalla, ma un amico lo trattenne di nuovo. 

- Ah, ecco Karenin! - gli disse l'amico col quale discorreva. - Cerca la moglie, e lei è al centro delle tribune. Non l'avete vista? 

- No, non l'ho vista - rispose Vronskij e, senza neppure voltarsi a guardare la tribuna nella quale gli avevano indicato la Karenina, si avvicinò alla cavalla. 

Vronskij non fece in tempo a osservare la sella per la quale aveva dato delle disposizioni, che chiamarono verso la tribuna i corridori per l'estrazione dei numeri. Diciassette ufficiali dal viso serio, severo, molti anche pallido, si ammassarono presso la tribuna ed estrassero i numeri. A Vronskij capitò il numero sette. Poi si udì: ``in sella!''. 

Avendo la sensazione di formare, insieme con gli altri che erano in gara, il centro dell'attenzione di tutti, Vronskij, in quel certo stato di tensione nel quale d'abitudine diveniva più calmo e lento nei movimenti, si avvicinò alla cavalla. Kord, in omaggio alle corse, si era messo l'abito di gala: un soprabito nero abbottonato, un solino inamidato che gli sosteneva le guance, un cappello tondo, nero, e gli stivaloni alla scudiera. Era, come sempre, calmo e grave e reggeva egli stesso tutte e due le briglie del cavallo, standogli ritto dinanzi. Frou-Frou continuava a tremare come se avesse la febbre. Il suo occhio pieno di fuoco guardava di traverso Vronskij che s'accostava. Vronskij le passò un dito nel sottopancia. La cavalla guardò ancor più di sbieco, mostrò i denti e drizzò l'orecchio. L'inglese fece una smorfia con le labbra, per esprimere un sorriso sul favorevole controllo alla sua abilità nel sellare. 

- Montate, sarete meno agitato. 

Vronskij si girò a guardare i suoi antagonisti per l'ultima volta. Sapeva che nella corsa non li avrebbe più visti. Due andavano avanti verso il luogo donde dovevano partire. Gal'cin, uno degli antagonisti temibili, amico di Vronskij, si aggirava intorno a uno stallone che non si lasciava montare. Un piccolo ussaro della guardia coi pantaloni stretti andava a galoppo, piegato come un gatto sul cavallo, per la mania di imitare gli inglesi. Il principe Kuzovlev montava, pallido, la sua giumenta purosangue della scuderia di Grabov, e un inglese la conduceva per il morso. Vronskij e tutti i suoi compagni conoscevano Kuzovlev, la sua particolare debolezza di nervi e il suo tremendo amor proprio. Sapevano che egli aveva paura di tutto, paura di montare un cavallo di classe; ma ora, proprio perché c'era da aver paura, proprio perché la gente si rompeva il collo e perché ad ogni ostacolo c'era un medico, l'ambulanza con la croce cucitavi sopra e una suora di carità, s'era deciso a correre. S'incontrarono con lo sguardo e Vronskij gli ammiccò con simpatia e approvazione. Soltanto uno non vide, l'antagonista principale, Machotin su Gladiator. 

- Non abbiate fretta - disse Kord a Vronskij - e ricordate una cosa sola: non la trattenete e non la spingete negli ostacoli; fatele fare quello che vuole. 

- Bene, bene - disse Vronskij, prendendo le redini. 

- Se è possibile, conducete voi la corsa, ma non perdete la speranza fino all'ultimo momento, anche foste in coda. 

La cavalla non fece in tempo a muoversi che Vronskij, con un movimento agile e forte, montò sulla staffa dentata d'acciaio e con disinvoltura e fermezza assestò il corpo ben fatto sulla sella di cuoio cigolante. Afferrando la staffa col piede destro, con gesto abituale, eguagliò fra le dita le redini che Kord lasciò scivolare dalle mani. Come non sapesse con quale zampa cominciare, Frou-Frou, distendendo col lungo collo le redini, si mosse come su delle molle, facendo oscillare il cavaliere sul dorso pieghevole. Kord, accelerando il passo, le teneva dietro. La cavalla, agitata, tirava le redini ora da una parte ora dall'altra, cercando di sfuggire al cavaliere, e Vronskij invano cercava di calmarla con la voce e con la mano. 

Si avvicinavano già al fiume sbarrato con la diga, in direzione del luogo dove avrebbero dato il via. Molti di quelli che prendevan parte alla gara erano avanti, molti indietro, quando a un tratto Vronskij udì dietro di sé, sul fango della via, il rumore del galoppare di un cavallo, e Machotin, sul suo Gladiator dalle orecchie lunghe e dalle zampe bianche, lo sorpassò. Machotin sorrise mostrando i denti lunghi, ma Vronskij lo guardò irritato. Non gli era simpatico, ora poi lo considerava il suo più temibile avversario, e gli dava fastidio il fatto che gli fosse passato accanto di galoppo, irritando la sua cavalla. Frou-Frou sollevò la zampa sinistra per mettersi al galoppo e fece due piccoli salti, ma, irritata dalla tensione delle redini, passò ad un trotto traballante che faceva sobbalzare il cavaliere. Anche Kord si accigliò e correva quasi per tener dietro a Vronskij. 

\capitolo{XXV}\label{xxv-1} 

Gli ufficiali che prendevano parte a questa corsa erano in tutto diciassette. La corsa doveva svolgersi su di un grande circuito a forma ellittica di quattro verste che si trovava dinanzi alla tribuna. Lungo questo circuito si trovavano nove ostacoli: un fiume, una grande barriera massiccia di circa due aršiny proprio davanti alla tribuna, un fosso asciutto e un altro con l'acqua, una scarpata, una banchina irlandese (uno degli ostacoli più difficili), che consisteva in un bastone ricoperto di ramaglie, dietro al quale, invisibile al cavallo, c'era ancora un fossato, così che il cavallo o doveva saltare tutti e due gli ostacoli insieme o ammazzarsi; poi ancora due fossati, uno con l'acqua e l'altro asciutto. Il traguardo era davanti alla tribuna. La corsa non iniziava dal circuito ma a cento sazeni da esso, di lato, e a questa distanza c'era il primo ostacolo, il fiume sbarrato da una diga di tre aršiny e mezzo di larghezza che i cavalieri potevano a loro piacere saltare o passare a guado. 

Per tre volte i cavalieri si misero in riga, ma ogni volta il cavallo di qualcuno ne usciva fuori e bisognava ricominciare daccapo. L'esperto di partenze, il colonnello Sestrin, cominciava già ad irritarsi, quando, finalmente, gridando per la quarta volta ``via'', i cavalieri si mossero. 

Tutti gli occhi, tutti i binocoli erano rivolti verso il gruppo multicolore dei cavalieri nel momento in cui si mettevano in riga. 

- Hanno dato il via, corrono! - si sentì da ogni parte, dopo il silenzio dell'attesa. 

E gli spettatori, a gruppi e isolati, cominciarono a correre da un posto all'altro per vedere meglio. Fin dal primo momento il gruppo dei cavalieri si allungò, e si vide come essi, a due a due, a tre a tre e uno dietro l'altro si avvicinassero al fiume. Agli spettatori pareva che fossero scattati tutti insieme; ma tra i corridori v'erano stati dei secondi di distacco che per loro avevano grande importanza. 

Frou-Frou, agitata e troppo nervosa, aveva perso il primo attimo, e alcuni cavalli si erano mossi prima di lei; ma ancor prima di arrivare al fiume Vronskij, trattenendo con tutte le forze il cavallo che tirava le briglie, ne sorpassò con facilità tre, e dinanzi a lui non rimase che Gladiator, il sauro di Machotin, che alzava con regolarità e leggerezza le zampe posteriori proprio davanti a Vronskij, e ancora, in testa a tutti, la splendida Diana che portava Kuzovlev più morto che vivo. 

Nei primi momenti Vronskij non riuscì a dominare se stesso, né la cavalla. Fino al primo ostacolo, il fiume, non poté dirigere i movimenti dell'animale. 

Gladiator e Diana si avvicinarono insieme e, quasi nello stesso momento, si sollevarono pari pari sul fiume e volarono dall'altra parte; inavvertita, quasi volando, Frou-Frou si sollevò dietro di loro; ma nello stesso attimo in cui Vronskij si sentiva sospeso in aria, vide quasi sotto le zampe della cavalla Kuzovlev che si dibatteva insieme a Diana sull'altra riva del fiume (Kuzovlev, dopo il salto, aveva abbandonato le briglie e il cavallo era capitombolato su di lui). Questi particolari Vronskij li venne a sapere dopo; in quell'attimo vide solo che proprio là dove sarebbero venute a cadere le zampe di Frou-Frou, poteva capitare una zampa o la testa di Diana. Ma Frou-Frou, come una gatta che cade, fece nel salto uno sforzo di zampe e di reni e, evitando il cavallo, galoppò oltre. 

``Oh, cara!'' pensò Vronskij. 

Dopo il fiume, Vronskij riacquistò il dominio pieno della cavalla e cominciò a trattenerla, pensando di saltare la grande barriera dietro a Machotin e di tentare di superarlo nella successiva distanza di duecento sazeni, non interrotta da ostacoli. 

La grande barriera era situata proprio dinanzi alla tribuna dello zar. L'imperatore e la corte e una folla di gente, tutti guardavano lui e Machotin, in testa per la lunghezza d'un cavallo, mentre si avvicinavano al ``diavolo'' (così veniva chiamata la barriera massiccia). Vronskij sentiva quegli sguardi rivolti su di lui da ogni parte, ma non vedeva nulla all'infuori della terra che gli correva incontro, e della groppa e delle zampe bianche di Gladiator che battevano veloci il tempo dinanzi a lui, rimanendo sempre alla stessa distanza. Gladiator saltò, senza urtare in nulla, agitò la coda e sparve agli occhi di Vronskij. 

- Bravo! - disse una voce. 

In quello stesso momento sotto gli occhi di Vronskij, proprio davanti a lui, balenarono le assi della barriera. Senza il più piccolo mutamento di andatura, la cavalla saltò sotto di lui; le assi scomparvero, ma dietro qualcosa picchiò. Eccitata da Gladiator che era in testa, la cavalla si era sollevata troppo presto sulla barriera e l'aveva urtata con lo zoccolo posteriore. Ma l'andatura non era mutata e Vronskij, nel ricevere in faccia uno schizzo di fango, capì che era sempre alla stessa distanza da Gladiator. Vide di nuovo dinanzi a sé la groppa, la coda corta e di nuovo quelle zampe bianche che si movevano rapide, ma senza allontanarsi. 

Proprio nel momento in cui Vronskij pensava di oltrepassare Machotin, Frou-Frou stessa, intuendone il pensiero, senza essere stimolata, accelerò notevolmente il galoppo, e cominciò ad avvicinarsi a Machotin dal lato più conveniente, cioè rasente la corda. Machotin però non lasciava andare la corda. Vronskij aveva appena pensato di oltrepassarlo dal lato esterno, che Frou-Frou aveva già cambiato piede e si era spinta ad oltrepassarlo proprio da questo lato. La spalla di Frou-Frou che aveva cominciato a scurirsi per il sudore, si portò alla stessa altezza del dorso di Gladiator. Per un po' galopparono insieme, ma davanti all'ostacolo al quale si avvicinavano, Vronskij, per non compiere un gran giro, si mise a lavorar di redini, e velocemente, sul pendio, oltrepassò Machotin. Vide di sfuggita la faccia di lui, inzaccherata di fango. Gli parve persino che sorridesse. Vronskij aveva superato Machotin, ma sentiva vicino e senza interruzioni, proprio dietro la schiena, il galoppo uguale e il respiro mozzato, ma ancora del tutto fresco, delle narici di Gladiator. 

I due ostacoli successivi, il fossato e la barriera, furono oltrepassati facilmente, ma Vronskij cominciò a sentire più vicini l'ansito e il galoppo di Gladiator. Lasciò andare la cavalla e con gioia sentì che essa con facilità aumentava l'andatura e che il suono degli zoccoli di Gladiator si faceva sentire di nuovo alla distanza di prima. 

Vronskij conduceva la corsa, cosa che egli stesso voleva fare e che gli aveva consigliato Kord, ed era ormai sicuro del successo. La sua agitazione, la gioia e la tenerezza per Frou-Frou divennero sempre maggiori. Voleva voltarsi indietro a guardare, ma non osava, e cercava di calmarsi e di non lanciare la cavalla per non sciupare in essa una riserva di forze eguale a quella che sentiva in Gladiator. Rimaneva un solo ostacolo e il più difficile: se egli l'avesse superato in testa, sarebbe giunto primo. Si avvicinava di galoppo alla banchina, e nello stesso momento tutti e due, lui e la cavalla, ebbero un attimo di esitazione. Egli notò nelle orecchie della cavalla indecisione e sollevò lo scudiscio, ma subito s'accorse che indecisione non c'era: la cavalla sapeva quello che occorreva fare. Accelerò l'andatura, e a tempo, proprio così come egli desiderava, si sollevò e, spintasi su da terra, si abbandonò alla forza d'inerzia che la trasportò lontano, di là dal fossato, e con la stessa cadenza, senza sforzo, senza cambiar passo, Frou-Frou riprese la corsa. 

- Bravo, Vronskij! - gli giunse da un gruppo di persone, ch'egli riconobbe come amici del reggimento, in piedi presso l'ostacolo. Non poté non distinguere la voce di Jašvin, ma non lo scorse. 

``Oh, tesoro mio!'' pensava di Frou-Frou, tendendo l'orecchio a quello che avveniva dietro di lui. ``Ha saltato'' pensò sentendo vicino il galoppo di Gladiator. Rimaneva solo l'ultimo fossato, pieno d'acqua e largo circa due aršiny. Vronskij non lo guardò neppure e, desiderando giungere di gran lunga primo, prese a lavorar di redini, alzando e abbassando la testa della cavalla. Sentiva che la cavalla sfruttava l'ultima riserva; non solo il collo e le spalle erano bagnati, ma sul garrese, sulla testa, sulle orecchie appuntite le veniva fuori il sudore, e aveva il respiro aspro e breve. Ma egli sapeva che questa riserva sarebbe stata più che sufficiente per gli ulteriori duecento sazeni. Solo da quel suo sentirsi più radente la terra e da quella particolare morbidezza dell'andatura, Vronskij poteva arguire di quanto la cavalla avesse aumentato la velocità. Essa volò sul fossato quasi senza avvedersene. Lo sorvolò come un uccello. Ma in quell'attimo stesso Vronskij sentì con orrore che, senza saper come, non era riuscito a secondare il movimento della cavalla, e, ricadendo pesantemente sulla sella, aveva fatto una mossa sbagliata, imperdonabile. E di colpo la sua posizione mutò ed egli sentì che qualcosa di spaventoso era accaduto. Prima ancora di rendersene conto gli balenarono di lato le zampe bianche dello stallone sauro, e Machotin gli passò dappresso a galoppo serrato. Vronskij si trovò a toccar terra con una gamba e la cavalla stava per abbattervisi sopra. Fece appena in tempo a liberar la gamba che quella cadde, riversa su di un fianco, rantolando pesantemente e facendo sforzi vani per rialzarsi con il sottile collo in sudore: come un uccello ferito a morte si dibatteva a terra ai piedi di lui. Il movimento malfatto di Vronskij le aveva spezzato le reni, ma egli lo capì molto tempo dopo. In quel momento vedeva solo che Machotin si allontanava veloce, e lui, barcollante, era rimasto solo sulla terra immota, fangosa; lì davanti, respirando greve, giaceva Frou-Frou che, piegando la testa verso di lui, lo guardava con i suoi occhi splendidi. Senza capire ancora quello che era avvenuto, Vronskij tirava la bestia per la briglia. Essa guizzò di nuovo tutta, come un pesciolino, facendo cricchiare le ali della sella; poggiò sulle zampe anteriori, ma non avendo la forza di sollevare il dorso, annaspò e cadde di nuovo sul fianco. Col volto sfigurato dall'emozione, pallido e col labbro inferiore che gli tremava, Vronskij la colpì col tacco nel ventre e prese di nuovo a tirarla per le briglie. Ma essa non si moveva e, ficcando il muso nel terreno, guardava il padrone con il suo sguardo parlante. 

- Aah! - muggì Vronskij, afferrandosi la testa. - Aah! Che ho fatto! - gridò. - E la corsa è perduta! E la colpa è mia, vergognosa, imperdonabile. E questa povera cara bestia perduta! Aah, che ho fatto! 

Un dottore e un infermiere, gli ufficiali del reggimento corsero, insieme con altra gente, verso di lui. Per sua disgrazia sentiva d'essere incolume e sano. La cavalla s'era spezzata la schiena, e fu deciso di finirla. Vronskij non poteva rispondere alle domande, non poteva parlare con nessuno. Si voltò e, senza raccattare il berretto che gli era saltato di testa, se ne andò via dall'ippodromo, non sapendo egli stesso dove. Si sentiva infelice. Per la prima volta in vita sua provava una pena così grande, così irreparabile, di cui la colpa era tutta sua. 

Jašvin lo raggiunse, portandogli il berretto e lo accompagnò fino a casa, e dopo mezz'ora Vronskij ritornò in sé. Ma il ricordo di questa corsa rimase per lungo tempo nell'animo suo come il ricordo più penoso e tormentoso della sua vita. 

\capitolo{XXVI}\label{xxvi-1} 

I rapporti esteriori di Aleksej Aleksandrovic con la moglie permanevano invariati. L'unica differenza consisteva nel fatto che egli era più occupato di prima. All'inizio della primavera andò all'estero per fare una cura di acque termali che ristabilisse la salute sua debilitata ogni anno dallo sforzo invernale. E, come al solito, tornò in luglio, e immediatamente, con aumentata energia, si dedicò alle occupazioni abituali. Come al solito sua moglie andò in campagna ed egli rimase a Pietroburgo. 

Dal tempo della conversazione avvenuta dopo la serata in casa della principessa Tverskaja, egli non aveva mai più parlato con Anna dei suoi sospetti e della sua gelosia; e quel suo solito tono di chi sente di essere qualcuno era quanto mai comodo per i presenti rapporti con la moglie. Era soltanto un po' freddo. Dava a vedere come fosse rimasta in lui una certa piccola scontentezza verso di lei per quella prima conversazione notturna ch'ella aveva voluto evitare. C'era pertanto nei suoi rapporti verso di lei, come un'ombra di dispetto, ma nulla di più. ``Tu non hai voluto avere una spiegazione - era come se le dicesse rivolgendosi a lei col pensiero - tanto peggio per te. Ormai sarai tu a pregarmene, ma io spiegazioni non ne darò. Tanto peggio per te'' diceva nel pensiero, come un uomo che abbia invano tentato di spegnere un incendio e, irritato contro i propri inutili sforzi, finisca col dire: ``Tanto peggio! Che bruci pure!''. 

Egli, intelligente e sottile negli affari di ufficio, non capiva tutta l'aberrazione di un simile comportamento. Non la capiva perché era troppo terribile per lui veder chiara la sua vera posizione, ed egli intanto nell'animo suo aveva nascosta, chiusa e sigillata quella tale cassetta nella quale si trovavano riposti i sentimenti suoi per la famiglia, per la moglie e per il figlio. Padre premuroso, dalla fine dell'inverno era diventato particolarmente freddo verso il figlio, e aveva verso di lui quello stesso tono canzonatorio che assumeva verso la moglie. ``Ohi, giovanotto'' gli diceva. 

Aleksej Aleksandrovic pensava e diceva che mai, come in quell'anno, aveva avuto tanto lavoro d'ufficio; e non voleva accorgersi che il lavoro se l'era creato lui stesso in quell'anno, che era stato uno dei mezzi per non aprire quella tale cassetta dove stavano rinchiusi i sentimenti suoi per la moglie e la famiglia: mentre il pensiero di costoro tanto più sgomentoso diveniva quanto più a lungo egli lo relegava là. E se qualcuno avesse avuto il diritto di chiedere ad Aleksej Aleksandrovic che cosa egli pensasse della condotta di sua moglie, quel pacifico, calmo Aleksej Aleksandrovic non avrebbe risposto nulla, e si sarebbe molto sdegnato contro la persona che gliene avesse chiesto. Proprio per questo vi era nell'espressione del viso di Aleksej Aleksandrovic qualcosa di sostenuto e di severo quando gli domandavano della salute di sua moglie. Aleksej Aleksandrovic non voleva pensare nulla circa la condotta di sua moglie, e realmente non ne pensava nulla. 

La dimora estiva consueta di Aleksej Aleksandrovic era a Petergof, dove abitualmente anche la contessa Lidija Ivanovna passava l'estate in compagnia e in continui rapporti con Anna. Quell'anno la contessa Lidija Ivanovna non aveva voluto soggiornare a Petergof, non era stata da Anna Arkad'evna neppure una volta, e aveva accennato ad Aleksej Aleksandrovic la sconvenienza dell'assiduità di Anna con Betsy e Vronskij. Aleksej Aleksandrovic le aveva chiuso la bocca, affermando con fermezza che sua moglie era al disopra di ogni sospetto; ma da allora aveva cercato di evitare la contessa Lidija Ivanovna. Non voleva vedere, e non vedeva che in società già molti guardavano di traverso sua moglie; non voleva capire e non capiva perché sua moglie insistesse per andare a Carskoe dove viveva Betsy e dove non sarebbe stata lontana dal campo del reggimento di Vronskij. Non si permetteva di pensare questo, e non lo pensava; tuttavia in cuor suo, pur senza dirselo mai, e pur senza averne non solo prova alcuna, ma neppure fondato sospetto, sapeva con certezza d'essere un marito ingannato, ed era per questo profondamente infelice. 

Quante volte durante i suoi otto anni di vita coniugale felicemente trascorsi, vedendo mogli infedeli e mariti ingannati, Aleksej Aleksandrovic si era detto: ``Ma come si può giungere a questo? Perché non troncare una situazione sconveniente?''. Ora, invece, che la disgrazia era piombata sul suo capo, non solo non pensava al modo di provvedere alla situazione, ma non voleva riconoscerla affatto, non voleva vederla, proprio perché era troppo penosa, troppo innaturale. 

Dal tempo del suo ritorno dall'estero, Aleksej Aleksandrovic era stato due volte in campagna. Una volta vi aveva pranzato, un'altra volta aveva passato la serata con ospiti, ma non vi aveva neanche una volta passato la notte, come era solito fare gli anni precedenti. 

Il giorno delle corse era un giorno pieno di lavoro per Aleksej Aleksandrovic; ma, predisposto fin dal mattino il programma della giornata, aveva deciso di andare, subito dopo colazione, in campagna dalla moglie, e di là alle corse, dove si sarebbe trovata tutta la corte e dove egli doveva andare. E dalla moglie sarebbe passato perché aveva deciso di andarle a far visita una volta alla settimana, per convenienza. Inoltre doveva consegnare alla moglie, proprio quel giorno che era il 15 del mese, secondo l'ordine da lui stabilito, il denaro per le spese. 

Dopo aver pensato tutto questo circa la moglie, con l'abituale dominio che aveva sui suoi pensieri, non permise loro di girovagare oltre, intorno a quanto la riguardava. 

La mattina fu tutta presa per Aleksej Aleksandrovic. Il giorno innanzi, la contessa Lidija Ivanovna gli aveva mandato un opuscolo di un noto viaggiatore della Cina, attualmente a Pietroburgo, con una lettera in cui lo pregava di ricevere il viaggiatore, uomo per varie considerazioni sempre interessante e utile. Aleksej Aleksandrovic non aveva fatto in tempo a leggere l'opuscolo la sera, e ne terminò la lettura la mattina. Dopo, s'erano presentati i consueti sollecitatori, erano cominciati i rapporti, i ricevimenti, le nomine, le rimozioni, le distribuzioni delle ricompense, delle pensioni, degli stipendi, la corrispondenza, quel lavoro quotidiano, infine, come lo chiamava Aleksej Aleksandrovic, che portava via tanto tempo. Poi c'erano state le occupazioni che lo riguardavano personalmente: la visita del dottore e dell'amministratore. L'amministratore non gli aveva preso molto tempo. Aveva consegnato solo il denaro necessario ad Aleksej Aleksandrovic ed aveva fatto un breve resoconto dello stato non troppo buono delle cose, giacché, in quell'anno, per i frequenti viaggi, si era speso di più, e c'era stato un certo dissesto. Ma il dottore, un celebre medico di Pietroburgo, che era in rapporti amichevoli con Aleksej Aleksandrovic, gli portò via molto tempo. Aleksej Aleksandrovic non lo aspettava quel giorno e fu stupito del suo arrivo e, ancor più, che il dottore lo interrogasse molto minutamente circa le sue condizioni di salute, gli ascoltasse il petto, picchiasse e tastasse il fegato. Non sapeva Aleksej Aleksandrovic che la sua amica Lidija Ivanovna, avendo notato che la salute di Aleksej Aleksandrovic quell'anno non era buona, aveva pregato il dottore di andare e di osservare il malato. ``Fatelo per me'' gli aveva detto la contessa Lidija Ivanovna. 

- Lo farò per la Russia, contessa - aveva risposto il dottore. 

- Un uomo inestimabile - aveva ribattuto la contessa Lidija Ivanovna. 

Il dottore era rimasto molto scontento di Aleksej Aleksandrovic. Aveva trovato il fegato notevolmente ingrossato, un certo esaurimento, nessun effetto della cura delle acque. Aveva ordinato molto esercizio fisico e poco sforzo intellettuale e, soprattutto, di guardarsi dai dispiaceri, il che per Aleksej Aleksandrovic era impossibile, così come è impossibile non respirare; e se n'era andato, lasciando in Aleksej Aleksandrovic la spiacevole consapevolezza che in lui qualcosa non andava e non si poteva aggiustare. 

Uscendo dalla camera di Aleksej Aleksandrovic il dottore si era imbattuto sulla scala con Šljudin, a lui ben noto, capogabinetto di Aleksej Aleksandrovic. Erano stati compagni di università e, sebbene si incontrassero di rado, si stimavano ed erano buoni amici; a nessuno perciò meglio che a Šljudin il dottore avrebbe detto tutta la sua sincera opinione sull'ammalato. 

- Come son contento che siate stato da lui - disse Šljudin. - Non sta bene, mi sembra. Che cos'ha? 

- Ecco, cos'ha - disse il dottore facendo un cenno al cocchiere di avanzare, al di sopra della testa di Šljudin. - Ecco vedete - disse il dottore prendendo nelle sue mani bianche il dito di un guanto di pelle e tirandolo. - Provate a spezzare una corda senza tenderla\ldots{} è molto difficile; tendetela invece fino all'estrema possibilità e poggiatevi sopra il peso di un dito\ldots{} si spezzerà. Per la sua assiduità, la sua scrupolosità nel lavoro, egli è teso fino all'estremo limite; e la pressione esterna c'è, e forte - concluse il dottore, aggrottando significativamente le sopracciglia. - Andate alle corse? - aggiunse, scendendo verso la carrozza che era stata fatta avanzare. - Sì, sì, s'intende, sarà una cosa lunga - rispose il dottore o rispose qualcosa di simile, a quello che aveva detto Šljudin e che egli non aveva afferrato. 

Dopo il dottore che gli aveva preso tanto tempo, si presentò il noto viaggiatore e Aleksej Aleksandrovic, profittando dell'opuscolo letto proprio allora e di una precedente conoscenza dell'argomento, stupì il viaggiatore con la profondità delle sue conoscenze e la larghezza delle sue vedute. 

Insieme al viaggiatore fu annunciato l'arrivo di un maresciallo della nobiltà giunto da poco a Pietroburgo e con il quale si doveva avere un colloquio. Dopo che questi se ne fu andato, dovette sbrigare le pratiche quotidiane col capo di gabinetto e dovette inoltre andare da un personaggio autorevole per un affare grave e importante. Aleksej Aleksandrovic fece appena in tempo a rientrare alle cinque, ora del suo pranzo, e dopo aver mangiato in compagnia del capogabinetto, lo invitò ad andare con lui in campagna e alle corse. 

Senza rendersene conto Aleksej Aleksandrovic cercava ormai l'occasione di avere una terza persona presente ai suoi incontri con la moglie. 

\capitolo{XXVII}\label{xxvii-1} 

Anna stava davanti allo specchio, appuntando, con l'aiuto di Annuška, l'ultimo nastro al vestito, quando sentì un rumore di ruote che calpestavano la ghiaia dell'ingresso. 

``Per Betsy è ancora presto - pensò e, guardando dalla finestra, vide una carrozza dalla quale uscirono il cappello nero e le ben note orecchie di Aleksej Aleksandrovic. - Che disdetta! Possibile che venga qui a passar la notte?'' e le parve così orribile e pauroso quello che poteva venirne fuori che, senza riflettere un attimo, gli uscì incontro col volto allegro e luminoso, e, sentendo vivo in sé lo spirito della menzogna e dell'inganno, subito vi si abbandonò, e cominciò a parlare senza sapere neppure lei cosa diceva. 

- Oh, come ciò è gentile! - disse, dando la mano al marito e salutando con un sorriso Šljudin che era persona di casa. - Passerai la notte qua, spero? - le suggerì per prima cosa lo spirito dell'inganno - e ora andiamo insieme alle corse. Peccato che abbia già promesso a Betsy. Deve passare a prendermi. 

Aleksej Aleksandrovic si accigliò al nome di Betsy. 

- Oh, non starò a separare le inseparabili - disse col suo abituale tono canzonatorio. - Andrò con Michail Vasil'evic. Anche i dottori mi ordinano di camminare. Farò la strada a piedi e immaginerò di essere alla stazione termale. 

- Non c'è bisogno di affrettarsi - disse Anna. - Volete il tè? - e sonò. 

- Servite il tè, e dite a Serëza che Aleksej Aleksandrovic è qui. Be', come va la tua salute? Michail Vasil'evic, voi non siete mai stato da me; guardate come si sta bene sul mio balcone - diceva rivolgendosi ora all'uno, ora all'altro. 

Parlava con semplicità e naturalezza, ma troppo e troppo in fretta. Lo sentiva lei stessa, tanto più che, nello sguardo incuriosito col quale la guardava Michail Vasil'evic, notava di essere osservata. 

Michail Vasil'evic uscì subito sulla terrazza. 

Anna sedette accanto al marito. 

- Non hai un buon aspetto - disse. 

- Già - disse egli - oggi il dottore è stato da me e mi ha portato via un'ora di tempo. Sospetto che qualcuno dei miei amici me lo abbia mandato: la mia salute è così preziosa\ldots{} 

- E cosa ha detto? 

Ella gli chiedeva della sua salute e delle sue occupazioni, voleva convincerlo a riposarsi e venire a stare da lei. 

Tutto questo lo diceva con vivacità, in fretta e con un particolare luccichio negli occhi; ma Aleksej Aleksandrovic, ora, non rilevava questo tono. Ascoltava le parole e dava loro solo il significato che avevano. E le rispondeva semplicemente, sia pure scherzando. In questa conversazione non ci fu nulla di particolare, ma in seguito Anna non poté mai ricordare questa breve scena senza un tormentoso senso di vergogna. 

Entrò Serëza, preceduto dalla governante. Se Aleksej Aleksandrovic avesse permesso a se stesso di osservare avrebbe notato lo sguardo timido, spaurito col quale Serëza guardava il padre e poi la madre. Ma egli non voleva vedere, e non vedeva. 

- Ohi, giovanotto! È cresciuto. Davvero, sta diventando un uomo. Buongiorno, giovanotto. 

E diede la mano a Serëza spaventato. 

Serëza, anche prima timido nei rapporti col padre, ora, da quando Aleksej Aleksandrovic aveva preso a chiamarlo giovanotto e da quando gli si era posto nella mente il dilemma se Vronskij fosse un amico o un nemico, sfuggiva il padre. Quasi per averne protezione, si rivolse alla madre. Con la madre stava bene quando era sola. Intanto Aleksej Aleksandrovic, parlando con la governante, teneva il figlio per la spalla. Serëza era così tormentosamente a disagio che Anna si accorse che stava lì lì per piangere. 

Anna, che era diventata rossa nel momento in cui era entrato il figlio, ne notò il disagio, si alzò in fretta, tolse la mano di Aleksej Aleksandrovic dalla spalla del figlio e, dopo averlo baciato, lo condusse sulla terrazza; e subito dopo rientrò. 

- Ma è già ora - disse, guardando l'orologio - come mai Betsy non viene?\ldots{} 

- Già - disse Aleksej Aleksandrovic e, alzatosi, intrecciò le mani e le fece scricchiolare. - Sono passato da te, anche per portarti del denaro, dal momento che gli usignuoli non vivono di fiabe - disse. - Ti occorre, penso. 

- No, non mi occorre\ldots{} sì, mi occorre - diss'ella senza guardarlo e arrossendo fino alla radice dei capelli. - Ma tu passerai di qua, spero, tornando dalle corse. 

- Oh, sì - rispose Aleksej Aleksandrovic. - Ed ecco ora la stella di Petergof, la principessa Tverskaja - soggiunse dopo aver guardato dalla finestra il tiro inglese con le bardature, che si avvicinava con una minuscola carrozzetta straordinariamente alta. - Che eleganza! Un incanto! Su, allora, andiamo anche noi. 

La principessa Tverskaja non uscì dalla vettura, ma solo il servitore in ghette, pellegrina e cappello nero, saltò giù all'ingresso. 

- Io vado, addio - disse Anna e, baciato il figlio, si avvicinò ad Aleksej Aleksandrovic, dandogli la mano. 

- Su, allora, arrivederci. Tu passerai a prendere il tè, benissimo! - ella disse, e uscì splendente e gaia. Ma appena non lo vide più, sentì sulla mano il punto preciso che le labbra di lui avevano toccato e rabbrividì di disgusto. 

\capitolo{XXVIII}\label{xxviii-1} 

Quando Aleksej Aleksandrovic apparve alle corse, Anna era già seduta nella tribuna accanto a Betsy, in quella tribuna dove era raccolta tutta l'alta società. Scorse il marito da lontano. Due esseri, il marito e l'amante, erano per lei i due centri della sua vita, ed ella ne avvertiva la vicinanza senza bisogno dei sensi esterni. Avvertì ancora da lontano l'avvicinarsi del marito, e suo malgrado lo seguì nella marea di folla fra la quale avanzava. Lo vide avvicinarsi alla tribuna, ora rispondendo con indulgenza ai saluti adulatori, ora salutando con cordialità e distrazione i colleghi, ora aspettando con desiderio lo sguardo dei potenti e sollevando il gran cappello tondo che gli premeva l'estremità delle orecchie. Conosceva tutti gli atteggiamenti di lui, e tutti le erano odiosi. ``Soltanto falsità, soltanto ambizione, ecco tutto quello che c'è nell'animo suo - pensava - e le idee di ordine superiore, l'amore per la cultura, la religione, tutte queste cose non sono altro che mezzi per affermarsi''. 

Dalla direzione del suo sguardo verso la tribuna delle signore (egli guardava dritto in questa, ma non riconosceva la moglie in quel mare di stoffe, nastri, piume, ombrellini e fiori), ella capì che la cercava ma finse di non accorgersene. 

- Aleksej Aleksandrovic! - gli gridò la principessa Betsy - voi probabilmente non vedete vostra moglie: eccola! 

Egli sorrise col suo sorriso freddo. 

- Qui c'è tanto splendore che gli occhi ne restano abbagliati - disse, e andò verso la tribuna. Sorrise alla moglie, come deve sorridere un marito che ritrova la moglie dopo averla vista un momento prima, e salutò la principessa e gli altri amici, dando a ciascuno il suo, scherzando, cioè, con le signore e scambiando dei convenevoli con gli uomini. Giù, accanto alla tribuna, stava in piedi un generale, un aiutante di campo che Aleksej Aleksandrovic stimava e che era noto per il suo ingegno e la sua cultura. Aleksej Aleksandrovic si mise a discorrere con lui. 

C'era intervallo fra una corsa e l'altra, e perciò nulla disturbava la loro conversazione. Il grande generale deprecava le corse. Aleksej Aleksandrovic ribatteva, prendendone le difese. Anna ascoltava la voce stridula, eguale di lui, senza perderne neppure una parola, e ogni parola le sembrava falsa e le colpiva dolorosamente l'orecchio. 

Quando cominciò la corsa a ostacoli su quattro verste, ella si sporse in avanti e, gli occhi fissi su Vronskij, prese a seguirlo mentre si avvicinava al cavallo e lo montava, e nello stesso tempo ascoltava l'odiosa, instancabile voce del marito. Era tormentata dal timore per Vronskij, ma ancora più dalla instancabile voce stridula del marito della quale conosceva tutte le intonazioni. 

``Sono una donna cattiva, sono una donna perduta - pensava - ma non mi piace mentire e non sopporto la menzogna, mentre Aleksej Aleksandrovic si pasce di menzogna. Egli sa tutto, vede tutto; che cosa mai c'è in lui, dunque, se può così tranquillamente parlare? Uccidesse me, uccidesse Vronskij, lo stimerei. Ma no, a lui bastano soltanto la menzogna e il rispetto delle convenienze'' si diceva Anna, senza pensare con precisione a quello che avrebbe voluto che il marito facesse, e sotto qual luce avrebbe voluto vederlo. Non capiva che anche quell'eccessiva verbosità di Aleksej Aleksandrovic, che tanto la irritava, era, in quel momento, l'espressione dell'inquietudine e dell'intima agitazione di lui. Come un bambino che, dopo aver urtato in qualche cosa, mette in moto, saltando, i propri muscoli per soffocare il dolore, così Aleksej Aleksandrovic aveva bisogno di un moto intellettuale per soffocare quei suoi pensieri sulla moglie che ora, alla presenza di lei e alla presenza di Vronskij, e alla continua ripetizione del nome di lui, urgevano perché si prestasse loro attenzione. E come al bambino vien naturale di saltare, così a lui veniva fatto di parlare bene e con intelligenza. Egli diceva: 

- Il pericolo nelle corse dell'arma di cavalleria, è un rischio che non si può eliminare in ogni corsa. Se l'Inghilterra può vantare nella sua storia militare le più brillanti azioni della cavalleria, è solo grazie al fatto che essa ha sviluppato, evolvendola nella storia, questa forza e di animali e di uomini. Lo sport, secondo la mia opinione, ha un grande valore, e, come sempre, noi ne vediamo soltanto il lato più superficiale. 

- Non tanto superficiale - disse la principessa Tverskaja. - Un ufficiale, dicono, si è rotto le costole! 

Aleksej Aleksandrovic sorrise col suo sorriso che gli scopriva soltanto i denti ma che non diceva nulla. 

- Ammettiamo, principessa, che questo non sia superficiale - egli disse - ma profondo. Ma non è qui la questione - ed egli si rivolse di nuovo al generale col quale parlava seriamente. - Non dimenticate che corrono dei militari i quali hanno scelto questa attività, e convenite che ogni attività ha il rovescio della medaglia. Questo rientra proprio nei doveri del militare. Lo sport scandaloso del pugilato o delle corride spagnole è un segno di barbarie. Ma uno sport specializzato è un segno di progresso. 

- No, non ci verrò più; tutto questo mi agita troppo - diceva la principessa Betsy. - Non è vero, Anna? Agita, ma non se ne possono distaccare gli occhi. Se fossi stata una romana, non avrei tralasciato un solo spettacolo del circo. 

Anna non parlava e senza abbandonare il binocolo, guardava in un punto solo. 

In quel momento, attraverso la tribuna, passò un ufficiale di alto grado. Interrotto il discorso, Aleksej Aleksandrovic si alzò in fretta, ma con dignità, e salutò profondamente l'ufficiale che passava. 

- Voi non correte? - gli disse, scherzando, l'ufficiale. 

- La mia corsa è più difficile - rispose rispettoso Aleksej Aleksandrovic. 

E sebbene la risposta non significasse nulla, l'ufficiale fece finta di aver colto una battuta di spirito intelligente, detta da un uomo d'ingegno e di aver capito in pieno la pointe de la sauce. 

- Qui vi sono due categorie di persone - riprese a dire Aleksej Aleksandrovic - quella dei partecipanti e quella degli spettatori. L'amore per questi spettacoli è il segno più sicuro del basso livello degli spettatori, ne convengo, ma\ldots{} 

- Principessa, una scommessa! - si sentì da basso la voce di Stepan Arkad'ic che si rivolgeva a Betsy. - Per chi tenete? 

- Io e Anna per il principe Kuzovlev - rispose Betsy. 

- Io per Vronskij. Un paio di guanti. 

- Vada pure! 

- Che bello spettacolo, non è vero? 

Aleksej Aleksandrovic tacque per un po' finché non finirono di parlare intorno a lui, ma poi ricominciò subito. 

- Ne convengo, non sono giuochi da uomini - e voleva continuare. 

Ma intanto davano il via ai cavalieri, e tutte le conversazioni cessarono. Aleksej Aleksandrovic tacque anche lui e tutti si alzarono e si volsero verso il fiume. Aleksej Aleksandrovic non si interessava alle corse e perciò non badava a quelli che correvano, ma distrattamente cominciò a girare intorno, sugli spettatori, i suoi occhi stanchi. Il suo sguardo si fermò su Anna. 

Il viso di lei era pallido e teso: ella evidentemente non vedeva niente e nessuno, tranne uno. Tratteneva il respiro, e la sua mano stringeva convulsa il ventaglio. Aleksej Aleksandrovic la guardò e si voltò in fretta a osservare altri visi. 

``Ma ecco, anche questa signora e le altre ancora sono agitate; ciò è molto naturale'' si diceva Aleksej Aleksandrovic. Non voleva guardare più; ma gli occhi erano involontariamente attratti verso di lei. Esaminava quel viso sforzandosi di non leggervi ciò che così chiaramente vi era scritto; e contro la sua volontà vi leggeva con terrore quello che non voleva sapere. 

La prima caduta di Kuzovlev nel compiere il salto del fiume impressionò tutti, ma Aleksej Aleksandrovic vide chiaramente sul pallido viso trionfante di Anna che quegli ch'ella seguiva non era caduto. Quando poi Machotin e Vronskij ebbero saltato la barriera, e l'ufficiale che veniva dopo cadde con la testa in giù e si abbatté come morto e un brivido di orrore percorse tutto il pubblico, Aleksej Aleksandrovic vide che Anna non aveva neppure notato questo, e che a stento capiva di che si parlasse intorno. E la osservava sempre di più e con maggiore ostinazione. Anna, tutta presa dalla vista di Vronskij che correva, sentiva di lato lo sguardo freddo del marito fisso su di lei. 

Si voltò per un attimo, lo fissò interrogativamente e, accigliandosi lievemente, si girò di nuovo. 

``Oh, non mi importa più!'' era come se gli avesse detto e non guardò più neppure una volta. 

La corsa fu disgraziata e su diciassette persone ne caddero e si fecero male più della metà. 

Alla fine delle corse tutti erano in uno stato di agitazione, tanto più che il sovrano se ne era mostrato scontento. 

\capitolo{XXIX}\label{xxix-1} 

Tutti esprimevano ad alta voce la loro disapprovazione, tutti ripetevano la frase messa in giro da qualcuno: ``non ci manca che il circo con i leoni''. Il terrore era sentito da tutti, sì che quando Vronskij cadde ed Anna emise un gemito, non ci fu nulla di straordinario. Ma subito dopo nel volto di Anna apparve un turbamento già troppo sconveniente. S'era smarrita del tutto; si dibatteva come un uccello al laccio; ora voleva alzarsi e andare chi sa dove, ora si volgeva a Betsy. 

- Andiamo, andiamo - diceva. 

Ma Betsy non l'ascoltava. Parlava, sporgendosi in giù, con un generale che le si era avvicinato. 

Aleksej Aleksandrovic si avvicinò ad Anna e le porse cortesemente la mano. 

- Andiamo, se vi fa piacere - disse in francese, ma Anna era intenta ad ascoltare quello che diceva il generale e non si curò del marito. 

- Anche lui si è rotto una gamba, dicono - diceva il generale. - Ma che senso c'è in tutto questo? 

Anna, senza rispondere al marito, aveva sollevato il binocolo e guardava il punto dove era caduto Vronskij: ma era così lontano e vi si era affollata così tanta gente che nulla di distingueva. Abbassò il binocolo e fece per andarsene; ma in quel momento giunse un ufficiale a cavallo a riferire qualcosa allo zar. Anna si sporse in avanti per ascoltarlo. 

- Stiva! Stiva! - gridò al fratello. 

Ma il fratello non la udì. Ella di nuovo voleva andar via. 

- Vi offro ancora una volta il braccio, se volete andare - disse Aleksej Aleksandrovic, toccandole il braccio. 

Ella si scostò da lui con ribrezzo e, senza guardarlo in viso, rispose: 

- No, no, lasciatemi, rimango. 

Vedeva adesso che dal punto dove era caduto Vronskij correva, attraversando tutto il circuito, un ufficiale diretto alla tribuna. Betsy gli faceva cenno col fazzoletto. L'ufficiale portò la notizia che il cavaliere era salvo, ma il cavallo si era rotto la schiena. 

Udito questo, Anna si sedette di colpo e si coprì il viso col ventaglio. Vedendo che ella piangeva e che, non solo non riusciva a trattenere le lacrime, ma neanche i singhiozzi che le sollevavano il petto, Aleksej Aleksandrovic la coprì con la propria persona, dandole il tempo di rimettersi. 

- Per la terza volta vi offro il mio braccio - disse dopo un po' di tempo, rivolgendosi a lei. Anna lo guardava e non sapeva cosa dire. La principessa Betsy venne in suo aiuto. 

- No, Aleksej Aleksandrovic, ho accompagnato io Anna, e io ho promesso di riaccompagnarla - s'intromise. 

- Perdonatemi, principessa - egli disse, sorridendo con cortesia, ma guardandola fermo negli occhi - io vedo che Anna non sta del tutto bene e desidero che venga con me. 

Anna si voltò a guardarlo spaventata, si alzò sottomessa e poggiò la mano sul braccio del marito. 

- Manderò da lui, m'informerò e poi farò sapere - le sussurrò Betsy. 

All'uscita della tribuna, Aleksej Aleksandrovic, come sempre, parlava con quelli che incontrava e Anna doveva come sempre rispondere e parlare; ma era proprio fuori di sé e come in sogno andava sotto il braccio del marito. 

``Si è ammazzato o no? È vero? Verrà o no? Lo vedrò stasera?'' pensava. 

In silenzio prese posto nella vettura di Aleksej Aleksandrovic e in silenzio rimase anche quando si furono allontanati dalla calca degli equipaggi. Malgrado tutto quello che aveva visto, Aleksej Aleksandrovic non si permetteva di pensare alla reale posizione della moglie. Egli coglieva solo i segni esteriori, aveva visto ch'ella si comportava in modo poco conveniente, e riteneva suo dovere dirglielo. Ma era molto difficile non dire nulla di più, dirle soltanto questo. Aprì la bocca per dirle che si era comportata in modo sconveniente, e invece, senza volere, disse tutt'altra cosa. 

- Ma come siamo tutti inclini a questi spettacoli feroci - disse. - Io noto\ldots{} 

- Cosa? Non capisco - disse Anna in tono sprezzante. 

Egli si offese e cominciò subito a dirle quello che voleva. 

- Devo dirvi\ldots{} - cominciò. 

``Eccola, la spiegazione'' pensò lei, e n'ebbe paura. 

- Devo dirvi che vi siete comportata in modo del tutto sconveniente - egli disse in francese. 

- In che cosa mi sono comportata in modo sconveniente? - ella disse forte, voltandosi rapida verso di lui e guardandolo dritto negli occhi, non più con quella sua allegria mordace di prima, ma con un'aria decisa che nascondeva a stento il terrore provato. 

- Non dimenticate - egli disse, indicandole il vetro aperto di contro al cocchiere. 

E si alzò e lo tirò su. 

- Che cosa avete trovato di sconveniente? - ella ripeté. 

- Quella disperazione che non avete saputo nascondere per la caduta di uno dei cavalieri. 

S'aspettava che ella ribattesse. Ma ella taceva, guardando davanti a sé. 

- Vi ho già pregata di comportarvi in modo che anche le male lingue non abbiano a dire nulla contro di voi. Un tempo vi ho parlato di rapporti interiori; ora non ne parlo più. Ora vi parlo solo dei rapporti esteriori. Vi siete comportata in modo sconveniente, e desidero che ciò non si ripeta. 

Ella non sentiva nemmeno metà delle sue parole; aveva paura di lui, ma intanto pensava se era vero che Vronskij non era rimasto ucciso. Era di lui che dicevano che era rimasto illeso, mentre il cavallo s'era spezzata la schiena? Appena egli ebbe finito di parlare, ella sorrise in quella sua maniera beffarda e falsa, e non rispose perché non aveva sentito quello che aveva detto. Aleksej Aleksandrovic allora riprese a parlare arditamente, ma appena ebbe coscienza di quello che diceva, il terrore di Anna si comunicò a lui. Notò quel riso, e una strana aberrazione lo prese. 

``Ride dei miei sospetti. Ecco, ora mi dirà subito quello che ha già detto l'altra volta: che i miei sospetti sono infondati, che tutto ciò è ridicolo''. 

Ora che era sospesa su di lui la scoperta di tutto, nulla desiderava tanto quanto ch'ella rispondesse beffarda, così come l'altra volta, che i suoi sospetti erano infondati e ridicoli. Così spaventoso era quello che sapeva che era pronto a credere a tutto. Ma l'espressione del viso di lei, atterrito e torvo, non prometteva ora neppure l'inganno. 

- Forse io mi sbaglio - disse. - In tal caso vogliate perdonarmi. 

- No, non vi siete sbagliato - ella disse lentamente, guardando con disperazione il suo viso impassibile. - Voi non vi siete sbagliato. Sono sconvolta e non posso non esserlo ancora. Io ascolto voi, e penso a lui. Io amo lui, sono la sua amante, e non posso più resistere. Ho paura, vi odio\ldots{} Fate di me quel che volete. 

E riversatasi all'indietro in un angolo della carrozza, scoppiò in singhiozzi, coprendosi il viso con le mani. Aleksej Aleksandrovic non si mosse e non mutò la direzione del suo sguardo, fisso davanti a sé. Ma tutto il suo viso prese ad un tratto l'immobilità solenne di un cadavere e questa espressione permase tale per tutto il tempo del percorso fino alla villa. Avvicinandosi alla casa, egli girò il capo verso di lei, sempre con la stessa espressione del viso. 

- Già, ma io pretendo l'osservanza delle forme esteriori fino al momento in cui - e qui la voce gli tremò - non avrò prese le misure necessarie per difendere il mio onore e ve le avrò comunicate. 

Uscì dalla carrozza e l'aiutò a discendere. In presenza della servitù le strinse in silenzio la mano, risalì in vettura e partì per Pietroburgo. 

Subito dopo venne un cameriere da parte della principessa Betsy e recò un biglietto per Anna. 

``Ho mandato da Aleksej per sapere della sua salute, ed egli mi scrive che è sano e salvo, ma desolato''. 

``Allora verrà - pensò. - Come ho fatto bene a dirgli tutto!''. 

Guardò l'orologio. Mancavano ancora tre ore, e il ricordo dei particolari dell'ultimo incontro le accese il sangue. 

``Dio mio, come è chiaro ancora! È terribile, ma io amo vederlo quel suo viso, e amo questa luce fantastica\ldots{} Mio marito, ah, già\ldots{} Ma, grazie a Dio, con lui tutto è finito''. 

\capitolo{XXX}\label{xxx-1} 

Come in tutti i luoghi dove si riunisce gente varia, così pure nella piccola stazione termale tedesca dove erano arrivati gli Šcerbackij, si era venuta a formare quella tale, per così dire, cristallizzazione della società che ad ogni suo membro fissa un posto definito e immutabile. Ogni nuovo personaggio che arrivava nel luogo di cura, si fissava nel posto che gli era proprio, così come una goccia d'acqua riceve dal freddo, definita e immutabile, una determinata forma di ghiacciuolo. 

Fürst Šcerbackij sammt Gemahlin und Tochter per il nome e per l'appartamento che occupavano e per gli amici che avevano trovato, si cristallizzarono nel loro posto definito e ad essi destinato. 

Quell'anno, alla stazione termale, c'era un'autentica Fürstin tedesca; perciò la ``cristallizzazione'' si operava in maniera ancor più rigida. La principessa Šcerbackaja volle assolutamente presentare sua figlia alla principessa di sangue reale, e fin dal giorno dopo l'arrivo compì questo rito. Kitty fece una profonda e graziosa riverenza nel suo vestito estivo molto semplice, e perciò molto elegante, ordinato a Parigi. La principessa reale disse: ``Spero che le rose torneranno presto a fiorire sul quel bel visino''. E da quel momento per gli Šcerbackij si fissò saldamente, e subito, un determinato tenore di vita al quale non era possibile sottrarsi. Fecero amicizia con la famiglia di una lady inglese, con una contessa tedesca e con il figlio che era stato ferito nell'ultima guerra, con uno scienziato svedese e con mr. Canut e la sorella. Ma la compagnia degli Šcerbackij si compose soprattutto, involontariamente, di una signora di Mosca, Mar'ja Evgenevna Rtišceva, con la figlia (che non piaceva a Kitty perché si era ammalata di amore come lei), e di un colonnello moscovita che Kitty ricordava fin dall'infanzia, in divisa e spalline, ma che qui, con i suoi piccoli occhi e il collo scoperto, la cravattina a colori, era straordinariamente ridicolo; e noioso poi, perché non si riusciva a liberarsi di lui. Quando tutto si assestò in modo preciso, Kitty cominciò ad annoiarsi molto, ancor più perché il principe era partito per Karlsbad, ed ella era rimasta sola con la madre. Non si interessava a quelli che conosceva, perché sentiva che non ne avrebbe cavato nulla di nuovo. Il suo più grande e intimo interesse consisteva invece nell'osservare quelli che non conosceva, o nel fare supposizioni circa il loro carattere. Per una particolare inclinazione del suo carattere, Kitty supponeva sempre negli altri quanto può esserci di più bello, e soprattutto in coloro che non conosceva. E ora, fantasticando così intorno alle persone, ai rapporti che intercorrevano fra di loro ed alla loro appartenenza a una o all'altra categoria, Kitty si figurava i più belli e meravigliosi caratteri, e cercava riconferma alle sue supposizioni. 

Tra queste persone le interessava in modo particolare una ragazza russa, arrivata al luogo di cura con una signora russa ammalata, la signora Stahl, come la chiamavano tutti. La signora Stahl apparteneva al gran mondo, ma era così malata da non poter camminare, e la si vedeva alle acque soltanto in qualche rara bella giornata, portatavi in una carrozzina. Ma non tanto per la malattia, quanto per alterigia, così spiegava la principessa, la signora Stahl non trattava nessuno dei russi. 

La ragazza russa curava la signora Stahl e, oltre a ciò, Kitty aveva notato ch'ella andava d'accordo con tutti i malati gravi, che erano ben numerosi nella stazione termale, e si occupava di loro con grande semplicità. Questa ragazza, secondo le supposizioni di Kitty, non era parente della signora Stahl, ma non era nemmeno un'infermiera retribuita. La signora Stahl la chiamava Varen'ka e gli altri la chiamavano ``m.lle Varen'ka''. Kitty amava non solo fantasticare intorno ai rapporti fra questa ragazza e la signora Stahl e le altre persone a lei sconosciute, ma provava, come talvolta accade, un'istintiva simpatia per m.lle Varen'ka e sentiva, nei loro sguardi che s'incontravano, d'esserne ricambiata. M.lle Varen'ka, pur essendo certamente giovane, sembrava un essere senza giovinezza: le si potevano dare diciannove o trenta anni indifferentemente. Malgrado il suo colorito malato, a giudicare dai tratti del viso, era piuttosto bella che brutta. E poteva sembrar anche ben fatta se non vi fosse stata in lei un'eccessiva magrezza del corpo e una certa sproporzione tra la testa e la sua figura di media altezza; ma certamente non poteva piacere agli uomini. Somigliava a un bellissimo fiore ancor pieno di petali, ma già sfiorito, senza profumo. Non poteva piacere agli uomini, anche perché le mancava quello che abbondava in Kitty: un fuoco di vitalità contenuta e la coscienza del proprio fascino. 

Sembrava tutta raccolta in qualche cosa di cui fosse certa in modo assoluto e non potesse pertanto interessarsi a nulla che ne fosse al di fuori. Ciò contrastava con quello che era nell'animo di Kitty e attirava questa verso di lei. Kitty sentiva che nell'altra, nel suo modo di vivere, avrebbe trovato un esempio di quanto ora tormentosamente cercava: l'interesse alla vita, il valore della vita all'infuori e al di là delle relazioni mondane tra una ragazza e gli uomini, relazioni che erano ormai odiose a Kitty e che le apparivano come un'umiliante esposizione di merce in attesa del compratore. Quanto più Kitty osservava l'amica sconosciuta, tanto più si convinceva che questa ragazza era proprio l'essere perfetto ch'ella immaginava, e tanto più desiderava di conoscerla. 

Le due ragazze si incontravano varie volte al giorno, e ad ogni incontro gli occhi di Kitty dicevano: ``Chi siete? Cosa mai siete? È vero che siete quell'essere delizioso che io mi figuro? Ma, per amor di Dio, non pensate - aggiungeva il suo sguardo - che io mi permetta di imporvi la mia conoscenza. Vi ammiro semplicemente e vi voglio bene''. ``Io pure vi voglio bene e voi siete molto carina. E vi vorrei ancora più bene, se avessi tempo'' rispondeva lo sguardo della ragazza sconosciuta. E invero Kitty vedeva ch'ella era sempre occupata: accompagnava fuori i bambini di una famiglia russa, o portava uno scialle per la malata e ve l'avviluppava dentro, o cercava di distrarre un malato inasprito, sceglieva e comprava per qualcuno i pasticcini per il caffè. 

Ben presto, dopo l'arrivo degli Šcerbackij alla stazione termale, apparvero altri due personaggi che attirarono l'attenzione, poco benevola, di tutti. Erano: un uomo alto, un po' curvo, con delle mani enormi, un cappotto corto non fatto su misura, degli occhi neri ingenui e insieme terribili, e una donna graziosa, butterata, vestita male e senza gusto. Riconosciute queste persone per russi, Kitty aveva cominciato a comporre su di loro, nella sua immaginazione, un bellissimo e commovente romanzo. Ma la principessa, scoperto nella Kurliste che erano Levin Nikolaj e Mar'ja Nikolaevna, spiegò a Kitty quale pessimo soggetto fosse questo Levin, e tutti i sogni su questi due esseri scomparvero. Non tanto perché la madre glielo avesse detto, quanto per il fatto che si trattava del fratello di Konstantin, queste due persone parvero a Kitty molto antipatiche. Anzi questo Levin, con quella sua abitudine di scuotere il capo, suscitava addirittura in lei un senso di repulsione. 

Le sembrava che in quei due grandi occhi terribili che la seguivano ostinatamente, ci fosse un sentimento di odio e di irrisione, ed ella cercava di evitare un incontro con lui. 

\capitolo{XXXI}\label{xxxi-1} 

Era una brutta giornata, la pioggia era caduta per tutta la mattina e i malati si affollavano con gli ombrelli sotto il portico. 

Kitty passeggiava insieme con la madre e il colonnello moscovita, che faceva allegramente l'elegantone con il suo soprabito all'europea, comprato già bell'e fatto a Francoforte. Camminavano da un lato del porticato, cercando di evitare Levin che camminava nell'altro senso. Varen'ka con il suo abito scuro, il cappello nero dalla falda ripiegata in giù, accompagnava una francese cieca lungo tutto il porticato, e ogni volta che s'incontrava con Kitty, scambiava con lei uno sguardo di simpatia. 

- Mamma, posso rivolgerle la parola? - disse Kitty, seguendo con gli occhi l'amica sconosciuta che si avviava alla fonte dove avrebbero potuto incontrarsi. 

- Se lo desideri tanto, prenderò informazioni sul suo conto, e l'avvicinerò io stessa - rispose la madre. - Cosa ci trovi di particolare? Deve essere una dama di compagnia. Se vuoi farò conoscenza con la signora Stahl. Conosco la sua belle-soeur - aggiunse la principessa, sollevando con orgoglio il capo. 

Kitty sapeva che la principessa era offesa dal fatto che la signora Stahl sembrava evitare di fare la sua conoscenza. 

- È un incanto, com'è cara! - ella disse, guardando Varen'ka, nel momento in cui porgeva un bicchiere alla francese. - Guardate come in lei tutto è schietto e grazioso. 

- Mi fanno ridere i tuoi engouements - disse la principessa. - No, torniamo indietro piuttosto - aggiunse poi, avendo notato Levin che moveva loro incontro con la sua donna e con un medico tedesco al quale andava dicendo qualcosa ad alta voce, con irritazione. 

Si voltarono per tornare indietro, quando improvvisamente sentirono, non più un parlare ad alta voce, ma un gridare. Levin, fermatosi, urlava, ed anche il dottore si accalorava. La folla si riuniva intorno a loro. La principessa e Kitty si allontanarono in fretta, mentre il colonnello si unì alla folla per sapere di che si trattasse. Dopo qualche minuto il colonnello le raggiunse. 

- Che cosa è successo? - domandò la principessa. 

- Un'infamia, un'ignominia - rispose il colonnello. - Una cosa sola c'è da temere: incontrare dei russi all'estero. Quel signore alto ha leticato col dottore, gli ha detto un sacco di insolenze perché non lo cura come si deve e ha levato il bastone su di lui. È proprio un'ignominia! 

- Ah, che cosa spiacevole! - disse la principessa. - E come è andata a finire? 

- Grazie a quella lì\ldots{} ci si è messa in mezzo, quella lì\ldots{} quella col cappello a fungo. Una russa, mi pare - disse il colonnello. 

- M.lle Varen'ka? - chiese Kitty con gioia. 

- Sì, sì. S'è trovata prima di tutti; ha preso quel signore sotto braccio e l'ha portato via. 

- Ecco, mamma - disse Kitty alla madre - voi vi meravigliate che io mi entusiasmi per lei! 

Fin dal giorno seguente, osservando la sua amica sconosciuta, Kitty notò che m.lle Varen'ka anche con Levin e la sua compagna usava già quei rapporti che usava con gli altri suoi protégés. Si avvicinava loro, conversava, faceva da interprete alla donna che non parlava nessuna lingua straniera. 

Kitty cominciò a supplicare ancora di più la madre perché le permettesse di conoscere Varen'ka. E per quanto dispiacesse alla principessa di fare, per così dire, il primo passo verso la signora Stahl, che si permetteva di essere orgogliosa di chi sa che cosa, ella assunse informazioni su Varen'ka. Ottenutele e concluso che non c'era nulla di male, pur non essendovi nulla di buono, in questa conoscenza, si avvicinò ella stessa per prima a Varen'ka, e si presentò. 

Nel momento in cui la figlia era andata alla fonte e Varen'ka era ferma dinanzi ad una panetteria, la principessa le si avvicinò. 

- Permettetemi di fare la vostra conoscenza - disse con il suo sorriso sostenuto. - Mia figlia è innamorata di voi. Voi forse non mi conoscete. Io\ldots{} 

- È una simpatia più che scambievole, principessa - rispose in fretta Varen'ka. 

- Che buona azione avete fatto ieri verso quel nostro povero compatriota! - disse la principessa. 

Varen'ka arrossì. 

- Non so, mi pare di non aver fatto nulla - ella disse. 

- Come! Avete salvato quel Levin da un incidente increscioso. 

- Sì, sa compagne mi ha chiamato ed io ho cercato di calmarlo: è molto malato e non è contento del dottore. Ma io sono avvezza a curare questi malati. 

- Sì, ho sentito che vivete a Mentone con vostra zia, mi pare, m.me Stahl. Conosco la sua belle-soeur. 

- No, non è mia zia. La chiamo maman, ma non le sono parente. Sono stata allevata da lei - rispose Varen'ka, arrossendo di nuovo. 

La cosa era stata detta con tanta semplicità, ed era così piena di grazia l'espressione sincera e aperta del suo viso, che la principessa capì perché Kitty avesse preso a voler bene a questa Varen'ka. 

- E ora che cosa fa qui quel Levin? - chiese la principessa. 

- Se ne parte - rispose Varen'ka. 

In quel momento, tornando dalla fonte, raggiante di gioia perché sua madre aveva fatto la conoscenza con l'amica sconosciuta, Kitty si avvicinò. 

- Ebbene, ecco Kitty, il tuo gran desiderio di far la conoscenza con m.lle\ldots{} 

- Varen'ka - suggerì, sorridendo, Varen'ka - mi chiamano tutti così. 

Kitty arrossì di gioia e tenne stretta a lungo, tacendo, la mano della nuova amica, che non rispondeva alla sua stretta, ma rimaneva immobile nella mano di lei. La mano non rispondeva alla stretta, ma il viso di m.lle Varen'ka si illuminò di un sorriso tranquillo, dolce anche se un po' triste, che scopriva i denti grandi, ma belli. 

- Anch'io lo desideravo da tempo - ella disse. 

- Ma voi siete così occupata\ldots{} 

- Oh, al contrario, non sono per nulla occupata - rispose Varen'ka, e intanto, proprio in quel momento, dovette lasciare le nuove conoscenti, perché due bambine russe, figlie di un malato, correvano verso di lei. 

- Varen'ka, la mamma chiama! - gridavano. 

E Varen'ka se ne andò con loro. 

\capitolo{XXXII}\label{xxxii-1} 

I particolari che la principessa era venuta a sapere sul passato di Varen'ka e sui suoi rapporti con la signora Stahl erano i seguenti. 

La signora Stahl, della quale alcuni dicevano che aveva tormentato il marito, e altri che costui aveva tormentato lei con la sua condotta immorale, era una donna eternamente ammalata ed esaltata. Il suo primo bambino, nato dopo il divorzio, era morto appena venuto al mondo, e i parenti della signora Stahl, conoscendo la sua sensibilità e temendo che questa notizia potesse ucciderla, sostituirono il bambino con la figlia del cuoco di corte, nata in quella stessa notte e nella stessa casa, a Pietroburgo. Era questa Varen'ka. La signora Stahl, in seguito, aveva saputo che Varen'ka non era sua figlia, ma aveva continuato ad allevarla, tanto più che, non molto dopo, Varen'ka era rimasta orfana di padre e di madre. 

La signora Stahl viveva da più di dieci anni continuamente all'estero, al sud, senza mai alzarsi dal letto. Alcuni dicevano che la signora Stahl si era creata in società la fama di donna virtuosa, profondamente religiosa; altri dicevano ch'ella era tale nell'anima quale appariva: un essere altamente morale che viveva solo per il bene del prossimo. Nessuno sapeva di quale religione fosse: cattolica, protestante o ortodossa; ma una cosa era fuor di dubbio: che era in relazioni amichevoli con i personaggi più alti di tutte le chiese e di tutte le confessioni. 

Varen'ka viveva sempre con lei all'estero e tutti quelli che conoscevano la signora Stahl conoscevano e amavano m.lle Varen'ka, così come era chiamata. 

Conosciuti questi particolari, la principessa non trovò nulla di riprovevole nel fare avvicinare la propria figliuola a Varen'ka, tanto più che Varen'ka aveva modi di educazione eccellenti: parlava perfettamente il francese e l'inglese; e poi, e ciò contava più di tutto, ella aveva riferito il rammarico della signora Stahl di essere privata, a causa della sua malattia, del piacere di fare la conoscenza della principessa. 

Conosciuta Varen'ka, Kitty ne fu sempre più entusiasta, e ogni giorno scopriva in lei nuove qualità. 

La principessa, avendo saputo che Varen'ka aveva una bella voce, la pregò di venire a cantare da loro una sera. 

- Kitty suona\ldots{} non abbiamo un buon pianoforte, è vero, ma voi ci farete un gran piacere - disse la principessa con il suo sorriso di occasione che ora spiaceva in modo particolare a Kitty perché aveva notato come Varen'ka non avesse nessuna voglia di cantare. Varen'ka, tuttavia, venne una sera e portò con sé gli spartiti. La principessa aveva invitato Mar'ja Evgenevna con la figlia e il colonnello. 

Varen'ka sembrava completamente indifferente al fatto che ci fossero persone a lei sconosciute, e subito si accostò al piano. Non sapeva accompagnarsi; ma leggeva benissimo le note con la voce. Kitty, che suonava bene, l'accompagnava. 

- Avete un talento straordinario - le disse la principessa dopo che Varen'ka ebbe cantato il primo pezzo. Mar'ja Evgenevna e la figlia la ringraziarono e si complimentarono. 

- Guardate un po' - disse il colonnello osservando dalla finestra - che pubblico s'è raccolto ad ascoltarvi. - Infatti sotto le finestre s'era raccolto un gruppo abbastanza folto. 

- Sono molto contenta che questo vi faccia piacere - rispose semplicemente Varen'ka. 

Kitty guardava l'amica con orgoglio. Era entusiasta dell'arte, della voce e del viso di lei, ma più di tutto era entusiasta del suo modo di fare, del fatto che Varen'ka, evidentemente, non dava alcun peso al proprio canto ed era del tutto indifferente alle lodi; pareva solo domandare se dovesse cantare ancora, o se bastasse. 

``Se fossi io - pensava Kitty fra sé - come andrei orgogliosa di questo! Come mi rallegrerei a guardar questa folla sotto le finestre! E a lei tutto è indifferente. La preoccupa solo il desiderio di non rifiutare e di far cosa gradita a maman. Che c'è mai in lei? Cos'è che le dà questa forza di rinunciare a tutto, di essere imperturbabilmente serena? Come vorrei sapere ciò e impararlo da lei!'' pensava Kitty guardando fisso quel viso calmo. La principessa pregò Varen'ka di cantare ancora e Varen'ka cantò un altro pezzo con eguale calma, con precisione ed accuratezza, stando in piedi accanto al piano e battendo il tempo su di esso con la sua mano magra e abbronzata. 

Fra gli spartiti, il pezzo che veniva dopo era una canzone italiana. Kitty ne accennò le prime battute e si voltò a guardare Varen'ka. 

- Saltiamola, questa - disse Varen'ka, arrossendo. 

Kitty fissò i suoi occhi turbati e interrogativi nel viso di Varen'ka. 

- Allora via, un'altra cosa - aggiunse in fretta, svolgendo i fogli e comprendendo subito che a questa canzone era legato un qualche ricordo. 

- No - rispose Varen'ka, poggiando la mano sulla musica e sorridendo - no, cantiamo questa - e cantò altrettanto bene, tranquilla e pacata come prima. 

Quando ebbe finito, tutti la ringraziarono ancora e uscirono a prendere il tè: Kitty e Varen'ka uscirono nel piccolo giardino che era accanto alla casa. 

- È vero che qualche vostro ricordo è legato a quella canzone? - disse Kitty. - Non me ne parlate - aggiunse in fretta, - ditemi soltanto se è vero. 

- Perché non dirlo? Io parlerò - disse semplicemente Varen'ka, e, senza aspettare la risposta, continuò: - Già, è un ricordo, ed è stato penoso un tempo. Ho amato un uomo e ho cantato per lui quella canzone. 

Kitty coi grandi occhi intenti taceva e guardava Varen'ka con tenerezza. 

- Lo amavo, anche lui mi amava, ma sua madre non volle, e lui ha sposato un'altra. Ora vive non lontano da noi, ed io lo vedo ogni tanto. Non pensavate che anch'io potevo avere una storia d'amore? - disse, e nel bel viso brillò appena appena quella fiammella che, Kitty lo sentiva, aveva dovuto, un tempo, illuminarla tutta. 

- Perché dovrei non pensarlo? Se fossi un uomo non avrei potuto amare nessun'altra dopo aver conosciuto voi. Non capisco, però, come egli abbia potuto dimenticare voi per compiacere sua madre e fare di voi un'infelice; non aveva cuore. 

- Oh, no, è un uomo molto buono, e io non sono infelice; al contrario, sono molto felice. Su, non canteremo più stasera? - aggiunse, dirigendosi verso casa. 

- Come siete buona, come siete buona! - esclamò Kitty e, fermatasi, la baciò. - Potessi assomigliarvi almeno un po'! 

- Perché mai dovreste assomigliare a qualcuno? Voi siete buona così come siete - disse Varen'ka, sorridendo col suo sorriso mite e stanco. 

- No, io non sono buona affatto. Su, ditemi\ldots{} Aspettate, sediamoci un po' - disse Kitty, facendola di nuovo sedere su di una panchina accanto a sé. - Ditemi, è possibile sentire come un'offesa il fatto che un uomo ha disdegnato il vostro amore, che non l'ha voluto? 

- Ma lui non l'ha disdegnato il mio amore; io credo che mi amasse, ma era un figlio sottomesso\ldots{} 

- Già, ma se lui si fosse comportato così non per volere della madre, ma per proprio volere? - disse Kitty, sentendo di aver rivelato il proprio segreto e che il suo viso, rosso di fiamma per la vergogna, la tradiva. 

- In tal caso egli avrebbe agito male, ed io avrei avuto pena di lui - rispose Varen'ka, comprendendo che ormai non si trattava più di lei, ma di Kitty. 

- E l'offesa? - disse Kitty. - L'offesa non si può dimenticare - diceva, ricordando quel suo sguardo all'ultimo ballo, mentre la musica taceva. 

- E in che cosa consiste quest'offesa? Che forse avete agito male voi? 

- Peggio che male, vergognosamente. 

Varen'ka scosse il capo e mise la mano su quella di Kitty. 

- Ma perché mai vergognosamente? - ella disse. - Non potevate certo dire a un uomo, cui voi eravate indifferente, che l'amavate? 

- S'intende che non l'ho detto; non ho detto neppure una parola, ma egli ha capito. No, no, ci sono degli sguardi, ci sono degli atteggiamenti!\ldots{} Vivessi cento anni, non potrò dimenticare. 

- Ebbene, allora? Non capisco. La questione è tutta in questo: se voi ora l'amate o no - disse Varen'ka, parlando chiaro. 

- Lo odio, e non riesco a perdonarmelo. 

- Ma che cosa dunque? 

- La vergogna mia, l'offesa ricevutane. 

- Ah, se tutti fossero sensibili come voi! - disse Varen'ka. - Non vi è una ragazza cui ciò non sia accaduto. Ma tutto questo è così poco importante! 

- E che cosa mai è importante? - chiese Kitty, guardando il viso di lei con curiosa attenzione. 

- Ah, molte cose sono importanti - disse sorridendo Varen'ka. 

- E che cosa mai? 

- Ah, molte cose sono più importanti - rispose Varen'ka, non sapendo cosa dire. Ma in quel momento dalla finestra si udì la voce della principessa: 

- Kitty, fa fresco! O prendi uno scialle o rientra in casa. 

- È vero, è ora - disse Varen'ka, alzandosi. - Devo ancora passare da m.me Berthe, me l'ha chiesto. 

Kitty le teneva la mano e con appassionata ansietà e preghiera le domandava con lo sguardo: ``Cos'è, cos'è mai questa cosa più importante di tutto che dà una simile pace? Voi la sapete, ditemela!''. Ma Varen'ka non capiva quello che le domandava lo sguardo di Kitty. Ricordava solo che quel giorno doveva ancora passare da m.me Berthe e che doveva arrivare in tempo a casa per il tè di maman, verso mezzanotte. Entrò nelle stanze, riunì la musica e, dopo aver salutato tutti, si preparò ad andar via. 

- Permettete che vi accompagni - disse il colonnello. 

- Ma certo; e come andar sola, di notte? - replicò la principessa. - Altrimenti vi farò accompagnare da Paraša. 

Kitty vedeva che Varen'ka tratteneva a stento un sorriso per questa convinzione che si dovesse accompagnarla. 

- No, io vado sempre da sola, e non mi accade mai nulla - disse, prendendo il cappello. E dopo aver baciato ancora una volta Kitty, e senza averle detto quale fosse la cosa importante, con passo svelto e con le carte sotto il braccio, scomparve nella penombra della notte estiva, portando con sé il segreto di quello che era importante e che le conferiva quella invidiabile, dignitosa pace. 

\capitolo{XXXIII}\label{xxxiii-1} 

Kitty aveva conosciuto anche la signora Stahl e questa conoscenza, unita all'amicizia di Varen'ka, non solo aveva avuto una grande influenza su di lei, ma l'aveva consolata della sua pena. Aveva trovato sollievo perché, grazie a questa conoscenza, le si era aperto nell'anima un mondo del tutto nuovo, che non aveva nulla di comune col suo passato, un mondo elevato, bellissimo, dall'alto del quale si poteva guardare con serenità al passato. Ebbe la rivelazione che oltre alla vita istintiva, alla quale ella si era finora abbandonata, esisteva anche una vita dello spirito. Questa vita era rivelata dalla religione, ma da una religione che non aveva nulla di comune con quella che Kitty praticava dall'infanzia e che tutta si esprimeva ed esauriva nell'assistere alla messa e ai vespri, nel recarsi alla ``Casa delle vedove'' dove si potevano incontrare dei conoscenti, e nello studiare a memoria col batjuška testi in slavo antico: quest'altra era una religione altissima, misteriosa, legata a una serie di pensieri e di sentimenti splendidi, in cui non solo si poteva credere, perché così era comandato, ma che si poteva amare. 

Kitty non apprese tutto ciò dalle parole. La signora Stahl parlava con Kitty come una bambina graziosa di cui ci si compiace quasi in ricordo della propria giovinezza, e soltanto una volta aveva detto che tutti i dolori umani traggono conforto soltanto dall'amore e dalla fede, e che nessun dolore è trascurato dalla compassione di Cristo per noi: ma subito aveva avviato il discorso su un altro argomento. Eppure Kitty in ogni movimento di lei, in ogni sua parola, in ogni suo sguardo che Kitty definiva celestiale, e in particolare in tutta la storia della vita di lei che ella conosceva attraverso Varen'ka, in tutto infine, riconosceva ``quello che è importante'', e che finora non aveva conosciuto. 

Ma per quanto elevato fosse il carattere della signora Stahl, per quanto commovente fosse tutta la sua storia, per quanto elevata e tenera la sua parola, Kitty notò un lei, e con disapprovazione, alcuni tratti che la sconcertarono. Aveva notato che, chiedendole dei suoi parenti, la signora Stahl aveva sorriso sprezzantemente, il che era contrario alla carità cristiana. Inoltre un giorno che aveva trovato da lei un prete cattolico aveva notato che la signora Stahl aveva tenuto con cura il viso nell'ombra del paralume e aveva sorriso in modo strano. Per quanto insignificanti, queste due osservazioni la sconcertarono ed ella dubitava ora della signora Stahl. In compenso Varen'ka, sola al mondo, senza parenti, senza amici, con la sua triste delusione nel cuore, Varen'ka che non desiderava nulla e di nulla si rammaricava, costituiva quella perfezione che Kitty soltanto in sogno aveva intravisto. Osservando Varen'ka aveva compreso che bastava solo dimenticare se stessi e amare gli altri per essere calmi, felici e sereni. Tale voleva essere Kitty. Avendo adesso chiaramente conosciuto quale fosse la cosa più importante, Kitty non si accontentò di ammirare, ma subito si diede con tutta l'anima a praticare questa nuova vita che le si era dischiusa. Seguendo i racconti di Varen'ka sull'attività della signora Stahl e di altre persone che ella nominava, Kitty si tracciò un piano di vita per l'avvenire. Dovunque avesse vissuto, ella avrebbe cercato, come Aline, la nipote della signora Stahl di cui Varen'ka parlava tanto, gli sventurati, li avrebbe aiutati per quanto possibile, avrebbe distribuito il Vangelo, lo avrebbe letto ai malati, ai delinquenti, ai moribondi. L'idea di leggere il Vangelo ai delinquenti, così come faceva Aline, tentava in modo particolare Kitty. Ma tutti questi erano segreti, dei quali Kitty non faceva parte né alla madre, né a Varen'ka. 

E, in attesa di poter eseguire su vasta scala i suoi piani, Kitty anche ora nella stazione termale, dove c'erano tanti malati e tanti disgraziati, imitando Varen'ka, trovò facile attuazione alle sue nuove direttive. 

Dapprima la principessa notò che Kitty si trovava sotto un forte influsso del suo engouement, così come lo chiamava lei, per la signora Stahl e in particolare per Varen'ka. Vedeva che Kitty, non solo imitava Varen'ka nella sua attività, ma senza accorgersene l'imitava anche nella maniera di camminare, di parlare e di battere le palpebre. Ma in seguito la principessa notò che nella figlia, a parte questo incantamento, si compiva una vera trasformazione spirituale. 

La principessa notava che Kitty, di sera, leggeva un Vangelo francese che le aveva regalato la signora Stahl, cosa che prima non faceva; sfuggiva le relazioni mondane e si accostava ai malati che erano sotto la protezione di Varen'ka, ed in particolare a una povera famiglia di un certo pittore, Petrov. Kitty evidentemente era orgogliosa di compiere i doveri di una suora di carità in questa famiglia. Tutto questo era bene e la principessa non trovava nulla da ridire, tanto più che la moglie di Petrov era una donna perfettamente a posto, e che la principessa reale, notando l'attività di Kitty, ne aveva fatto le lodi chiamandola l'angelo consolatore. Tutto questo sarebbe andato molto bene se non avesse raggiunto l'eccesso. E la principessa, vedendo che la figlia cadeva nell'eccesso, glielo faceva notare. 

- Il ne faut jamais rien outrer - le diceva. Ma la figlia non rispondeva nulla. In cuor suo pensava che non si può parlare di eccesso nell'attività cristiana. Quale eccesso poteva esserci in una dottrina che insegnava a porgere la guancia sinistra quando avessero percosso la destra, e a dar via la camicia, quando avessero tolto il mantello? Ma alla principessa questo eccesso non piaceva e ancor più le spiaceva il fatto che Kitty, ella lo sentiva, non le aprisse tutta l'anima sua. In realtà Kitty nascondeva alla madre le sue nuove visioni e i suoi sentimenti. Li nascondeva, non perché non stimasse o non amasse sua madre, ma solo perché era sua madre; li avrebbe svelati a chiunque anziché alla madre. 

- È un bel po' che Anna Pavlovna non è venuta da noi - disse un giorno la principessa a proposito della Petrova. - L'ho invitata; ma mi pare offesa. 

- No, non l'ho notato, maman - disse Kitty, avvampando. 

- È da molto che manchi da loro? 

- Pensiamo di fare domani una passeggiata in montagna - rispose Kitty. 

- Ebbene, andate - disse la principessa, notando la confusione apparsa sul viso della figlia e cercando di indovinarne la causa. 

Quel giorno stesso Varen'ka venne a pranzo e riferì che Anna Pavlovna aveva rinunciato ad andare l'indomani in montagna. E la principessa notò che Kitty era improvvisamente diventata rossa. 

- Kitty, non è mica successo qualcosa di spiacevole tra te e i Petrov? - chiese la principessa quando restarono sole. - Perché non ha più mandato le bambine da noi? 

Kitty rispose che nulla era successo fra di loro e che proprio non capiva perché Anna Pavlovna sembrasse scontenta di lei. Kitty aveva detto tutta la verità. Non conosceva le cause del cambiamento di Anna Pavlovna nei suoi riguardi, ma indovinava. Indovinava una tal cosa che non poteva dire alla madre, che non poteva dire nemmeno a se stessa. Era una di quelle cose che si intuiscono, ma che non si possono dire neanche a se stessi: tanto è terribile e vergognoso lo sbagliarsi. 

Riesaminò ancora una volta nel ricordo tutti i suoi rapporti con quella famiglia. Ricordò la gioia ingenua che si esprimeva sul viso tondo, bonario di Anna Pavlovna nei loro incontri; ricordò i loro discorsi segreti a proposito del malato, le congiure per distrarlo dal lavoro che gli era stato proibito, e per portarlo a passeggio; l'attaccamento del bambino più piccolo che la chiamava ``la mia Kitty'' e che non voleva andare a letto senza di lei. Come tutto era bello! Poi ricordò la figura magra di Petrov, il suo collo lungo, il soprabito marrone, i radi capelli ondulati, gli occhi azzurri che sembravano interrogare e che impressionavano Kitty nei primi tempi, e gli sforzi morbosi di lui per sembrare valido e vivace in sua presenza. Ricordò il proprio sforzo per vincere nei primi tempi la ripugnanza che provava per lui come per tutti i tisici, e lo sforzo per escogitare cosa dirgli. Ricordò quello sguardo timido, commosso col quale egli la guardava, e lo strano senso di compassione e di imbarazzo, seguìto alla coscienza della propria virtù, ch'ella provava in quel momento. Come tutto ciò era bello! Ma tutto questo era accaduto nei primi tempi. Ora invece, da alcuni giorni, tutto si era improvvisamente sciupato. Anna Pavlovna l'accoglieva con una cortesia finta e non cessava d'osservare lei e il marito. 

Possibile che quella commovente gioia di lui al suo avvicinarsi fosse la causa del raffreddamento di Anna Pavlovna? 

``Sì - ricordava - c'era qualcosa di poco naturale in Anna Pavlovna, del tutto diverso dalla sua bontà, quando l'altro giorno ha detto con rancore: `Ecco, tutto per aspettare voi, non ha voluto prendere il caffè senza di voi, pur essendo spaventosamente debole'\,''. 

``Sì, forse le è spiaciuto anche quando gli ho dato lo scialle. Tutto questo è così semplice, ma lui l'ha accolto con tanto impaccio, ha ringraziato così a lungo che io ero a disagio. E quel mio ritratto che ha dipinto così bene! E poi ancora, soprattutto, quello sguardo, confuso e tenero! Sì, sì, è così! - si ripeteva con orrore. - No, questo non può, non deve essere! Fa tanta pena!'' diceva a se stessa subito dopo. 

E questo dubbio le avvelenava l'incanto della nuova vita. 

\capitolo{XXXIV}\label{xxxiv-1} 

Prima della chiusura della stagione termale, il principe Šcerbackij che, dopo Karlsbad, era stato a Baden e Kissingen, da conoscenti russi per fare, come egli diceva, provvista di spirito russo, tornò dai suoi. 

Le opinioni del principe e della principessa sulla vita all'estero erano completamente opposte. La principessa trovava tutto bellissimo e, malgrado la sua salda posizione nella società russa, all'estero faceva di tutto per sembrare una dama europea, quale non era, dal momento che era una vera signora russa, e in questo suo voler essere diversa da quello che era, si sentiva un po' a disagio. Il principe, al contrario, all'estero criticava tutto, si sentiva oppresso dalla vita europea, conservava le sue abitudini russe, sforzandosi di mostrarsi all'estero meno europeo di quanto non lo fosse in realtà. 

Il principe era tornato dimagrito, con le borse sotto gli occhi, ma di ottimo umore. E questo suo buon umore aumentò quando vide Kitty completamente ristabilita. La notizia dell'amicizia di Kitty con la signora Stahl e Varen'ka e le osservazioni della principessa su di un certo cambiamento prodottosi in Kitty, sconcertarono il principe e ridestarono in lui il solito senso di gelosia verso tutto quello che appassionava la figlia a sua insaputa, e la paura che la figlia sfuggisse alla sua influenza, rifugiandosi in qualche regione a lui inaccessibile. Ma queste notizie poco piacevoli affondarono in quel mare di bonarietà e di allegria che sempre era in lui e che la cura di Karlsbad aveva accresciuto. 

Il giorno dopo il suo arrivo, il principe, di ottimo umore, nel suo lungo cappotto, con le sue rughe tipicamente russe e le guance gonfie sostenute dal colletto inamidato, andò alla fonte in compagnia della figlia. 

La mattina era splendida: le case linde e allegre con i giardinetti, le cameriere tedesche dal viso rosso, dalle mani rosse, sature di birra e allegramente intente al lavoro, il sole gagliardo, rallegravano il cuore; ma più si avvicinavano alla fonte e più numerosi incontravano i malati, e il loro aspetto sembrava ancor più desolante sullo sfondo di vita tedesca solitamente ben organizzata. Questo contrasto non colpiva ormai più Kitty. Il sole splendente, l'allegro luccichio del verde, i suoni della musica erano per lei una cornice naturale di tutti quei visi ormai noti e dei loro mutamenti in peggio o in meglio ch'ella notava; ma al principe la luce e lo splendore di quella mattina di luglio, i suoni dell'orchestra che eseguiva un allegro valzer di moda e soprattutto la vista della rubiconde, robuste cameriere facevan l'effetto di cosa disadatta e innaturale ad accogliere quelle larve umane convenute da ogni parte d'Europa, lentamente deambulanti. 

Malgrado il senso d'orgoglio e quasi di rinnovata giovinezza ch'egli provava quando la figliuola preferita camminava al suo braccio, sentiva ora quasi un senso di disagio e di mortificazione per il proprio passo deciso, per le proprie membra robuste, ricoperte di carne. Provava la sensazione di un uomo che andasse svestito in società. 

- Presentami, presentami ai tuoi nuovi amici - chiedeva alla figliuola, premendole il braccio col gomito. - Ho finito col voler bene anche a questo tuo sudicio Soden che ti ha fatto rimettere così. Solo che è triste, triste qui da voi. Questo chi è? 

Kitty gli veniva nominando le persone conosciute e quelle non conosciute che incontrava. Proprio all'ingresso del giardino incontrarono m.me Berthe, la cieca, l'accompagnatrice, e il principe si rallegrò dell'espressione commossa della vecchia francese nel sentir la voce di Kitty. Ella subito si mise a parlar con lui, con quell'eccessiva cortesia francese, felicitandosi per la figliola così straordinaria e innalzando al cielo Kitty che chiamava tesoro, perla, angelo consolatore. 

- Via, però è sempre l'angelo numero due - disse il principe sorridendo. - Perché l'angelo numero uno è m.lle Varen'ka, a dir di mia figlia. 

- Oh, m.lle Varen'ka è un angelo del cielo, allez - replicò m.me Berthe. 

Sotto il portico incontrarono Varen'ka in persona. Veniva svelta incontro a loro, con un'elegante borsetta rossa. 

- Ecco, è arrivato anche papà! - le disse Kitty. 

Varen'ka fece con semplicità e naturalezza, come del resto faceva tutto, un movimento fra l'inchino e la riverenza, e cominciò subito a parlare col principe come parlava con tutti, in maniera semplice e spontanea. 

- Ma io vi conosco, naturalmente, e vi conosco da molto - le disse il principe con un sorriso dal quale Kitty capì con gioia che l'amica sua era piaciuta al padre. - Dove vi affrettate tanto? 

- Maman è qui - ella disse, volgendosi a Kitty. - Non ha dormito tutta la notte e il dottore le ha consigliato di uscire. Le porto il lavoro. 

- Così questo è l'angelo numero uno - disse il principe, quando Varen'ka se ne fu andata. 

Kitty vedeva ch'egli avrebbe voluto scherzare su Varen'ka, ma che non poteva riuscirci in nessun modo, perché Varen'ka gli era piaciuta. 

- Sì, ecco che vedremo tutti i tuoi amici - aggiunse - anche la signora Stahl, se mi concederà l'onore di riconoscermi. 

- Ma tu l'hai forse conosciuta, papà? - chiese Kitty con terrore, avendo notato un lampo di irrisione negli occhi del principe al ricordo della signora Stahl. 

- Conoscevo suo marito e lei, ancora prima che si iscrivesse fra le pietiste. 

- Che cosa vuol dire pietista, papà? - chiese Kitty, già spaventata del fatto che quello che ella apprezzava così altamente nella signora Stahl avesse un nome. 

- Neanche io lo so con precisione. So soltanto ch'ella ringrazia Dio di tutto; di ogni sventura, e anche della morte del marito ringrazia Iddio. Ebbene, questo fa ridere, perché loro due non andavano d'accordo. 

- Chi è quello là? Che viso da far pena! - chiese dopo aver notato un malato non alto, seduto su di una panchina, con un cappotto marrone e dei pantaloni bianchi che facevano delle strane pieghe sulle ossa scarnite delle gambe. 

Il signore sollevò il cappello di paglia sui radi capelli ondulati, scoprendo una fronte alta, arrossata dal cappello. 

- È Petrov, il pittore - rispose Kitty, arrossendo. - E questa è sua moglie - aggiunse indicando Anna Pavlovna la quale, come apposta, nel momento in cui essi si avvicinavano, si era messa a rincorrere un bambino scappato via per un viale. 

- Come fa pena, ma che viso simpatico che ha! - disse il principe. - Come mai non ti sei avvicinata? Non ti voleva forse dire qualcosa? 

- Su, via, andiamo! - disse Kitty voltandosi risoluta. - Come state oggi? - chiese a Petrov. 

Petrov si alzò, appoggiandosi al bastone e guardando timidamente il principe. 

- È mia figlia - disse il principe. - Permettetemi di fare la vostra conoscenza. 

Il pittore si inchinò e sorrise, scoprendo i denti bianchi straordinariamente lucidi. 

- Vi abbiamo aspettato ieri, principessina - disse egli a Kitty. 

Vacillò, dicendo questo, ma, ripetendo il movimento, si sforzava di far parere che l'avesse fatto apposta. 

- Io volevo venire ma Anna Pavlovna mi ha fatto sapere per mezzo di Varen'ka che non sareste andati. 

- Come non saremmo andati! - disse Petrov, arrossendo e tossendo subito, cercando con gli occhi la moglie. - Aneta! Aneta! - chiamò con voce aspra e sul collo bianco si tesero, come corde, le grosse vene. 

Anna Pavlovna si avvicinò. 

- Come mai hai mandato a dire alla principessina che non saremmo andati? - mormorò irritato, già senza voce. 

- Buon giorno, principessina - disse Anna Pavlovna, con un sorriso finto, affatto dissimile dalle sue maniere d'una volta. - Piacere di conoscervi - disse rivolta al principe. - Vi aspettavamo da lungo tempo, principe. 

- Come mai hai mandato a dire alla principessina che non saremmo andati? - mormorò rauco, una seconda volta, il pittore ancor più irritato, perché la voce gli veniva a mancare e non riusciva a dare alle parole l'intonazione che avrebbe voluto. 

- Ah, Dio mio! Pensavo che non saremmo andati - rispose la moglie con dispetto. 

- Ma, come se\ldots{} - e cominciò a tossire e a far un gesto con la mano. 

Il principe sollevò il cappello e si allontanò con la figlia. 

- Oh, oh - sospirò penosamente; - oh, che disgraziati! 

- Sì, papà - ripose Kitty. - E devi sapere che hanno tre bambini, e sono senza donna di servizio e quasi senza mezzi. Egli riceve qualcosa dall'Accademia - raccontò vivacemente Kitty sforzandosi di soffocare l'agitazione dalla quale era stata presa per lo strano mutamento di Anna Pavlovna nei suoi riguardi. 

- Ed ecco anche la signora Stahl! - disse Kitty, indicando una carrozzina nella quale, avvolta fra i cuscini e in un groviglio grigio-azzurro, sotto un ombrellino, giaceva una certa cosa. 

Era la signora Stahl! Dietro di lei stava dritto un robusto lavoratore tedesco dall'aria burbera che la trasportava. Accanto veniva un biondo conte svedese che Kitty conosceva di nome. Alcuni malati si fermarono attorno alla carrozzina, guardando questa signora come una cosa rara. 

Il principe si avvicinò. E subito negli occhi di lui Kitty notò la piccola luce di irrisione che l'aveva sconcertata. Si avvicinò alla signora Stahl e cominciò a parlare in quell'ottimo francese che ormai così pochi parlano, straordinariamente cortese e gentile. 

- Non so se vi ricordate di me, ma io devo richiamarmi alla vostra memoria per ringraziarvi della bontà usata verso la mia figliuola - egli disse, dopo essersi tolto il cappello e senza rimetterlo. 

- Il principe Aleksandr Šcerbackij - disse la signora Stahl alzando su di lui i suoi occhi celesti, nei quali Kitty notò lo scontento. - Molto lieta. Io voglio molto bene alla vostra figliuola. 

- La vostra salute è sempre poco buona? 

- Sì, ormai mi ci sono abituata - disse la signora Stahl e presentò al principe il conte svedese. 

-Ma voi siete molto poco cambiata - disse il principe. - Io non ho avuto l'onore di vedervi da dieci o undici anni. 

- Sì, Dio dà la croce e Dio dà la forza per portarla. Spesso ci si meraviglia perché si prolunga questa vita\ldots{} Dall'altra parte! - disse con stizza a Varen'ka che le avvolgeva le gambe nello scialle non precisamente come voleva lei. 

- Per far del bene, probabilmente - disse il principe, ridendo con gli occhi. 

- Questo non spetta a noi giudicare - disse la signora Stahl, che aveva colto la sfumatura di irrisione nel viso del principe. - Così voi mi manderete questo libro, caro conte? Vi ringrazio molto - disse rivolta al giovane svedese. 

- Ah - esclamò il principe, vedendo il colonnello di Mosca che era in piedi lì accanto e, salutata la signora Stahl, si allontanò con la figlia e con il colonnello moscovita che si era unito a loro. 

- Questa è la nostra aristocrazia, principe - disse, cercando d'essere ironico, il colonnello moscovita, che ce l'aveva con la signora Stahl perché non aveva fatto amicizia con lui. 

- Sempre la stessa - rispose il principe. 

- Ma voi l'avete conosciuta ancora prima della sua malattia, cioè prima che si fosse messa a letto? 

- Già, s'è messa a letto quando già la conoscevo. 

- Dicono che non si alzi da dieci anni. 

- Non si alza perché ha una gamba più corta dell'altra. È fatta molto male\ldots{} 

- Papà, ma non può essere! - gridò Kitty. 

- Le cattive lingue dicono così, figlia mia. E la tua Varen'ka deve saperne abbastanza - aggiunse. - Oh queste signore malate! 

- Oh, no, papà! - ribatté Kitty con calore. - Varen'ka l'adora. E poi è una donna che fa tanto bene. Domanda a chi vuoi. Lei ed Aline Stahl sono conosciute da tutti. 

- Può darsi - disse egli, stringendole il braccio col gomito. - Ma vale di più quando si fa in modo che, a chiunque si chieda, nessuno lo sappia. 

Kitty tacque, non perché non avesse nulla da ribattere, ma perché non voleva svelare neanche al padre i suoi segreti pensieri. Però, cosa strana, pur preparandosi a non sottostare all'introspezione del padre, a non dargli accesso nel suo santuario, sentì che quella immagine sublime della signora Stahl, che per un mese intero aveva portato nell'anima, era irrimediabilmente scomparsa, così come scompare la figura formata da un vestito abbandonato, quando ci si accorge che è solo un vestito. Era rimasta ormai una donna con una gamba più corta dell'altra che stava a letto perché era fatta male e tormentava la docile Varen'ka perché non ravvolgeva lo scialle così come andava fatto. E ormai nessuno sforzo dell'immaginazione poteva far rivivere la signora Stahl di prima. 

\capitolo{XXXV}\label{xxxv} 

Il principe aveva trasmesso il suo buon umore ai familiari e agli amici e persino all'albergatore tedesco presso il quale stavano gli Šcerbackij . 

Tornando dalla fonte con Kitty e invitati per il caffè il colonnello, Mar'ja Evgenevna e Varen'ka, il principe ordinò di portare il tavolo e le poltrone nel giardino, sotto il castagno, e di apparecchiare là per la colazione. L'albergatore e la servitù si rianimarono per effetto del suo buon umore. Essi conoscevano la sua liberalità; mezz'ora dopo un dottore d'Amburgo, ammalato, che era alloggiato al piano superiore, guardava con invidia dalla finestra quell'allegra brigata di russi, formata di persone sane, raccolta sotto il castagno. All'ombra tremula, in cerchi, delle foglie, vicino a una tavola coperta da una tovaglia bianca e cosparsa di caffettiere, pane, burro, formaggio, selvaggina fredda, sedeva la principessa con un'acconciatura ornata di nastri lilla, che distribuiva tazze e tartine. All'altra estremità sedeva il principe che mangiava abbondantemente e discorreva a voce alta, allegra. Aveva disposto accanto a sé le compere fatte in grande quantità nei vari luoghi di cura: cofanetti scolpiti, gingilli, coltellini intagliati d'ogni specie, e li andava regalando a tutti, compresa Lischen, la cameriera, e l'albergatore, col quale scherzava in quel suo comico, pessimo tedesco, assicurandolo che non erano le acque che avevano guarito Kitty, ma la sua ottima cucina, in particolare la zuppa con le prugne secche. La principessa prendeva in giro il marito per le sue abitudini russe, ma era così vivace e allegra come non lo era mai stata in tutto il suo soggiorno nel luogo di cura. Il colonnello, come sempre, sorrideva agli scherzi del principe; ma in quanto all'Europa, che egli credeva di aver studiato a fondo, teneva dalla parte della principessa. La buona Mar'ja Evgenevna scoppiava a ridere a ogni facezia che diceva il principe, e perfino Varen'ka, cosa che Kitty non aveva notato mai, si sfiniva in un debole, ma contagioso riso suscitatole dagli scherzi del principe. 

Tutto questo rallegrava Kitty, ma ella non riusciva a superare le sue preoccupazioni. Non poteva risolvere il problema che involontariamente le aveva posto il padre con la propria scherzosa opinione sui suoi amici e su quella vita che ella tanto aveva preso ad amare. A questo problema si aggiungeva inoltre il mutamento dei suoi rapporti coi Petrov che quel giorno si era rivelato così evidente e spiacevole. Tutti erano allegri, ma Kitty non poteva esserlo, e questo ancor più la tormentava. Provava una sensazione simile a quella che aveva provato nell'infanzia quando, chiusa in castigo in camera sua, sentiva il riso allegro delle sorelle. 

- Ebbene, perché l'hai comprata tutta questa roba? - diceva la principessa, sorridendo e porgendo al marito una tazza di caffè 

- Che vuoi fare? Vai a passeggio, ti avvicini a una botteguccia, ti pregano di comprare: ``Erlaucht Excellenz, Durchlaucht''. Ecco, quando hanno detto Durchlaucht, io non resisto più, ed ecco, dieci talleri sono andati via. 

- Così, solo per sfuggire alla noia - disse la principessa. 

- Si sa, per la noia. Una noia tale, moglie mia, che non sai dove batter la testa. 

- Ma come ci si può annoiare, principe? Ci sono tante cose interessanti, ora, in Germania - disse Mar'ja Evgenevna. 

- Sì, lo so tutto quello che c'è d'interessante: la zuppa con le prugne secche, lo so, le salsicce coi piselli, lo so. 

- Ma no, vi prego, principe, le loro istituzioni sono interessanti - disse il colonnello. 

- Che c'è di interessante? Sono tutti contenti come tanti soldoni di rame; hanno vinto tutti gli altri. Be', e io perché dovrei essere contento? Io non ho vinto nessuno; e là anche gli stivali te li devi togliere da solo e poi metterli dietro la porta. La mattina alzati, vestiti subito, vai nel salone a bere un pessimo tè. Ben altra cosa a casa! Ti svegli senza fretta, t'arrabbi contro qualcosa, brontoli un po', ritorni in te per benino, rifletti a tutto, non ti affanni. 

- Ma il tempo è denaro, voi dimenticate ciò - disse il colonnello. 

- Ma che tempo e tempo! A volte è tale, che dareste via tutto un mese per mezzo rublo, e altre volte non c'è denaro bastante per una mezz'ora. È così, Katen'ka? Che hai, così triste? 

- Io, nulla. 

- Ma dove andate? Restate ancora un po' - disse rivolto a Varen'ka. 

- Devo andare a casa - disse Varen'ka, alzandosi e scoppiando di nuovo a ridere. 

Ricompostasi, salutò ed entrò a prendere il cappello. Kitty la seguì. Perfino Varen'ka pareva ora un'altra. Non era peggiore, ma era un'altra da quella ch'ella aveva immaginato. 

- Ah, da tempo non ridevo così - disse Varen'ka, raccogliendo ombrellino e borsa. - Com'è simpatico il vostro papà! 

Kitty taceva. 

- Quando ci vediamo? - chiese Varen'ka. 

- Maman voleva passare dai Petrov. Voi non sarete là? - disse Kitty, mettendo Varen'ka alla prova. 

- Sì, ci sarò - rispose Varen'ka. - Si preparano a partire e io ho promesso di aiutare a fare le valigie. 

- Su, verrò anch'io. 

- No, che ve ne importa? 

- Perché, perché, perché? - si mise a dire Kitty, dilatando gli occhi e afferrando l'ombrellino per non lasciare andar via Varen'ka. - No, aspettate, perché? 

- Ma dicevo così; è arrivato vostro padre, e poi hanno soggezione di voi. 

- No, ditemi perché non volete che io vada spesso dai Petrov. Voi non volete, dunque? Perché? 

- Io non ho detto questo - disse tranquilla Varen'ka. 

- No, vi prego, ditelo! 

- Devo dir tutto? - chiese Varen'ka. 

- Tutto, tutto! - replicò Kitty. 

- Ma non c'è nulla di particolare, c'è solo questo, che Michail Alekseevic - così si chiamava il pittore - prima voleva partir subito, e ora non vuole più partire - disse Varen'ka, sorridendo. 

- Ebbene, ebbene - sollecitava Kitty, guardando torva Varen'ka. 

- Ebbene, chi sa perché Anna Pavlovna ha detto che egli non vuole partire perché voi siete qui. Certo era inopportuno dir questo, ma a causa di questo, a causa vostra, ne è venuto fuori un litigio. E voi sapete come questi malati siano irritabili. 

Kitty, accigliatasi sempre più, taceva e Varen'ka parlava da sola cercando di placarla e di calmarla, prevedendo la crisi che si andava preparando, non sapeva bene se di lacrime o di parole. 

- Così è meglio che non andiate\ldots{} Dovete capire, e non offendervi. 

- E mi sta bene e mi sta bene - cominciò a dire in fretta Kitty, afferrando l'ombrellino dalle mani di Varen'ka e guardando al di là degli occhi dell'amica. 

Varen'ka voleva sorridere, vedendo l'arrabbiatura da bimba dell'amica, ma temeva di offenderla. 

- Come, vi sta bene? Non capisco - disse. 

- Mi sta bene perché tutto questo era una finzione, perché tutto questo è artificioso, e non viene dal cuore. Che me ne importa a me di un estraneo! Ed ecco che per colpa mia è venuto fuori un litigio, perché ho fatto quello che nessuno mi ha chiesto di fare. Perché tutto è finzione, finzione, finzione! 

- Ma a quale scopo fingere? - disse piano Varen'ka. 

- Ah, che cosa brutta, stupida! Io non avevo alcun bisogno\ldots{} Tutto è finzione! - diceva, aprendo e chiudendo l'ombrellino. 

- Ma a quale scopo mai? 

- Per parer migliori agli occhi della gente, a se stessi, per ingannare tutti. No, adesso non mi sottometterò più a questo. Esser cattiva, sia pure, ma almeno bugiarda, falsa, no! 

- Ma chi mai è falsa? - disse Varen'ka con rimprovero. - Voi parlate come se\ldots{} 

Ma Kitty era tutta presa dall'ira. Non le dava modo di finir di parlare. 

- Non parlo di voi, non parlo affatto di voi, voi siete la perfezione. Sì, sì, io lo so che voi siete la perfezione; ma che fare, se io sono cattiva? Questo non sarebbe accaduto se io non fossi cattiva. Che io sia quale sono, ma non falsa. Che me ne importa di Anna Pavlovna? Che vivano pure come piace loro, e io come piace a me. Io non posso esser diversa\ldots{} E tutto questo non è quel che dovrebbe essere, non è! 

- Ma cosa mai non è quel che dovrebbe essere? - diceva Varen'ka perplessa. 

- Tutto non è come dovrebbe essere. Io non posso vivere altrimenti che secondo il cuore, e voi vivete secondo le regole. Io ho preso ad amarvi semplicemente, e voi, forse, solo per salvarmi e istruirmi! 

- Siete ingiusta! - disse Varen'ka. 

- Ma io non dico nulla degli altri, parlo di me. 

- Kitty - si udì la voce della madre, - vieni, mostra a papà i tuoi coralli. 

Kitty con aria sdegnosa, senza far pace con l'amica, prese dalla tavola i coralli nella scatolina e andò dalla madre. 

- Che ti è successo, che sei così rossa? - le dissero padre e madre a una voce. 

- Nulla - ella rispose - vengo subito - e corse via. 

``È ancora qui! - pensò. - Cosa le dirò, Dio mio! Che ho fatto, che ho detto! Perché l'ho offesa? Cosa fare? Cosa dirle?'' pensava Kitty, e si fermò presso la porta. 

Varen'ka col cappello e con l'ombrellino in mano sedeva vicino alla tavola, esaminando una molla che Kitty aveva spezzato. Ella alzò il capo. 

- Varen'ka, perdonatemi, perdonate! - sussurrò Kitty, avvicinandosi a lei. - Io non mi ricordo quello che ho detto. Io\ldots{} 

- Non volevo addolorarvi, proprio no - disse Varen'ka, sorridendo. 

La pace fu conclusa. Ma da quando era arrivato suo padre, tutto quel mondo in cui ella aveva vissuto le parve cambiato. Non rinnegò tutto quello che aveva ultimamente conosciuto, ma capì che ingannava se stessa, illudendosi di poter essere quello che voleva essere. Come se fosse tornata in sé, sentì tutta la difficoltà di mantenersi, senza finzione e senza vanteria, all'altezza alla quale aspirava; inoltre sentì tutto il peso di quel mondo di dolore, di malattie, di moribondi in cui viveva; le parvero tormentosi gli sforzi che faceva su di sé per amare tutto questo, e desiderò di andare al più presto via, all'aria fresca, in Russia, ad Ergušovo, dove, come aveva saputo da una lettera, era già andata Dolly coi bambini. 

Ma il suo amore per Varen'ka non si affievolì. Nel congedarsi, Kitty la pregò di venire da loro in Russia. 

- Verrò quando vi sposerete - disse Varen'ka. 

- Io non mi sposerò. 

- E allora non verrò mai. 

- E allora mi sposerò, soltanto perché possiate venire. Badate, dunque, di non dimenticare la promessa! - disse Kitty. 

Le previsioni del medico curante si erano avverate. Kitty ritornò a casa, in Russia, guarita. Non era più spensierata e allegra come una volta, ma era tranquilla. I suoi dolori di Mosca erano diventati un ricordo. 

\parte{PARTE TERZA}\label{parte-terza} 

\capitolo{I}\label{i-2} 

Sergej Ivanovic Koznyšev voleva prendersi un po' di riposo dal lavoro intellettuale e, invece di andarsene, come al solito, all'estero, verso la fine di maggio, si recò in campagna dal fratello. Secondo la sua convinzione, la vita di campagna era la migliore. Era quindi venuto dal fratello a godersela questa vita. Konstantin Levin ne fu molto contento; tanto più che per quell'estate non aspettava suo fratello Nikolaj. Ma, pur avendo stima ed affetto per Sergej Ivanovic, in campagna Konstantin Levin non si trovava a suo agio con lui. Non si sentiva a suo agio, e perfino gli spiaceva l'atteggiamento del fratello verso la vita di campagna. Per Konstantin Levin la campagna era un luogo di vita, cioè di gioia, di sofferenza e di lavoro; per Sergej Ivanovic la campagna era, da una parte, il riposo dal lavoro, dall'altra un utile controveleno alla corruzione, ch'egli prendeva con piacere, consapevole della sua efficacia. Per Konstantin Levin la campagna era tanto bella perché rappresentava il campo di azione per un lavoro indubbiamente utile; per Sergej Ivanovic la campagna era bella perché vi si poteva e vi si doveva restare oziosi. Inoltre anche l'atteggiamento di Sergej Ivanovic verso la gente di campagna offendeva un po' Konstantin Levin. Sergej Ivanovic diceva di amarla e di conoscerla, quella gente, e spesso se ne stava a discorrere con i contadini, cosa che faceva con garbo, senza infingimenti o affettazioni, e da ognuna di queste conversazioni ricavava dei dati generali in favore del popolo e a conferma della conoscenza che diceva di averne. Un simile atteggiamento non piaceva a Konstantin Levin. Per lui il contadino era solo il collaboratore primo al lavoro comune, e malgrado tutta la considerazione che gli accordava e un certo amore che aveva probabilmente succhiato, come egli stesso diceva, insieme al latte della balia contadina, tuttavia egli, come collaboratore al lavoro comune, pure estasiandosi talvolta dinanzi alla forza, all'umiltà, alla verità di quella gente, molto spesso, quando il lavoro comune richiedeva altre attitudini, inveiva contro il contadino per la sua trascurataggine e sporcizia, per la tendenza all'ubriachezza e l'abitudine a mentire. Se avessero chiesto a Konstantin Levin se amasse o no quella gente, egli invero non avrebbe saputo rispondere. L'amava e non l'amava, così come gli uomini in generale. Istintivamente di animo buono, era più incline ad amare anziché a non amare gli uomini, e così pure quella gente. Ma amarla o non amarla come qualcosa a sé, non poteva, perché non solo viveva con essa, non solo tutti i suoi interessi erano con essa collegati, ma riteneva di farne parte egli stesso, e non vedeva fra se stesso e quella gente nessuna differenza positiva o negativa, e perciò non poteva contrapporsi ad essa. Inoltre, pur vivendo da tempo nei più stretti rapporti coi contadini, e come padrone e come arbitro e soprattutto come consigliere (i contadini avevano fiducia in lui e venivano a lui per consiglio sin da quaranta verste all'intorno), non era riuscito a formarsene, peraltro, un concetto preciso, e si sarebbe trovato imbarazzato a rispondere alla domanda se li amasse oppure no. Dire di conoscere il contadino sarebbe stato per lui come dire di conoscere gli uomini. Conosceva e osservava continuamente uomini di ogni categoria e contadini, che considerava come gli uomini migliori e più interessanti, ma continuamente notava tratti nuovi per cui mutava i giudizi precedenti e ne formulava altri. Sergej Ivanovic, invece, aveva idee del tutto diverse. Come amava e lodava la vita di campagna, contrapponendola a quella che non amava, così pure amava la gente di campagna, contrapponendola a quella categoria di persone che egli non amava: considerava, dunque, il contadino qualcosa di diverso dagli uomini in genere. Nella sua mente ordinata si erano chiaramente fissate le forme definite della vita rurale, tratte, in parte, dalla stessa vita del contadino, ma in prevalenza da quella contrapposizione. Egli non cambiava mai la sua opinione e il suo atteggiamento di simpatia verso i contadini. 

Nella discussione fra i due fratelli sul giudizio sui contadini, Sergej Ivanovic vinceva sempre il fratello, proprio perché Sergej Ivanovic aveva idee precise sul contadino e sul suo carattere, sulle sue peculiarità e usanze; Konstantin Levin, invece, non aveva nessuna idea definita, così che in queste discussioni finiva per convincersi della propria incongruenza. 

Per Sergej Ivanovic il fratello minore era un buon ragazzo, dal cuore ben formato (così egli si esprimeva in francese), dalla mente sia pure abbastanza sveglia, ma influenzabile dalle impressioni del momento, e perciò piena di contraddizioni. Con la condiscendenza di fratello maggiore verso il minore, gli spiegava il senso delle cose, ma non trovava gusto a discutere con lui perché con troppa facilità lo metteva fuori combattimento. 

Konstantin Levin giudicava il fratello un uomo di straordinario ingegno e cultura, nobile nel più alto senso della parola e dotato della facoltà di agire per il bene generale. Ma in fondo all'anima sua, quanto più gli appariva grande e quanto più nell'intimo lo conosceva, tanto più spesso gli veniva in mente che questa facoltà di lavorare per il bene collettivo, della quale egli si sentiva assolutamente sprovvisto, poteva anche non essere un valore concreto, ma piuttosto l'indice dell'insufficienza di qualche cosa; non già di buoni, onesti e nobili propositi e aspirazioni, ma di slancio vitale, di quello che si chiamava ``cuore'', di quell'anelito che costringe l'uomo, fra le innumerevoli vie della vita che gli si parano davanti, a sceglierne una, e a questa sola dedicarsi. Quanto più conosceva il fratello tanto più notava che Sergej Ivanovic e molte altre persone che agivano per il bene comune, non erano stati portati dal cuore verso questo amore per la collettività, ma dal cervello che aveva giudicato esser bene occuparsene, e solo per questo se ne occupavano. Levin si confermò ancor più in questa supposizione nel notare che il fratello si interessava alle questioni sul bene comune o sull'immortalità dell'anima, così come si interessava a una partita a scacchi o al complicato congegno di una macchina nuova. 

Oltre a ciò Konstantin Levin non si trovava a suo agio, in campagna, col fratello, anche perché, specie d'estate, egli era continuamente occupato per l'azienda e non gli bastava neppure la lunga giornata estiva per compiere quanto era necessario, mentre Sergej Ivanovic era in ferie. Ma anche in vacanze, anche senza attendere, cioè, al proprio lavoro, egli era così abituato all'attività intellettuale, che amava esporre in bella e precisa forma le idee che gli venivano in mente, e amava che ci fosse qualcuno ad ascoltarle. E il suo più abituale e naturale ascoltatore era il fratello. Perciò, malgrado l'amichevole semplicità dei loro rapporti, Levin si sentiva imbarazzato a lasciarlo solo. Sergej Ivanovic amava sdraiarsi sull'erba al sole e rimanere a crogiolarsi e a chiacchierare oziosamente. 

- Tu non crederai - diceva al fratello - che piacere è per me quest'ozio degno di un chochol. Neppure un'idea nel cervello, neanche a cercarla col lumicino. 

Ma Konstantin Levin si angustiava a star lì seduto ad ascoltarlo, tanto più che sapeva che proprio in quel momento trasportavano, lui assente, il letame su di un campo non arato e, non sorvegliati, i contadini l'avrebbero ammucchiato Dio sa come; e i dentali negli aratri non li avrebbero svitati, ma strappati e dopo avrebbero detto che gli aratri sono una sciocca invenzione da non potersi paragonare con l'aratro di legno di mastro Andrej, e via di seguito. 

- Ma finiscila di andare su e giù con questo caldo - gli diceva Sergej Ivanovic. 

- No, devo fare una cosa in amministrazione, un attimo solo - diceva Levin e scappava verso i campi. 

\capitolo{II}\label{ii-2} 

Nei primi giorni di giugno accadde che Agaf'ja Michajlovna, la njanja e ora governante, portando in cantina un vasetto di funghi allora da lei salati, scivolò e cadde, slogandosi un braccio. Venne il medico condotto, un giovane chiacchierone che da poco aveva terminato gli studi universitari. Osservò il braccio, disse che non s'era affatto slogato, ordinò delle compresse e, rimasto a pranzo, ebbe il piacere di conversare con il famoso Sergej Ivanovic. Gli raccontò, per far mostra del proprio illuminato punto di vista, tutti i pettegolezzi del distretto, lamentando la cattiva condizione degli affari dell'amministrazione distrettuale. Sergej Ivanovic ascoltava attento, faceva delle domande e, eccitato dalla circostanza di avere un nuovo ascoltatore, prese a parlare ed esporre alcune sue giuste e ponderate osservazioni, apprezzate con deferenza dal giovane dottore, ponendosi così in quella lieta disposizione d'animo, nota al fratello, alla quale egli abitualmente perveniva dopo una conversazione brillante e vivace. Quando il dottore se ne fu andato, Sergej Ivanovic manifestò il desiderio di andare sul fiume a pescare con la lenza. Gli piaceva pescare con la lenza, ed era quasi orgoglioso di provar piacere in un'occupazione così sciocca. 

Konstantin Levin, che doveva andare a sorvegliare l'aratura e sui prati, si offrì di accompagnarlo in calesse. 

Si era al colmo dell'estate, quando il raccolto dell'annata in corso è già assicurato e cominciano le cure della semina per l'anno nuovo e si avvicina la fienagione; quando la segale grigioverde, tutta in spighe, ma non turgida, con la pannocchia ancora leggera, ondeggia al vento; quando le avene verdi, coi cespi d'erba gialla sparsa qua e là, spiccano fra le seminagioni tardive; quando il grano saraceno primaticcio già matura, ricoprendo il terreno; quando i maggesi, calpestati dal bestiame fino a diventar di pietra e coi viottoli rimasti intatti perché il vomere non li addenta, sono arati fino a metà; quando i mucchi del concio disseccato, all'aperto, odorano all'alba insieme alle erbe mielate, e quando sui pianori, simili a un mare ininterrotto, si distendono, in attesa della falce, i prati circondati dai mucchi nereggianti degli steli dell'acetosella estirpata. 

Era il tempo in cui nel lavoro dei campi subentra una breve pausa prima di iniziare il raccolto che ogni anno ridesta tutte le energie dei campagnoli. Il raccolto si presentava splendido e le giornate estive erano chiare, calde, con brevi notti rugiadose. 

I fratelli dovevano attraversare il bosco per giungere ai prati. Sergej Ivanovic lungo il percorso non si stancava di ammirare la bellezza del bosco soffocato dal fogliame, e mostrava al fratello ora un vecchio tiglio, scurito nella parte ombrosa, screziato di stipole gialle già pronte a fiorire, ora i giovani germogli verde smeraldo, rilucenti sugli alberi. Konstantin Levin non amava parlare, né sentir parlare della bellezza della natura. Le parole, per lui, toglievano l'incanto di quello che vedeva. Faceva eco al fratello, ma istintivamente pensava ad altro. Quando ebbero attraversato il bosco, tutta la sua attenzione fu attratta da un maggese su di una collina, ricoperto in un punto di chiazze gialle d'erba secca, in un altro battuto e tagliato a riquadri, in un altro ricoperto di mucchi di letame, e in un altro ancora perfino arato. Attraverso il campo andavano in fila dei carri. Levin li contò e fu contento pensando che così sarebbe stato portato via tutto quello che si doveva, e alla vista dei prati i suoi pensieri si rivolsero alla questione della falciatura. Quando si doveva provvedere alla raccolta del fieno, egli provava sempre qualcosa che lo toccava nel vivo. Accostandosi al prato, fermò il cavallo. 

C'era ancora guazza nel folto del prato e Sergej Ivanovic, per non bagnarsi i piedi, chiese d'esser portato in calesse fino al cespuglio di citiso presso cui si pescava il pesce persico. Per quanto dispiacesse a Konstantin Levin di calpestare l'erba, entrò nel prato. L'erba alta si avvinceva morbida intorno alle ruote del calesse e alle zampe del cavallo, lasciando i semi sui raggi bagnati e sui mozzi. 

Sergej Ivanovic, approntate le lenze, sedette sotto il cespuglio e Levin allontanò il cavallo, lo legò, ed entrò nell'immenso mare grigioverde del prato non mosso dal vento. L'erba, morbida come seta, coi semi maturi, gli arrivava fin quasi alla cintola nel luogo fecondato dalle piene. 

Attraversato di sghembo il prato, Konstantin Levin uscì sulla strada e incontrò un vecchio con un occhio gonfio che portava uno sciame di api. 

- Oh che, ne hai prese delle altre, Formic? - chiese. 

- Altro che prendere, Konstantin Dmitric! A stento ti restano le tue! Ecco che è scappata per la seconda volta la regina\ldots{} Grazie, i ragazzi sono arrivati di galoppo. Da voi arano. Hanno staccato il cavallo, sono arrivati di galoppo\ldots{} 

- Be', che ne dici, Formic, si deve falciare ora o aspettare ancora? 

- Macché! Da noi si deve aspettare fino al giorno di san Pietro. Voi invece falciate sempre prima. Ma se Dio vuole, le erbe son buone. Il bestiame ne avrà a sazietà. 

- E il tempo, come credi che sia? 

- Questo è affar di Dio. Può darsi che anche il tempo sia buono. 

Levin si avvicinò al fratello. Non un pesce abboccava, ma Sergej Ivanovic non s'annoiava, e sembrava nella più lieta disposizione di spirito. Levin si accorse che, eccitato dalla conversazione col dottore, avrebbe voluto parlare un po'; egli, invece, voleva tornarsene a casa a convocare i falciatori per l'indomani e risolvere la questione della falciatura che lo occupava tanto. 

- Be', andiamo - disse. 

- Affrettarsi per andar dove? Rimaniamo a sedere un po'. Anche senza pescar nulla, si sta bene qui. Ogni caccia è buona perché mette a contatto con la natura. Eh, che delizia quest'acqua d'acciaio! - egli disse. - I bordi di questi prati - continuò - mi ricordano sempre un vecchio indovinello, lo conosci? ``L'erba dice all'acqua: e noi ondeggeremo, ondeggeremo''. 

- No, non lo conosco - rispose Levin con tristezza. 

\capitolo{III}\label{iii-2} 

- E sai, ho pensato a te - disse Sergej Ivanovic. - Non c'è nulla di paragonabile a quello che avviene nel vostro distretto, a quanto dice quel dottore; ma non è mica sciocco quel giovane. E io ti ho detto e ti ripeto: non è bene che tu non vada alle riunioni e che in genere ti renda estraneo all'attività del consiglio distrettuale. Se le persone dabbene se ne allontanano, tutto andrà, s'intende, Dio sa come. Le tasse che si pagano, servono per gli stipendi, ma non vi sono scuole, né infermieri, né levatrici, né farmacie, non c'è nulla. 

- Ma io ho provato - rispose piano e svogliato Levin - non posso! Ebbene, che fare? 

- Che cosa non puoi? Io, confesso, non capisco. L'indifferenza, l'inesperienza, non le ammetto; possibile che sia solo pigrizia? 

- Né la prima, né la seconda, e nemmeno la terza. Ho provato e credo di non poterci far nulla - disse Levin. 

Egli non prestava attenzione a quello che diceva il fratello. Guardava l'aratura di là dal fiume, e scorgeva qualcosa di scuro senza riuscire a distinguere se fosse un cavallo o l'amministratore a cavallo. 

- Perché non puoi farci nulla? Hai fatto una prova e secondo te non è andata bene e ti rassegni. Ma com'è che non hai amor proprio? 

- L'amor proprio - disse Levin, punto nel vivo dalle parole del fratello - io non lo capisco. Se all'università mi avessero detto che gli altri capivano il calcolo integrale e io no, allora ci sarebbe entrato l'amor proprio. Ma qui bisogna prima esser convinti di avere delle speciali attitudini a queste cose e, quel che più conta, esser convinti che queste cose siano molto importanti. 

- Eh, già! Che forse tutto ciò non è importante? - disse Sergej Ivanovic, tocco nel vivo perché il fratello non trovava importante quel che interessava lui e perché, evidentemente, non lo ascoltava quasi. 

- Non mi sembra importante, non mi tocca, che vuoi mai?\ldots{} - rispose Levin mentre s'accorgeva che quel che vedeva era l'amministratore, e l'amministratore, probabilmente, aveva mandato via gli operai dall'aratura. Essi voltavano gli aratri. ``Possibile che abbiano già arato?'' pensò. 

- Su, ma ascolta - disse il fratello maggiore, corrugando il suo bel viso intelligente - vi sono dei limiti a tutto. È molto bello essere un originale e un uomo schietto e spregiare ogni falsità, questo lo so; ma ecco, quello che tu dici, o non ha senso, o ha un senso tutt'altro che buono. Come puoi trovare poco importante che questa popolazione che tu ami, come mi assicuri\ldots{} 

``Io non l'ho mai assicurato'' pensò Konstantin Levin. 

- \ldots{} muoia senza aiuti? Queste mammane fanno morir di fame i bambini e il popolo marcisce nell'ignoranza e rimane in potere di un qualsiasi scribacchino, mentre tu hai in mano i mezzi per riparare a questo, e non te ne dài pensiero perché, secondo te, la cosa non è importante. - E Sergej Ivanovic gli pose il dilemma: - O sei così poco evoluto da non riuscire a intravedere tutto quello che puoi fare, o non vuoi rinunciare alla tua tranquillità, alla tua vanità o che so io, per fare ciò. 

Konstantin Levin sentiva che non gli restava ormai che dichiararsi vinto e confessare la mancanza di interesse per una causa comune. E questo lo offendeva e lo addolorava. 

- E l'uno e l'altro - disse reciso - non vedo proprio come si possa\ldots{} 

- Come? Non si può, ripartendo bene il denaro, creare un'assistenza medica? 

- Non si può, a quanto pare. Per le quattromila verste quadrate del nostro distretto, con le nostre zazory, con le tempeste di neve, con la stagione dei lavori, non vedo la possibilità di dare in ogni luogo un'assistenza medica. E poi, in genere, io non credo alla medicina. 

- Via, permettimi, questo è ingiusto\ldots{} Io ti porterò migliaia di esempi\ldots{} via, e le scuole? 

- Perché le scuole? 

- Che dici? Può esservi mai dubbio sull'utilità delle scuole? Se la scuola è buona per te, lo è anche per gli altri. 

Konstantin Levin si sentiva moralmente messo con le spalle al muro e perciò si accalorava dando prova, senza volerlo, della sua indifferenza al benessere collettivo. 

- Può darsi che tutto questo vada bene; ma io, perché devo curarmi di istituire dei posti di assistenza medica di cui non farò mai uso, e delle scuole dove non manderò certo i miei figli, dove neanche i contadini vorranno mandare i loro e dove non credo ancora che proprio ci si debbano mandare? - disse. 

Questo modo inatteso di vedere la questione disorientò Sergej Ivanovic per un attimo; ma subito egli preparò un nuovo piano di attacco. 

Stette un po' di tempo in silenzio, tirò fuori un amo, lo gettò in acqua e, sorridendo, si volse al fratello. 

- Su, permettimi\ldots{} In primo luogo, il posto di assistenza medica è servito anche a te. Ecco, noi per Agaf'ja Michajlovna abbiamo mandato a chiamare il medico condotto. 

- Già ma io penso che il braccio resterà storto. 

- Questo è ancora da vedere\ldots{} Poi un contadino, un lavoratore istruito ti è più utile e più accetto. 

- No, domanda a chi vuoi - rispose deciso Konstantin Levin - uno che sappia leggere e scrivere, come lavoratore, è peggiore degli altri. E le strade non si possono fare aggiustare; e i ponti, appena messi a posto, li portano via. 

- Del resto - disse, aggrottando le sopracciglia Sergej Ivanovic, che non amava le contraddizioni e particolarmente quelle che saltavano continuamente di palo in frasca e senza alcuna connessione introducevano nella discussione elementi nuovi, così che non si poteva sapere a quali di essi rispondere - del resto non si tratta di questo. Permetti. Riconosci che l'istruzione è un bene per il popolo? 

- Lo riconosco - disse Levin senza riflettere, e subito pensò di non aver detto quello che pensava. Sentiva che, riconoscendo ciò, gli sarebbe stato dimostrato che diceva delle sciocchezze che non avevano alcun senso. Come questo gli sarebbe stato dimostrato, non lo sapeva, ma sapeva che, senza dubbio, gli sarebbe stato dimostrato, a fil di logica, e aspettava questa dimostrazione. 

La dimostrazione fu più semplice di quella che Levin si aspettava. 

- Se riconosci come un bene l'istruzione - disse Sergej Ivanovic - allora tu, come uomo onesto, non puoi non amare e non aderire a quest'opera e non desiderare di lavorare per essa. 

- Ma io ancora non la riconosco buona - disse arrossendo Levin. 

- Come? O ora hai detto di sì\ldots{} 

- Cioè, non la riconosco né buona né possibile. 

- Questo non lo puoi sapere, senza aver prima fatto tutti i tentativi. 

- Su, ammettiamo - disse Levin, sebbene non lo ammettesse per nulla - ammettiamo pure che sia così; ma io tuttavia non vedo la necessità di dovermi affannare per questo. 

- Sarebbe a dire? 

- No, giacché abbiamo preso a parlarne, spiegamelo dal lato filosofico - disse Levin. 

- Non capisco cosa c'entri qui la filosofia - disse Sergej Ivanovic, con un tono che a Levin parve tale da non volergli riconoscere il diritto di discutere di filosofia, e questo lo irritò. 

- Ecco come - disse, accalorandosi. - Io penso che il movente di tutte le nostre azioni sia l'interesse personale. Ora nelle istituzioni provinciali io, nella mia qualità di nobile, non ci vedo nulla che cooperi al mio benessere. Le strade non diventano migliori e, se pure rimangono quali sono, i miei cavalli mi portano anche per quelle cattive. Del dottore e del posto di assistenza medica non ho bisogno; il giudice conciliatore non mi occorre; io non mi rivolgo e non mi rivolgerò mai a lui. Le scuole non solo non mi occorrono, ma mi sono persino dannose, come ti ho detto. Per me le istituzioni distrettuali hanno il solo scopo di obbligarmi a pagare diciotto copeche per desjatina, e farmi andare in città a pernottare con le cimici e ascoltare ogni sorta di sciocchezze e brutture: ed in questo l'interesse personale non mi stimola affatto. 

- Permettimi - interruppe con un sorriso Sergej Ivanovic - l'interesse personale non ci stimolava a lavorare per la liberazione dei contadini, eppure noi abbiamo lavorato! 

- No - interruppe, sempre più accalorandosi, Konstantin. - La liberazione dei contadini era un'altra cosa. Lì c'era, sì, un interesse personale. Volevamo scrollar da noi questo giogo che opprimeva tutti noi uomini giusti. Ma essere delegato, discutere sulla quantità necessaria di cloache e sulla maniera di far passare le fogne in una città in cui non vivo; essere giurato e giudicare un contadino che ha rubato un prosciutto e ascoltare per sei ore di seguito tutte le sciocchezze che inventano i difensori e i procuratori e star lì a sentire come il presidente interroga il vecchio Alëška, lo scemo che sta da me: ``Confessate, voi, signor imputato, il furto del prosciutto?''. ``Eh?''. 

Konstantin Levin aveva già smarrito il filo del discorso e s'era messo a rifare il presidente e Alëška lo scemo, e gli pareva che tutto questo riguardasse la questione. 

Ma Sergej Ivanovic alzò le spalle. 

- Ebbene, con questo che vuoi dire? 

- Io voglio dire che quei diritti che mi\ldots{} che toccano il mio interesse personale, io li difenderò sempre con tutte le mie forze; che quando eravamo studenti e i gendarmi facevano le perquisizioni e leggevano le nostre lettere, io ero pronto con tutte le mie forze a difendere i miei diritti, a difendere il mio diritto alla libertà e alla cultura. Capisco il servizio militare perché interessa la sorte dei miei figli, dei miei fratelli e di me stesso; sono pronto a giudicare tutto quanto mi riguarda; ma giudicare se e come distribuire quarantamila rubli di denaro del distretto o giudicare Alëška lo scemo, io questo non lo capisco e non posso farlo. 

Konstantin Levin parlava come se si fosse rotta la diga che tratteneva la sua loquela; Sergej Ivanovic sorrideva. 

- E domani potrai essere giudicato tu stesso: ti piacerebbe forse essere giudicato dalla vecchia Camera criminale? 

- Io non sarò giudicato. Io non sgozzerò nessuno, e non ne avrò bisogno. Su via! - continuò, passando di nuovo a cosa che non riguardava affatto la questione - le nostre istituzioni distrettuali e tutto il resto somigliano alle piccole betulle che noi ficchiamo in terra dovunque il giorno di Pentecoste, perché sembrino un bosco venuto su spontaneamente in Europa; ma io non posso innaffiare e credere in queste piccole betulle con tutta l'anima. 

Sergej Ivanovic alzò le spalle, esprimendo con questo gesto la sua meraviglia per queste betulle spuntate ora nella questione chi sa mai da quale parte; mentre aveva capito subito a cosa volesse alludere il fratello. 

- Scusami, ma così non si può ragionare - osservò. 

Ma Konstantin Levin voleva giustificare quella manchevolezza che riconosceva in se stesso, l'indifferenza cioè verso il bene comune e continuò. 

- Io penso - disse che nessuna attività può essere salda se non ha le radici nell'interesse personale. Questa è una verità d'ordine generale, filosofico - disse, ripetendo con intenzione la parola ``filosofico'', quasi desiderasse mostrare che anche lui aveva il diritto, come tutti, di parlare di filosofia. 

Sergej Ivanovic ancora una volta sorrise. ``E anche lui - pensò - ha una certa filosofia al servizio delle proprie tendenze''. 

- Su via, la filosofia lasciala stare - disse. - Il compito della filosofia di tutti i secoli consiste proprio nel trovare il legame indispensabile fra l'interesse personale e quello generale. Ma questo non riguarda la questione, mentre, per quel che la concerne, io devo soltanto correggere il tuo paragone. Le betulle non sono conficcate, ma alcune sono piantate e altre seminate; e a queste ultime ci si deve rivolgere con maggior cura. Soltanto i popoli che guardano all'avvenire, soltanto quelli si possono chiamare storici, quelli che sentono ciò che è importante e significativo nelle loro istituzioni, e ne hanno cura. 

E Sergej Ivanovic trasportò la questione sul terreno storico-filosofico inaccessibile a Konstantin Levin, dimostrandogli tutta l'infondatezza del suo punto di vista. 

- Che questo poi non ti piaccia, questo, perdonami, fa parte della nostra pigrizia russa e del barstvo, e io sono sicuro che, quanto a te, si tratta di una deviazione momentanea che passerà. 

Konstantin taceva. Sentiva d'essere sconfitto da ogni lato, ma nello stesso tempo sentiva che quello che egli intendeva dire non era stato capito dal fratello, non sapeva bene perché: perché non aveva saputo esporlo lui chiaramente o perché il fratello non aveva voluto o non aveva potuto capirlo? Ma non stette a riflettere e, senza replicare, cominciò a pensare a una faccenda del tutto diversa, tutta sua personale. 

Sergej Ivanovic avvolse l'ultimo amo, slegò il cavallo e insieme si avviarono. 

\capitolo{IV}\label{iv-2} 

La faccenda personale che era venuta in mente a Levin durante la conversazione col fratello, era questa: l'anno precedente, recatosi un giorno ad assistere alla fienagione, e irritatosi con l'amministratore, aveva adoperato, per riconquistare la propria calma, il solito suo sistema: aveva tolto dalle mani di un contadino la falce e s'era messo a falciare. 

Questo lavoro gli era piaciuto tanto che diverse altre volte aveva falciato; aveva falciato tutto il prato davanti alla casa, e per questo, fin dalla primavera, si era proposto di falciare insieme con i contadini per giornate intere. Da quando era arrivato il fratello era in dubbio: falciare o no? Gli rincresceva lasciare il fratello solo per giornate intere, e poi temeva che non avesse a prendersi giuoco di lui per questo. Ma, camminando per il prato, ricordando le impressioni della falciatura, aveva deciso di falciare. Ora, dopo il colloquio irritante avuto col fratello, s'era nuovamente ricordato della decisione. 

``Ho bisogno di movimento fisico, altrimenti il mio carattere si guasta'' pensò e decise di andare a falciare, pur rincrescendogli di fronte al fratello e alla gente. 

La sera Konstantin passò in amministrazione, diede le disposizioni per i lavori e mandò in giro per i villaggi a convocare per l'indomani i falciatori per il prato Kalinovyj, il più grande e il migliore. 

- E la mia falce mandatela a Tit perché me l'affili e me la porti domani; forse falcerò anch'io - disse, cercando di non turbarsi. 

L'amministratore sorrise e disse: 

- Sissignore. 

La sera, al tè, Levin lo disse anche al fratello. 

- Sembra che il tempo si sia messo al bello. Domani comincio a falciare. 

- Mi piace molto questo lavoro - disse Sergej Ivanovic. 

- A me straordinariamente. Io stesso ho falciato qualche volta insieme con i contadini, e domani voglio falciare tutta la giornata. 

Sergej Ivanovic alzò la testa e guardò con curiosità il fratello. 

- E così, al pari dei contadini, tutta la giornata? 

- Sì, è una cosa piacevole - disse Levin. 

- È bellissimo come esercizio fisico, ma è difficile che tu possa farcela - disse Sergej Ivanovic, senza alcuna ironia. 

- Ho provato. In principio è duro, poi ci si abitua. Io penso che non resterò indietro\ldots{} 

- Ecco, ma di' un po', che ne pensano i contadini? Probabilmente rideranno della stramberia del signore. 

- No, non credo; ma è un lavoro così piacevole e nello stesso tempo così difficile che non si ha il tempo di pensare. 

- E così tu pranzerai con loro? Mandarti là del Lafite e un tacchino arrosto non sta mica bene. 

- No, ma io, durante la sosta del lavoro, verrò a casa. 

La mattina dopo Konstantin Levin si alzò più presto del solito, ma le disposizioni da dare per l'azienda lo trattennero e, quando giunse, i falciatori andavano già per la seconda falciata. 

Sin dall'alto della collina gli si era rivelata la parte in ombra del prato, quella già tagliata, con le falciate d'erba grigiastra e i mucchi neri dei gabbani dei falciatori tolti nel punto dal quale avevano preso l'avvio per la prima falciatura. 

A misura che si avvicinava, scorgeva i contadini in fila, uno dietro l'altro, alcuni coi gabbani, altri con la sola camicia, che menavano la falce in modo vario. Ne contò quarantadue. 

Si movevano lentamente per il fondo ineguale del campo dove c'era una vecchia diga. Levin riconosceva già qualcuno di loro. C'era il vecchio Ermil con la camicia bianca molto lunga che menava la falce stando curvo; c'era Vas'ka, il giovane che stava da Levin come cocchiere, e che prendeva la falciata con tutta la forza del braccio. C'era anche Tit, un contadino piccolo e asciutto, che aveva iniziato Levin alla fienagione. Andava avanti senza curvarsi, come se giocasse con la falce nel tagliare la sua larga falciata. 

Levin scese dal cavallo e, legatolo presso la strada, raggiunse Tit che, presa da un cespuglio un'altra falce, gliela diede. 

- È pronta, padrone, taglia come un rasoio, falcia da sé - disse Tit con un sorriso, togliendosi il berretto e dandogli la falce. 

Levin prese la falce e cominciò a provare. I falciatori che avevano finito la loro fila, uscivano sudati e allegri, uno dopo l'altro, sulla strada e salutavano, sorridendo, il padrone. Tutti lo guardavano, ma nessuno aprì bocca finché un vecchio, uscendo sulla strada, alto, col viso rugoso e glabro, con un giubbotto di montone, si rivolse a lui. 

- Attento a te, padrone. Se hai preso l'avvio, non restare addietro! - disse, e Levin udì un riso contenuto fra i falciatori. 

- Cercherò di non restare addietro - disse, mettendosi accanto a Tit e aspettando il momento per cominciare. 

- Bada a te - ripeté il vecchio. 

Tit fece posto a Levin che gli tenne dietro. L'erba era bassa, vicino alla strada, e Levin, che da tempo non falciava e si sentiva confuso sotto gli sguardi di tutti, falciò male al primo momento, pur agitando con forza la falce. Dietro di lui si sentirono delle voci. 

- È impostata male, il manico è troppo alto; guarda come deve abbassarsi - disse uno. 

- Pòggiati di più col tallone - disse un altro. 

- Non fa niente, va bene, taglia lo stesso - continuò il vecchio. - Guarda\ldots{} è andata\ldots{} Stai prendendo la falciata troppo larga, ti stancherai\ldots{} Il padrone, non c'è che dire, si sforza per sé. Ma guarda che sgorbio! Per una cosa simile a noi ce la danno sul groppone. 

L'erba diventò più morbida, e Levin, ascoltando senza rispondere, cercando di falciare come meglio poteva, teneva dietro a Tit. Erano andati avanti di cento passi. Tit procedeva senza fermarsi: ma Levin aveva già il terrore di non resistere, tanto era stanco. 

Sentiva che ormai falciava con le sue ultime riserve, e decise di pregare Tit di fermarsi. Ma proprio in quel momento Tit si fermò per conto suo e, chinatosi, prese dell'erba, asciugò la falce e si mise ad affilarla. Levin si raddrizzò e, dopo aver respirato, si guardò in giro. Dietro di lui procedeva un contadino che, evidentemente, era stanco anche lui, perché subito, senza raggiungere Levin, si fermò e prese ad affilare. Tit finì di affilare la falce sua e quella di Levin, e insieme proseguirono. 

Alla seconda ripresa fu lo stesso. Tit procedeva, un colpo dietro l'altro, senza fermarsi e senza stancarsi. Levin lo seguiva, sforzandosi di non restare indietro, ma gli era sempre più difficile: veniva il momento in cui sentiva di non avere più forze, ma proprio in quel momento Tit si fermava e si metteva ad affilare. 

Così passarono la prima falciata. E questa lunga falciata parve particolarmente difficile a Levin; in compenso quando fu terminata e Tit, gettandosi la falce sulla spalla, si mise a passo lento a percorrere, sulle orme lasciate dai tacchi, la falciata, anche Levin s'incamminò sulla propria. E sebbene il sudore gli scendesse a rivoli per il viso e gocciolasse giù dal naso e tutta la schiena fosse bagnata, come immersa nell'acqua, egli si sentiva bene. Lo rallegrava in modo particolare la sicurezza di poter resistere. 

La sua soddisfazione era amareggiata solo dal fatto che la falciata non gli riusciva bene. ``Moverò meno la mano e più il torso'' pensava, confrontando la falciata di Tit come tesa su di un filo, con la sua sparpagliata e disposta in modo ineguale. 

Nel passare la prima falciata, Tit, come aveva notato Levin, era andato particolarmente in fretta, forse per mettere alla prova il padrone e la falciata era capitata lunga. Le altre erano già più facili; Levin tuttavia doveva tendere tutte le sue forze per non restare indietro ai contadini. 

Egli non pensava a nulla, non desiderava nulla, altro che non restare indietro ai contadini e terminare nel modo migliore. Sentiva solo lo stridere delle falci e vedeva dinanzi a sé la figura diritta di Tit che si allontanava, il semicerchio curvo del terreno falciato, le erbe e le corolle dei fiori che si chinavano lente, a onda, intorno alla lama della falce e dinanzi a sé il termine della falciata, là dove sarebbe giunto il riposo. 

Nel mezzo del lavoro, senza capir che fosse e donde venisse, provò improvvisamente una piacevole sensazione di fresco giù per le spalle accaldate e sudate. Guardò il cielo mentre affilava la falce. Una nuvola bianca e greve s'era addensata e ne veniva giù una pioggia pesante. Alcuni contadini corsero ai gabbani e se li infilarono; altri, come Levin, si strinsero nelle spalle con gioia sotto la piacevole rinfrescata. 

Passarono ancora una falciata e poi ancora un'altra. Passavano falciate lunghe e corte, con l'erba buona e con l'erba cattiva. Levin aveva perso ogni nozione del tempo e proprio non sapeva se fosse tardi o presto. Nel suo lavoro si era verificato un cambiamento che gli fece grande piacere. Mentre lavorava, aveva dei momenti nei quali dimenticava quello che faceva, si sentiva leggero, e proprio in quei momenti la falciata gli veniva fuori uguale e bella quasi come quella di Tit. Ma appena si ricordava di quello che faceva, e si sforzava di far meglio, provava subito tutta la pesantezza del lavoro e la falciata gli riusciva male. 

Passata un'altra falciata, egli voleva di nuovo riprendere a camminare, ma Tit si fermò, e accostandosi al vecchio, gli disse qualcosa sottovoce. Guardarono insieme il sole. ``Di che stanno a parlare, e perché non continua a falciare?'' pensò Levin, senza rendersi conto che i contadini avevano falciato ininterrottamente non meno di quattro ore e che per loro era tempo di far colazione. 

- A colazione, padrone - disse il vecchio. 

- È forse ora? Di già a colazione? 

Levin rese la falce a Tit e, insieme coi contadini, che si erano avviati verso i gabbani a prendere il pane, si avviò verso il cavallo in mezzo alle falciate leggermente spruzzate di pioggia del lungo spazio lavorato. Ora soltanto capì che non aveva indovinato il tempo giusto e che la pioggia avrebbe rovinato il fieno. 

- Sciuperà il fieno - disse. 

- Non fa nulla, padrone: con la pioggia falcia, col bel tempo rastrella! - disse il vecchio. 

Levin sciolse il cavallo e andò a casa a prendere il caffè. 

Sergej Ivanovic s'era appena alzato. Preso il caffè, Levin tornò a falciare, prima che Sergej Ivanovic facesse in tempo a vestirsi e a venire in sala da pranzo. 

\capitolo{V}\label{v-2} 

Dopo la colazione, Levin non capitò più, nella fila, al posto di prima, ma fra il vecchio scherzoso che l'aveva invitato ad essere suo vicino e il contadino giovane, sposato solo dall'autunno, e che era venuto a falciare per la prima volta. 

Il vecchio, tenendosi diritto, andava avanti con un movimento eguale ed ampio delle gambe ricurve, e con un gesto preciso e uniforme, che ormai non gli costava, evidentemente, più che il dimenar delle braccia nel camminare, tagliava una falciata eguale, alta, come se giocasse. Proprio come se non lui, ma la falce affilata tagliasse da sola l'erba sugosa. 

Dietro a Levin andava il giovane Miška. Il giovane dal viso simpatico, coi capelli stretti da un laccio d'erba fresca, lavorava sempre con sforzo; ma appena lo guardavano, sorrideva. Evidentemente era pronto a morire anzi che confessare di far fatica. 

Levin camminava fra loro due. Nel pieno del caldo la falciatura non gli parve tanto difficile. Il sudore che lo inondava lo rinfrescava, e il sole che gli bruciava la schiena, la testa e il braccio dalla manica rimboccata fino al gomito, dava vigore e tenacia al lavoro; e sempre più spesso gli capitavano quei tali momenti di incoscienza, in cui si può non pensare a quello che si fa. La falce allora tagliava da sola. Erano questi i momenti felici. Ancora più felici quelli in cui, avvicinandosi al fiume verso il quale andavano a finire le falciate, il vecchio puliva con l'erba umida e folta la falce, ne sciacquava l'acciaio nell'onda fresca, vi immergeva un barattolo e lo offriva a Levin. 

- Su, ecco il mio kvas! Buono, eh? - diceva, ammiccando. 

E invero Levin non aveva mai bevuto una bevanda simile a quell'acqua tiepida con l'erba che ci sguazzava dentro e il senso di ruggine della latta del barattolo. E subito dopo seguiva una beata, lenta passeggiata con la mano sulla falce, durante la quale ci si poteva asciugare il sudore che scorreva a rivoli, si poteva respirare a pieni polmoni e si poteva guardare la schiera disseminata dei falciatori e tutto quello che avveniva in giro nel bosco e nel campo. 

Quanto più a lungo Levin falciava, tanto più spesso sentiva dei momenti di oblio durante i quali non eran le mani che menavano la falce, ma la falce stessa che trascinava con sé tutto il corpo di lui, cosciente e pieno di vita; e allora, come per incanto, senza pensarci, il lavoro si compiva da sé, regolare e preciso. Erano questi i momenti più beati. 

La cosa diveniva difficile solo quando si doveva far cessare questo moto inconsapevole e bisognava riflettere: quando cioè si doveva o falciare intorno a un monticello o intorno all'acetosella non estirpata. Il vecchio lo faceva con facilità. S'imbatteva in un monticello, ed ecco cambiava movimento, e dove col tallone, dove con l'estremità della falce abbatteva il monticello da tutte e due le parti a piccoli colpi. E nel far questo guardava e osservava sempre quello che gli si parava innanzi; ora strappava una radichetta, la mangiava o l'offriva a Levin, ora gettava via con la punta della falce un ramo, ora osservava un piccolo nido di quaglie, dal quale, proprio di sotto alla falce, volava via la femmina, ora afferrava una vipera capitatagli sul cammino, e alzandola con la falce, come su di una forchetta, la mostrava a Levin e la buttava via. 

A Levin invece e al giovane dietro di lui, queste variazioni di movimento riuscivano difficili. Tutti e due, dato l'avvio ad un unico movimento di tensione, si trovavano presi nella foga del lavoro e non erano in grado di mutar movimento e di osservare nel tempo stesso quello che si parava innanzi. 

Levin non s'accorgeva dello scorrer del tempo. Se gli avessero chiesto quanto tempo era che falciava, avrebbe risposto da una mezz'ora, e invece s'era già avvicinata l'ora del desinare. Avviandosi per la falciata, il vecchio richiamò l'attenzione di Levin su alcune bambine e alcuni ragazzetti che da varie parti, appena visibili, camminavano fra l'erba alta e sulla strada verso i falciatori, portando il pane e le brocche di kvas, chiusi in fagotti di stracci, che stiravano loro le piccole braccia. 

- Guarda, i moscerini che strisciano! - disse, indicandoli e, facendosi schermo con la mano, guardò il sole. 

Passarono altre due falciate e il vecchio si fermò. 

- Su, via, padrone, a mangiare! - disse deciso. E, avviandosi al fiume, i falciatori si diressero in mezzo alle falciate, verso i gabbani, accanto ai quali sedevano, aspettandoli, i bambini che avevano portato il desinare. I contadini si riunirono, alcuni lontani sotto i carri, altri vicino presso un ciuffo di citiso sotto il quale avevano gettato dell'erba. 

Levin sedeva accanto a loro; non aveva voglia di andarsene. 

Ogni imbarazzo di fronte al padrone era ormai scomparso da un pezzo. I contadini si preparavano a mangiare. Alcuni si lavavano, i giovani facevano il bagno nel fiume, altri si accomodavano un posto per la siesta, scioglievano gli involti col pane e aprivan le brocche col kvas. Il vecchio sbriciolò del pane nella ciotola, l'impastò col manico del cucchiaio, versò dell'acqua dalla brocca, tagliò ancora del pane, e, sparsovi sopra del sale, si volse verso oriente per pregare. 

- Ecco, barin, prendi la mia tjur'ka - disse, mettendosi in ginocchio davanti alla ciotola. 

La zuppa era così gustosa che Levin decise di non andare a casa a pranzare. Pranzò col vecchio e si mise a discorrere con lui delle sue faccende di casa, prendendovi il più vivo interesse; gli parlò poi di tutte le proprie cose e con tutti i particolari che potevano interessare il vecchio. Si sentiva più vicino a lui che al fratello, e involontariamente sorrideva per la tenerezza che provava per quell'uomo. Quando il vecchio si alzò di nuovo e, dopo aver pregato e dopo essersi approntato un fascio d'erbe, si sdraiò lì sotto al cespuglio, Levin fece lo stesso, e malgrado le mosche e i moscerini appiccicosi e molesti che gli solleticavano il viso e il corpo sudati, si addormentò immediatamente, e si svegliò solo quando il sole, dall'altra parte del cespuglio, cominciò a raggiungerlo. Il vecchio già da tempo era sveglio e sedeva affilando le falci dei giovani. 

Levin guardò attorno e non riconobbe il luogo; tanto era cambiato tutto. Un enorme spazio del campo era stato falciato e brillava di uno splendore particolare, nuovo, con le falciate che odoravano sotto i raggi obliqui del sole calante. E i cespugli intorno ai quali s'era falciato, vicino al fiume, e lo stesso fiume prima invisibile e ora risplendente d'acciaio nelle sue anse, e i contadini che si movevano e si sollevavano e la parete erta dell'erba del campo non ancora falciato, e gli sparvieri che roteavano sul prato spoglio, tutto questo era affatto nuovo. Risvegliatosi, Levin cominciò a considerare quanto era stato falciato e quanto ancora si poteva falciare nella giornata. 

S'era lavorato proprio di buona lena, tenendo conto che gli operai erano quarantadue. Tutto il prato grande, che al tempo della servitù si falciava in due giorni con trenta opre, era già stato falciato. Restavano solo gli angoli delle falciate corte. Ma Levin voleva falciare quanto più era possibile per quel giorno, e se la prendeva col sole che calava così presto. Non sentiva più nessuna stanchezza; voleva solo lavorare sempre più svelto e sempre di più. 

- E ce la faremo a falciare anche il Maškin Verch? che ne dici? - disse al vecchio. 

- Come Dio vuole, il sole non è alto. Posso promettere un po' di vodka ai ragazzi? 

Così durante la refezione, quando di nuovo si furon seduti e i fumatori si erano messi a fumare, il vecchio fece intendere ai ragazzi che ``a falciare il Maškin Verch ci sarebbe stata la vodka''. 

- E che, non falciarlo? Via, Tit! Sbrighiamoci alla svelta. Finirai di mangiare stanotte! va', va'! - si sentirono delle voci e, terminando di mangiare il pane, i falciatori si misero subito in cammino. 

- Su, ragazzi, forza! - disse Tit e, quasi al trotto, andò avanti. 

- Va', va', - diceva il vecchio, canterellando dietro di lui, e, dopo averlo raggiunto facilmente: - taglio! Bada! 

E giovani e vecchi falciavano come a gara. Ma pur facendo in fretta, non sciupavano l'erba e le falciate si adagiavano in modo preciso e netto. Il tratto di campo che era rimasto in angolo fu tagliato in cinque minuti. Non ancora gli ultimi falciatori tagliavano la falciata, che già quelli avanti avevano gettato i gabbani sulle spalle e si avviavano sulla strada verso il Maškin Verch. 

Il sole inclinava già verso gli alberi, quando i falciatori, con rumor di ciotole, entrarono nel piccolo burrone boscoso del Maškin Verch. L'erba al centro del vallone arrivava alla cintola, ed era tenera e morbida, soffice, colorata qua e là di violacciocche. 

Dopo un breve parlottare: se andare in lungo o in largo, Prochor Ermilin, un bravo falciatore anche lui, un contadino enorme, abbronzato, andò avanti. Andò avanti per una falciata, si voltò indietro e fece largo, e tutti cominciarono ad allinearsi dietro di lui, procedendo in discesa per il vallone, e in salita accanto al margine del bosco. Il sole era calato dietro il bosco. Cadeva già la brina: soltanto i falciatori che erano sull'altura erano esposti al sole, ma in basso, dove si era levata la nebbia, e di lato, procedevano all'ombra fresca, rugiadosa. Il lavoro ferveva. L'erba tagliata con un suono pieno ed esalante un odore acuto, si adagiava nelle falciate alte. I falciatori che si stringevano da ogni parte perché le falciate erano corte, si sollecitavano l'un l'altro con grida allegre, facendo rumore con le ciotole, e risonando col cozzar delle falci e lo stridere dell'acciarino sulla lama. 

Levin camminava sempre fra il giovane e il vecchio. Il vecchio, rivestito di un giubbotto di montone, era sempre allegro, scherzoso e agile nei movimenti. Nel bosco capitavano continuamente dei funghi, gonfiatisi nell'erba sugosa, che venivan tagliati via dalle falci. Ma il vecchio, incontrando i funghi, si chinava ogni volta, tirava su e metteva in petto: ``Ancora un regalo per la vecchia'' diceva. 

Per quanto fosse facile falciare l'erba umida e tenera, era però difficile scendere e salire per i ripidi pendii del burrone. Ma il vecchio non era in imbarazzo. Menava la falce sempre allo stesso modo, col piccolo passo fermo dei suoi piedi infilati nei grandi lapti, s'arrampicava lentamente lungo il pendio, e pur traballando con tutto il corpo e coi pantaloni che pendevano di sotto la camicia, non tralasciava nel cammino neppure un filo d'erba, né un fungo, e scherzava allo stesso modo coi contadini e con Levin. Levin gli teneva dietro e spesso temeva di cadere nel salir con la falce su di un'erta così ripida dove anche senza falce era difficile arrampicarsi; ma s'arrampicava e faceva quello che doveva. Si sentiva sospinto da una forza esterna. 

\capitolo{VI}\label{vi-2} 

Falciarono il Maškin Verch, terminarono le ultime file, indossarono i gabbani e andarono allegramente verso casa. Levin montò a cavallo e, congedatosi con rammarico dai contadini, prese la via del ritorno. Dall'alto si voltò a guardarli; non si vedevano più nella nebbia che saliva dal basso; si udivano solo le grosse voci allegre, il riso e il suono delle falci che si cozzavano. 

Sergej Ivanovic da tempo aveva finito di pranzare e stava sorbendo acqua e limone e ghiaccio in camera sua, guardando le riviste e i giornali ricevuti proprio allora con la posta, quando Levin, coi capelli arruffati e appiccicati, la schiena e il petto anneriti e bagnati dal sudore, irruppe in camera con un allegro vociare. 

- Abbiamo finito tutto il prato! Ah, com'è bello, meraviglioso! E tu come te la sei passata? - disse Levin del tutto dimentico della conversazione poco piacevole della sera prima. 

- Dio mio! cosa sembri - disse Sergej Ivanovic, voltandosi a guardare, scontento sulle prime, il fratello. - Sì, la porta, la porta, chiudila! - gridò.- Ne avrai fatte entrare certamente una dozzina. 

Sergej Ivanovic non poteva sopportare le mosche e nella sua stanza apriva le finestre solo di notte e chiudeva con cura le porte. 

- Eh, via, neppure una. E se le ho fatte entrare, le acchiapperò. Tu non puoi immaginare, che piacere! E tu come hai passato la giornata? 

- Bene. Ma hai forse falciato tutto il giorno? Avrai una fame da lupo, penso. Kuz'ma ti ha preparato tutto. 

- No, non ho voglia di mangiare. Ho mangiato là. Ma ecco, vado a lavarmi. 

- Be', vai, vai; vengo subito da te - disse Sergej Ivanovic, scotendo il capo nel guardare il fratello. - E fai presto - aggiunse sorridendo e, riuniti i suoi libri, si preparò a muoversi. A un tratto anche egli era diventato allegro e non voleva separarsi dal fratello. - Dimmi, quando ha piovuto, dov'eri? 

- Ma quale pioggia? Appena poche gocce. Allora vengo subito. Così hai passato bene la giornata? Su, benissimo. - E Levin andò a vestirsi. 

Dopo cinque minuti i due fratelli si ritrovavano in sala da pranzo. Sebbene a Levin sembrasse di non aver appetito, e si fosse seduto a tavola solo per non dispiacere Kuz'ma, quando cominciò a mangiare, il pranzo gli parve straordinariamente gustoso. Sergej Ivanovic guardava sorridendo. 

- Ah, già, c'è una lettera per te - disse. - Kuz'ma, portala giù, per piacere. E guarda di chiudere la porta. 

La lettera era di Oblonskij. Levin la lesse ad alta voce. Oblonskij scriveva da Pietroburgo: ``Ho ricevuto una lettera da Dolly; è a Ergušovo e là le cose non vanno troppo bene. Ti prego, va' da lei e aiutala un po' con il tuo consiglio; tu sai tutto. Sarà lieta di vederti. È proprio sola, poverina. Mia suocera con gli altri è ancora all'estero''. 

- Bene! Andrò certamente da loro - disse Levin. - Anzi, andremo insieme. Lei è così simpatica. Non è vero? 

- Stanno lontano da qui? 

- Trenta verste. Forse anche quaranta. Ma la strada è ottima. Andremo comodamente. 

- Sono molto contento - disse Sergej Ivanovic, sempre sorridendo. La presenza del fratello minore lo predisponeva subito all'allegria 

- Eh, che appetito che hai! - disse, guardando il volto abbronzato rosso-scuro e il collo di lui chino sul piatto. 

- Ottimo! Non puoi credere che cura utile contro ogni balordaggine. Voglio arricchire la medicina di un termine nuovo: Arbeitskur. 

- Su, questo a te non occorre, mi pare. 

- Già, ma alle varie specie di malati di nervi, sì. 

- E già, bisognerebbe provarlo. Avrei voluto venire alla fienagione per vederti, ma il caldo era così insopportabile che non sono andato più in là del bosco. Son rimasto un po' a sedere e attraverso il bosco sono andato al villaggio, ho incontrato la tua governante e l'ho saggiata un po' circa l'opinione che i contadini hanno di te. A quanto ho potuto capire, non approvano questo. Ha detto: ``non è affar da signori''. In genere, mi pare che, nella concezione popolare, la manifestazione di una certa attività che essi chiamano ``da signori'' sia molto ben delimitata. E non ammettono che i signori escano fuori dal quadro delle loro concezioni. 

- Forse; ma questo è un tale godimento, quale non avevo mai provato in tutta la vita. E non c'è nulla di male. Vero? - rispose Levin. - Che farci se a loro non va? Del resto, io credo che non importi nulla. Eh? 

- In generale - proseguì Sergej Ivanovic - come vedo, sei soddisfatto della tua giornata. 

- Molto soddisfatto. Abbiamo falciato l'intero prato. E sapessi con che razza di vecchietto ho fatto amicizia! Non te lo puoi immaginare: un incanto! 

- Dunque, sei contento della tua giornata. E io pure. Per primo, ho risolto due mosse di scacchi di cui una è molto carina, si apre con un pedone, te la mostrerò. E poi ho pensato alla nostra conversazione di iersera. 

- Cosa? Alla conversazione di ieri? - disse Levin, socchiudendo beatamente gli occhi e riprendendo fiato dopo la fine del pasto, nell'impossibilità assoluta di ricordare quale fosse stata la conversazione del giorno innanzi. 

- Ti dimostrerò che hai ragione solo in parte. Il nostro disaccordo consiste in questo: che tu poni come movente l'interesse personale, e io suppongo che ogni uomo che abbia un certo grado di cultura debba interessarsi del bene comune. Può anche darsi che tu abbia ragione, che sia più desiderabile un'attività materiale spronata dall'interesse. In generale tu sei una natura troppo primesautière, come dicono i francesi; per te o un'attività appassionata, energica, o niente. 

Levin udiva le parole del fratello, ma non capiva proprio nulla e non voleva capire. Temeva solo che il fratello gli rivolgesse qualche domanda, perché allora sarebbe subito apparso che non lo ascoltava affatto. 

- Così è, amico mio - disse Sergej Ivanovic, toccandogli la spalla. 

- Sì, s'intende. Ma cosa mai! Io non mi intestardisco mica - rispose Levin con un colpevole sorriso infantile. 

``Ma di che si discuteva? - pensava. - S'intende, ho ragione io ed ha ragione lui e tutto va benissimo. Ma debbo passare in amministrazione a dare gli ordini''. Si alzò stirandosi e sorridendo. 

Anche Sergej Ivanovic sorrise. 

- Vuoi fare una passeggiata, andiamo insieme - disse, desideroso di non staccarsi dal fratello che emanava freschezza e vigore. - Andiamo, passiamo pure in amministrazione, se ci devi andare. 

- Ah, Dio mio! - gridò Levin così forte che Sergej Ivanovic si spaventò. 

- Che hai? 

- E il braccio di Agaf'ja Michajlovna! - disse Levin, battendosi la fronte. - Me n'ero proprio scordato! 

- Va molto meglio. 

- Via, faccio una corsa da lei. Non farai in tempo a metterti il cappello che sarò qui. 

E, correndo giù per la scala, fece risonare i tacchi come una raganella. 

\capitolo{VII}\label{vii-2} 

Stepan Arkad'ic era andato a Pietroburgo per compiere il più elementare dei doveri, i doveri di tutti i funzionari, il più necessario dei doveri, anche se incomprensibile a chi non è funzionario, omesso il quale non c'è modo di mantenere un impiego, far notare, cioè la propria esistenza al ministero. E mentre per compiere questo dovere, dopo aver preso con sé quasi tutto il denaro di casa, passava allegramente e piacevolmente il tempo alle corse e nei dintorni, Dolly coi bambini era andata a starsene in campagna, per diminuire, quanto più era possibile, le spese. Era andata nella sua proprietà dotale di Ergušovo, quella stessa dove in primavera era stato venduto il legname e che distava cinquanta verste da Pokrovskoe di Levin. 

La grande vecchia casa di Ergušovo era da tempo mal ridotta, ed era stata riparata dal vecchio principe che ne aveva anche ingrandita un'ala. Quest'ala, venti anni prima, quando Dolly era ancora ragazza, era spaziosa e comoda, pur rimanendo di lato, come tutte le ali, rispetto al viale d'ingresso e all'esposizione a mezzogiorno. Ma ormai anch'essa era cadente e ammuffita. Quando in primavera Stepan Arkad'ic vi si era recato a vendere il legname, Dolly gli aveva raccomandato di guardare con attenzione la casa e di ordinare e fare eseguire le riparazioni più necessarie. Stepan Arkad'ic che, come tutti i mariti colpevoli, si adoperava molto perché la moglie si trovasse a suo agio, aveva egli stesso dato un'occhiata alla casa e aveva impartito ordini per tutto quello che, secondo lui, era necessario. Secondo lui era necessario rivestire tutto il mobilio di cretonne, mettere le tende, ripulire il giardino, costruire un ponticello vicino allo stagno e piantare dei fiori: ma aveva dimenticato molte altre cose indispensabili, la cui mancanza costituiva ora un tormento per Dar'ja Aleksandrovna. 

Per quanto Stepan Arkad'ic si sforzasse di essere un marito e un padre premuroso, non riusciva in nessun modo a ricordarsi d'aver moglie e figli. Aveva gusti da celibe, e solo ad essi si conformava. Tornato a Mosca, aveva detto alla moglie che la casa era pronta, che era proprio un gioiello e che le consigliava di andarvi. Sotto tutti i riguardi a Stepan Arkad'ic piaceva molto che la moglie partisse; era salutare per i ragazzi, le spese sarebbero state minori, ed egli sarebbe stato più libero. Dar'ja Aleksandrovna a sua volta considerava il soggiorno in campagna indispensabile per i ragazzi, soprattutto per la bambina che non riusciva a riaversi dai postumi della scarlattina, e vedeva in esso anche la liberazione dalle piccole umiliazioni, dai piccoli debiti col legnaiuolo, col pescivendolo, col calzolaio, che la tormentavano tanto. Inoltre, la partenza l'attraeva perché sognava di far venire presso di sé la sorella Kitty, che sarebbe dovuta rientrare a mezza estate e alla quale avevano consigliato i bagni. Kitty le aveva già scritto dalla stazione termale che nulla le sorrideva tanto quanto passare l'estate a Ergušovo, così pieno di ricordi d'infanzia per tutte e due. 

Il primo periodo della vita in campagna fu molto difficile per Dolly. Aveva vissuto in campagna nell'infanzia e ne aveva serbata l'impressione che la campagna fosse una specie di liberazione da tutti i dispiaceri cittadini, che là, malgrado la vita non brillante (e a tutto questo Dolly si rassegnava presto), tutto fosse almeno accessibile e comodo; che tutto fosse a buon prezzo, che vi si potesse trovare tutto, e che ai bambini la campagna facesse bene. Ma ora, giuntavi come padrona di casa, vide che non era così come pensava. 

Il giorno dopo il loro arrivo, venne giù una pioggia dirotta e la notte cominciò a gocciolare nel corridoio e anche nella camera dei bambini, sì che si dovettero trasportare i lettini nel salotto. La sguattera non c'era; di nove mucche, a stare a sentire la donna addettavi, alcune erano pregne, altre al primo vitello, altre erano vecchie, altre avevano i capezzoli stretti; latte e burro non bastavano per i bambini. Uova non ce n'era. Una gallina non si poteva trovare, venivan serviti arrosto certi galli vecchi color viola, filamentosi. Non si trovavano donne per lavare i pavimenti; erano tutte a raccogliere le patate. Non si poteva uscire in carrozza perché il cavallo s'impennava e strapazzava il timone. Non c'era dove fare i bagni: tutta la sponda del fiume veniva calpestata dal bestiame ed era tutta aperta dal lato della strada; non si poteva neppure passeggiare nel giardino, perché il bestiame vi entrava attraverso uno squarcio dello steccato, e c'era un terribile toro che muggiva e sembrava pronto a dare cornate. Non c'erano armadi per i vestiti. Quelli che c'erano non chiudevano e s'aprivano da soli quando ci si passava accanto. Non c'erano né pentole né tegami; non c'era la caldaia per la lavanderia, e nella stanza delle donne nemmeno la tavola da stiro. 

In quel primo periodo, Dar'ja Aleksandrovna, che sperava di trovare tranquillità e riposo, capitata tra tutti questi guai, per lei enormi, si sentiva disperata; si dava da fare con tutte le sue energie, ma sentiva che non c'era via d'uscita, e ogni momento tratteneva le lacrime che le spuntavano negli occhi. L'amministratore, un ex sottufficiale, che Stepan Arkad'ic aveva preso a benvolere e che aveva promosso, per la sua prestanza e il suo fare ossequioso, dall'ufficio di portiere, non prese parte alcuna alle pene di Dar'ja Aleksandrovna; si limitava a dire rispettosamente: ``Non è proprio possibile, è gente così cattiva'' e non l'aiutava in nulla. 

La situazione sembrava senza via d'uscita. Ma anche in casa Oblonskij, come in tutte le case dove ci sono molti membri di famiglia, c'era la persona che non si faceva notare, ma che era tanto importante e utile: Matrëna Filimonovna. Costei calmava la signora, la rassicurava che tutto si ``sarebbe appianato'' (era questo il suo intercalare e da lei l'aveva preso Matvej), ed ella stessa, senza affrettarsi e senza agitarsi, operava. 

Andò subito d'accordo con la moglie dell'amministratore, e sin dal primo giorno bevve con lei e con l'amministratore il tè sotto le acacie e prese in esame tutte le questioni. Ben presto, lì, sotto le acacie, si venne a formare il circolo formato dalla moglie dell'amministratore, dallo starosta e dall'impiegato d'ufficio; cominciarono ad appianarsi a poco a poco tutte le difficoltà della vita: dopo una settimana, infatti, realmente tutto s'era ``appianato''. Il tetto fu accomodato, si trovò la cuoca, una comare dello starosta, le galline furono comprate; le mucche ripresero a dare il latte, il giardino fu recinto, furono messi dei ganci agli armadi che non si aprirono più arbitrariamente, e la tavola da stiro, ravvolta in un panno da soldato, fu distesa dal bracciuolo di una poltrona al cassettone, sì che nella stanza delle donne si sentì odor di stiro. 

- Su, ecco, e voi non facevate altro che disperarvi! - diceva Matrëna Filimonovna, mostrando la tavola. 

Fu costruito perfino un recinto per fare i bagni con paraventi di paglia. Lily cominciò a fare il bagno e per Dar'ja Aleksandrovna si avverarono, almeno in parte, le sue aspirazioni a una vita di campagna, se non tranquilla, almeno comoda. Dar'ja Aleksandrovna, con sei bambini, tranquilla non poteva mai essere. Uno si ammalava, l'altro rischiava di ammalarsi, al terzo mancava qualcosa, il quarto mostrava i segni d'un cattivo carattere e così via di seguito. Di rado, molto di rado, venivano brevi periodi di tranquillità. Ma queste cure e questi affanni erano per Dar'ja Aleksandrovna l'unica felicità possibile. Se non vi fossero stati, sarebbe rimasta sola col pensiero rivolto al marito che non l'amava. Ma, a parte ciò, per quanto fossero penosi per la madre la paura delle malattie, le malattie stesse e il dolore suo nel constatare le cattive inclinazioni dei figli, questi stessi figliuoli già adesso, con tante piccole gioie, la ripagavano delle sue pene. Queste gioie però eran così piccole che non si notavano, così come non si nota l'oro fra la sabbia; e nei momenti cattivi ella vedeva solo i dolori, cioè la sabbia, mentre c'erano pure i momenti buoni in cui vedeva solo le gioie, solo l'oro. 

Adesso, nella solitudine della campagna, ella sempre più spesso si rendeva conto di queste gioie. Spesso, guardando i figli, faceva tutti gli sforzi possibili per convincersi che si sbagliava, che come madre era parziale verso di loro; tuttavia non poteva non dirsi che aveva dei bambini deliziosi, tutti e sei, tutti così diversi, ma come non è facile trovarne, e ne era felice e orgogliosa. 

\capitolo{VIII}\label{viii-2} 

Verso la fine di maggio, quando già tutto era più o meno in ordine, ebbe risposta dal marito alle sue lamentele sui disagi campestri. Le scriveva chiedendole venia di non aver pensato a tutto e promettendo di venire alla prima occasione. Questa occasione non si era presentata, e fino ai primi di giugno Dar'ja Aleksandrovna visse sola in campagna. 

La vigilia di san Pietro, di domenica, Dar'ja Aleksandrovna era andata alla messa per far fare la comunione a tutti i suoi ragazzi. Dar'ja Aleksandrovna, nei suoi discorsi intimi, filosofici con la sorella, con la madre, con gli amici, molto spesso meravigliava per la sua libertà di pensiero in materia di religione. Credeva stranamente nella metempsicosi, poco preoccupandosi dei dogmi della Chiesa. Ma nella famiglia, e non solo per dare l'esempio, ma con tutta l'anima, adempiva rigorosamente tutti i precetti della Chiesa; e il fatto che i ragazzi per quasi un anno non si fossero comunicati l'agitava molto; sì che, con la piena approvazione e partecipazione di Matrëna Filimonovna, aveva deciso di far avvenire ciò in quella estate. 

Alcuni giorni prima aveva pensato all'abbigliamento di tutti i bambini. Furono cuciti, rifatti e lavati i vestiti, messi fuori gli orli e le finte; cuciti i bottoni e preparati i nastri. Un vestito per Tanja, che la signorina inglese s'era incaricata di cucire, fece masticar veleno a Dar'ja Aleksandrovna. L'inglese, nel ricucire, non aveva rifatto le pieghe al posto giusto, aveva tirato le maniche troppo in fuori e sembrava aver completamente sciupato l'abito. E in tal modo Tanja aveva delle spalle così strette che faceva male a guardarla. Ma Matrëna Filimonovna pensò di far delle riprese e di aggiungere una pellegrina. Alla cosa si pose riparo, ma ne venne fuori una baruffa con l'inglese. La mattina però tutto era in ordine e verso le nove, il termine fino al quale avevano pregato il sacerdote di attendere prima di iniziare la messa, i bambini, raggianti di gioia, tutti agghindati, erano presso la scalinata davanti alla carrozza, in attesa della madre. 

Alla carrozza, invece di Voron che s'impennava, era stato attaccato, per raccomandazione di Matrëna Filimonovna, Buryj dell'amministratore; e Dar'ja Aleksandrovna, che s'era trattenuta per curare il proprio abbigliamento, montò in carrozza in abito bianco di mussolina. 

Si adornava e vestiva con ansia e preoccupazione. Un tempo s'era vestita per sé, per essere bella e piacente; poi, con l'andar degli anni, l'abbigliarsi le era divenuto increscioso; s'accorgeva di non esser più bella. Ma ora di nuovo si vestiva con gusto e trepidazione. Ora non si abbigliava più per sé, né per la sua bellezza, ma per non sciupare, come madre di quei tesori di figli, l'impressione generale. Guardatasi nello specchio l'ultima volta, rimase contenta di sé. Stava bene. Non così bene come quando, ai suoi tempi, voleva figurare a un ballo, ma stava bene per lo scopo che perseguiva. 

In chiesa non c'era nessuno all'infuori dei contadini, dei portieri e delle donne. Ma a Dar'ja Aleksandrovna parve di scorgere l'incanto suscitato dai suoi bambini e da lei. I bambini non solo erano splendidi nei loro vestitini di gala, ma erano graziosi perché si comportavano proprio bene. È vero che Alëša non stava proprio del tutto composto: non faceva che voltarsi per rimirarsi il dietro del giubbetto; tuttavia era straordinariamente aggraziato. Tanja stava lì composta come una personcina grande e badava ai piccoli. Ma la più piccola, Lily, era deliziosa nel suo ingenuo stupore di tutto e fu difficile non sorridere quando, ricevuta la comunione, disse: ``please, some more''. 

Tornando a casa, i bambini sentivano che qualcosa di solenne era stato compiuto, ed erano tranquilli. 

Tutto andò bene anche a casa; ma a colazione Griša cominciò a fischiare, e quel che fu peggio, non obbedì all'inglese, così che fu privato del dolce. Dar'ja Aleksandrovna, se si fosse trovata presente, non avrebbe inflitto, proprio in quel giorno, una punizione, ma ormai era necessario sostenere la punizione dell'inglese, e la decisione che per Griša non ci sarebbe stato il dolce, venne riconfermata. Questo sciupò un po' la felicità generale. 

Griša piangeva, e diceva che anche Nikolen'ka aveva fischiato, eppure non era stato punito, e che egli non piangeva per la torta\ldots{} tanto era lo stesso\ldots{} ma perché si era ingiusti con lui. Questo era troppo triste, e Dar'ja Aleksandrovna aveva deciso di perdonare Griša e, per interpellare l'inglese, andò da lei. Ma, attraversando la sala, vide una scena che le riempì il cuore di una tale gioia che le lacrime le vennero agli occhi ed ella stessa perdonò senz'altro il malandrino. 

Il colpevole sedeva nella sala sul davanzale della finestra ad angolo; vicino a lui stava Tanja con un piatto. Col pretesto di voler imbandire un pranzo alle bambole, ella aveva chiesto all'inglese il permesso di portare una parte del suo dolce nella camera dei bambini, e invece l'aveva portato al fratellino. Griša, continuando a piangere sull'ingiustizia patita, mangiava la torta e fra i singhiozzi diceva: ``mangia anche tu, mangiamo insieme\ldots{} insieme''. 

Su Tanja aveva agito prima la pena per Griša, poi la coscienza della propria buona azione, sì che anche a lei venivano le lacrime agli occhi, ma non rifiutava e mangiava la propria parte. 

Scorta la madre, i piccoli si spaventarono, ma dal viso di lei capirono d'aver fatto bene, si misero a ridere con le bocche piene di torta, e cominciarono a pulire le labbra sorridenti con le mani, impiastricciando così di lacrime e marmellata i loro visi splendenti. 

- Mamma mia! Il vestito nuovo bianco! Tanja! Griša! - diceva la mamma, sforzandosi di salvare il vestito, ma sorridendo, con le lacrime agli occhi, d'un riso beato, entusiastico. 

I vestiti nuovi furono tolti, si fecero indossare alle bimbe dei camiciotti e ai bambini delle vecchie giacchette, e si fece attaccare alla carrozza lunga, di nuovo e con rincrescimento dell'amministratore, Buryj al timone per andare in cerca di funghi e al bagno. Un entusiastico grido si levò nella camera dei bambini e non si chetò fino alla partenza per il bagno. 

Di funghi se ne raccolse un cestino colmo; perfino Lily trovò un prugnolo. Era stata miss Hull a trovarne per prima e a mostrarglieli; ma ora aveva trovato da sola una grossa cappella di prugnolo, e questo fatto la rese oggetto di un'entusiastica ovazione generale: ``Lily ha trovato una cappella!''. 

Poi si andò verso il fiume; i cavalli furono lasciati all'ombra delle piccole betulle, e si andò al bagno. Il cocchiere Terentij, legati ad un albero i cavalli liberi dai freni, si sdraiò, schiacciando l'erba, all'ombra d'una betulla e si mise a fumare tabacco in foglie, mentre dal fiume gli giungeva l'allegro stridio infantile che non si chetava. 

Sebbene fosse faticoso badare a tutti i ragazzi e frenare le loro birichinate, sebbene fosse difficile ricordare e non confondere tutte quelle calzine, quei pantaloncini, quelle scarpette dei vari piedini, e inoltre snodare e sbottonare e riannodare fettuccine e bottoncini, Dar'ja Aleksandrovna, cui sempre era piaciuto fare i bagni, e che li riteneva utili per i ragazzi, di niente godeva tanto come di quel bagno fatto insieme con tutti i suoi bambini. Toccare quelle gambette paffute, stendendo su di esse le calzine, prendere nelle braccia e bagnare tutti quei corpicini nudi e sentir le strida ora gioiose ora spaventate; vedere quei volti ansanti, con gli occhi spalancati, impauriti e allegri, di quei suoi cherubini che si spruzzavano, tutto questo era un godimento grande per lei. 

Quando già una metà dei bambini fu rivestita, alcune donne parate a festa, che andavano a raccogliere erba egizia ed euforbia, si avvicinarono al bagno e si fermarono impacciate. Matrëna Filimonovna ne chiamò una, per darle a stendere un lenzuolo e una camicia caduti nell'acqua, mentre Dar'ja Aleksandrovna prese a discorrere con loro. Queste, che in principio ridevano nascondendosi il viso con la mano, e sembravano non capire le domande, si fecero presto coraggio e cominciarono a parlare, conquistando subito Dar'ja Aleksandrovna con la sincera ammirazione per i bambini che andavano via via indicando. 

- Guarda che bellezza, è bianca come lo zucchero - diceva una, ammirando Tanja e scotendo il capo. - Ma è magra\ldots{} 

- Sì, è stata malata. 

- Guarda, han fatto il bagno anche a lei - diceva un'altra, indicando la bambina lattante. 

- No, ha solo tre mesi - rispondeva con orgoglio Dar'ja Aleksandrovna. 

- Guarda! 

- E tu quanti ne hai? 

- Ne avevo quattro; ma ne sono restati due: un maschio e una femmina. Ecco, l'ho svezzata a carnevale. 

- E quanto ha? 

- Va per i due anni. 

- E perché l'hai allattata così a lungo? 

- È usanza nostra: le tre vigilie\ldots{} 

E la conversazione divenne quanto mai interessante per Dar'ja Aleksandrovna: come aveva partorito? di che cosa s'era ammalata? dov'era il marito? veniva spesso? 

Dar'ja Aleksandrovna non voleva staccarsi dalle donne, tanto l'interessava la conversazione avviata con loro, tanto identici in tutto erano i loro interessi. E la cosa più piacevole per Dar'ja Aleksandrovna era veder chiaramente come quelle donne l'ammirassero soprattutto perché aveva tanti figliuoli e tutti così belli. Le donne fecero anche ridere Dar'ja Aleksandrovna, ma l'inglese, che era stata la causa di quel riso per lei incomprensibile, si turbò. Una delle giovani donne, osservando l'inglese che si vestiva da ultima e che indossava una terza sottana, non poté trattenersi dall'osservare: ``Guarda, quella se ne mette e se ne mette e ancora ce ne ha!'' al che tutte erano scoppiate a ridere. 

\capitolo{IX}\label{ix-2} 

Circondata da tutti i bambini che avevano fatto il bagno e avevan le testine ancora umide, Dar'ja Aleksandrovna, con un fazzoletto in testa, si avvicinava già a casa, quando il cocchiere disse: 

- C'è un signore che arriva, mi pare, quello di Pokrovskoe. 

Dar'ja Aleksandrovna guardò innanzi a sé e si rallegrò vedendo la figura di Levin che le veniva incontro in cappello e cappotto grigio. Era sempre contenta di vederlo, ma in questo momento era particolarmente lieta ch'egli la vedesse in tutta la sua gioia. Nessuno più di Levin poteva apprezzarne il valore. 

Vistala, egli si trovò dinanzi uno di quei quadri di vita familiare che aveva sognato per sé per il futuro. 

- Sembrate una chioccia, Dar'ja Aleksandrovna. 

- Oh, come sono contenta! - diss'ella, tendendogli la mano. 

- Siete contenta, ma intanto non me lo avete fatto sapere. Da me c'è mio fratello. Soltanto ora ho ricevuto un biglietto di Stiva e ho saputo che eravate qua. 

- Di Stiva? - chiese con sorpresa Dar'ja Aleksandrovna. 

- Sì, scrive che avete cambiato residenza, e pensa che mi permetterete d'aiutarvi in qualche cosa - disse Levin; ma, appena detto questo, si confuse e, interrotto il discorso, seguitò a camminare in silenzio accanto alla carrozza, strappando i germogli dei tigli e spezzandoli coi denti. Si era confuso temendo che Dar'ja Aleksandrovna potesse non gradire l'aiuto di una persona estranea in cose che sarebbero dovute spettare al marito. A Dar'ja Aleksandrovna, invero, non garbava punto l'abitudine di Stepan Arkad'ic di affidare le faccende familiari a estranei. Ma capì subito che Levin aveva compreso. Anche per questa sua finezza d'intuito, per questa delicatezza, Dar'ja Aleksandrovna voleva bene a Levin. 

- Ma io ho capito, s'intende - disse Levin - questo vuol dire soltanto che volete vedermi, e io ne sono molto contento. Immagino che voi, padrona di casa cittadina, vi troviate a disagio in questo posto un po' selvaggio e, se vi occorre qualcosa, sono completamente ai vostri ordini. 

- Oh, no - disse Dolly. - Nei primi tempi sono stata a disagio; ma ora tutto si è aggiustato nel modo migliore, grazie alla mia vecchia njanja - disse, mostrando Matrëna Filimonovna, la quale, avendo compreso che si parlava di lei, sorrideva a Levin allegra e cordiale. Lo conosceva e giudicava che sarebbe stato un buon marito per la signorina e desiderava che la faccenda si concludesse. 

- Vogliate sedervi, ci stringeremo in qua - egli disse. 

- No, andrò a piedi. Ohé, ragazzi, chi di voi viene con me a fare a chi arriva prima coi cavalli? 

I bambini conoscevano molto poco Levin, non ricordavano neppure se e quando l'avessero visto, ma non mostrarono nei suoi riguardi quello strano senso di timidezza e repulsione che tanto spesso i bambini provano per le persone adulte che fingono, e per cui sono spesso così dolorosamente puniti. La finzione può ingannare, in ogni caso, la persona più intelligente e accorta, ma non il bambino, anche il più sciocco, ché, per quanto abilmente nascosta, la riconosce e se ne ritrae. Quali che fossero i difetti di Levin, di finzione non esisteva in lui neppure il più piccolo segno, e perciò i bambini gli mostrarono una simpatia pari a quella che scorsero sul viso della madre. All'invito di Levin, i due più grandi saltarono subito giù dalla carrozza, e corsero con lui così semplicemente come avrebbero corso con la njanja o con miss Hull o con la madre. Anche Lily volle andare da lui e la madre gliela consegnò; egli la mise a sedere su di una spalla e corse con lei. 

- Non abbiate paura, non abbiate paura, Dar'ja Aleksandrovna! - diceva sorridendo allegramente alla madre. - Non è possibile che le faccia del male e la faccia cadere. 

E guardando i suoi movimenti agili, forti, prudentemente accorti, e fin troppo tesi, la madre si tranquillizzò e sorrise allegra, guardandolo con approvazione. 

Ora, in campagna, con i bambini e Dar'ja Aleksandrovna che gli era simpatica, Levin si abbandonò a quella disposizione d'animo infantilmente gioiosa che Dar'ja Aleksandrovna amava tanto in lui. Correndo con i bambini, egli insegnava loro la ginnastica, faceva ridere miss Hull col suo pessimo inglese, e raccontava a Dar'ja Aleksandrovna le sue faccende campestri. 

Dopo pranzo Dar'ja Aleksandrovna, seduta sola con lui sul balcone, cominciò a parlare di Kitty. 

- Sapete? Kitty verrà qui e passerà l'estate con me. 

- Davvero? - egli disse, accendendosi, e subito, per cambiar discorso, disse: - Così vi devo mandare due mucche? Se volete regolare dal punto di vista economico la faccenda, allora vogliate pagarmi cinque rubli al mese, se non vi rincresce. 

- No, grazie, abbiamo già provveduto. 

- Su, allora andrò a dare un'occhiata alle vostre mucche, e se permettete darò delle istruzioni per nutrirle. Tutto sta nel foraggio. 

E Levin, solo per stornare il discorso, espose a Dar'ja Aleksandrovna la teoria sull'industria del latte che consisteva nel far conto che ogni vacca altro non fosse che una macchina per la trasformazione del foraggio in latte, e via di seguito. 

Egli parlava di queste cose, ma nello stesso tempo desiderava con tutta l'anima sentire particolari di Kitty pur temendo di averne. Lo terrorizzava il sospetto che potesse esserne sconvolta la propria calma conquistata. 

- Sì, è giusto, tutto questo va seguìto, ma chi lo farà? - rispondeva controvoglia Dar'ja Aleksandrovna. 

Ella aveva finalmente assestato le faccende domestiche per mezzo di Matrëna Filimonovna che non amava i cambiamenti; inoltre non credeva alla competenza di Levin in materia di economia rurale. Il ragionamento secondo cui la vacca è una macchina per fare il latte, la metteva in sospetto. Le pareva che ragionamenti simili potessero solo intralciare l'economia. Le pareva che molto più semplice fosse dare più foraggio e più beveraggio, come diceva Matrëna Filimonovna, a Petrucha e a Belopachaja e che il cuoco non prelevasse dalla cucina le risciacquature per farne usufruire la vacca della lavandaia. Questo sì che era chiaro. Invece i ragionamenti sul mangime farinoso erano vaghi e poco chiari. La verità però era ch'ella aveva voglia di parlare di Kitty. 

\capitolo{X}\label{x-2} 

- Kitty mi scrive che desidera soltanto solitudine e calma - disse Dolly dopo il silenzio sopravvenuto. 

- E come va la sua salute, meglio? - domandò Levin agitato. 

- Grazie a Dio, è guarita del tutto. Io non ho mai creduto che avesse una malattia di petto. 

- Ah, son molto contento! - disse Levin, e a Dolly parve di scorgere una certa emozione, come una distensione, nel viso di lui mentre diceva questo e la guardava in silenzio. 

- Sentite, Konstantin Dmitric - disse Dar'ja Aleksandrovna, sorridendo del suo sorriso buono e un po' canzonatorio - perché siete arrabbiato con Kitty? 

- Io? Io non sono arrabbiato. 

- No, voi siete arrabbiato. Perché non siete passato né da noi né da loro quando siete stato a Mosca? 

- Dar'ja Aleksandrovna - disse egli, arrossendo fino alla radice dei capelli - mi meraviglio perfino che voi, tanto buona, non lo abbiate compreso. Come è che non avete almeno pietà di me, quando sapete\ldots{} 

- Cosa so? 

- Sapete che ho fatto una domanda di matrimonio e mi si è detto di no - pronunciò Levin e tutta la tenerezza che un momento prima lo aveva invaso per Kitty si tramutò in un senso di rancore per l'offesa ricevutane. 

- E perché credete che io lo sappia? 

- Perché tutti lo sanno. 

- Ecco, già in questo vi sbagliate; io non lo sapevo, pur immaginandolo. 

- Ebbene, lo sapete ora. 

- Sapevo soltanto che c'era stato qualcosa che l'aveva terribilmente tormentata, e che mi si pregava di non parlare mai di questo. E se non l'ha detto a me, non si è confidata con nessuno. Ma cosa mai vi è accaduto? Ditemelo. 

- Vi ho detto quello che è accaduto. 

- Quando? 

- Quando sono stato l'ultima volta da voi. 

- E sapete cosa vi dirò - disse Dar'ja Aleksandrovna: - ho un'enorme, enorme compassione di lei. Voi soffrite solo per orgoglio. 

- Può darsi - disse Levin - ma\ldots{} 

Ella lo interruppe. 

- Ma di lei, poveretta, ho un'enorme, enorme compassione. Adesso capisco tutto. 

- Eh, Dar'ja Aleksandrovna, scusatemi - diss'egli alzandosi. - Addio, Dar'ja Aleksandrovna, arrivederci. 

- No, aspettate - diss'ella agguantandolo per una manica. - Aspettate, sedetevi. 

- Vi prego, vi prego, non parliamo di questo - egli disse, sedendosi e sentendo nello stesso tempo che nel suo cuore si sollevava e s'agitava una speranza che gli era parsa sepolta. 

- Se non vi volessi bene - disse Dar'ja Aleksandrovna, e le vennero le lacrime agli occhi - se non vi conoscessi come vi conosco\ldots{} 

Il sentimento che era parso morto si ravvivava sempre di più, si sollevava e si impadroniva del cuore di Levin. 

- Sì, adesso ho capito tutto - proseguì Dar'ja Aleksandrovna. - Voi non potete capire questo; per voi, uomini, che siete liberi e potete scegliere, è sempre chiaro chi amate. Ma una ragazza nello stato d'attesa, col suo pudore femminile, virginale, una ragazza che vede voi, uomini, di lontano, prende tutto sulla parola; una ragazza ha e può avere un sentimento tale, da non saper cosa dire. 

- Sì, se il cuore non parla\ldots{} 

- No, il cuore parla, ma pensate: voi, uomini, avete delle intenzioni su una ragazza, andate in casa, fate amicizia, osservate, aspettate per vedere se troverete quel che vi piace, e poi, quando siete convinti di amare, fate la proposta di matrimonio\ldots{} 

- Via, non è affatto così. 

- È lo stesso, voi fate la proposta di matrimonio quando il vostro amore è venuto a maturità o quando fra due da scegliere s'è fatto il soprappeso. Ma una ragazza non la interrogano. Vogliono che ella scelga da sé, ma lei non può scegliere e risponde soltanto: ``sì'' e ``no''. 

``Sì, la scelta fra me e Vronskij'' pensò Levin e il cadavere che si ravvivava nell'anima sua morì di nuovo e premeva solo tormentosamente il suo cuore. 

- Dar'ja Aleksandrovna - diss'egli - così si sceglie un vestito, oppure non so che compera, ma non l'amore. La scelta è fatta, e tanto meglio\ldots{} E una ripetizione non può esserci. 

- Ah, quanto orgoglio, quanto orgoglio! - disse Dar'ja Aleksandrovna, quasi disprezzandolo per questo suo sentire che le appariva ben basso in confronto del sentimento che solo le donne conoscono. - Mentre voi facevate la proposta a Kitty, lei si trovava proprio nella situazione di non poter rispondere. C'era in lei il dilemma: o voi o Vronskij. Vedeva lui ogni giorno, e non vedeva voi da tempo. Supponiamo che fosse stata meno giovane: ad esempio, per me, al suo posto, non ci sarebbe stato dilemma. Mi era sempre stato antipatico quello lì, e i fatti\ldots{} 

Levin ricordò la risposta di Kitty. Ella aveva detto: ``No, questo non può essere\ldots{}''. 

- Dar'ja Aleksandrovna - egli disse asciutto - io apprezzo la vostra fiducia in me; penso che non sia giusta. Ma che io abbia ragione o torto\ldots{} questo orgoglio che voi disprezzate tanto, fa sì che ogni mio disegno su Katerina Aleksandrovna sia impossibile, voi comprendete, completamente impossibile. 

- Io dirò ancora una cosa sola: voi capite che parlo di una sorella che amo come i miei figli. Non dico che ella vi ami, ma volevo soltanto dire che il suo rifiuto in quel momento non significa nulla. 

- Non so - disse Levin, alzandosi. - Se sapeste quanto male mi fate! È come se vi fosse morto un bambino e vi si dicesse: ecco, era così e così, avrebbe potuto vivere e voi avreste potuto gioirne. Ed è morto, morto, morto\ldots{} 

- Come siete strano! - disse Dar'ja Aleksandrovna, osservando con un sorriso triste l'agitazione di Levin. - Sì, adesso capisco sempre di più - continuò pensosa. - Così voi non verrete da noi quando ci sarà Kitty. 

- No, non verrò. S'intende, io non sfuggirò Katerina Aleksandrovna, ma ogni volta che potrò, cercherò di liberarla del fastidio della mia presenza. 

- Siete molto, molto strano - ripeté Dar'ja Aleksandrovna, guardandolo con tenerezza in viso. - Su, va bene, non parliamone più. Perché sei venuta, Tanja? - disse Dar'ja Aleksandrovna in francese alla bambina che era entrata. 

- Dov'è la mia paletta, mamma? 

- Io ti sto parlando in francese, e tu rispondimi in francese. 

La bambina voleva, ma aveva dimenticato come si dice ``paletta'' in francese; la madre le suggerì la parola e poi, sempre in francese, disse dove poteva trovarla. Questo spiacque a Levin. 

Tutto ormai in casa di Dar'ja Aleksandrovna non gli piaceva più come poco prima, perfino i bambini. 

``E perché parla in francese coi bambini? - pensò. - Com'è poco naturale, com'è falso! E i bambini lo sentono. Si fa loro apprendere il francese e disimparare la sincerità'' e non sapeva che Dar'ja Aleksandrovna aveva già cambiato idea venti volte a questo proposito e tuttavia, pure a danno della sincerità, aveva trovato indispensabile educare in tal modo i figliuoli. 

- Ma dove mai dovete andare? Rimanete. 

Levin rimase lì fino al tè, ma la sua allegria era scomparsa e si sentiva ormai a disagio. 

Dopo il tè uscì in anticamera per ordinare di fare accostare al portone i cavalli; e quando rientrò trovò Dar'ja Aleksandrovna col viso sconvolto e le lacrime agli occhi. Durante la breve assenza di Levin era accaduto qualcosa che aveva distrutto in Dar'ja Aleksandrovna tutta la felicità di quel giorno e tutto il suo orgoglio per i suoi bambini. Griša e Tanja s'erano azzuffati per una palla. Dar'ja Aleksandrovna, sentendo gridare nella camera dei bambini, era accorsa e li aveva trovati avvinti in modo orribile: Tanja aveva afferrato Griša per i capelli e questi, col volto mostruoso di cattiveria, tirava pugni dove capitava. Nel vedere questa scena, qualcosa si era spezzato nel cuore di Dar'ja Aleksandrovna. Come se le tenebre si fossero addensate sulla sua vita, capiva che quei bambini, che erano il suo vanto, erano non solo bambini come tanti altri, ma tutt'altro che buoni, male educati, con tendenze perverse e volgari, bambini cattivi. 

Non poteva pensare ad altro, non poteva parlare d'altro e non seppe trattenersi dal raccontare a Levin la sua pena. 

Levin vedeva ch'ella soffriva, e cercava di consolarla, dicendo che non si trattava di cosa grave, che tutti i bambini vengono alle mani; eppure, mentre diceva questo, pensava dentro di sé: ``No, io non farò moine ai miei figli e non parlerò in francese con loro; ma non avrò bambini come questi. Non si devon guastare, non si devono deformare i bambini e allora saranno deliziosi. No, io non avrò bambini come questi''. 

Si accomiatò e andò via, né lei lo trattenne. 

\capitolo{XI}\label{xi-2} 

Verso la metà di luglio si presentò da Levin lo starosta del villaggio di sua sorella che era a venti verste da Pokrovskoe, a render conto dell'andamento degli affari e della fienagione. La rendita principale della tenuta della sorella di Levin proveniva dai prati fertilizzati dalle inondazioni. Negli anni precedenti, la fienagione era stata fatta dai contadini per venti rubli a desjatina. Quando Levin assunse l'amministrazione del fondo, esaminata la questione, aveva trovato che la fienagione poteva rendere di più e ne aveva fissato il prezzo a venticinque rubli per desjatina. 

I contadini non avevano accettato questo prezzo e, come aveva sospettato Levin, avevano allontanato gli altri compratori. Allora, recatosi sul posto, dispose che si raccogliesse il fieno, in parte per ingaggio e in parte trattenendo una quota. I contadini intralciarono con tutti i mezzi l'innovazione, ma l'affare andò bene e, fin dal primo anno, i prati resero quasi il doppio di prima. Per due anni ancora, fino all'anno precedente, era continuata la medesima reazione da parte dei contadini, ma il raccolto era andato bene lo stesso. Quell'anno poi i contadini si erano assunta tutta la fienagione, trattenendo per loro un terzo del prodotto e ora lo starosta era venuto a dire che il fieno era stato raccolto e che, temendo la pioggia, egli aveva fatto venire l'amministratore, aveva fatto la divisione e aveva ammucchiato, presente lui, undici covoni di fieno quale spettanza padronale. Dalle risposte evasive alla domanda sul quantitativo di fieno raccolto nel prato più esteso, dalla fretta con la quale aveva proceduto alla ripartizione del prodotto senza chiedere l'autorizzazione, da tutto il tono del discorso del contadino, Levin aveva capito che in quella divisione qualcosa di poco chiaro c'era stato e decise di andare lui stesso a sincerarsene. 

Giunto al villaggio all'ora di pranzo e lasciato il cavallo da un suo vecchio amico, marito della balia del fratello, Levin, desideroso di conoscere egli stesso i particolari della falciatura, si recò dal vecchio addetto alle api che si trovava nell'arniaio. Questi, il vecchio Parmenyc, loquace e bello a vedersi, accolse festoso Levin; gli mostrò tutta la sua azienda, si diffuse in mille particolari circa le api e la sciamatura di quell'anno; ma alle domande di Levin riguardanti la fienagione, rispose impreciso e svogliato. Questo confermò ancor più Levin nei suoi sospetti. Andò allora sui prati per vedere le biche del fieno. Da ciascuna bica non ne potevano venir fuori cinquanta carrettate e, per convincere di ciò i contadini, Levin ordinò di chiamare subito i carri da trasporto, di caricarvi su una bica e di trasportarla nella rimessa. Ne vennero fuori trentadue carrettate solamente. Malgrado lo starosta assicurasse che il fieno si rigonfia e poi nelle biche si abbassa e malgrado spergiurasse che tutto questo era andato secondo il volere di Dio, Levin rimase fermo nella sua determinazione: il fieno era stato diviso senza suo ordine e perciò egli non accettava quel fieno per cinquanta carrettate a bica. Dopo lunghe discussioni decisero la questione nel senso che i contadini avrebbero ritenute per sé quelle undici biche, calcolandole a cinquanta carrettate ciascuna, e che per determinare la parte spettante al padrone, si sarebbe fatta una nuova ripartizione. Tra trattative e decisioni si giunse all'ora del pranzo. Mentre si divideva l'ultimo fieno, Levin, affidato il rimanente controllo all'amministratore, andò a sedersi su una bica segnata da una canna e se ne stette a contemplare il prato che brulicava di gente. 

Dinanzi a lui, in un'ansa del fiume, al di là di un pantanello, si moveva allegramente, scoppiettando di voci sonore, una schiera variegata di donne, e su per il guaime verdechiaro si distendevano le onde grigie, sinuose, del fieno sparpagliato. Dietro le donne venivano i contadini coi forconi, e dalle onde del fieno si formavano alte, larghe e rigonfie le biche. A sinistra del prato già sgombro risonavano i carri e, tirate giù a grandi forcate, una dopo l'altra, le biche scomparivano e al loro posto venivano caricate pesanti carrettate del fieno profumato che sporgevano fin sulla groppa dei cavalli. 

- Bisogna raccoglierlo col tempo buono! Fieno ce ne sarà - disse il vecchio, sedendosi accanto a Levin. - Sembra tè, e non fieno! Guarda come lo tirano su! Sembra che abbian dato a beccare i semi agli anatroccoli! - aggiunse mostrando le biche che venivano caricate. - Dall'ora di pranzo ne han caricate una buona metà. - È l'ultimo carro, eh? - gridò rivolgendosi a un giovanotto che, in piedi sul fondo del carro e agitando le estremità delle redini di canapa, gli passava accanto. 

- L'ultimo, batjuška - gridò il giovane, trattenendo per un momento il cavallo; poi, sorridendo, si voltò a guardare la contadina, allegra, rubiconda e sorridente anch'essa, che sedeva nella parte posteriore del carro. 

- E questo chi è, un figliuolo? - chiese Levin. 

- L'ultimo - disse il vecchio con un sorriso carezzevole. 

- Che bel ragazzo! 

- Non c'è male. 

- È già sposato? 

- Sì, son tre anni, alla vigilia di san Filippo. 

- Be', e figliuoli niente? 

- Macché figliuoli! Per un anno intero si è comportato come uno stupido\ldots{} c'era da vergognarsi - rispose il vecchio. - Su, il fieno! È vero tè! - ripeté, desiderando cambiar discorso. 

Levin osservò attento Van'ka Parmenov e la moglie. Non lontano da lui la coppia caricava una bica. Ivan Parmenov stava diritto sul carro, riceveva il fieno, eguagliava e pestava le forcate colme, che gli tendeva, agile, la giovane bella moglie, prima a bracciate, poi con la forca. La donna lavorava con facilità, allegra e svelta. Il fieno grosso, riscaldato, non veniva subito alla forca. Ella dapprima l'assestava, ci conficcava dentro il bidente, poi, con un movimento elastico e veloce, vi si appoggiava con tutto il peso del corpo e subito, piegando la schiena stretta da una cintura rossa, si raddrizzava e, sporgendo il seno turgido sotto la pettina bianca, con una mossa rapida afferrava la forca e gettava la forcata su, in alto, sul carro. Ivan, svelto, cercando evidentemente di risparmiarle ogni più piccolo sforzo, afferrava, aprendo larghe le braccia, la forcata che gli veniva tesa e l'assestava sul carro. Tesogli l'ultimo fieno raccolto col rastrello, la donna scosse via le festuche che le si erano insinuate intorno al collo e, rimesso a posto il fazzoletto rosso sulla fronte bianca non ancora abbronzata, si ficcò sotto il carro per legare il carico. Ivan le diceva come dovesse agganciarlo alla freccia, e a una cosa detta da lei, scoppiò a ridere forte. Nell'espressione di tutti e due i volti c'era un amore forte, giovane, risvegliato da poco. 

\capitolo{XII}\label{xii-2} 

Il carico fu legato. Ivan saltò giù e condusse per la briglia il cavallo ben nutrito. La donna gettò il rastrello sul carro e con passo sicuro, agitando le braccia, andò verso le contadine che s'erano riunite formando quasi un cerchio. Ivan, raggiunta la strada, entrò a far parte del convoglio dei carri. Le donne, col rastrello sulle spalle, splendendo nei colori vivaci delle vesti e facendo strepito con le voci sonore, seguivano allegre i carri. Una voce rozza, selvatica di donna, intonò una canzone e la cantò fino al ritornello; poi, insieme, in coro, una cinquantina di voci varie, sane, robuste e acute, ripresero daccapo la stessa canzone. 

Le donne, così cantando, si avvicinarono a Levin, e a lui sembrò che una nuvola e un tuono gonfio di allegria si avanzasse venendogli incontro. La nuvola avanzò, lo afferrò, e la bica sulla quale era sdraiato e le altre biche e i carri e il bosco intero, il campo lontano, tutto cominciò a camminare e ad agitarsi al ritmo di quella selvaggia, ilare canzone tra gridi, sibili e sobbalzi. Levin ebbe invidia di quella sana allegria, e avrebbe voluto prender parte a quella gioiosa espressione di vita. Ma non poteva, e doveva tacere, guardare e ascoltare. Quando quella gente e il canto si sottrassero alla sua vista e al suo udito, un senso grave di tristezza per la propria solitudine, per il proprio ozio fisico, per la propria avversione verso quel mondo lo afferrò. 

Alcuni di quegli stessi contadini che più di tutti avevano discusso con lui per il fieno, quelli ch'egli aveva offeso o quelli che volevano ingannarlo, questi stessi contadini lo salutavano ora allegramente, e appariva chiaro che non avevano e che non avrebbero potuto avere verso di lui alcun'ombra di rancore, e non solo nessun pentimento, ma neppure più il ricordo di aver cercato di ingannarlo. Tutto era affondato nel gioioso mare del lavoro comune. Dio dà il giorno, Dio dà la forza. E il giorno e la forza sono consacrati al lavoro; e in se stesso il lavoro trova la sua ricompensa. Ma per chi il lavoro? Quali saranno i frutti del lavoro? Queste sono considerazioni secondarie e inconsistenti. Spesso Levin aveva ammirato questa vita, aveva spesso provato invidia per le persone che la vivevano; ma ora per la prima volta, sotto l'impressione dei rapporti che aveva notato tra Parmenov e la sua giovane moglie, per la prima volta si delineò chiara nella mente di Levin l'idea che dipendeva da lui sostituire alla vita che viveva, così greve e oziosa, così artificiale ed egoista, questa deliziosa, semplice vita di lavoro fatto in comune. 

Il vecchio, che era stato a sedere accanto a lui, già da tempo se n'era andato; la gente si era tutta dispersa. Quelli che abitavano vicini erano andati alle loro case e quelli che stavano lontano s'eran raccolti per cenare e dormire sul prato. Levin, inosservato, continuava a starsene sdraiato sulla bica e a guardare, ad ascoltare e riflettere. La gente che era rimasta a passar la notte nel prato, non dormì quasi, in quella breve notte estiva. Dapprima, durante la cena, si udirono chiacchiere e risa, poi di nuovo canti e risa. 

Tutta la lunga giornata di lavoro non aveva lasciato in quei contadini altra traccia che l'allegria. Prima dell'alba tutto si chetò. Si udivano solo l'incessante gracidare notturno delle ranocchie nella palude e lo sbuffare dei cavalli in giro per il prato nella nebbia levatasi all'alba. 

Rientrato in sé, Levin si alzò dalla bica e, guardate le stelle, capì che la notte era passata. 

``Be', e allora che farò? Come farò tutto questo?'' si chiese, cercando di esprimere a se stesso ciò che aveva pensato e sentito in quella breve nottata. Tutto quello che aveva pensato e sentito si divideva in tre distinti ordini di pensieri. Il primo comprendeva la rinuncia alla sua vecchia vita, alle cognizioni inutili, alla cultura che non serviva a nulla. Questa rinuncia l'avrebbe fatta con piacere, ed era per lui facile e semplice. Gli altri pensieri e le altre figurazioni riguardavano la vita che egli avrebbe voluto vivere ora. Sentiva chiaramente la purezza e la semplicità, la legittimità di questa vita, ed era convinto che vi avrebbe trovato quella soddisfazione tranquilla e dignitosa di cui avvertiva così morbosamente l'assenza. Ma l'ultimo ordine di idee si aggirava sul modo di compiere questo passaggio dalla vecchia vita alla nuova. E qui non vedeva più con chiarezza. ``Prender moglie? Avere un lavoro e la necessità di lavorare? Lasciare Pokrovskoe? Comprare della terra? Iscriversi in una società? Sposare una contadina? Come farò tutto questo? - si chiedeva di nuovo e non trovava risposta. - Ma io non ho dormito tutta la notte e non posso darmi una risposta chiara - si disse. - Chiarirò dopo. Una cosa è certa, che questa notte ho deciso il mio destino. Tutti i miei precedenti sogni sulla vita familiare non rispondono a questo. Questo sì, è molto più semplice ed è certamente migliore''. 

``Com'è bello! - pensò, guardando la strana conchiglia quasi madreperlacea di nuvole bianche a pecorelle fermatasi proprio sulla sua testa, a mezzo il cielo. - Come tutto è bello in questa notte luminosa! E quando ha avuto il tempo di formarsi questa conchiglia? Un momento fa ho guardato il cielo e non c'era nulla, solo due strisce bianche. Ecco, proprio così, inavvertitamente, sono cambiate le mie idee sulla vita!''. 

Uscì dal prato e andò verso il villaggio per la strada maestra. S'era levato un po' di vento e il tempo s'era fatto grigio, scuro. Era il momento della foschia che di solito precede l'alba, la vittoria completa della luce sulle tenebre. 

Rannicchiandosi in se stesso per il freddo, Levin camminava in fretta, guardando a terra. ``Cos'è questo? Viene qualcuno'' pensò, sentendo rumore di sonagli, e sollevò la testa. A quaranta passi da lui, venendogli incontro sulla strada maestra sulla quale egli camminava, avanzava un tiro a quattro con le valigie sull'imperiale. I cavalli del centro si stringevano all'asse per evitare le carreggiate, ma il cocchiere abile, che sedeva di sghembo a cassetta, teneva l'asse sopra una carreggiata sola, di modo che le ruote correvano sul liscio. 

Levin notò questo soltanto e, senza pensare a chi potesse trovarsi in viaggio, guardò distratto la vettura. 

Nella vettura sonnecchiava, in un angolo, una signora anziana e, al finestrino, evidentemente da poco risvegliata, sedeva una giovinetta che teneva stretti con le due mani i nastri di una cuffia bianca. Limpida e pensosa, tutta piena di una vita interiore bella e complessa, ignota a Levin, guardava, di là da lui, il sopravvenire dell'alba. 

Al momento in cui la visione stava per scomparire due occhi sinceri lo guardarono. Ella lo riconobbe, e uno stupore gioioso illuminò il suo viso. 

Egli non poteva sbagliarsi. Unici al mondo erano quegli occhi. Per lui un solo essere c'era al mondo capace di esprimere tutta la luce e tutto il senso della vita. Era lei. Kitty. Egli capì che veniva dalla stazione ferroviaria e andava a Ergušovo. E tutto quello che aveva agitato Levin in quella notte insonne, tutte quelle decisioni che egli aveva prese, tutto improvvisamente scomparve. Ricordò con ripugnanza la sua fantasticheria di sposare una contadina. Soltanto là, in quella vettura che si allontanava veloce e che era passata dall'altra parte della strada, soltanto là c'era la possibilità di risolvere l'enigma della sua vita che così tormentosamente l'opprimeva negli ultimi tempi. 

Ella non guardò più fuori. Il suono delle balestre cessò d'essere percettibile, si udì appena un bubbolio di campanelli. L'abbaiare dei cani significò che la vettura era passata attraverso il villaggio. Intorno rimasero i campi deserti, più avanti il villaggio e lui che, solitario ed estraneo a tutto, camminava solo sulla strada abbandonata. 

Guardò il cielo, sperando di trovarvi ancora quella conchiglia che aveva ammirato e che rappresentava per lui tutto il corso dei suoi pensieri e dei suoi sentimenti di quella notte. In cielo non c'era più nulla che somigliasse a una conchiglia. Ad altezza irraggiungibile, s'era compiuto già un misterioso cambiamento. Non c'era più traccia di conchiglia e c'era invece un tappeto eguale e disteso per l'intera metà del cielo di nuvole a pecorelle che rimpicciolivano sempre più. Il cielo s'inazzurrava e schiariva e con la stessa tenerezza, ma anche con la stessa irraggiungibilità, rispondeva al suo sguardo indagatore. 

``No - egli si disse - per quanto bella sia questa vita semplice e laboriosa, io non posso tornarvi. Io amo lei''. 

\capitolo{XIII}\label{xiii-2} 

Nessuno, ad eccezione delle persone che vivevano proprio vicino ad Aleksej Aleksandrovic, sapeva che quest'uomo dall'aspetto impassibile, quest'uomo ragionatore avesse una debolezza contrastante con la linea fondamentale del suo carattere. Aleksej Aleksandrovic non poteva sentire e vedere con indifferenza le lacrime di un bambino o di una donna. La vista delle lacrime produceva in lui come un certo smarrimento, e gli faceva perdere del tutto la facoltà di riflettere. Il capo della sua cancelleria e il suo segretario, che sapevano ciò, avvertivano le postulanti di reprimere con ogni sforzo le lacrime, se non volevano guastare le loro cose. ``Si arrabbierà, smetterà di ascoltarvi'', dicevano. Infatti, in simili casi, il turbamento prodotto dalle lacrime in Aleksej Aleksandrovic si esprimeva con uno scatto d'ira: ``Non posso, non posso far nulla. Vi prego di andarvene'' gridava in simili casi. 

Quando, tornando dalle corse, Anna gli ebbe rivelato i suoi rapporti con Vronskij e, subito dopo, copertosi il viso con le mani, era scoppiata a piangere, Aleksej Aleksandrovic, malgrado il rancore che provava verso di lei, sentì nello stesso tempo un afflusso di quello sconvolgimento d'animo che sempre producevano in lui le lacrime. Sapendo ciò e sapendo che in quel momento la manifestazione dei suoi sentimenti non sarebbe stata adatta alla situazione, si era sforzato di contenere dentro di sé ogni moto di vita, e perciò era rimasto immobile senza guardarla. Proprio da questo gli era venuto nel volto quello strano livore di morte che aveva tanto colpito Anna. 

Quando furono giunti a casa, egli l'aiutò a scendere dalla vettura e, energicamente dominandosi, la salutò con l'abituale cortesia e disse quelle parole che non lo impegnavano in nulla: disse che l'indomani le avrebbe comunicata la sua decisione. 

Le parole della moglie, che erano venute a confermare i suoi sospetti peggiori, avevano prodotto un dolore atroce nel cuore di Aleksej Aleksandrovic. Questo dolore era reso ancora più aspro da quello strano senso di pena fisica verso di lei che gli avevano prodotto le lacrime. Ma, rimasto solo nella vettura, con gioiosa sorpresa, si sentì completamente liberato e da questa pena e dai dubbi e dai tormenti che negli ultimi tempi lo avevano fatto soffrire. 

Provava la sensazione di chi si fosse fatto cavare un dente che da tempo dolesse. Dopo un dolore tremendo e la sensazione che una cosa enorme, più grande della propria testa, sia stata tolta dalla mascella, il paziente, pur non credendo ancora al riacquistato benessere, sente però che non c'è più la cosa che per lungo tempo gli ha avvelenato l'esistenza, avvincendo tutta la sua attenzione; sente che può di nuovo vivere, pensare e interessarsi non più esclusivamente al proprio dente. Questa era la sensazione che provava Aleksej Aleksandrovic. Il suo dolore era stato non comune e terribile, ma era passato; ora egli sentiva che poteva di nuovo vivere e non pensare esclusivamente a sua moglie. 

``Senza onore e senza cuore e senza religione: una donna depravata. L'ho sempre pensato e sempre constatato, sebbene cercassi, per compassione, di ingannare me stesso'' diceva tra sé. E gli pareva davvero d'aver sempre constatato ciò; richiamò alla mente i particolari della vita di lei, nei quali prima non aveva notato mai nulla di riprovevole: adesso questi stessi particolari diventavano la prova evidente che ella era sempre stata una donna depravata. ``Ho sbagliato nel legare a lei la mia vita, ma nulla c'è di censurabile in questo mio errore, e non debbo per questo rendermi infelice. Non sono io il colpevole - diceva a se stesso - ma lei. Essa non esiste più per me''. 

Tutto quello che sarebbe poi accaduto di lei e del figlio, verso il quale, così come verso di lei, si eran mutati i suoi sentimenti, non lo interessava più. Ora occorreva una cosa sola: trovare il modo migliore, più decoroso e che importasse il minor sacrificio da parte sua, e fosse perciò il più giusto, per scuotersi quel fango ch'ella, cadendo, gli aveva schizzato addosso, e riprendere il cammino della sua vita attiva, onesta, utile. ``Non può rendere infelice la mia vita il fatto che una donna disprezzabile si sia resa colpevole; io debbo soltanto trovare una via d'uscita da questa situazione penosa nella quale ella mi ha posto. E la troverò - diceva a se stesso, accigliandosi sempre più. - Non sono il primo, e non sarò l'ultimo''. E per non parlare degli esempi storici, a cominciar da Menelao, tornato nella mente di tutti per La bella Elena, tutta una serie di esempi contemporanei di infedeltà di mogli verso mariti dell'alta società si presentò alla mente di Aleksej Aleksandrovic. ``Dar'jalov, Poltavskij, il principe Karibanov, il conte Paskudin; Dram\ldots{} sì\ldots{} anche Dram, uomo così degno e laborioso\ldots{} Semënov, cagin, Sigonin - ricordava. - Poniamo che un certo irragionevole ridicule cada su queste persone, ma io in questo non ho visto mai altro che una sventura e ne ho sempre avuto pietà - si andava dicendo Aleksej Aleksandrovic, e non era vero che egli avesse avuto compassione per sventure di tal genere; anzi, quanto più frequenti erano i casi di mogli che tradivano i mariti, tanto più in alto egli si considerava. - Questa è una sventura che può capitare a chiunque. E questa sventura ora ha colpito me. Tutto sta nel risolvere la situazione nel miglior modo possibile''. E cominciò a elencare ed esaminare i diversi modi con i quali avevano reagito le persone che si erano trovate in una posizione simile alla sua. 

``Dar'jalov si è battuto in duello\ldots{}''. 

Nella sua giovinezza, il duello, per Aleksej Aleksandrovic, era stato addirittura un incubo, perché egli era costituzionalmente un pavido, e lo sapeva bene. Aleksej Aleksandrovic non poteva pensare a una pistola puntata contro di lui senza inorridire, e in vita sua non aveva mai adoperato armi. Questo suo orrore lo aveva indotto, quando era giovane, a pensare spesso al duello e a immaginare se stesso in una situazione in cui fosse necessario esporre la propria vita al pericolo. Raggiunto poi il successo e una solida posizione sociale, aveva dimenticato da tempo questa sensazione; ma l'averci troppo pensato prevalse ora, e il terrore della propria vigliaccheria, anche adesso, gli parve così grande, che Aleksej Aleksandrovic indugiò a lungo e da ogni verso ad accarezzare col pensiero la soluzione del duello, pur sapendo in precedenza che mai, per nessuna ragione al mondo, si sarebbe battuto. 

``Senza dubbio la nostra società è tuttora tanto barbara (non così in Inghilterra), che moltissimi - e nel numero di questi `moltissimi' vi erano proprio le persone alla cui opinione Aleksej Aleksandrovic maggiormente teneva - sarebbero favorevoli al duello; ma quale risultato mai si raggiungerebbe? Mettiamo che io lo sfidi - continuò tra sé Aleksej Aleksandrovic mentre, rappresentatasi con chiarezza la notte che avrebbe passata dopo aver mandata la sfida, e la pistola puntata contro di lui, rabbrividiva e capiva che questo non sarebbe mai avvenuto - ammettiamo che io lo sfidi a duello. Ammettiamo che mi insegnino come fare - continuò a pensare - che mi mettano in posizione\ldots{} io premo il grilletto - si disse socchiudendo gli occhi - e ammettiamo che lo uccida - disse fra sé Aleksej Aleksandrovic e scosse il capo per scacciare questi sciocchi pensieri. - Quale senso può avere l'uccisione di un uomo al fine di regolare i propri rapporti con una moglie infedele e un figlio? Dovrei poi sempre necessariamente decidere come regolarmi con lei. Ma la cosa ancor più probabile, e che senza dubbio accadrebbe, è che sarei io a rimanere ucciso o ferito. Io, vittima innocente, ucciso o ferito. Cosa ancor più insensata. Ma questo è il meno: una sfida sarebbe, da parte mia, un'azione disonesta. Non so io, forse, che i miei amici eviterebbero a tutti i costi di farmi scendere sul terreno, che non lascerebbero mai esporre al pericolo la vita di un uomo di stato, necessario alla Russia? E che cosa accadrebbe allora? Che io, sicuro di non giungere al momento pericoloso, avrei cercato, con questa sfida, di acquistare soltanto un falso prestigio. E questo è disonesto, è falso, è un ingannare gli altri e se stesso. La soluzione del duello è, dunque, assurda, e nessuno la esige da me. Il mio scopo è quello di garantire la mia reputazione, quella reputazione che mi è necessaria per poter proseguire senza ostacoli la mia attività''. La sua attività al servizio dello stato che, anche prima, aveva un grande valore agli occhi di Aleksej Aleksandrovic, gli appariva ora particolarmente rilevante. 

Esaminata e respinta la soluzione del duello, passò a considerare quella offerta dal divorzio, altro espediente cui eran ricorsi alcuni di quei mariti ch'egli aveva ricordato. Passando in rassegna tutti i casi di divorzio a lui noti (molti ce ne erano stati nell'alta società), non ne trovò neppure uno in cui il divorzio avesse ottenuto lo scopo ch'egli perseguiva. In tutti quei casi il marito aveva o ceduto o venduto la moglie infedele, e quella stessa parte che, in conseguenza della propria colpa, non avrebbe più dovuto avere diritto a contrarre matrimonio, entrava in rapporti legalizzati, sia pure in modo fittizio e immaginario, con il nuovo coniuge. Inoltre, nel caso suo personale, non era possibile conseguire un divorzio legale, cioè un divorzio nel quale fosse proclamata solo la colpevolezza della moglie. Egli vedeva che le condizioni complesse di vita in cui egli era, non ammettevano la possibilità di quelle prove volgari che la legge pretendeva per il riconoscimento della colpevolezza della moglie; vedeva che quella certa raffinatezza di ambiente in cui viveva non permetteva neppure l'uso di queste prove, se pure ci fossero state, e che queste prove avrebbero fatto decadere più lui che lei nella pubblica opinione. 

Un tentativo di divorzio poteva portare soltanto a un processo scandaloso che avrebbe offerto ai nemici la gradita occasione per infamare e avvilire la sua alta posizione nel mondo. E quindi lo scopo principale: risolvere la situazione col minimo turbamento possibile, non si sarebbe ottenuto nemmeno col divorzio. Era inoltre evidente che col divorzio, ed anche con una semplice minaccia di divorzio, la moglie avrebbe rotto ogni rapporto col marito e si sarebbe unita con l'amante. E nell'animo di Aleksej Aleksandrovic, malgrado la sua attuale, completa, come gli pareva, sprezzante indifferenza verso la moglie, un sentimento permaneva nei riguardi di lei: il desiderio che ella non riuscisse a unirsi senza ostacoli a Vronskij, ch'ella non ottenesse così un vantaggio dalla propria colpa. Questo pensiero solo irritava a tal punto Aleksej Aleksandrovic che, rappresentatosi con chiarezza la cosa, emise un mugolio di intimo dolore e si sollevò e cambiò posizione nella carrozza; poi a lungo, dopo questo, avviluppò, accigliato, le gambe ossute e freddolose nello scialle di lana. 

Calmatosi, continuava a pensare: ``Oltre al divorzio formale, si potrebbe ricorrere alla separazione legale, come hanno fatto Karibanov, Paskudin e quel bravo Dram''. Ma anche questa soluzione presentava gli stessi inconvenienti umilianti del divorzio, gettava sua moglie nella braccia di Vronskij. ``No, non è possibile, non è possibile, non è possibile - disse ad alta voce rigirandosi di nuovo lo scialle. - Io non devo essere infelice, e lei e lui non devono poter essere felici''. 

La gelosia che lo aveva tormentato quando non era ancora a conoscenza di tutto era svanita nel momento in cui con le parole della moglie gli era stato strappato il dente. Ma questo sentimento si era trasformato in un altro: nel desiderio che non solo ella non avesse a trionfare, ma che della propria colpa dovesse sopportare la pena. Egli non lo confessava questo sentimento, ma in fondo all'anima voleva che ella soffrisse per aver turbato la sua pace, e per aver offeso il suo onore. E di nuovo, riesaminate le soluzioni del duello, del divorzio e della separazione, e nuovamente respintale, Aleksej Aleksandrovic si convinse che l'unica via d'uscita era questa: trattenere presso di sé la moglie, nascondendo al mondo quello che era accaduto e adoperando tutti i mezzi per spezzare quella relazione e punire in tal modo, ma questo non lo confessava a se stesso, la colpa di lei. ``Devo annunciarle la mia decisione, che cioè, avendo riflettuto sulla penosa situazione nella quale ella ha posto la famiglia, tutte le vie d'uscita sarebbero, per entrambe le parti, peggiori di uno statu quo apparente; e che a questo consento a patto che sia rigorosamente osservata da parte sua la inderogabile mia volontà, che ella cioè tronchi ogni suo rapporto con l'amante''. A confermarlo in questa decisione, quando però già era stata presa in maniera definitiva, balenò nella mente di Aleksej Aleksandrovic un'altra importante considerazione. ``Soltanto in questo modo agirò anche secondo i precetti della religione, soltanto con questa decisione non respingerò da me la moglie colpevole e le offrirò la possibilità di ravvedersi e consacrerò perfino, per quanto possa essermi penoso, una parte delle mie energie alla sua riabilitazione e alla sua salvezza''. Sebbene Aleksej Aleksandrovic sapesse di non avere alcun ascendente morale sulla moglie, e che da tutto quel tentativo di ravvedimento sarebbe venuta fuori soltanto menzogna, pur non avendo mai pensato, anche in momenti penosi, neppure una volta sola a cercare una guida nella religione, ora che la sua decisione sembrava coincidere con i precetti religiosi, questa solenne sanzione alla propria decisione lo soddisfaceva in pieno e lo tranquillizzava in parte. Gli era di sollievo pensare che nessuno potesse dire che, pur in momenti così gravi della sua vita, egli non avesse agito in conformità alle regole di quella religione, la cui bandiera aveva sempre tenuta alta tra la freddezza e la indifferenza generali. Pensando a particolari lontani nel futuro, Aleksej Aleksandrovic vedeva anche possibile che i suoi rapporti con la moglie potessero permanere quasi gli stessi di prima. Certamente egli non avrebbe mai più potuto ridarle la stima; ma non c'era alcuna ragione che la vita di lui fosse sconvolta, e venisse a soffrire lui del fatto che sua moglie era stata infedele e perversa. ``Sì, passerà del tempo, il tempo che accomoda tutto, e si ristabiliranno i rapporti di prima - diceva a se stesso Aleksej Aleksandrovic - si ristabiliranno cioè in modo che io non subirò più alcun squilibrio nel corso della mia vita. Deve essere infelice lei che è colpevole, non io che non lo sono, e che perciò non posso essere infelice''. 

\capitolo{XIV}\label{xiv-2} 

Avvicinandosi a Pietroburgo, Aleksej Aleksandrovic non solo era risolutamente fermo in questa sua decisione, ma aveva già composta nella sua mente la lettera che avrebbe scritto alla moglie. Entrato in portineria, dette uno sguardo alle lettere e alle pratiche del ministero, e ordinò di portarle subito nel suo studio. 

- Rimandate chiunque: non ricevo nessuno - rispose alla domanda del portiere, accentuando le parole ``non ricevo nessuno'' con un certo compiacimento, il che indicava una buona disposizione di spirito. 

Giunto nello studio, Aleksej Aleksandrovic lo percorse due volte; si fermò presso l'enorme scrittoio, sul quale erano già state accese dal cameriere, che vi era entrato prima, sei candele; fece scricchiolare le dita e sedette, disponendo l'occorrente per scrivere. Poggiò i gomiti sul tavolo, inclinò la testa da un lato, pensò per circa un minuto, e cominciò a scrivere senza fermarsi un attimo. Scriveva a lei senza intestazione, e in francese, usando il pronome ``voi'' che in francese non ha quel tono di freddezza che ha in russo. 

\begin{quote}
``Nell'ultima nostra conversazione vi ho espresso il proposito di comunicarvi la mia decisione riguardo all'oggetto di essa. Dopo aver attentamente riflettuto a tutto, vi scrivo per adempiere quella promessa. La mia decisione è la seguente: quali che siano le vostre azioni, io reputo di non aver il diritto di spezzare quei vincoli con i quali siamo legati da un potere che viene dall'alto. La famiglia non può essere distrutta dal capriccio, dall'arbitrio o, peggio, dalla colpa di uno dei coniugi, e la nostra vita deve procedere come è proceduta sinora. Ciò è indispensabile per me, per voi, per nostro figlio. Sono convinto che siate pentita o vi pentiate di quanto ha dato occasione a questa lettera e che coadiuverete con me per estirpare dalla radice la causa del nostro contrasto e dimenticare il passato. In caso contrario, potete voi stessa prevedere quel che attende voi e vostro figlio. Di tutto questo spero parlarvi più dettagliatamente a voce. Giacché il tempo della villeggiatura sta per finire, vi prego di rientrare al più presto a Pietroburgo, non più tardi di martedì. Saranno date tutte le disposizioni necessarie per il vostro trasferimento. Vi prego notare che attribuisco una particolare importanza all'adempimento di questa mia richiesta.

A. Karenin

P. S. - In questa lettera è accluso il denaro che potrà essere necessario per le vostre spese''.
\end{quote} 

Lesse la lettera e ne rimase soddisfatto, particolarmente per il fatto che si era ricordato di accludervi il denaro; non vi era una sola parola dura, né una recriminazione, ma non vi era neppure indulgenza. C'era soprattutto un ponte d'oro per il ritorno. Piegata la lettera, spianatala col grosso tagliacarte di avorio massiccio e postala in una busta insieme col denaro, con la soddisfazione che sempre gli procurava l'uso dei suoi oggetti da scrittoio disposti in bell'ordine, sonò. 

- La consegnerai al corriere perché la faccia avere ad Anna Arkad'evna , domani in campagna - disse e si alzò. 

- Va bene, eccellenza. Volete il tè nello studio? 

Aleksej Aleksandrovic ordinò di servire il tè nello studio e, giocando col tagliacarte massiccio, andò verso la poltrona accanto alla quale erano preparati una lampada e un libro francese sulle Tavole eugubine del quale aveva iniziato la lettura. Sopra la poltrona era appeso un ritratto ovale di Anna, molto ben fatto, di un noto artista. Aleksej Aleksandrovic lo guardò. Gli occhi impenetrabili lo fissavano con ironia e impudenza, come in quell'ultima sera della loro spiegazione. Intollerabilmente impudente e provocante era per lui la vista del merletto nero, posato sulla testa, ed eseguito in modo perfetto dall'artista, dei capelli neri e della bellissima mano bianca dall'anulare coperto di anelli. Dopo aver guardato il ritratto per circa un minuto, si agitò tanto che le labbra gli tremarono ed emisero il suono ``brr'', ed egli si voltò dall'altra parte. Sedutosi in fretta nella poltrona, aprì il libro. Cominciò a leggere, ma non seppe ritrovare l'interesse, prima sempre vivo, per le Tavole eugubine. Scorreva il libro, ma pensava ad altro. Non alla moglie, ma a un complicato affare venuto fuori nell'ultimo periodo della sua attività di statista e che, in quel momento, costituiva la cosa più interessante del suo ufficio. Aveva esaminato profondamente quel problema, e ora nella sua mente nasceva un'idea geniale che, poteva dirlo senza vanteria, avrebbe risolto in pieno quell'affare, avrebbe fatto progredir lui nella carriera, debellato i suoi nemici ed arrecato grande utilità allo stato. Non appena il cameriere, portato il tè, uscì dalla stanza, Aleksej Aleksandrovic si alzò e andò allo scrittoio. Fatta avanzare nel centro la cartella degli affari in corso, ebbe un impercettibile sorriso di soddisfazione, tirò fuori dal portapenna una matita e si sprofondò nella lettura della pratica di quel complesso affare che s'era fatto portare e che riguardava la complicazione sopraggiunta. La complicazione era la seguente. La speciale qualità di Aleksej Aleksandrovic come uomo di governo, il tratto caratteristico tutto suo, e del resto di ogni impiegato che fa carriera, che insieme alla sua ostinata ambizione, al suo contegno, all'onestà e alla presunzione era stata la ragione della sua rapida carriera, consisteva nel disdegno verso il carattere burocratico degli affari, nella riduzione al minimo della corrispondenza, nel porsi a contatto diretto, per quanto possibile, col punto vivo delle questioni e nella economia di queste. Nella famosa commissione del 2 giugno, era stata esumata la questione della irrigazione dei campi del governatorato di Zarajsk, pratica che giaceva accantonata presso il ministero di Aleksej Aleksandrovic e che rappresentava un evidente esempio di spese infruttuose e di lungaggini burocratiche. Aleksej Aleksandrovic riteneva giusta la faccenda. Questa pratica era stata iniziata dal predecessore di Aleksej Aleksandrovic e infatti per tale affare si era speso e si spendeva molto denaro in modo del tutto improduttivo e non si veniva mai a capo di nulla. Aleksej Aleksandrovic, entrato in carica, s'era subito accorto di tali manchevolezze, e parve voler mettere mano alla cosa; ma nei primi tempi, quando non si sentiva ancora del tutto sicuro, sapeva che la faccenda ledeva troppi interessi e non era stata bene impostata; sopravvenute poi altre questioni, se ne era semplicemente dimenticato. Così questa, come tutte le altre questioni, andava avanti da sé, per forza di inerzia. (Molte persone ci mangiavano su, specialmente una famiglia molto per bene e amante della musica in cui tutte le figlie sonavano strumenti a corda. Aleksej Aleksandrovic conosceva questa famiglia, ed era padrino di una delle figlie maggiori). 

Secondo l'opinione di Aleksej Aleksandrovic, non era stato onesto che un ministero contrario avesse richiamato l'attenzione su quell'affare, perché in ogni ministero vi erano affari che, per certe convenienze di servizio, nessuno metteva in luce; ma poiché gli era stato lanciato quel guanto, egli coraggiosamente lo raccolse, e cominciò col richiedere una commissione speciale per lo studio e la verifica dei lavori affidati alla commissione per l'irrigazione dei campi del governatorato di Zarajsk; eliminando, però, ogni favoritismo anche nei rapporti di quei tali signori. 

Pretese la nomina di una commissione speciale per l'affare della sistemazione degli allogeni. Questo affare era stato messo in evidenza, per caso, nel comitato del 2 giugno ed era sostenuto con energia da Aleksej Aleksandrovic come cosa che non sopportava dilazione, dato lo stato deplorevole degli allogeni. Nel comitato questo affare servì di pretesto a dispute tra alcuni ministeri. Il ministero contrario ad Aleksej Aleksandrovic asseriva che la condizione degli allogeni era diventata sempre più florida, che la sistemazione della proposta poteva rovinare questa floridezza, e che, se qualcosa c'era da lamentare, questo qualcosa dipendeva solo dall'inadempimento, da parte del ministero di Aleksej Aleksandrovic, delle misure prescritte dalla legge. Ora Aleksej Aleksandrovic aveva intenzione di pretendere: in primo luogo, che venisse costituita una commissione nuova a cui affidare l'incarico di accertare sul posto quale fosse la condizione reale degli allogeni; in secondo luogo, se la condizione degli allogeni fosse risultata tale quale appariva dai dati ufficiali che erano nelle mani del comitato, si nominasse un'altra commissione, scientifica questa, per lo studio delle cause della desolante condizione degli allogeni dal punto di vista: a) politico, b) amministrativo, c) economico, d) etnografico, e) materiale, f) religioso; in terzo luogo, che si ordinasse al ministero contrario di fornire notizie sulle misure da esso attuate nell'ultimo decennio per eliminare le condizioni svantaggiose in cui versavano ora gli allogeni; infine, in quarto luogo, che si imponesse allo stesso ministero di dar conto delle ragioni per le quali esso, come risultava dalle notizie fornite al comitato sotto i nn. 17015 e 108308 del 5 dicembre 1863 e 7 giugno 1864, aveva agito proprio in modo opposto alle prescrizioni della legge fondamentale e organica, vol. \ldots{} art. 18 e nota dell'art. 36. Un colorito di animazione copriva a poco a poco il viso di Aleksej Aleksandrovic mentre egli fissava in breve lo schema di queste idee. Dopo aver riempito un foglio di carta, sonò e dette un biglietto per il direttore della cancelleria per avere dei dati che gli occorrevano. Alzatosi e fatto un giro per la stanza, guardò di nuovo il ritratto, aggrottò le sopracciglia e sorrise con disprezzo. Dopo aver letto ancora un po' il libro sulle Tavole eugubine e risvegliato in sé l'interesse verso di queste, Aleksej Aleksandrovic, alle undici, andò a dormire, e quando, supino nel letto, ricordò quello che era accaduto con la moglie, la cosa non gli apparve più sotto un aspetto così fosco. 

\capitolo{XV}\label{xv-2} 

Sebbene Anna avesse ribattuto Vronskij con tenacia e irritazione, quando egli le aveva detto che la posizione era insostenibile e l'aveva esortata a dire tutto al marito, ella intimamente sentiva falsa e disonesta la propria posizione e desiderava con tutta l'anima di cambiarla. Tornando col marito dalle corse, in un momento di impeto, gli aveva detto tutto; malgrado la pena provata, era contenta. Dopo che il marito l'aveva lasciata, ella si andava dicendo che ora ogni cosa si sarebbe definita e che, per lo meno, non ci sarebbero più stati la menzogna e l'inganno. Le sembrava fuor di dubbio che adesso la sua posizione si sarebbe definita per sempre. Poteva anche essere non buona questa nuova sua posizione, ma sarebbe sempre stata definita, e non più ambigua e mendace. Il dolore ch'ella aveva causato a se stessa e al marito nel pronunziare quelle parole, sarebbe stato compensato, così ella immaginava, dal fatto che tutto si sarebbe definito. La sera stessa, però, ella si era vista con Vronskij, e non gli aveva riferito nulla di quello che era accaduto tra lei e il marito, sebbene fosse evidente la necessità di parlargliene per chiarire la situazione. 

Quando l'indomani si svegliò, la prima cosa che le si presentò alla mente fu il colloquio col marito, e quelle parole pronunciate le risonarono così orribili che non riusciva più a capire come si fosse decisa a pronunciarle, quelle strane parole, e non riusciva a immaginare gli effetti che ne sarebbero derivati. Ma le parole erano state dette, e Aleksej Aleksandrovic era andato via senza dir nulla. ``Ho visto Vronskij e non gliene ho parlato. Mentre andava via, volevo richiamarlo per parlargli ma poi ho cambiato idea, perché mi pareva strano non avergli detto nulla al primo momento. Perché glielo volevo dire e non l'ho detto?''. E in risposta a questa domanda una vampa di rossore le si diffuse sul viso. Capiva ora perché aveva taciuto; capiva ora perché se ne vergognava. La posizione sua, che le era sembrata chiarita la sera innanzi, le si presentava ora, non solo poco chiara, ma senza via d'uscita. Aveva lo sgomento del disonore, al quale prima non aveva mai neppure pensato. Appena immaginava quello che il marito avrebbe fatto, le si affacciavano i pensieri più paurosi. Le venne in mente che sarebbe venuto subito l'intendente a cacciarla di casa, che il suo disonore sarebbe stato rivelato a tutto il mondo. Si chiedeva dove sarebbe andata a finire se fosse stata cacciata di casa, e non trovava risposta. 

Quando pensava a Vronskij le pareva che egli non l'amasse più, che già cominciasse ad essere stanco di lei, che lei non potesse offrirglisi più, e sentiva per questo un astio verso di lui. Le sembrava che le parole, che ella aveva dette al marito e che incessantemente ripeteva nella mente, le avesse dette a tutti e che tutti le avessero udite. Non poteva decidersi a guardare negli occhi le persone con le quali viveva. Non poteva decidersi a chiamare la cameriera e ancor meno a uscir fuori dove erano il figlio e la governante. 

La cameriera, che già da tempo era in ascolto presso la porta, entrò senza essere chiamata. Anna la guardò interrogativamente negli occhi e arrossì come spaurita. La donna si scusò di essere entrata, dicendo che le era parso di sentir sonare. Aveva portato un vestito e un biglietto. Il biglietto era di Betsy. Betsy le ricordava che quella mattina si riunivano da lei Liza Merkalova e la baronessa Stoltz coi loro corteggiatori Kaluzskij ed il vecchio Stremov per una partita di croquet. ``Venite anche solo per guardare, come studio di costumi. Vi aspetto'' terminava. 

Anna lesse il biglietto e sospirò profondamente. 

- Non mi occorre nulla, nulla - disse ad Annuška che le cambiava di posto le boccette e le spazzole sul tavolo da toeletta. - Va', mi vesto ed esco. Non ho bisogno di nulla, di nulla. 

Annuška uscì, ma Anna rimase a sedere nella medesima posizione in cui era, il capo e le braccia abbandonati. Di tanto in tanto un tremito le percorreva tutto il corpo: avrebbe voluto fare un movimento qualsiasi, dire qualche cosa e invece si immobilizzava di nuovo. Ripeteva continuamente: ``Dio mio, Dio mio''; ma né ``Dio'', né ``mio'' avevano senso alcuno. Il pensiero di cercare aiuto nella fede, malgrado non avesse mai avuto dubbi sulla religione nella quale era cresciuta, le era altrettanto impossibile quanto cercare aiuto presso lo stesso Aleksej Aleksandrovic. Sapeva bene, infatti, che l'aiuto della religione sarebbe stato possibile solo se avesse rinunciato a tutto quello che ora costituiva per lei la ragione stessa della vita. Non solo era in pena, ma cominciava ad avere spavento del suo nuovo stato d'animo, finora mai provato. Sentiva che dentro di sé tutto cominciava a sdoppiarsi, come talvolta si sdoppiano gli oggetti agli occhi stanchi. Non sapeva di che cosa avesse paura, che cosa volesse: se avesse paura o desiderasse quello che era stato o quello che sarebbe stato, e che cosa precisamente desiderasse, non sapeva. 

``Ahi, ma che cosa sto facendo!'' si disse, sentendo a un tratto dolore ai due lati del capo. Quando rientrò in sé, si accorse che stringeva con tutte e due le mani le ciocche dei capelli delle tempie e le tirava. Si levò di scatto e fece dei passi. 

- Il caffè è pronto, e mamzel' e Serëza aspettano - disse Annuška, che, entrata di nuovo, aveva trovato Anna nella stessa posizione di prima. 

- Serëza? Che fa Serëza? - chiese Anna, rianimandosi a un tratto e ricordandosi, per la prima volta in tutta la mattina, dell'esistenza del figlio. 

- A quanto pare, si è reso colpevole - rispose Annuška sorridendo. 

- Come s'è reso colpevole? 

- Nella stanza d'angolo c'erano delle pesche; sembra che lui, di nascosto, ne abbia mangiata una. 

Il ricordo del figlio strappò immediatamente Anna da quella situazione senza uscita nella quale si trovava. Ricordò l'atteggiamento, in parte sincero se pur molto esagerato, che ella aveva assunto negli ultimi anni, di madre che vive tutta pel suo figliuolo, e sentì con gioia che, pur nella condizione nella quale si trovava, era sempre in possesso di qualche cosa che stava a sé e per sé, indipendentemente da quelli che sarebbero stati i rapporti suoi col marito e con Vronskij. Questo suo possesso era il figlio. In qualsiasi condizione si fosse venuta a trovare non poteva abbandonare il figlio. La umiliasse pure e la scacciasse il marito, si raffreddasse pure Vronskij nei suoi riguardi e riprendesse a vivere la sua vita libera (pensò di nuovo a lui con rancore e rimprovero), ella non poteva abbandonare il figlio. Aveva uno scopo nella vita. E doveva agire, agire, per garantire questo suo rapporto col figlio, agire perché non glielo togliessero. Anzi presto, al più presto possibile doveva agire perché non avessero a sottrarglielo. Doveva prendere con sé il figlio e partire. Era l'unica cosa che doveva fare adesso. Doveva quindi calmarsi e uscire da quello stato di angoscia. Il pensiero di un'azione immediata collegata col figlio, il fatto di dover partire con lui per un luogo lontano, le dette questa tranquillità. 

Si vestì in fretta, scese a passi decisi, entrò nel salotto dove, di solito, era apparecchiato il caffè e dove l'attendevano Serëza e la governante. Serëza, tutto in bianco, stava in piedi presso la tavola sotto lo specchio e, la schiena e la testa chine, con una espressione di attenzione intensa che ben le era nota e che lo faceva rassomigliare al padre, giocava con dei fiori che aveva preso con sé. La governante aveva un'aria severa. Serëza proruppe in un grido acuto, come spesso gli accadeva: ``Oh mamma'' e stette incerto se correre ad abbracciarla e lasciare i fiori, o terminare la coroncina di fiori e andare poi da lei. 

La governante, dopo aver salutato, cominciò con lentezza e precisione a raccontare il misfatto commesso da Serëza; ma Anna non l'ascoltava: pensava se l'avrebbe portata o no con sé. ``No, non la condurrò - decise. - Partirò sola con mio figlio''. 

- Sì, va molto male - disse Anna, e preso il figlio per una spalla, lo guardò con occhio tutt'altro che severo, con uno sguardo timido che confuse e rallegrò il ragazzo, e lo baciò. - Lasciatelo con me - disse alla governante che la guardava sorpresa, e, senza lasciare la mano del figlio, sedette alla tavola dove era preparato il caffè. 

- Mamma, io\ldots{} io\ldots{} non\ldots{} - disse il ragazzo, cercando di capire dalla espressione del volto della madre che cosa dovesse venirgliene pel fatto della pesca. 

- Serëza - disse lei non appena la governante fu uscita dalla stanza - quel che hai fatto è male. Ma tu non lo farai più\ldots{} Mi vuoi bene? 

Sentiva le lacrime venirle agli occhi. ``Posso forse non amarlo? - si diceva, fissando lo sguardo del figlio spaventato e nello stesso tempo allegro. - È possibile mai che egli sia d'accordo col padre nel punirmi? È possibile che non abbia pietà di me?''. Già le lacrime le scorrevano pel viso, e per nasconderle, si alzò di scatto e si avviò quasi di corsa in terrazza. 

Dopo le piogge temporalesche degli ultimi giorni, era seguito un tempo freddo, ma limpido. Malgrado il sole vivido che passava attraverso le foglie lavate, l'aria era fredda. 

Al contatto dell'aria libera, rabbrividì, e per il freddo e per l'interno sgomento che con forza rinnovata l'assaliva. 

- Va', va' da Mariette - disse a Serëza che l'aveva seguita, e si mise a camminare sulla stuoia della terrazza. ``Possibile che non mi perdonino, che non capiscano che tutto quello che è accaduto non poteva non accadere?''. 

Indugiatasi a guardare le cime delle tremule che si dondolavano al vento con le foglie scintillanti e vivide nel sole freddo, capì che non le avrebbero perdonato, che tutto e tutti sarebbero stati spietati con lei, come quel cielo, come quel verde. E di nuovo sentì che nell'animo suo avveniva quel tale sdoppiamento. ``Non si deve pensare, non si deve - disse a se stessa - bisogna prepararsi. Per dove? Quando? Chi prendere con me? Sì\ldots{} a Mosca, col treno della sera. Annuška e Serëza, e soltanto le cose più indispensabili. Ma prima devo scrivere a tutti e due''. Entrò in fretta in casa, sedette al tavolo dello studio e scrisse al marito. 

``Dopo quello che è accaduto, non posso più rimanere nella vostra casa. Me ne vado e prendo con me mio figlio. Non conosco la legge e perciò non so con quale dei genitori debba stare il figlio; ma io lo prendo con me, perché senza di lui non posso vivere. Siate generoso, lasciatemelo''. 

Sino ad allora aveva scritto speditamente e con naturalezza, ma l'appello alla generosità che ella non gli riconosceva, e la opportunità di chiudere la lettera con qualche cosa di commovente, la fecero sostare. 

``Parlare della mia colpa e del mio pentimento non posso, perché\ldots{}''. 

Ristette di nuovo, non trovando un nesso tra i suoi pensieri. ``No, non ci vuol nulla più di quanto è strettamente necessario''. Ricopiò la lettera, eliminando l'accenno alla generosità, e la sigillò. 

L'altra lettera la doveva scrivere a Vronskij. ``Ho confessato a mio marito'' scrisse, ma poi non seppe andare avanti. Era così volgare, così poco attraente. ``E poi cosa posso scrivergli?''. Di nuovo il rossore della vergogna le coprì il volto. Pensò alla pace perduta, sentì rancore verso l'amante e con dispetto strappò in piccoli pezzi il foglietto con la frase scrittavi. ``Non occorre nulla'' si disse e, riposta la cartella, andò su, disse alla governante e alle persone di servizio che quel giorno sarebbe partita per Mosca, e subito si accinse a mettere insieme le sue robe. 

\capitolo{XVI}\label{xvi-2} 

Per tutte le stanze della villa era un correre di portieri, giardinieri e servitori che portavano via la roba. Gli armadi e i cassettoni erano aperti; si era mandato già due volte di corsa alla bottega per lo spago; per terra si trascinava carta di giornali. Due bauli, le sacche e gli scialli da viaggio erano stati già portati in anticamera. Una carrozza padronale e due da nolo stavano ferme presso l'ingresso. Anna, che per il lavoro dei preparativi aveva abbandonato l'interna agitazione, approntava la sua sacca da viaggio stando in piedi accanto alla tavola nello studio, quando Annuška ne attirò l'attenzione verso un rumore di vettura che s'avvicinava. Anna guardò dalla finestra e vide presso la scala il fattorino di Aleksej Aleksandrovic che sonava alla porta d'ingresso. 

\begin{itemize} \itemsep1pt\parskip0pt\parsep0pt \item Va' a veder cos'è - disse e, in calma attesa, sedette in una poltrona, incrociando le mani sulle ginocchia. Il servitore portò un grosso plico con l'indirizzo di mano di Aleksej Aleksandrovic. \end{itemize} 

- Il fattorino ha l'incarico di portare la risposta - disse. 

- Va bene - rispose Anna e, appena l'uomo fu uscito, con le dita tremanti, strappò la busta. Ne venne fuori un plico di assegni non piegati, incollati in una fascetta. Ella liberò la lettera dalla busta e cominciò a scorrerla dalla fine. ``Ho disposto i preparativi per il trasferimento, do grande importanza all'adempimento della mia richiesta'' leggeva. Scorse più avanti, poi tornò indietro, lesse tutto e ancora una volta rilesse tutta la lettera daccapo. Quando ebbe finito sentì che aveva freddo e che su di lei era piombata una sventura così grande quale non poteva attendersi mai. 

La mattina s'era pentita d'aver parlato al marito e aveva desiderato una sola cosa, e cioè che quelle parole fossero come non dette. Ed ecco, la lettera riconosceva le parole come non dette, e le dava proprio quello che aveva desiderato. Ma ora questa lettera le appariva più terribile di tutto quello che avrebbe potuto immaginare. 

``Ha ragione, ha ragione! - si disse. - S'intende, egli ha sempre ragione; è cristiano, è magnanimo! Ma no! è vile, disgustoso! E questo, nessuno, all'infuori di me, lo capisce e nessuno lo capirà mai; e io non posso spiegarlo a nessuno. Gli altri dicono: è un uomo religioso, morale, onesto, intelligente; ma non sanno quello che so io. Non sanno ch'egli ha soffocato tutto quello che c'era di vivo in me; che neppure una volta gli è venuto in mente che io ero una donna viva che aveva bisogno d'amare. Non sanno che in ogni occasione mi ha umiliato, compiacendosene. Non ho forse cercato con tutte le mie forze di trovare uno scopo alla mia vita? Non ho forse provato ad amarlo, ad amare mio figlio quando già non potevo più amare lui? Ma è venuto poi il momento in cui ho compreso, in cui non mi è stato più possibile ingannare me stessa, in cui ho sentito che ero viva, che non avevo colpa se Dio mi aveva fatto così per l'amore e per la vita. E ora? Avesse ucciso me, avesse ucciso lui, avrei sopportato tutto, avrei perdonato tutto, ma no, egli\ldots{}''. 

``Com'è che non ho indovinato prima quello che avrebbe fatto? Che avrebbe fatto quello che è proprio conforme al suo carattere meschino? Egli avrà ragione e io sarò rovinata, io precipiterò ancora, ancora più in basso\ldots{}''. ``Voi stessa potrete supporre quello che attende voi e vostro figlio'' ricordava le parole della lettera. ``Questa è la minaccia di togliermi il figlio, e probabilmente, secondo la loro stupida legge, ciò è possibile. Ma forse non so perché dice così? Egli non crede neanche al mio amore per mio figlio e lo disprezza (così come l'ha sempre irriso), disprezza questo mio sentimento; non sa che io non abbandonerò mio figlio, che non posso abbandonarlo, che senza mio figlio non saprei vivere neppure con l'essere che amo, e sa pure che se abbandonassi mio figlio e fuggissi via da lui, agirei come la donna più abietta e svergognata, questo egli lo sa, e sa che io questo non avrò la forza di farlo''. 

``La nostra vita deve procedere come prima''; ella ricordò un'altra frase della lettera. ``Questa vita era tormentosa anche prima, è stata orribile negli ultimi tempi. Che sarà mai ora? Ed egli sa tutto questo, sa che io non potrò pentirmi di quello che ho fatto, che di questo io vivo, che amo; sa che oltre a menzogna e inganno non ne verrebbe fuori altro; ma egli sente il bisogno di continuare a tormentarmi. Io lo conosco, so che, come un pesce nell'acqua, egli nuota e gode nella menzogna. Ma no, io non glielo darò questo piacere, io spezzerò questa rete di menzogna nella quale egli vuole avvilupparmi; sarà quel che sarà. Tutto sarà preferibile alla menzogna e all'inganno!''. 

``Ma come, Dio mio! Dio mio! C'è forse al mondo una donna più infelice di me?''. 

- No, la spezzerò, la spezzerò! - gridò, scattando e trattenendo le lacrime. E si accostò allo scrittoio per scrivergli un'altra lettera, ma in fondo all'anima già sentiva che non avrebbe avuto la forza di uscire dalla situazione che era durata fino ad allora, per quanto falsa e disonorevole. Sedette allo scrittoio e, invece di scrivere, incrociò le braccia sul tavolo, poggiò la testa su di esse e pianse, così come piangono i bambini, singhiozzando e scotendo il petto. Piangeva perché quanto aveva sognato sulla chiarificazione e sistemazione del suo stato era distrutto per sempre. Tutto sarebbe rimasto così come prima, anzi molto peggio di prima. Sentiva che la posizione che occupava nella società alla quale apparteneva e che la mattina le era parsa cosa del tutto insignificante, quella posizione le era cara, sentiva che non avrebbe mai avuto l'ardire di cambiarla con l'altra ignominiosa della donna che lascia il marito e il figlio e si unisce all'amante; che per quanti sforzi facesse non sarebbe mai riuscita a far violenza a se stessa. Non avrebbe mai provato la libertà dell'amore, e sarebbe per sempre rimasta una moglie colpevole, sotto la minaccia continua d'essere accusata d'aver ingannato il marito per un legame infame con un altro uomo che era libero, ma col quale non poteva vivere una unica vita. Sapeva che così sarebbe stato, ed era tanto orribile tutto questo, che non poteva neppure immaginare come sarebbe andato a finire. E piangeva senza ritegno, così come piangono i bambini puniti. 

Il rumore dei passi del servitore la obbligò a ritornare in sé, e, nascondendo il viso, finse di scrivere. 

- Il fattorino vuole la risposta - riferì il servitore. 

- La risposta?\ldots{} sì - disse Anna - che aspetti, sonerò. 

``Che posso scrivere? - pensava. - Che posso decidere da sola? Che cosa so? Che cosa voglio? Che cosa desidero?''. Sentì che di nuovo nell'animo suo avveniva lo sdoppiamento. Ebbe di nuovo paura di questa sensazione e si aggrappò al primo pretesto di attività che le si parò innanzi e che potesse distrarla dal pensare a se stessa. ``Devo vedere Aleksej - così nel pensiero chiamava Vronskij - egli solo potrà dirmi cosa devo fare. Andrò da Betsy, forse lo vedrò là'' pensò, dimenticando completamente che proprio il giorno prima, quando gli aveva detto che non sarebbe andata dalla principessa Tverskaja, egli aveva soggiunto che perciò neanche lui ci sarebbe andato. Si accostò allo scrittoio, scrisse al marito: ``Ho ricevuto la vostra lettera. A.'' e, dopo aver sonato, la dette al servitore. 

- Non partiamo più - disse ad Annuška che era entrata. 

- Non partiamo proprio? 

- No, ma non disfate i bauli fino a domani e trattenete la carrozza. Io vado dalla principessa. 

- Quale abito devo preparare? 

\capitolo{XVII}\label{xvii-2} 

Il gruppo della partita a croquet, alla quale la principessa Tverskaja aveva invitato Anna, doveva essere formato da due signore e dai loro rispettivi adoratori. Queste due donne erano le esponenti più in vista di un nuovo circolo scelto di Pietroburgo che si chiamava, a imitazione di qualche cosa già imitata, Les sept merveilles du monde. Queste signore appartenevano, è vero, a un ambiente elevato, ma questo era completamente ostile a quello frequentato da Anna. Inoltre il vecchio Stremov, una delle persone più influenti di Pietroburgo, l'adoratore di Liza Merkalova, era nemico, per ragioni di ufficio, di Aleksej Aleksandrovic. Per tutte queste considerazioni Anna non aveva voluto andare da Betsy, e a questo suo rifiuto si riferivano le allusioni della principessa Tverskaja nel biglietto che le aveva scritto. Ma ora Anna, nella speranza di incontravi Vronskij, volle andare. 

Anna giunse dalla principessa Tverskaja prima degli altri ospiti. 

Nel momento in cui entrava, il servitore di Vronskij, con le fedine ben lisce, simile a un gentiluomo di camera, entrava anch'esso. Si fermò sulla porta, e, toltosi il berretto, la lasciò passare. Anna lo conosceva, e solo in quel momento si ricordò che Vronskij il giorno innanzi le aveva detto che non sarebbe andato dalla principessa. Forse proprio per questo aveva mandato un biglietto. 

Ella aveva sentito, togliendosi il mantello nell'anticamera, come il servitore, che pronunciava perfino la lettera ``r'' come un gentiluomo di camera, aveva detto: ``da parte del conte alla principessa'' e aveva consegnato un biglietto. 

Avrebbe voluto chiedergli dove era il padrone. Avrebbe voluto tornare indietro e scrivere a Vronskij che venisse da lei, oppure andare lei stessa da lui. Ma né questa, né l'altra, né la terza cosa si potevano fare: si sentivano risonare già i campanelli che annunziavano nelle stanze attigue il suo arrivo e il servitore della principessa Tverskaja stava già di lato accanto alla porta aperta, aspettando che ella passasse nelle stanze interne. 

- La principessa è in giardino, sarà avvertita subito. Vuole avere la compiacenza di favorire in giardino? - disse un altro servo nella stanza accanto. 

La situazione era sempre la stessa, oscura, come a casa; anche peggiore, perché niente poteva fare, e non poteva vedere Vronskij e doveva restare in quell'ambiente così estraneo e così contrario alle sue condizioni di spirito. Ma Anna aveva un vestito che, lo sapeva, le stava bene; non era sola; intorno a lei c'era quell'abituale sfondo di ozio imperante e quindi stava meglio qui che a casa. Qui non doveva pensare a quel che avrebbe dovuto fare. Qui tutto andava da sé. A Betsy che le venne incontro in un abito bianco, la cui eleganza la colpì, Anna sorrise come sempre. La principessa Tverskaja stava con Tuškevic e una parente nubile che, con grande gioia dei genitori di provincia, passava l'estate presso la famosa principessa. 

Probabilmente in Anna c'era qualcosa d'insolito, perché Betsy lo notò subito. 

- Ho dormito male - rispose Anna, guardando il servitore che veniva loro incontro e che, secondo i suoi calcoli, portava il biglietto di Vronskij. 

- Come sono contenta che siate venuta! - disse Betsy. - Sono stanca, e proprio ora volevo prendere una tazza di tè, prima che gli altri arrivino. E voi - si rivolse a Tuškevic - potreste andare con Maša a provare il croquet-ground, là dove hanno tagliato l'erba. Io e voi avremo un po' di tempo per parlare un po' tra di noi prendendo il tè: will have a cosy chat, vero? - disse rivolta ad Anna con un sorriso, stringendole la mano che reggeva l'ombrellino. 

- Tanto più che non posso trattenermi a lungo da voi; devo andare dalla vecchia Vrede. Gliel'ho promesso da cento anni - disse Anna alla quale la bugia, estranea alla sua natura, non solo era divenuta facile e naturale, ma procurava perfino piacere. 

Perché avesse detto quello cui un minuto prima non pensava, non avrebbe potuto spiegarlo in nessun modo. Aveva detto ciò solo perché, non essendoci Vronskij, le era necessario esser sicura della propria libertà per cercare di vederlo in qualche modo. Ma perché proprio le fosse venuto sulle labbra il nome della vecchia damigella d'onore Vrede, dalla quale avrebbe dovuto andare come da tanti altri, non sapeva spiegarselo; e intanto, come apparve poi, nell'escogitare i mezzi accorti per incontrarsi con Vronskij, non riusciva a trovare niente di meglio. 

- No, non vi lascerò andare affatto! - rispose Betsy, guardando attenta Anna. - Invero mi offenderei, se non vi volessi bene. È come se temeste che la mia compagnia possa compromettervi. Per favore il tè per noi nel salottino - disse, socchiudendo come sempre gli occhi nel rivolgersi al servitore. Preso da lui il biglietto, lo lesse. - Aleksej ci ha fatto un brutto tiro - disse in francese - scrive che non verrà - aggiunse con un tono così naturale e semplice, come se mai le fosse passato per la mente che Vronskij potesse interessare Anna altrimenti che come giocatore di croquet. 

Anna sapeva che Betsy era al corrente di tutto, ma quando la sentiva parlare in sua presenza di Vronskij, per un momento si persuadeva ch'ella non sapesse nulla. 

- Ah - disse con indifferenza, come se poco le interessasse la cosa, e continuò sorridendo: - Come può compromettere qualcuno la vostra compagnia? - Questi giuochi di parole, questo voler celare il segreto avevano, come del resto per tutte le donne, un grande fascino per Anna. E non solo la necessità di nascondere, o lo scopo per cui si nasconde, ma lo stesso procedimento del nascondere la seduceva. - Io non posso essere più cattolica del papa - ella disse. - Stremov e Liza Merkalova sono il fior fiore della società. Poi sono ricevuti dovunque e io - accentuò in particolare quell'io - non sono mai stata severa ed intollerante. Non ne ho il tempo, ecco perché. 

- No, voi forse non volete incontrarvi con Stremov? Lasciate pure che lui e Aleksej Aleksandrovic spezzino delle lance al comitato; questo non ci riguarda. Ma in società egli è l'uomo più amabile che io conosca, ed è un appassionato giocatore di croquet. Ecco, vedrete. Malgrado la sua posizione ridicola di vecchio amatore di Liza, bisogna vedere come se la cava bene. È molto simpatico. Safo Stoltz, non la conoscete. È un tipo originale, proprio originale. 

Mentre Betsy parlava, Anna nello stesso tempo capiva, dallo sguardo vivace e intelligente di lei, ch'ella aveva intuito in parte la situazione sua, e stava ideando qualcosa. Esse erano in un piccolo studio. 

- Però bisogna scrivere ad Aleksej - e Betsy sedette al tavolo, scrisse alcune righe e mise in busta. - Scrivere che venga a pranzo. A pranzo mi rimane una signora senza cavaliere. Guardate, è efficace? Scusatemi, vi lascio un momento. Vi prego, sigillate e mandate. Devo dare un ordine. 

Senza esitare un attimo, Anna sedette al tavolo e, senza leggere la lettera di Betsy, vi scrisse in fondo: ``Mi è indispensabile vedervi. Venite nei pressi del giardino di Vrede. Vi sarò alle sei''. Sigillò, e Betsy, rientrata, consegnò, lei presente, la lettera. 

Proprio come aveva detto la principessa Tverskaja, durante il tè, che fu portato su di un tavolino-vassoio, in un piccolo salotto fresco, s'avviò fra le due donne a cosy chat, fino all'arrivo degli ospiti. Esse malignarono sulle signore che aspettavano e la conversazione indugiò su Liza Merkalova. 

- È molto carina, e mi è sempre stata simpatica - disse Anna. 

- Voi dovete volerle bene. Va pazza per voi. Ieri si è avvicinata a me dopo le corse ed era desolata di non avervi trovata. Dice che voi siete una vera eroina da romanzo e che se fosse un uomo farebbe mille sciocchezze per voi. Stremov dice che le fa lo stesso. 

- Ma ditemi, vi prego, io non ho mai potuto capire - disse Anna, dopo aver taciuto un po', e con un tono tale che mostrava chiaramente che la sua non era una domanda oziosa, ma era per lei più importante di quanto sarebbe dovuto apparire. - Ditemi, per favore, che c'è fra lei e il principe Kaluzskij, il cosiddetto Miška. Li ho visti poco. Che c'è? 

Betsy sorrise con gli occhi e guardò attentamente Anna. 

- C'è una maniera nuova - disse. - Hanno scelto questa maniera qua. Non badano alle convenienze. Ma c'è modo e modo di non curarsene. 

- Già, ma quali sono i suoi rapporti con Kaluzskij ? 

Betsy d'improvviso cominciò a ridere allegramente, cosa che accadeva di rado. 

- Voi invadete il campo della principessa Mjagkaja. Questa è una domanda da enfant terrible - e Betsy, evidentemente, voleva contenersi, ma non ci riusciva, e scoppiò in quel riso comunicativo delle persone che ridono di rado. - Bisogna chiederlo a loro - disse ridendo fino alle lacrime. 

- No, voi ridete - disse Anna, involontariamente contagiata dal riso - ma io non ho mai potuto capire. Non capisco, in questo, la parte del marito. 

- Il marito? Il marito di Liza Merkalova le porta gli scialli ed è sempre pronto a servirla. E più in fondo, in queste faccende, nessuno vuol ficcarci il naso. Vedete, nella buona società non si parla, e neppure si pensa, a certi particolari della toletta intima. Così anche per queste cose. 

- Sarete alla festa dei Rolandaki? - disse Anna per cambiar discorso. 

- Non credo - rispose Betsy e, senza guardare l'amica, riempì di tè le piccole tazze trasparenti. Avvicinata la tazza ad Anna, tirò fuori una sigaretta e, introdottala in un bocchino d'argento, si mise a fumare. 

- Ecco, vedete, io sono in una condizione felice - cominciò a dire ormai seria, dopo aver preso in mano la tazza. - Capisco voi e capisco Liza. Liza è una di quelle nature ingenue che, come i bambini, non capiscono che cosa sia bene e che cosa male. Almeno, non lo capiva quando era molto giovane. Ora sa che questa mancanza di discernimento le si addice: può darsi pure che non voglia capire di proposito - diceva Betsy con un sorriso sottile. - Tuttavia questo le si addice. Vedete, la stessa cosa può essere vista tragicamente, e divenire un tormento, mentre può essere considerata come di lieve importanza e divenire perfino piacevole. Voi forse sareste incline a considerare le cose troppo tragicamente. 

- Come vorrei conoscer gli altri, come conosco me stessa! - disse Anna seria e pensosa. - Sono peggiore o migliore degli altri? Peggiore, credo. 

- Siete un enfant terrible, un enfant terrible - ripeté Betsy. - Ma ecco che arrivano. 

\capitolo{XVIII}\label{xviii-2} 

Si sentirono dei passi, una voce maschile, poi una voce femminile e delle risa; ed entrarono gli ospiti attesi: Safo Stoltz e un giovanotto sprizzante salute, il cosiddetto Vas'ka. Si vedeva che gli aveva giovato nutrirsi di carne sanguinolenta, di tartufi e di vino di Borgogna. Vas'ka s'inchinò alle signore e le guardò, ma solo per un attimo. Era entrato dopo Safo e l'aveva seguita nel salotto come se le fosse stato legato, senza staccar da lei gli occhi sfavillanti, che sembrava volessero mangiarsela. Safo Stoltz era una bionda dagli occhi neri. Entrò a piccoli passi svelti, sui tacchi alti delle scarpette, e strinse forte, da uomo, le mani alle signore. 

Anna non aveva finora incontrato mai, neanche una volta, questa nuova celebrità, e fu sorpresa della sua bellezza, dell'eccentricità del suo abbigliamento e dell'arditezza dei suoi modi. Sulla testa di capelli suoi e non suoi, d'un tenero color d'oro, era innalzata una tale impalcatura che la testa sembrava eguagliare in altezza il busto armoniosamente sporgente e molto scollato sul davanti. Lo slancio in avanti era tale che, ad ogni movimento, si disegnavano sotto al vestito le forme delle ginocchia e della parte superiore delle gambe, e involontariamente ci si chiedeva dove in realtà finisse, sotto quella costruzione ondeggiante, il suo vero corpo, piccolo e snello, tanto scoperto di sopra e tanto nascosto nelle sue parti inferiori. 

Betsy si affrettò a presentarla ad Anna. 

- Figuratevi, stavamo quasi per schiacciare due soldati - ella cominciò subito a raccontare, ammiccando, sorridendo e tirando indietro lo strascico che aveva al primo momento gettato troppo da un lato. - Andavo con Vas'ka\ldots{} Ah, sì, non vi conoscete. - E, pronunciando il nome di lui, presentò il giovanotto e, arrossendo, rise sonoramente del proprio errore, di averlo presentato cioè come Vas'ka a chi non lo conosceva. 

Vas'ka si inchinò ancora una volta ad Anna, ma non le disse nulla. Si rivolse a Safo: 

- La scommessa è perduta. Siamo arrivati prima, pagate - egli disse, sorridendo. 

Safo rise ancor più allegramente. 

- Non ora però - ella disse. 

- È lo stesso, incasserò dopo. 

- Va bene, va bene. Ah, sì - si rivolse improvvisamente alla padrona di casa. - Sono brava io\ldots{} Vi ho condotto un ospite. Ecco anche lui. 

Il giovane ospite inatteso che Safo aveva condotto, e di cui si era dimenticata, era però un ospite così importante che, malgrado la sua giovinezza, tutte e due le signore si alzarono ad accoglierlo. 

Era costui un nuovo adoratore di Safo. Anch'egli, come Vas'ka, la seguiva dappertutto. 

Ben presto giunsero il principe Kaluzskij e Liza Merkalova con Stremov. Liza Merkalova era una bruna magra con un viso sonnolento di tipo orientale e con degli occhi deliziosi, indefinibili, come dicevano tutti. Il genere del suo abbigliamento (Anna lo notò subito e lo apprezzò) era pienamente rispondente alla sua bellezza. Quanto Safo era brusca e sostenuta, tanto Liza era morbida e abbandonata. 

Ma Liza, secondo il gusto di Anna, era molto più attraente. Betsy aveva detto di lei ad Anna che aveva preso il tono della bambina incosciente; ma quando Anna la vide, sentì che non era vero. Era proprio la donna incosciente, corrotta, ma simpatica e docile. È vero che il suo tono era lo stesso di quello di Safo; così come per Safo, due adoratori la seguivano, uno giovane e l'altro vecchio, quasi fossero cuciti alle sue vesti, e la divoravano con gli occhi; ma in lei c'era qualcosa che era al di sopra di quanto la circondava, c'era in lei lo splendore schietto dell'acqua di un brillante fra i vetri. Questo splendore illuminava i suoi occhi deliziosi, davvero indefinibili. Lo sguardo stanco e nello stesso tempo appassionato di quegli occhi circondati da un cerchio scuro, stupiva per la sua completa sincerità. Dopo aver guardato in quegli occhi, sembrava a ognuno di conoscerla tutta e, conosciutala, di non poterla non amare. Alla vista di Anna, il suo viso si illuminò improvvisamente di un sorriso gioioso. 

- Ah, come son contenta di vedervi! - ella disse, avvicinandosi. - Ieri alle corse stavo per raggiungervi proprio nel momento in cui andavate via. Volevo tanto vedervi proprio ieri. Non è vero che è stato orribile? - ella disse, guardando Anna col suo sguardo che sembrava scoprire tutta l'anima. 

- Già, non m'aspettavo proprio che potesse impressionare tanto - disse Anna, arrossendo. 

Il gruppo si alzò in quel momento per andare in giardino. 

- Io non vengo - disse Liza, sorridendo e sedendosi accanto ad Anna. - Voi neppure andate? Non so che gusto ci sia a giocare a croquet! 

- No, mi piace - disse Anna. 

- Ecco, ecco, come fate voi a non annoiarvi? Si guarda voi e ci si rallegra. Voi vivete, e io mi annoio. 

- Come, vi annoiate? Fate parte del gruppo più allegro di Pietroburgo! - disse Anna. 

- Forse quelli che non sono della nostra compagnia si annoiano ancora di più; ma noi, noi non siamo allegri, io sicuramente no, e ci annoiamo terribilmente, terribilmente. 

Safo, accesa una sigaretta, uscì in giardino con i due giovanotti. Betsy e Stremov rimasero a prendere il tè. 

- Come, vi annoiate? - disse Betsy. - Safo ha detto che ieri si sono tanto divertiti a casa vostra. 

- Oh, una tale malinconia! - disse Liza Merkalova. - Sono venuti tutti da me dopo le corse. E sempre gli stessi, sempre gli stessi! E sempre la stessa cosa. Tutta la sera ci siamo trascinati per i divani. Che c'è di allegro? No, come fate voi a non annoiarvi? - si rivolse di nuovo ad Anna. - Basta guardarvi per dire: ecco una donna che può essere felice o infelice, ma che non si annoia. Insegnatemi, come fate? 

- Non faccio in nessun modo - rispose Anna, arrossendo per quelle domande insistenti. 

- Ecco il modo migliore - disse Stremov, entrando nella conversazione. 

Stremov era un uomo sui cinquant'anni, dai capelli grigi, ma ancora fresco, molto brutto, ma con un viso espressivo e intelligente. Liza Merkalova era nipote di sua moglie ed egli passava con lei le sue ore libere. Incontrata Anna Karenina, egli, nemico per ragioni di ufficio di Aleksej Aleksandrovic, come uomo di mondo e intelligente, aveva cercato di essere particolarmente gentile con lei, moglie del suo nemico. 

- In nessun modo - replicò, sorridendo con finezza - è il mezzo migliore. Da tempo dico - proseguì, rivolgendosi a Liza Merkalova - che, per non annoiarsi, bisogna non pensare che ci si annoia. Così come non si deve temere di non dormire se si ha paura dell'insonnia. Lo stesso vi ha detto Anna Arkad'evna. 

- Sarei molto contenta d'aver detto questo, perché non solo è intelligente, ma è la verità - disse Anna, sorridendo. 

- No, ditemi perché non si può dormire e non ci si può non annoiare? 

- Per dormire bisogna lavorare, ed anche per divertirsi, bisogna lavorare. 

- Perché dovrei lavorare, quando il mio lavoro non serve a nessuno? E fingere io non so e non voglio. 

- Siete incorreggibile - disse Stremov senza guardarla, e si rivolse di nuovo ad Anna. 

Incontrando di rado Anna, egli non poteva dirle che delle cose banali, e di queste cose le parlava: di quando sarebbe andata a Pietroburgo, del bene che le voleva la contessa Lidija Ivanovna, ma con una espressione tale che mostrava come egli desiderasse con tutta l'anima di riuscirle simpatico e mostrarle la sua considerazione e anche più. 

Entrò Tuškevic, annunziando che tutta la compagnia aspettava i giocatori di croquet. 

- No, non andate, vi prego - pregava Liza Merkalova avendo sentito che Anna andava via. Stremov si unì a lei. 

- È un troppo grande contrasto - egli diceva - andare, dopo di qua, dalla vecchia Vrede. Dopo tutto per lei sarete un'occasione per fare un po' di maldicenza, mentre qui voi potete suscitare soltanto i migliori sentimenti, i più lontani e opposti alla maldicenza - egli diceva. 

Anna rimase un attimo pensosa, per la indecisione. I discorsi lusinghieri di quell'uomo intelligente, la simpatia ingenua, infantile che le mostrava Liza Merkalova, e tutto quell'abituale apparato mondano, tutto ciò era così facile, mentre l'attendeva una cosa tanto difficile, che per un attimo fu incerta se rimanere e allontanare ancora il momento penoso della spiegazione. Ma, prospettandosi quello che l'avrebbe attesa poi nella solitudine della casa se non avesse preso alcuna decisione, ricordatasi di quel gesto terribile per lei, anche nella memoria, dei capelli tirati con tutte e due le mani, si scusò e andò via. 

\capitolo{XIX}\label{xix-2} 

Vronskij, malgrado la vita mondana apparentemente leggera, era un uomo che detestava il disordine. Ancora giovane, al corpo dei paggi aveva provato l'umiliazione di un rifiuto quando, trovandosi in cattive condizioni finanziarie, aveva chiesto del denaro in prestito, e da quella volta non si era messo mai più in una condizione simile. 

Per tenere sempre in ordine le sue cose, più o meno spesso, a seconda delle circostanze, si appartava un cinque volte all'anno e metteva in chiaro i suoi affari. Chiamava questo la resa dei conti, ovvero faire la lessive. 

Il giorno dopo le corse, svegliatosi tardi, senza radersi né fare il bagno, Vronskij indossò l'uniforme e, distribuiti sulla tavola il denaro, i conti e le lettere, si mise al lavoro. Petrickij, svegliatosi e visto il compagno alla scrivania, sapendo che in un momento simile era solito arrabbiarsi, si vestì piano e uscì senza dargli noia. 

Ogni uomo, conoscendo fin nei più piccoli particolari la complessità della propria situazione, presuppone involontariamente che tale complessità e la difficoltà di scioglierla siano cose esclusivamente attinenti alla propria persona, e non pensa in nessun modo che altri si trovino assediati da affari altrettanto complessi quanto i propri. Così pure sembrava a Vronskij. Ed egli, non senza un intimo compiacimento, e non senza ragione, pensava che chiunque altro, trovatosi in così difficili condizioni, si sarebbe da tempo messo negli impicci, e sarebbe stato costretto ad agire male. Ma Vronskij sentiva che proprio ora gli era indispensabile fare i conti e chiarire la sua situazione per non trovarsi negli impicci. 

La prima cosa a cui Vronskij si accinse, come alla più facile, furon gli affari di denaro. Copiato con la sua scrittura minuta sulla busta d'una lettera tutto quello che egli doveva, tirò la somma e trovò che doveva 17.000 rubli e alcune centinaia, che accantonò per sistemare tutto. Contato il denaro e aggiuntovi quello risultante dal libretto di banca, trovò che gli restavano 1.800 rubli, mentre incassi fino all'anno nuovo non se ne prevedevano. Rifatto il conto dei debiti, lo ricopiò dopo averlo diviso in tre gruppi. Nel primo gruppo trovavano posto i debiti che dovevano essere pagati subito, o per i quali, in ogni caso, bisognava tener pronto il denaro, in modo che alla richiesta seguisse il pagamento senza neppure un attimo di indugio. Questi debiti ammontavano a circa 4.000 rubli: 1.500 per il cavallo e 2.500 per la garanzia prestata al giovane compagno Veneskij che, in presenza di Vronskij, aveva perduto questo denaro, giocando con un baro. Vronskij voleva dare subito la somma (la possedeva), ma Veneskij e Jašvin avevano insistito per pagare loro e non Vronskij che non aveva neppure giocato. Tutto questo era una bellissima cosa, ma Vronskij sapeva che, pur avendo preso parte in questo sordido affare solo coll'assumere sulla parola la garanzia per Veneskij, gli era indispensabile aver pronti quei 2.500 rubli da buttare all'imbroglione per non aver più nulla a che fare con lui. Così, per questo primo importantissimo gruppo di debiti occorreva avere sotto mano 4.000 rubli. Nel secondo gruppo, di ottomila rubli, erano compresi debiti meno importanti. Erano in prevalenza debiti di scuderia da corsa, con l'inglese, col sellaio e via di seguito. Per tali debiti occorreva tenere da parte circa 2.000 rubli per essere completamente tranquillo. L'ultimo gruppo di debiti, verso fornitori, verso alberghi e verso il sarto, poteva essere trascurato. Ci volevano dunque almeno seimila rubli per le spese correnti e ce n'erano solo 1.800. Per un uomo con 100.000 rubli di rendita, a tanto si riteneva ammontasse il patrimonio di Vronskij, questi debiti potevano non sembrare troppo gravosi; ma erano ben lontani da lui quei 100.000 rubli! L'enorme patrimonio paterno, che rendeva da solo 200.000 rubli all'anno, era indiviso fra i fratelli. Al tempo in cui il fratello maggiore, pieno di debiti, s'era ammogliato con la principessina Varja cirkova, figlia del decabrista, senza un soldo, Aleksej aveva ceduto al fratello maggiore tutta la rendita del patrimonio paterno, riservando per sé solo 25.000 rubli all'anno. Aleksej aveva detto allora al fratello che questa somma gli sarebbe stata sufficiente fino al giorno in cui non si sarebbe ammogliato, il che probabilmente non si sarebbe verificato mai. E il fratello, comandante di uno dei reggimenti più fastosi e da poco sposato, fu ben lieto di accettare un simile dono. La madre, che aveva un patrimonio a sé, oltre ai 25.000 rubli fissi, dava ad Aleksej 20.000 rubli all'anno, e Aleksej li spendeva presto. Nell'ultimo tempo la madre, indispettita con lui per la sua relazione e per la sua partenza da Mosca, non gli aveva più mandato quel denaro. E perciò Vronskij, abituato a vivere con 45.000 rubli e ricevutine all'anno solo 25.000, si trovava in difficoltà. Per uscirne non poteva chiedere il denaro alla madre. L'ultima sua lettera, ricevuta il giorno prima, l'aveva particolarmente irritato, perché ella faceva intendere d'esser pronta ad aiutarlo perché avesse successo in società e in carriera, ma non per condurre una vita che scandalizzava tutta la buona società. L'intento della madre di ricattarlo l'aveva offeso nel profondo dell'anima ed aveva aumentato la sua freddezza verso di lei. D'altra parte, egli non poteva ritrattare la sua generosa rinunzia a favore del fratello, pur sentendo confusamente, in previsione di alcune eventualità derivanti dalla sua relazione con la Karenina, che quella generosa rinunzia era stata fatta con leggerezza, e che a lui, pur non sposato, potevano far comodo tutti i 100.000 rubli di rendita. Ma ritrattarsi non si poteva. Gli bastava solo pensare alla moglie del fratello, ricordare come quella gentile e simpatica Varja in ogni occasione opportuna gli ripetesse ch'ella ricordava la sua generosità e che l'apprezzava tanto, per capire l'impossibilità di togliere quello che era stato dato. Era impossibile quanto percuotere una donna, quanto rubare o mentire. Una sola cosa era possibile e si doveva fare, e ad essa Vronskij si decise senza un attimo di esitazione: prendere in prestito da un usuraio diecimila rubli, e questo non sarebbe stato difficile, ridurre in genere le proprie spese e vendere i cavalli da corsa. Deciso ciò, egli scrisse subito un biglietto a Rolandaki che più di una volta gli aveva proposto di comprargli i cavalli. Dopo mandò a chiamare l'inglese e l'usuraio, e distribuì secondo i conti i denari che aveva. Terminati questi affari, scrisse una fredda e tagliente risposta alla lettera della madre. Dopo, tirati fuori dal portafoglio tre biglietti di Anna, li rilesse, li bruciò e, riandando con la mente alla conversazione del giorno innanzi, si fece pensoso. 

\capitolo{XX}\label{xx-2} 

La vita di Vronskij era così particolarmente serena perché egli si era fatto un codice di regole che definiva in modo sicuro quello che si doveva e quello che non si doveva fare. Questo codice abbracciava una cerchia di casi molto limitata, ma in compenso queste norme erano sicure e Vronskij, non uscendo mai da quella cerchia, non aveva mai tentennamenti nelle sue azioni. Queste norme stabilivano in modo non dubbio che un baro lo si dovesse pagare, ma che non era necessario pagare il sarto; che non si dovesse mentire agli uomini, ma alle donne sì; che non si dovesse ingannare nessuno, ma che un marito lo si poteva ingannare senz'altro; che si dovessero perdonare le offese, ma che si poteva offendere, e via di seguito. Regole, queste, che potevano essere assurde, cattive, ma che erano sicure; adempiendole, Vronskij si sentiva tranquillo e poteva andare a testa alta. Negli ultimi tempi, però, in seguito alla sua relazione con Anna, Vronskij aveva cominciato a rendersi conto che il codice delle sue norme non contemplava proprio tutti i casi e che in futuro si sarebbero presentati dubbi e difficoltà nei quali egli già non trovava il filo conduttore. 

Gli attuali rapporti suoi con Anna e il marito erano per lui semplici e chiari. Essi erano chiaramente ed esattamente definiti nel codice di regole dalle quali egli si faceva guidare. 

Ella era una donna per bene che gli aveva dato il proprio amore, ed egli l'amava, perciò ella era per lui una donna degna dello stesso, e anche maggiore, rispetto che una moglie legittima. Si sarebbe fatto tagliare una mano prima di offenderla con una parola, con un'allusione, o di non mostrarle tutta la considerazione sulla quale può contare una donna. 

I rapporti con la società erano chiari anch'essi. Tutti potevano sapere, sospettare, ma nessuno doveva osare di parlare della sua relazione. In caso contrario era pronto a far tacere quelli che avrebbero parlato e a far rispettare l'onore, non più esistente, della donna che egli amava. 

I rapporti col marito erano i più chiari di tutti. Sin dal momento in cui Anna si era innamorata di lui, egli riteneva di avere su di lei, egli solo, un suo proprio diritto indiscutibile. Il marito era solo un personaggio superfluo e fastidioso. Senza dubbio ci faceva una figura pietosa, ma che farci? Un solo diritto aveva il marito: quello di pretendere soddisfazione con l'arma alla mano, e a questa eventualità Vronskij era stato pronto fin dal primo momento. 

Ma recentemente erano apparsi dei rapporti nuovi, intimi tra lui e lei, che avevano sconvolto Vronskij per la loro indeterminatezza. Appena ieri, ella gli aveva detto di essere incinta. Ed egli aveva sentito che questa notizia e la risposta ch'ella si aspettava da lui esigevano qualcosa che non rientrava nel codice delle norme che dirigevano la sua vita. Infatti era stato preso alla sprovvista, e al primo momento, quando ella gli aveva detto la cosa, il cuore gli aveva suggerito di pretendere che lasciasse il marito. Lo aveva subito detto, ma ora, riflettendo, vedeva chiaro che sarebbe stato meglio farne a meno; e intanto, dicendosi questo, temeva che ciò fosse riprovevole. 

``Se ho detto di lasciare il marito, questo significa unirsi con me. Sono io pronto a questo? Come la porterò via adesso, se non ho denari? Ammettiamo che a questo potrei provvedere\ldots{} ma come portarla via se sono tuttora in servizio? Ma se l'ho detto, è necessario che io sia pronto a farlo, debbo cioè avere del denaro e debbo dare le dimissioni''. 

E rifletteva. La questione di dare o no le dimissioni lo aveva portato a meditare su un altro suo intimo interesse, noto a lui solo, ma essenziale, anche se nascosto, per la sua vita. 

Il successo era una vecchia ambizione della sua infanzia e della sua giovinezza; sogno ch'egli non confessava neppure a se stesso, ma che era così forte che anche ora questa passione lottava col suo amore. I suoi primi passi nella società e nella carriera erano stati fortunati, ma due anni addietro aveva commesso un grosso errore. Per dar prova della propria indipendenza e di voler progredire, aveva rifiutato una posizione offertagli, sperando che questo rifiuto potesse conferirgli maggior prestigio; accadde invece che fu giudicato troppo temerario, e fu lasciato stare; e ora, volente o nolente, acquistatasi questa fama di uomo libero, cercava di sostenerla, comportandosi con finezza e intelligenza, in modo da parere che non avesse rancore contro nessuno, che non si considerasse offeso da nessuno, e che volesse solo starsene in pace, perché contento di sé. Ma, in fondo, fin dall'anno scorso, quando era andato a Mosca, aveva cessato di esserlo. Sentiva che questa condizione di uomo indipendente, che può tutto e non vuole nulla, cominciava a diventar piatta; già molti cominciavano a pensare ch'egli non avrebbe potuto nulla, fuorché essere un onesto e bravo ragazzo. La sua relazione con la Karenina, che aveva fatto tanto scalpore, e che aveva attirato l'attenzione generale, dandogli nuovo prestigio, aveva calmato per un certo tempo in lui il tarlo dell'ambizione; ma da una settimana in qua questo tarlo s'era ridestato con rinnovata energia. Un amico d'infanzia, della stessa cerchia, dello stesso ambiente, e suo compagno al corpo dei paggi, Serpuchovskoj, licenziatosi con lui e suo rivale in classe e in ginnastica, in birbonate e in sogni ambiziosi, era tornato in quei giorni dall'Asia centrale, dopo aver ricevuto due promozioni e una ricompensa che era data di rado a generali così giovani. 

Appena giunto a Pietroburgo, si era parlato di lui come di un astro di prima grandezza che sorgeva. Coetaneo di Vronskij e compagno suo di collegio, egli era generale e aspettava una nomina che poteva avere influenza sul corso degli affari di stato, mentre lui, Vronskij, sebbene indipendente e brillante e amato da una donna deliziosa, era un semplice capitano al quale si lasciava la libertà di essere indipendente quanto e come voleva. ``S'intende, io non invidio e non posso invidiare Serpuchovskoj, ma il suo successo mi dimostra che basta aspettare il momento buono, e la carriera di un uomo come me può essere fatta ben presto. Tre anni fa egli era nella stessa condizione nella quale mi trovo io ora. Dando le dimissioni, brucerei le mie navi. Rimanendo in servizio non perdo nulla. Ella stessa ha detto che non vuole cambiare lo stato delle cose. E io che posseggo il suo amore, non posso invidiare Serpuchovskoj''. E, arricciandosi con un movimento lento i baffi, si alzò dalla tavola e fece un giro per la stanza. I suoi occhi splendevano in modo particolarmente chiaro ed egli sentiva quella disposizione d'animo ferma, tranquilla e gioiosa che lo prendeva sempre quando aveva chiarito la propria posizione. Tutto era così netto e preciso come dopo i conti che aveva sistemato poco prima. Si rase la barba, s'immerse in un bagno freddo e uscì. 

\capitolo{XXI}\label{xxi-2} 

- E io ti vengo dietro. Il bucato è durato un pezzo, oggi - disse Petrickij. - Be', è finito? 

- È finito - rispose Vronskij, sorridendo soltanto con gli occhi e arricciando la punta dei baffi così cautamente come se, dopo l'ordine in cui erano stati messi i suoi affari, ogni movimento troppo ardito e lesto potesse distruggerlo. 

- Fatto questo sembra proprio che tu esca da un bagno - disse Petrickij. - Io vengo da Griška - così chiamavano il comandante del reggimento - ti aspettano. 

Vronskij, senza rispondere, guardò il compagno, pensando ad altro. 

- Sì, c'è musica da lui? - disse, prestando orecchio alle note emesse dalla cornetta a tempo di polca e di valzer che giungevano fino a lui. - Cos'è, c'è festa? 

- È arrivato Serpuchovskoj. 

- Ah - disse Vronskij - nemmeno lo sapevo. 

Il sorriso dei suoi occhi brillò ancor più chiaramente. 

Una volta che aveva stabilito con se stesso d'esser felice del suo amore e di aver sacrificato ad esso la propria ambizione, assunta, almeno, questa parte, Vronskij non poteva sentire né invidia per Serpuchovskoj, né irritazione verso di lui perché, arrivato al reggimento, non era venuto da lui per primo. Serpuchovskoj era un buon amico, ed egli era felice di rivederlo. 

- Ah, ne sono lieto. 

Il comandante del reggimento, Demin, occupava una grande casa di possidenti. Tutta la compagnia era sul vasto terrazzo di sotto. Nel cortile, la prima cosa che saltò agli occhi furono i cantanti in uniforme estiva, in piedi, accanto a una piccola botte di vodka, e la sana, allegra figura del comandante circondato dagli ufficiali. Venendo fuori sul primo gradino del terrazzo, costui, gridando più forte della musica che sonava una quadriglia di Offenbach, ordinò qualcosa e fece alcuni cenni ai soldati che stavano da un lato. Il gruppo di soldati, di marescialli e di sottufficiali si accostò al terrazzo insieme a Vronskij. Tornato presso al tavolo, il comandante del reggimento venne fuori sulla scala con una coppa in mano e pronunciò il brindisi: ``Alla salute del nostro antico compagno e valoroso generale, principe Serpuchovskoj. Urrà!''. 

Dietro il comandante uscì anche Serpuchovskoj con una coppa in mano. 

- Tu diventi sempre più giovane, Bondarenko - disse rivolto a un ben fatto, rubicondo maresciallo che era stato richiamato in servizio per la seconda volta, e che stava diritto davanti a lui. 

Vronskij non vedeva Serpuchovskoj da tre anni. Questi aveva preso un aspetto più maschio con le fedine più folte, ma era rimasto snello quale era e sorprendeva, non tanto per la bellezza, quanto per la delicatezza e nobiltà del viso e della figura. Il solo mutamento che Vronskij notò in lui, fu quel calmo continuo splendore che si fissa sul volto delle persone che hanno successo e che sono sicure del riconoscimento di questo successo da parte di tutti. Vronskij conosceva questo splendore e subito lo notò in Serpuchovskoj. 

Scendendo la scala, Serpuchovskoj scorse Vronskij. Un sorriso di gioia gli illuminò il volto. Fece un cenno con la testa, sollevò la coppa, salutando Vronskij e mostrando con questo gesto che voleva avvicinarsi prima al maresciallo che, inchinatosi, piegava già le labbra al bacio. 

- Su, ecco anche lui! - gridò il comandante del reggimento. - E Jašvin mi ha detto che eri di umore nero! 

Serpuchovskoj dette un bacio sulle umide e fresche labbra del bel giovane maresciallo e, asciugandosi la bocca col fazzoletto, si accostò a Vronskij. 

- Eh, come son contento! - disse stringendogli la mano e appartandosi con lui. 

- Occupatevi di lui! - gridò a Jašvin il comandante del reggimento, indicando Vronskij, e scese giù dai soldati. 

- Perché ieri non eri alle corse? Pensavo di vederti là - disse Vronskij esaminando Serpuchovskoj. 

- Sono venuto, ma tardi. Perdona - soggiunse, e si rivolse all'aiutante di campo. - Per favore ordinate di distribuire da parte mia a ognuno il suo. 

Ed in fretta, tirò fuori dal portafogli tre biglietti da cento rubli e arrossì. 

- Vronskij! Qualcosa da mangiare, o da bere? - chiese Jašvin. - Ehi, da' da mangiare qui al conte. Ed ecco, bevi. 

La baldoria dal comandante si protrasse a lungo. 

Si bevve molto. Dondolarono e gettarono in aria Serpuchovskoj. Dopo si fece dondolare il comandante del reggimento. Poi, davanti ai cantanti, ballò lo stesso comandante con Petrickij. Dopo, il comandante del reggimento, già infiacchito, sedette su di una panca nel cortile e cominciò a dimostrare a Jašvin la superiorità della Russia sulla Prussia, specie nell'attacco di cavalleria, e per un momento la baldoria si chetò. Serpuchovskoj entrò in casa, nella stanza da toletta, per lavarsi le mani, e ci trovò Vronskij che si versava addosso dell'acqua. Toltasi la divisa estiva e messo il collo rosso, coperto di peli, sotto il getto d'acqua del lavabo, frizionava il corpo con le mani. Finita l'abluzione, Vronskij sedette accanto a Serpuchovskoj. Tutti e due s'erano seduti su di un divanetto e tra loro cominciò una conversazione che interessava molto entrambi. 

- Io di te ho saputo tutto attraverso mia moglie - disse Serpuchovskoj. - Sono contento che tu la veda spesso. 

- È amica di Varja, e queste sono le uniche donne di Pietroburgo con le quali mi vedo volentieri - rispose, sorridendo Vronskij. Sorrideva perché prevedeva il tema su cui si sarebbe svolta la conversazione e gli faceva piacere. 

- Le uniche? - chiese di rimando, sorridendo, Serpuchovskoj. 

- Sì, e anch'io sapevo di te, ma non solo attraverso tua moglie - disse Vronskij, respingendo quella vaga allusione con un'espressione severa del volto. - Sono stato molto contento del tuo successo, ma per nulla affatto sorpreso. Mi aspettavo ancora di più. 

Serpuchovskoj sorrise. Gli faceva piacere, era evidente, l'opinione che si aveva di lui e non cercava di nasconderlo. 

- Io, al contrario, lo confesso sinceramente, m'aspettavo di meno. Ma sono contento, molto contento. Sono ambizioso, è questa la mia debolezza, lo confesso. 

- Forse non lo confesseresti, se non avessi successo - disse Vronskij. 

- Non credo - disse Serpuchovskoj, sorridendo di nuovo. - Non dico che non potrei vivere senza di questo, ma mi annoierei. S'intende, forse sbaglio, ma mi sembra di avere delle possibilità in quella sfera di azione che ho scelto, e mi pare che nelle mie mani il potere, quale che sia, se ci sarà, starà meglio che nelle mani di molti a me noti - disse Serpuchovskoj con la raggiante consapevolezza del successo. - E perciò quanto più sono vicino alla mèta, tanto più sono contento. 

- Forse questo va così per te, ma non per tutti. Io pensavo lo stesso, ma ecco che vivo e trovo che non vale la pena vivere solo per questo - disse Vronskij. 

- Eccolo, eccolo! - disse ridendo Serpuchovskoj. - Io avevo già cominciato a dire che avevo sentito parlare di te, del rifiuto\ldots{} S'intende, io ti ho approvato. Ma in ogni cosa ci vuole la misura. E io penso che il gesto in sé è stato buono, ma tu non hai agito così come si sarebbe dovuto. 

- Quel ch'è fatto è fatto; e tu sai, io non rimpiango mai. E poi sto benissimo. 

- Benissimo\ldots{} per un po' di tempo. Ma poi questo non ti basterà. Non direi così a tuo fratello. È un caro ragazzo, come questo nostro padrone di casa. Vedi - aggiunse, prestando orecchio al grido di ``urrà'' - anche lui si diverte, ma questo non può accontentare te. 

- Io non dico d'esser soddisfatto. 

- Già, ma non è solo questo. Uomini come te sono necessari. 

- A chi? 

- A chi? Alla società. La Russia ha bisogno di uomini, ha bisogno di un partito, altrimenti tutto va alla deriva. 

- Che cosa allora? Il partito di Bertenev contro i comunisti russi? 

- No - disse Serpuchovskoj, accigliandosi per la stizza di vedersi sospettato di una simile sciocchezza. - Tout ça est une blague. Questo è sempre stato e sarà. Non c'è nessun comunista. Ma le persone intriganti hanno sempre sentito la necessità di inventare un partito nocivo, pericoloso. Questo è un vecchio sistema. No, c'è bisogno di un partito di governo, di persone indipendenti come te e come me. 

- Ma perché mai? - e Vronskij nominò alcune persone che erano al potere. - Ma perché dici che non vi sono uomini indipendenti? 

- Solo perché non hanno o non hanno avuto dalla nascita una posizione indipendente, non hanno avuto un nome, né quella vicinanza al sole così come abbiamo avuto noi sin dalla nascita. Costoro si possono comprare col denaro o con la protezione. E lasciano passare delle idee e certe tendenze in cui essi non credono affatto, che danneggiano, al solo fine di avere una casa dal governo e tanto di stipendio. Cela n'est pas plus fin que ça, quando guardi nelle loro carte. Forse io sarò peggiore o più sciocco di loro. Ma ho certamente un vantaggio rilevante: che è più difficile comprarmi. E uomini cosiffatti sono più che mai necessari. 

Vronskij ascoltava attentamente, ma lo interessava non tanto il contenuto delle parole, quanto il modo col quale considerava le cose Serpuchovskoj, che pensava già di lottare per il potere e in quel mondo aveva già le sue simpatie e antipatie; mentre per lui nella carriera rientravano soltanto gli interessi dello squadrone. Vronskij intendeva quanto potesse essere forte Serpuchovskoj con la sua indubbia capacità a comprendere le cose, con la sua intelligenza e con il dono della parola così raro nella sfera in cui viveva. E per quanto se ne vergognasse, provava invidia. 

- Tuttavia per questo mi manca la dote principale - rispose - il desiderio del potere. L'ho avuto ma è passato. 

- Perdonami, non è vero - disse, sorridendo, Serpuchovskoj. 

- No, è vero, è vero, ora, ad essere sincero - aggiunse Vronskij. 

- Se è vero ora è un'altra cosa; ma questa ora non ci sarà sempre. 

- Può darsi - rispose Vronskij. 

- Tu dici, può darsi - continuò Serpuchovskoj, come indovinando il suo pensiero - e io ti dico certamente. E per questo volevo vederti. Tu hai agito così come dovevi. Questo lo capisco, ma perseverare non devi. Io ti chiedo solo carte blanche. Io non ti proteggo\ldots{} Benché, poi, perché non dovrei proteggerti? Tu hai protetto me tante volte! Spero che la nostra amicizia sia al di sopra di questo. Sì - egli disse, sorridendo teneramente come una donna. - Dammi carte blanche, esci dal reggimento e io ti rimetterò dentro inavvertitamente. 

- Ma capisci, non ho bisogno di nulla - disse Vronskij - se non di questo, che tutto continui ad essere così com'è stato. 

Serpuchovskoj si alzò e gli si mise di fronte. 

- Tu hai detto: che tutto continui ad essere così com'è stato. Io capisco perché dici così. Ma ascolta: noi siamo coetanei, può darsi che tu abbia conosciuto donne in maggior numero di me. - Il sorriso e i gesti di Serpuchovskoj dicevano che Vronskij non doveva temere, ch'egli avrebbe sfiorato con delicatezza, con riguardo il punto dolente. - Ma io sono ammogliato e, credimi, che pur conoscendo soltanto la propria moglie (come ha scritto qualcuno), se la ami, conosci tutte le donne meglio che se ne avessi conosciute mille. 

- Veniamo subito - gridò Vronskij all'ufficiale che era entrato un momento nella stanza e li invitava ad andare dal comandante del reggimento. 

Vronskij voleva ora ascoltare e sapere che cosa l'amico gli avrebbe detto. 

- Ed eccoti la mia opinione. Le donne sono la principale pietra di inciampo nell'attività di un uomo. È difficile amare una donna e fare qualcosa. C'è un solo mezzo per amare comodamente e scansare gli ostacoli, e questo mezzo è il matrimonio. Come, come dirti quello che penso - disse Serpuchovskoj, cui piacevano i paragoni. - Aspetta, aspetta! Sì, è come portare un fardeau e fare qualcosa con le mani; si può solo quando il fardeau è legato alla schiena, e questo è il matrimonio. E questo io l'ho sentito dopo essermi sposato. Mi si sono liberate a un tratto le mani. Ma senza il matrimonio, a trascinarsi dietro questo fardeau, le mani sono così impegnate, che non si può far nulla. Guarda Mazankov, Krupov. Si son giocata la carriera per le donne. 

- Quali donne! - disse Vronskij, pensando alla francese e all'attrice con cui erano in relazione le due persone nominate. 

- Tanto peggio se è più alta la posizione della donna in società: tanto peggio. È come se, invece di trascinare il fardeau con le mani, lo si strappasse a un altro. 

- Tu non hai mai amato - disse piano Vronskij, guardando avanti a sé e pensando ad Anna. 

- Forse. Ma ricordati quel che ti ho detto. E ancora. Le donne hanno tutte più senso pratico che non gli uomini. Noi facciamo dell'amore qualcosa d'immenso, ma esse sono sempre terre-à-terre. 

- Subito, subito! - disse rivolto al servo che era entrato. Ma il servo non era venuto per chiamarli, come egli pensava. Il servo portava un biglietto a Vronskij. 

- Un uomo ha portato questo da parte della principessa Tverskaja. 

Vronskij dissuggellò la lettera e diventò rosso. 

- M'è venuto mal di testa, vado a casa - disse a Serpuchovskoj. 

- Allora, addio. Mi dai carte blanche? 

- Ne riparleremo dopo, ti troverò a Pietroburgo. 

\capitolo{XXII}\label{xxii-2} 

Erano già le sei e perciò, per giungere in tempo e non andare con i propri cavalli che tutti conoscevano, Vronskij prese posto nella vettura di Jašvin e ordinò di andare il più presto possibile. La vecchia carrozza di piazza a quattro posti era ampia. Sedette in un angolo, distese le gambe sul sedile davanti e si fece pensieroso. 

La coscienza confusa di quella sistemazione che aveva dato ai suoi affari, il ricordo vago dell'amicizia di Serpuchovskoj che lo riteneva un essere necessario e, soprattutto, l'attesa dell'incontro, tutto si fondeva in un unico gioioso senso di vita. Questa sensazione era così forte che egli involontariamente sorrise. Tirò giù le gambe, mise l'una sul ginocchio dell'altra, e, presala in mano, tastò il polpaccio elastico della gamba ferita il giorno prima nella caduta, e, riversatosi all'indietro, respirò varie volte a pieni polmoni. 

``Bene, molto bene!'' si disse. Anche altre volte aveva provato la gioiosa sensazione del proprio corpo come ora. Gli piaceva sentire quel leggero dolore nella gamba solida, gli piaceva la sensazione muscolare del movimento del proprio petto nel respirare. Quella stessa chiara e fresca giornata d'agosto, che così disperatamente aveva agito su Anna, pareva a lui eccitante e vivificante e gli rinfrescava il viso e il collo accaldati dall'abluzione. L'odore della brillantina dei suoi baffi gli pareva particolarmente piacevole in quell'aria fresca. Tutto ciò che vedeva dal finestrino della carrozza, in quell'aria fredda e tersa, nella luce pallida del tramonto era egualmente fresco, allegro e forte come lui; così i tetti delle case, che rilucevano ai raggi del sole calante, e i contorni netti dei recinti e degli angoli delle costruzioni, così le sagome dei pedoni e delle vetture che si incontravano di rado, così il verde immobile degli alberi e delle erbe, e il campo con i solchi regolari delle patate, così le ombre contorte, cadenti dalle case e dagli alberi, dai cespugli, e dagli stessi solchi delle patate. Tutto era bello come un grazioso paesaggio allora allora finito e ricoperto di lacca. 

- Va', va' - disse, sporgendosi dal finestrino, e, tirato fuori dalla tasca un biglietto da tre rubli, lo ficcò in mano al vetturino che s'era voltato verso di lui. La mano del vetturino tastò qualcosa vicino al fanale, si sentì il fischio della frusta e la vettura rotolò in fretta sul lastrico levigato. 

``Non ho bisogno di nulla, oltre questa felicità - pensava, guardando il bottoncino d'osso del campanello tra gli spazi dei finestrini e immaginandosi Anna così come l'aveva vista l'ultima volta. - E più passa il tempo e più l'amo. Ecco anche il giardino della villa governativa della Vrede. Dov'è mai? Dove? Come? Perché ha fissato qui l'appuntamento e ha scritto in una lettera di Betsy?'' pensava soltanto ora; ma non aveva già più tempo di pensare. Fece fermare i cavalli prima di arrivare al viale e, aperto lo sportello, saltò giù dalla carrozza in corsa e andò per il viale che porta alla casa. Nel viale non c'era nessuno ma, guardando a destra, scorse lei. Aveva il viso nascosto da un velo, ma egli, in uno sguardo gioioso, avvolse il movimento particolare, tutto suo, dell'andatura, dell'abbandono delle spalle, e della posizione del capo, e immediatamente qualcosa di simile a una corrente elettrica percorse il suo corpo. Sentì con rinnovata forza se stesso, dai movimenti elastici delle gambe, fino al moto dei polmoni in respirazione, e qualcosa gli vellicò le labbra. 

Incontratisi ella gli strinse forte la mano. 

- Non ti dispiace se ti ho fatto venire? Mi era indispensabile vederti - ella disse, e la piega seria e severa delle labbra ch'egli scorse di sotto al velo mutò di colpo la sua disposizione d'animo. 

- Io spiacente! Ma come sei venuta, da dove? 

- Non mette conto - ella disse, poggiando il braccio su quello di lui - andiamo, devo parlarti. 

Egli capì che qualcosa era accaduto e che quell'incontro non sarebbe stato lieto. Quando era con lei non aveva una volontà propria: non sapeva le ragioni dell'agitazione di lei e sapeva già che quella stessa agitazione gli si sarebbe comunicata. 

- Che c'è, che c'è? - chiedeva stringendo il braccio di lei col gomito e cercando di leggerle i pensieri nel viso. 

Ella fece qualche passo in silenzio e, facendosi coraggio, improvvisamente si fermò. 

- Non ti ho raccontato ieri - cominciò, respirando in fretta e con pena - che tornando a casa con Aleksej Aleksandrovic, io gli ho detto che non potevo più essere sua moglie, che\ldots{} tutto gli ho detto. 

Egli l'ascoltava, chinandosi involontariamente e con tutto il corpo, desiderando con questo di alleviare a lei il peso della sua situazione. Ma dopo quelle parole si drizzò improvvisamente e il suo viso prese un'espressione orgogliosa e severa. 

- Sì, sì, è meglio, mille volte meglio! Capisco come sia stato penoso - disse. 

Ma lei non ascoltava le sue parole, gli leggeva i pensieri nell'espressione del viso. Ella non poteva sapere che quell'espressione del viso si collegava alla prima idea che era venuta in mente a Vronskij: all'inevitabilità, adesso, del duello. A lei non era neppure venuta in mente l'idea del duello, e perciò dette una diversa spiegazione a questa fugace espressione di severità. 

Ricevuta la lettera del marito, ella sapeva già in fondo all'anima che tutto sarebbe rimasto come prima e ch'ella non avrebbe avuto la forza di buttar via la sua posizione sociale, di abbandonare il figlio e di unirsi all'amante. La mattinata trascorsa dalla principessa Tverskaja l'aveva rafforzata ancor più in questo: tuttavia questo incontro era straordinariamente importante per lei. Ella sperava che l'incontro avrebbe cambiato la loro situazione, che l'avrebbe salvata. Se egli a quella notizia, risolutamente, appassionatamente, senza un attimo di esitazione le avesse detto: ``lascia tutto e fuggi con me'' ella avrebbe abbandonato il figlio e sarebbe andata con lui. Ma la notizia datagli non produsse in lui l'effetto ch'ella s'attendeva: egli stava lì come offeso di qualcosa. 

- Non mi è stato per nulla penoso. È avvenuto da sé - ella disse con irritazione - ed ecco\ldots{} - ella trasse fuori dal guanto la lettera del marito. 

- Capisco, capisco - egli la interruppe, dopo aver preso la lettera e cercando, senza leggerla, di calmarla; - io desideravo una cosa sola, chiedevo una cosa sola, uscir fuori da questa situazione per dedicare la mia vita alla tua felicità. 

- Perché mi dici questo? - ella disse. - Posso forse dubitarne? Se dubitassi\ldots{} 

- Chi è che viene? - disse a un tratto Vronskij, indicando due signori che venivano alla loro volta. - Può darsi che ci conoscano - e in fretta si diresse in un viottolo laterale, tirandosela appresso. 

- Ah, per me è lo stesso! - ella disse. Le sue labbra tremavano. E a lui pareva che gli occhi di lei lo guardassero di sotto il velo con una strana cattiveria. - Così io dico che non è questo che importa, ora: di questo io non posso dubitare; ma ecco, cosa egli mi scrive. Leggi. - Si fermò di nuovo. 

Di nuovo come nel primo momento della notizia della rottura di lei col marito, Vronskij, nel leggere la lettera, si lasciò andare a quella sensazione istintiva che destavano in lui i rapporti col marito offeso. Ora, mentre teneva la lettera in mano, involontariamente si raffigurava la sfida che forse quel giorno stesso o l'indomani avrebbe trovato a casa, e persino il duello durante il quale, con quella stessa fredda e orgogliosa espressione che aveva in quel momento, avrebbe sparato in aria, e sarebbe rimasto sotto la mira del marito offeso. E a questo punto gli era balenato in mente quello che poco prima gli aveva detto Serpuchovskoj e che egli stesso aveva pensato la mattina, che sarebbe stato meglio non legarsi, e sentiva che questo suo pensiero non poteva certo comunicarlo a lei. 

Leggendo la lettera, egli alzò gli occhi su di lei, ma nel suo sguardo non c'era decisione alcuna. Ella capì subito che egli aveva già prima pensato qualcosa su questo dentro di sé. Ella sapeva che ora, qualunque cosa dicesse, non le avrebbe detto tutto quello che pensava. L'ultima speranza era delusa. E questo non se lo aspettava. 

- Tu vedi che uomo è - ella disse con voce tremante; - egli\ldots{} 

- Perdonami, ma io sono contento di questo - aggiunse Vronskij. - Grazie a Dio, lasciami finire di parlare - soggiunse, supplicandola con uno sguardo di dargli il tempo di spiegare le sue parole. - Sono contento perché questa faccenda non può, non può assolutamente rimanere così come egli suppone. 

- Perché non può? - prese a dire Anna, trattenendo le lacrime, evidentemente non dando ormai alcun valore a quello che egli avrebbe detto. Ella sentiva che il suo destino era deciso. 

Vronskij voleva dire che dopo il duello, inevitabile secondo lui, quello stato di cose non sarebbe potuto continuare, ma disse altro. 

- Non può continuare. Spero che adesso lo lascerai. Io spero - si confuse e arrossì - che mi permetterai di dare ordine e provvedere alla nostra vita. Domani\ldots{} - e voleva continuare. 

Ella non lo lasciò finire. 

\begin{itemize} \itemsep1pt\parskip0pt\parsep0pt \item E mio figlio? - gridò. - Vedi cosa scrive? Dovrei lasciarlo, ma io non voglio e non posso fare questo. \end{itemize} 

- Ma, in nome di Dio, cosa è meglio? Lasciare il figlio o continuare a vivere in questa situazione umiliante? 

- Umiliante per chi? 

- Per tutti, e più di tutti per te. 

- Tu dici, umiliante\ldots{} non lo dire. Queste parole non hanno senso per me - ella disse con voce che le tremava. Non voleva, ora, che egli le dicesse ciò che non sentiva. Le rimaneva solo l'amore di lui e voleva amarlo. - Tu capisci che dal giorno che ho cominciato ad amarti, tutto per me è cambiato. Per me non c'è che una sola cosa, il tuo amore. Se questo è mio, allora mi sento così in alto, così forte che nulla per me può essere umiliante. Sono orgogliosa del mio stato perché\ldots{} orgogliosa che\ldots{} orgogliosa\ldots{} - Non finì di pronunciare di che cosa fosse orgogliosa. Lacrime di vergogna e di disperazione soffocarono la sua voce. Tacque e scoppiò in singhiozzi. 

Anch'egli sentiva qualcosa venirgli su verso la gola e vellicargli il naso, e per la prima volta nella sua vita sentì che stava per piangere. Non avrebbe potuto dire che cosa proprio l'avesse commosso tanto; aveva pena di lei e sentiva che non poteva aiutarla, mentre egli era colpevole dell'infelicità sua, egli le aveva fatto del male. 

- Non è forse possibile il divorzio? - disse piano. Ella scosse il capo senza rispondere. - Non si può forse pretendere tuo figlio e lasciare lui? 

- Sì, ma tutto dipende da lui. Ora è da lui che devo andare - ella disse seccamente. Il suo presentimento che tutto sarebbe rimasto come prima non l'aveva ingannata. - Martedì sarò a Pietroburgo e si deciderà. 

- Sì - disse. - Ma non parliamo più di questo. 

La vettura che Anna aveva mandato via e che aveva fatto poi venire al cancello del giardino delle Vrede, si accostò. Ella salutò Vronskij e andò a casa. 

\capitolo{XXIII}\label{xxiii-2} 

Il lunedì c'era la solita seduta della commissione del 2 giugno. Aleksej Aleksandrovic entrò nell'aula della riunione, salutò, come al solito, i membri e il presidente, e sedette al suo posto, poggiando le mani sulle carte preparate davanti a lui. Fra queste carte c'erano anche le notizie necessarie e lo schema della proposta che aveva deciso di fare. Del resto non gli occorrevano neppure gli appunti. Ricordava tutto e non aveva bisogno di ripetersi mentalmente quello che avrebbe detto. Sapeva che, al momento opportuno, visto davanti a sé il viso dell'avversario che invano avrebbe cercato di darsi un'aria indifferente, il discorso sarebbe venuto fuori da sé, molto meglio che se lo avesse preparato adesso. Prevedeva che il contenuto del discorso sarebbe stato elevato, che ogni parola avrebbe avuto un significato. Frattanto, ascoltando la solita relazione, aveva l'aspetto più innocente, più inoffensivo. Nessuno avrebbe potuto sospettare, guardando le mani bianche dalle vene gonfie che palpavano così delicatamente con le dita lunghe le due estremità dei fogli di carta bianca posti davanti a lui, e quel capo chino da un lato con una impronta di stanchezza, che da un momento all'altro sarebbero usciti dalle sue labbra discorsi tali da scatenare una tempesta, da suscitare tra i commissari grida e reciproche interruzioni, tanto da costringere il presidente a richiamare all'ordine. Quando la relazione fu terminata, Aleksej Aleksandrovic con la sua voce calma, stridula, dichiarò ch'egli aveva da comunicare alcune sue considerazioni sulla questione della sistemazione degli allogeni. L'attenzione si rivolse a lui. Aleksej Aleksandrovic tossì e, senza guardare l'avversario, ma scelto, come sempre faceva nel pronunciare i suoi discorsi, il primo individuo che stava seduto dinanzi a lui - questa volta un vecchietto piccolo, tranquillo, che non aveva mai nessuna opinione - cominciò ad esporre le sue considerazioni. Quando si arrivò alla legge fondamentale e organica, l'avversario saltò su e cominciò a ribattere. Stremov, anche lui membro della commissione e anche lui colto nel vivo, cominciò a giustificarsi, e nell'insieme ne venne fuori una seduta tempestosa; ma Aleksej Aleksandrovic trionfò, e la sua proposta fu accolta; furono nominate tre nuove commissioni e il giorno dopo, in un certo ambiente di Pietroburgo, non si fece altro che parlare di questa seduta. Il successo di Aleksej Aleksandrovic fu persino maggiore di quello che egli si aspettava. 

La mattina dopo, martedì, Aleksej Aleksandrovic, svegliatosi, ricordò con soddisfazione la vittoria del giorno innanzi e non poté non sorridere, pur tentando di mostrarsi indifferente, quando il direttore della cancelleria, adulandolo, lo informò delle voci giunte fino a lui su quello ch'era accaduto in seno alla commissione. 

Intrattenendosi con il capo della cancelleria, Aleksej Aleksandrovic dimenticò completamente che quel giorno era martedì, giorno da lui fissato per l'arrivo di Anna Arkad'evna, e si meravigliò e dispiacque quando il servitore venne ad annunziarne l'arrivo. 

Anna era giunta a Pietroburgo la mattina presto, era stata mandata per lei la carrozza in seguito a un suo telegramma, perciò Aleksej Aleksandrovic doveva pur sapere del suo arrivo. Ma quando giunse, egli non le andò incontro. Le dissero che non era uscito dalla sua camera e che era occupato con il capo della cancelleria. Ella fece sapere al marito che era arrivata, andò nel proprio studiolo e si occupò di disfare le valigie, in attesa ch'egli venisse da lei. Ma passò un'ora, ed egli non si fece vivo. Ella uscì in sala da pranzo col pretesto di dare un ordine e parlò a voce alta proprio perché egli udisse e la raggiungesse là; ma non comparve sebbene ella si fosse accorta che era venuto fin sulla porta dello studio ad accompagnare il capo della cancelleria. Ella sapeva che, come al solito, sarebbe andato via presto per recarsi in ufficio, e desiderava vederlo prima per definire i loro rapporti. 

Fece un giro per la sala e si diresse decisamente verso lo studio. Quando vi entrò, egli, in uniforme d'ufficio, evidentemente pronto per andar via, era seduto accanto a un tavolino sul quale aveva poggiato i gomiti, e guardava tristemente davanti a sé. Ella lo vide prima che lui la scorgesse e capì che pensava a lei. 

Vistala, egli volle alzarsi, cambiò idea, ma il suo viso s'infiammò, cosa del tutto nuova per Anna, e in fretta s'alzò dirigendosi verso di lei e guardandola non negli occhi ma più in alto, sulla fronte e sull'acconciatura. Le si avvicinò, le prese la mano e la pregò di sedersi. 

- Sono molto contento che siate venuta - disse, sedendosi accanto a lei e, desiderando evidentemente di dire qualcosa, esitò. Parecchie volte egli fece per parlare, ma tacque. Sebbene nel prepararsi a quell'incontro ella avesse imparato a disprezzarlo e ad accusarlo, non sapeva cosa dirgli e aveva pietà di lui. E così il silenzio durò abbastanza a lungo. 

- Serëza sta bene? - egli disse e, senza aspettar risposta, soggiunse: - oggi non pranzerò a casa e ora devo andar via. 

- Io volevo andare a Mosca - ella disse. 

- No, avete fatto molto, molto bene a venire - egli disse e di nuovo tacque. 

Vedendo che egli non aveva la forza di cominciare a parlare, cominciò lei stessa. 

- Aleksej Aleksandrovic - disse guardandolo e senza abbassare gli occhi sotto lo sguardo di lui fisso sulla pettinatura - io sono una donna colpevole, sono una donna cattiva, ma sono la stessa che allora vi ha parlato, e sono venuta a dirvi che non posso cambiare in nulla. 

- Non vi ho detto questo - egli disse deciso e guardandola con odio diritto negli occhi - e questo proprio mi aspettavo. - Nell'impeto d'ira, era ritornato di nuovo padrone di tutte le sue facoltà. - Ma come vi ho detto allora e come vi ho scritto - prese a dire con voce tagliente, stridula - ora vi ripeto che io non sono obbligato a sapere questo. Io lo ignoro. Non tutte le mogli sono come voi generose tanto da affrettarsi a comunicare una notizia così piacevole ai mariti. - S'indugiò in modo particolare sulla parola ``piacevole''. - Io ignoro tutto ciò finché il mondo lo ignora, finché il mio nome non è svergognato. E perciò vi dico soltanto che i nostri rapporti devono essere quali sono sempre stati e che solo in caso che vi compromettiate, io sarò costretto a prendere delle misure per difendere il mio onore. 

- Ma i nostri rapporti non possono essere quelli di prima - disse Anna con voce timida, guardandolo con spavento. 

Nel vedere di nuovo quei gesti calmi, nel sentire quella voce penetrante, infantile e canzonatoria, la repulsione ch'ella sentiva per lui fece svanire quel sentimento di pietà che poco prima aveva sentito, e ora aveva soltanto paura; ma voleva, a ogni costo, chiarire la sua situazione. 

- Io non posso essere vostra moglie, quando\ldots{} - stava per cominciare. 

Egli si mise a ridere d'un riso cattivo. 

- Si vede che il genere di vita che avete scelto, si è riflesso sulle vostre idee. Per quel tanto che io rispetto e disprezzo e questo e quello\ldots{} rispetto il vostro passato, ma disprezzo il presente\ldots{} ero ben lontano dalla interpretazione che voi avete dato alle mie parole. 

Anna sospirò e chinò il capo. 

- D'altra parte non capisco come, avendo voi tanta spregiudicatezza - continuò, riscaldandosi - da annunziare a vostro marito la vostra infedeltà, senza trovare, a quanto sembra, nulla di biasimevole in questo, stimiate ora riprovevole l'adempimento dei doveri di moglie nei riguardi del marito. 

- Aleksej Aleksandrovic che cosa mai vi occorre da me? 

- Mi occorre non incontrare qui quell'uomo e che vi comportiate in modo che né il mondo né la servitù possano accusarvi\ldots{} che non lo vediate. Mi pare che non sia molto. E in compenso di questo godrete dei diritti di una moglie onesta, senza adempierne i doveri. Ecco tutto quello che ho da dirvi. Ora devo andar via. Non pranzo a casa. 

Si alzò e si diresse verso la porta. Anche Anna si alzò. Egli, inchinandosi senza proferire parola, si fece precedere da lei. 

\capitolo{XXIV}\label{xxiv-2} 

La notte che Levin trascorse sulla bica di fieno non passò invano per lui. L'azienda agricola che conduceva gli era divenuta d'un tratto odiosa, ed era divenuta priva di qualsiasi interesse per lui. Malgrado l'ottimo raccolto, non vi erano mai stati, o almeno mai gli era parso che ci fossero stati, tanto insuccesso e tanta ostilità tra lui e i contadini, come in quell'anno, e la causa di questo insuccesso e di questa ostilità gli si rivelava ora, in piena luce. Il fascino che aveva esercitato per lui lo stesso lavoro dei contadini, il contatto più intimo che, per questo, aveva avuto con loro, il senso di invidia che aveva provato per loro, per la loro vita, il desiderio di viverla, quella stessa loro vita, che, in quella notte, non era stato più un sogno, ma un proposito, di cui aveva riflettuto i particolari, tutto questo aveva mutato talmente la sua opinione circa l'azienda da lui condotta, che non poteva in alcun modo ritrovare, ora, in essa l'interesse di prima, e non poteva non vedere chiara la ragione dei suoi rapporti spiacevoli con i contadini, rapporti che costituivano la base della questione. Armenti di vacche bellissime, belle come la Pava, tutta la terra concimata, e rivoltata con gli aratri, nove campi eguali circondati da giunchi, novanta desjatiny di concio rivoltato in profondità, i seminativi in fila e via di seguito, tutto questo sarebbe stato bellissimo se fosse stato fatto da lui stesso e dai collaboratori, da uomini che simpatizzassero con lui. Ma egli ora vedeva chiaramente (il suo studio per un volume di economia rurale nel quale era proclamato come elemento essenziale in tale economia l'elemento lavoratore, lo aveva molto aiutato in questo) che l'azienda che egli conduceva rappresentava soltanto una crudele e ostinata lotta tra lui e i lavoratori, nella quale da una parte, la sua, c'era un'incessante, intensa aspirazione a rifare tutto su di un modello ritenuto il migliore, dall'altra, invece, c'era l'ordine naturale delle cose. E scorgeva che in questa lotta, la massima tensione di forze da parte sua e la mancanza di ogni sforzo e perfino di ogni proponimento dall'altra, pervenivano soltanto alla conseguenza che l'azienda non andava avanti e che si sciupavano attrezzi bellissimi, bestiame superbo, e terra. La cosa preminente era che non solo andava perduta del tutto la propria energia, ma che egli non poteva non sentire, ora che il senso della sua azienda gli si era rivelato, che lo scopo di questa energia fosse il meno degno. In sostanza, in che consisteva la lotta? Egli teneva dietro a ogni suo soldo (e non poteva non tenerci dietro perché gli bastava allentare per poco la sorveglianza per non avere denaro sufficiente per pagare i lavoratori); essi invece pensavano solo a lavorare tranquillamente e piacevolmente secondo la loro abitudine. Egli aveva interesse a che ogni lavoratore rendesse quanto più possibile, che non si distraesse, che badasse a non rompere i vagli, che riflettesse a quello che faceva; il lavoratore, invece, aveva interesse a lavorare nel modo più piacevole possibile, con respiro, e soprattutto senza preoccupazione, lasciandosi andare, senza pensare. E proprio in quell'estate Levin aveva constatato ciò ad ogni passo. Aveva mandato a falciare del trifoglio per seminarvi il fieno, scegliendo le desjatiny di terra meno buona, dove erano cresciute le erbe e l'artemisia, che non servivano per sementa, e gli avevano falciato le migliori desjatiny da semi, asserendo per giustificarsi che così aveva detto l'amministratore, e lo consolavano dicendo che il fieno sarebbe stato ottimo; ma egli sapeva che avevan fatto così perché quelle desjatiny di terra si falciavano con minor fatica. Aveva mandato un'essiccatrice a ventilare il fieno e l'avevano rotta ai primi giri, perché il contadino s'era annoiato di starci su seduto a cassetta sotto le ali che si agitavano. E gli dicevano: ``Degnatevi di non inquietarvi; le donne sparnazzeranno alla svelta''. Gli aratri s'erano mostrati inadatti, perché al lavoratore non entrava in mente di dover abbassare il dentale, e, girando con forza, l'aratro tormentava i cavalli e sciupava il terreno: e lo pregavano di non inquietarsene. I cavalli li avevan lasciati pascolare nel frumento perché non uno solo dei lavoratori voleva fare da guardiano notturno e, dato l'ordine di farlo senz'altro, avevan fatto a turno la guardia di notte e Van'ka, dopo aver lavorato tutto il giorno, si era addormentato, e aveva confessato il suo peccato, dicendo: ``Come volete voi''. Avevano fatto crepare le tre migliori vacche perché le avevano lasciate andare a pascolare là dove non c'era abbeveratoio su per il guaime del trifoglio, e non avevano voluto credere in nessun modo che si erano gonfiate col trifoglio, e raccontavano, per consolarsi, che al vicino erano morti centoventi capi di bestiame. Tutto questo lo facevano, non perché qualcuno di loro volesse male a Levin o alla sua azienda, al contrario, egli sapeva che gli volevano bene, lo consideravano un signore alla mano (che è la lode più alta); ma lo facevano solo perché volevano lavorare allegramente e senza affanno; e gl'interessi suoi erano non solo estranei e incomprensibili a loro, ma fatalmente opposti ai loro più legittimi interessi. Già da tempo Levin si sentiva insoddisfatto del suo modo di condurre l'azienda. Vedeva che la barca faceva acqua, ma non trovava e non cercava neppure la falla, ingannando di proposito se stesso. Ma ormai non poteva ingannarsi più. Quell'azienda che egli conduceva gli era divenuta non solo priva di interesse, ma odiosa, e non poteva occuparsene più. A questo si aggiungeva anche la presenza a trenta verste da lui di Kitty Šcerbackaja che egli voleva e non poteva vedere. Dar'ja Aleksandrovna Oblonskaja, quando egli era stato da lei, l'aveva invitato a tornare: andare per rinnovare la proposta di matrimonio a sua sorella che ora, da quanto gli si faceva capire, l'avrebbe accolta? Levin, rivedendo Kitty Šcerbackaja, aveva capito che non aveva cessato di amarla; ma egli non poteva andare dagli Oblonskij, sapendo che era là. Il fatto che la sua domanda di matrimonio era stata respinta, poneva fra lui e lei una barriera insormontabile. ``Io non posso chiederle di essere mia moglie solo perché ella non può essere la moglie di colui che desiderava'' diceva fra sé. Il pensiero di questo lo rendeva freddo e ostile. ``Non avrò la forza di parlare con lei senza doverle rimproverare qualche cosa, di guardarla senza rancore: ed ella mi odierà ancora di più, come del resto è prevedibile. E poi, come posso io, ora, dopo tutto quello che mi ha detto Dar'ja Aleksandrovna, andare da loro? Posso forse fingere di non sapere quello che mi ha detto? E andrei io, pieno di generosità, a perdonarla, a farle grazia? Io dinanzi a lei nella parte di chi la perdona e la degna del proprio amore! Perché mai Dar'ja Aleksandrovna mi ha parlato di questo? Avrei potuto vederla per caso, e allora tutto si sarebbe svolto da sé, ma ora è impossibile!''. 

Dar'ja Aleksandrovna gli mandò un biglietto, chiedendogli una sella da signora per Kitty. ``Mi hanno detto che avete una sella - gli scriveva. - Spero che la porterete voi stesso''. 

Questa cosa non la poteva proprio sopportare. Come mai una donna intelligente e delicata poteva umiliare a tal punto la sorella? Scrisse dieci biglietti e li strappò uno dopo l'altro, e mandò la sella senza rispondere. Scrivere che sarebbe andato, non poteva, perché non poteva andare; scrivere che non poteva andare perché qualcosa glielo impediva o perché partiva, era ancora peggio. Mandò la sella senza la risposta, e il giorno dopo, con la coscienza di aver compiuto qualcosa di vergognoso, affidata l'azienda divenutagli odiosa all'amministratore, partì per un lontano distretto dove c'erano bellissime paludi da beccacce, ospite del suo amico Svijazskij che da poco gli aveva scritto, pregandolo di attuare l'antico progetto di recarsi un po' da lui. Le paludi da beccacce nel distretto di Surov tentavano già da tempo Levin, ma per gli affari dell'azienda aveva sempre rinviato questo viaggio. Ora invece era contento di allontanarsi dagli Šcerbackij e, soprattutto, dall'azienda, e di andare a caccia, cosa che, in tutte le sue amarezze, era sempre stata per lui la migliore delle consolazioni. 

\capitolo{XXV}\label{xxv-2} 

Per il distretto di Surov non c'era strada ferrata né diligenza, e Levin andò coi cavalli suoi, in un tarantas. 

A mezza strada si fermò a mangiare da un ricco contadino. Il vecchio calvo, arzillo, con una barba rossiccia, canuta sulle guance, aprì il portone, serrandosi contro lo stipite, per lasciar passare la trojka. Mostrato al cocchiere il posto sotto la tettoia nel cortile vasto, nuovo, pulito e ben curato, dove c'eran degli aratri bruciacchiati, il vecchio invitò Levin a entrare nella stanza. Una giovane donna pulitamente vestita, con gli zoccoli ai piedi scalzi, strofinava curva il pavimento di un ingresso nuovo. Ella si spaventò del cane che era corso dietro a Levin e dette un grido, ma subito rise del proprio spavento, accortasi che il cane non l'avrebbe toccata. Mostrata a Levin col braccio dalla manica rimboccata la porta della stanza, nascose di nuovo, curvandosi, il suo bel viso e continuò a lavorare. 

- Il samovar, eh? - chiese. 

- Sì, per favore. 

La stanza era grande, con una stufa olandese e un'intelaiatura. Sotto le icone c'erano una tavola pitturata a disegni, una panca e due sedie. All'entrata un armadietto con le stoviglie. Le imposte erano chiuse, c'erano poche mosche e tutto era così pulito che Levin si preoccupò che Laska, avendo corso per via ed essendosi bagnata nelle pozzanghere, non avesse a sporcare il pavimento, e le indicò un posto in un angolo accanto alla porta. Dopo aver guardato la stanza, Levin uscì nel cortile dietro la casa. La giovane donna, bella a vedersi, con gli zoccoli e i secchi vuoti che faceva oscillare sulla stanga, corse davanti a lui a prendere acqua dal pozzo. 

- Fa' presto - gridò allegramente dietro di lei il vecchio, e si accostò a Levin. - Ebbene, signore, andate da Nikolaj Ivanovic Svijazskij? Anche lui si ferma da noi - cominciò ciarliero, appoggiandosi coi gomiti alla balaustra della scala. 

Mentre il vecchio raccontava della sua conoscenza con Svijazskij, il portone cigolò ed entrarono i lavoratori che tornavan dal campo con gli aratri e gli erpici. I cavalli attaccati agli aratri ed agli erpici erano ben pasciuti e grandi. I lavoratori evidentemente erano gente di casa: due erano giovani e avevano le camicie d'indiana e i berretti; gli altri due, uno vecchio e l'altro giovane, erano a opra, e avevano le camicie di canapa. Allontanatosi dall'ingresso, il vecchio si avvicinò ai cavalli e prese a staccarli. 

- Che cosa hanno arato? - chiese Levin. 

- Hanno arato in giro in giro per le patate. Anche noi teniamo un pezzetto di terra. Tu, Fedot, non lasciare andare il castrato, mettilo invece vicino al trogolo, ne attaccheremo un altro. 

- Ohi, babbo, quei vomeri che avevo ordinato di prendere, li ha portati sì o no? - chiese il giovane, robusto di statura, evidentemente figlio del vecchio. 

- Nel\ldots{} nella slitta - rispose il vecchio, avvolgendo ad anello le redini abbandonate e gettate a terra. - Metti in ordine intanto che mangiano. 

La giovane donna, bella a vedersi, con le brocche piene che le facevano tendere le spalle, attraversò l'ingresso. Apparvero da qualche parte altre donne, alcune giovani, belle, di mezza età ed altre vecchie, brutte, con bambini e senza bambini. 

Il samovar cominciò a brontolare nel tubo; gli operai e quelli di casa, posti in stalla i cavalli, andarono a mangiare. Levin tirò fuori dalla carrozza le sue provviste e invitò il vecchio a bere il tè. 

- L'ho già bevuto oggi - disse il vecchio, accettando con evidente piacere la proposta. - Ma se è per farvi compagnia\ldots{} 

Prendendo il tè, Levin seppe tutta la storia dell'azienda del vecchio. Il vecchio aveva affittato dieci anni addietro centoventi desjatiny da una proprietaria, e l'anno precedente se le era comprate e ne aveva prese in affitto altre trecento da un proprietario vicino. Una piccola parte del terreno, la peggiore, l'aveva data in affitto, ma quaranta desjatiny nel campo le arava lui con la famiglia e con due opre a giornata. Il vecchio si lamentava che gli affari andavano male. Ma Levin capiva che egli si lamentava solo per abitudine, e che l'azienda andava bene. Se le cose fossero andate male, egli non avrebbe comprato a centocinque rubli, non avrebbe dato moglie a tre suoi figliuoli e a un nipote, non avrebbe ricostruito due volte dopo gli incendi e tutto non sarebbe andato sempre di bene in meglio. Malgrado le lamentele del vecchio si vedeva che era giustamente orgoglioso del proprio benessere, orgoglioso dei figli, dei nipoti, delle nuore, dei cavalli, delle mucche e, in particolare, del fatto che egli mandava avanti tutta quella azienda. Dalla conversazione col vecchio, Levin capì ch'egli non era alieno dalle innovazioni. Seminava molte patate, e le patate, che Levin aveva visto mentre si avvicinava alla fattoria, sfiorivano già e cominciavano a germogliare, quando da lui cominciavano appena a spuntare. Egli arava sotto le patate con ``l'aratra'', come egli chiamava l'aratro preso in prestito dal proprietario. Seminava il frumento. Il particolare che, sarchiando la segala, il vecchio dava da mangiare ai cavalli la segala sarchiata, colpì molto Levin. Quante volte, vedendo questo ottimo mangime andar perduto, voleva che si raccogliesse!, ma ciò risultava sempre impossibile. Il contadino invece lo utilizzava e non si stancava di far le lodi di un mangime simile. 

- E le donnette allora che devono fare? Portano i mucchi sulla strada e il carro si avvicina. 

- Ed ecco, invece, da noi proprietari tutto va male coi lavoratori - disse Levin, dandogli un bicchiere col tè. 

- Grazie - rispose il vecchio; prese il bicchiere, ma rifiutò lo zucchero, indicando la pallottolina ch'era restata, rosicchiata da lui. - E come condurre l'azienda con gli operai? - egli disse. - Una rovina. Ecco, prendiamo, sia pure Svijazskij. Noi sappiamo che la terra è la sua: una bellezza; pure anche lui non ha a lodarsi del raccolto. Tutta incuria! 

- Ma, dimmi, tu fai andare avanti l'azienda con gli operai? 

- Ma questo è affar nostro di contadini. Possiamo arrivare a far tutto da soli. Se un operaio non rende, va via: ce la facciamo anche da soli. 

- Babbo, Finogen ha detto di procurargli del catrame - disse, entrando, la donna con gli zoccoli. 

- Proprio così, signore! - disse il vecchio, alzandosi, si fece il segno della croce lentamente, ringraziò Levin e uscì. 

Quando Levin entrò nella capanna da lavoro per chiamare il cocchiere, vide tutti gli uomini della famiglia a tavola. Le donne in piedi servivano. Un giovane e robusto figliuolo, con la bocca piena di zuppa, raccontava qualcosa di buffo e tutti ridevano, e in particolare la donna con gli zoccoli che scodellava la minestra di cavolo nella tazza. 

Può darsi benissimo che il bel viso della donna con gli zoccoli avesse contribuito molto a dare a Levin l'impressione di buona amministrazione in quella casa di contadini; ma quest'impressione fu così profonda che Levin non poté in nessun modo distoglierne il pensiero. E per tutta la strada, dalla casa del vecchio a quella di Svijazskij, vi pensò continuamente, come se l'impressione che quell'azienda gli aveva fatto esigesse da lui una particolare attenzione. 

\capitolo{XXVI}\label{xxvi-2} 

Svijazskij era maresciallo della nobiltà nel suo distretto. Aveva cinque anni più di Levin ed era ammogliato da molto tempo. Nella sua casa viveva una giovane cognata, che era molto simpatica a Levin. E Levin sapeva che Svijazskij e sua moglie desideravano molto dargli in moglie questa ragazza. Lo sapeva con certezza, come lo sanno sempre i cosiddetti pretendenti, tuttavia non lo avrebbe mai detto a nessuno, e sapeva pure che, nonostante volesse ammogliarsi, nonostante che quella ragazza molto attraente sarebbe dovuta essere, secondo le informazioni, un'ottima moglie, tuttavia gli sembrava tanto impossibile, anche se non fosse stato innamorato di Kitty, di sposare lei, quanto volare in cielo. E questa consapevolezza gli avvelenava il piacere che sperava di ricavare dalla visita a Svijazskij. 

Ricevuta la lettera di Svijazskij con l'invito per la caccia, Levin aveva pensato a questa circostanza, e aveva finito per giudicare un'infondata supposizione le mire che attribuiva a Svijazskij e così, malgrado tutto, decise di andare. Inoltre, in fondo all'animo, voleva provarsi, voleva misurarsi ancora una volta con questa ragazza. La vita domestica degli Svijazskij era straordinariamente piacevole, e lo stesso Svijazskij era il miglior tipo di amministratore pubblico della provincia e, appena l'aveva conosciuto, aveva suscitato in Levin un grande interesse. 

Svijazskij era una di quelle persone che sempre s'imponevano all'attenzione di Levin: persone, il cui modo di ragionare molto coerente, anche se non originale, fila diritto per conto proprio, e la cui vita precisamente definita e salda nelle direttive, scorre poi da sé, in modo del tutto autonomo e quasi sempre in senso opposto al ragionamento. Svijazskij era un uomo di idee più che mai liberali e disprezzava la nobiltà perché riteneva che in maggioranza i nobili fossero segreti fautori della servitù della gleba e restassero in ombra per vigliaccheria. Considerava la Russia un paese rovinato, tipo Turchia, e stimava il governo russo spregevole tanto da non degnarsi, lui, neppure di criticarne seriamente gli atti, e intanto prestava servizio per quel governo, ed era un perfetto maresciallo della nobiltà e per istrada portava sempre il berretto con la coccarda e l'orlo rosso. Riteneva che una vita degna di un uomo si potesse condurla soltanto all'estero, dove andava non appena se ne presentasse l'occasione, e intanto in Russia conduceva con vivissimo interesse un'azienda molto vasta e perfezionata, e seguiva e conosceva tutto quello che avveniva in Russia. Considerava il contadino russo qualcosa di intermedio fra la scimmia e l'uomo, e nello stesso tempo, alle elezioni provinciali, era lui che più volentieri di tutti stringeva la mano ai contadini e ascoltava le loro opinioni. Non credeva a nulla, né al diavolo, né all'acqua santa, ma si preoccupava molto del problema del miglioramento delle condizioni del clero e della riduzione delle parrocchie, e si affannava inoltre perché la chiesa del suo villaggio non fosse soppressa. 

Nella questione femminile era dalla parte degli estremi fautori della completa libertà della donna e particolarmente del suo diritto al lavoro, ma viveva con la moglie in modo che tutti ammiravano la loro affettuosa vita familiare priva di figli, e organizzava la vita di sua moglie in modo ch'ella non dovesse e non potesse far nulla, oltre che occuparsi, in compagnia del marito, del come passar meglio e il più allegramente possibile il tempo. 

Se Levin non avesse posseduto la speciale facoltà di dar peso alla parte migliore delle persone, il carattere di Svijazskij non avrebbe presentato per lui nessuna difficoltà e nessun problema; si sarebbe detto: ``è uno sciocco o un poco di buono'' e tutto sarebbe stato chiaro. Ma egli non poteva dire che fosse stupido perché Svijazskij era senza dubbio non solo un uomo molto intelligente, ma anche molto colto e portava il suo bagaglio culturale con non comune semplicità. Non c'era materia ch'egli non conoscesse; ma delle sue cognizioni dava prova solo quando vi era costretto. Ancora meno Levin poteva dire che fosse un poco di buono, perché indubbiamente Svijazskij era un uomo onesto, buono, intelligente, che, allegramente, con vivacità e senza posa, faceva un lavoro molto apprezzato da tutti quelli che lo circondavano, e certo poi non faceva e non avrebbe saputo fare nulla di male con intenzione. 

Levin cercava di capire, e non capiva: e guardava sempre a lui e alla sua vita come a un enigma vivente. 

Erano amici e perciò Levin si permetteva di insistere con Svijazskij pel desiderio di attingere il fondo della sua visione della vita; ma non gli riusciva mai. Ogni volta che Levin tentava di penetrare più in là delle stanze di ricevimento della mente di Svijazskij, aperte a tutti, notava che Svijazskij si turbava un poco; un'ansietà appena percettibile appariva nel suo sguardo, come se temesse che Levin potesse capirlo, e gli opponeva perciò una benevola e allegra resistenza. 

Ora, dopo la sua delusione per l'azienda, era per Levin oltremodo piacevole starsene un po' da Svijazskij. A parte la vista di quei colombi felici, contenti di loro stessi e di tutti, del loro nido ben ordinato, piaceva ora a Levin, sentendosi così scontento della vita, veder raggiunto in Svijazskij quel segreto che gli dava tanta chiarezza, precisione e allegria nella vita. Oltre a ciò, Levin sapeva che avrebbe incontrato dagli Svijazskij alcuni proprietari vicini e in questo momento lo interessava ascoltar quei tali discorsi sull'azienda, sul raccolto, sull'assunzione degli operai e via di seguito; discorsi che Levin era solito considerare come qualcosa di molto utile, ma che adesso gli sembravano i soli importanti. ``Questo, forse, non era importante ai tempi della servitù della gleba, o non è più importante in Inghilterra. In tutti e due i casi c'erano e ci sono delle condizioni quanto mai definite; ma da noi, poiché ora tutto ciò è stato messo sottosopra e si va appena appena assestando, la questione della sistemazione delle nuove condizioni è l'unica per la Russia'' pensava Levin. 

La caccia fu molto meno fortunata di quello che s'aspettava Levin. La palude era senz'acqua, e non c'erano beccacce. Camminò una giornata intera e ne ammazzò appena tre, ma in compenso riportò, come sempre dalla caccia, un appetito eccellente, un'eccellente disposizione d'animo e quel certo risveglio intellettuale che in lui si accompagnava sempre al moto fisico. E a caccia, quando gli pareva di non pensare a nulla, che è che non è, di nuovo gli tornava in mente il vecchio con la sua famiglia, e la viva impressione che gli era rimasta sembrava esigesse attenzione non solo per se stessa, ma anche perché gli pareva si collegasse alla soluzione di un qualche cosa. 

La sera, al tè, la conversazione con i due proprietari che erano giunti per certi affari di tutele, si svolse interessante così come Levin si era ripromesso. 

Levin sedeva accanto alla padrona di casa presso la tavola da tè e doveva tener viva la conversazione con lei e con la cognata che era seduta di fronte a lui. La padrona era una donna dal viso tondo, bionda e non alta, tutta splendente di fossette e sorrisi. Levin cercava attraverso lei di giungere alla soluzione dell'enigma per lui interessante del marito; ma non aveva piena libertà di pensiero, perché si sentiva tormentosamente a disagio. Si sentiva tormentosamente a disagio, perché davanti a lui sedeva la cognata con un vestito che gli pareva indossato apposta per lui, con una scollatura speciale a trapezio sul petto bianco; questo scollo quadrangolare, malgrado il petto fosse molto bianco, o proprio perché questo era molto bianco, toglieva a Levin la libertà di pensiero. Egli immaginava, probabilmente ingannandosi, che quello scollo era stato scelto proprio per lui, e non si riteneva in diritto di guardarlo e cercava di non guardarlo; ma sentiva di essere colpevole solo per il fatto che lo scollo era stato fatto. A Levin pareva di ingannare qualcuno, di dover chiarire qualche cosa, che però non si poteva in nessun modo chiarire, perciò arrossiva continuamente, era inquieto e a disagio. Il suo disagio si comunicava anche alla graziosa cognata. Ma la padrona pareva non avvedersene e faceva in modo che la ragazza partecipasse alla conversazione. 

- Voi dite - continuava la padrona - che tutto quello che è russo non può interessare mio marito. Al contrario, egli vive, sì, felice all'estero, ma non mai come qua. Qua egli si sente nel suo ambiente. Ha tanto da fare, e ha il dono di interessarsi a tutti. Ah, non siete stato nella nostra scuola? 

- Ho visto\ldots{} È una casetta circondata di edera? 

- Sì, è l'occupazione di Nast'ja - disse, indicando la sorella. 

- Insegnate proprio voi? - chiese Levin, cercando di guardare al di là dello scollo, sentendo però che, guardando da quella parte, dovunque fissasse gli occhi, avrebbe sempre visto quello scollo. 

- Sì, proprio io vi ho insegnato, e vi insegno, ma abbiamo un'ottima maestra. Anche la ginnastica abbiamo introdotto. 

- No, vi ringrazio, non voglio più tè - disse Levin e, pur sapendo di commettere una scortesia, non avendo più la forza di continuare la conversazione, si alzò, arrossendo. - Sento che là si parla di un argomento molto interessante - soggiunse e si avvicinò all'altro estremo della tavola dove sedeva il padrone di casa con i due proprietari. Svijazskij sedeva di fianco alla tavola, girando una tazza nella mano poggiata sul gomito, e con l'altra raccogliendo nel pugno la barba ch'egli avvicinava al naso e lasciava poi cader giù come se l'annusasse. Con gli occhi neri lucenti guardava fisso il proprietario dai baffi grigi che si animava, ed evidentemente quei discorsi lo divertivano. Il proprietario si lamentava dei contadini. Per Levin era ovvio che Svijazskij fosse in grado di dare una risposta alle lamentele del proprietario sì da annientare d'un tratto la sostanza di quei discorsi, ma che per la sua posizione non potesse darla, questa risposta, e che perciò ascoltasse, non senza compiacimento, il comico discorso del proprietario. 

Il proprietario dai baffi grigi, evidentemente, era un fautore convinto della servitù della gleba, un vecchio abitante di campagna, appassionato proprietario di terre. Questi segni Levin li scorgeva nell'abito, un soprabito fuori moda, frusto e decisamente non adatto per un proprietario, nei suoi occhi intelligenti, accigliati, nella parlata armoniosa, nel tono di comando acquisito ormai per lunga abitudine, e nei gesti risoluti delle mani grandi, belle, abbronzate, adorne solo del vecchio anello nuziale all'anulare. 

\capitolo{XXVII}\label{xxvii-2} 

- Se non ci dispiacesse lasciare quel che s'è avviato\ldots{} fatica se n'è fatta tanta\ldots{} butterei tutto all'aria, venderei, me ne andrei, come Nikolaj Ivanyc, a sentire La bella Elena - disse il proprietario con un sorriso cordiale che illuminò il suo vecchio viso intelligente. 

- Già, ecco, eppure non lo lasciate - disse Nikolaj Ivanovic Svijazskij - dunque c'è il tornaconto. 

- L'unico vantaggio è che vivo a casa mia, non è roba comprata, né presa in affitto. Già: e poi speri sempre che il popolo rinsavisca! Ma lo credereste? questa è un'ubriacatura, uno sbandamento. Si son divisi tutto, non c'è più una cavalla, né una vaccarella. Crepan di fame, ma intanto, prendete un operaio a giornata! Cercherà l'occasione per rovinarvi e ancora di mandarvi innanzi al giudice di pace. 

- In compenso anche voi lo denuncerete al giudice di pace - disse Svijazskij. 

- Lo denuncerò? Ma per nulla al mondo! Cominceranno tali discussioni che non ci sarà da stare allegri con la denuncia! Ecco: alla fabbrica hanno preso la caparra e sono andati via. Che ha fatto il giudice di pace? Li ha assolti. Tutto dovrebbe essere tenuto su dal tribunale del distretto e dall'anziano. Lui sì che li bastona all'uso antico! E se non ci fosse questo\ldots{} lascia via tutto! Fuggi in capo al mondo! 

Evidentemente il proprietario stuzzicava Svijazskij, ma Svijazskij non solo non si arrabbiava, ma, si vedeva, ci si divertiva. 

- Sì, ecco, noi conduciamo la nostra azienda senza ricorrere a codeste misure - egli disse sorridendo - io, Levin, il signore. 

E indicò l'altro proprietario. 

- Già, va' da Michail Petrovic, a chiedergli come fa. È forse razionale la sua azienda domestica? - disse il proprietario, facendo sfoggio evidente della parola ``razionale''. 

- Io ho un'azienda modesta - disse Michail Petrovic. - Ringrazio Iddio. Il mio modo di condurla consiste tutto nel far che siano pronti i soldi per le tasse d'autunno. Vengono i contadini: ``padrone, salvaci tu''. Ebbene, tutti i miei vicini son contadini, ti fan pena. E via, dài per il primo trimestre e di' soltanto: ``ragazzi, ricordatevi, vi ho aiutato, date anche voi una mano quando ci sarà bisogno: la semina dell'avena, la raccolta del fieno, la mietitura'': e così si stabilisce a tanto per tassa. Ci son quelli senza coscienza anche fra loro, è vero. 

Levin, che conosceva da tempo questi sistemi patriarcali, scambiò uno sguardo con Svijazskij e interruppe Michail Petrovic, rivolgendosi di nuovo al proprietario dai baffi grigi. 

- Allora, voi come la pensate? - chiese. - Come bisogna condurre, ora, un'azienda? 

- Sì, condurla al metodo di Michail Petrovic o darla a mezzadria o in fitto ai contadini si può; ma è proprio così che si distrugge la ricchezza generale della nazione. Da me, dove la terra rendeva nove col lavoro dei servi della gleba e una buona amministrazione, a mezzadria rende tre. L'emancipazione ha rovinato la Russia. 

Svijazskij guardò Levin con occhi ridenti e gli fece persino un segno canzonatorio appena percettibile, ma Levin non trovava ridicole le parole del proprietario; egli ne capiva il valore più di quanto non lo capisse Svijazskij. E molto di quello che disse poi il proprietario, per dimostrare perché la Russia era rovinata dall'emancipazione, gli parve perfino molto vero per lui, nuovo e incontestabile. Il proprietario, evidentemente, esponeva un'idea sua propria, il che accade di rado, e un'idea che era sorta in lui non come effetto del desiderio di occupare con qualche cosa il cervello ozioso, ma un'idea che era venuta fuori dalle contingenze stesse della vita, che egli aveva elaborato nella solitudine della campagna e che aveva esaminato da ogni lato. 

- La questione, permettetemi di osservare, è che ogni progresso lo si attua solo d'autorità - egli disse evidentemente per mostrare che anche lui non era estraneo alla cultura. - Prendete le riforme di Pietro, di Caterina, di Alessandro. Prendete la storia d'Europa. Tanto più per quanto riguarda il progresso della vita rurale. Anche la patata, quella pure è stata introdotta da noi d'autorità. Certamente c'è stato un tempo in cui non si conosceva nemmeno l'aratro primitivo. Anche questo l'hanno introdotto forse solo al tempo degli appannaggi, e certamente d'autorità. Al tempo nostro, quando c'era la servitù della gleba, noi proprietari conducevamo l'azienda attuando dei perfezionamenti; gli essiccatoi, i vagli, le concimaie, e tutti gli altri strumenti, tutto introducevamo d'autorità; e i contadini dapprincipio si opponevano, dopo ci imitavano. Ora, con l'abolizione della servitù, ci hanno tolto l'autorità, e le nostre aziende anche se già portate a un livello più alto, devono discendere a un livello più barbaro e primitivo. È così che io la intendo. 

- Ma perché mai? Se l'azienda è razionale la potete condurre con l'affitto - disse Svijazskij. 

- Ma se non c'è autorità! Con che mai la posso condurre? permettetemi di chiedere. 

``Eccola, la forza lavoratrice, l'elemento principale dell'economia'' pensò Levin. 

- Con i lavoratori. 

- Già, ma i lavoratori non vogliono lavorare bene e con strumenti buoni. Il nostro lavoratore fa solo una cosa: s'ubriaca come un porco, e guasta tutto quello che gli date. Abbevera i cavalli così da farli scoppiare, una bardatura buona la rompe, una ruota cerchiata ve la cambia e se la beve; nella macchina per la battitura ci getta un perno, per spezzarla. Tutto quello che non è fatto da lui lo disgusta. Proprio per questo si è abbassato tutto il livello dell'azienda rurale. Le terre sono abbandonate, sono coperte di assenzio o sono distribuite ai contadini; e là, dove ne producevano un milione, producono qualche centinaia di migliaia di stai di grano; in genere la ricchezza è diminuita. Se avessero fatto lo stesso, ma con misura! 

E cominciò a svolgere il suo piano di emancipazione secondo il quale questi inconvenienti sarebbero stati eliminati. 

A Levin non interessava questo piano; ma quando egli finì, Levin tornò alla sua prima tesi e disse, rivolto a Svijazskij e cercando di indurlo ad esprimere la sua ponderata opinione: 

- Il fatto che il livello dell'azienda si sia abbassato e che, dati i nostri rapporti con i lavoratori, non sia possibile condurre in maniera vantaggiosa un'azienda razionale, è del tutto vero - egli disse. 

- Non sono d'accordo - ribatté ormai con serietà Svijazskij. - Io vedo solo che noi non sappiamo condurre l'azienda e che, d'altra parte, quest'azienda che noi abbiamo condotto durante la servitù della gleba, certamente non era troppo alta, ma invece troppo bassa di livello. Ma noi non abbiamo macchine né buon bestiame da lavoro, non abbiamo una buona amministrazione, e non sappiamo fare i conti. Chiedete a un proprietario; egli non sa quello che gli conviene e quello che non gli conviene. 

- Contabilità all'italiana - disse ironico il proprietario. - In qualunque modo fai i conti, quando ti sciupano tutto, non c'è guadagno. 

- Perché ti sciupano? Una cattiva macchina per battere, il vostro topcak russo li spezzeranno; ma la mia macchina a vapore non la spezzeranno. Un ronzino russo, di razza da tiro, uno di quelli da trascinar per la coda, ve lo sciuperanno, ma mettete su dei percesi o almeno dei buoni cavalli da tiro, non ve li sciuperanno, e così per tutto. Occorre portare a un livello più alto l'azienda. 

- Ma ci fossero i mezzi Nikolaj Ivanovic! Voi state bene, ma io che debbo mantenere un figlio all'università, mandare i piccoli al ginnasio, io i percesi non me li posso comprare. 

- E per questo ci sono le banche. 

- Per costringermi a vendere all'asta le ultime cose? No, grazie. 

- Io non sono d'accordo che si debba o si possa sollevare lo stato dell'azienda domestica - disse Levin. - È di questo che mi occupo io e ne ho i mezzi, e non posso fare niente. Le banche non so a chi siano utili. Quanto a me, per qualunque cosa abbia speso del denaro nell'azienda, ho sempre perduto tutto: bestiame\ldots{} perdita, macchine\ldots{} perdita. 

- Ecco, la precisa verità - affermò, persino ridendo dalla soddisfazione, il proprietario dai baffi grigi. 

- E non sono il solo - continuò Levin - io mi appello a tutti i proprietari che conducono razionalmente una azienda; tutti, salvo rare eccezioni, lavorano in perdita. Su, dite voi, è forse attiva la vostra azienda? - disse Levin, e subito nello sguardo di Svijazskij notò quella fugace espressione di spavento che egli vi scorgeva ogni volta che voleva andare oltre le stanze da ricevimento della mente di Svijazskij. 

Inoltre questa domanda, da parte di Levin, non era del tutto onesta. La padrona di casa, durante il tè, gli aveva detto proprio allora che quell'estate avevano fatto venire da Mosca un tedesco esperto di computisteria che per cinquecento rubli aveva verificato i conti della loro azienda e aveva trovato tremila rubli e più di deficit. Non ricordava con precisione quanto, ma sembrava che il tedesco avesse spaccato il millesimo. 

Il proprietario, a sentir l'allusione ai profitti dell'azienda di Svijazskij, sorrise, conoscendo, evidentemente, quale potesse essere il profitto del vicino maresciallo della nobiltà. 

- Può darsi che sia infruttuosa - rispose Svijazskij. - Questo dimostra soltanto o che sono un cattivo padrone o che spendo il capitale per aumentare la rendita. 

- Ah, la rendita! - esclamò Levin con orrore. - Può darsi che in Europa si percepisca una rendita là dove la terra è stata migliorata dal lavoro, ma da noi tutta la terra è stata peggiorata, o rovinata dal lavoro; dunque niente rendita. 

- Come, non c'è rendita? Ma questa è di regola. 

- Allora siamo fuori regola. La rendita per noi non spiega nulla, al contrario, confonde. No, ditemi, come la teoria della rendita può essere\ldots{} 

- Volete del latte cagliato? Maša, portaci qua del latte cagliato o dei lamponi - disse rivolto alla moglie. - Quest'anno i lamponi si mantengono straordinariamente a lungo. 

E nel più piacevole stato d'animo Svijazskij si alzò e si allontanò, supponendo, evidentemente, che la conversazione fosse finita proprio nel punto in cui a Levin sembrava che fosse appena cominciata. 

Rimasto privo di un interlocutore, Levin continuò la conversazione col proprietario, cercando di dimostrargli che tutte le difficoltà dipendevano dal fatto che noi non vogliamo conoscere la peculiarità e le abitudini del lavoratore; ma il proprietario era, come tutti gli uomini che pensano col proprio cervello e in solitudine, tetragono alla comprensione d'un pensiero altrui e particolarmente appassionato al proprio. Egli insisteva sempre nel dire che il contadino russo è un porco, che ama la sporcizia e che per farlo uscire dalla sporcizia ci vuole autorità e questa non c'è, che occorre il bastone, e che invece si è diventati così liberali da cambiar d'un tratto il millenario bastone in certi avvocati e in certi incarceramenti, grazie ai quali si dava da mangiare una buona zuppa a questi cialtroni di contadini puzzolenti, preoccupandosi di calcolare il loro spazio vitale. 

- Perché non pensate - disse Levin, cercando di tornare alla questione - che si possa trovare con la forza lavoratrice un rapporto tale che il lavoro risulti produttivo? 

- Ciò non avverrà mai col popolo russo! Non c'è autorità - rispose il proprietario. 

- E come si possono trovare rapporti nuovi? - disse Svijazskij che, dopo aver mangiato del latte cagliato e fumato una sigaretta, si era avvicinato di nuovo a quelli che discutevano. - Tutti i rapporti possibili con la forza lavoratrice sono definiti e studiati - egli disse. - La comunità primordiale, un resto di barbarie, con la mutua garanzia, va in rovina da sé; il diritto di servitù è stato annientato, rimane solo il lavoro libero, e le sue forme sono definite e già bell'e pronte; non si può che prendere queste: l'operaio a giornata, il bracciante, il fittavolo, e di qua non si esce. 

- Ma l'Europa non è contenta di queste forme. 

- Non è contenta e ne cerca di nuove. Ne troverà, probabilmente. 

- Io dico solo questo - rispose Levin. - Perché non dobbiamo cercare anche noi da parte nostra? 

- Perché è lo stesso che inventare dei nuovi metodi per la costruzione delle strade ferrate. Sono pronti, sono già inventati. 

- Ma se non ci convengono, se sono idioti? - disse Levin. 

E di nuovo notò l'espressione di spavento negli occhi di Svijazskij. 

- Sì, questo: canteremo vittoria, abbiamo trovato quello che l'Europa cerca! Io so tutto questo, ma, perdonatemi, voi conoscete quel che è stato fatto in Europa a proposito della questione dell'organizzazione dei lavoratori? 

- No, non abbastanza. 

- Questa questione occupa ora le menti migliori in Europa. La tendenza di Schulze-Delitzsch\ldots{} Poi questa enorme letteratura sulla questione operaia, la tendenza più liberale di Lassalle\ldots{} L'organizzazione di Mühlhausen è già un fatto, forse lo sapete. 

- Ne ho un'idea, ma molto confusa. 

- No, lo dite voi; probabilmente saprete tutto questo non peggio di me. Io, s'intende, non sono un professore di sociologia, ma ciò mi interessa e, davvero, se v'interessa, occupatevene. 

- Ma a che mai sono pervenuti? 

- Perdonate\ldots{} 

I proprietari di terre s'erano alzati, e Svijazskij, avendo di nuovo fermato Levin nella sua antipatica abitudine di guardare quello che si trovava al di là delle stanze di ricevimento del proprio cervello, andò ad accompagnare gli ospiti. 

\capitolo{XXVIII}\label{xxviii-2} 

Levin s'annoiava terribilmente quella sera, con le signore: lo agitava, come non mai prima, il pensiero che quell'insoddisfazione che egli provava circa la sua azienda, fosse, non una sua personale situazione, ma una condizione generale delle cose in Russia; che la sistemazione dei rapporti con i lavoratori in modo che producessero così come da quel contadino presso il quale s'era fermato a mezza strada, non fosse un sogno, ma un problema che urgesse risolvere. E gli pareva che si dovesse tentare di risolverlo, questo problema. 

Dopo essersi congedato dalle signore e dopo aver promesso di rimanere ancora l'indomani per tutto il giorno, per andare insieme a cavallo a visitare una frana interessante nel bosco demaniale, Levin, prima di andare a dormire, passò un momento per lo studio del padrone di casa a prendere dei libri sulla questione operaia che Svijazskij gli aveva offerto. 

Lo studio di Svijazskij era una enorme stanza mobiliata con armadi pieni di libri e due tavole: uno scrittoio massiccio posto al centro della stanza, l'altra, una tavola rotonda, coperta degli ultimi numeri di giornali e di riviste in varie lingue disposti a raggiera intorno alla lampada. Presso lo scrittoio c'era un banco con delle cassette piene di pratiche di vario genere contrassegnate da etichette dorate. 

Svijazskij tirò fuori i libri e sedette su di una poltrona a dondolo. 

- Cos'è che guardate? - disse a Levin che, fermo vicino alla tavola, dava una scorsa alle riviste. 

- Ah, sì, qui c'è un articolo molto interessante - disse Svijazskij, indicando la rivista che Levin teneva tra le mani. - Vi è dimostrato - aggiunse con allegra animazione - che il responsabile principale della spartizione della Polonia non è stato affatto Federico. Risulta\ldots{} 

E con quella chiarezza che gli era propria, espose in breve quelle nuove, molto importanti e interessanti scoperte. Malgrado la mente di Levin fosse ora tutta occupata dal pensiero della azienda, egli ascoltava il padrone di casa domandandosi: ``Che c'è in lui? E perché, perché gli interessa la spartizione della Polonia?''. Quando Svijazskij finì, Levin involontariamente chiese: ``E allora?''. Ma non c'era nulla di interessante, c'era solo che ``risultava''. Ma Svijazskij non spiegò né trovò necessario spiegare perché questo era interessante. 

- Sì, ma mi ha molto interessato il proprietario rabbioso - disse Levin dopo aver sospirato. - È intelligente e ha detto molte cose esatte. 

- Oh, via! Un inveterato sostenitore segreto della servitù della gleba, come tutti loro - disse Svijazskij. 

- Di cui voi siete il capo. 

- Già, solo che io li capeggio dall'altra parte - disse, ridendo, Svijazskij. 

- Ecco cosa mi interessa molto - disse Levin. - Egli ha ragione di dire che gli affari nostri, cioè l'azienda razionale, non va; va solo l'azienda usuraia, come quella di quel bonaccione, o la più semplice\ldots{} Di chi la colpa? 

- S'intende, nostra. Ma già\ldots{} non è poi vero che non vada. Da Vasil'cikov va. 

- Una fabbrica\ldots{} 

- Tuttavia non so che cosa vi sorprenda. Il popolo è in uno stato così basso di sviluppo materiale e morale che, evidentemente, si oppone a tutto quello di cui ha invece bisogno. In Europa l'azienda razionale va perché il popolo è istruito; dunque da noi bisogna educare il popolo, ecco tutto. 

- Ma come istruire il popolo? 

- Per istruire il popolo ci vogliono tre cose: le scuole, le scuole, le scuole. 

- Ma voi avete detto che il popolo si trova a un grado di sviluppo materiale bassissimo. In che modo, allora, possono aiutare le scuole? 

- Sentite, voi mi ricordate l'aneddoto sui consigli a un malato: ``Dovreste provare un purgante''. ``Me l'hanno dato\ldots{} peggio\ldots{}''. ``Dovreste provare le sanguisughe''. ``Me le hanno applicate\ldots{} peggio''. E così siamo anche noi, io e voi. Io dico: economia politica, voi dite\ldots{} peggio. Io dico: socialismo, voi\ldots{} peggio, l'istruzione, voi\ldots{} peggio. 

- Ma in che modo mai le scuole potranno essere di aiuto? 

- Daranno al popolo altre esigenze. 

- Ecco, io questo non l'ho mai capito - ribatté Levin con calore. - In che modo le scuole possano aiutare il popolo a migliorare la sua condizione materiale. Voi dite, la scuola, l'istruzione, gli daranno nuovi bisogni. Tanto peggio, perché ancor meno avrà la capacità di soddisfarli. E in che modo la conoscenza dell'addizione e della sottrazione e del catechismo lo aiuteranno a migliorare il proprio stato materiale, io non l'ho mai potuto capire. L'altro ieri sera ho incontrato una donna con un bambino poppante e le ho chiesto dove fosse andata. Mi ha detto: ``Sono andata dalla mamma. Al bambino gli era venuto il frigno, così gliel'ho portato per farlo curare''. Le ho chiesto come la donnetta curasse quel piangere continuo. ``Mette a sedere il bambino sulla gruccia delle galline e poi dice qualche cosa''. 

- Ebbene, ecco, lo dite voi stesso. Per questo c'è bisogno di istruire il popolo, perché non vada a far curare il frigno sulla gruccia - disse Svijazskij, sorridendo allegramente. 

- Ah no - disse Levin con stizza. - Per me questo metodo di cura è simile a quello che vuol curare il popolo con le scuole. Il popolo è povero e ignorante; questo noi lo vediamo con la stessa chiarezza con la quale la fattucchiera vede il frigno, che è reso evidente dal fatto che il ragazzo frigna. Ma come e perché contro questo malanno della povertà e dell'ignoranza, debbano giovare le scuole è così incomprensibile come è incomprensibile che contro il frigno giovino le grucce delle galline. Bisogna aiutare il popolo in quanto è povero. 

- Bene, almeno in questo andate d'accordo con Spencer che amate così poco; anche lui dice che l'educazione può essere la conseguenza di un grande benessere e di un'agiatezza, di frequenti lavaggi, come egli dice, ma non del saper leggere e far di conto. 

- Ebbene, ecco, io sono molto contento, o, al contrario, molto poco contento di andar d'accordo con lo Spencer; io questo l'ho sempre pensato. Non sono le scuole che possono aiutare, ma gioverà un'organizzazione economica in virtù della quale il popolo sarà più ricco, e avrà più tempo libero; allora ci saranno anche le scuole. 

- Ma in tutta Europa attualmente le scuole sono obbligatorie. 

- E come mai voi stesso concordate in questo con lo Spencer? - domandò Levin. 

Ma negli occhi di Svijazskij balenò quella tale espressione di spavento, ed egli disse sorridendo: 

- No, questa faccenda del frigno è magnifica! Ma è proprio capitata a voi? 

Levin si accorse che non avrebbe mai trovato il nesso tra la vita di quell'uomo e le sue idee. Evidentemente a lui era del tutto indifferente la conseguenza a cui l'avrebbe portato il ragionamento. E si rammaricava quando il filo del ragionamento lo conduceva a volte in un vicolo cieco. Da questo rifuggiva, portando il discorso su qualcosa di piacevolmente allegro. 

Tutte le impressioni del giorno, a cominciare da quella del contadino presso il quale s'era fermato a mezza strada e che era servita come base fondamentale alle altre impressioni e a tutti i pensieri della giornata, avevano fortemente scosso Levin. Quel simpatico Svijazskij, che sfruttava le sue idee soltanto per uso pubblico ed evidentemente aveva altre basi di vita, misteriose per Levin, e che pertanto guidava l'opinione pubblica, una folla il cui nome è legione, con idee che non applicava alla vita propria; quel proprietario rabbioso, che diceva cose giustissime nei suoi ragionamenti, basati su esperienze concrete, ma che era ingiusto nella sua avversione contro tutta una classe, e la migliore classe della Russia; la sua stessa insoddisfazione della propria attività e la confusa speranza di trovar rimedio, tutto ciò confluiva in un senso di intima inquietudine e di attesa in una soluzione prossima. 

Rimasto nella camera assegnatagli, coricato su di un saccone a molle che lo faceva sobbalzare a ogni movimento del braccio o della gamba, Levin non dormì a lungo. Neppure la conversazione con Svijazskij, sebbene molte cose intelligenti fossero state dette fra di loro, interessava Levin; ma gli argomenti di quel proprietario esigevano un esame. Gli tornarono in mente tutte le sue parole e corresse dentro di sé la risposta che gli aveva dato. 

``Già avrei dovuto dirgli: voi dite che la nostra azienda non va perché il contadino odia tutti i perfezionamenti, e che occorre introdurli di autorità; ma se, senza questi perfezionamenti, l'azienda agraria non andasse del tutto, allora, sì, voi avreste ragione; ma essa invece va, ma soltanto dove il contadino opera in conformità con le sue abitudini, come lì dal vecchio che ho incontrato a mezza strada. Il fatto che non siamo soddisfatti della mia e della vostra azienda dimostra che colpevoli siamo o noi o i lavoratori. Noi già da tempo andiamo innanzi a modo nostro, all'europea, senza tener conto delle peculiarità dell'elemento lavoratore. Proviamo a considerare la forza lavoratrice non come una forza lavoratrice astratta, ma come contadino russo, con i suoi istinti, e costruiamo conformemente ad esso l'azienda. Immaginatevi, avrei dovuto dirgli, che da voi l'azienda vada come dal vecchio; che voi abbiate trovato il modo di interessare i lavoratori al buon successo del lavoro e che abbiate trovato nei perfezionamenti quello stesso punto di mezzo che essi apprezzano, e che infine, senza isterilire il suolo, ricaviate il doppio o il triplo in confronto di prima. Dividete a metà, date la metà alla forza lavoratrice; la differenza che vi rimane sarà più remunerativa per voi, e alla forza lavoratrice spetterà di più. Ma per fare ciò, occorre abbassare il livello dell'azienda e interessare i lavoratori alla prosperità di questa. Come riuscire, è questione di particolari, ma indubbiamente ciò è possibile''. 

Questa idea gettò Levin in una forte agitazione. Non dormì per metà della notte, pensando ai particolari dell'esecuzione della sua idea. Egli non aveva intenzione di partire il giorno dopo, ma ora decise di partire la mattina presto per tornare a casa. Inoltre quella cognata con la scollatura suscitava in lui un sentimento simile alla vergogna e al pentimento di una cattiva azione. La ragione precipua dell'immediata partenza era poi questa: occorreva proporre presto ai contadini il nuovo programma, prima della semina d'autunno, in modo che si potesse seminare già secondo i suoi nuovi criteri. Aveva deciso di dare un aspetto del tutto diverso all'azienda di prima. 

\capitolo{XXIX}\label{xxix-2} 

La realizzazione del piano di Levin presentava molte difficoltà; ma egli si batté con quanta forza aveva, e ne ottenne, sia pure non proprio quello che voleva, ma la convinzione almeno di poter credere, senza ingannarsi, che l'impresa valesse il lavoro. Una delle difficoltà principali consisteva nella circostanza che l'azienda era già avviata, che non si poteva fermare e riordinare daccapo, e che bisognava rinnovare la macchina mentre era in moto. 

Quando, la sera stessa in cui arrivò a casa, comunicò all'amministratore i suoi piani, l'amministratore con evidente soddisfazione, concordò con lui in quella parte del discorso in cui si dimostrava che quanto si era fatto finora era stato assurdo e poco conveniente. L'amministratore disse che egli l'aveva detto da tempo, e che non si era voluto ascoltarlo. Per quello poi che si riferiva alla proposta fattagli da Levin: prender parte come consocio insieme ai lavoratori in tutta l'azienda agricola, l'amministratore dette a vedere solo un grande scoraggiamento e nessuna opinione definita, e subito portò il discorso sulla necessità di trasportare l'indomani gli ultimi covoni di segala e di mandare a fare la seconda aratura; di modo che Levin capì che si era ancora lontani dal suo programma. 

Quando, infatti, lo prospettava ai contadini e faceva ad essi la proposta dell'affitto di terre a nuove condizioni, s'imbatteva pure nella difficoltà principale che essi, cioè, erano così occupati dalla fatica giornaliera da non avere il tempo di riflettere ai vantaggi e agli svantaggi dell'impresa. 

Un contadino semplicione, Ivan il pecoraio, parve capire in pieno la proposta di Levin, di prendere parte cioè, con la sua famiglia, agli utili della stalla, e aderì in pieno a questa impresa. Ma appena Levin gli prospettava i vantaggi futuri, sulla faccia di Ivan si esprimevano agitazione e rammarico per non poter rimanere ad ascoltare tutto fino all'ultimo, ed egli si dava in fretta a una qualche faccenda che non ammetteva dilazione: o afferrava la forca per finir di cacciare il fieno dal recinto, o cominciava a versare l'acqua, o a riassestare il letame. 

Un'altra difficoltà consisteva nell'invincibile sospetto dei contadini che lo scopo del proprietario non potesse consistere in altro che non fosse il desiderio di spogliarli il più possibile. Essi erano fermamente convinti che qualunque cosa avesse detto loro, il vero scopo sarebbe stato quello che egli non avrebbe detto. Ed essi stessi, nell'esporre il proprio modo di vedere, parlavano molto, ma non dicevano mai in che cosa consistesse il loro vero scopo. Inoltre (Levin sentiva che il proprietario bilioso aveva ragione), i contadini, come prima ed immutabile condizione per un qualsiasi accordo, ponevano quella di non essere costretti a introdurre un qualsiasi nuovo procedimento nell'azienda e l'uso di attrezzi nuovi. Essi convenivano che l'aratro arava meglio, che l'estirpatrice lavorava con maggiore rendimento, ma trovavano mille ragioni per non dover usare né l'uno né l'altra, e, sebbene egli fosse convinto che occorresse abbassare il livello dell'azienda, gli spiaceva rinunziare a perfezionamenti il cui vantaggio era tanto evidente. Malgrado tutte queste difficoltà, egli raggiunse il suo scopo, e verso l'autunno la cosa cominciò ad andare, così almeno gli pareva. 

Dapprima Levin pensò di dare tutta l'azienda, nello stato in cui era, in affitto ai contadini, ai lavoratori e all'amministratore, a nuove condizioni di compartecipazione; ma ben presto si convinse che ciò non era possibile, e si decise a suddividere l'azienda. La stalla, il giardino, l'orto, i prati, i campi, tutti divisi in settori, dovevano costituire parti separate. Quel semplicione d'Ivan il pecoraio, che pareva a Levin avesse capito la cosa meglio di tutti quanti, messa insieme un'artel' nella sua massima parte formata da persone di famiglia, diventò consocio della stalla. I campi lontani, che erano rimasti otto anni incolti sotto gli sterpi, furono assunti, con l'aiuto di un legnaiolo intelligente, Fëdor Rezunov, da sei famiglie di contadini alle nuove condizioni di compartecipazione, e il contadino Šuraev prese in affitto, alle stesse condizioni, tutti gli orti. Il resto rimase ancora come prima, ma queste tre parti erano già il principio della nuova sistemazione, e tenevano occupato completamente Levin. 

Era vero che nella stalla le cose finora non andavano meglio di prima, e Ivan si era vivamente opposto al locale caldo per le mucche e alla produzione del burro, asserendo che, se si tenevano le mucche in ambiente freddo, ci sarebbe voluta minor quantità di foraggio e che il burro di crema acida era più redditizio; ed era vero che pretendeva una paga come per l'innanzi, e non si dava per inteso che il denaro così pagatogli non era una paga, ma un'anticipazione sulla sua quota di partecipazione al guadagno. 

Era vero che i contadini dell'artel' di Fëdor Rezunov non avevano arato per la seconda volta con gli aratri prima della semina, così come era stato convenuto, giustificandosi che il tempo era stato breve. Era vero che i contadini di quest'artel', pur avendo pattuito di condurre la cosa su nuove basi, non dicevano che quella terra era condotta in comune, ma a mezzadria; e più di una volta i contadini di quell'artel' e lo stesso Rezunov avevano detto a Levin: ``se prendeste l'affitto per la terra, sareste più tranquillo voi e per noi sarebbe una liberazione''. Inoltre questi contadini rimandavano sempre, con vari pretesti, la convenuta costruzione di una stalla e di un granaio e avevan tirato in lungo fino all'inverno. 

Era vero che Šuraev avrebbe voluto distribuire gli orti da lui assunti in compartecipazione in piccoli lotti ai contadini. Evidentemente aveva capito tutto a rovescio, e pareva che, coscientemente, avesse capito a rovescio le condizioni alle quali era stata data la terra. Era vero che spesso, parlando con i contadini e spiegando loro tutti i vantaggi dell'impresa, Levin sentiva ad ogni momento che i contadini udivano solo il suono della sua voce e sapevano fermamente che qualunque cosa dicesse, essi non si sarebbero fatti ingannare da lui. In particolare egli sentiva questo, quando parlava con il più intelligente dei contadini, con Rezunov, e notava una certa scintilla negli occhi di lui che, mentre era irrisoria verso Levin, esprimeva anche la ferma convinzione che, se qualcuno dovesse essere ingannato, questo qualcuno non sarebbe certamente stato lui, Rezunov. 

Malgrado tutto questo Levin pensava che la faccenda andava e che, tenendo strettamente i conti e insistendo nelle sue idee, egli avrebbe mostrato loro, nel futuro, i vantaggi di una simile sistemazione, e la cosa allora sarebbe andata da sé. 

Tutte queste faccende, insieme al resto dell'azienda che era rimasta nelle sue mai, insieme allo studio per il suo libro, occuparono tanto Levin per tutta l'estate, che egli non andò quasi mai a caccia. Alla fine di agosto seppe, dalla persona venuta a riportare la sella, che gli Oblonskij erano andati a Mosca. Sentiva che non avendo risposto alla lettera di Dar'ja Aleksandrovna, aveva bruciato le sue navi con questa scortesia, che non poteva ricordare senza arrossire; e ormai non sarebbe più andato da loro. Proprio allo stesso modo aveva agito con Svijazskij, essendo partito senza salutare. Ma neanche da loro sarebbe più andato. Tutto questo gli era però indifferente. L'affare della nuova sistemazione dell'azienda lo occupava come mai finora nessun'altra cosa in vita sua. Lesse i libri datigli da Svijazskij, e, ordinati quelli che non aveva, lesse anche i libri di economia politica e di socialismo che riguardavano questo argomento, e, come si aspettava, non trovò nulla che si riferisse all'opera da lui intrapresa. Nei libri di economia politica, nello Stuart Mill, per esempio, che studiò per primo con grande entusiasmo, sperando di trovare ad ogni passo la soluzione delle questioni che lo interessavano, trovò delle leggi tratte dalla situazione dell'economia europea, ma non poté in nessun modo rendersi conto perché queste leggi, inapplicabili in Russia, dovessero essere dichiarate generali. Lo stesso vide anche nei libri sul socialismo; o erano bellissime fantasie, ma inattuabili, alle quali si era appassionato quando era ancora studente, o erano correzioni, revisioni di uno stato di cose in cui si era venuta a trovare l'Europa e con il quale l'agricoltura non aveva nulla in comune. L'economia politica diceva che le leggi, secondo le quali si era sviluppata e si sviluppava la ricchezza dell'Europa, erano leggi universali e indubitabili; e la dottrina socialista diceva invece che andare innanzi secondo queste leggi portava alla rovina. Né l'una né l'altra davano non già una soluzione, ma neppure il più piccolo accenno a quello che lui, Levin, e tutti i contadini e i proprietari terrieri russi dovessero fare con i loro milioni di braccia e di desjatiny di terra, perché fossero le une e le altre il più possibile produttive per il benessere generale. 

Datosi a questi studi, leggeva coscienziosamente tutto quello che riguardava la materia, e aveva intenzione di andare in ottobre all'estero per studiare la questione sul posto, affinché non gli capitasse più, in questa faccenda, quello che spesso gli era capitato in varie altre. Gli era accaduto che, appena aveva cominciato a capire il pensiero dell'interlocutore e a esprimere il suo, gli era stato detto a un tratto: ``E Kauffmann, e Johns, e Dubois, e Miceli? Non li avete letti? Leggeteli, hanno sviscerato tutta la questione''. 

Egli vedeva, ora, chiaramente, che Kauffmann e Miceli non avevano nulla da dirgli. Sapeva lui quello che ci voleva. Vedeva che la Russia aveva terre bellissime, ottimi lavoratori e che in alcuni casi, come dal contadino a mezza strada, i lavoratori e le terre rendevano molto; nella maggioranza dei casi, invece, quando i capitali venivano impiegati all'europea, producevano poco, e ciò dipendeva unicamente dal fatto che i lavoratori volevano lavorare e lavoravano bene solo a quel modo che era loro proprio, e che la loro opposizione alle novità non era casuale ma costante, avendo radici nello spirito stesso del popolo. Pensava che il popolo russo, che aveva la vocazione di dissodare con alacrità e di popolare enormi estensioni disabitate, finché le terre non fossero tutte occupate, si atteneva ai metodi necessari per questo lavoro, e che questi metodi non erano poi così cattivi come di solito si pensava. E voleva dimostrare ciò in teoria, nel libro, e, nella pratica, con la sua azienda agraria. 

\capitolo{XXX}\label{xxx-2} 

Alla fine di settembre fu trasportato il legname per la costruzione della stalla sul terreno concesso all'artel' e fu venduto il burro e se ne divise l'utile. Nell'azienda, in pratica, le cose andavano ottimamente, o, per lo meno, così sembrava a Levin. Per chiarire però teoricamente tutta la faccenda e terminare l'opera che, secondo i suoi sogni, avrebbe dovuto non solo portare una rivoluzione nell'economia politica, ma annientare completamente questa scienza, e instaurarne un'altra, basata sui rapporti del contadino con la terra, occorreva non solo andare all'estero e studiare sul posto tutto quello che era stato fatto con questo indirizzo, ma trovare argomenti convincenti per dimostrare che tutto quello che era stato fatto là, non era quello che occorreva fare. Levin attendeva solo la consegna del frumento per ritirare il denaro e andare all'estero. Ma cominciarono le piogge, che non permisero di raccogliere le patate e il grano rimasti nel campo, e fermarono tutti i lavori e perfino la consegna del frumento. Per le strade c'era un fango insormontabile; due mulini furono divelti dalla piena, e il tempo divenne sempre peggiore. Il 30 settembre, fin dal mattino, apparve il sole e, sperando nel sereno, Levin si decise a fare i preparativi per il viaggio. Ordinò d'insaccare il frumento, mandò l'amministratore dal compratore a ritirare il denaro, ed egli stesso andò in giro per la fattoria a dare gli ultimi ordini prima della partenza. 

Dopo aver fatto tutto questo, madido d'acqua che, a rivoli, gli scendeva per la giacca di pelle giù per il collo e nei gambali, ma nella migliore disposizione d'animo, sveglio ed alacre, Levin prese la via del ritorno, a sera. Ma il maltempo, verso sera, si scatenò ancor peggio; la grandine sferzava il cavallo che, tutto bagnato, scoteva le orecchie e la testa, sì che andava di traverso; ma Levin, sotto il cappuccio, si sentiva a suo agio e guardava allegramente intorno a sé, ora i ruscelli torbidi che correvano per le carreggiate, ora le gocce d'acqua che gocciavano giù da ogni ramo nudo, ora il bianco delle piazzuole di grandine minuta che non si era ancora sciolta sulle tavole del ponte, ora il fogliame dell'olmo, carnoso e ancora denso di umori, che s'era ammassato per terra in uno strato spesso, intorno all'albero spoglio. Malgrado la tetraggine della natura circostante, egli si sentiva fortemente eccitato. I discorsi fatti con i contadini in un villaggio lontano gli avevano dimostrato che cominciavano ad abituarsi ai nuovi rapporti. Il vecchio portiere, dal quale s'era fermato per asciugarsi, evidentemente approvava il piano di Levin se, di sua iniziativa, proponeva di entrare in società per l'allevamento del bestiame. 

``Occorre solo andare tenacemente verso il proprio scopo, ed io riuscirò ad attuare la mia idea - pensava Levin. - E c'è una ragione al mio lavoro e alla mia fatica. Questa faccenda non riguarda solo la mia persona, ma qui si tratta del bene generale. Tutta l'economia domestica, la cosa principale, la situazione di tutto il popolo deve cambiare completamente. Invece della povertà, una ricchezza generale, un'agiatezza; invece dell'odio, la concordia e il legame degli interessi. In una parola, una rivoluzione incruenta, ma una profondissima rivoluzione; inizialmente nella piccola cerchia del nostro distretto, poi nel governatorato, poi nella Russia, nel mondo intero. Perché un'idea giusta non può non essere feconda. Sì, questo è uno scopo per cui vale la pena di lavorare. E il fatto che ad aver avuto questa idea sia stato io, Kostja Levin, quello stesso che giunse al ballo in cravatta nera e al quale la Šcerbackaja disse di no, e che, quanto alla propria persona, è così pietoso e insignificante, questo non significa nulla. Io sono sicuro che Franklin si sentiva insignificante allo stesso mio modo, ed aveva, così come faccio io, poca stima di sé, quando considerava se stesso. Questo non significa nulla. E anche lui, probabilmente, avrà avuto la sua Agaf'ja Michajlovna alla quale confidare i suoi piani''. 

In tali pensieri Levin giunse che era già buio a casa. 

L'amministratore, che era andato dal compratore, venne e portò una parte del denaro del frumento. Il contratto col portiere era stato concluso e per istrada l'amministratore aveva saputo che il grano altrove era rimasto nei campi, così che i loro centosessanta covoni non raccolti erano poca cosa in confronto di quello che era rimasto agli altri. 

Dopo aver pranzato, Levin sedette, come al solito, con un libro sulla poltrona, e, leggendo, continuava a pensare al suo viaggio imminente collegato alla sua opera. Ora gli si presentava in modo particolarmente chiaro tutta la sostanza della questione e nella sua mente si formavano interi periodi che esprimevano di per sé l'essenza del suo pensiero. ``Questo bisogna scriverlo - pensò. - Devo scrivere una breve introduzione che prima non ritenevo necessaria''. Si alzò per andare allo scrittoio, e Laska, che era sdraiata ai suoi piedi, stirandosi, si alzò essa pure e lo guardò come a chiedergli dove andasse; ma non c'era tempo per scrivere, perché erano venuti i capoccia a prendere gli ordini, e Levin andò da loro in anticamera. 

Dopo aver impartito gli ordini, cioè le disposizioni per i lavori dell'indomani e dopo aver ricevuto tutti i contadini che avevano da parlargli, Levin andò nello studio e tornò al suo lavoro. Laska si coricò sotto la tavola, Agaf'ja Michajlovna con la calza in mano si mise a sedere al proprio posto. 

- Ma non c'è ragione che vi annoiate - gli disse Agaf'ja Michajlovna. - Perché restate a casa? Se andaste alle acque termali, dal momento che vi siete già preparato? 

- Parto proprio domani, Agaf'ja Michajlovna. Bisogna risolvere la faccenda. 

- Ma che lavoro il vostro! Come se non aveste già ricompensato abbastanza i contadini! E dicono: il vostro signore riceverà per questo una grazia dallo zar. Ed è strano: perché dovete voi angustiarvi tanto per i contadini? 

- Io non mi angustio per loro; lo faccio per me. 

Agaf'ja Michajlovna conosceva tutti i particolari dei piani economici di Levin. Spesso Levin le esponeva in tutti i particolari i suoi pensieri e non di rado discuteva con lei che a volte non conveniva con le sue spiegazioni. Ma ora aveva capito in modo affatto diverso quello che egli le aveva detto. 

- All'anima propria, è certo, bisogna pensarci più di tutto - ella disse con un sospiro. - Ecco, Parfën Denisyc, un analfabeta, è morto così come Iddio lo conceda a ognuno - ella disse, alludendo a un domestico morto da poco. - L'hanno comunicato, gli hanno dato l'olio santo. 

- Non è questo che dico - egli rispose. - Io dico che lo faccio per mio vantaggio. È maggiore il mio vantaggio, se i contadini lavorano meglio. 

- Ma per quanto voi facciate, se uno è pigro cadrà sempre come un sacco. Se la coscienza c'è lavorerà, altrimenti non c'è niente da fare. 

- Su, via, voi stessa dite che Ivan s'è messo d'impegno a custodire il bestiame. 

- Io dico solo - rispose Agaf'ja Michajlovna, evidentemente non a caso, ma con severa coerenza di pensiero - che voi avete bisogno di prender moglie, ecco! 

Il riferimento di Agaf'ja Michajlovna alla stessa cosa cui egli aveva proprio allora pensato, lo amareggiò e lo offese. Levin si accigliò e, senza risponderle, si accinse di nuovo al suo lavoro, ripetendosi tutto quello che aveva pensato sul significato di questo lavoro. Solo ogni tanto tendeva l'occhio allo sferruzzare di Agaf'ja Michajlovna, e, ricordando quello che non voleva ricordare, si accigliava di nuovo. 

Alle nove si sentì un campanellino e il sordo traballare di una carrozza nel fango. 

- Su, ecco che sono arrivati anche gli ospiti; la noia passerà - disse Agaf'ja Michajlovna, alzandosi e dirigendosi verso la porta. Ma Levin la sorpassò. In quel momento il suo lavoro non andava, ed egli era felice e contento di un ospite qualsiasi. 

\capitolo{XXXI}\label{xxxi-2} 

Dopo essere sceso giù fino a mezza scala, Levin sentì nell'ingresso un tossicchiare a lui noto; ma lo sentì poco chiaro a causa del rumore dei propri passi e sperò di essersi sbagliato; poi scorse una figura alta, ossuta, nota, e gli parve, non si poteva ormai più sbagliare, eppure ancora lo sperava, che quell'uomo lungo che si toglieva la pelliccia e tossiva fosse suo fratello Nikolaj. 

Levin voleva bene a suo fratello, ma stare con lui a lungo era un tormento. Ora poi che, per effetto del pensiero che gli era venuto in mente e dell'accenno di Agaf'ja Michajlovna, si trovava in uno stato d'animo oscuro e confuso, l'incontro imminente col fratello gli parve proprio penoso. Invece di un ospite allegro, sano, estraneo, ch'egli sperava lo distraesse dalla sua nebulosità spirituale, doveva rivedere il fratello che gli leggeva dentro da parte a parte, che avrebbe suscitato in lui i pensieri più intimi, che lo avrebbe obbligato a confidarsi in pieno. E questo non voleva. 

Irritatosi contro se stesso per questo senso di repulsione, Levin corse nell'ingresso. Appena vide il fratello, quel senso di intima delusione scomparve immediatamente e si mutò in pena. Per quanto spaventoso fosse anche prima suo fratello Nikolaj per la magrezza e l'aria malandata, in quel momento era ancora più smagrito, ancora più debole. Era uno scheletro ricoperto di pelle. 

Stava dritto nell'ingresso, storcendo il collo lungo e magro e strappandone la sciarpa, e sorrideva in modo strano e penoso. Visto questo sorriso, umile e sottomesso, Levin sentì che un convulso gli afferrava la gola. 

- Ecco, son venuto da te - disse Nikolaj con voce cupa, senza staccare un attimo gli occhi dal viso del fratello. - Da tempo ne avevo voglia, ma non stavo mai bene. Adesso invece mi sono rimesso - diceva, asciugandosi la barba con le grandi palme magre. 

- Sì, sì - rispose Levin. E provò ancor più terrore quando nell'abbraccio sentì con le labbra l'aridità della carne del fratello e vide da vicino i suoi grandi occhi stranamente luccicanti. 

Qualche settimana prima, Levin aveva scritto al fratello che, per la vendita della piccola parte di patrimonio che era rimasta indivisa fra i germani, spettavano a lui, per la sua quota, duemila rubli. 

Nikolaj disse che era venuto per ritirare questo denaro, per stare un po' nel suo nido, per toccare un po' la terra e riprendere forza, come i giganti, per la sua prossima attività. Malgrado l'accentuato incurvamento, malgrado la magrezza che sorprendeva in rapporto alla statura, i suoi movimenti, come al solito, erano agili e a scatti. Levin lo introdusse nello studio. 

Nikolaj cambiò d'abito con particolare cura, cosa che prima non faceva, pettinò i capelli radi e dritti ed entrò, sorridendo, di sopra. 

Era di umore carezzevole e allegro come quando era fanciullo e come spesso lo ricordava Levin. Ricordò perfino Sergej Ivanovic senza rancore. Vista Agaf'ja Michajlovna, scherzò con lei e le chiese notizie dei vecchi servi. La notizia della morte di Parfën Denisyc agì in modo spiacevole su di lui. Sul suo viso si espresse lo spavento, ma si riprese subito. 

- Già, era vecchio ormai - egli disse e cambiò discorso. - Eh, sì, starò un mese, due da te, e poi a Mosca; tu sai, Mjagkov mi ha promesso un posto ed entrerò in servizio. Ora ordinerò la mia vita in maniera del tutto diversa. Lo sai, ho allontanato quella donna. 

- Mar'ja Nikolaevna? come mai, perché? 

- Ah, una donna disgustosa! M'ha dato un mucchio di dispiaceri. - Ma egli non raccontò quali erano stati questi dispiaceri. Non poteva dire che aveva cacciato Mar'ja Nikolaevna perché il tè che faceva era debole e, principalmente, perché ella aveva cura di lui come di una persona malata. - Poi, in genere, adesso voglio cambiar vita del tutto. Io, s'intende, come tutti del resto, ho fatto delle sciocchezze, ma il patrimonio è l'ultima cosa, non lo rimpiango. Purché ci sia la salute, e la salute, grazie a Dio, s'è rimessa. 

Levin ascoltava e cercava e non sapeva trovare cosa dire. Forse Nikolaj s'accorgeva di questo; cominciò a interrogare il fratello sulle sue cose; e Levin era contento di parlare di sé perché poteva parlare senza fingere. Raccontò al fratello i suoi piani e la sua attività. 

Il fratello ascoltava; ma evidentemente non si interessava. 

Questi due uomini erano così consanguinei e così vicini l'uno all'altro che il minimo movimento, il minimo cambiamento di voce diceva per entrambi più di tutto quello che si può dire a parole. 

Ora per entrambi vi era un solo pensiero: la malattia e la prossimità della morte di Nikolaj, e questo pensiero soffocava tutto il resto. Ma né l'uno né l'altro aveva il coraggio di parlarne e perciò, qualunque cosa dicessero, senza esprimere quello che unicamente interessava, aveva un tono falso. Levin non s'era mai sentito così contento come adesso che, finita la serata, si doveva andare a letto. Mai con nessun estraneo, né in nessuna visita ufficiale, era stato così poco naturale, così falso come ora. E la coscienza e il pentimento di questa falsità lo rendevano ancora più insincero. Avrebbe voluto piangere sul suo caro fratello morente, e doveva ascoltare e parlare su come avrebbe vissuto. 

Poiché in casa c'era umido e una sola camera era riscaldata Levin mise il fratello a dormire nella propria camera, di là da un tramezzo. 

Il fratello s'era coricato, dormiva e non dormiva: ma, come un ammalato, si rivoltava, tossiva, e, quando non poteva espettorare, brontolava qualcosa. A volte, mentre respirava con difficoltà, diceva: ``Ah, Dio mio!''. A volte, quando lo spurgo lo soffocava, esclamava con stizza: ``Ah, diavolo!''. Levin a lungo non dormì ascoltandolo. I pensieri di Levin erano i più difformi, ma tutti finivano in una sola cosa: la morte. 

La morte, l'inevitabile fine di tutto, per la prima volta gli si presentava con una violenza ineluttabile, e questa morte che era là in quel fratello caro che gemeva nel sonno e che per abitudine invocava indifferentemente ora Dio ora il diavolo, non era così lontana come gli era sempre parsa. Era anche in lui: lo sentiva. Se non ora domani, se non domani fra trenta anni, non era forse lo stesso? E cosa fosse questa morte inevitabile, egli non solo non lo sapeva, né mai ci aveva neppure pensato, ma non sapeva e non osava pensarci. 

``Io lavoro, voglio fare qualche cosa, ma ho dimenticato che tutto finisce, che c'è la morte''. Stava seduto sul letto nel buio, rannicchiato, stringendo fra le braccia le ginocchia e, trattenendo il respiro per la tensione del pensiero, meditava. Ma quanto più tendeva il pensiero tanto più chiaramente gli appariva che era senza dubbio così, che egli aveva realmente dimenticato, tralasciato una piccola circostanza della vita, che sarebbe cioè venuta la morte e che tutto sarebbe finito, che non valeva la pena d'intraprendere cosa alcuna e che rimediare a questo non si poteva in nessun modo. Sì, era terribile, ma era così. 

``Eppure io sono vivo ancora. E adesso che farò mai, che farò?'' diceva con disperazione. Accese una candela e cautamente si alzò e andò verso lo specchio e cominciò a guardarsi il viso e i capelli. Sì, sulle tempie v'erano dei capelli bianchi. Aprì la bocca. I molari cominciavano a guastarsi. Scoprì le braccia muscolose. Sì, c'era ancora vigore. Ma anche Nikolen'ka, che là respirava coi resti dei suoi polmoni, aveva avuto un corpo sano. E a un tratto ricordò che, bambini, si coricavano insieme e aspettavano solo che Fëdor Bogdanyc uscisse dalla porta, per gettarsi addosso l'un l'altro i guanciali e ridere, ridere irresistibilmente, così che neanche il terrore di Fëdor Bogdanyc riusciva a frenare quella consapevolezza di gioia di vivere che scaturiva e spumeggiava oltre i limiti. ``E ora questo petto vuoto e incurvato\ldots{} e io che non so perché e che cosa mi accadrà\ldots{}''. 

- Ach, ach! Ah, diavolo! Che fai in giro? non dormi? - lo chiamò la voce del fratello. 

- Ma non so, l'insonnia. 

- E io dormivo bene, adesso non sono più sudato. Guarda, tocca la camicia. Niente sudore? 

Levin palpò, andò di là dal tramezzo, spense la candela, ma ancora per molto tempo non dormì. Gli si era appena chiarita la questione di come vivere, che gli era apparsa un'altra insolubile questione: la morte. 

``Eh, sì, egli muore, sì, morirà in primavera. Come venirgli in aiuto? Che posso dirgli? Che cosa ne so io? Avevo perfino dimenticato che questa cosa esistesse''. 

\capitolo{XXXII}\label{xxxii-2} 

Levin aveva già da tempo osservato che, quando tra persone ci si sente a disagio per eccessiva cedevolezza e sottomissione, allora ben presto nasce qualcosa d'insopportabile per eccessiva pretesa e cavillosità. Sentiva che questo sarebbe accaduto anche col fratello. E invero, la mansuetudine del fratello Nikolaj non durò a lungo. Fin dalla mattina dopo divenne irritabile, e molestò con insistenza il fratello, toccandolo nei punti più dolorosi per lui. 

Levin si sentiva colpevole e non poteva rimediarvi. Sentiva che se tutti e due non avessero finto, e avessero invece parlato come si dice, a cuore aperto, dicendo solo quello che sentivano, allora si sarebbero solo guardati negli occhi l'un l'altro e Konstantin avrebbe detto soltanto: ``morirai, morirai, morirai'' e Nikolaj avrebbe risposto: ``lo so che morirò, e ho paura, paura, paura!''. E niente più avrebbero detto se avessero parlato a cuore aperto. Ma così non si poteva vivere, perché Konstantin si provava a fare quello che per tutta la vita aveva tentato e non aveva saputo mai fare, quello cioè che, secondo lui, molti sapevano fare tanto bene e di cui non si può fare a meno nella vita: si provava a dire quello che non pensava; e continuamente sentiva che ciò gli riusciva falso, che il fratello se ne accorgeva e se ne irritava. 

Due giorni dopo Nikolaj invitò il fratello a esporgli di nuovo il suo piano e prese non solo a giudicarlo, ma si mise di proposito a confonderlo col comunismo. 

- Tu hai preso soltanto un'idea non tua, poi l'hai fatta diventar mostruosa e vuoi applicarla all'inapplicabile. 

- Ma io ti dico che questo non ha nulla di comune. Loro, negano il diritto di proprietà privata, il capitale, l'ereditarietà, mentre io, senza negare questo stimolo principale - Levin era contrariato con se stesso di usare questi termini, ma da che s'era appassionato al suo lavoro, aveva cominciato involontariamente a usare sempre più spesso termini non russi - voglio solo regolare il lavoro. 

- È proprio così, hai preso un pensiero non tuo, ne hai tolto via tutto quello che ne costituisce la forza e vuoi far credere che è qualcosa di nuovo - disse Nikolaj, torcendo irritato il collo dentro la cravatta. 

- Ma la mia idea non ha nulla in comune\ldots{} 

- Là - diceva Nikolaj con gli occhi scintillanti di cattiveria e sorridendo ironicamente - là almeno vi è il fascino, per così dire, geometrico della chiarezza, della certezza. Può darsi che sia un'utopia. Ma ammettiamo che di tutto il passato si possa fare tabula rasa: non c'è proprietà, non c'è famiglia, e allora anche il lavoro si organizza. Ma da te non c'è nulla\ldots{} 

- Perché confondi? Io non sono mai stato comunista. 

- E io lo sono stato, e trovo che sia una cosa prematura, ma ragionevole e che avrà un avvenire come il cristianesimo dei primi secoli. 

- Io ritengo solo che la forza lavoratrice debba essere considerata da un punto di vista naturale, debba cioè essere studiata, riconosciuta nelle sue peculiarità e\ldots{} 

- Ma questo è completamente inutile. Questa forza trova da sé, secondo il suo grado di sviluppo, un certo modo di estrinsecarsi. Dappertutto ci sono stati gli schiavi, poi i metayers; e da noi c'è il lavoro a mezzadria, c'è il fitto, il lavoro del bracciante, che cosa cerchi di più? 

Levin d'un tratto s'accalorò a queste parole, perché in fondo all'anima temeva che ciò fosse vero, vero il fatto ch'egli in fondo volesse tenersi in equilibrio tra il comunismo e le forme passate, e che questo fosse difficilmente possibile. 

- Io cerco mezzi per lavorare produttivamente e per me e per i lavoratori. Voglio organizzare\ldots{} - rispose con calore. 

- Tu non vuoi organizzare niente; come hai sempre fatto in tutta la tua vita, hai soltanto voglia di fare l'originale, di mostrare che non sfrutti così semplicemente i contadini, ma che lo fai per un'idea. 

- Già, tu così pensi\ldots{} e basta! - rispose Levin, sentendo che il muscolo della sua guancia sinistra guizzava irresistibilmente. 

- Tu non avevi e non hai convinzioni, ma vuoi solo soddisfare il tuo amor proprio. 

- Sta benissimo; ma lasciami stare! 

- Ti lascerò! ed è un pezzo che ne era ora, e va' al diavolo! E mi spiace molto d'esser venuto. 

Per quanto poi Levin cercasse di calmare il fratello, Nikolaj non volle sentir nulla; diceva che era molto meglio separarsi, e Konstantin vedeva che al fratello era semplicemente venuta a noia la vita. 

Nikolaj era già del tutto pronto a partire, quando Konstantin di nuovo andò da lui e gli chiese, senza spontaneità, di scusarlo se in qualche modo l'aveva offeso. 

- Ah, che magnanimità! - disse Nikolaj, e sorrise. - Se vuoi aver ragione, posso farti questo favore. Hai ragione, ma tuttavia io parto! 

Proprio solo al momento di partire, Nikolaj scambiò l'abbraccio e disse a un tratto, guardando stranamente serio il fratello: 

- Tuttavia non serbarmi rancore, Kostja! - e la voce gli tremò. 

Furono queste le uniche parole che furono dette sinceramente. Levin capì che sotto queste parole si sottointendeva: ``tu vedi e sai che sto male e che forse non ci vedremo più''. Levin intese ciò, e le lacrime gli sgorgarono dagli occhi. Baciò ancora una volta il fratello, ma non poté e non seppe dirgli nulla. 

Due giorni dopo la partenza del fratello, anche Levin andò all'estero. Incontratosi alla stazione ferroviaria con Šcerbackij, il cugino di Kitty, questi fu colpito dalla tristezza di Levin. 

- Cosa c'è? - gli chiese Šcerbackij . 

- Ma, niente; al mondo c'è poco da stare allegri. 

- Come poco? ecco, venite con me a Parigi, invece di andare a quel non so che Mühlhausen. Vedrete come si sta allegri. 

- No, per me è finita. È tempo di morire. 

- Ecco un bello scherzo! - disse ridendo Šcerbackij. - Io mi sono appena preparato a cominciare. 

- Già, anch'io pensavo così fino a poco fa, ma ora so che morirò presto. 

Levin diceva quello che realmente pensava negli ultimi tempi. In tutto vedeva soltanto la morte o l'avvicinarsi di essa. Ma l'opera da lui intrapresa l'interessava ancora di più. Era pur necessario concludere in qualche modo la vita, finché non fosse giunta la morte. L'oscurità per lui copriva tutto; ma proprio a causa di questa oscurità sentiva che l'unico filo conduttore era la sua opera, e con le ultime forze si aggrappava ad essa e vi si teneva stretto. 

\parte{PARTE QUARTA}\label{parte-quarta} 

\capitolo{I}\label{i-3} 

I Karenin, marito e moglie, continuavano a vivere nella stessa casa, s'incontravano ogni giorno, ma erano completamente estranei l'uno all'altra. Aleksej Aleksandrovic si era imposto la regola di vedere ogni giorno sua moglie, perché la servitù non avesse il diritto di sospettare, ma evitava di pranzare a casa. Vronskij non andava mai in casa di Aleksej Aleksandrovic, ma Anna lo vedeva fuori e il marito lo sapeva. 

La situazione era tormentosa per tutti e tre, e nessuno di loro sarebbe stato in grado di protrarla neppure di un giorno se non ne avesse atteso il mutamento e se non avesse avuto la convinzione che si trattava di una dolorosa difficoltà del momento che sarebbe passata. Aleksej Aleksandrovic aspettava che questo amore passasse così come passavano tutte le cose, così che tutti avrebbero poi dimenticato e il suo nome non sarebbe rimasto disonorato. Anna, dalla quale dipendeva la situazione, e che più di tutti ne era tormentata, la sopportava, perché non soltanto sperava, ma era fermamente convinta che tutto questo si sarebbe presto risolto e chiarito. Davvero non sapeva che cosa avrebbe potuto risolvere la situazione, ma era fermamente convinta che sarebbe avvenuto ora, molto presto. Vronskij, che senza volere si sottometteva a lei, aspettava anche lui qualche cosa di indipendente dal proprio volere che chiarisse tutte queste difficoltà. 

A metà dell'inverno Vronskij trascorse una settimana molto noiosa. Era stato addetto alla persona di un principe straniero venuto a Pietroburgo, per fargli conoscere le cose più notevoli della città. Vronskij era indubbiamente rappresentativo; inoltre aveva l'arte di essere deferente con dignità e aveva l'abitudine di trattare con persone di tale rango; perciò era stato addetto al principe. Ma l'incarico gli parve molto pesante. Il principe non voleva tralasciare di veder nulla di quanto, tornato a casa, gli avrebbero chiesto se avesse visto in Russia; e per quel che riguardava se stesso, voleva godersi quanto più possibile gli svaghi russi. Vronskij era obbligato a condurlo di qua e di là. La mattina andavano a visitare le cose notevoli, la sera partecipavano ai divertimenti nazionali. Il principe godeva di una salute non comune anche fra i principi, e con la ginnastica e le cure continue del corpo aveva acquistato un tale vigore che, malgrado gli eccessi a cui si abbandonava nei bagordi, era sempre fresco come un grosso cetriolo verde di qualità olandese. Aveva viaggiato molto e trovava che uno dei maggiori vantaggi dell'attuale facilità di comunicazioni consisteva nella possibilità di accedere agli svaghi di ogni paese. Era stato in Spagna, e là aveva fatto serenate e stretto amicizia con una spagnola che sonava il mandolino. In Svizzera aveva ucciso una Gemse. In Inghilterra aveva saltato a cavallo gli ostacoli in frac rosso, e aveva ucciso, per scommessa, duecento fagiani. in Turchia era stato in un harem, in India aveva viaggiato su di un elefante, e ora in Russia voleva assaggiare tutti gli svaghi peculiarmente russi. 

A Vronskij, che gli era accanto come un maestro di cerimonie, costava fatica distribuire tutti i divertimenti offerti al principe da varie personalità; c'erano, infatti, i cavalli trottatori, i bliny, la caccia all'orso, le trojke e gli zigani e le baldorie russe con la rottura delle stoviglie. E il principe con straordinaria facilità aveva fatto sua l'anima russa, fracassava vassoi e stoviglie, faceva seder sulle ginocchia una zigana e pareva chiedesse: ``Dunque, che altro c'è? possibile che solo in questo consista l'anima russa?''. 

In sostanza, a tutti gli svaghi russi il principe preferiva le attrici francesi, una ballerina dei balletti e lo champagne di marca bianca. Vronskij aveva le stesse abitudini del principe; ma o perché egli stesso era cambiato negli ultimi tempi, o per la troppo grande dimestichezza acquistata col principe, quella settimana gli parve terribilmente faticosa. Durante tutto il tempo, provò continuamente una sensazione simile a quella che prova un uomo addetto a un pazzo pericoloso che abbia paura del pazzo e che tema, nello stesso tempo, stando vicino a lui, di uscir di cervello. Vronskij sentiva ogni momento la necessità di non attenuare neppure per un attimo il tono di severa deferenza ufficiale verso il principe, per non essere da lui offeso. La maniera del principe di trattare con quelle stesse persone che, con stupore di Vronskij, non stavan più nella pelle dalla voglia di offrirgli divertimenti russi, era una maniera sprezzante. I suoi giudizi sulle donne russe che aveva desiderato conoscere avevano più di una volta costretto Vronskij ad arrossire d'indignazione. Ma la ragione principale per la quale il principe riusciva fastidioso a Vronskij, era che in lui vedeva riprodotto se stesso. E il fatto di guardare in quello specchio non lusingava il suo amor proprio. Era infatti un uomo molto fatuo e molto presuntuoso, molto sano e molto pulito, e null'altro. Era un gentleman, questo sì, era vero, e Vronskij non poteva negarlo. Era costante e dignitoso verso i superiori, schietto e semplice con i suoi pari e sprezzantemente buono con gli inferiori. Anche Vronskij si comportava così e riteneva tale comportamento una qualità; ma nei rapporti col principe egli era un inferiore e l'atteggiamento sprezzante e benevolo verso di lui lo indignava. 

``Che stupido bue! Possibile che anch'io sia fatto così?'' pensava. 

Così, quando dopo sei giorni si congedò da lui, prima che egli partisse per Mosca, e ne ricevette i ringraziamenti, fu felice d'essersi liberato da quella posizione di disagio e da quello specchio sgradevole. Lo salutò alla stazione, di ritorno dalla caccia all'orso dove, per tutta la notte, avevano assistito all'esibizione della bravura russa. 

\capitolo{II}\label{ii-3} 

Tornato a casa Vronskij trovò un biglietto di Anna. Diceva: ``Sono malata ed infelice. Non posso uscire in vettura, ma non posso resistere ancora senza vedervi. Venite di sera. Alle sette Aleksej Aleksandrovic va al consiglio e vi rimane fino alle dieci''. Dopo aver riflettuto un attimo sul fatto strano ch'ella, malgrado l'ingiunzione del marito di non riceverlo, lo invitasse a casa sua, decise di andare. 

In quell'inverno Vronskij era stato promosso colonnello, era uscito dal reggimento e viveva solo. Dopo aver fatto colazione, si sdraiò subito su di un divano, e in pochi minuti i ricordi delle disgustose scene della sua vita di quegli ultimi giorni, si confusero e si collegarono con l'immagine di Anna e di un certo contadino, esattore delle imposte, che aveva avuto una parte importante nella caccia all'orso, ed egli si addormentò. Si svegliò che era buio, tremando di terrore, e accese subito una candela: ``Che cos'è, che cosa ho visto di terribile nel sogno? Sì, sì, quel contadino esattore delle imposte, mi pare, piccolo, sudicio, con la barba arruffata, curvo, faceva qualcosa, e, a un tratto, s'è messo a dire certe strane parole in francese. Già, non c'era nient'altro che questo, nel sogno - si disse. - Ma perché era così spaventoso?''. Ricordò di nuovo con chiarezza il contadino e le incomprensibili parole francesi che quello pronunciava, e il terrore gli corse come un brivido per la schiena. 

``Che sciocchezza!'' pensò Vronskij, e guardò l'ora. Erano già le otto e mezzo. Chiamò il servo, si vestì in fretta e uscì sulla scala, del tutto dimentico del sogno e tormentato solo dal fatto di essere in ritardo. Avvicinandosi all'ingresso dei Karenin, guardò l'orologio e vide che erano le nove meno dieci. Una carrozza alta, stretta, alla quale era attaccata una coppia di cavalli grigi, stava ferma davanti all'ingresso. Riconobbe la carrozza di Anna. ``Viene lei da me - pensò Vronskij - e sarebbe meglio. Mi spiace entrare in questa casa. Ma è lo stesso, ormai non posso nascondermi'' si disse e, con quel modo di fare, sin dall'infanzia tutto suo, di chi sa di non aver da vergognarsi, uscì dalla slitta e si avvicinò alla porta. La porta si aprì e il portiere con uno scialle da viaggio in mano chiamò la carrozza. Vronskij, pur non abituato a notare i particolari, in quel momento osservò l'espressione di sorpresa con cui il portiere lo guardò. Proprio sulla porta, Vronskij quasi si scontrò con Aleksej Aleksandrovic. Un lume a gas illuminava in pieno il suo viso esangue, smagrito sotto il cappello nero e la cravatta bianca che spiccava fra il castoro del cappotto. Gli occhi immobili, appannati di Karenin guardarono in faccia Vronskij. Vronskij s'inchinò e Aleksej Aleksandrovic, dopo aver masticato un po' fra i denti, portò la mano al cappello e passò. Vronskij vide ch'egli, senza più guardare, sedeva nella vettura, prendeva attraverso il finestrino lo scialle e il binocolo e si rannicchiava dentro. Vronskij entrò in anticamera. Aveva le sopracciglia aggrottate e negli occhi brillava una scintilla cattiva e sprezzante. 

``Ecco che situazione! - pensò. - Se lottasse, se difendesse il suo onore, io potrei agire, dare sfogo ai miei sentimenti, ma questa debolezza o vigliaccheria\ldots{} mi fa apparire un traditore, e io non volevo e non voglio esserlo''. 

Dal tempo del suo colloquio con Anna nel giardino della Vrede, le idee di Vronskij erano molto cambiate. Egli, sottomettendosi involontariamente alla debolezza di Anna che gli si dava tutta e aspettava da lui la decisione del proprio destino, sottomettendosi fin dal principio a tutto, non pensava più da tempo che quel legame potesse finire, come aveva pensato in un primo momento. I suoi piani di ambizione si erano di nuovo ritirati in buon ordine e, sentendo di essere uscito ormai da quel tipo di attività in cui ogni cosa è ben definita, si era abbandonato al sentimento, e questo sentimento lo avvinceva sempre e sempre più forte a lei. 

Mentre era ancora nell'ingresso udì i passi di lei che si allontanavano. Capì che lo aspettava, che tendeva l'orecchio e che proprio allora era tornata nel salotto. 

- No - gridò vedendolo, e al primo suono della propria voce le vennero le lacrime agli occhi - no, se continuerò così, questo accadrà ancora molto, molto prima! 

- Che cosa, amica mia? 

- Che cosa? Io ti aspetto, mi tormento, un'ora, due. No, basta! Io non posso arrabbiarmi con te. Forse non potevi. No, non lo farò! 

Poggiò tutte e due le mani sulle spalle di lui e guardò a lungo, con uno sguardo profondo, entusiastico e nello stesso tempo indagatore, il viso di lui. Lo andava indagando come se volesse guardarlo per tutto il tempo che non lo aveva visto. Come sempre ad ogni incontro, fondava l'immaginaria figurazione di lui (incomparabilmente migliore, impossibile nella realtà), con lui, così com'era. 

\capitolo{III}\label{iii-3} 

- L'hai incontrato? - chiese quando furono seduti presso la tavola sotto la lampada. - Ecco la punizione per essere venuto in ritardo! 

- Già, ma come mai? Non doveva essere al consiglio? 

- C'è stato, ed è tornato per andare non so dove. Ma questo è nulla. Non ne parlare. Dove sei stato? Sempre col principe? 

Ella conosceva tutti i particolari della vita di lui. Egli voleva dire che non aveva dormito la notte e che perciò aveva preso sonno, ma guardando il viso agitato e felice di lei, ebbe rimorso. E disse che era dovuto andare a riferire sulla partenza del principe. 

- Ma ora è finito. Tu non crederai come ciò mi sia stato insopportabile. 

- Perché poi? Dopo tutto, è questa la vita che fate sempre voi uomini scapoli - ella disse, aggrottando le sopracciglia e, messasi al suo lavoro a maglia che era sulla tavola, cominciò a liberarne l'uncinetto, senza guardare Vronskij. 

- Io già da tempo l'ho abbandonata questa vita - egli disse, meravigliandosi del cambiamento di espressione del viso di lei, e cercando di penetrarne il senso. - E confesso - disse, mostrando con un sorriso i suoi denti bianchi e regolari - che, osservando questa vita in questa settimana, mi sono visto come in uno specchio, e me ne è venuta una sensazione sgradevole. 

Ella teneva in mano il lavoro a maglia, non lavorava e guardava lui con uno sguardo strano, luminoso e ostile. 

- Stamane Liza è venuta da me - saltò su a dire - quelli là ancora non hanno paura di venire da me, malgrado la contessa Lidija Ivanovna: e mi ha raccontato della vostra serata ateniese. Che orrore! 

- Io volevo dire solo che\ldots{} 

Ella lo interruppe. 

- Era quella Thérèse che conoscevi prima? 

- Volevo dire\ldots{} 

- Come siete disgustosi, voi uomini! Come mai non riuscite a immaginare che una donna questo non lo può dimenticare? - disse, riscaldandosi sempre più e rivelando così la ragione della sua irritazione. - Specialmente una donna che non conosce la tua vita. Che ne so io? che cosa potevo saperne? - ella diceva - quello che tu mi avresti detto. E come posso sapere che hai detto la verità? 

- Anna! Tu mi offendi. Non mi credi forse? Forse non ti ho detto che non c'è un pensiero in me che io non ti abbia rivelato? 

- Sì, sì - ella disse, cercando evidentemente di scacciare i pensieri di gelosia. - Ma se sapessi come è penoso per me! Io credo, sì, ti credo\ldots{} Così, che dicevi? 

Ma egli non poté ricordare quello che voleva dire. Questi accessi di gelosia, che negli ultimi tempi la prendevano sempre più spesso, lo atterrivano; e benché cercasse di nasconderlo, lo raffreddavano verso di lei, anche se riconosceva che la causa della gelosia era il suo amore per lui. Quante volte s'era detto che l'amore di lei era la felicità! ed ecco, ella lo amava come può amare una donna che per l'amore abbia distrutto tutti gli altri beni di questa vita, ma egli era molto più lontano dalla felicità ora, che quando era partito da Mosca per seguire lei. Allora egli si considerava infelice, ma la felicità era una cosa da raggiungere; ora, invece, sentiva che la felicità più piena era già nel passato. Ella già non era più quale egli l'aveva veduta nei primi tempi. E moralmente e fisicamente era peggiorata. Si era dilatata, e, nel momento in cui aveva accennato all'attrice, un'espressione cattiva era passata sul suo viso, alterandolo. Egli la guardava ora come un uomo guarda un fiore da lui colto e già appassito, nel quale a stento riconosce la bellezza che lo ha spinto a coglierlo e a distruggerlo. Malgrado questo, sentiva che allora, quando l'amore era più violento, egli avrebbe potuto, se l'avesse voluto fermamente, strapparlo quell'amore dal proprio cuore, ma adesso, quando, come in questo momento, gli pareva di non sentire più amore per lei, egli avvertiva che questo legame non poteva più essere spezzato. 

- Su, via, cosa mi volevi dire del principe? L'ho scacciato, l'ho scacciato ``il diavolo'' - aggiunse. Chiamavano fra di loro ``il diavolo'' la gelosia. - Già, cosa avevi cominciato a dire del principe? Perché per te è stato così faticoso? 

- Oh, insopportabile - egli disse, cercando di afferrare il filo del pensiero smarrito. - Non ci guadagna, a conoscerlo da vicino. Per definirlo, è un animale ottimamente nutrito, di quelli che alle esposizioni vincono i primi premi e niente più - disse con un dispetto che interessò lei. 

- No, come mai? - ribatté. - Però ha visto molte cose, è una persona colta. 

- È tutta un'altra cultura la loro. Serve solo per poterla disprezzare la cultura, così come disprezzano tutto, tranne i piaceri animaleschi. 

- Sì, però voi tutti li amate questi piaceri! - ella disse, e di nuovo egli notò quello sguardo torvo che lo sfuggiva. 

- Come mai lo difendi tanto? - egli disse, sorridendo. 

- Non lo difendo, mi è del tutto indifferente; ma penso che se a te stesso non fossero più piaciuti questi svaghi, avresti rifiutato di fargli compagnia. Ma a te piace guardare Teresa in costume d'Eva\ldots{} 

- Daccapo, daccapo ``il diavolo'' - disse Vronskij, prendendole la mano ch'ella aveva poggiato sulla tavola e baciandola. 

- Già, ma io non posso. Tu non sai come io mi sia tormentata, aspettandoti! Io penso di non essere gelosa. Non sono gelosa. Ti credo quando sei qui con me; ma quando tu sei chi sa dove, solo, e conduci quella tua vita per me incomprensibile\ldots{} 

Ella si scostò da lui, liberò finalmente l'uncinetto dal lavoro, e in fretta, con l'aiuto dell'indice, cominciarono a sovrapporsi una dopo l'altra le maglie di lana bianca che risplendevano sotto la luce della lampada; e, in fretta, il polso sottile nel polsino ricamato prese a girare nervosamente. 

- Dunque, dove l'hai incontrato Aleksej Aleksandrovic? - risonò a un tratto, senza naturalezza, la voce di lei. 

- Ci siamo scontrati sulla porta. 

- E lui ti ha salutato così? 

Ella allungò il viso e, socchiusi gli occhi, cambiò espressione, piegò le braccia, e sul suo bel volto Vronskij vide ad un tratto la stessa espressione con la quale Aleksej Aleksandrovic l'aveva salutato. Sorrise, ed ella rispose gaia con quel suo simpatico riso che era uno dei suoi incanti maggiori. 

- Non ci capisco nulla - disse Vronskij. - Se, dopo la tua rivelazione in campagna, l'avesse rotta con te, se mi avesse sfidato a duello\ldots{} ma questo non capisco: come può sopportare una simile posizione? Soffre, evidentemente. 

- Lui? - disse lei con un sorriso. - È completamente soddisfatto. 

- Perché ci tormentiamo, quando tutto potrebbe andar tanto bene? 

- Solo lui no. Che forse non lo conosco io, non so che è tutto impastato di menzogna? Si può forse, sentendo qualcosa, vivere come vive lui con me? Non capisce, non sente niente. Può forse un uomo che sente qualcosa, vivere nella stessa casa con la moglie colpevole? Può forse parlarle? Darle del tu? 

E di nuovo involontariamente gli rifaceva il verso: ``Tu, ma chère, tu, Anna!''. 

- Non è un uomo, non è un uomo, è un fantoccio. Nessuno sa quello che so io. Oh, al suo posto, da molto tempo avrei ucciso, avrei fatto a pezzi questa moglie come me, e non le direi, ``tu ma chère, Anna''. Non è un uomo, è una macchina ministeriale. Non capisce che io sono tua moglie, che lui è un estraneo, una cosa superflua. Non ne parliamo, non ne parliamo più\ldots{} 

- Tu hai torto, hai torto, amica mia - disse Vronskij, cercando di calmarla. - Ma fa lo stesso, non parliamo più di lui. Dimmi, cosa hai fatto? Come stai? Cos'è questo malessere e cosa ha detto il dottore? 

Ella lo guardava con una gioia irridente. Aveva trovato, si vedeva, altri lati ridicoli e deformi nel marito e aspettava il momento per rifarli. Ma egli continuò: 

- Io credo che non sia un malessere, ma che tutto dipende dal tuo stato. A quando? 

Lo scintillio irridente si spense negli occhi di lei, ma un altro sorriso, in cui c'era qualcosa di sconosciuto per lui e una calma tristezza, tramutò l'espressione di prima. 

- Preso, presto. Tu dicevi che la nostra posizione è tormentosa, che bisogna romperla. Se sapessi come mi è penosa, che cosa non darei per amarti liberamente e coraggiosamente! Non mi tormenterei e non ti tormenterei con la mia gelosia\ldots{} Sarà presto, ma non tanto come pensiamo. 

E al pensiero di come si sarebbe svolta la cosa, ella parve avere tanta pietà di se stessa che le lacrime le vennero agli occhi e non poté continuare. Posò la mano, che brillava sotto la lampada per gli anelli e la bianchezza, sulla manica della sua giacca. 

- Non sarà così come noi pensiamo. Io non volevo dirtelo, ma tu mi ci hai costretta. Presto, presto, tutto si risolverà, e noi tutti ci calmeremo e non ci tormenteremo più. 

- Non capisco - disse lui, pur comprendendola. 

- Tu mi domandavi quando? Presto. E io non sopravviverò. Non m'interrompere! - e si affrettò a parlare. - Lo so, lo so con certezza. Morirò e sono contenta di morire e di liberare me stessa e voi. 

Le lacrime le sgorgarono dagli occhi; egli si chinò sulla mano di lei e cominciò a baciarla, cercando di nascondere la propria agitazione che, egli sapeva, non aveva nessun fondamento, ma che non riusciva a dominare. 

- Ecco, così va meglio - ella diceva, stringendogli forte la mano. - Ecco la sola, la sola cosa che ci è restata. 

Egli tornò in sé e chinò il capo. 

- Che sciocchezza! Che assurda sciocchezza stai dicendo! 

- No, è vero. 

- Che cosa è vero? 

- Che morirò. Ho fatto un sogno. 

- Un sogno? - ripeté Vronskij, e in un baleno ricordò il contadino esattore del proprio sogno. 

- Già, un sogno - diceva lei. - Da tempo ho spesso questo sogno. Mi vedo correre in camera; devo prendere qualcosa, devo sapere qualcosa; sai, come succede in sogno - diceva lei, dilatando gli occhi pieni di terrore - e nella camera in un angolo c'è qualcosa. 

- Che sciocchezza, come puoi credere\ldots{} 

Ma lei non si lasciò interrompere. Quello che diceva era troppo importante per lei. 

- Ecco, questo qualcosa si volta e io vedo che è un contadino con la barba arruffata, piccolo che fa paura. Voglio fuggire, ma lui, ecco, si china sopra un sacco e con le mani ci rimesta dentro. - E rifaceva il gesto di quest'uomo che rimestava nel sacco. Il terrore era sul suo viso. E Vronskij, ricordando il proprio sogno, sentiva lo stesso terrore che riempiva l'animo di lei. - Rimesta e aggiunge in francese presto presto e, sai, biascica: ``Il faut battre le fer, le broyer, le pétrir\ldots{}''. E io dalla paura volevo svegliarmi, mi sono svegliata\ldots{} ma mi sono svegliata nel sogno. E ho cominciato a chiedermi che cosa significasse tutto questo. E Kornej mi dice: ``di parto, di parto morirete, di parto, padrona mia\ldots{}''. E mi sono svegliata. 

- Che sciocchezza, che sciocchezza! - diceva Vronskij, ma egli stesso sentiva che non c'era convinzione nella propria voce. 

- Non ne parliamo più. Suona, ordinerò di portare il tè. Ma aspetta, ora non c'è molto, io\ldots{} 

Ma improvvisamente tacque. L'espressione del suo viso si era istantaneamente cambiata. Il terrore e l'agitazione si erano mutati in una espressione di calma, pacata e felice attenzione. Egli non poté capire il segreto di questo mutamento. Ella aveva sentito agitarlesi in seno la nuova vita. 

\capitolo{IV}\label{iv-3} 

Aleksej Aleksandrovic, dopo l'incontro con Vronskij sulla scala di casa sua, era andato, come aveva stabilito, all'opera italiana. Ci era rimasto per due atti e aveva visto tutti quelli che aveva interesse di vedere. Tornato a casa, esaminò attentamente l'attaccapanni, e, notato che il cappotto militare non c'era più, passò, secondo il solito, in camera sua. Ma, contrariamente al solito, non si mise a letto, passeggiò avanti e indietro per lo studio fino alle tre di notte. Un senso di rancore verso la moglie, che non aveva voluto rispettare le convenienze e adempiere l'unica condizione impostale: non ricevere in casa sua l'amante, non gli dava pace. Ella non aveva adempiuto la sua richiesta, ed egli doveva punirla e dar corso alla sua minaccia: chiedere il divorzio e toglierle il figlio. Egli conosceva tutte le difficoltà collegate a questa faccenda, ma aveva detto che lo avrebbe fatto ed ora doveva mettere in atto la minaccia. La contessa Lidija Ivanovna aveva accennato che questa era la migliore delle soluzioni per il suo caso e, negli ultimi tempi, la pratica dei divorzi era giunta a un punto di perfezione che Aleksej Aleksandrovic vedeva la possibilità di superare le difficoltà formali. Inoltre, una disgrazia non viene mai sola, e l'affare della sistemazione degli allogeni e quello dell'irrigazione dei campi del governatorato di Zarajsk avevano procurato ad Aleksej Aleksandrovic tali dispiaceri burocratici che in tutto quest'ultimo tempo egli era stato in uno stato di estrema agitazione. Non aveva dormito tutta la notte e la sua irritazione, con un enorme crescendo, era giunta al mattino agli estremi limiti. Si vestì in fretta e, come se portasse in mano una coppa piena di fiele e temesse di versarla e di perdere insieme col fiele l'energia necessaria per la spiegazione con la moglie, andò da lei non appena seppe che si era alzata. 

Anna, che credeva di conoscere bene suo marito, fu sorpresa dal suo aspetto quando egli entrò. La fronte era corrugata, gli occhi guardavano torvi in avanti, evitando lo sguardo di lei; la bocca era come sigillata dalla durezza e dallo sprezzo. Nell'andatura, nei gesti, nel tono della voce, c'erano una decisione e una fermezza, quali la moglie non aveva mai visto in lui. Entrò nella stanza e, senza salutarla, si diresse diritto verso lo scrittoio e, afferrate le chiavi, aprì il cassetto. 

- Che vi occorre? - gridò lei. 

- Le lettere del vostro amante - egli disse. 

- Non sono qui - ella disse, chiudendo il cassetto, ma da questo gesto egli capì che aveva indovinato giusto e, respinta rudemente la mano di lei, afferrò rapido il portafogli nel quale sapeva ch'ella metteva le carte più interessanti. Ella fece per strapparglielo di mano, ma egli la respinse. 

- Sedete, devo parlarvi - disse, mettendosi il portafogli sotto il braccio e stringendolo così forte col gomito da sollevar la spalla. 

Ella lo guardava sorpresa e spaventata. 

- Vi ho detto che non vi avrei mai permesso di ricevere il vostro amante in casa mia. 

- Avevo bisogno di vederlo per\ldots{} 

Si fermò non trovando nessuna giustificazione. 

- Non entro nei particolari del perché una donna debba veder l'amante. 

- Io volevo, io soltanto\ldots{} - disse, avvampando. La sua villania l'aveva irritata e le aveva dato coraggio. - Possibile che non sentiate come vi sia facile offendermi? - ella disse. 

- Si può offendere un uomo onesto e una donna onesta, ma dire a un ladro che è un ladro è solo la constatation d'un fait. 

- Questo nuovo tratto di crudeltà non mi era ancora noto in voi. 

- Voi chiamate crudeltà il fatto che un marito lasci la libertà alla propria moglie, dandole l'onesto asilo del suo nome sotto la sola condizione di salvare le convenienze? Questa è crudeltà? 

- È peggiore della crudeltà, questa è vigliaccheria, se volete saperlo - gridò Anna in uno scoppio di rabbia e, alzatasi, fece per andar via. 

- No - gridò lui con la sua voce stridula di un tono più alto del normale e, afferratala con le sue dita grosse per un braccio in maniera così forte da farle restare impressi i segni rossi del bracciale che vi aveva premuto, la costrinse a sedere al suo posto. - Vigliaccheria? Se la volete usare questa parola, vigliaccheria è questo: abbandonare il marito e il figlio per un amante, e mangiare il pane del marito! 

Ella chinò il capo. Non solo non disse quello che aveva detto all'amante, che era lui suo marito e che il marito era un intruso: non lo pensò neppure. Sentiva tutta la verità che era nelle parole di lui e disse solo, sottovoce: 

- Voi non potevate definire la mia situazione in modo peggiore di quello che io non la senta; ma perché dite tutto ciò? 

- Perché lo dico? perché? - continuò, sempre irritato, lui. - Perché lo sappiate: poiché non avete adempiuto la mia volontà nell'osservare le convenienze, io agirò in modo che questa situazione finisca. 

- Presto, presto finirà anche se dura così - ella disse e di nuovo le lacrime, al pensiero della morte vicina, ora desiderata, le vennero agli occhi. 

- Finirà più presto di quanto non immaginiate, con il vostro amante! Voi avete bisogno di soddisfare una passione carnale\ldots{} 

- Aleksej Aleksandrovic! Io non dico che questo sia poco generoso, ma è disonesto percuotere chi è a terra. 

- Già! è solo a voi che pensate! Ma le sofferenze di un uomo che è stato vostro marito non vi interessano. Per voi è indifferente che tutta la sua vita sia crollata, che egli abbia sop\ldots{} sop\ldots{} sopperto. 

Aleksej Aleksandrovic parlava così in fretta che s'era impappinato e non riusciva in alcun modo a pronunciare la parola ``sofferto''. Aveva finito col pronunciare ``sopperto''. A lei venne da ridere, ma immediatamente ebbe vergogna che qualcosa potesse eccitarle il riso in un momento simile. E per la prima volta, per un attimo, si trasferì in lui, soffrì il suo dolore, e ne ebbe pena. Ma cosa mai poteva dire o fare? Abbassò la testa e tacque. Egli pure tacque per un po' di tempo e poi cominciò a parlare con una voce già meno stridula, con una voce fredda, sottolineando le parole scelte a caso, che non avevano nessuna particolare importanza. 

- Sono venuto a dirvi\ldots{} - egli disse. 

Ella lo guardò. ``No, mi è parso - pensò, ricordando la espressione del suo viso quando s'era impappinato sulla parola''sofferto``; - no, può forse un uomo con quegli occhi appannati, con quella presuntuosa calma sentire qualche cosa?''. 

- Io non posso cambiare nulla - ella mormorò. 

- Son venuto a dirvi che domani parto per Mosca, e non tornerò più in questa casa, e voi avrete notizia delle mie decisioni attraverso l'avvocato, al quale affiderò la pratica del divorzio. Mio figlio andrà da mia sorella - disse Aleksej Aleksandrovic, ricordando con uno sforzo quel che voleva dire del figlio. 

- Vi occorre Serëza per farmi del male - ella disse guardandolo di sotto in su. - Voi non lo amate\ldots{} Lasciatemi Serëza! 

- E sì, ho perso anche l'amore per mio figlio, perché la mia avversione per voi ha trascinato insieme anche lui. Tuttavia lo prenderò. Addio! 

E voleva andar via, ma ora fu lei a trattenerlo. 

- Aleksej Aleksandrovic, lasciatemi Serëza! - mormorò lei ancora una volta. - Io non ho altro da dire. Lasciatemi Serëza fino al mio\ldots{} Partorirò presto, lasciatemelo! 

Aleksej Aleksandrovic s'infiammò, e, svincolato bruscamente il braccio, uscì dalla stanza in silenzio. 

\capitolo{V}\label{v-3} 

La sala d'aspetto del noto avvocato di Pietroburgo era piena di gente quando Aleksej Aleksandrovic vi entrò. Tre signore, una anziana, una giovane e la moglie di un mercante; tre signori: il primo, un banchiere tedesco con un anello al dito, il secondo, un commerciante con la barba, il terzo, un impiegato rabbioso in piccola tenuta con una decorazione al collo, aspettavano già da tempo. Due procuratori scrivevano sui tavoli, facendo stridere le penne. Il materiale per scrivere che Aleksej Aleksandrovic amava in maniera particolare, era veramente buono. Aleksej Aleksandrovic non poté non notarlo. Uno dei procuratori, senza alzarsi, accigliatosi, si voltò irritato verso Aleksej Aleksandrovic. 

- Che vi occorre? 

- Ho un affare con l'avvocato. 

- L'avvocato è occupato - rispose severo il procuratore, indicando con la penna quelli che aspettavano, e continuò a scrivere. 

- Non può trovare un po' di tempo? - disse Aleksej Aleksandrovic. 

- Non ha tempo libero, è sempre occupato. Favorite aspettare. 

- Volete allora compiacervi di dargli il mio biglietto da visita? - disse dignitosamente Aleksej Aleksandrovic, vedendo l'impossibilità di mantenere l'incognito. 

Il procuratore prese il biglietto e, disapprovandone evidentemente il contenuto, sparì dietro una porta. 

Aleksej Aleksandrovic era propenso in teoria a un'azione giudiziaria; ma alcuni particolari di procedura non li approvava del tutto per certe altre ragioni di ufficio a lui note, e li criticava per quanto poteva criticare una qualunque cosa che avesse avuto la sanzione sovrana. Tutta la sua vita era trascorsa in attività burocratiche, perciò quando non concordava con qualche cosa, questo disaccordo veniva attenuato dalla consapevolezza dell'inevitabilità degli errori e della possibilità della loro correzione in un qualsiasi settore. Nelle nuove istituzioni giudiziarie non approvava le condizioni in cui era stata posta l'avvocatura. Ma finora non aveva mai avuto a che fare con essa, perciò la disapprovava solo in teoria; ora invece la sua disapprovazione s'era rafforzata per l'impressione sfavorevole prodottagli dalla sala d'aspetto dell'avvocato. 

- Viene subito - disse il procuratore, e, dopo due minuti, apparvero sulla soglia la figura lunga di un vecchio giurista che parlava con l'avvocato e l'avvocato stesso. 

L'avvocato era piccolo, tarchiato, calvo, con una barba rossoscura, le sopracciglia chiare, lunghe e la fronte rialzata. Era parato come uno sposo, dalla cravatta e dalla catena doppia fino alle scarpe di coppale. Il viso era intelligente, maschio; gli ornamenti da zerbinotto e di pessimo gusto. 

- Accomodatevi - disse l'avvocato, rivolgendosi ad Aleksej Aleksandrovic e, fatto passare l'accigliato Karenin davanti a sé, chiuse la porta. 

- Non vi dispiace? - e indicò la poltrona presso la scrivania carica di carte, mentre egli stesso sedeva al posto presidenziale, fregandosi le piccole mani dalle dita corte, coperte di peli bianchi, e piegando la testa da un lato. Ma s'era appena assestato nella sua posa, che sulla tavola volò una tignola. L'avvocato con una velocità quale non ci si poteva aspettare da lui, aprì le mani, afferrò la tignola e riprese la posa di prima. 

- Prima di parlare del mio affare - disse Aleksej Aleksandrovic seguendo sorpreso con gli occhi i movimenti dell'avvocato - devo farvi osservare che la questione della quale vengo a parlarvi, deve rimanere segreta. 

Un sorriso appena percettibile sollevò i baffi rossicci e prominenti dell'avvocato. 

- Non sarei un avvocato se non sapessi custodire i segreti affidatimi. Ma se desiderate una conferma\ldots{} 

Aleksej Aleksandrovic lo guardò in faccia e vide che gli occhi grigi, intelligenti, ridevano e pareva che sapessero già tutto. 

- Voi sapete il mio nome? - proseguì Aleksej Aleksandrovic. 

- Conosco voi e la vostra utile\ldots{} - e qui di nuovo afferrò una tignola - attività, come ogni russo, del resto - disse l'avvocato, inchinandosi. 

Aleksej Aleksandrovic sospirò, raccogliendo le proprie forze. Ma una volta deciso, continuò con la sua voce stridula, senza timori, senza esitazioni e sottolineando alcune parole. 

- Io ho la sventura - cominciò Aleksej Aleksandrovic - di essere un marito tradito e desidero rompere legalmente i rapporti con mia moglie, cioè divorziare, ma in modo che mio figlio non rimanga con la madre. 

Gli occhi grigi dell'avvocato cercavano di non ridere, ma saltellavano qua e là per una gioia irrefrenabile, e Aleksej Aleksandrovic vedeva che in essi c'era non la sola gioia di un uomo che riceveva un incarico vantaggioso, ma come una certa entusiastica festosità, c'era uno scintillio simile a quello cattivo che aveva sorpreso negli occhi della moglie. 

- Voi desiderate la mia collaborazione per ottenere il divorzio? 

- Sì, proprio, vi devo però dire che rischio di abusare della vostra cortesia. Sono venuto solo a consigliarmi preventivamente con voi. Desidero il divorzio, ma per me è importante conoscere le forme attraverso le quali è possibile ottenerlo. Molto probabilmente, dunque, se le forme non coincideranno con le mie aspirazioni, rinuncerò all'azione legale. 

- Oh, è sempre così - disse l'avvocato - e questo dipende del tutto dalla vostra volontà. 

L'avvocato abbassò gli occhi verso i piedi di Aleksej Aleksandrovic, sentendo di poter offendere il cliente con la propria incontenibile ilarità. Guardò una tignola che volava davanti al naso di lui e tese la mano, ma non l'afferrò per un riguardo alla situazione di Aleksej Aleksandrovic. 

- Benché nelle linee generali le nostre disposizioni legislative in materia mi siano note - continuò Aleksej Aleksandrovic - desidererei conoscere le forme con cui in pratica si compiono gli affari di questo genere. 

- Voi desiderate - rispose l'avvocato senza alzare gli occhi, entrando non senza soddisfazione nel tono del discorso del suo cliente - che io vi esponga le vie attraverso le quali è possibile attuare il vostro proposito. 

E, a un cenno affermativo del capo di Aleksej Aleksandrovic, continuò, guardando solo di rado, di sfuggita, il viso chiazzato di Aleksej Aleksandrovic. 

- Il divorzio, secondo le nostre leggi - disse con una leggera disapprovazione verso le leggi - è possibile, come vi è noto, nei seguenti casi\ldots{} Aspettate! - disse rivolto al procuratore che s'era infilato attraverso la porta, tuttavia si alzò, gli disse alcune parole e sedette di nuovo. - Nei seguenti casi: incapacità fisica dei coniugi, assenza di notizie per cinque anni - disse, piegando il pollice ricoperto di peli - poi adulterio - pronunciò questa parola con evidente soddisfazione. - Le suddivisioni sono le seguenti - egli continuava a piegare le dita grasse, sebbene i casi e le loro suddivisioni non potessero essere elencati insieme: - difetti fisici del marito o della moglie, quindi adulterio del marito o della moglie. - Avendo piegato tutte le dita, le raddrizzò e continuò: - Questo è il punto di vista teorico, ma io suppongo che voi mi abbiate fatto l'onore di rivolgervi a me per conoscere l'applicazione pratica. E perciò, facendomi guidare dagli antecedenti, devo dirvi che i casi di divorzio si riducono tutti ai seguenti: difetti fisici, niente, a quanto posso capire, e così pure assenza di notizie?\ldots{} 

Aleksej Aleksandrovic chinò il capo affermativamente. 

- Si riducono ai seguenti: adulterio di uno dei coniugi e prova della colpevolezza di una delle parti risultante da reciproco accordo, oppure, fuori di un tale accordo, prova legale. Devo far presente che nella pratica l'ultimo caso s'incontra di rado - disse l'avvocato e, guardando di sfuggita Aleksej Aleksandrovic, tacque come un venditore di pistole che, descritti i vantaggi di questa o di quell'arma, attenda la scelta del compratore. Ma Aleksej Aleksandrovic taceva e perciò l'avvocato proseguì: - La cosa più usuale e semplice e la più ragionevole a mio modo di vedere, è l'adulterio ammesso da accordo reciproco. Non mi sarei permesso di esprimermi così, parlando con una persona di scarsa cultura - disse l'avvocato - ma suppongo che per voi sia comprensibile. 

Aleksej Aleksandrovic era così sconvolto che non capì subito la ragionevolezza dell'adulterio ammesso per reciproco accordo e si leggeva la perplessità nel suo sguardo; ma l'avvocato gli venne subito in aiuto. 

- Due persone non possono più vivere insieme, ecco il fatto. E se tutti e due sono in questo d'accordo, allora i particolari e i modi diventano indifferenti. E nello stesso tempo questo è il mezzo più semplice e sicuro. 

Aleksej Aleksandrovic ora capiva in pieno. Ma aveva le sue esigenze religiose che gli impedivano l'accettazione di questa procedura. 

- Questo è fuori questione nel caso presente - egli disse. - Qui una sola cosa è possibile: la prova del fatto è confermata dalle lettere che sono in mio possesso. 

L'avvocato a sentir nominare le lettere, strinse le labbra ed emise un suono flebile tra il compassionevole e lo sprezzante. 

- Abbiate la bontà di considerare - egli riprese - che gli affari di questo genere sono decisi, come vi è noto, dalla giurisdizione ecclesiastica; e i preti, in affari di questo genere, sono ghiotti di particolari minutissimi - egli disse con un sorriso che mostrava simpatia verso il gusto dei preti. - Le lettere, senza dubbio, possono confermare altri elementi, ma le prove devono essere ottenute per via diretta, cioè per testimoni. D'altra parte, poi, se mi fate l'onore di degnarmi della vostra fiducia, riservatemi la scelta dei mezzi da adoperare. Chi vuole il resultato, accetti anche i mezzi. 

- Se è così\ldots{} - cominciò, improvvisamente impallidendo Aleksej Aleksandrovic, ma, in quel momento, l'avvocato si alzò e andò di nuovo verso la porta dove era il procuratore che l'aveva interrotto. 

- Ditele che non siamo al mercato! - disse e ritornò da Aleksej Aleksandrovic. 

Tornando al posto, afferrò, inosservato, un'altra tignola. ``Sarà bello il mio reps in estate\ldots{}'' pensò, accigliandosi. 

- Così voi state dicendo\ldots{} - disse. 

- Vi comunicherò la mia decisione per iscritto - disse Aleksej Aleksandrovic alzandosi, e si appoggiò alla tavola. Dopo essere rimasto in piedi un po' in silenzio, disse: - Dalle vostre parole posso concludere che il conseguimento del divorzio è possibile. Vi pregherei di comunicarmi pure quali sono le vostre condizioni\ldots{} 

- Tutto è possibile se mi concederete piena libertà di azione - disse l'avvocato senza rispondere in tutto alla domanda. - Quando posso contare di ricevere vostre comunicazioni? - chiese, dirigendosi verso la porta e brillando con gli occhi e gli stivaletti di coppale. 

- Fra una settimana. E, se assumete il patrocinio di questo affare, sarete gentile di comunicarmi a quali condizioni. 

- Molto bene. 

L'avvocato si inchinò ossequioso, fece uscire dalla porta il cliente, e, rimasto solo, si abbandonò alla propria ilarità. Divenne così allegro che, contrariamente alle sue abitudini, fece uno sconto alla signora che contrattava con lui, cessò di afferrar tignole e decise che per l'inverno prossimo occorreva ricoprire il mobilio con del velluto, come da Sigonin. 

\capitolo{VI}\label{vi-3} 

Aleksej Aleksandrovic aveva ottenuto una brillante vittoria nella seduta della commissione del 17 agosto, ma le conseguenze di questa vittoria gli spezzarono le ali. La nuova commissione per investigare sotto tutti i rapporti la vita degli allogeni, era stata sostituita e mandata sul posto con inconsueta sollecitudine ed energia suscitate da Aleksej Aleksandrovic. Dopo tre mesi era stata presentata la relazione. La vita degli allogeni era stata osservata sotto l'aspetto politico, amministrativo, economico, etnografico, materiale e religioso. E a tutti questi problemi erano state date delle risposte ottimamente redatte ed inequivocabili, che non ammettevano dubbi, poiché non erano il prodotto del pensiero di un uomo soggetto a errori, ma erano tutte frutto di un'attività burocratica. Le risposte erano il risultato dei dati ufficiali, dei rapporti dei governatori e degli arcivescovi, basati sui rapporti dei capi dei distretti e dei sovrintendenti ecclesiastici, basati, a loro volta, sui rapporti delle amministrazioni comunali e dei prelati e perciò tutte queste risposte erano indubitabili. Tutte le questioni a proposito del perché, per esempio, i raccolti fossero scarsi, del perché gli abitanti si attenessero alle loro credenze e così via, questioni che, senza la comodità della macchina burocratica, non si risolvono e non possono essere risolte per secoli, ebbero una chiara indubitabile soluzione. E questa soluzione era in favore dell'idea di Aleksej Aleksandrovic. Ma Stremov, sentitosi toccato nel vivo nell'ultima seduta, all'arrivo delle relazioni della commissione usò una tattica che Aleksej Aleksandrovic non sospettava, e, non solo difese con molto calore l'attuazione delle misure proposte da Karenin, ma ne propose altre che di queste erano le estreme conseguenze. Queste misure, eccessive rispetto a quello che era il pensiero fondamentale di Aleksej Aleksandrovic, furono accolte, e allora la tattica di Stremov si scoprì. Queste misure, portate all'estremo, apparvero d'un tratto così assurde che sia gli uomini di stato che l'opinione pubblica e le signore intellettuali e i giornali, tutti, nello stesso tempo, vi si scagliarono contro, esprimendo la propria indignazione, e si scagliarono contro le misure stesse e contro il loro fautore, Aleksej Aleksandrovic. Stremov allora si fece da parte, fingendo d'aver solo voluto ciecamente seguire il piano di Karenin, mentre egli stesso si mostrava sorpreso e confuso di quello che era stato fatto. Questo spezzò le ali ad Aleksej Aleksandrovic. Ma nonostante la salute che deperiva e i dispiaceri familiari, Aleksej Aleksandrovic non si arrese. In seno alla commissione si era prodotta una scissione. Alcuni membri con Stremov a capo, giustificavano il proprio errore dicendo d'aver avuto fiducia nella commissione di revisione guidata da Aleksej Aleksandrovic, la quale aveva presentato un rapporto che altro non era risultato che un'assurdità e della carta scritta. Aleksej Aleksandrovic, insieme col partito delle persone che vedevano il pericolo di considerar le pratiche in modo così rivoluzionario, seguitava a sostenere i dati elaborati dalla commissione ispettiva. In seguito a ciò, nelle alte sfere e persino nei salotti si confuse tutto e, malgrado questo interessasse estremamente tutti, nessuno riusciva a capire se gli allogeni fossero realmente nella miseria e perissero o se prosperassero. La posizione di Aleksej Aleksandrovic, in conseguenza di ciò e in parte in conseguenza del disonore caduto su di lui per l'infedeltà della moglie, si fece molto vacillante. Ma pure in questo stato di cose, Aleksej Aleksandrovic prese una decisione importante. Con stupore della commissione, dichiarò che avrebbe chiesto l'autorizzazione ad andare sul posto per ispezionare di persona. E, sollecitatane l'autorizzazione, Aleksej Aleksandrovic si mise in viaggio per governatorati lontani. 

Il viaggio di Aleksej Aleksandrovic suscitò grande scalpore, tanto più che, proprio all'atto di partire, egli restituì ufficialmente, con documento, il denaro assegnatogli per le spese dei dodici cavalli necessari per raggiungere il luogo della missione. 

- Giudico questo molto nobile - diceva a questo proposito Betsy con la principessa Mjagkaja. - Perché dare il denaro per i cavalli da posta, quando tutti sanno che adesso ci sono dovunque le strade ferrate? 

Ma la Mjagkaja non era d'accordo e l'opinione della Tverskaja la irritava persino. 

- Voi parlate bene - ella disse - voi che avete non so quanti milioni, ma a me piace che mio marito vada in missione nell'estate. Gli fa bene alla salute, gli fa piacere fare un viaggio e io ormai ho stabilito che con quel denaro, in casa mia, si pagano carrozza e cocchiere. 

Di passaggio per governatorati lontani, Aleksej Aleksandrovic si fermò tre giorni a Mosca. 

Il giorno dopo il suo arrivo, andò in carrozza a far visita al governatore generale. All'incrocio del vicolo Gazetnyj, dove si affollano sempre carrozze e vetture, Aleksej Aleksandrovic sentì ad un tratto il proprio nome gridato da una voce così forte e vivace che non poté non rimanere colpito. All'angolo del marciapiedi, in cappotto corto alla moda, con un cappello a falde strette messo di lato e un sorriso splendente fra le labbra rosse e i denti bianchi, allegro, giovane, raggiante, stava Stepan Arkad'ic che, energicamente e imperiosamente gridava e pretendeva che egli si fermasse. Si teneva con una mano al finestrino di una carrozza che s'era fermata all'angolo, dalla quale si sporgevano una testa di donna con un cappello di velluto e due testoline di bimbi, e sorrideva e faceva segno con l'altra mano al cognato. La signora sorrideva con un sorriso buono, e faceva anche lei dei gesti al Aleksej Aleksandrovic. 

Era Dolly con i bambini. 

Aleksej Aleksandrovic aveva deciso di non vedere nessuno a Mosca e meno di tutti il fratello di sua moglie. Sollevò il cappello, e voleva passar via, ma Stepan Arkad'ic ordinò al cocchiere di fermare e corse verso di lui sulla neve. 

- Ma come non far sapere niente! È molto che sei qui? E io ieri sono stato da Dussau e ho visto sulla tabella ``Karenin'' e non m'è venuto in mente che fossi tu! - diceva Stepan Arkad'ic , ficcandosi con la testa nel finestrino della carrozza. - E sarei venuto io. Come sono contento di vederti! - diceva, battendo un piede contro l'altro per scuoter via la neve. - Ma come non senti di essere colpevole a non farti vedere! - ripeteva. 

- Non ne ho avuto il tempo, sono molto occupato - rispose secco Aleksej Aleksandrovic. 

- Andiamo da mia moglie, vuole tanto vederti. 

Aleksej Aleksandrovic si sbarazzò dello scialle nel quale erano avviluppate le sue gambe freddolose e, uscito dalla carrozza, si fece strada fra la neve verso Dar'ja Aleksandrovna. 

- Che c'è mai, Aleksej Aleksandrovic, perché ci evitate così? - disse Dolly sorridendo. 

- Sono stato molto occupato. Sono molto contento di vedervi - disse con un tono che chiaramente diceva che ne era invece contrariato. - Come va la vostra salute? 

- Ebbene, che ne è della mia cara Anna? 

Aleksej Aleksandrovic mugolò qualcosa e voleva andarsene, ma Stepan Arkad'ic lo trattenne. 

- Ma, ecco cosa facciamo domani. Dolly, invitalo a pranzo! Inviteremo anche Koznyšev e Pescov per offrirgli dell'intelligencija moscovita. 

- Così, vi prego, venite - disse Dolly - noi vi aspettiamo alle cinque, alle sei se volete. E la mia cara Anna? Come da tempo\ldots{} 

- Sta bene - mugolò Aleksej Aleksandrovic accigliandosi. - Molto lieto! - e si diresse verso la carrozza. 

- Verrete? - gridò Dolly. 

Aleksej Aleksandrovic pronunciò qualcosa che Dolly non poté sentire fra il rumore delle vetture che si movevano. 

- Passerò domani - gli gridò Stepan Arkad'ic . 

Aleksej Aleksandrovic sedette nella vettura e vi si sprofondò in modo da non vedere e da non essere visto. 

- Che originale! - disse Stepan Arkad'ic alla moglie e, guardata l'ora, fece un movimento con la mano che voleva essere una carezza sul viso della moglie e dei bambini e s'incamminò spavaldo per il marciapiedi. 

- Stiva! Stiva! - disse Dolly, arrossendo. 

Egli si voltò. 

- Ho bisogno di denaro, sai, per comprare un cappotto a Griša e un altro a Tanja. Dammi dunque il denaro. 

- Non fa nulla; di' che pagherò poi - e scomparve dopo aver salutato allegramente, con un cenno del capo, un conoscente che passava. 

\capitolo{VII}\label{vii-3} 

Il giorno dopo era domenica. Stepan Arkad'ic andò al Bol'šoj Teatr per assistere alle prove di un balletto e per consegnare a Maša cibisova, una graziosa ballerina che di recente ne faceva parte perché da lui protetta, i coralli promessile il giorno innanzi, e, nell'oscurità diurna del teatro, dietro una quinta, riuscì a baciarne il visetto simpatico, splendente di gioia per il regalo. Oltre al regalo di coralli doveva prendere accordi con lei per un appuntamento dopo il balletto. Spiegatole perché non poteva trovarsi all'inizio del ballo, promise di venire all'ultimo atto e di condurla a cena. Dal teatro Stepan Arkad'ic andò all'Ochotnyj Rjad, scelse egli stesso il pesce e gli asparagi per il pranzo e alle dodici era già da Dussau per recarsi dai tre personaggi che, per sua fortuna, alloggiavano nello stesso albergo; da Levin che s'era fermato lì ed era tornato da poco dall'estero, dal suo nuovo capo che era allora allora entrato in carica, e che ispezionava Mosca, e dal cognato Karenin per averlo a ogni costo a pranzo. 

A Stepan Arkad'ic piaceva mangiar bene, ma ancor più dare un pranzo, non grandioso, ma raffinato e per cibi e per la scelta dei commensali. La lista del pranzo odierno era proprio di suo gusto: ci sarebbero stati i persici vivi, gli asparagi e, come pièce de résistance, un meraviglioso, ma semplice roastbeef, e i vini adatti: questo per il cibo e le bevande. Come invitati ci sarebbero stati Kitty e Levin e, perché questo non desse nell'occhio, anche una cugina e il giovane Šcerbackij, e, come pièce de résistance, Sergej Koznyšev e Aleksej Aleksandrovic: Sergej Ivanovic moscovita e filosofo, Aleksej Aleksandrovic pietroburghese e uomo d'azione; inoltre avrebbe invitato quell'originale di Pescov, liberale, chiacchierone, musicista, storico e simpaticissimo scapolo cinquantenne che avrebbe costituito la salsa o il contorno a Koznyšev e Karenin. Egli li avrebbe eccitati e aizzati l'un contro l'altro. 

Il denaro del compratore del legname era stato incassato alla seconda scadenza e non era ancora speso; Dolly era molto carina e buona in questi ultimi tempi, e l'idea del pranzo, sotto tutti gli aspetti, rallegrava Stepan Arkad'ic . Egli, si trovava nella più felice disposizione d'animo. Aveva solo due ragioni di malcontento; ma entrambe si sperdevano nel mare di benevola allegria che fluttuava nell'animo suo. Esse erano: la prima, che il giorno avanti, incontrato per via Aleksej Aleksandrovic, aveva notato ch'egli era stato asciutto e brusco con lui e, associando questa espressione del viso di Aleksej Aleksandrovic e il fatto che non era venuto da loro e non aveva fatto sapere nulla di sé con le voci che circolavano sul conto di Anna e Vronskij, Stepan Arkad'ic indovinò che c'era qualcosa che non andava tra marito e moglie. 

Questa era una delle cose spiacevoli. L'altra alquanto spiacevole era che il nuovo capo, come tutti i nuovi capi, aveva fama di uomo terribile, che si alzava la mattina alle sei, lavorava come un bue e pretendeva un egual lavoro dai dipendenti. Inoltre questo nuovo capo aveva anche fama di orso pel suo modo di trattare, ed era, secondo le voci, una persona di tendenze completamente contrarie a quelle del capo precedente e seguite da Stepan Arkad'ic. Il giorno prima egli si era presentato in ufficio in divisa, e il nuovo capo era stato molto amabile e s'era messo a parlare con Oblonskij come con un amico; perciò Stepan Arkad'ic reputava suo dovere fargli visita in finanziera. Il pensiero che il nuovo capo potesse riceverlo male, rappresentava la seconda circostanza spiacevole. Ma Stepan Arkad'ic sentiva istintivamente che tutto si sarebbe ``appianato'' nel modo migliore. ``Tutti sono esseri umani, uomini, come noi, peccatori; perché arrabbiarsi e litigare?'' pensava, entrando nell'albergo. 

- Salve, Vasilij - disse, passando col cappello di sghembo per un corridoio e rivolgendosi a un cameriere di sua conoscenza - ti sei fatto crescere le fedine? Levin è al n. 7, eh? Accompagnami, per favore. E informati se il conte Anickin - era il suo nuovo capo - riceve o no. 

- Sissignore - rispose Vasilij, sorridendo. - È un pezzo che attendiamo i vostri ordini. 

- Sono stato qui, ieri, ma dall'altro ingresso. È il n. 7? 

Levin stava in piedi in mezzo alla stanza con un contadino di Tver' e misurava con un aršin una pelle d'orso fresca, quando entrò Stepan Arkad'ic. 

- Ehi! L'avete ucciso voi? - gridò Stepan Arkad'ic. - Bel giocattolo! Un'orsa? Buongiorno, Archip! 

Strinse la mano al contadino e sedette su di una sedia senza togliersi cappello e cappotto. 

- Ma metti via questa roba, siedi un po' - disse Levin, togliendogli il cappello. 

- Non ho tempo, son venuto solo per un attimo - rispose Stepan Arkad'ic. Sbottonò il cappotto e poi se lo tolse e rimase un'ora intera, discorrendo con Levin di caccia e di cose intime. 

- Dimmi, ti prego, cosa hai fatto all'estero? dove sei stato? - disse Stepan Arkad'ic quando il contadino fu uscito. 

- Sì, sono stato in Germania, in Prussia, in Francia e in Inghilterra; ma non nelle capitali, nelle città industriali, e ho visto molte cose nuove. E sono contento d'esserci stato. 

- Già, io so la tua idea di organizzare il lavoratore. 

- Per nulla affatto: in Russia non può esistere una questione operaia. In Russia c'è la questione dei rapporti del popolo lavoratore con la terra; c'è anche là, ma là si tratta di riparare quel che s'è guastato, invece da noi\ldots{} 

Stepan Arkad'ic ascoltava attentamente Levin. 

- Già, già - diceva. - È molto probabile che tu abbia ragione - disse. - Ma io sono felice che tu sia di buon umore; vai a caccia di orsi e lavori, e sei tutto preso dal lavoro. E invece Šcerbackij mi diceva d'averti incontrato, che eri in non so quale stato d'abbattimento, che parlavi sempre di morte. 

- Già, ma non smetto di pensare alla morte - disse Levin. - È vero che è ora di morire. E che tutto questo è vanità. Io ti dirò il vero: ho straordinariamente cara la mia idea e il lavoro, ma in sostanza, pensaci: tutto questo mondo è in fin dei conti una piccola muffa che è spuntata su di un minuscolo pianeta. E noi pensiamo di essere in possesso di qualcosa di grande\ldots{} pensieri, affari! Granelli di sabbia tutti questi! 

- Ma questo, amico mio, è vecchio come il mondo! 

- È vecchio; ma quando lo capisci chiaramente, allora tutto, in un certo modo si riduce a niente. Quando capisci che oggi o domani morirai, e non resterà più nulla, che tutto sarà annientato! Ecco, io considero molto importante la mia idea, eppure questa anche a pensarla attuata appare così insignificante come fare il giro della pelle di quest'orso. Allora passi la vita distraendoti con la caccia, col lavoro, solo per non pensare alla morte. 

Stepan Arkad'ic sorrideva finemente e benevolmente ascoltando Levin. 

- Su, s'intende! Ecco che tu sei venuto dalla parte mia. Ricordi che m'investivi perché cercavo i piaceri nella vita? 

- Non esser, moralista, così duro! 

- No, tuttavia nella vita vi è tanto di buono che\ldots{} - Levin si confuse. - Ma non so. So soltanto che morirò presto. 

- E perché presto? 

- E sai, c'è meno gioia nella vita quando pensi alla morte, ma c'è più calma. 

- Al contrario, quando si è alla fine si sta più allegri. Su, però per me è ora - disse Stepan Arkad'ic, alzandosi per la decima volta. 

- Ma no, siedi! - diceva Levin, trattenendolo. - Quando ci vedremo ora? Io vado via domani. 

- E io, che bravo! son venuto apposta\ldots{} Vieni assolutamente oggi a pranzo da me. Ci sarà tuo fratello, mio cognato Karenin. 

``Forse lei è qui'' pensò Levin, e voleva chiedere di Kitty. Aveva sentito che al principio dell'inverno era stata a Pietroburgo dalla sorella sposata a un diplomatico e non sapeva se ne era tornata o no; ma rinunciò a chiedere. ``Che ci sia o non ci sia, per me è lo stesso''. 

- Allora, verrai? 

- Eh, s'intende. 

- Alle cinque e in finanziera. 

E Stepan Arkad'ic si alzò e andò giù dal nuovo capo. L'istinto non l'aveva ingannato. Il nuovo terribile capo si mostrò un uomo molto affabile, e Stepan Arkad'ic fece colazione con lui e si trattenne così a lungo che solo verso le quattro si recò da Aleksej Aleksandrovic. 

\capitolo{VIII}\label{viii-3} 

Aleksej Aleksandrovic, tornato dalla messa, passò tutta la mattina in casa. In quella mattinata aveva da fare due cose: in primo luogo ricevere e dare le direttive a una deputazione di allogeni che andava a Pietroburgo e che attualmente si trovava a Mosca; in secondo luogo scrivere la lettera promessa all'avvocato. La deputazione fatta venire per iniziativa di Aleksej Aleksandrovic presentava molti svantaggi e persino dei pericoli, e Aleksej Aleksandrovic era molto contento d'averla incontrata a Mosca. Infatti i membri di questa deputazione non avevano la minima idea della parte che rappresentavano e del compito loro affidato. Erano ingenuamente convinti che tutto consistesse nell'esporre la necessità e la vera situazione delle cose, chiedendo l'aiuto del governo, ma non pensavano menomamente che alcune loro dichiarazioni e pretese potessero essere sostenute dal partito opposto ad Aleksej Aleksandrovic e rovinare perciò tutto l'affare. Aleksej Aleksandrovic li catechizzò a lungo, stese per loro un programma dal quale non dovevano deviare e, congedatili, scrisse delle lettere a Pietroburgo circa l'indirizzo che la deputazione avrebbe dovuto seguire. Il più valido aiuto in questa faccenda doveva venire dalla contessa Lidija Ivanovna. Era specialista in fatto di deputazioni, e nessuno come lei sapeva farle valere e dare loro un preciso indirizzo. Sistemate queste faccende, Aleksej Aleksandrovic scrisse all'avvocato. Senza la minima esitazione gli diede l'autorizzazione a procedere come meglio riteneva. Nella lettera incluse i tre biglietti di Vronskij ad Anna, trovati nel portafogli sottratto. 

Dal momento in cui Aleksej Aleksandrovic era andato via di casa, con l'intenzione di non tornare più in famiglia, dal momento in cui era stato dall'avvocato, e aveva detto, sia pure a una persona, la sua intenzione, e proprio dal momento in cui aveva tradotto questa faccenda della sua vita in un affare legale, si era abituato sempre più alla decisione, e adesso vedeva chiara la possibilità di attuarla. Stava sigillando la busta per l'avvocato, quando sentì il suono forte della voce di Stepan Arkad'ic che discuteva col cameriere e insisteva perché lo si annunciasse. 

``È lo stesso - pensò Aleksej Aleksandrovic - meglio: dichiarerò subito la mia posizione nei confronti di sua sorella e spiegherò perché non posso pranzare da lui''. 

- Fa' passare - disse a voce alta, raccogliendo le carte e mettendole nella cartella. 

- Ecco, vedi che menti, è in casa! - rispondeva la voce di Stepan Arkad'ic al cameriere che non lo lasciava passare e, togliendosi il cappotto nel camminare, Oblonskij entrò nella stanza. - Sì, son molto contento d'averti trovato! Così io spero\ldots{} - cominciò allegramente. 

- Non posso venire - disse Aleksej Aleksandrovic freddo, in piedi e senza far sedere l'ospite. 

Aleksej Aleksandrovic pensava di stabilire subito quei rapporti di fredda cortesia, che gli pareva dovessero ora correre tra lui e il fratello della moglie, contro la quale andava intentando un giudizio di divorzio; ma non aveva fatto i conti con quel mare di bonarietà che straripava dall'animo di Stepan Arkad'ic . 

Stepan Arkad'ic aprì i suoi occhi scintillanti, chiari. 

- Perché non puoi? Cosa vuoi dire? - disse con perplessità, in francese. - No, è già promesso. E noi tutti contiamo su di te! 

- Io voglio dire che non posso venire, perché quei rapporti di parentela che esistevano tra noi, devono cessare. 

- Come? cioè, come mai? perché? - pronunciò con un sorriso Stepan Arkad'ic . 

- Perché do inizio a una causa di divorzio contro vostra sorella, contro mia moglie. Ho dovuto\ldots{} 

Ma Aleksej Aleksandrovic non aveva ancora fatto in tempo a finire il suo discorso che Stepan Arkad'ic aveva già agito in modo del tutto diverso da quello ch'egli si aspettava. Stepan Arkad'ic mise un gemito e si sedette in una poltrona. 

- No, Aleksej Aleksandrovic, che dici mai? - gridò Oblonskij e la sofferenza si espresse sul suo viso. 

- È così. 

- Perdonami, io non posso e non voglio crederci. 

Aleksej Aleksandrovic sedette, vedendo che le sue parole non avevano avuto quell'effetto ch'egli si aspettava e che ormai inevitabilmente avrebbe dovuto spiegarsi e che i suoi rapporti col cognato, quali che fossero state le sue spiegazioni, sarebbero rimasti inalterati. 

- Sì, sono stato messo nella penosa necessità di esigere il divorzio - egli disse. 

- Io una cosa sola ti dico, Aleksej Aleksandrovic. Io ti conosco per un uomo eccellente, giusto; conosco Anna, scusami, non posso cambiare l'opinione che ho di lei, di bravissima, ottima donna, e perciò, perdonami, non posso credere a questo. Qui c'è un equivoco - egli disse. 

- Già, se si trattasse solo di un equivoco\ldots{} 

- Permetti, io capisco - interruppe Stepan Arkad'ic . - Ma s'intende\ldots{} Una cosa sola: non bisogna precipitare. Non si deve, non si deve avere fretta. 

- Io non ho avuto fretta - disse freddo Aleksej Aleksandrovic - e in simili casi nessuno può dar consigli. Io ho fermamente deciso. 

- È terribile! - disse Stepan Arkad'ic sospirando penosamente. - Avrei fatto una sola cosa, Aleksej Aleksandrovic. Ti supplico, fa' ciò che ti dico - disse. - La causa non è ancora cominciata, a quanto ho capito. Prima di iniziare il giudizio, vedi mia moglie, parla con lei. Ella vuol bene ad Anna come a una sorella, vuol bene a te, ed è una donna sorprendente. Per amor di Dio, parla con lei. Dammi questa prova di amicizia, te ne supplico! 

Aleksej Aleksandrovic si fece pensieroso e Stepan Arkad'ic lo guardava con simpatia, senza interrompere il suo silenzio. 

- Ci andrai da lei? 

- Ma, non so. È per questo che non son venuto da voi. Suppongo che i nostri rapporti debbano cambiare. 

- E perché mai? Non ne vedo la necessità. Permettimi di tener presente che, oltre ai nostri rapporti di parentela, tu hai avuto per me, almeno in parte, quei sentimenti di amicizia che io ho sempre avuto per te. È stima sincera - disse Stepan Arkad'ic stringendogli la mano. - Se anche le tue peggiori supposizioni fossero vere, io non prendo e non prenderò mai su di me la responsabilità di giudicare in favore dell'una o dell'altra parte, e non vedo la ragione per la quale i nostri rapporti debbano cambiare. Ma ora, fa' una cosa, vieni da mia moglie. 

- Già, noi consideriamo in modo diverso questa faccenda - disse freddamente Aleksej Aleksandrovic. - Del resto, non ne parliamo più. 

- No, perché non venire? Magari oggi a pranzo. Mia moglie ti aspetta. Ti prego, vieni. E soprattutto, parla con lei. È una donna sorprendente. Per amor di Dio, ti supplico in ginocchio. 

- Se lo volete tanto, verrò - disse, sospirando, Aleksej Aleksandrovic. 

E desiderando cambiar discorso, domandò di quello che interessava entrambi: del nuovo capo di Stepan Arkad'ic, uomo non ancora vecchio, che aveva improvvisamente ricevuto una così alta promozione. 

- Ebbene, l'hai visto? - disse Aleksej Aleksandrovic, con un sorriso velenoso. 

- Come no, ieri è stato da noi in tribunale. Sembra che sappia molto bene il fatto suo e sia molto attivo. 

- Sì, ma a che è diretta la sua attività? - disse Aleksej Aleksandrovic. - Ad agire o a ricalcare quello che è stato fatto? La disgrazia del nostro stato è quest'amministrazione burocratica di cui egli è un degno rappresentante. 

- Davvero non so, ma so una cosa sola: è un ottimo giovane - rispose Stepan Arkad'ic . - Sono stato subito da lui, e davvero è un ottimo giovane. Abbiamo fatto colazione insieme, e io gli ho insegnato a fare quella bevanda, sai, il vino con le arance. Rinfresca molto. È strano che non la conoscesse. Gli è piaciuta molto. Sì, davvero è una simpatica persona. 

Stepan Arkad'ic guardò l'orologio. 

- Ah, Dio mio, sono già più delle quattro, e io devo ancora andare da Dolgovušin! Allora, ti prego, vieni a pranzo! Non puoi credere come addoloreresti mia moglie! 

Aleksej Aleksandrovic lo accompagnò in tutt'altro modo da come l'aveva ricevuto. 

- Ho promesso e verrò - rispose con tristezza. 

- Credimi, apprezzo ciò e spero che non te ne pentirai - rispose sorridendo Stepan Arkad'ic. E, infilatosi il cappotto camminando, urtò col braccio la testa del cameriere, rise e uscì. - Alle cinque e in finanziera, per favore! - gridò ancora, tornando verso la porta. 

\capitolo{IX}\label{ix-3} 

Erano le cinque passate e già alcuni ospiti cominciavano ad arrivare, quando giunse anche il padrone di casa. Entrò insieme con Sergej Ivanovic Koznyšev e Pescov che si erano trovati nello stesso momento sul pianerottolo. Erano costoro i due principali esponenti dell'intelligencija moscovita così come li aveva definiti Oblonskij. Erano persone stimabili per carattere e per ingegno. Si stimavano reciprocamente, ma erano quasi in tutto completamente e irrimediabilmente discordi fra loro, non perché avessero tendenze opposte, ma proprio perché erano d'uno stesso partito (i nemici li fondevano in uno), e in questo partito ciascuno aveva la propria sfumatura. E poiché è molto difficile metter d'accordo dissensi lievi, indefiniti, essi non solo non concordavano mai nelle opinioni, ma erano da tempo abituati, senza irritarsi, a irridere l'uno l'incorreggibile aberrazione dell'altro. 

Stavano per entrare, conversando del tempo, quando li raggiunse Stepan Arkad'ic. Nel salotto sedevano già il principe Aleksandr Dmitrievic, suocero di Oblonskij, il giovane Šcerbackij, Turovcyn, Kitty e Karenin. 

Stepan Arkad'ic si accorse subito che nel salotto senza di lui, le cose non andavano bene. Dar'ja Aleksandrovna, in abito di gala di seta grigia, evidentemente preoccupata dei bambini che dovevano pranzare in camera loro, e del marito che non arrivava ancora, non aveva saputo fondere bene quella riunione. Stavano tutti seduti come figli di pope in visita (così diceva il vecchio principe), visibilmente perplessi come mai fossero capitati là, spremendo le parole per non star zitti. Il buon Turovcyn, si vedeva, non si sentiva a suo agio, e il sorriso delle grosse labbra col quale accolse Stepan Arkad'ic, diceva chiaramente: ``Ehi, amico, mi hai piantato qui con gli intelligentoni! Ecco, andare a bere sia pure allo Château des fleurs, questo sì che è affar mio!''. Il vecchio principe sedeva in silenzio, guardando di sbieco con i suoi occhi scintillanti Karenin, e Stepan Arkad'ic capì che egli aveva già pensato una qualche battuta di spirito per bollare quest'uomo di stato per il quale si imbandiva un banchetto come se fosse uno storione. Kitty guardava la porta, raccogliendo le proprie forze per non arrossire all'entrata di Konstantin Levin. Il giovane Šcerbackij, che non era stato presentato a Karenin cercava di mostrare che ciò non lo imbarazzava affatto. Karenin, secondo l'uso di Pietroburgo, a quel pranzo con signore, era in frac e cravatta bianca, e Stepan Arkad'ic capì dalla faccia di lui che era venuto soltanto perché aveva promesso, ma che, facendo atto di presenza in quella compagnia, compiva un dovere increscioso. Era proprio lui il principale responsabile di quella freddezza che aveva gelato tutti gli ospiti prima dell'arrivo di Stepan Arkad'ic. 

Entrando in salotto, Stepan Arkad'ic si scusò, spiegò che era stato trattenuto da quel tale principe che era l'eterno capro espiatorio di tutti i suoi ritardi e di tutte le sue assenze e, in un attimo, presentò tutti e, messi insieme Aleksej Aleksandrovic con Sergej Koznyšev, lanciò loro il tema della russificazione della Polonia al quale essi si aggrapparono subito insieme a Pescov. Battendo sulla spalla di Turovcyn gli sussurrò qualcosa di ameno e lo mise a sedere accanto a sua moglie e al principe. Poi disse a Kitty che quel giorno era molto carina, e presentò Šcerbackij a Karenin. In un momento maneggiò tutta quella pasta sociale in modo tale che il salotto cominciò ad andare a gonfie vele, e le voci cominciarono a risonare animate. Mancava soltanto Konstantin Levin. Ma fu per il meglio, perché, andando in sala da pranzo Stepan Arkad'ic si accorse con terrore che il Porto e lo Xeres erano stati presi da Deprès e non da Levais, e, dato ordine di mandare al più presto il cocchiere da Levais, si diresse di nuovo nel salotto. 

In sala da pranzo si incontrò con Konstantin Levin. 

- Non sono in ritardo? 

- Puoi forse arrivare in ritardo tu? - disse Stepan Arkad'ic, prendendolo sotto il braccio. 

- Hai molta gente? E chi mai? - chiese Levin, involontariamente arrossendo mentre faceva cadere col guanto la neve dal berretto. 

- Tutti dei nostri. Kitty è qui. Andiamo dunque: ti farò fare la conoscenza di Karenin. 

Stepan Arkad'ic, malgrado il suo liberalismo sapeva che la conoscenza di Karenin non poteva non essere lusinghiera e perciò la offriva ai suoi amici migliori. Ma in quel momento Konstantin Levin non era in grado di provare tutto il piacere di questa conoscenza. Egli non aveva più visto Kitty da quella serata per lui memorabile in cui aveva incontrato Vronskij, a non voler tener conto dell'attimo in cui l'aveva intraveduta sulla strada maestra. In fondo all'anima sapeva che l'avrebbe vista quel giorno. Ma, per voler tener la mente libera si sforzava di convincersi che non lo sapeva. Adesso invece, nel sentire ch'ella era là, provò, a un tratto, una tale gioia e nello stesso tempo una tale agitazione, che gli si mozzò in gola il respiro e non poté dire quello che voleva. 

``Come, come sarà? Così com'era prima, o com'era nella carrozza? Che succederà se Dar'ja Aleksandrovna ha detto il vero? E perché poi non potrebbe esser vero?'' pensava. 

- Ah, ti prego, fammi fare la conoscenza di Karenin - rispose a stento e, con passo disperatamente deciso, entrò nel salotto e la vide. 

Ella non era più quella di prima, né com'era nella carrozza; era del tutto un'altra. 

Era spaurita, timida, confusa, e, per questo, ancora più incantevole. Lo scorse nell'attimo stesso in cui egli entrò nella stanza. Lo aspettava. Si rallegrò e si confuse della sua gioia a un punto tale che vi fu un momento, proprio quando egli si avvicinò alla padrona di casa e la guardò di nuovo, che a lei e a lui e a Dolly, che vedeva tutto, parve ch'ella non avrebbe resistito e che sarebbe scoppiata a piangere. Arrossì, impallidì, arrossì di nuovo e restò sospesa, con le labbra che le tremavano appena, aspettandolo. Egli le si avvicinò, s'inchinò e tese la mano in silenzio. Se non fosse stato il lieve tremito delle labbra e l'umore che le velava gli occhi e ne aumentava lo splendore, il suo sorriso sarebbe stato quasi calmo, quando disse: 

- Da quanto tempo non ci vediamo! - e con disperata risolutezza strinse con la sua mano fredda la mano di lui. 

- Voi non mi avete visto, ma io sì che vi ho visto - disse Levin, splendido d'un sorriso di gioia. - Vi ho veduto quando dalla ferrovia andavate a Ergušovo. 

- Quando? - ella chiese con sorpresa. 

- Andavate a Ergušovo - diceva Levin, sentendo di soffocar per la gioia che gli invadeva l'anima. ``Come ho osato di associare il pensiero di qualcosa d'impuro con quest'essere commovente? E già, pare che sia vero quello che diceva Dar'ja Aleksandrovna'' pensava. 

Stepan Arkad'ic lo prese per un braccio e lo portò verso Karenin. 

- Permettete che vi presenti. - E disse i loro nomi. 

- Molto piacere di incontrarvi di nuovo - disse freddo Aleksej Aleksandrovic, stringendo la mano di Levin. 

- Vi conoscete? - chiese Stepan Arkad'ic con stupore. 

- Abbiamo passato tre ore insieme in treno - disse Levin, sorridendo - ma ne siamo usciti, come da un ballo in maschera, con la curiosità, io almeno. 

- Ecco, signori, accomodatevi - disse Stepan Arkad'ic, facendo segno verso la sala da pranzo. 

Gli uomini andarono in sala da pranzo e si accostarono alla tavola con gli antipasti, coperta di sei qualità di vodka e di altrettante specie di formaggi con le palettine d'argento o senza, i caviali, le aringhe, le conserve di varie specie e i piatti con le fettine di pane francese. 

Gli uomini stavano in piedi presso le vodke profumate e gli antipasti, e la conversazione sulla russificazione della Polonia fra Sergej Ivanovic Koznyšev, Karenin e Pescov tacque in attesa del pranzo. 

Sergej Ivanovic che, come nessun altro, sapeva chiudere la discussione più astratta e seria spargendo inaspettatamente un po' di sale attico e cambiando così l'umore degli interlocutori, lo fece anche adesso. 

Aleksej Aleksandrovic aveva dimostrato che la russificazione della Polonia poteva compiersi solo in virtù degli alti principi che dovevano essere introdotti dall'amministrazione russa. 

Pescov aveva insistito sul fatto che un popolo assimila un altro popolo solo quando ha una popolazione più densa. Koznyšev aveva annuito all'una ed all'altra idea ma con riserve. E quando uscirono dal salotto, per concludere la conversazione, Koznyšev disse sorridendo: 

- Perciò per la russificazione degli allogeni vi è un mezzo solo: far nascere quanti più bambini è possibile. Ecco, io e mio fratello agiamo nel modo peggiore. Ma voi, signori uomini ammogliati, e in particolare voi Stepan Arkad'ic, agite del tutto patriotticamente. Quanti ne avete? - e si voltò affettuosamente sorridendo al padrone di casa, tendendogli un minuscolo bicchierino. 

Tutti risero, e in modo particolarmente allegro Stepan Arkad'ic. 

- Sì, ecco il mezzo migliore! - egli disse, masticando bene il formaggio e versando una certa vodka di una speciale qualità nel bicchierino teso. La conversazione era realmente finita nello scherzo. 

- Questo formaggio non è cattivo. Ne volete? - diceva il padrone di casa. - Possibile che tu ti sia dato di nuovo alla ginnastica? - disse rivolto a Levin, tastando con la sinistra i suoi muscoli. Levin sorrise, tese il braccio e, sotto le dita di Stepan Arkad'ic, come un rotondo formaggio, si sollevò, al di sotto del panno sottile della finanziera, una massa di acciaio. 

- Che bicipite! Un vero Sansone! 

- Credo che occorra molta forza per la caccia all'orso - disse Aleksej Aleksandrovic che aveva le idee più nebulose sulla caccia, mentre stendeva il formaggio e ne copriva, come una ragnatela, la midolla del pane. 

Levin sorrise. 

- Nessuna. Al contrario, anche un bambino può uccidere un orso - disse, facendosi da parte con un lieve inchino alle signore che, insieme con la padrona di casa, si avvicinavano al tavolo degli antipasti. 

- E voi avete ucciso un orso, mi han detto - disse Kitty, cercando invano di afferrare con la forchetta un fungo bizzarro che scivolava via, e scotendo le trine attraverso le quali biancheggiava il suo braccio. - Ci sono forse degli orsi da voi? - aggiunse, volgendo verso di lui a metà la deliziosa testolina e sorridendo. 

Sembrava che non ci fosse nulla di straordinario in quello ch'ella diceva, ma quale significato impossibile a dirsi c'era per lui in ogni suono, in ogni muover delle labbra, degli occhi, delle mani, mentre ella diceva queste parole! C'era l'implorazione del perdono e la fiducia in lui, e una carezza, una lieve, timida carezza, e una promessa e una speranza, e l'amore per lui al quale egli non poteva non credere e che lo soffocava di gioia. 

- Siamo andati nel governatorato di Tver'. Ritornando di là mi sono incontrato col vostro beau-frère, ossia col cognato del vostro beau-frère - diss'egli con un sorriso. - È stato un incontro buffo. 

E allegramente e con brio raccontò come, non avendo dormito tutta la notte, avesse fatto irruzione nello scompartimento di Aleksej Aleksandrovic con indosso un pellicciotto. 

- Il conduttore, giudicandomi, contrariamente al proverbio, dall'abito, mi voleva cacciar fuori; ma allora io ho cominciato ad esprimermi in uno stile elevato, e\ldots{} voi pure - disse rivolto a Karenin, avendone dimenticato il nome - in principio, a causa di quel pellicciotto, volevate cacciarmi via, ma poi avete preso le mie difese, del che vi sono molto grato! 

- In genere sono assai poco definiti i diritti dei passeggeri sulla scelta del posto - disse Aleksej Aleksandrovic, asciugando col fazzoletto la punta delle dita. 

- Vedevo che eravate dubbioso sul mio conto - disse Levin, sorridendo cordialmente - ma mi sono affrettato a cominciare un discorso intelligente per riparare al mio pellicciotto. 

Sergej Ivanovic, continuando una conversazione con la padrona di casa e ascoltando con un orecchio il fratello, lo guardò di sbieco. ``Che gli succede oggi? Sembra un trionfatore!'' pensò. Non sapeva che Levin sentiva che gli eran spuntate le ali. Levin sapeva che ella ascoltava le sue parole e che le piaceva udirle. E quest'unica cosa l'interessava. Non solo in quella stanza, ma in tutto il mondo, esistevano ormai soltanto lui, che aveva acquistato di fronte a se stesso e a lei un enorme significato e importanza, e lei. Si sentiva trasportato ad un'altezza che gli faceva girar la testa, e là, chi sa dove, ma in basso e lontano, c'erano tutti quei buoni e bravi Karenin, Oblonskij e il mondo intero. 

Senza che nessuno se n'avvedesse, senza guardarli e come se ormai non ci fosse più posto per metterli a sedere, Stepan Arkad'ic fece sedere vicini Levin e Kitty. 

- Via, tu siediti magari qui - disse a Levin. 

Il pranzo fu altrettanto raffinato quanto il vasellame di cui Stepan Arkad'ic era un appassionato. La minestra à la Marie Louise era riuscita ottima; i minuscoli sfogliantini che si scioglievano in bocca erano irreprensibili. Due camerieri e Matvej, in cravatta bianca, facevano l'ufficio loro con le pietanze e coi vini senza farsi notare, accorti e abili. Il pranzo dal lato materiale riuscì benissimo; non meno bene riuscì dal lato spirituale. La conversazione, ora generale, ora a gruppi, non languì mai e, verso la fine del pranzo, si avvivò tanto che gli uomini si alzarono senza cessare di parlare, e persino Aleksej Aleksandrovic si era animato. 

\capitolo{X}\label{x-3} 

A Pescov piaceva toccare il fondo delle questioni e non si acquietò alle parole di Sergej Ivanovic, tanto più che egli sentiva l'infondatezza dell'opinione di lui. 

- Io non ho mai inteso parlare - egli disse, mentre si consumava la minestra, rivolto ad Aleksej Aleksandrovic - della sola densità di popolazione, bensì di questa in concomitanza con la struttura sociale e non con i sistemi. 

- Mi sembra - rispondeva senza fretta e senza voglia Aleksej Aleksandrovic - che sia la stessa cosa. Secondo me, può influire su di un altro popolo solo quello che ha un più alto grado di progresso, che\ldots{} 

- Ma proprio qui sta la questione - interruppe con la sua voce di basso Pescov che si affrettava sempre a interloquire e sembrava metter tutta l'anima in quello che diceva - che significa un più alto grado di progresso? Gli inglesi, i francesi, i tedeschi, chi di questi è al più alto grado di sviluppo? Chi assimilerà l'altro? Noi vediamo che il Reno è francesizzato, eppure i tedeschi non sono da meno dei francesi - egli gridò. - Qui ci deve essere un'altra legge. 

- Mi sembra che il predominio stia sempre dalla parte della vera cultura - disse Aleksej Aleksandrovic, alzando lievemente le sopracciglia. 

- Ma in che cosa consisteranno i segni di una vera cultura? - disse Pescov. 

- Io suppongo che questi segni siano noti - disse Aleksej Aleksandrovic. 

- Pienamente noti? - s'introdusse con un sottile sorriso Sergej Ivanovic. - Ora io ritengo che la cultura vera può essere soltanto quella classica; ma vediamo quanto sono aspre le dispute dall'una e dall'altra parte, e non possiamo negare che anche il campo avversario ha forti argomenti a proprio favore. 

- Voi siete un classico, Sergej Ivanovic. Bevete del rosso? - disse Stepan Arkad'ic. 

- Io non esprimo la mia opinione su questa o quella cultura - disse Sergej Ivanovic, tendendo il proprio bicchiere, con un sorriso di condiscendenza, come verso un bambino - io dico solo che entrambe le parti dispongono di validi argomenti - continuò, rivolgendosi ad Aleksej Aleksandrovic. - Io, per cultura, sono un classico, ma in questa questione non posso avere un'opinione certa. Io non vedo argomenti chiari per cui alla cultura classica si debba dare la preferenza di fronte alle scienze positive. 

- Quelle naturali hanno un'influenza pedagogica-formativa - replicò Pescov. - Prendete l'astronomia, prendete la botanica, la zoologia con il suo sistema di leggi. 

- Non posso concordare in pieno su ciò - rispose Aleksej Aleksandrovic; - mi sembra che non si possa non riconoscere che lo stesso processo delle forme filologiche agisca in modo particolarmente benefico sullo sviluppo spirituale. Oltre a ciò non si può negare che l'influenza degli scrittori classici sia in sommo grado morale, mentre, per disgrazia, all'insegnamento delle scienze naturali sono collegate quelle false, nocive dottrine che formano la piaga del nostro tempo. 

Sergej Ivanovic voleva dire qualcosa, ma Pescov, con la sua grossa voce di basso, lo interruppe. Egli cominciò a dimostrare l'infondatezza di quella opinione. Sergej Ivanovic tranquillamente aspettava il proprio turno e aveva evidentemente pronta una vittoriosa obiezione. 

- Ma - disse Sergej Ivanovic, sorridendo con finezza e rivolto a Karenin - non si può disconoscere che ponderare esattamente tutti i vantaggi e gli svantaggi delle une e delle altre scienze sia cosa difficile, e che la questione su quali debbano preferirsi non andrebbe decisa così alla svelta e in maniera definitiva, se dalla parte della cultura classica non ci fosse quel vantaggio che avete enunciato poc'anzi; l'influenza morale, disons le mot, anti-nichilista. 

- Senza dubbio. 

- Se non ci fosse questo vantaggio dell'influenza anti-nichilista da parte della cultura classica, noi avremmo pensato ancora, avremmo ponderato le argomentazioni di tutte e due le correnti - diceva Sergej Ivanovic con un sorriso sottile - avremmo fatto largo all'una e all'altra tendenza. Ma adesso sappiamo che in queste pillole di cultura classica sta la salutare forza dell'anti-nichilismo, e noi coraggiosamente le propiniamo ai nostri pazienti\ldots{} E che fare quando non ci sarà più neppure questo sale benefico? - concluse, con un pizzico di sale attico. 

Le pillole di Sergej Ivanovic provocarono il riso di tutti, e specialmente di Tuškevic che, ascoltando la conversazione, aveva atteso che arrivasse il momento spiritoso. 

Stepan Arkad'ic non s'era sbagliato a invitare Pescov. Con Pescov una conversazione intelligente non poteva venire meno neppure un istante. Sergej Ivanovic aveva appena chiuso con lo scherzo quella discussione che Pescov subito ne sollevò un'altra. 

- Nemmeno si può dire - disse - che il governo abbia questo scopo. Il governo sembra lasciarsi guidare da considerazioni di ordine generale, rimanendo poi indifferente agli effetti che possono avere le misure adottate. Per esempio, la questione dell'istruzione femminile dovrebbe essere considerata dannosa, ma il governo apre corsi e università femminili. 

E la conversazione subito si diresse verso il nuovo tema dell'istruzione femminile. 

Aleksej Aleksandrovic espresse il pensiero che di solito l'istruzione delle donne si confonde con la questione della libertà delle donne, e soltanto per questo può essere considerata dannosa. 

- Io, al contrario, considero le due questioni indissolubilmente legate - disse Pescov. - È un circolo vizioso. La donna è priva di diritti per insufficienza di istruzione, e l'insufficienza di istruzione deriva dalla mancanza di diritti. Non bisogna dimenticare che l'asservimento delle donne è così grande e inveterato che noi spesso non vogliamo renderci conto dell'abisso che le divide da noi - egli disse. 

- Voi avete detto ``diritti'' - disse Sergej Ivanovic, dopo aver atteso una pausa di Pescov - diritto di occupare le cariche di giurato, di consigliere, di presidente del tribunale; diritto d'impiegato, di membro del parlamento\ldots{} 

- Senza dubbio. 

- Ma se le donne, sia pure come rara eccezione, possono occupare questi posti, mi pare che abbiate usato in modo non proprio l'espressione ``diritti''. Meglio avreste detto: doveri. Ognuno sarà d'accordo che, coprendo una qualche carica di giurato, di consigliere, d'impiegato telegrafico, sentiamo di adempiere un dovere. E perciò è più giusto dire che le donne cercano dei doveri, e del tutto legittimamente. E non si può non simpatizzare verso questo loro desiderio di aiutare il comune lavoro maschile. 

- Perfettamente giusto - affermò Aleksej Aleksandrovic. - La questione, io credo, consiste solo nel vedere se esse sono adatte a compiere quei doveri. 

- Probabilmente ve ne saranno molte adatte - intervenne Stepan Arkad'ic quando l'istruzione sarà diffusa fra di loro. Lo vediamo\ldots{} 

- E il proverbio? - disse il principe che, da tempo, andava prestando orecchio alla conversazione facendo brillare i suoi piccoli occhi canzonatori - anche in presenza delle figliuole si può dire: ``Capelli lunghi\ldots{}''. 

- Pensavamo proprio così dei negri fino alla loro liberazione - disse Pescov urtato. 

- Io trovo strano soltanto questo: che le donne vadano in cerca di nuovi doveri - disse Sergej Ivanovic - quando, per disgrazia nostra, vediamo che gli uomini fanno di tutto per eluderli. 

- I doveri sono congiunti ai diritti; il potere, il denaro, gli onori; è questo che cercano le donne - disse Pescov. 

- Sarebbe lo stesso se io pretendessi il diritto di fare la balia, e mi rammaricassi che le donne fossero pagate e che a me non volessero dar la paga - disse il vecchio principe. 

Tuškevic scoppiò a ridere forte, e a Sergej Ivanovic spiacque di non aver detto lui quella battuta. Perfino Aleksej Aleksandrovic sorrise. 

- Già, ma l'uomo non può allattare - disse Pescov - mentre la donna\ldots{} 

- No, un inglese ha allattato, su di una nave il suo bambino - disse il vecchio principe, permettendosi tanta libertà di linguaggio alla presenza delle figliuole. 

- Quanti inglesi ci sono di questa specie, tante donne funzionario ci saranno - disse Sergej Ivanovic. 

- Sì, ma che deve fare una ragazza che non abbia famiglia? - s'intromise Stepan Arkad'ic, pensando alla cibisova che aveva sempre davanti agli occhi, simpatizzando con Pescov e sostenendolo. 

- Se esaminate bene la storia di una tale ragazza, troverete che essa ha abbandonato la famiglia propria o quella della sorella dove avrebbe potuto avere un lavoro femminile - disse con irritazione Dar'ja Aleksandrovna, entrando improvvisamente nella conversazione e indovinando probabilmente quale ragazza avesse presente Stepan Arkad'ic . 

- Ma noi combattiamo per un principio, per un'idea! - ribatteva Pescov con la sua voce sonora di basso. - La donna può avere il diritto di essere indipendente, colta. Ella è impacciata, oppressa dalla consapevolezza dell'impossibilità di esserlo. 

- E io sono impacciato e oppresso dal fatto che non mi prenderanno come balia nell'ospizio dei trovatelli! - disse di nuovo il vecchio principe con grande gioia di Tuškevic che per ridere lasciò cadere nella salsa, dalla parte grossa, un asparagio. 

\capitolo{XI}\label{xi-3} 

Tutti prendevano parte alla conversazione generale, tranne Kitty e Levin. 

In principio, quando si parlava dell'influenza che un popolo può avere su di un altro, a Levin veniva involontariamente in mente quello che avrebbe potuto dire in proposito, ma questi pensieri, prima molto importanti per lui, non avevano ora il più piccolo interesse. Gli pareva persino strano come mai cercassero di parlare tanto di cose di cui nessuno aveva bisogno. Proprio allo stesso modo, a Kitty sembrava che dovesse essere interessante quello che dicevano a proposito dei diritti e dell'istruzione delle donne. Quante volte ella aveva pensato a questo, a proposito della sua amica Varen'ka, del suo penoso stato di dipendenza, quante volte aveva pensato fra di sé cosa sarebbe avvenuto di lei stessa, se non si fosse maritata, e quante volte aveva discusso di questo con la sorella! Ma ora tutto ciò non la interessava per nulla. Fra lei e Levin si era avviata una conversazione a parte, anzi non una conversazione, ma una certa misteriosa comunione che a ogni minuto li legava più da vicino e suscitava in entrambi un sentimento di gioioso spavento innanzi all'ignoto nel quale entravano. 

Alla domanda di Kitty come avesse potuto scorgerla in carrozza l'anno prima, Levin aveva raccontato che tornava dalla falciatura per la strada maestra e che l'aveva incontrata. 

- Era la prima mattina. Voi probabilmente vi eravate appena svegliata. Vostra maman sonnecchiava in un angolino. Era un mattino meraviglioso. Io cammino e penso: chi sarà mai nella carrozza dal tiro a quattro? Una bella quadriglia coi bubboli, ed ecco per un attimo mi siete balenata voi, e attraverso il finestrino vedo che siete seduta così, e con tutte e due le mani tenete i nastri della cuffietta e pensate intensamente a qualcosa - diceva sorridendo. - Come vorrei sapere a che pensavate allora! Era una cosa importante? 

``Non ero forse spettinata?'' ella pensò; ma visto il sorriso entusiastico che suscitavano nel ricordo di lui questi particolari, sentì che, al contrario, l'impressione da lei prodotta era stata ottima. Arrossì e rise felice. 

- Davvero, non ricordo. 

- Come ride contento Tuškevic! - disse Levin, compiacendosi del fatto che avesse le lacrime agli occhi e il corpo sobbalzante dal ridere. 

- È molto che lo conoscete? - chiese Kitty. 

- Chi non lo conosce! 

- Ma mi pare che lo riteniate un uomo cattivo. 

- Cattivo no, insignificante. 

- E non è vero! E smettete subito di giudicarlo così! - disse Kitty. - Anch'io avevo una cattiva opinione di lui; è invece gentilissimo, straordinariamente buono. Ha un cuore d'oro. 

- Com'è che avete potuto conoscere il suo cuore? 

- Noi siamo stati grandi amici. Lo conosco molto bene. L'inverno scorso, subito dopo\ldots{} che eravate stato da noi - ella disse con un sorriso colpevole e fiducioso insieme - Dolly aveva tutti i bambini con la scarlattina e lui passò da lei per caso. E figuratevi - ella diceva sottovoce - gliene venne tanta pena che si fermò, e cominciò ad aiutarla a curar bambini come una balia. Sai, racconto a Konstantin Dmitric quel che ha fatto Tuškevic durante la scarlattina - disse chinandosi verso la sorella. 

- Sì, meraviglioso, un tesoro! - disse Dolly guardando Tuškevic, il quale, accortosi che si parlava di lui, sorrise affabile. Levin guardò ancora una volta Tuškevic e si meravigliò di non aver capito prima tutto il fascino di quell'uomo. 

- Sono colpevole, colpevole e non penserò mai più male della gente! - disse allegro, esprimendo con sincerità ciò che in quel momento sentiva. 

\capitolo{XII}\label{xii-3} 

Nella conversazione avviatasi sui diritti delle donne vi erano argomenti scabrosi a trattarsi in presenza delle signore, sulla ineguaglianza dei diritti nel matrimonio. Pescov, durante il pranzo, era incappato diverse volte in tali argomenti; ma Sergej Ivanovic e Stepan Arkad'ic lo avevano prudentemente distolto. 

Quando poi si furono alzati da tavola e le signore uscirono, Pescov, senza seguirle, si rivolse ad Aleksej Aleksandrovic e si mise ad esporre la ragione principale di tale ineguaglianza. L'ineguaglianza dei coniugi, secondo lui, consisteva nel fatto che l'infedeltà della moglie e l'infedeltà del marito erano punite in modo ineguale e dalla legge e dall'opinione pubblica. 

Stepan Arkad'ic si accostò in fretta ad Aleksej Aleksandrovic, offrendogli da fumare. 

- No, non fumo - rispose con calma Aleksej Aleksandrovic e, quasi volesse dimostrare di proposito che non temeva un tale discorso, si voltò con un sorriso freddo a Pescov. 

- Suppongo che le basi di questo modo di vedere siano nell'essenza delle cose - egli disse cercando di passare in salotto; ma qui a un tratto cominciò inaspettatamente a parlare Tuškevic, rivolto ad Aleksej Aleksandrovic. 

- E avete sentito di Priacnikov? - disse Tuškevic, eccitato dallo champagne bevuto e ansioso di rompere da tempo il silenzio che gli pesava. - Vasja Priacnikov - disse con quel buon sorriso tra le labbra umide e rosse, rivolgendosi di preferenza all'ospite più importante, Aleksej Aleksandrovic - m'hanno raccontato oggi s'è battuto in duello a Tver' con Kvytskij e l'ha ucciso. 

Come sempre pare che la lingua batte dove il dente duole, anche ora Stepan Arkad'ic sentiva che, disgraziatamente, quel giorno il discorso cadeva ogni momento sul punto dolente di Aleksej Aleksandrovic. Voleva di nuovo allontanare il cognato, ma lo stesso Aleksej Aleksandrovic chiese con curiosità: 

- Perché si é battuto Priacnikov? 

- Per sua moglie. È stato bravo. L'ha sfidato e l'ha ucciso. 

- Ah! - disse con indifferenza Aleksej Aleksandrovic e, sollevate le sopracciglia, passò in salotto. 

- Come son contenta che siate venuto! - gli disse Dolly con un sorriso apprensivo, venendogli incontro nel salotto di passaggio. - Devo parlare un po' con voi. Sediamoci qui. 

Sempre con quella espressione di indifferenza che gli davano le sopracciglia sollevate, Aleksej Aleksandrovic sedette accanto a Dar'ja Aleksandrovna, fingendo di sorridere. 

- Tanto più - egli disse - che volevo chiedere il vostro permesso per congedarmi subito. Devo partire domani. 

Dar'ja Aleksandrovna era pienamente convinta dell'innocenza di Anna e sentiva che impallidiva e che le labbra le tremavano di rabbia verso quell'uomo freddo e insensibile che così pacatamente si accingeva a distruggere la sua innocente amica. 

- Aleksej Aleksandrovic - ella disse con una decisione disperata, guardandolo negli occhi. - Vi ho chiesto di Anna, voi non mi avete risposto. Cosa ne è di lei? 

- Sta bene, mi pare, Dar'ja Aleksandrovna - rispose Aleksej Aleksandrovic senza guardarla. 

- Aleksej Aleksandrovic, perdonatemi\ldots{} io non ho il diritto\ldots{} ma amo e stimo Anna come una sorella, vi chiedo, vi supplico di dirmi: che c'è fra voi? di che cosa l'accusate? 

Aleksej Aleksandrovic si accigliò e, quasi chiudendo gli occhi, abbassò la testa. 

- Suppongo che vostro marito vi abbia riferito le ragioni per le quali considero necessario cambiare i miei precedenti rapporti con Anna Arkad'evna - egli disse, senza guardarla negli occhi, e guardando contrariato Šcerbackij che passava per il salotto. 

- Io non credo, non credo, non posso credere a questo! - esclamò Dolly con un gesto energico, stringendo davanti a sé le mani ossute. Si alzò in fretta e poggiò la mano sulla manica di Aleksej Aleksandrovic. - Qui ci disturbano, passiamo di qua, vi prego. 

L'agitazione di Dolly agiva su Aleksej Aleksandrovic. Egli si alzò e la seguì umilmente nella stanza di studio dei ragazzi. Sedettero a una tavola ricoperta di tela cerata tagliuzzata dai temperini. 

- Io non credo, non credo a questo! - esclamò Dolly cercando di afferrare lo sguardo di lui che la sfuggiva. 

- Non si può non credere ai fatti, Dar'ja Aleksandrovna - egli disse, accentuando la parola ``fatti''. 

- Ma che ha fatto mai? - disse Dar'ja Aleksandrovna. - Che cosa ha fatto precisamente? 

- Ha violato i suoi doveri e ha tradito suo marito. Ecco quello che ha fatto - egli disse. 

- No, no, non può essere! No, per amor di Dio, voi vi sbagliate! - diceva Dolly, toccandosi le tempie con le mani e chiudendo gli occhi. 

Aleksej Aleksandrovic sorrise freddamente, desiderando mostrare a lei e a se stesso la fermezza della propria convinzione; ma quella calorosa difesa, pur senza farlo tentennare, gli inaspriva la ferita. Egli prese a parlare con grande animazione. 

- È assai difficile sbagliarsi quando la moglie stessa confessa tutto al marito e gli dice che otto anni di vita in comune e un figlio non contano, che tutto questo è uno sbaglio e che vuol vivere daccapo - disse lui furioso, aspirando forte col naso. 

- Anna e la colpa\ldots{} non riesco ad associarli, non ci posso credere. 

- Dar'ja Aleksandrovna!- egli disse guardando ora diritto nel buon viso agitato di Dolly e sentendo che la lingua involontariamente gli si scioglieva. - Io darei molto perché un dubbio fosse ancora possibile. Quando dubitavo, c'era ancora la speranza; ma ora non c'è speranza, e tuttavia io dubito di tutto. Dubito a tal punto di tutto, che odio mio figlio e a volte non credo che sia figlio mio. Sono molto infelice. 

Non gli era necessario dir questo: Dar'ja Aleksandrovna lo aveva capito subito, appena egli l'aveva guardata in viso; ed ebbe pietà di lui e la fede nell'innocenza della sua amica vacillò. 

- Ah, è terribile, terribile! Ma è possibile che sia vero, che vi siate deciso a divorziare? 

- Son ricorso all'ultima misura. Non ho più nulla da fare. 

- Niente da fare, niente da fare\ldots{} - ripeteva lei con le lacrime agli occhi. - No, non niente da fare! - disse. 

- Appunto questo è orribile in questa specie di dolore, che non si può, come in un qualsiasi altro, in una perdita, in una morte, portare la croce; qui bisogna agire - egli disse quasi indovinando il pensiero di lei. - Bisogna uscire dalla situazione umiliante in cui siete stato posto: non si può vivere in tre. 

- Capisco, capisco bene - disse Dolly, e abbassò il capo. Taceva pensando a se stessa e al proprio dolore familiare e improvvisamente, con un gesto energico, alzò il capo e con fare supplichevole giunse le mani: - Ma, aspettate! Voi siete cristiano. Pensate a lei! Che ne sarà di lei, se l'abbandonate? 

- Ci ho pensato, Dar'ja Aleksandrovna, e ci ho pensato molto - disse Aleksej Aleksandrovic. Il suo viso era arrossato a chiazze e gli occhi appannati guardavano dritto in lei. Dar'ja Aleksandrovna ora lo compativa con tutta l'anima. - Ho fatto questo, dopo che mi fu annunciato da lei stessa il mio disonore: ho lasciato tutto come prima. Ho dato la possibilità di una resipiscenza, ho cercato di salvarla. Ebbene? Lei non ha adempiuto la mia più piccola pretesa, il rispetto delle convenienze - egli disse, accalorandosi. - Si può salvare un essere che non si vuole perdere; ma se una natura è tutta così corrotta, pervertita, che la stessa rovina le sembra una salvezza, che fare mai? 

- Tutto, tranne il divorzio! - rispose Dar'ja Aleksandrovna. 

- Ma cosa tutto? 

- No, è orribile. Ella non sarà la moglie di nessuno, si perderà. 

- Che posso farci? - disse Aleksej Aleksandrovic, dopo aver alzato le spalle e le sopracciglia. Il ricordo dell'ultima azione della moglie lo irritò talmente che tornò freddo come al principio della conversazione. - Vi ringrazio molto della vostra simpatia, ma devo andare - disse, alzandosi. 

- No, aspettate. Voi non dovete rovinarla. Aspettate, vi dirò di me. Mi sono sposata e mio marito mi ha ingannato, nell'ira della gelosia volevo abbandonare tutto, volevo io stessa\ldots{} Ma poi sono tornata in me. Chi mai? Anna mi ha salvato. Ed ecco, io vivo. I bambini crescono, mio marito ritorna in famiglia e sente il torto suo e diventa sempre migliore, e io vivo\ldots{} Io ho perdonato, anche voi dovete perdonare! 

Aleksej Aleksandrovic ascoltava, ma le parole di lei non agivano più. Nella sua anima erompeva di nuovo tutto il rancore di quel giorno in cui aveva deciso il divorzio. Si scosse e cominciò a parlare con voce penetrante, forte: 

- Perdonare non posso e non voglio, e lo considero ingiusto. Per questa donna ho fatto tutto, e lei ha calpestato tutto nel fango che le è proprio. Io non sono un uomo cattivo, io non ho mai odiato nessuno, ma odio lei con tutte le forze dell'anima mia e non posso perdonare perché troppo la odio per tutto il male che mi ha fatto! - disse con lacrime di rancore nella voce. 

- Amate chi vi odia - mormorò timidamente Dar'ja Aleksandrovna. 

Aleksej Aleksandrovic rise sprezzante. Questo lo sapeva da tempo, ma questo non poteva essere applicato al suo caso. 

- Amate chi vi odia, ma amare chi si odia non è possibile. Perdonatemi d'avervi sconvolta. A ciascuno il suo dolore! - E, ritornato padrone di sé, Aleksej Aleksandrovic salutò tranquillo e andò via. 

\capitolo{XIII}\label{xiii-3} 

Quando si erano alzati da tavola, Levin avrebbe voluto seguire Kitty; ma temeva di spiacerle con una corte troppo evidente. Rimase nella cerchia degli uomini, prendendo parte alla conversazione generale e, senza guardare Kitty, ne percepiva i movimenti, gli sguardi e il posto dove era in salotto. 

Subito, e senza il più piccolo sforzo, eseguiva la promessa che le aveva fatto, di pensar sempre bene di tutti e di voler bene a tutti. La conversazione volgeva sulla ``comunità'' nella quale Pescov vedeva un certo principio particolare, da lui detto principio corale. Levin non era d'accordo né con Pescov né col fratellastro che, in un certo modo, a modo uso, riconosceva e non riconosceva il valore della comunità russa. Ma parlava con loro cercando solo di metterli d'accordo e di smussare le loro obiezioni. Non si interessava affatto a quello che egli stesso diceva, ancora meno a quello che dicevano loro, e desiderava una cosa sola, che tutti si trovassero bene e a loro agio. Adesso per lui una cosa sola era importante. E questa cosa sola dapprima era là nel salotto, ma poi si era mossa e si era fermata presso la porta. Senza voltarsi, egli ne sentiva lo sguardo fisso su di lui e il sorriso, e non poté non voltarsi. Ella stava dritta sulla porta con Šcerbackij e lo guardava. 

- Pensavo che andaste al piano - disse, avvicinandosi a lei. - Ecco quello che mi manca in campagna: la musica. 

- Noi siamo venuti solo per tirarvi fuori di qua - disse lei, ricompensandolo di un sorriso come di un dono - e vi ringrazio che siate venuto via. Che gusto c'è a discutere? Tanto nessuno mai convincerà l'altro. 

- Già, è vero - disse Levin - succede il più delle volte che si discute calorosamente perché non si riesce in alcun modo a capire che cosa voglia precisamente dimostrare l'avversario. 

Levin spesso aveva notato che, anche nelle discussioni fra le persone più intelligenti, dopo sforzi enormi, dopo un'enorme quantità di sottigliezze e di parole, coloro che discutono giungono alla fine a riconoscere che quello per cui a lungo si sono battuti per dimostrare l'uno all'altro, era loro noto da gran tempo, fin dall'inizio della conversazione, ma che piacevano loro cose diverse e non volevano nominare quello che piaceva per non essere contraddetti. Spesso aveva sperimentato che durante una discussione si capisce quello che piace al contraddittore, e improvvisamente quella stessa cosa comincia a piacere e allora tutti gli argomenti cadono come cosa inutile; talvolta aveva sperimentato il contrario: si espone finalmente quello che piace per cui si sono escogitati tutti gli argomenti, e se accade che si espongano bene e con sincerità, a un tratto l'avversario è d'accordo e smette di discutere. Proprio questo egli voleva dire. 

Ella corrugò la fronte, cercando di capire. Ma Levin aveva appena cominciato a spiegare che ella aveva già capito. 

- Capisco: bisogna sapere per che cosa egli discute, cosa gli piace, allora si può\ldots{} 

Ella aveva indovinato in pieno e aveva esposto il pensiero espresso male da lui. Levin sorrise con gioia; tanto era stato rapido il passaggio dall'oscura verbosa discussione di Pescov con suo fratello a questa laconica e chiara, quasi tacita comunicazione dei pensieri più complicati. 

Šcerbackij si era allontanato, e Kitty, avvicinandosi a un tavolo da giuoco che era lì aperto, sedette e, preso in mano un pezzo di gesso, cominciò a disegnare sul panno verde fiammante. 

Ripresero la conversazione che s'era tenuta a tavola, sulla libertà e sulle occupazioni delle donne. Levin era d'accordo con Dar'ja Aleksandrovna che una ragazza non sposata può trovare un lavoro femminile nella propria famiglia. Egli lo confermava col fatto che nessuna famiglia può fare a meno di un aiuto, che in ogni famiglia, ricca o povera, ci sono e ci devono essere della bambinaie salariate o delle parenti. 

- No - disse Kitty arrossendo, ma guardando tanto più coraggiosamente con i suoi occhi sinceri - una ragazza può essere posta in una condizione tale che non può, senza sentirsene umiliata, occuparsi in famiglia, e lei stessa\ldots{} 

Egli capì l'allusione. 

- Oh, sì - disse - sì, sì, avete ragione! 

E comprese il valore di tutto quello che aveva dimostrato a pranzo Pescov sulla libertà delle donne, per il solo fatto che vedeva nel cuore di Kitty il terrore dello stato nubile e della umiliazione e, poiché l'amava, aveva sentito quel terrore e quell'umiliazione in se stesso, e aveva rinunciato, d'un tratto, alle proprie argomentazioni. 

Sopravvenne un silenzio. Ella disegnava col gesso sul tappeto del tavolo. I suoi occhi brillavano d'una luce calma. Immedesimandosi nello stato d'animo di lei, egli sentiva in tutto il suo essere una tensione di felicità che diventava sempre più intensa. 

- Ah, ho disegnato tutto il tavolo! - disse lei e, deposto il gesso, fece per alzarsi. 

``E come farò a rimanere solo senza di lei?'' pensò Levin con terrore e prese il gesso. 

- Aspettate - disse, sedendo al tavolo. - Da tempo volevo chiedervi una cosa. 

Egli la guardava dritto negli occhi carezzevoli, sebbene spaventati. 

- Domandate, vi prego. 

- Ecco - egli disse, e scrisse le lettere iniziali: q, m, a, r: q, n, p, e, s, m, o, a? Quelle lettere volevano significare: ``quando mi avete risposto: questo non può essere, significava mai o allora?''. Non c'era nessuna probabilità che ella potesse decifrare questa frase complicata; ma egli la guardò con tanta ansia come se la sua vita dipendesse dall'aver ella capito o no quelle parole. 

Kitty lo guardò seria, poi poggiò la fronte corrugata sulla mano e cominciò a leggere. Di tanto in tanto dava un'occhiata a lui, domandandogli con lo sguardo: ``È quello che penso?''. 

- Ho capito - disse, arrossendo. 

- Che parola è questa? - egli disse, indicando l'm con cui era significata la parola ``mai''. 

- Questa parola significa ``mai'' - ella disse - ma non è vero! 

Egli cancellò in fretta quel che era scritto, le dette il gesso e si alzò. Ella scrisse: a, i, n, p, r, d. 

Dolly si consolò completamente del dolore arrecatogli dalla conversazione con Aleksej Aleksandrovic, quando sorprese queste due figure: Kitty col gesso in mano e con un sorriso timido che guardava di sotto in su Levin, e la bella figura di lui curva sul tavolo, con gli occhi ardenti, fissi ora sul tavolo ora su di lei. A un tratto egli s'illuminò tutto: aveva capito. La scritta significava: ``allora io non potevo rispondere diversamente''. 

Egli la guardò interrogativamente con timore. 

- Soltanto allora? 

- Sì - rispose il sorriso di lei. 

- E o\ldots{} ora? - egli domandò. 

- Ebbene, ecco leggete. Dirò quello che desidererei. Quello che desidererei tanto. - Ella scrisse le iniziali: ``c, p, d, e, p, q, c, e, s''. Questo significava: ``Che possiate dimenticare e perdonare quello che è stato''. 

Egli afferrò il gesso con le dita tese, tremanti, e, spezzatolo, scrisse le iniziali di quel che segue: ``Non ho nulla da dimenticare e perdonare, non ho mai cessato di amarvi''. 

Ella lo guardò con un sorriso che s'era fermato sul suo volto. 

- Ho capito - disse piano. 

Egli sedette e scrisse una lunga frase. Ella capì tutto e senza chiedere: ``È così?'' prese il gesso e rispose immediatamente. 

Egli per parecchio tempo non riuscì a capire quello ch'ella aveva scritto e la guardava spesso negli occhi. La mente gli si annebbiò di gioia. Non riusciva in nessun modo a sostituire alle lettere le parole ch'ella intendeva; ma negli occhi di lei, raggianti di felicità, capì tutto quello che doveva sapere. E scrisse tre lettere sole. Ma non aveva ancora finito di scriverle e lei già leggeva dietro il suo braccio e terminava lei stessa e scriveva la risposta: ``Sì''. 

- Giocate al secrétaire? - disse il vecchio principe accostandosi. - Su, però, andiamo, se vuoi arrivare in tempo a teatro. 

Levin si alzò e accompagnò Kitty alla porta. 

Nella loro conversazione era stato detto tutto; ch'ella lo amava e che avrebbe detto al padre e alla madre ch'egli sarebbe andato da loro l'indomani mattina. 

\capitolo{XIV}\label{xiv-3} 

Quando Kitty se ne fu andata e Levin rimase solo, egli sentì una tale inquietudine e un tale impaziente desiderio di giungere presto all'indomani mattina, al momento in cui l'avrebbe vista di nuovo e si sarebbe unito a lei per sempre, che ebbe paura, come della morte, di quelle quattordici ore da passare ancora lontano da lei. Gli era indispensabile rimanere a parlare con qualcuno, per non restare solo, per ingannare il tempo. Stepan Arkad'ic sarebbe stato per lui l'interlocutore preferito, ma se ne andava via, come diceva, a una serata, e in realtà, al balletto. Levin fece solo in tempo a dirgli che era felice, che gli voleva bene e che mai, mai avrebbe dimenticato quello ch'egli aveva fatto per lui. Lo sguardo e il sorriso di Stepan Arkad'ic dimostrarono a Levin ch'egli capiva in pieno questo sentimento. 

- Be', non è più ora di morire? - disse Stepan Arkad'ic, stringendo la mano di Levin con emozione. 

- N-n-no - disse Levin. 

Dar'ja Aleksandrovna, salutandolo, si congratulò in un certo modo, dicendo: 

- Sono proprio contenta che vi siate di nuovo incontrato con Kitty; si debbono tener care le vecchie amicizie. 

Ma a Levin non erano piaciute le parole di Dar'ja Aleksandrovna. Ella non poteva capire come tutto questo fosse elevato e inaccessibile a lei, e non doveva osare accennarvi. Levin si congedò da loro, ma, per non restar solo, si attaccò a suo fratello. 

- Dove vai? 

- A una seduta. 

- Su, vengo con te. Posso? 

- Perché no, andiamo - disse, sorridendo, Sergej Ivanovic. - Ma che hai oggi? 

- Che ho? La felicità! - disse Levin, abbassando il finestrino della carrozza in cui avevano preso posto. - Non ti fa niente? Altrimenti si soffoca. Ho la felicità. Perché non ti sei sposato? 

Sergej Ivanovic sorrise. 

- Sono molto contento, pare che sia una brava rag\ldots{} -cominciò Sergej Ivanovic. 

- Non parlare, non parlare, non parlare! - gridò Levin, afferrandolo con tutte e due le mani per il bavero della pelliccia e avviluppandovelo. ``È una brava ragazza'' erano parole così semplici, così comuni, così inadatte al suo sentimento! 

Sergej Ivanovic rise di un riso allegro come di rado gli accadeva. 

- Be', però, si può dire che ne sono molto contento? 

- Questo si potrà domani, domani e basta! Niente, niente, silenzio! - disse Levin, e, avvoltolo ancora una volta nella pelliccia, aggiunse: - Ti voglio tanto bene. Dunque, posso rimanere alla seduta? 

- Ma s'intende, si può. 

- Di che si parla oggi, da voi? - chiese Levin, senza cessare di sorridere. 

Giunsero alla seduta. Levin ascoltava il segretario, che, balbettando, leggeva il verbale, evidentemente a lui stesso incomprensibile; ma Levin scorgeva dalla faccia di questo segretario che era un simpatico, buono e brav'uomo; lo vedeva da come si confondeva e si turbava quando leggeva il verbale. Dopo cominciarono i discorsi. Si discuteva di dover defalcare alcune somme e situare alcuni tubi, e Sergej Ivanovic disse qualcosa di spiacevole a due membri del consiglio e parlò trionfante a lungo di qualcosa; ma l'altro membro, segnato qualcosa su di un foglietto di carta, in principio fu timido, ma poi gli rispose molto velenosamente e con garbo. E poi anche Svijazskij (c'era anche lui) disse qualcosa molto bene e con nobiltà. Levin li ascoltava e s'accorgeva che né quelle somme defalcate, né quei tubi esistevano, che non c'era nulla di tutto questo, e che essi non si arrabbiavano affatto e che erano buone e brave persone e che tutto fra di loro andava bene, in modo tanto cordiale. Non davano noia a nessuno e tutti si trovavano bene. Notava Levin che quel giorno tutti erano per lui trasparenti, e che da piccoli segni prima inosservati, egli riusciva a conoscere l'animo di ciascuno e vedeva chiaramente che tutti erano brave persone. Di più, quel giorno, tutti gli volevano straordinariamente bene. Lo si vedeva dal modo col quale gli rivolgevano la parola, e nel modo affabile e affettuoso col quale lo guardavano anche quelli che non lo conoscevano. 

- Su, allora, sei contento? - gli chiedeva Sergej Ivanovic. 

- Molto. Non avrei mai pensato che la cosa fosse così interessante. Bello, bellissimo. 

Svijazskij si avvicinò a Levin e lo invitò a prendere il tè a casa sua. Levin non poteva in nessun modo rendersi conto e ricordare perché mai Svijazskij gli fosse stato antipatico, e che cosa avrebbe voluto trovare in lui. Era un uomo intelligente e buono in modo straordinario. 

- Molto contento - disse e chiese della moglie e della cognata. E per una strana connessione di idee (nella sua immaginazione il pensiero della cognata di Svijazskij si collegava al matrimonio) gli parve che a nessuno meglio che alla moglie e alla cognata di Svijazskij si potesse parlare della propria felicità, e fu molto contento di andare da loro. 

Svijazskij lo interrogò sul suo lavoro in campagna, sostenendo, come al solito, che non c'era nessuna possibilità di creare qualcosa che non fosse stata già trovata in Europa; ma adesso questo non urtava più Levin. Al contrario, sentiva che Svijazskij aveva ragione, che tutto il lavoro che si faceva era insignificante e scorgeva la sorprendente mitezza e delicatezza con cui Svijazskij evitava di far valere le proprie ragioni. Le signore Svijazskij erano particolarmente gentili. A Levin pareva che esse sapessero già tutto e simpatizzassero con lui, ma che non parlassero della cosa solo per delicatezza. Rimase da loro un'ora, due, tre, parlando di argomenti vari, e celava solo quello che gli riempiva l'anima, e non si accorgeva che ormai era diventato oltremodo noioso, e che da tempo le signore avrebbero preferito andare a dormire. Svijazskij lo accompagnò in anticamera, sbadigliando e meravigliandosi dello strano stato in cui era l'amico. Era l'una passata. Levin tornò in albergo, e il pensiero di restare solo con la propria impazienza a passare le dieci ore che ancora gli restavano, lo spaventò. Il cameriere di turno gli accese le candele e voleva andar via, ma Levin lo trattenne. Questo cameriere, Egor, che prima Levin non aveva mai notato, gli si rivelò, d'un tratto, come un uomo molto intelligente e bravo, e soprattutto buono. 

- Be', Egor, è difficile riuscire a non dormire? 

- Che fare? È il nostro mestiere che è fatto così. Nelle case dei signori si sta tranquilli, ma, in compenso, qui c'è più guadagno. 

Venne in chiaro che Egor aveva famiglia, tre maschi e una figlia sarta che egli voleva maritare al commesso di una valigeria. 

Levin, a questo proposito, comunicò a Egor la sua idea che nel matrimonio la cosa principale è l'amore, e che con l'amore si sarebbe stati sempre felici, perché la felicità è solo dentro di noi. 

Egor ascoltava con attenzione ed evidentemente aveva capito in pieno il pensiero di Levin, ma per confermarlo uscì nell'osservazione, inattesa per Levin, che, quando viveva dai suoi buoni signori, era stato sempre contento di loro, e ora era pienamente contento del suo padrone sebbene fosse un francese. 

``È un uomo straordinariamente buono'' pensava Levin. 

- Be', e tu, Egor, quando hai preso moglie, l'amavi tua moglie? 

- E come se l'amavo! - rispose Egor. 

E Levin credette che anche Egor si trovasse in uno stato di euforia e avesse l'intenzione di manifestare tutti i suoi sentimenti più intimi. 

- Anche la mia vita è straordinaria. Io sin da piccolo\ldots{} - cominciò con gli occhi che gli luccicavano, evidentemente contagiato dall'entusiasmo di Levin, così come si è contagiati dallo sbadiglio. 

Ma in quel momento si udì una scampanellata; Egor andò via e Levin rimase solo. Non aveva mangiato a pranzo, aveva rifiutato il tè e la cena da Svijazskij, ma non poteva neppure pensare a dormire. Nella camera c'era fresco, eppure il caldo lo soffocava. Aprì ambedue i finestrini e sedette sulla tavola proprio di fronte ad essi. Di là, oltre un tetto coperto di neve, si vedevano una croce lavorata con delle catene e, sopra a questa, il triangolo della costellazione sorgente dell'Auriga col giallo chiaro della Capra. Egli guardava ora la croce, ora la costellazione, aspirando l'aria gelata che entrava con uniformità in camera, e seguiva, come in sogno, le immagini e i ricordi che gli sorgevano nella mente. Dopo le tre, notò dei passi nel corridoio e guardò dalla porta. Era Mjaskin, il giocatore a lui noto che rientrava dal club. Camminava con aria cupa, aggrottando le sopracciglia e spurgando. ``Povero disgraziato!'' pensò Levin, e lacrime di amore e di compassione per quell'uomo gli vennero agli occhi. Voleva parlare con lui, consolarlo; ma, accortosi di aver indosso la sola camicia, cambiò idea, e sedette di fronte al finestrino per fare un bagno nell'aria fredda e guardare quella croce silenziosa, ma per lui piena di significato, dalla forma sorprendente, e la stella giallo-chiara che si levava. Dopo le sei cominciarono a far rumore i lucidatori dei pavimenti, qualcuno sonò per qualche servizio, e Levin sentì che cominciava a gelare. Chiuse il finestrino, si lavò, si vestì e uscì in strada. 

\capitolo{XV}\label{xv-3} 

Le strade erano ancora deserte. Levin si diresse verso casa Šcerbackij. L'ingresso principale era chiuso e tutto dormiva. Tornò indietro, rientrò in casa e ordinò un caffè. Il cameriere di turno di giorno, già non più Egor, glielo portò. Levin voleva attaccar discorso con lui, ma questi fu chiamato da una scampanellata e andò via. Levin provò a bere un po' di caffè e mise in bocca una ciambellina, ma la bocca non sapeva proprio che farsene delle ciambelle. Sputò la ciambella, infilò il cappotto e uscì di nuovo. Erano le nove passate, quando, per la seconda volta, si accostò alla scala degli Šcerbackij. In casa s'erano appena alzati, e il cuoco andava a far la spesa. Dovevano passare almeno altre due ore. 

Tutta quella notte e la mattina seguente Levin aveva vissuto inconsciamente e si era sentito del tutto fuori della vita materiale. Non aveva mangiato durante il giorno, non aveva dormito per due notti, aveva passato alcune ore, svestito, al gelo, e si sentiva non solo fresco e sano come non mai, ma come staccato completamente dal corpo; si moveva senza alcuno sforzo di muscoli e sentiva di poter fare qualsiasi cosa. Era sicuro che, se fosse stato necessario, sarebbe volato in cielo o avrebbe smosso l'angolo di una casa. Passò il resto del tempo in istrada, guardando continuamente l'ora e voltandosi di qua e di là. 

E ciò che vide in quel momento non lo vide mai più. Bambini che andavano a scuola, colombi grigio-azzurri che volavano dal tetto sul marciapiede e ciambelle cosparse di farina, esposte da mani invisibili, lo commossero in modo particolare. Quelle ciambelle, quei colombi, i due bambini non erano di questa terra. In un solo attimo, uno dei bambini corse verso il colombo e guardò Levin sorridendo; il piccione starnazzò e volò via luccicando al sole fra i granelli di neve che tremavano nell'aria, e da una vetrina esalò odor di pane cotto al forno e le ciambelle furono esposte. Tutto questo insieme era straordinariamente bello, e Levin rideva e piangeva di gioia. Fatto un giro per il vicolo Gazetnyj e la Kislovka, tornò di nuovo in albergo, e dopo aver posto dinanzi a sé l'orologio, sedette, aspettando le dodici. Nella camera accanto parlavano di macchine e di una certa truffa e tossivano della tosse di prima mattina. Costoro non capivano che la lancetta dell'orologio si avvicinava alle dodici. La lancetta si avvicinò. Levin uscì sulla scala. I vetturini, evidentemente, già sapevano tutto. Con le facce gioconde circondarono Levin, discutendo fra di loro e offrendo i loro servigi. Senza offendere gli altri e promettendo di andare in seguito anche con loro, Levin ne scelse uno e ordinò di andare dagli Šcerbackij. Il cocchiere era proprio bello con il colletto bianco della camicia tirato fuori dal gabbano e teso sul collo pieno, rosso e forte. Questo cocchiere aveva una slitta alta e comoda, quale Levin non ne trovò mai più. Il cavallo poi era un buon cavallo e cercava di correre, ma non riusciva a muoversi dal posto. Il cocchiere conosceva gli Šcerbackij e, dopo aver fatto cerchio delle braccia e detto ``ih!'', in modo particolarmente deferente verso il cliente, si fermò all'ingresso del palazzo. Il portiere degli Šcerbackij sapeva tutto certamente. Si vedeva dal sorriso degli occhi e da come disse: 

- Ehi, da un pezzo non vi si vede Konstantin Dmitric ! 

Non solo sapeva tutto ma evidentemente giubilava di gioia e si sforzava di nasconderla. Guardando i suoi cari occhi di vecchio, Levin intravide perfino qualcosa di nuovo nella propria felicità. 

- Sono alzati? 

- Favorite! E quello lasciatelo qua - disse, sorridendo quando Levin tornò indietro a prendere il cappello. Questo voleva significare qualcosa. 

- A chi devo annunciare? - chiese il cameriere. 

Il cameriere, sebbene giovane e di quelli nuovi, un bell'imbusto, era tuttavia buono e per bene e anche lui capiva tutto. 

- Alla principessa\ldots{} al principe\ldots{} alla principessina - disse Levin. 

Il primo personaggio che vide fu m.lle Linon. Attraversava la sala e i suoi riccioli e il suo viso splendevano. Egli aveva appena cominciato a parlare con lei che improvvisamente, di là dalla porta, si udì un fruscio di vesti: m.lle Linon scomparve agli occhi di Levin, e gli si comunicò un gioioso terrore della propria imminente felicità. M.lle Linon si affrettò a lasciarlo e andò verso un'altra porta. Appena fu uscita, dei passi leggeri, svelti svelti risonarono sul pavimento di legno e la sua felicità, la sua vita, egli stesso, quello che aveva cercato e desiderato tanto a lungo, si avvicinò veloce a lui. Ella non aveva camminato, ma era stata portata verso di lui come da una forza invisibile. 

Egli vedeva soltanto i chiari occhi di lei, sinceri, spaventati dalla stessa gioia di amore che riempiva il cuore di lui. Brillavano questi occhi sempre più a misura che si avvicinavano e lo accecavano con la loro luce d'amore. Ella si fermò dinanzi a lui, fino a toccarlo. Le mani si sollevarono e gli si abbandonarono sulle spalle. 

Ella aveva fatto tutto quello che poteva, era corsa a lui e gli si era data tutta, con timore e con gioia. Egli l'abbracciò, e premette le labbra su quella bocca che cercava il suo bacio. 

Anche lei non aveva dormito per tutta la notte e per tutta la mattina l'aveva atteso. La madre e il padre acconsentivano in pieno ed erano felici della sua felicità. Ella lo aspettava. Voleva dirgli la sua felicità e quella di lui. Si era preparata ad andargli incontro da sola, e si era rallegrata a questo pensiero, ma era timida e vergognosa ed ella stessa non sapeva quello che avrebbe fatto. Aveva udito i passi e la voce di lui e aveva atteso di là dalla porta che m.lle Linon fosse andata via. M.lle Linon era andata via. Senza pensare, senza chiedersi come e perché, era corsa a lui e aveva fatto tutto quello che aveva fatto. 

- Andiamo dalla mamma - disse prendendolo per mano. Egli non poté dire nulla per molto tempo, non tanto perché temesse di sciupare con le parole la elevatezza del proprio sentimento, quanto perché ogni volta che voleva dire qualcosa, invece delle parole, sentiva che gli sarebbero sfuggite lacrime di gioia. Le prese la mano e la baciò. 

- Possibile che sia vero? - disse alla fine con voce sorda. - Non posso credere che tu mi ami! 

Ella sorrise di questo ``tu'' e della timidezza con la quale egli la guardava. 

- Sì - ella disse, significativamente, lentamente. - Sono così felice! 

Senza lasciare la mano di lui, entrò nel salotto. La principessa, dopo che li ebbe visti, cominciò a respirare forte, e subito si mise a piangere, e poi a ridere, e con un passo così energico quale Levin non si aspettava da lei, corse verso di loro e, abbracciato il capo di Levin, gli baciò e bagnò di lacrime le guance. 

- Allora tutto è concluso! Sono contenta. Amala. Sono felice\ldots{} Kitty. 

- Avete fatto presto - disse il vecchio principe, cercando di fare l'indifferente: ma Levin notò che i suoi occhi erano umidi quando si voltò verso di lui. - Da tanto, sempre, ho desiderato questo! - egli disse, prendendo la mano di Levin e tirandolo a sé. - E ancora, quando questa sventatella, s'era messa in testa\ldots{} 

- Papà - gridò Kitty e gli chiuse la bocca con le mani. 

- Su, basta - disse. - Sono molto\ldots{} Ah! Come sono sciocco\ldots{} 

Abbracciò Kitty, le baciò il viso, la mano, di nuovo il viso, e le fece un segno di croce. E Levin, nel vedere Kitty che a lungo e teneramente baciava la mano carnosa di lui, fu preso da un impeto d'amore per il vecchio principe, per quest'uomo che prima gli era estraneo. 

\capitolo{XVI}\label{xvi-3} 

La principessa sedeva in poltrona, tacendo e sorridendo; il principe le stava accanto; Kitty stava presso la poltrona del padre senza lasciargli la mano. Tutti tacevano. 

La principessa per prima, con poche parole, riportò i pensieri e i sentimenti di tutti loro alle questioni pratiche della vita. E questo parve a tutti strano e persino penoso al primo momento. 

- Quando allora? Bisogna dare la benedizione e annunciarlo. E a quando le nozze? Che ti pare, Aleksandr? 

- Eccolo - disse il vecchio principe, indicando Levin - è lui il personaggio di centro. 

- Quando? - disse Levin, arrossendo. - Domani. Se domandate a me, per me, oggi la benedizione e domani le nozze. 

- Su, basta, mon cher, non dire sciocchezze! 

- Allora, fra una settimana. 

- È quasi pazzo. 

- No, perché mai? 

- Ma pensa! - disse la madre, sorridendo, felice di questa fretta. - E il corredo? 

``Possibile che ci sia di mezzo il corredo e tutto il resto? - pensò Levin con terrore. - E forse, il corredo e la benedizione e tutto il resto possono sciupare la mia felicità? Niente può guastarla! - Guardò Kitty e notò che non era per nulla contrariata dal fatto che si pensasse al corredo. - Si vede che è necessario'' pensò Levin. 

- Io di questo non capisco nulla, ho espresso soltanto il desiderio mio - disse, scusandosi. 

- Così, decideremo noi. Ora dobbiamo dare la benedizione e partecipare il fidanzamento. Si fa così. 

La principessa si accostò al marito, lo baciò e voleva andar via, ma egli la trattenne, l'abbracciò e, teneramente, come un giovane innamorato, la baciò più volte sorridendo. I vecchi, evidentemente, s'erano confusi per un attimo e non sapevano bene se erano loro a essere innamorati, o soltanto la figliuola. Quando il principe e la principessa uscirono, Levin si accostò alla fidanzata e le prese la mano. Egli era ora padrone di sé e poteva parlare e doveva dirle tante cose. Ma disse tutt'altro. 

- Come sapevo che sarebbe stato così! Non ho sperato mai, ma dentro di me sono sempre stato sicuro - disse. - Credo che fosse predestinato. 

- E io - ella disse - perfino allora\ldots{} - Si fermò e poi continuò, guardandolo decisa coi suoi occhi sinceri. - Perfino allora, quando ho respinto da me la mia felicità, ho amato sempre voi solo, ma mi ero esaltata. Devo dire\ldots{} Potete dimenticare? 

- Forse è stato per il meglio. Voi dovete perdonare molte cose. Io vi devo dire\ldots{} 

Era una di quelle cose che aveva deciso di dirle. Aveva deciso di dirle, fin dai primi giorni, due cose: una, che non era puro come lei, e l'altra che non aveva fede. Era tormentoso, ma riteneva di dover dire e l'una e l'altra cosa. 

- No, non ora, dopo! - egli disse. 

- Bene, dopo, ma me lo direte: assolutamente. Io non ho paura di nulla. Ho bisogno di sapere tutto. Ormai è concluso\ldots{} 

Egli terminò: 

- È concluso nel senso che mi prenderete così come sono, non rinuncerete a me? Sì? 

- Sì, sì. 

La loro conversazione fu interrotta da m.lle Linon che sorridendo con tenerezza, anche se con affettazione, venne a congratularsi con la sua allieva prediletta. Non era ancora andata via che vennero a congratularsi i domestici. Poi vennero i parenti, e cominciò quel beato stordimento dal quale Levin non uscì se non il giorno dopo le nozze. Levin si sentiva continuamente a disagio, si annoiava ma la tensione della sua felicità continuava sempre, aumentava. Sentiva ogni momento che si pretendevano da lui molte cose che egli ignorava, ma faceva tutto quello che gli dicevano e tutto questo gli procurava piacere. Pensava che il suo fidanzamento non dovesse aver nulla in comune con quello degli altri, che le solite abitudini avrebbero sciupato la sua particolare felicità, ma finì col fare tutto quello che fanno gli altri e malgrado ciò la sua felicità divenne sempre più grande e sempre più fuori del comune, una felicità che non aveva nulla di paragonabile. 

- Ora mangeremo i confetti - diceva m.lle Linon, e Levin andava a comprare i confetti. 

- Via, sono molto contento- diceva Svijazskij. - Vi consiglio di prendere i mazzi di fiori di Fomin. 

- Ci vogliono?- ed egli andava da Fomin. 

Il fratello gli diceva che occorreva prendere in prestito del denaro, perché ci sarebbero state molte spese, regali\ldots{} 

- Ci vogliono dei regali? - ed egli correva da Ful'de. 

E dal pasticciere e da Fomin e da Ful'de gli sembrava che lo aspettassero, che fossero contenti di lui e solennizzassero la sua felicità così come tutti quelli che avevano a che fare con lui in quei giorni. Era straordinario non solo che tutti lo amassero, ma che anche le persone antipatiche, fredde, indifferenti, ammirandolo, gli si sottomettessero in tutto e avessero pel suo sentimento tatto e tenerezza, condividendo la persuasione sua d'essere l'uomo più felice del mondo perché la sua fidanzata era il vertice di ogni perfezione. Lo stesso provava anche Kitty. Quando la contessa Nordston si permise di accennare al fatto che avrebbe desiderato per lei qualcosa di meglio, Kitty si accalorò tanto e dimostrò con tanta convinzione che non poteva esservi al mondo alcuno migliore di Levin, che la contessa Nordston dovette riconoscerlo e da allora in poi, in presenza di Kitty, accolse Levin con un sorriso di ammirazione. 

La spiegazione da lui promessa fu l'unico avvenimento penoso di quel periodo. Si consigliò col vecchio principe, e, ottenutone il permesso, consegnò a Kitty il suo diario nel quale era scritto quello che lo tormentava. Aveva scritto questo diario proprio pensando alla sua futura sposa. Lo tormentavano due cose: la sua impurità e la sua mancanza di fede. Quest'ultima confessione passò inosservata. Ella era religiosa, non aveva mai avuto dubbi sulla verità della religione; ma l'ateismo esteriore di lui non la turbava quasi. Ella conosceva per mezzo dell'amore tutta l'anima sua e nell'anima sua vedeva tutto quello che voleva, e che un tale stato d'animo significasse non aver fede, le era assolutamente indifferente. L'altra confessione invece la fece piangere amaramente. 

Levin le aveva dato il suo diario non senza lotta interiore. Sapeva che fra di loro non potevano e non dovevano esserci segreti e perciò aveva deciso di far così, ma non si era reso conto degli effetti che ne sarebbero potuti derivare, non si era trasferito in lei. Appena giunse da lei quella sera, e, prima del teatro, entrò nella stanza di lei e vide il caro e pietoso viso piangente, infelice per un dolore irrimediabilmente prodottole da lui, solo allora comprese l'abisso che separava il proprio passato dalla innocenza di colomba di lei ed ebbe orrore di quello che aveva fatto. 

- Prendete, prendete con voi questi orribili scritti - ella disse respingendo i quaderni che le stavano davanti sul tavolo. - Perché me li avete dati? No, ma forse è meglio - aggiunse, avendo pena del viso disperato di lui. - Ma è orribile, orribile. 

Egli abbassò il capo e tacque. Non poteva dir nulla. 

- Voi non mi perdonerete - mormorò. 

- No, io ho già perdonato, ma è orribile! 

Ma la felicità di lui era così grande che questa confessione non la turbò, le diede solo una sfumatura nuova. Gli aveva perdonato; ma da quel momento in poi egli si considerò ancora più indegno di lei, ancora più si inchinò moralmente dinanzi a lei e ancor più apprezzò la propria immeritata felicità. 

\capitolo{XVII}\label{xvii-3} 

Rivolgendo involontariamente nella mente le impressioni dei discorsi fatti durante e dopo il pranzo, Aleksej Aleksandrovic tornava nella sua solitaria stanza d'albergo. Le parole di Dar'ja Aleksandrovna circa il perdono lo avevano soltanto irritato. L'adempiere o meno la regola cristiana era questione difficile a risolversi nel suo caso, non se ne poteva parlare alla leggera ed era stata già da tempo decisa da Aleksej Aleksandrovic negativamente. Di quanto si era detto gli erano impresse maggiormente le parole dello sciocco e buon Tuškevic: È stato bravo, lo ha sfidato e lo ha ucciso. Tutti, evidentemente, avevano approvato, sebbene, per cortesia, non l'avessero detto. 

``Del resto, questa faccenda è conclusa, è inutile pensarci su'' si disse Aleksej Aleksandrovic. E, pensando solo al viaggio imminente e all'ispezione da compiere, entrò in camera e chiese al portiere che l'accompagnava, dove fosse il suo servitore. Il portiere disse che era uscito proprio in quel momento. Aleksej Aleksandrovic ordinò di portargli il tè, sedette alla tavola e, preso il Frum, cominciò a combinare l'itinerario del viaggio. 

- Due telegrammi - disse il servitore di ritorno, entrando in camera. - Scusate, eccellenza, ero appena uscito. 

Aleksej Aleksandrovic prese i telegrammi e li dissuggellò. Il primo telegramma portava la notizia della nomina di Stremov allo stesso posto cui aspirava Karenin. Aleksej Aleksandrovic gettò via il dispaccio, e, fattosi rosso in viso, si alzò e prese a camminare per la stanza. ``Quos vult perdere dementat'' disse, riferendo quel quos alle persone che avevano contribuito a quella nomina. Non lo irritava il fatto di non aver conseguito lui quel posto, né d'essere stato, evidentemente, messo da parte; ma per lui era inconcepibile, sorprendente come costoro non si accorgessero che quel ciarlatano, quel parolaio di Stremov fosse meno di ogni altro adatto. Come non si accorgevano che rovinavano se stessi, il proprio prestige con quella nomina! 

``Ancora qualcosa di questo genere'' disse con bile, aprendo il secondo dispaccio. Il telegramma era della moglie. La firma a matita turchina ``Anna'' gli saltò per prima agli occhi. ``Muoio, prego, supplico venire. Morirò più tranquilla col perdono'' egli lesse. Sorrise sprezzante e gettò via il telegramma. Che questo fosse inganno e astuzia era fuor di dubbio, così gli parve al primo momento. 

``Non c'è inganno dinanzi al quale ella si arresti. Deve partorire. Forse una malattia di parto. Ma quale ne è lo scopo? Legittimare il neonato, compromettere me e ostacolare il divorzio - pensava. - Ma lì è detto qualcosa: muoio\ldots{}''. Rilesse il telegramma; e improvvisamente il senso letterale di quello che aveva letto lo colpì. ``E se fosse vero? - si disse. - Se è vero che in un momento di sofferenza e di prossimità alla morte, ella si penta sinceramente e io, sospettando in questo un inganno, mi rifiuti di andare? Non solo sarebbe crudele e tutti mi giudicherebbero male, ma sarebbe stolto da parte mia''. 

- Pëtr, trattieni la carrozza. Vado a Pietroburgo - disse al servitore. 

Aleksej Aleksandrovic decise di andare a Pietroburgo e di vedere la moglie. Se la malattia era un inganno, avrebbe taciuto e sarebbe ripartito. Se realmente era malata, in punto di morte, egli le avrebbe perdonato se la trovava ancora in vita, e le avrebbe reso gli estremi onori se fosse arrivato troppo tardi. 

Per tutto il viaggio non pensò più a quello che doveva fare. 

Con un senso di stanchezza e di trascuratezza, derivato dalla notte in treno, Aleksej Aleksandrovic andava nella nebbia mattutina di Pietroburgo per il Nevskij Prospekt deserto e guardava dinanzi a sé senza pensare a ciò che lo aspettava. Non poteva pensare perché, immaginando quello che sarebbe accaduto, non poteva scacciare l'idea che la morte di lei avrebbe risolto d'un tratto la difficoltà della propria situazione. I fornai, le botteghe chiuse, i vetturini notturni, i portieri che spazzavano i marciapiedi baluginavano dinanzi ai suoi occhi ed egli osservava tutto ciò cercando di soffocare il pensiero di quello che lo aspettava e che tuttavia desiderava. Si accostò alla scala. Una vettura con un cocchiere addormentato stava all'ingresso. Entrando nel vestibolo Aleksej Aleksandrovic trasse come da un angolo remoto del cervello la propria decisione e si consultò; in quell'angolo era scritto: ``Se c'è inganno, allora calmo disprezzo, e via. Se è vero, allora salvare le convenienze''. 

Il portiere aprì la porta ancora prima che Aleksej Aleksandrovic sonasse. Il portiere Petrov, chiamato Kapitonyc, aveva un aspetto strano, così com'era in quella vecchia finanziera senza cravatta e in pantofole. 

- Come sta la signora? 

- Ha partorito ieri felicemente. 

Aleksej Aleksandrovic si fermò e impallidì. Adesso capiva con chiarezza quanto intensamente avesse desiderato la morte di lei. 

- E come va la salute? 

Kornej in grembiule da mattina, scese di corsa dalla scala. 

- Molto male - rispose. - Ieri c'è stato un consulto e ora il dottore è qui. 

- Prendi la roba - disse Aleksej Aleksandrovic e, provando un certo sollievo alla notizia che c'era pur sempre speranza di morte, entrò nell'ingresso. 

Sull'attaccapanni c'era un cappotto militare. Aleksej Aleksandrovic lo notò e chiese: 

- Chi c'è qui? 

- Il dottore, la levatrice e il conte Vronskij. 

Aleksej Aleksandrovic passò nelle stanze interne. 

Nel salotto non c'era nessuno: dallo studio, al suono dei passi di lui, uscì la levatrice con una cuffia di nastri lilla. 

Si avvicinò ad Aleksej Aleksandrovic e, con la confidenza che dà la prossimità della morte, presolo per mano, lo tirò verso la camera. 

- Sia lodato Iddio che siete arrivato. Solo di voi, solo di voi domanda - ella disse. 

- E date del ghiaccio, presto! - si sentì dalla camera la voce imperiosa del medico. 

Aleksej Aleksandrovic passò nello studio di lei. Vicino alla tavola, di traverso contro la spalliera, su di una sedia bassa, stava Vronskij e, coperto il viso con le mani, piangeva. Saltò su alla voce del dottore, tolse le mani dal viso e vide Aleksej Aleksandrovic. Visto il marito, si confuse al punto da doversi sedere di nuovo, ritraendo il capo nelle spalle quasi desiderando di sprofondare in qualche posto, poi fece uno sforzo su di sé, si alzò e disse: 

- Muore. I dottori hanno detto che non c'è più speranza. Io sono in vostro potere, ma permettetemi di stare qui\ldots{} del resto, io sono a vostra disposizione. 

Aleksej Aleksandrovic, viste le lacrime di Vronskij, sentì un afflusso di quello sconvolgimento d'animo che produceva in lui la vista della sofferenza altrui, e voltando il viso, senza finir di ascoltare le sue parole, si avviò in fretta verso la porta. Dalla camera si udiva la voce di Anna che diceva qualcosa. La voce era animata, viva, con intonazioni straordinariamente precise. Aleksej Aleksandrovic entrò in camera e si accostò al letto. Ella giaceva col viso rivolto verso di lui. Le guance erano rosse, gli occhi lucidi; le piccole mani bianche, uscendo dal polso della camicia, brancicavano, attorcigliandolo, un angolo del lenzuolo. Sembrava che non solo ella fosse sana e fresca, ma nella migliore disposizione d'animo. Parlava in fretta, ad alta voce, con un'intonazione di voce insolitamente giusta e precisa. 

- Perché Aleksej, io parlo di Aleksej Aleksandrovic (che strano, orribile destino che siano tutte e due Aleksej, non è vero?), Aleksej non mi direbbe di no. Io dimenticherei, lui perdonerebbe\ldots{} Ma come mai non viene? Lui è buono, non lo sa neanche lui quanto è buono. Ah, Dio mio! Che pena! Datemi presto dell'acqua. Ah, ma questo a lei, alla mia bambina, farà male! Su, via, va bene, datele una balia. Su, io consento, è anche meglio. Egli arriverà, gli farà male vederla. Datela via. 

- Anna Arkad'evna, egli è arrivato. Eccolo - diceva la levatrice, cercando di richiamare su Aleksej Aleksandrovic l'attenzione di lei. 

- Ah, che sciocchezza! - continuava Anna, senza scorgere il marito. - Su, datemela, la bambina, datemela. Non è venuto ancora. Voi dite che non perdonerà perché non lo conoscete. Nessuno lo conosce. Io sola, e ne ho tanta pena ora. I suoi occhi, dovete sapere, sono quelli di Serëza e perciò non posso guardarli. Hanno dato da mangiare a Serëza? Perché io so che tutti dimenticano. Egli non avrebbe dimenticato. Bisogna trasferire Serëza nella stanza d'angolo e pregare Mariette di dormire con lui. 

Improvvisamente si contrasse, ammutolì e con spavento, come se aspettasse un colpo e volesse difendersene, portò le mani al volto. Aveva visto il marito. 

- No, no - prese a dire - non ho paura di lui, ho paura della morte. Aleksej, avvicinati qua. Ho fretta perché non ho tempo, mi resta poco da vivere, subito ricomincerà la febbre e non capirò più nulla. Adesso capisco tutto e vedo tutto. 

Il viso corrugato di Aleksej Aleksandrovic prese un'espressione martoriata; le afferrò la mano e voleva dire qualcosa, ma non poté articolare parola; il suo labbro inferiore tremava, egli lottava ancora sempre con la propria agitazione e solo di rado la guardava. E ogni volta che la guardava, scorgeva che i suoi occhi lo guardavano con una tenerezza così commossa e incantata quale egli non aveva mai vista in lei. 

- Aspetta, tu non sai\ldots{} Fermatevi, fermatevi\ldots{} - ella si fermò, come per raccogliere le idee. - Ecco quello che volevo dire. Non meravigliarti di me. Io sono sempre la stessa. Ma in me c'è un'altra donna; lei ha cominciato ad amare quell'altro e io volevo odiarti e non potevo dimenticare quello che c'era stato prima. Quella non sono io. Adesso io sono la vera, sono una. Adesso muoio, so che morirò, chiedilo a lui. Io ora sento, ecco, i pesi sulle braccia, sulle gambe, sulle dita. Le dita\ldots{} ecco come sono enormi! Ma tutto questo finirà presto\ldots{} Solo una cosa mi è necessaria: perdonami, perdonami di tutto! Sono detestabile, ma la mia njanja mi diceva che una santa martire\ldots{} come si chiamava?\ldots{} era stata peggiore di me. Andrò a Roma, là ci sono degli eremi, e allora non darò fastidio a nessuno, prenderò solo Serëza e la bambina\ldots{} No, tu non puoi perdonare! no, va' via, sei troppo buono! - Ella teneva con una mano che scottava la mano di lui, con l'altra lo respingeva. 

Lo sconvolgimento d'animo di Aleksej Aleksandrovic si faceva sempre più forte ed era giunto a un tale punto che egli aveva già smesso di dominarlo; a un tratto sentì che quello che egli considerava uno sconvolgimento, era, al contrario, un beato stato d'animo che gli dava a un tratto una nuova felicità mai prima provata. Egli non pensava più che la legge cristiana che avrebbe voluto seguire per tutta la vita gli prescriveva di perdonare e di amare i nemici, ma un gioioso sentimento d'amore e di perdono verso i nemici gli riempiva ora l'anima. Stava in ginocchio e, posto il capo sulla giuntura del braccio di lei che lo bruciava come fuoco attraverso la camiciola, singhiozzava come un bambino. Ella abbracciò la sua testa quasi calva, lo accostò a sé e levò gli occhi in su con una espressione di orgoglio e di sfida. 

- Eccolo, io lo sapevo! Ora addio a tutti, addio!\ldots{} Sono venuti di nuovo, perché non vanno via? Ma toglietemi queste pellicce! 

Il dottore le sciolse le braccia e, dopo averla adagiata con precauzione sul guanciale, le coprì le spalle. Ella si sdraiò docile e guardò dinanzi a sé con uno sguardo raggiante. 

- Ricordati una cosa sola, che avevo bisogno solo del perdono e che non voglio altro\ldots{} E perché lui non viene? - cominciò a dire volgendosi attraverso la porta a Vronskij. - Avvicinati, avvicinati, dagli la mano. 

Vronskij si accostò alla sponda del letto e, guardandola si coprì di nuovo il viso con le mani. 

- Scopri il viso, guardalo! È un santo - ella disse. - Ma scopri, scopri il viso - prese a dire con rabbia. - Aleksej Aleksandrovic, scoprigli il viso. Lo voglio vedere. 

Aleksej Aleksandrovic prese le mani di Vronskij e le allontanò dal viso sfigurato dall'espressione di tormento e di vergogna. 

- Dàgli la mano, perdonagli. 

Aleksej Aleksandrovic gli dette la mano, senza trattenere le lacrime che gli correvano giù dagli occhi. 

- Sia lodato Iddio. Sia lodato Iddio - ella cominciò a dire. - Ora tutto è pronto. Bisogna stendere solo un po' le gambe. Ecco, così, benissimo. Come son fatti senza gusto questi fiori! non sono per nulla simili alle viole - diceva mostrando la tappezzeria. - Dio mio, Dio mio! Quando finirà tutto questo? Datemi della morfina. Dottore datemi della morfina. Dio mio, Dio mio! 

E cominciò ad agitarsi sul letto. 

Il medico curante e i dottori dicevano trattarsi di una febbre puerperale nella quale su cento probabilità novantanove erano di morte. Tutto il giorno ella ebbe febbre, delirio e deliquio. A mezzanotte la malata giaceva priva di sensi e quasi senza polso. 

Si aspettava la fine da un momento all'altro. 

Vronskij andò a casa, ma la mattina venne a prendere notizie, e Aleksej Aleksandrovic, incontrandolo nell'ingresso, disse: 

- Restate, può darsi che cerchi di voi - ed egli stesso lo introdusse nello studio della moglie. 

Verso la mattina cominciò di nuovo l'agitazione, l'eccitamento, la velocità del pensiero e del discorso e sopravvenne un nuovo deliquio. Per due giorni avvenne lo stesso, e i dottori dissero che c'era speranza. Quel giorno Aleksej Aleksandrovic andò nello studio dove era Vronskij e, chiusa la porta, sedette di fronte a lui. 

- Aleksej Aleksandrovic - disse Vronskij, sentendo che si avvicinava la spiegazione - io non posso parlare, non posso intendere. Risparmiatemi! Per quanto voi soffriate, credetemi, per me è ancora più orribile. 

Voleva alzarsi. Ma Aleksej Aleksandrovic lo prese per una mano e disse: 

- Vi prego di ascoltarmi, è indispensabile. Devo spiegarvi i miei sentimenti, quelli che mi hanno guidato e che mi guideranno, perché voi non abbiate a sbagliare sul mio conto. Sapete che mi ero deciso a chiedere il divorzio e avevo persino iniziato la causa. Non vi nascondo che, dando inizio a un giudizio, ero indeciso, mi tormentavo; vi confesso che il desiderio di vendicarmi di voi e di lei mi perseguitava. Quando ho ricevuto il telegramma, sono venuto qua con gli stessi sentimenti nell'animo, dirò di più: desideravo la sua morte. Ma\ldots{} - tacque un po' dubbioso se aprirgli o no l'animo suo. - Ma l'ho vista e ho perdonato. E la gioia del perdono mi ha rivelato il mio dovere. Ho perdonato completamente. Voglio porgere l'altra guancia, voglio dare la tunica quando mi si è preso il mantello. Prego Iddio che non allontani da me la gioia del perdono! - Le lacrime erano nei suoi occhi e il suo sguardo chiaro, tranquillo colpì Vronskij. - Ecco la mia situazione. Voi potete calpestarmi nel fango, fare di me lo zimbello del mondo; io non abbandonerò lei, e non dirò mai a voi una parola di recriminazione - continuò. - Il mio dovere è, per me, chiaramente segnato: devo essere con lei e ci sarò. Se desidera di vedervi, ve lo farò sapere, ma ora suppongo che sia meglio per voi allontanarvi. 

Si alzò e i singhiozzi gli spezzarono le parole. Anche Vronskij si era alzato e, restando curvo, lo guardò di sotto in su. Non capiva i sentimenti di Aleksej Aleksandrovic; ma sentiva che vi era in essi qualcosa di molto più alto, e persino di inaccessibile a lui e alla propria visione del mondo. 

\capitolo{XVIII}\label{xviii-3} 

Dopo il colloquio con Aleksej Aleksandrovic, Vronskij uscì sulla scala di casa Karenin e vi si fermò, ricordando a stento dove si trovasse e se dovesse andare a piedi o in vettura. Si sentiva svergognato, umiliato, colpevole e nella impossibilità di riscattare la propria umiliazione. Si sentiva lanciato al di fuori di quella carreggiata sulla quale aveva finora camminato con tanto orgoglio e tanta disinvoltura. Tutte le abitudini e regole di vita che gli erano parse sempre così salde, gli si erano improvvisamente mostrate mendaci e inapplicabili. Il marito ingannato, che gli si era fino allora presentato come un essere pietoso, un impedimento casuale e un po' ridicolo alla propria felicità, a un tratto era stato invocato proprio da lei e si era elevato tanto in alto da ispirargli rispetto; e quello stesso marito, pur così in alto ora, non s'era mostrato cattivo, falso, ridicolo, ma buono, semplice, generoso. Tutto questo Vronskij non poteva non sentirlo. Le parti si erano improvvisamente cambiate. Vronskij sentiva la superiorità di lui e la umiliazione propria. Sentiva che quel marito era grande anche nel suo dolore, ed egli basso, meschino nel suo inganno. Ma questa coscienza della propria bassezza, di fronte all'uomo che ingiustamente aveva oltraggiato, rappresentava solo una piccola parte del suo dolore. Si sentiva ora indicibilmente infelice, perché la sua passione per Anna, che gli sembrava intiepidita negli ultimi tempi, nel momento in cui sapeva di averla perduta per sempre, era diventata più forte di quanto non lo fosse mai stata. L'aveva vista tutta durante la sua malattia; aveva imparato a conoscerne l'anima, e gli pareva di non averla mai amata fino ad allora. E proprio adesso, quando aveva imparato a conoscerla e l'aveva presa ad amare così come si deve amare, egli era umiliato davanti a lei, e la perdeva per sempre, lasciando di sé solo un ricordo vergognoso. Più terribile di tutto era stata quella ridicola, umiliante situazione, in cui Aleksej Aleksandrovic gli aveva tolto le mani dalla faccia svergognata. Egli stava fermo sulla scala di casa Karenin, come smemorato, senza sapere cosa fare. 

- Volete una vettura? - chiese il portiere. 

- Sì, una vettura. 

Tornato a casa dopo tre notti insonni, Vronskij, senza spogliarsi, si coricò bocconi sopra un divano, incrociando le mani e poggiandovi sopra la testa. La testa gli pesava. Le immagini, le memorie e le idee più strane si susseguivano le une alle altre con straordinaria velocità e chiarezza: ora la medicina che aveva versato all'ammalata e che aveva fatto gocciolare dal cucchiaino, ora le braccia bianche della levatrice, o la strana posizione di Aleksej Aleksandrovic sul pavimento, davanti al letto. 

``Addormentarsi, dimenticare!'' si disse con la calma certezza dell'uomo sano che se è stanco e vuol dormire, s'addormenta subito. E invero, in quello stesso momento, nella sua testa sopraggiunsero confusione e oblio. Le onde della vita subcosciente avevano già cominciato ad affluirgli alla testa. Ma a un tratto, proprio come se una fortissima scarica elettrica si fosse scaraventata su di lui, rabbrividì in modo che tutto il corpo sussultò sulle molle del divano e, poggiatosi con le mani, saltò su in ginocchio, spaventato. I suoi occhi erano spalancati, come se non si fosse mai addormentato. La pesantezza di testa e la debolezza delle membra erano scomparse d'un tratto. 

``Potete calpestarmi nel fango'' riudiva le parole di Aleksej Aleksandrovic e lo vedeva davanti a sé e vedeva il viso di Anna arrossato dalla febbre e gli occhi scintillanti che guardavano con tenerezza e amore, non lui, ma Aleksej Aleksandrovic; vedeva la propria figura, così com'era apparsa, fatua e ridicola, mentre Aleksej Aleksandrovic gli toglieva le mani dal viso. Di nuovo distese le gambe e si gettò sul divano nella posizione di prima e chiuse gli occhi. 

``Addormentarsi, addormentarsi!'' ripeteva a se stesso. Ma con gli occhi chiusi vedeva ancora più chiaramente il viso di Anna così come era quella sera per lui memorabile delle corse. 

``Questo non è e non sarà, ed ella desidera cancellarlo dalla sua mente. Io invece non posso vivere senza questo. Come potremo mai fare pace, come potremo mai fare pace'' diceva ad alta voce e inconsciamente cominciò a ripetere queste parole. Questa ripetizione di parole tratteneva il sorgere di nuove immagini e di nuovi ricordi che, lo sentiva, gli si affollavano in capo. Ma quella ripetizione di parole non trattenne a lungo l'immaginazione. Di nuovo uno dopo l'altro cominciarono ad apparire i momenti migliori e con essi la recente umiliazione. ``Togli le mani'' diceva la voce di Anna. E lui toglieva le mani e sentiva l'espressione confusa e stupida della propria faccia. 

Era sempre disteso, cercando di addormentarsi, sebbene sentisse che fosse vano, e ripeteva sempre sottovoce le parole casuali di qualche pensiero, tentando così di trattenere il sorgere di nuove immagini. Si pose in ascolto e sentì ripetere con uno strano, pazzo mormorio: ``non hai saputo apprezzare, non hai saputo profittare''. 

``Cos'è, divento pazzo forse? - si disse. - Forse. E del resto, non si diventa forse pazzi, non ci si ammazza, addirittura?'' rispose a se stesso, e, aperti gli occhi, vide con sorpresa, sotto la testa, il cuscino ricamato da Varja, la moglie del fratello. Toccò il fiocco del cuscino e cercò di ricordarsi di Varja, di quando l'aveva vista l'ultima volta. Ma pensare a qualcosa di estraneo era tormentoso. ``No, bisogna addormentarsi''. Accostò il cuscino e si strinse ad esso colla testa, ma doveva fare uno sforzo per tenere gli occhi chiusi. Saltò su e sedette. ``È finita per me - disse. - Bisogna riflettere quel che occorre fare. Che cosa è rimasto?''. Il suo pensiero percorse veloce la propria vita all'infuori del suo amore per Anna. 

L'ambizione? Serpuchovskoj? Il bel mondo? La Corte? Non poteva fermare la mente su nessuna di queste cose. Tutto questo aveva un senso prima, ma ora non esisteva più nulla. Si alzò dal divano, si tolse la finanziera, slacciò la cintura e, scoperto il petto villoso, per respirare più liberamente, fece un giro per la stanza. ``È così che s'impazzisce, è così che ci si spara\ldots{} per non sentire la vergogna'' aggiunse lentamente. 

Si avvicinò alla porta e la chiuse; dopo, con uno sguardo fisso e coi denti fortemente stretti, si accostò alla tavola e, presa la rivoltella, la esaminò, la rigirò dalla parte della canna carica e si fece pensieroso. Rimase due minuti così, col capo chino e l'espressione nel volto di uno sforzo mentale, con la rivoltella in mano, immobile, e pensava. ``S'intende'' disse, come se un logico, prolungato, chiaro passaggio di idee lo avesse condotto a una conclusione indubitabile. In realtà questo ``s'intende'' per lui persuasivo, non era che la conseguenza della ripetizione di un giro sempre identico di ricordi e di figurazioni attraverso il quale era già passato, diecine di volte, nello spazio di un'ora. Sempre gli stesi ricordi della felicità perduta, sempre identica la rappresentazione della mancanza di senso di tutto quello che gli offriva la vita, identica la coscienza della propria umiliazione. Identica era anche la successione di queste immagini e di questi sentimenti. 

``S'intende'' ripeté, quando per la terza volta il pensiero si volse verso la stessa cerchia incantata di ricordi e di pensieri e, poggiata la rivoltella al lato sinistro del petto, strettala forte con tutta la mano, come se d'un tratto l'avesse impugnata, tirò il grilletto. Non sentì il rumore dello sparo, ma un colpo violento nel petto lo buttò a terra. Voleva reggersi all'orlo della tavola, lasciò cadere la rivoltella, vacillò e si sedette per terra, guardandosi intorno con sorpresa. Non riconosceva la sua camera, guardando dal basso i piedi curvi della tavola, il cestino per le carte e la pelle di tigre. I passi veloci scricchiolanti del servitore che camminava nel salotto lo fecero tornare in sé. Fece uno sforzo per pensare e capì che era a terra e, visto il sangue sulla pelle di tigre e sulla mano, capì che s'era sparato. 

- È sciocco! Ho fallito il colpo! - esclamò, con la mano che tastava in cerca della rivoltella. La rivoltella era vicino a lui; lui cercava più in là. Continuando a cercare si protese dall'altra parte e, non avendo la forza di mantenere l'equilibrio, cadde, perdendo sangue. 

L'elegante servitore con le fedine, che più di una volta s'era lamentato con gli amici per la propria debolezza di nervi, si spaventò a tal punto nel vedere il padrone per terra, che lo lasciò a perdere sangue e corse a chiedere aiuto. Dopo un'ora Varja, la moglie del fratello, giunta insieme con tre dottori chiamati da ogni parte e arrivati nello stesso momento, adagiò sul letto il ferito e rimase da lui per curarlo. 

\capitolo{XIX}\label{xix-3} 

L'errore di Aleksej Aleksandrovic di non aver pensato, nel rivedere la moglie, e nel caso che il rimorso di lei fosse sincero e ch'egli perdonasse, alla eventualità che ella non morisse, questo errore, due mesi dopo il suo ritorno da Mosca, gli si parò innanzi in tutta la sua gravità. Ma l'errore suo era derivato non solo dal non aver supposto questa eventualità, ma anche dal non aver mai, fino a quel giorno dell'incontro con la moglie morente, conosciuto il proprio cuore. Egli per la prima volta in vita sua, presso il letto della moglie malata, si era abbandonato a quel sentimento di commossa compassione che in lui suscitavano le sofferenze altrui e di cui prima si vergognava come di una debolezza nociva; e la pena verso di lei e il rimorso di averne desiderato la morte e soprattutto la stessa gioia del perdono avevano fatto sì ch'egli, improvvisamente, avesse sentito non solo un lenimento alle proprie pene, ma anche una tranquillità d'animo che non aveva mai provato prima. Improvvisamente aveva sentito che proprio quello che era la causa delle sue pene, diveniva la sorgente della sua gioia spirituale, quello che pareva insolubile, quando egli rimproverava, recriminava e odiava, era divenuto semplice e chiaro, ora che perdonava e amava. 

Aveva perdonato alla moglie e aveva avuto pena di lei per le sofferenze sue e per il suo rimorso. Aveva perdonato Vronskij e lo commiserava, specialmente dopo che erano giunte a lui le voci del suo atto insano. Anche del figlio aveva più pena di prima e si rimproverava ora di essersi tanto poco occupato di lui. Per la piccola neonata, poi, provava un sentimento particolare, non solo di pena, ma di tenerezza. In principio, per pura compassione egli si era occupato di quella fragile creatura appena nata che non era figlia sua e che era stata trascurata durante la malattia della madre, e che certamente sarebbe morta se egli non si fosse preoccupato di lei; ma non s'era accorto neppur lui che aveva cominciato a volerle bene. Varie volte al giorno andava nella camera dei bambini e a lungo restava là a sedere, tanto che la balia e la njanja, prima timide per la sua presenza, s'erano poi abituate a lui. A volte guardava in silenzio per una mezz'ora intera il visino addormentato rosso-zafferano, lanuginoso e grinzoso della bambina e osservava i movimenti della fronte aggrottata e le manine paffute con le dita ripiegate che col dorso si fregavano gli occhietti e la radice del naso. Proprio in quei momenti Aleksej Aleksandrovic si sentiva completamente tranquillo e in armonia con se stesso, e non vedeva nella sua posizione nulla di eccezionale, nulla che fosse da cambiare. 

Ma quanto più tempo passava, tanto più chiaramente egli scorgeva che, per quanto ora questa posizione gli paresse naturale, non gli avrebbero consentito di permanervi. Sentiva che oltre alla felice forza spirituale che aveva guidato la sua anima, c'era un'altra forza di natura materiale, ma altrettanto e ancor più potente, che dirigeva la sua vita e che non gli consentiva di fermarsi in quella umile tranquillità che desiderava. Sentiva che tutti guardavano a lui con interrogativo stupore, che non lo capivano e che si aspettavano qualcosa da lui. Sentiva inoltre l'instabilità e, in modo particolare, la falsità dei suoi rapporti con la moglie. 

Appena dileguata quella commozione prodotta in lei dalla prossimità della morte, Aleksej Aleksandrovic cominciò a notare che Anna lo temeva, che si sentiva oppressa da lui e che non poteva guardarlo dritto negli occhi. Era come s'ella desiderasse qualcosa e non si decidesse a dirglielo, e anche per lei era come se intuisse che quei rapporti non potevano durare, e attendesse qualcosa da lui. 

Alla fine di febbraio, la neonata di Anna, chiamata anch'essa Anna, s'ammalò. Aleksej Aleksandrovic era stato la mattina nella camera dei bambini e, dato ordine di mandare a chiamare il medico, era andato al ministero. Finiti i suoi affari, tornò a casa dopo le tre. Entrando in anticamera, vide un servitore, un bel giovane con una pellegrina d'orso e alamari, che reggeva un mantello bianco di cane americano. 

- Chi è qui? - domandò Aleksej Aleksandrovic. 

- La principessa Elizaveta Fëdorovna Tverskaja - rispose il servitore con un sorriso, come parve ad Aleksej Aleksandrovic. 

Durante quel penoso periodo, Aleksej Aleksandrovic notava che le sue conoscenze mondane, in particolare le donne, s'interessavano vivamente a lui e a sua moglie. Notava in tutti questi amici come una certa gioia a stento contenuta, quella stessa che egli aveva sorpreso negli occhi dell'avvocato e che ora scorgeva negli occhi del servitore. Come se tutti fossero in una certa esaltazione, come se si dovesse sposare qualcuno. Quando lo incontravano domandavano della salute di sua moglie con una gioia appena celata. 

La presenza della principessa Tverskaja, e per i ricordi legati a lei, e perché in complesso non gli era simpatica, non era gradita ad Aleksej Aleksandrovic, ed egli andò di filato nella camera dei bambini. Nella prima stanza Serëza, disteso col petto sulla tavola e con i piedi su di una sedia, disegnava qualcosa commentando allegramente. L'inglese, che durante la malattia di Anna aveva sostituito la governante francese, seduta accanto al ragazzo, con un lavoro di merletto a maglia tra le mani, si alzò in fretta, fece un inchino e scosse Serëza. 

Aleksej Aleksandrovic carezzò con la mano il figlio sui capelli, rispose alla governante sulla salute della moglie, e domandò che cosa avesse detto il medico della baby. 

- Il dottore ha detto che non c'è nulla da impensierirsi e ha ordinato dei bagni, signore. 

- Ma lei soffre sempre - disse Aleksej Aleksandrovic, prestando orecchio al piagnucolio della bambina nella camera accanto. 

- Io penso che la balia non sia adatta, signore - disse decisa l'inglese. 

- Perché lo pensate? - egli domandò fermandosi. 

- È successo così dalla contessa Pol', signore. Curavano il bambino e poi capirono che il bambino aveva semplicemente fame: la balia non aveva latte, signore. 

Aleksej Aleksandrovic si fece pensieroso e, fermatosi per alcuni secondi, entrò nell'altra stanza. La bambina era lì supina, con la testa rovesciata all'indietro, e, dibattendosi in braccio alla balia, non voleva attaccarsi al petto pieno che le veniva offerto, né acquietarsi malgrado il doppio zittio della balia e della njanja, curve su di lei. 

- Nessun miglioramento ancora? - disse Aleksej Aleksandrovic. 

- È molto inquieta - rispose sottovoce la njanja. 

- Miss Edward dice che forse la balia non ha latte - egli disse. 

- Anch'io lo penso, Aleksej Aleksandrovic. 

- Come mai non lo dite? 

- A chi dirlo? Anna Arkad'evna è sempre ammalata - disse la njanja scontenta. 

La njanja era una vecchia donna di casa. Anche in queste sue semplici parole parve ad Aleksej Aleksandrovic di scorgere un'allusione alla propria situazione. 

La bambina gridava ancora più forte rimanendo senza fiato e rantolando. La njanja fece un gesto sconsolato con la mano, le si avvicinò, la prese dalle braccia della balia e cominciò a cullarla camminando. 

- Bisogna pregare il dottore di visitare la balia - disse Aleksej Aleksandrovic. 

La nutrice, tutta adorna, sana all'aspetto, spaventata all'idea di essere licenziata, mormorò qualcosa fra i denti e, nascondendo il vasto petto, sorrise sprezzante del dubbio sull'allattamento. Anche in questo sorriso parve ad Aleksej Aleksandrovic di scorgere una certa ironia verso la propria situazione. 

- Povera bambina! - disse la njanja acquietando la piccola, e seguitò a camminare. 

Aleksej Aleksandrovic si era seduto su di una sedia e col viso abbattuto e sofferente, guardava la njanja che andava avanti e indietro. 

Quando alla fine posero la bambina, finalmente acquietata, nel lettino profondo, e la njanja, accomodato il guancialino, se ne fu allontanata, Aleksej Aleksandrovic si alzò e, avanzando cauto in punta di piedi, si accostò alla piccola. Per un momento stette in silenzio, e con quello stesso viso triste guardò la bambina: ma improvvisamente un sorriso, che gli aggrinzò i capelli e la pelle sulla fronte, gli salì al viso, ed egli ugualmente piano uscì dalla camera. 

In sala da pranzo, bussò e ordinò al cameriere di mandare di nuovo per il dottore. Era irritato con la moglie che non si preoccupava di quella bambina deliziosa e non voleva in tale disposizione di spirito andare da lei, né vedere la principessa Betsy; ma la moglie avrebbe potuto meravigliarsi ch'egli non passasse da lei come al solito, e perciò, dominandosi, si diresse in camera. Si avvicinò sul tappeto morbido accanto alla porta, e sentì involontariamente una conversazione che non avrebbe voluto ascoltare. 

- Se non partisse, capirei il vostro rifiuto e anche quello di lui. Ma vostro marito deve essere superiore a questo - diceva Betsy. 

- Io non lo voglio, né per mio marito, né per me. Non me ne parlate - rispondeva la voce agitata di Anna. 

- Già, ma voi non potete non desiderare di perdonare a un uomo che si è sparato per voi\ldots{} 

- Appunto per questo non voglio. 

Aleksej Aleksandrovic si fermò con una espressione di spavento, come se fosse colpevole, e fece per tornare indietro; ma questo gli parve indegno, si voltò di nuovo e, dopo aver tossito, si diresse verso la camera. Le voci tacquero ed egli entrò. 

Anna in vestaglia grigia, coi capelli neri tagliati corti a spazzola sulla testa rotonda, sedeva su di un letto basso. Come sempre, alla vista del marito, l'animazione sul suo viso scomparve; abbassò la testa e guardò inquieta Betsy. Betsy vestiva all'ultima moda, stravagante: con un cappellino che si poggiava chissà dove sulla testa, come un paralume su di una lampada, e con un abito nero-azzurro a strisce marcate trasversali che correvano sulla vita in un senso e sulla gonna nell'altro, sedeva accanto ad Anna, drizzando il busto alto e piatto, e, chinando il capo, accolse con un sorriso ironico Aleksej Aleksandrovic. 

- Ah - disse con sorpresa. - Sono molto contenta che siate in casa. Non vi si vede in nessun posto, e io non vi ho visto dal tempo della malattia di Anna. Ho saputo tutto\ldots{} le vostre premure! Sì, siete un marito straordinario! - disse con un fare significativo e affabile, come se gli concedesse l'ordine cavalleresco della magnanimità per il suo comportamento verso la moglie. 

Aleksej Aleksandrovic si inchinò con freddezza e, baciata la mano alla moglie, le chiese della sua salute. 

- Sto meglio, mi pare - ella disse, evitando lo sguardo di lui. 

- Ma avete un colorito di febbre - egli disse, calcando la parola ``febbre''. 

- Abbiamo parlato troppo io e lei - disse Betsy; - sento che è egoismo da parte mia e me ne vado. - Si alzò, ma Anna, divenuta rossa d'un tratto, le afferrò in fretta il braccio. 

- No, rimanete ancora, vi prego. Ho bisogno di dire a voi\ldots{} no, a voi - si rivolse ad Aleksej Aleksandrovic, e il rosso le coprì il collo e la fronte. - Io non voglio e non posso avere nulla che vi sia nascosto - ella disse. 

Aleksej Aleksandrovic fece scricchiolare le dita e abbassò il capo. 

- Betsy diceva che il conte Vronskij desiderava venire da noi per salutare prima della sua partenza per Taškent. - Ella non guardava il marito e, evidentemente, si affrettava a dire tutto, per quanto questo le fosse penoso. - Io ho detto che non posso riceverlo. 

- Voi avete detto, amica mia, che questo dipendeva da Aleksej Aleksandrovic - corresse Betsy. 

- Ma no, non lo posso ricevere, e questo non avrà nessun\ldots{} - D'un tratto si fermò e guardò interrogativamente il marito (egli non la guardava). - In una parola, non voglio. 

Aleksej Aleksandrovic si mosse e fece per prenderle la mano. Seguendo il primo impulso, ella ritirò la mano da quella di lui, umida, dalle grosse vene gonfie, che cercava la sua, ma, facendo uno sforzo evidente su di sé, gliela strinse. 

- Vi ringrazio molto per la vostra fiducia, ma\ldots{} - disse, sentendo turbamento e irritazione per il fatto che quello che avrebbe potuto decidere facilmente e chiaramente da solo, non lo poteva fare in presenza della principessa Tverskaja che gli si presentava come la personificazione di quella tale forza volgare che avrebbe dovuto dirigere la sua vita agli occhi del mondo e che gli impediva di abbandonarsi al suo sentimento di amore e di perdono. Si fermò, guardando la principessa Tverskaja. 

- Su, addio, tesoro mio - disse Betsy, alzandosi. Baciò Anna e uscì. Aleksej Aleksandrovic la accompagnò. 

- Aleksej Aleksandrovic! Io vi conosco per un uomo veramente generoso - disse Betsy fermandosi nel piccolo salotto e stringendogli in modo particolarmente forte ancora una volta la mano. - Io sono una persona estranea, ma voglio tanto bene a lei e stimo tanto voi che mi permetto darvi un consiglio. Ricevetelo; Aleksej, Vronskij è l'onore personificato, ed egli parte per Taškent. 

- Vi ringrazio, principessa, per il vostro interessamento e per i vostri consigli. Ma la questione se mia moglie possa o non possa ricevere qualcuno la risolverà lei stessa. 

Disse ciò, sollevando per abitudine le sopracciglia e subito pensò che, quali che fossero le sue parole, non poteva esserci dignità nella sua posizione. E questo egli scorse nel sorriso contenuto, cattivo e ironico, col quale Betsy lo guardò dopo questa frase. 

\capitolo{XX}\label{xx-3} 

Aleksej Aleksandrovic fece un inchino a Betsy nella sala e andò dalla moglie. Ella s'era sdraiata, ma, sentendo i passi di lui, riprese in fretta la posizione di prima, e lo guardò con spavento. Egli s'accorse che aveva pianto. 

- Ti sono molto grato per la fiducia che hai in me - ripeté sommessamente in russo la frase detta in francese davanti a Betsy, e sedette accanto a lei. Quando egli parlava in russo e le dava del ``tu'', questo ``tu'' irritava irresistibilmente Anna. - E molto grato per la tua decisione. Anch'io ritengo che, poiché parte, non c'è nessun bisogno per il conte Vronskij di venire qua. Del resto\ldots{} 

- Ma l'ho già detto, perché ripeterlo? - l'interruppe Anna con un'irritazione che non fece in tempo a contenere. ``Nessun bisogno - ella pensava - per un uomo di venire a salutare la donna che ama, per la quale voleva morire e s'è rovinato, per la donna che non può vivere senza di lui. Nessuna necessità!''. Strinse le labbra e abbassò gli occhi scintillanti sulle mani di lui dalle vene gonfie che lentamente si fregavano l'una contro l'altra. - Non ne parliamo più - soggiunse, più calma. 

- Io ti ho lasciata libera di decidere da sola questa faccenda, e sono molto contento di vedere\ldots{} - cominciò a dire Aleksej Aleksandrovic. 

- \ldots{}che il mio desiderio si incontra col vostro - finì svelta lei, irritata dal fatto ch'egli parlasse così lentamente, quando ella sapeva già in precedenza quello che avrebbe detto. 

- Già - confermò lui - e la principessa Tverskaja s'intromette del tutto a sproposito nelle più delicate situazioni familiari. Proprio lei\ldots{} 

- Io non credo affatto a quello che si dice sul suo conto - disse in fretta Anna - so soltanto che mi vuole bene sinceramente. 

Aleksej Aleksandrovic sospirò e tacque. Ella giocava nervosamente con le nappine della vestaglia, guardandolo con quel tormentoso senso di repulsione fisica che si rimproverava, ma che non riusciva a vincere. Adesso ella desiderava una cosa sola: essere liberata dalla presenza spiacevole di lui. 

- Ora ho mandato a chiamare il medico - disse Aleksej Aleksandrovic. 

- Io sto bene; perché il medico per me? 

- No, la piccola grida e dicono che la balia abbia poco latte. 

- Perché non mi hai consentito di allattarla, quando io lo volevo tanto? Sempre lo stesso - Aleksej Aleksandrovic capì cosa significava questo ``sempre lo stesso''; - è una bambina, e la fanno morire. - Sonò e ordinò di portare la bambina. - Ho chiesto di allattarla, non me l'hanno permesso, e ora si rimprovera proprio me. 

- Io non rimprovero\ldots{} 

- No, voi rimproverate. Dio mio! Perché non sono morta! - E si mise a singhiozzare. - Perdonami, sono irritata, sono ingiusta - disse rientrando in sé. - Ma va'\ldots{} 

``No, così non può durare'' si disse deciso Aleksej Aleksandrovic, uscendo dalla camera della moglie. 

L'assurdità della sua posizione agli occhi del mondo e l'odio di sua moglie verso di lui e, più di tutto, la prepotenza di quella forza volgare che, pur nell'orientamento del suo spirito, guidava la sua vita pratica e chiedeva l'adempimento delle sue esigenze, il cambiamento, cioè, dei suoi rapporti con la moglie, non gli si erano mai finora presentati innanzi alla mente con tanta evidenza come in quel momento. Vedeva chiaramente che tutto il mondo e la moglie pretendevano da lui qualche cosa, ma che cosa precisamente pretendessero non gli riusciva di capire. Sentiva che, per questo, nell'animo suo si faceva strada un sentimento cattivo che distruggeva la sua calma e tutto il merito della sua azione. Considerava che per Anna sarebbe stato meglio spezzare i rapporti con Vronskij, ma se tutti gli altri trovavano che questo era possibile, era pronto persino ad ammettere di nuovo questi rapporti, pur di non coprire di vergogna i bambini, pur di non esserne privato e di non cambiare la sua posizione. Per quanto questo fosse male, era sempre preferibile a una rottura per cui ella sarebbe rimasta in una posizione senza via d'uscita, umiliante e lui stesso sarebbe stato privato di tutto quello che amava. Ma si sentiva senza forze; sapeva già che tutti erano contro di lui e che non gli avrebbero consentito di fare quello che sembrava così naturale e buono e che l'avrebbero obbligato a fare quello che era un male, ma che credevano si dovesse fare. 

\capitolo{XXI}\label{xxi-3} 

Betsy non aveva ancora fatto in tempo a uscire dalla sala che Stepan Arkad'ic, venuto or ora da Eliseev, dove erano arrivate le ostriche fresche, le venne incontro sulla porta. 

- Oh, principessa, quale felice incontro! - cominciò a dire. - E io che sono stato da voi. 

- Incontro di un attimo, perché vado via - disse Betsy, sorridendo e infilando un guanto. 

- Aspettate, principessa, a infilarvi il guanto, datemi a baciare la vostra manina. Per nessuna cosa sono così grato al ritorno delle vecchie mode, quanto per il bacio delle mani. - E baciò la mano di Betsy. - Quando ci vedremo allora? 

- Voi non lo meritate - rispose Betsy sorridendo. 

- No, lo merito molto, perché sono diventato più serio. Non solo metto a posto le mie, ma anche le faccende familiari altrui - disse con un'espressione significativa del viso. 

- Ah, sono molto contenta! - rispose Betsy, avendo subito capito che parlava di Anna. E, tornati in sala, stettero in piedi in un angolo. - Egli la farà morire - disse Betsy con un mormorio significativo. - È impossibile, impossibile\ldots{} 

- Sono molto contento che voi pensiate così - disse Stepan Arkad'ic scotendo il capo con un'espressione seria e piena di compassione; - sono venuto per questo a Pietroburgo. 

- Tutta la città ne parla - ella disse. - È una situazione impossibile. Lei si consuma. Egli non capisce che lei è una di quelle donne che non possono scherzare con i loro sentimenti. Una delle due: o egli la porta via con un atto energico, o dà il divorzio. Ma questo stato la soffoca. 

- Sì, sì, proprio\ldots{} - disse Oblonskij, sospirando. - Io perciò sono venuto. Cioè non proprio per questo\ldots{} Mi hanno fatto ciambellano, via, bisogna pure ringraziare. Ma, soprattutto, bisogna accomodare questa faccenda. 

- Che Iddio vi aiuti - disse Betsy. 

Accompagnata la principessa Betsy fino all'ingresso, baciatale ancora una volta la mano più su del guanto, là dove batte il polso, e, lanciatele ancora delle amenità tanto poco convenienti ch'ella non sapeva più se arrabbiarsi o riderne, Stepan Arkad'ic entrò dalla sorella. La trovò in lacrime. 

Malgrado la disposizione d'animo sprizzante allegria in cui si trovava, Stepan Arkad'ic passò subito con naturalezza a quel tono compassionevole, poeticamente eccitato che si confaceva all'umore di lei. Le chiese della sua salute e come avesse passato la mattina. 

- Molto, molto male. E così il giorno e la mattina e tutti i giorni passati e futuri - ella disse. 

- Mi pare che tu ti abbandoni alla tetraggine. Bisogna scuotersi; bisogna guardare in faccia la vita. Lo so che è penoso, ma\ldots{} 

- Ho sentito che le donne amano gli uomini anche per i loro vizi - cominciò improvvisamente Anna - ma io lo odio per la sua virtù. Io non posso vivere con lui. Intendimi, il suo aspetto fisico agisce su di me, esco fuori di me. Non posso, non posso vivere con lui. Che fare mai? Ero infelice e pensavo non si potesse essere ancora più infelice di così, ma questa orribile posizione nella quale ora sono, non potevo immaginarla. Lo credi che, pur sapendo che egli è un uomo buono, eccellente, che io non valgo una sua unghia, tuttavia, io lo odio? Lo odio proprio per la sua generosità. Non mi resta nulla, tranne\ldots{} 

Voleva dire la morte, ma Stepan Arkad'ic non le dette il tempo di finire. 

- Sei malata ed eccitata - egli disse, - credimi, esageri terribilmente. Qui non c'è nulla di così spaventoso. 

E Stepan Arkad'ic sorrise. Nessuno al posto di Stepan Arkad'ic dinanzi a una simile disperazione, si sarebbe permesso di sorridere (il sorriso sarebbe parso volgare), ma nel sorriso di lui v'era tanta bontà e quasi una tenerezza femminile, che invece di offendere, raddolciva e calmava. 

I suoi discorsi, calmi e rasserenanti, e i suoi sorrisi agivano in modo da ammorbidire come olio di mandorla. E Anna ne sentì subito l'effetto. 

- No, Stiva - ella disse. - Sono perduta, sono perduta! Peggio che perduta. Ancora non sono perduta, non posso dire che tutto sia finito; al contrario, sento che non è finito. Sono come una corda tesa che deve spezzarsi. Ma ancora non è finito\ldots{} e finirà in modo orribile. 

- Ma no, si può adagio adagio allentare la corda. Non vi è situazione dalla quale non si possa uscire. 

- Ho pensato e ho pensato. Soltanto una\ldots{} 

Di nuovo egli capì dal suo sguardo spaventato che quest'unica via d'uscita, per lei, era la morte, e non le permise di finire. 

- Nient'affatto - disse. - Permetti. Tu non puoi vedere la situazione come me. Permettimi di dire apertamente la mia opinione. - Di nuovo egli sorrise timido col suo sorriso all'olio di mandorla. - Comincio dal principio; ti sei sposata con un uomo che aveva vent'anni più di te. Ti sei sposata senz'amore o senza conoscere l'amore. Questo è stato un errore, ammettiamolo. 

- Terribile errore! - disse Anna. 

- Ma io ripeto: quel che è fatto è fatto. Dopo hai avuto, diciamo pure, la sventura di amare chi non era tuo marito. È una sventura, ma anche questo è un fatto compiuto. E tuo marito l'ha riconosciuto come tale e ti ha perdonato. - Egli si fermava dopo ogni frase, aspettando le obiezioni di lei, ma lei non rispondeva nulla. - Così è. Adesso la questione è questa: puoi continuare a vivere con tuo marito? Lo desideri? Lo desidera lui? 

- Io non so nulla, nulla. 

- Ma tu stessa hai detto che non puoi sopportarlo. 

- No, non l'ho detto. Lo ritratto. Io non so nulla e non capisco nulla. 

- Sì, ma permetti\ldots{} 

- Tu non puoi capire. Io sento che precipito con la testa in giù in un abisso, ma che non devo salvarmi, e non posso! 

- Non è niente, noi stenderemo qualcosa sotto di te e ti afferreremo. Ti capisco, capisco che non ti senti di assumere la responsabilità di esprimere il tuo desiderio, il tuo sentimento. 

- Io non desidero nulla, nulla\ldots{} solo che tutto finisca. 

- Ma egli lo vede questo e lo sa. E credi forse che non ne senta, quanto te, tutta la pena? Tu ti tormenti, lui si tormenta e che ne viene fuori? Mentre il divorzio risolverebbe tutto - disse Stepan Arkad'ic, manifestando non senza sforzo il proprio pensiero preminente e guardandola con intenzione. 

Ella non rispose nulla e scosse negativamente il capo dai capelli corti. Ma, dall'espressione del viso che improvvisamente s'era acceso della bellezza d'un tempo, egli capiva ch'ella rifiutava tale soluzione solo perché le pareva una felicità irraggiungibile. 

- Mi fate tanta tanta pena! E come sarei felice se potessi compier tutto questo! - disse Stepan Arkad'ic sorridendo ormai più coraggiosamente. - Non dire, non dire nulla. Se Dio mi concedesse solo di parlare così come sento. Andrò da lui. 

Anna con gli occhi pensosi e splendenti lo guardò, e non disse nulla. 

\capitolo{XXII}\label{xxii-3} 

Stepan Arkad'ic, con quell'aria alquanto solenne con cui prendeva posto nella poltrona presidenziale del suo ufficio, entrò nello studio di Aleksej Aleksandrovic. Aleksej Aleksandrovic camminava con le mani dietro la schiena su e giù per la stanza e pensava alle stesse cose di cui Stepan Arkad'ic aveva parlato con la moglie. 

- Non ti disturbo? - disse Stepan Arkad'ic, provando un insolito senso di smarrimento alla vista del cognato. Per nascondere il turbamento, cacciò fuori un portasigarette, da poco acquistato, munito di un nuovo sistema di apertura e, annusandone la pelle, ne trasse fuori una sigaretta. 

- No. Ti occorre qualcosa? - rispose di malavoglia Aleksej Aleksandrovic. 

- Sì, volevo\ldots{} ho bisogno\ldots{} già, ho bisogno di parlarti - disse Stepan Arkad'ic, sentendo con sorpresa un'insolita timidezza. 

E ciò era così insolito e strano per lui che Stepan Arkad'ic non volle pensare che potesse essere dovuto alla voce della coscienza che gli presentava come male quello che stava per fare. Fece uno sforzo su di sé e vinse la timidezza che lo aveva pervaso. 

- Spero che tu creda al mio affetto per mia sorella e al sincero legame e alla stima che ho per te - disse, arrossendo. 

Aleksej Aleksandrovic si fermò e non rispose nulla, ma il suo volto colpì Stepan Arkad'ic per l'espressione di vittima sottomessa. 

- Avevo intenzione, volevo parlare di mia sorella, della vostra reciproca situazione - disse Stepan Arkad'ic, lottando ancora con l'insolita timidezza. 

Aleksej Aleksandrovic sorrise triste, guardò il cognato e, senza rispondere, si accostò al tavolo, ne trasse fuori una lettera cominciata e la dette al cognato. 

- Penso continuamente alla stessa cosa. Ed ecco quello che avevo cominciato a scrivere, ritenendo più opportuno parlarle per lettera, evitando così che la mia presenza la irriti - disse, porgendo la lettera. 

Stepan Arkad'ic prese la lettera, e con stupore e perplessità guardò gli occhi appannati, immobilmente fissi su di lui, e cominciò a leggere. 

\begin{quote}
``Vedo che la mia presenza vi è di peso. Per quanto possa essere doloroso per me convincermene, vedo che è così e che non può essere diversamente. Io non vi accuso, e Dio mi è testimone che, da quando vi ho visto durante la vostra malattia, ho deciso con tutta l'anima di dimenticare tutto quello che era stato tra di noi e di cominciare una nuova vita. Io non mi pento e non mi pentirò mai di quello che ho fatto; desideravo una cosa sola: il vostro bene, il bene della vostra anima, e adesso vedo che questo non l'ho raggiunto. Ditemi voi stessa che cosa può dare pace e felicità all'anima vostra. Io mi rimetto alla vostra volontà e al vostro senso di giustizia''.
\end{quote} 

Stepan Arkad'ic restituì la lettera e con la stessa perplessità di prima continuò a guardare il cognato senza sapere che dire. Questo silenzio era per entrambi così penoso che le labbra di Stepan Arkad'ic ebbero un tremito morboso, mentre taceva senza levar gli occhi dal viso di Karenin. 

- Ecco quello che io volevo dirle - disse Aleksej Aleksandrovic, voltandosi da un'altra parte. 

- Sì, sì\ldots{} - disse Stepan Arkad'ic, senza avere la forza di rispondere, giacché le lacrime gli venivano alla gola. - Sì, sì, vi capisco - pronunciò alla fine. 

- Io desidero sapere quello ch'ella vuole - disse Aleksej Aleksandrovic. 

- Penso ch'ella stessa non intenda la propria situazione. Ella non può esserne l'arbitra - diceva Stepan Arkad'ic, riprendendosi. - È schiacciata, proprio schiacciata dalla tua generosità. Se leggerà questa lettera non avrà la forza di nulla, abbasserà solo più in giù il capo. 

- Già, ma che fare, dunque, in un caso simile? Come intuire, come conoscere i suoi desideri? 

- Se mi permetti di dire la mia opinione, io penso che dipenda da te indicare quella misura che ritieni necessaria per far cessare questo stato di cose. 

- Allora tu ritieni che è necessario far cessare? - lo interruppe Aleksej Aleksandrovic. - Ma come? - aggiunse, dopo aver fatto con le mani un insolito gesto davanti agli occhi. - Non vedo possibile nessuna via d'uscita. 

- In qualsiasi situazione c'è sempre una via d'uscita - disse, alzandosi e animandosi, Stepan Arkad'ic. - Vi è stato un momento quando tu volevi rompere\ldots{} Se adesso ti convincerai che non potete formare l'uno la felicità dell'altra\ldots{} 

- La felicità si può intendere in vari modi. Ma poniamo che io sia d'accordo, che non voglia nulla. Quale via d'uscita dalla nostra posizione? 

- Se tu vuoi la mia opinione - disse Stepan Arkad'ic con lo stesso tenero, dolce sorriso all'olio di mandorla, col quale aveva parlato ad Anna. Quel buon sorriso era così suadente che Aleksej Aleksandrovic senza volere, sentendo la propria debolezza e sottomettendovisi, era pronto a credere quello che avrebbe detto Stepan Arkad'ic. - Ella non lo dirà mai. Ma una cosa sola può desiderare - continuò Stepan Arkad'ic - questa: che cessino i rapporti e i ricordi ad essi collegati. Secondo me, nella vostra situazione, è indispensabile la chiarificazione di nuovi rapporti. E questi rapporti possono stabilirsi solo con la libertà delle due parti. 

- Il divorzio - interruppe con avversione Aleksej Aleksandrovic. 

- Sì, io ritengo che sia necessario il divorzio. Sì, il divorzio - ripeté, arrossendo, Stepan Arkad'ic. - Sotto tutti gli spetti è la via d'uscita più ragionevole per coniugi che siano in rapporti come i vostri. Che fare mai se i coniugi sentono che per loro la vita in comune è impossibile? Questo può sempre accadere. - Aleksej Aleksandrovic sospirò pesantemente e chiuse gli occhi. - Qui c'è solo una considerazione da fare: desidera uno dei coniugi contrarre altro matrimonio? Se no, la cosa è molto semplice - disse Stepan Arkad'ic liberandosi sempre più dal disagio. 

Aleksej Aleksandrovic, corrugato il viso per l'emozione, mormorò qualcosa fra sé e sé e non rispose nulla. Tutto quello che a Stepan Arkad'ic pareva così semplice, Aleksej Aleksandrovic lo aveva pensato mille volte. E tutto questo gli pareva, non solo tutt'altro che semplice, ma del tutto impossibile. Il divorzio, i cui particolari già conosceva, gli pareva impossibile perché il sentimento di dignità personale e il rispetto per la religione non gli consentivano di assumere l'accusa fittizia di adulterio e ancora meno di ammettere che la moglie da lui perdonata e amata fosse riconosciuta colpevole e perduta. Il divorzio si presentava, inoltre, impossibile per altre, ancora più importanti ragioni. 

Che ne sarebbe stato del figlio in caso di divorzio? Lasciarlo con la madre non era possibile. La madre divorziata avrebbe avuto una sua famiglia illegittima nella quale la situazione di figliastro e la sua educazione sarebbero state, con ogni probabilità, poco buone. Tenerlo con sé? Questo, lo sapeva, sarebbe stata una cattiva vendetta da parte sua e non voleva. Inoltre, il divorzio sembrava ad Aleksej Aleksandrovic la cosa più inopportuna, perché, acconsentendo al divorzio, egli avrebbe proprio con questo fatto la rovina di Anna. Gli era rimasta nell'animo la frase detta da Dar'ja Aleksandrovna, a Mosca, che, decidendosi al divorzio, avrebbe pensato soltanto a sé, ma che avrebbe rovinato lei irrimediabilmente. E quella frase, congiunta nel suo pensiero, al perdono, all'affetto per i suoi bambini, egli la intendeva, ora, a modo suo. Consentire al divorzio, dare a lei la libertà significava troncare l'ultimo suo legame con la vita dei bambini che amava e privar lei dell'ultimo sostegno sulla via del bene, e gettarla nella rovina. S'ella fosse diventata una moglie divorziata, si sarebbe unita, egli lo sapeva, a Vronskij, e questo legame era illegale e colpevole perché per la moglie, secondo la legge della Chiesa, non può esservi altro matrimonio finché è vivo il marito. ``Si unirà a lui e, dopo un anno o due, egli l'abbandonerà e lei contrarrà un nuovo legame - pensava Aleksej Aleksandrovic. - E io, avendo acconsentito a un divorzio al di fuori della legge, sarò il responsabile della sua perdizione''. Aveva riflettuto a questo centinaia di volte ed era convinto che la soluzione del divorzio, non solo non era molto semplice come diceva il cognato, ma del tutto inaccettabile. Non credeva neppure a una parola di quello che diceva Stepan Arkad'ic, per ogni sua parola aveva migliaia di obiezioni, ma l'ascoltava, perché sentiva che nelle parole di lui trovava espressione quella tale prepotente forza volgare che guidava la sua vita e alla quale avrebbe dovuto sottomettersi. 

- La questione è solo nello stabilire a quali condizioni tu acconsentirai al divorzio. Ella non vuole nulla, e non osa chiederti nulla, si rimette completamente alla tua generosità. 

``Dio mio! Dio mio! Perché?'' pensava Aleksej Aleksandrovic ricordando i particolari di quel tale divorzio nel quale il marito assumeva la colpa dell'adulterio, e con quello stesso gesto con quale Vronskij si era coperto il viso, egli coprì con le mani il suo, sopraffatto dalla vergogna. 

- Tu sei agitato, capisco. Ma se rifletterai\ldots{} 

``E a colui che avrà percosso la tua guancia destra, tendi la sinistra; e a colui che ti avrà tolto il mantello da' la tunica'' diceva a se stesso Aleksej Aleksandrovic. 

- Sì, sì - gridò con voce stridula - prenderò su di me il disonore, darò anche mio figlio, ma\ldots{} non sarebbe meglio? Del resto, fa' quello che vuoi\ldots{} 

E voltatosi in modo che il cognato non potesse vederlo, sedette su di una seggiola accanto alla finestra. Quanta amarezza, quanta vergogna! ma insieme con quest'amarezza e con questa vergogna erano in lui la gioia e la commozione che gli venivano dall'altezza della propria umiltà. 

Stepan Arkad'ic era commosso. Tacque per un po'. 

- Aleksej Aleksandrovic, credimi, ella apprezzerà la tua generosità - egli disse. - Ma si vede, era la volontà di Dio - soggiunse e, detto questo, sentì d'essere stato idiota, e trattenne a stento un sorriso sulla propria idiozia. 

Aleksej Aleksandrovic voleva rispondere qualcosa, ma le lacrime glielo impedirono. 

- È una sventura del destino, ci si deve rassegnare. Riconosco questa sventura come un fatto compiuto e cerco di venire in aiuto a lei e a te - disse Stepan Arkad'ic . 

Quando Stepan Arkad'ic uscì dalla stanza del cognato, era commosso, ma questo non gl'impedì d'essere soddisfatto d'aver condotto a termine felicemente la faccenda, giacché era convinto che Aleksej Aleksandrovic non avrebbe ritrattato le sue parole. A questa soddisfazione si frammischiava l'idea venutagli in mente che, ad affare concluso, avrebbe potuto chiedere alla moglie e agli intimi:``Che differenza c'è fra me e un monarca? Il monarca fa il cambio della guardia e di questo nessuno si avvantaggia ed io, invece, ho portato a termine un divorzio e tre persone ne trarranno vantaggio''. Oppure: ``Che somiglianza c'è fra un monarca e me? Quando\ldots{} Ma\ldots{} ci penserò meglio'' disse fra sé con un sorriso. 

\capitolo{XXIII}\label{xxiii-3} 

La ferita di Vronskij era stata pericolosa, pur avendo risparmiato il cuore. E per alcuni giorni fu tra la vita e la morte. Quando per la prima volta si sentì in condizioni di parlare, nella camera c'era solamente Varja, la moglie del fratello. 

- Varja - egli disse, guardandola - mi sono sparato addosso inavvertitamente. E, ti prego, cerca di non parlarne; ad ogni modo di' così a tutti. Altrimenti, è troppo sciocco! 

Senza rispondere alle sue parole, Varja si chinò su di lui e ne scrutò il viso. Gli occhi erano limpidi, non più febbrili, ma la loro espressione era dura. 

- Via, sia lodato Iddio! - ella disse. - Non ti fa male? 

- Un poco, qui. - E indicò il petto. 

- Allora da' qua, che ti fascio. 

Stringendo in silenzio gli zigomi larghi, Vronskij la guardava mentre ella lo fasciava. Quando ebbe finito, disse: 

- Non sono in stato di delirio; ti prego, fa' in modo che non corra voce che mi sono sparato di proposito. 

- Nessuno lo dice neppure. Soltanto spero che non sparerai più inavvertitamente - ella disse con un sorriso interrogativo. 

- Così dev'essere, non lo farò più, ma sarebbe stato meglio\ldots{} 

E sorrise cupo. 

Malgrado queste parole e il sorriso che avevano spaventato Varja, quando passò l'infiammazione e cominciò a rimettersi, sentì d'essersi completamente liberato di una parte del suo dolore. Era come se, con quell'atto, si fosse liberato della vergogna e dell'umiliazione che aveva provato prima. Ora poteva tranquillamente pensare al Aleksej Aleksandrovic. Riconosceva tutta la generosità di lui, ma non sentiva più la propria umiliazione. Inoltre s'era rimesso di nuovo sulla carreggiata della sua vita precedente. Vedeva che poteva guardare senza vergogna la gente negli occhi e vivere, facendosi guidare dalle sue abitudini. L'unica cosa che non poteva strappare dal proprio cuore, malgrado lottasse senza tregua con quel sentimento, era il rimpianto, che giungeva fino alla disperazione, di averla perduta per sempre. Che egli adesso, riscattata davanti al marito la propria colpa, dovesse rinunciare a lei e non porsi mai più fra il rimorso di lei e il marito, era cosa fermamente decisa nel suo cuore; ma dal cuore non poteva strappare il rimpianto d'aver perduto l'amore di lei, non poteva cancellare dalla memoria quei momenti di felicità che aveva conosciuto con lei e che aveva così poco apprezzato allora, mentre adesso lo perseguitavano con tutto il loro fascino. 

Serpuchovskoj aveva escogitato per lui la nomina a Taškent, e Vronskij, senza la minima esitazione, aveva acconsentito a questa proposta. Ma quanto più si avvicinava il momento della partenza, tanto più penoso si faceva per lui il sacrificio che compiva perché lo considerava doveroso. 

La ferita era cicatrizzata ed egli già usciva per fare i preparativi per la partenza per Taškent. 

``Vederla una volta e poi scomparire, morire'' pensava e, facendo le visite di congedo, espresse questo pensiero a Betsy. Con questo incarico Betsy era andata da Anna e aveva riferito la risposta negativa. 

``Tanto meglio - aveva pensato Vronskij, dopo aver ricevuto questa notizia. - Era una debolezza che avrebbe rovinato le mie ultime risorse''. 

Il giorno dopo la stessa Betsy andò da lui la mattina e gli disse di aver ricevuto per mezzo di Oblonskij la sicura notizia che Aleksej Aleksandrovic consentiva al divorzio e che perciò egli poteva vederla. 

Senza curarsi neppure di accompagnare a casa Betsy, dimentico di tutte le proprie decisioni, senza chiedere quando si poteva, e dove fosse il marito, Vronskij andò subito dai Karenin. Corse per la scala, senza veder nulla e nessuno e, a passo veloce, entrò, trattenendosi appena dal correre, nella camera di lei. E senza pensare e senza notare se ci fosse qualcuno in camera, abbracciò e coprì di baci il viso, le mani, il collo di lei. 

Anna si era preparata a quest'incontro, aveva pensato a quello che gli avrebbe detto, ma non ebbe il tempo di dire nulla: la passione l'afferrò. Voleva calmar lui, calmare se stessa, ma ormai era troppo tardi. L'eccitazione di lui le si era comunicata. Le sue labbra tremavano tanto che a lungo non poté articolar parola. 

- Sì, ti sei impadronito di me e io sono tua - pronunciò alla fine, stringendosi al petto le mani di lui. 

- Così doveva essere - egli disse. - Finché siamo vivi, così deve essere. È questo che io so, adesso. 

- È vero - diceva lei, impallidendo sempre più e abbracciandogli la testa. - Tuttavia c'è qualcosa di tremendo in questo, dopo quello che è stato. 

- Tutto passerà, tutto passerà: noi saremo tanto felici! Il nostro amore, se potesse, diverrebbe più forte, perché in esso c'è qualcosa di tremendo - egli disse, alzando la testa e scoprendo nel sorriso i suoi denti forti. 

Ella non poté non rispondere con un sorriso, non alle parole, ma agli occhi innamorati di lui. Gli prese una mano e si carezzò con questa le guance divenute fredde e i capelli tagliati. 

- Non ti riconosco con questi capelli corti. Così sei più bella. Sembri un ragazzo. Ma come sei pallida! 

- Sì, sono molto debole - ella disse, arrossendo. E le sue labbra di nuovo tremarono. 

- Andremo in Italia, ti rimetterai - egli disse. 

- Ma è possibile che noi siamo come marito e moglie, soli, con una famiglia nostra, io e tu? - disse lei, guardandolo da vicino negli occhi. 

- Mi stupiva soltanto, come una volta potesse essere diversamente. 

- Stiva dice che lui acconsente a tutto, ma io non posso accettare la sua generosità - ella disse, guardando pensosa al di là del viso di Vronskij. - Non voglio il divorzio; ora per me è lo stesso. Solo non so cosa deciderà per Serëza. 

Egli non riusciva in nessun modo a capire come ella potesse, nel momento del loro incontro, ricordarsi del figlio, e pensare al divorzio. Non era forse lo stesso? 

- Non parlare di questo, non pensare - disse, rigirando le mani di lei nella sua e cercando di attirarne l'attenzione; ma lei non lo guardava. 

- Ah, perché non sono morta, sarebbe stato meglio! - disse e, senza singhiozzi, le lacrime le colarono giù per le guance; ma cercava di sorridere per non addolorarlo. 

Secondo le idee che poco prima erano nella mente di Vronskij, rinunziare alla lusinghiera e pericolosa nomina a Taškent, sarebbe stato disonorevole e impossibile. Ma adesso, senza pensarci un attimo, vi rinunziò e, avendo notato che i suoi superiori disapprovavano il gesto, diede immediatamente le dimissioni. 

Dopo un mese, Aleksej Aleksandrovic rimase solo col figlio nel suo appartamento, e Anna partì per l'estero con Vronskij, senza aver ottenuto il divorzio e avendovi decisamente rinunciato. 

\parte{PARTE QUINTA}\label{parte-quinta} 

\capitolo{I}\label{i-4} 

La principessa Šcerbackaja trovava che celebrare il matrimonio prima della quaresima, alla quale mancavano cinque settimane, non era possibile, poiché metà del corredo non poteva essere pronto per quella data; ma non poteva non convenire con Levin che, dopo la quaresima, sarebbe stato già troppo tardi, poiché la vecchia zia paterna del principe Šcerbackij era molto ammalata e poteva morire da un momento all'altro, e allora il lutto avrebbe ostacolato il matrimonio. E perciò, dopo aver deciso di dividere il corredo in due parti, corredo di casa e corredo personale, la principessa acconsentì a celebrare il matrimonio prima della quaresima. Aveva deciso di preparare subito il corredo personale e di spedir dopo quello di casa, ma si irritava con Levin che non riusciva in nessun modo a risponderle seriamente se acconsentiva oppure no a questo. Una decisione simile era quanto mai opportuna, perché la giovane coppia, subito dopo il matrimonio, si sarebbe recata in campagna, dove il corredo di casa non sarebbe stato necessario. 

Levin continuava a essere sempre in quello stato di esaltazione nel quale gli era dato ritenere che lui e la sua felicità formassero il principale ed essenziale scopo di tutto quello che esisteva e che per adesso non gli occorresse pensare a preoccuparsi di nulla, che tutto per lui farebbero e avrebbero fatto gli altri. Non aveva neppure progetti e scopi per la vita futura; ne lasciava la decisione agli altri, sapendo che tutto sarebbe stato bellissimo. Suo fratello Sergej Ivanovic, Stepan Arkad'ic e la principessa lo guidavano in quello che doveva fare. Era completamente d'accordo su quello che gli proponevano. Il fratello aveva preso del denaro in prestito per lui, la principessa consigliava di partire per Mosca dopo il matrimonio, Stepan Arkad'ic di andare all'estero. Egli consentiva a tutto. ``Fate quel che volete, se ciò vi rallegra. Io sono felice e la felicità mia non può essere maggiore né minore, qualunque cosa facciate'' pensava. Quando riferì a Kitty il consiglio di Stepan Arkad'ic di andare all'estero, si sorprese molto ch'ella non acconsentisse e che avesse certe pretese ben definite sulla loro vita avvenire. Ella sapeva che Levin in campagna aveva un lavoro che amava. E non solo non intendeva quel lavoro, ma egli lo vedeva, non voleva intenderlo. Questo, però, non le impediva di ritenerlo molto importante. E poiché sapeva che la loro dimora sarebbe stata in campagna, non desiderava di andare all'estero dove non avrebbe vissuto, ma là dove sarebbe stata la loro dimora. Questa intenzione, espressa in modo definito, sorprese Levin. Ma poiché per lui ciò era indifferente, pregò subito Stepan Arkad'ic, come se questo fosse stato un suo dovere, di andare in campagna e di preparare là tutto quello ch'egli sapeva fare, con quel gusto di cui tanto disponeva. 

- Però senti - disse una volta Stepan Arkad'ic a Levin, tornando dalla campagna dove aveva preparato tutto per l'arrivo degli sposi - ce l'hai un certificato che attesti che ti sei confessato? 

- No. E perché? 

- Senza di questo non ci si può sposare. 

- Ahi, ahi, ahi! - gridò Levin. - Io, vedi, mi pare che sono nove anni che non mi son più comunicato. Non ci pensavo. 

- Bene! - disse ridendo Stepan Arkad'ic - e chiami nichilista me! Non se ne può fare a meno, tuttavia. Ti devi confessare. 

- E quando? Non ci sono che quattro giorni. 

Stepan Arkad'ic accomodò anche questo. Levin cominciò a prepararsi alla comunione. Per Levin, come per qualsiasi essere che non crede, ma che nello stesso tempo rispetta la fede degli altri, la presenza e la partecipazione a qualsiasi rito della Chiesa erano molto incresciose. Adesso, in quello stato in cui era di sensibilità e di intenerimento verso tutto, la necessità di fingere non solo gli era penosa, ma gli sembrava del tutto impossibile. Adesso, in quel suo stato di esaltazione ed effusione avrebbe dovuto o mentire o compiere un sacrilegio. Non si sentiva in grado di fare né l'una né l'altra cosa. Ma per quanto interrogasse Stepan Arkad'ic se fosse possibile ricevere il certificato senza confessarsi, Stepan Arkad'ic dichiarava che non era possibile. 

- Ma, del resto, che cosa ti costa? due giorni. Ed è un vecchietto simpaticissimo, intelligente. Ti caverà questo dente senza fartene accorgere. 

Nell'assistere alla prima messa, Levin si sforzò di ravvivare in sé le reminiscenze giovanili di quel forte sentimento religioso che aveva provato fra i sedici e i diciassette anni. Ma subito si convinse che era assolutamente impossibile. Cercò allora di considerare la cosa come una consuetudine senza senso, vuota, simile a quella di far le visite; ma sentì che neanche così poteva compierla in nessun modo. Levin si trovava, nei riguardi della religione, nella situazione più indefinita, come, del resto, la maggior parte dei suoi contemporanei. Credere non poteva, ma nello stesso tempo non era fermamente convinto che tutto questo fosse falso. E perciò, non essendo in grado di credere al significato di quello che compiva, né di considerarlo con indifferenza come vuota formalità, durante tutto il tempo della preparazione alla comunione, provò un senso di disagio e di vergogna, compiendo cose che lui stesso non intendeva e, perciò, come gli diceva una voce interna, qualcosa da falso e di poco buono. 

Durante il tempo delle funzioni sacre, ora ascoltava le preghiere, cercando di dar loro un significato che non si allontanasse dalle sue opinioni, ora, sentendo di non poterle intendere e di doverle criticare, si sforzava di non ascoltarle, e si occupava dei suoi pensieri, delle osservazioni e dei ricordi che con straordinaria vivezza gli erravano per la testa durante quell'ozioso stare in piedi in chiesa. 

Rimase durante tutta la messa, fino alle preghiere della sera e ai vespri, e l'indomani, alzatosi prima del solito, senza bere il tè, giunse alle otto in chiesa per ascoltare il mattutino e confessarsi. 

In chiesa non c'era nessuno all'infuori di un povero soldato, due vecchiette e i sacrestani. 

Un giovane diacono, con le due metà della schiena marcatamente delineate sotto la tunica sottile, gli venne incontro, e subito avvicinatosi a un tavolino presso il muro, cominciò a leggere le preghiere. A misura che la lettura andava avanti, in particolare alla frequente e veloce ripetizione delle parole: ``Signore, abbi pietà'' che risonavano come ``pietasign'', Levin sentiva che il suo pensiero era chiuso e sigillato e che in quel momento scuoterlo e commuoverlo non conveniva, ché ne sarebbe venuta fuori una confusione; e perciò, rimanendo in piedi dietro al diacono, continuava a pensare a se stesso, senza ascoltare né intendere. ``È sorprendente quanta espressione ci sia nella sua mano'' pensava, ricordando che il giorno prima erano rimasti a sedere accanto al tavolo d'angolo. Non avevano nulla da dirsi, come sempre capitava in quel tempo, e lei, posta una mano sul tavolo, l'aveva aperta e richiusa, e poi s'era messa a ridere da sola, guardandone il movimento. Egli ricordò come avesse baciato la mano e come poi si fosse messo a osservarne le linee convergenti sulla palma rosea: ``Di nuovo pietasign - pensò Levin, segnandosi, inchinandosi e guardando l'agile movimento della schiena del diacono che s'inginocchiava. - Lei poi ha preso la mia mano e ne ha esaminate le linee. Hai una bella mano, ha detto. - Ed egli guardò la propria mano e la mano tozza del diacono. - Ecco, ora finirà presto - pensava. - No, pare che cominci daccapo - pensò, prestando orecchio alle preghiere. - No, finisce; ecco che si è inchinato fino a terra. Questo avviene sempre prima della fine''. 

Il diacono, dopo aver preso, senza farsene accorgere, un biglietto da tre rubli nel manichino di velluto, disse che lo avrebbe iscritto e, facendo risonare sveltamente gli stivali nuovi sulle lastre di pietra della chiesa vuota, si avviò verso l'altare. Dopo un minuto ne uscì fuori e fece cenno a Levin. Il pensiero, fino a quel momento chiuso nella mente di Levin, si agitò, ma egli si affrettò a respingerlo. ``In qualche modo si farà'' pensò e si avviò verso l'ambone. Salì i gradini e, voltando a destra, vide il sacerdote. Un piccolo prete vecchio, con una barba rada, bianca a metà, con gli occhi stanchi, buoni, stava in piedi dinanzi a un leggio e volgeva i fogli del messale. Dopo essersi lievemente inchinato a Levin, cominciò subito a leggere le preghiere con la voce di chi ne ha l'abitudine. Quando finì, s'inchinò fino a terra e rivolse il viso verso Levin. 

- Qui Cristo è invisibilmente presente nell'accogliere la vostra confessione - disse, indicando un Crocifisso. - Credete voi tutto quello che ci insegna la Santa Chiesa Apostolica? - continuò il prete, distogliendo gli occhi dal viso di Levin e incrociando le mani sotto la stola. 

- Io ho dubitato, io dubito di tutto - disse Levin con voce sgradevole a se stesso, e tacque. 

Il sacerdote attese qualche secondo, se mai egli dicesse ancora qualcosa; poi, chiusi gli occhi, nella parlata veloce, in ``o'', di Vladimir, disse: 

- I dubbi sono propri della debolezza umana, ma noi dobbiamo pregare, affinché il Signore misericordioso ci renda forti. Quali peccati particolari avete? - aggiunse senza il più piccolo intervallo, quasi cercando di non perdere tempo. 

- Il mio peccato principale è il dubbio. Io dubito di tutto e mi trovo sempre nel dubbio. 

- Il dubbio è proprio della debolezza umana - ripeté il sacerdote con le stesse parole. - In che cosa, soprattutto, avete dei dubbi? 

- Io dubito di tutto. Dubito, a volte, perfino dell'esistenza di Dio - disse involontariamente Levin, e inorridì della sconvenienza di quel che aveva detto. Ma le parole di Levin non avevano prodotto impressione sul sacerdote, a quanto parve. 

- E quali dubbi vi possono essere sull'esistenza di Dio? - egli disse in fretta, con un sorriso appena percettibile. Levin taceva. 

- E che dubbio potete avere sul Creatore, quando guardate le Sue creazioni? - continuò il sacerdote, con la solita voce affrettata. - E chi ha abbellito di astri la volta celeste? Chi ha rivestito la terra della sua bellezza? Come sarebbe avvenuto tutto ciò senza un Creatore? - disse, guardando interrogativamente Levin. 

Levin sentiva che sarebbe stato sconveniente entrare in una discussione filosofica col sacerdote, e perciò dette in risposta solo quello che si riferiva all'interrogazione. 

- Non so - disse. 

- Non sapete? Ma come potete dubitare che Dio abbia creato tutto? - disse il sacerdote con sorpresa ilare. 

- Io non capisco nulla - disse Levin, diventando rosso e sentendo che le sue parole erano sciocche e non potevano non essere sciocche in una situazione come la sua. 

- Pregate Iddio e chiedete a Lui. Perfino i santi padri hanno avuto dubbi e hanno chiesto a Dio la conferma della loro fede. Il demonio ha un grande potere, e noi non dobbiamo sottometterci a lui. Pregate Iddio, chiedete a Lui. Pregate Iddio - ripeté in fretta. 

Il sacerdote tacque un poco, come se si fosse messo a riflettere. 

- Voi, come ho sentito, vi preparate a contrarre matrimonio con la figlia del mio fedele e figlio spirituale principe Šcerbackij? - aggiunse con un sorriso. - Una bellissima fanciulla! 

- Sì - rispose Levin, arrossendo per il sacerdote. ``Perché gli occorre di domandar questo in confessione?'' pensò. E, quasi in risposta al pensiero di lui, il prete disse: 

- Voi vi preparate a contrarre matrimonio, e Dio, forse, vi concederà una prole, non è vero? Ebbene, quale educazione potete dare ai vostri figli se non vincerete in voi stesso la tentazione del demonio, che vi trascina verso l'incredulità? - disse con rimprovero mite. - Se amate la vostra prole, voi, come buon padre di famiglia, augurerete al vostro figliuolo non solo la ricchezza, il fasto, gli onori; gli augurerete anche la salvezza, l'illuminazione spirituale attraverso la luce della verità. Non è forse così? E che cosa risponderete quando il fanciullo innocente vi chiederà: ``Padre, chi ha fatto tutto quello che mi rallegra in questo mondo, la terra, le acque, il sole, i fiori, le erbe?''. Possibile che gli rispondiate:``Non lo so''? Voi non potete non saperlo, quando, per Sua grande misericordia, il Signore Iddio ve lo ha rivelato. Oppure il figlio vostro vi chiederà: ``Cosa mi aspetta nella vita d'oltretomba?''. Cosa gli direte se non saprete nulla? E come gli risponderete? Lo abbandonerete alle tentazioni del diavolo e del mondo? Questo non è bene - disse, e si fermò, chinando la testa da un lato e guardando Levin con gli occhi miti, benevoli. 

Levin adesso non rispondeva nulla, non perché non volesse entrare in discussione con il sacerdote, ma perché nessuno mai gli aveva fatto tali domande, e prima che i suoi figli gli avessero poste queste domande, ci sarebbe stato ancora tempo per pensare e rispondere. 

- Voi entrate in un periodo della vita - proseguì il sacerdote - in cui bisogna scegliere una via e attenervisi. Pregate Dio, affinché per Sua bontà vi aiuti e abbia pietà di voi - concluse. - Il Signore e Iddio nostro Gesù Cristo, con la grazia divina e la liberalità del Suo amore per gli uomini, ti perdoni, o figlio\ldots{} - e, finita la preghiera di assoluzione, il sacerdote lo benedisse e lo lasciò andare. 

Tornato a casa, quel giorno, Levin provava la sensazione gioiosa che il disagio fosse finito e fosse finito in modo tale da non aver dovuto mentire. Inoltre gli era rimasto il ricordo confuso che ciò che aveva detto quel bravo e caro vecchietto non fosse così sciocco come gli era parso in principio, ma che in esso ci fosse qualcosa che bisognava chiarire. 

``S'intende, non adesso - pensava Levin - ma prima o poi sì''. Levin adesso sentiva più di prima di aver nell'animo qualcosa di confuso e di poco chiaro e che nei rapporti con la religione egli era in quella stessa situazione che così chiaramente scopriva negli altri, che non gli piaceva affatto e che riprovava nell'amico Svijazskij. 

Levin, trascorrendo quella sera con la fidanzata in casa di Dolly, fu particolarmente allegro e, spiegando a Stepan Arkad'ic quel suo stato di eccitamento, disse che si sentiva allegro come un cane al quale abbiano insegnato a saltare attraverso un cerchio e che, avendo capito alla fine e compiuto quel che si pretende da lui, si mette a guaire e, agitando la coda, salta per l'entusiasmo sui tavoli e sulle finestre. 

\capitolo{II}\label{ii-4} 

Il giorno del matrimonio, secondo l'usanza (la principessa e Dar'ja Aleksandrovna insistevano che ci si attenesse alle usanze), Levin non vide la sposa e pranzò nel suo albergo con tre scapoli venuti da lui per caso: Sergej Ivanovic, Katavasov, un compagno di università ora professore di scienze naturali, che Levin, incontrato per strada, si era trascinato a casa, e cirikov, il compare d'anello, giudice di pace a Mosca, compagno di Levin nella caccia all'orso. Il pranzo fu molto allegro. Sergej Ivanovic era di ottimo umore ed era divertito dalla originalità di Katavasov. Katavasov, sentendo che la sua originalità era apprezzata e capita, ne faceva sfoggio. cirikov sosteneva con bonarietà e allegria qualsiasi conversazione. 

- Ecco dunque - diceva Katavasov, strascinando le parole per un'abitudine dovuta all'insegnamento cattedratico - quale ragazzo pieno di possibilità era il nostro amico Konstantin Dmitric! Parlo di assenti, perché lui ormai non c'è più. E amava la scienza allora, finita l'università, e aveva interessi umani; mentre adesso, una parte delle sue abitudini è diretta a ingannare se stesso, e l'altra metà a giustificare questo inganno. 

- Un nemico più deciso di voi del matrimonio non l'ho mai visto - disse Sergej Ivanovic. 

- No, non sono nemico. Sono un amico della suddivisione del lavoro. Le persone che non possono far nulla devono creare gli uomini, ma le altre devono cooperare alla loro educazione e felicità. Così l'intendo io. Un'infinità di persone ama confondere queste due funzioni, io non sono fra queste. 

- Come sarò felice, quando verrò a sapere che vi siete ingannato! - disse Levin. - Per favore invitatemi alle vostre nozze. 

- Io sono già innamorato. 

- Già, della seppia. Sai - disse Levin al fratello - Michail Semënyc scrive un'opera sulla nutrizione e\ldots{} 

- Via, non fate confusioni! Non ha importanza su che cosa. Fatto sta che io amo proprio la seppia. 

- Ma essa non vi impedirà di amare vostra moglie. 

- Lei, no, ma la moglie, sì che me lo impedirà. 

- E perché? 

- Ma ecco, vedrete. Ecco, a voi piace l'azienda domestica, la caccia, ebbene, vedrete! 

- E oggi c'è stato Archip e ha detto che nel Prudnoj c'è un branco di cervi e ci sono due orsi - disse cirikov. 

- Be', li prenderete senza di me. 

- Anche questo è vero - disse Sergej Ivanovic. - D'ora in poi da' un addio alla caccia all'orso. Tua moglie non ti lascerà andare. 

Levin sorrise. L'immagine della moglie, che non gli permetteva di andare, gli era così cara che era pronto a rinunciare per sempre al piacere di vedere gli orsi. 

- Ma è pur peccato che quei due orsi li prendano senza di voi. E vi ricordate a Chapilovo l'ultima volta? Sarà una caccia meravigliosa - disse cirikov. 

Levin non voleva togliersi l'illusione che, senza di lei, ci potesse essere, in una qualche parte, qualcosa di buono, e perciò non disse nulla. 

- Non per nulla si è stabilito l'uso di dire addio alla vita da scapolo - disse Sergej Ivanovic. - Per quanto si possa essere felici si rimpiange sempre la libertà. 

- Ma, dite la verità, non avete la sensazione, come lo sposo di Gogol' d'aver voglia di saltar via dalla finestra? 

- Sicuro che ce l'ha, ma non lo confessa! - disse Katavasov e prese a ridere forte. 

- Ebbene, la finestra è aperta\ldots{} Andiamo subito a Tver'! Uno dei due orsi è una femmina e si può prendere nella tana. Davvero, andiamo col treno delle cinque. E qui, facciano quello che vogliono - disse, sorridendo, cirikov. 

- Ma ecco, in verità di Dio - disse, sorridendo, Levin - non posso trovare nell'anima mia questo senso di rimpianto per la libertà. 

- Ma voi adesso avete nell'animo un tale caos che non ci trovate nulla - disse Katavasov. - Aspettate a raccapezzarvici un po', e poi troverete! 

- No, sentirei, sia pure poco, a parte il mio sentimento - non voleva dire dinanzi a lui ``amore'' - e la felicità mia, che tuttavia mi spiacerebbe perdere la libertà\ldots{} E invece è proprio di questa sottrazione di libertà che sono contento. 

- Male! soggetto senza speranza! - disse Katavasov. - Via, beviamo alla sua guarigione, o auguriamogli soltanto che una centesima parte dei suoi sogni si avveri. E questa sarà una felicità quale non c'è mai stata sulla terra! 

Poco dopo il pranzo gli ospiti se ne andarono per fare in tempo a mutar d'abito per la cerimonia. 

Rimasto solo e riandando ai discorsi di quegli scapoli, Levin si chiese ancora una volta se avesse nell'animo quel senso di rimpianto per la libertà di cui essi avevano parlato. Sorrise a un simile quesito. ``La libertà? e perché la libertà? La felicità è solo nell'amore e nel desiderare, nel pensare con i suoi pensieri e con i suoi desideri, cioè nessuna libertà, questa è la felicità''. 

``Ma conosco io forse i suoi pensieri, i suoi desideri, i suoi sentimenti?'' gli mormorò all'improvviso chi sa quale voce. Il sorriso gli sparì dal volto, ed egli si fece pensieroso. E a un tratto lo afferrò una strana sensazione. Lo afferrò il terrore e il dubbio, il dubbio di tutto. 

``E se non mi ama? E se mi sposa solo per prender marito? Se lei stessa non sa quello che fa? - si chiedeva. - Potrebbe ravvedersi e, appena sposata, capire che non mi ama e che non può amarmi''. E cominciarono a venirgli i pensieri più strani su di lei, i pensieri peggiori. Era geloso di Vronskij come un anno prima, come se quella sera in cui l'aveva vista con Vronskij fosse stata la sera precedente. Sospettava ch'ella non gli avesse detto tutto. 

Saltò su in fretta. ``No, così non si può! - disse a se stesso disperato. - Andrò da lei le chiederò, le dirò per l'ultima volta: `noi siamo liberi, non è meglio fermarsi? Tutto è preferibile a una infelicità continua, all'infamia, all'infedeltà!'\,''. Con la disperazione nell'animo e con un senso di rancore verso tutti, verso se stesso, verso di lei, uscì dall'albergo e si diresse verso casa Šcerbackij. 

La trovò nelle stanze interne. Sedeva su di un baule e dava disposizioni a una donna, scegliendo fra mucchi di abiti di vario colore distribuiti sulle spalliere delle seggiole e sul pavimento. 

- Ah! - gridò nel vederlo e si illuminò di gioia. - Come tu, come voi? - fino a quell'ultimo giorno gli parlava ora col tu ora col voi. - Ecco, non me l'aspettavo! E io sto scegliendo i miei vestiti da ragazza, a chi questo\ldots{} 

- Ah, questo è molto bello! - disse lui guardando torvo la donna. 

- Va' via, Dunjaša, sonerò poi - disse Kitty. - Che hai? - domandò, dandogli decisamente del tu, appena la donna fu uscita. Ella aveva notato il viso di lui, agitato e cupo, e il terrore l'aveva afferrata. 

- Kitty, mi tormento. Non posso tormentarmi solo - disse con la disperazione nella voce, fermandosi dinanzi a lei e guardandola supplichevole negli occhi. Egli vedeva già dal viso di lei pieno di amore, di sincerità, che nulla poteva venir fuori da quel che s'era proposto di dire, tuttavia gli era necessario ch'ella stessa lo dissuadesse. - Sono venuto a dirti che siamo ancora in tempo. Tutto questo si può annientare e riparare. 

- Che cosa? Non capisco nulla. Che cosa ti è successo? 

- Quello che ho detto mille volte e che non posso non pensare\ldots{} che io non merito te. Tu non puoi acconsentire a sposarmi. Pensaci. Ti sei ingannata. Pensaci proprio bene. Tu non puoi amarmi\ldots{} Se\ldots{} dimmelo piuttosto - disse senza guardarla. - Sarò infelice. Lascia che tutti dicano quello che vogliono; tutto è preferibile alla infelicità\ldots{} Tutto è meglio ora, finché siamo in tempo\ldots{} 

- Non capisco - rispondeva lei, spaventata - sarebbe allora che tu vuoi rinunciare\ldots{} che non si deve? 

- E già, se tu non mi ami. 

- Ma sei impazzito! - gridò lei, avvampando di stizza. Ma il viso di lui era così pietoso ch'ella trattenne la stizza e, gettati via gli abiti dalla poltrona, cambiò di posto, sedendosi accanto a lui. - Cosa pensi? dimmi tutto. 

- Io penso che tu non puoi amarmi. Perché dovresti amarmi? 

- Dio mio! cosa posso mai\ldots{} - disse lei, e cominciò a piangere. 

- Ah, che ho fatto! - egli gridò e, postosi in ginocchio dinanzi a lei, cominciò a baciarle le mani. 

Quando la principessa, cinque minuti dopo, entrò nella stanza, li trovò completamente rappacificati. Kitty non solo lo aveva rassicurato di amarlo, ma gli aveva persino spiegato perché lo amava, rispondendo alla sua domanda. Gli aveva detto che lo amava perché lo capiva interamente, perché sapeva che cosa gli doveva piacere, e che tutto quello che piaceva a lui, era bene. E questo a lui parve del tutto chiaro. Quando la principessa entrò, sedevano l'uno accanto all'altra sul baule, scegliendo gli abiti e discutendo sul fatto che Kitty voleva dare a Dunjaša quel vestito marrone che aveva indosso quando Levin le aveva fatto la sua proposta di matrimonio, e lui insisteva perché non fosse dato ad alcuno e diceva che a Dunjaša si poteva dare quello azzurro. 

- Ma come, non capisci? Lei è bruna e non le starà bene\ldots{} Io ho tutto calcolato\ldots{} 

Avendo saputo perché era venuto, la principessa, un po' per scherzo e un po' sul serio, si arrabbiò e lo mandò a casa a vestirsi, invece di impedire a Kitty di pettinarsi, ora che Charles stava per arrivare. 

- Anche così non mangia niente, in questi giorni, e si è fatta brutta, e tu vieni ancora a sconvolgerla con le tue sciocchezze - gli disse. - Vattene via, caro. 

Levin, colpevole e vergognoso, ma rasserenato, tornò in albergo. Suo fratello, Dar'ja Aleksandrovna e Stepan Arkad'ic, tutti in gran gala, lo aspettavano già per benedirlo con l'icona. Non c'era tempo da perdere. Dar'ja Aleksandrovna doveva ancora passare da casa per prendere un figliuolo tutto impomatato e arricciato, che doveva portare l'icona insieme con la fidanzata. Poi una carrozza bisognava mandarla a rilevare il compare d'anello, e l'altra, che avrebbe portato Sergej Ivanovic, bisognava mandarla indietro\ldots{} Insomma, di considerazioni molto complicate ce n'erano tante. Una cosa era fuor di dubbio, che non si poteva indugiare, perché erano già le sei. 

La benedizione con l'icona non riuscì per niente. Stepan Arkad'ic si mise in una posa comicamente solenne accanto alla moglie, prese l'icona e, ordinato a Levin di inchinarsi fino a terra, lo benedisse con il suo sorriso buono e canzonatorio e lo baciò tre volte; lo stesso fece Dar'ja Aleksandrovna e subito si affrettò ad andar via e si confuse nel dare le disposizioni per le carrozze. 

- Su, allora, ecco cosa faremo: tu vai a prender lui con la nostra carrozza, e se poi Sergej Ivanovic fosse così buono di passarlo a prendere e rimandare la vettura indietro\ldots{} 

- Ma certo, sono molto contento. 

- E noi verremo subito con lui. La roba è stata spedita? - chiese Stepan Arkad'ic. 

- È stata spedita - rispose Levin, e ordinò a Kuz'ma di aiutarlo a vestirsi. 

\capitolo{III}\label{iii-4} 

Una folla di gente, in gran parte femminile, inondava la chiesa illuminata per il matrimonio. Le signore che non avevano fatto in tempo a ficcarsi dentro, si affollavano intorno alle finestre, urtandosi, discutendo e guardando attraverso le grate. 

Più di venti carrozze erano già state disposte lungo la via dai gendarmi. Un ufficiale di polizia, senza curarsi del gelo, stava dinanzi all'ingresso, splendente nella sua uniforme. Le carrozze si avvicinavano ininterrottamente, e ora signore con fiori e con gli strascichi sollevati, ora uomini che si toglievano il chepì o il cappello nero, entravano in chiesa. Nell'interno erano già accesi i due candelabri e tutte le candele dinanzi alle immagini. Lo splendore dell'oro sul fondo rosso dell'iconostasi, l'intaglio dorato delle icone e l'argento dei candelabri e dei candelieri, le lastre del pavimento, gli arazzi e gli stendardi al di sopra e accanto ai cori, gli scalini dell'ambone, e i vecchi libri anneriti, le stole e le cotte, tutto era inondato di luce. Nella parte destra della chiesa riscaldata, in un mare di frac, cravatte bianche e divise, di seta, velluto, raso, acconciature e fiori, di spalle, braccia nude e guanti lunghi, si svolgeva una conversazione animata che risonava stranamente nell'alto della cupola. Ogni volta che strideva la porta nell'aprirsi, la conversazione si quietava, e tutti guardavano, aspettandosi di vedere entrare lo sposo e la sposa. Ma la porta era stata aperta più di dieci volte, e ogni volta era un invitato o un'invitata in ritardo che si univa alla cerchia degli altri a destra, o una spettatrice che, ingannato o commosso l'ufficiale di polizia, si univa alla folla estranea a sinistra. E i parenti e gli estranei erano passati attraverso tutte le fasi dell'attesa. 

In principio avevano creduto che lo sposo e la sposa sarebbero venuti subito, e non avevano dato alcun significato a questo ritardo. Poi avevano cominciato a guardare la porta sempre più spesso, domandandosi di tanto in tanto se non fosse accaduto qualcosa. Infine, questo ritardo divenne increscioso, e i parenti e gli invitati cercarono di non apparire preoccupati degli sposi e di essere presi dalla loro conversazione. 

Il protodiacono, come per ricordare che il suo tempo era prezioso, tossicchiava con impazienza, facendo tremare i vetri delle finestre. Sul coro si sentivano le voci che provavano e i cantori che, presi dalla noia, si soffiavano il naso. Il sacerdote mandava ogni momento il sacrestano o il diacono a vedere se arrivava lo sposo, e lui stesso, con la tunica viola e la cintura ricamata, si affacciava sempre più spesso alla porta laterale in attesa dello sposo. Finalmente una delle signore, guardando l'ora, disse: ``ma è proprio strano!'' e tutti gli invitati si misero in agitazione e cominciarono a esprimere ad alta voce la propria sorpresa e il proprio scontento. Uno dei testimoni andò ad informarsi che cosa era successo. Kitty, intanto, da tempo già completamente pronta, con l'abito bianco, il velo lungo e la corona di fiori di arancio, stava dritta nella sala di casa Šcerbackij, con la madrina e la sorella L'vova, e guardava dalla finestra, aspettando invano, già da più di mezz'ora, notizie dal suo testimone sull'arrivo dello sposo in chiesa. 

E Levin intanto, coi pantaloni, ma senza il panciotto e il frac, camminava su e giù per la camera d'albergo, affacciandosi continuamente alla porta del corridoio e guardando. Ma nel corridoio non compariva la persona ch'egli aspettava e, tornando indietro disperato e agitando le braccia, si rivolgeva a Stepan Arkad'ic che fumava tranquillamente. 

- C'è mai stato un uomo in una posizione così terribilmente stupida? 

- Sì, è sciocco - confermò Stepan Arkad'ic, sorridendo teneramente. - Ma calmati, porteranno subito. 

- No, ma come! - diceva con rabbia trattenuta Levin. - E questi stupidi panciotti aperti! È impossibile! - diceva, guardando lo sparato sgualcito della propria camicia. - E che succederà se hanno già portato la roba alla ferrovia! - gridò con disperazione. 

- Allora ne metterai una mia. 

- E bisognava far così da un pezzo. 

- Ma non sta bene esser ridicoli\ldots{} Aspetta, ``s'appianerà''. 

Il fatto era che, quando Levin aveva chiesto di vestirsi, Kuz'ma, il vecchio servo di Levin, aveva portato il frac, il panciotto e tutto quello che occorreva. 

- E la camicia? - aveva detto Levin. 

- La camicia l'avete indosso - aveva risposto Kuz'ma con un sorriso tranquillo. 

Una camicia pulita Kuz'ma non aveva pensato a lasciarla e, avuto l'ordine di prendere e portare tutto in casa Šcerbackij, dalla quale la sera sarebbero partiti gli sposi, aveva messo tutto dentro, tranne il frac. La camicia, indossata fin dalla mattina, era sgualcita e impossibile a mettersi con la moda dei panciotti aperti. Mandar dagli Šcerbackij era lontano. Avevano mandato a comprare una camicia. Il servitore era tornato indietro: era tutto chiuso, perché era domenica. Avevano mandato da Stepan Arkad'ic e avevano portato una camicia; ma era stretta e corta in modo impossibile. Avevano, infine, mandato dagli Šcerbackij a disfare i bagagli. Si aspettava lo sposo in chiesa e lui camminava per la stanza come una belva in gabbia, guardando fuori nel corridoio e ricordando con terrore e disperazione quel che aveva detto a Kitty e quello che ella poteva pensare, adesso, di lui. 

Alla fine Kuz'ma, il colpevole, respirando a stento, irruppe nella stanza con la camicia. 

- Li ho appena trovati. Caricavano già tutto sul carro - disse Kuz'ma. 

Dopo tre minuti, senza guardare l'ora, per non inasprire le ferite, Levin correva per il corridoio. 

- Tanto non ci fai nulla - diceva Stepan Arkad'ic con un sorriso, tenendogli dietro senza furia. - ``S'appianerà, s'appianerà''\ldots{} ti dico. 

\capitolo{IV}\label{iv-4} 

- Sono arrivati! Eccolo! Ma qual'è? Quello più giovane, eh? E lei, matuška, più morta che viva! - si cominciò a dire tra la folla quando Levin, incontrata la sposa all'ingresso, entrò in chiesa insieme con lei. 

Stepan Arkad'ic raccontò alla moglie la causa del ritardo e gli invitati parlottarono fra di loro, sorridendo. Levin non notava nulla e nessuno; senza abbassar gli occhi, guardava la sposa. 

Tutti dicevano che, in quegli ultimi giorni, lei s'era sciupata e che con l'acconciatura da sposa era molto meno carina del solito; ma Levin non trovava questo. Guardava quella sua pettinatura alta con il lungo velo bianco e i fiori bianchi, il colletto alto a pieghe, che in modo così verginale chiudeva di lato e scopriva davanti il collo lungo e la vita meravigliosamente sottile, e a lui sembrava ch'ella stesse come non mai, non perché quei fiori, quel velo, quel vestito ordinato a Parigi aggiungessero qualcosa alla sua bellezza, ma perché, malgrado il fasto dell'acconciatura, l'espressione del viso gentile, dello sguardo, delle labbra, era sempre quella stessa espressione di innocente sincerità, tutta sua. 

- Credevo che già te ne volessi fuggire - disse lei, sorridendogli. 

- È così sciocco quello che m'è successo che fa vergogna a dirlo - rispose Levin arrossendo, e fu costretto a voltarsi verso Sergej Ivanovic che gli si era avvicinato. 

- Carina la tua storia della camicia! - disse Sergej Ivanovic, scotendo la testa e sorridendo. 

- Già - rispose Levin senza capire di che cosa gli parlassero. 

- Eh, via, Kostja, ora bisogna decidere - disse Stepan Arkad'ic con un'aria di finto spavento - è una questione seria. Tu ora sei proprio in grado di capirne tutta l'importanza. Mi hanno chiesto: si devono accendere le candele bruciate o quelle non bruciate? La differenza è di dieci rubli - aggiunse, atteggiando le labbra a un sorriso. - Ho deciso io, ma temo che tu non me ne dia l'approvazione. 

Levin capì che si trattava di uno scherzo, ma non riuscì a sorridere. 

- E allora, come? quelle bruciate o quelle non bruciate? qui sta la questione. 

- Sì, sì, quelle non bruciate. 

- Ah, sono proprio contento! La questione è decisa! - disse Stepan Arkad'ic, sorridendo. - Ma come ci si istupidisce in questa situazione! - disse rivolto a cirikov, mentre Levin, dopo averlo guardato con aria smarrita, si era avvicinato alla sposa. 

- Attenta, Kitty, metti per prima il piede sul tappeto - disse la contessa Nordston, avvicinandosi. - Come state bene! - disse rivolta a Levin. 

- Be', niente paura? - chiese Mar'ja Dmitrievna, una vecchia zia. 

- Non hai freddo? Sei pallida. Aspetta, fermati! - disse la sorella di Kitty, la L'vova, e, disponendo a cerchio le braccia piene, bellissime, con un sorriso le acconciò i fiori sul capo. 

Dolly si avvicinò, voleva dire qualcosa ma non riuscì a parlare, si mise a piangere e poi a ridere forzatamente. 

Kitty guardava tutti con i suoi occhi assenti, come Levin. A tutti i discorsi rivolti a lei poteva rispondere solo con un sorriso di gioia, che in lei, in quel momento, era del tutto naturale. 

Intanto i sacerdoti avevano di nuovo indossato i paramenti e il prete e il diacono si erano diretti verso il leggio posto nel vestibolo della chiesa. Il sacerdote si rivolse a Levin, dicendogli qualcosa. Levin non sentì quel che il sacerdote gli aveva detto. 

- Prendete per mano la sposa e conducetela - disse il compare d'anello a Levin. 

Per un pezzo Levin non capì che cosa si volesse da lui. Per un pezzo lo corressero e stavano già per desistere - infatti o non la prendeva con la mano giusta o le prendeva quella che non doveva - quando finalmente capì che la doveva prendere con la sua destra, senza cambiar posto, proprio per la destra. Quando finalmente ebbe presa la sposa con la mano destra, così come si doveva, il sacerdote fece alcuni passi in avanti e si fermò dinanzi al leggio. La folla dei parenti e degli amici, con un fruscio e ronzio di strascichi e discorsi, fece ressa dietro di loro. Qualcuno, chinatosi, acconciò il velo della sposa. Nella chiesa s'era fatto un silenzio tale che si sentiva gocciolar la cera. 

Il sacerdote, un vecchietto con la cotta, le ciocche di capelli lucide d'argento divise in due parti dietro le orecchie, liberate le piccole mani di vecchio sotto la pesante pianeta d'argento dalla croce dorata sulla schiena, sfogliava qualcosa sul leggio. 

Stepan Arkad'ic si avvicinò cauto, sussurrò qualcosa e, dopo aver ammiccato a Levin, si fece di nuovo indietro. 

Il sacerdote accese due ceri ornati di fiori e, tenendoli inclinati nella mano sinistra così che la cera ne gocciolava, si voltò con il viso verso gli sposi. Era lo stesso prete che aveva confessato Levin. Guardò con uno sguardo stanco e triste lo sposo e la sposa e, liberata di sotto la pianeta la mano destra, benedisse lo sposo e nello stesso modo, ma con una sfumatura di cauta tenerezza, impose le dita ripiegate sul capo chino di Kitty. Poi dette loro i ceri e, preso il turibolo, si allontanò. 

``Possibile che sia vero?'' pensò Levin e si voltò a guardare la sposa. Vedeva, un po' dall'alto, il profilo di lei e, dal moto appena percettibile delle labbra e delle ciglia, sapeva ch'ella sentiva il suo sguardo. Ella non si voltò, ma l'alto colletto a pieghe si mosse, sollevandosi, verso il piccolo orecchio rosa. Egli vedeva che il respiro si era fermato nel petto di lei e che, nel guanto lungo, la piccola mano che reggeva il cero aveva cominciato a tremare. 

Tutta la sua agitazione per la camicia e per il ritardo, la conversazione con gli amici e i parenti, il loro disappunto, la sua situazione ridicola, tutto scomparve a un tratto, ed egli provò gioia e sgomento. 

Un bel protodiacono alto, in dalmatica d'argento, con i ricci ondulati spartiti al centro, si avanzò con sicurezza e, sollevato su due dita, con gesto consueto, il manipolo, si fermò di fronte al prete. 

``Be-ne-di-ci Si-gno-re!'' echeggiarono lente le note solenni, l'una dietro l'altra, facendo oscillare delle onde d'aria. 

``Sia benedetto Iddio nostro sempre, adesso e ognora e nei secoli dei secoli'' rispose umilmente il vecchietto seguitando a sfogliare qualcosa sul leggio. E, effondendosi per tutta la chiesa, dalle finestre fino alla volta, l'accordo pieno del coro invisibile dei chierici si levò ampio e armonioso, si rafforzò, si fermò per un attimo e si spense piano. 

Pregavano, come del resto sempre, per la pace suprema e per la salvezza, per il Sinodo e per lo zar; pregavano in questo momento anche per il servo di Dio Konstantin e per Ekaterina che si univano in matrimonio. 

``Perché sia mandato loro l'amore perfetto, la pace e l'aiuto, preghiamo il Signore'' pareva respirare tutta la chiesa nella voce del protodiacono. 

Levin ascoltava le parole e queste lo stupivano. 

``Come hanno indovinato; che cos'è mai l'aiuto, l'aiuto? - pensava ricordando tutti i suoi recenti timori e dubbi. - Che cosa so io? che cosa posso in questa terribile cosa - pensava - senza aiuto? Proprio di aiuto ho bisogno ora''. 

Quando il diacono ebbe finito la preghiera, il sacerdote si rivolse agli sposi con il libro. ``Iddio eterno, Tu che hai congiunto quelli che erano lontani - leggeva con voce mite, intonata - e che hai stabilito un'unione d'amore indistruttibile; Tu che hai benedetto Isacco e Rebecca, che hai mostrato ai loro discendenti la Tua promessa: benedici Tu stesso i Tuoi servi Konstantin ed Ekaterina, indirizzandoli verso ogni opera di bene. Poiché misericordioso e pieno d'amore sei, o Dio, a Te la gloria innalziamo, al Padre, al Figlio e allo Spirito Santo, ora e sempre, e nei secoli''. ``Amen'' risonò di nuovo nell'aria l'invisibile coro. 

``\,`Che hai congiunto quelli che erano lontani e hai stabilito un'unione d'amore': come sono profonde queste parole e come corrispondono a quello che io sento in questo momento! - pensava Levin. - Sente anche lei come me?''. E voltatosi a guardarla, incontrò lo sguardo di lei. 

E da questo sguardo, egli concluse ch'ella sentiva così come lui. Ma non era vero; ella non intendeva quasi nulla delle parole del servizio divino e non le ascoltava neppure, durante la funzione. Non voleva sentirle, né intenderle; tanto forte era quell'unico sentimento che le invadeva l'anima e diveniva sempre più forte. Questo sentimento era la gioia del pieno compiersi di ciò che da un mese e mezzo si era compiuto nell'animo suo e che durante quelle sei settimane l'aveva rallegrata e tormentata. Nell'animo suo, in quel giorno in cui, in abito marrone, nella sala della casa sull'Arbat, si era avvicinata a lui in silenzio e gli si era data, nell'animo suo, in quel giorno e in quell'ora si era compiuto un completo distacco da tutta la sua vita di prima ed era cominciata, pur continuando in realtà l'antica, un'altra vita, completamente nuova, completamente sconosciuta. Queste sei settimane erano state il periodo più beato e più tormentato per lei. Tutta la sua vita, i suoi desideri, le sue speranze si erano concentrati in quel solo uomo, per lei ancora incomprensibile, al quale la legava un sentimento ancora più incomprensibile dell'uomo stesso, che ora avvicinava, che ora respingeva; e nello tesso tempo ella aveva continuato a vivere nelle stesse condizioni di vita di prima. Vivendo la sua vita di prima, aveva orrore di sé, della sua completa, insormontabile indifferenza verso tutto il suo passato: verso le cose e le abitudini, verso le persone che le avevano voluto bene e gliene volevano, verso la madre rammaricata di quell'indifferenza, verso il padre tenero e caro, fino allora amato più di tutti al mondo. Un momento aveva orrore di quell'indifferenza, un momento si rallegrava di quello che ve l'aveva condotta. Non poteva più pensare, né desiderare altro che la vita con quell'uomo; ma questa nuova vita non c'era ancora, ed ella non riusciva neppure a immaginarsela con chiarezza. Non c'era che attesa: lo sgomento e la gioia del nuovo e dell'ignoto. Ed ecco, da un momento all'altro, l'attesa e l'ignoto, e il rimorso di aver rinunciato alla vita di prima: tutto sarebbe finito e sarebbe cominciato qualcosa di nuovo. Questo qualcosa di nuovo non poteva non essere terribile per la sua incertezza; ma, per quanto pauroso fosse, s'era già compiuto sei settimane prima nell'animo suo, e adesso si santificava soltanto quello che già da tempo era avvenuto nell'animo suo. 

Voltosi di nuovo verso il leggio, il sacerdote afferrò con difficoltà il piccolo anello di Kitty e, chiesta la mano di Levin, glielo infilò nella prima falange del dito. ``Si sposa il servo di Dio Konstantin con la serva di Dio Ekaterina''. E, infilato l'anello grande nel dito piccolo e roseo, commovente di fragilità, di Kitty, il sacerdote pronunciò le stesse parole. 

Varie volte gli sposi cercarono di indovinare che cosa si dovesse fare, e ogni volta sbagliarono, e il prete li corresse sottovoce. Finalmente, fatto quello che occorreva, dopo averli segnati d'un segno di croce con gli anelli, il sacerdote consegnò di nuovo a Kitty quello grande e a Levin quello piccolo; di nuovo essi si confusero e due volte fecero passare l'anello da una mano all'altra senza che ne venisse fuori quello che si richiedeva. 

Dolly, cirikov e Stepan Arkad'ic si fecero avanti per correggerli. Si produssero confusione, bisbigli e sorrisi, ma l'espressione solennemente commossa degli sposi non mutò; al contrario, confondendo le mani, essi apparvero ancor più seri e solenni, e il sorriso col quale Stepan Arkad'ic sussurrò che ognuno infilasse il proprio anello, si spense involontariamente sulle sue labbra. Sentiva che qualsiasi sorriso li avrebbe offesi. 

``Poiché Tu dall'origine hai creato l'uomo e la donna - leggeva il sacerdote dopo lo scambio degli anelli - e da Te è congiunta la moglie al marito per la provvidenza e la procreazione del genere umano. Così Tu stesso, o Signore Iddio nostro, che hai inviato la verità alla Tua discendenza e la Tua promessa ai servi Tuoi, padri nostri, per generazioni e generazioni Tuoi eletti, guarda il servo Tuo Konstantin e la serva Tua Ekaterina, e conferma le nozze loro nella fede e nella concordia, nella verità e nell'amore\ldots{}''. 

Levin sentiva sempre più che tutte le sue idee sul matrimonio, i suoi sogni su come avrebbe costruito la sua vita, erano fanciullaggini, e che questo era qualcosa che lui non aveva finora inteso e che in quel momento ancor meno intendeva, sebbene si compisse in lui; un fremito sempre più alto gli sollevava il petto e le lacrime indocili gli venivano agli occhi. 

\capitolo{V}\label{v-4} 

Nella chiesa c'era tutta Mosca, tra parenti e amici. E durante il rito nuziale, nella chiesa illuminata a giorno, fra le donne adornate, le fanciulle e gli uomini in frac e cravatta bianca e in uniforme, non veniva mai meno un discorrere convenientemente sommesso, tenuto vivo soprattutto dagli uomini, mentre le signore erano prese dallo studio di tutti i particolari della cerimonia che sempre le commuove tanto. 

Nel circolo più vicino alla sposa c'erano le due sorelle: Dolly e la maggiore, la L'vova, donna calma e bella, giunta dall'estero. 

- Come mai Marie è in quel viola quasi nero, a un matrimonio? - chiedeva la Korsunskaja. 

- Con quel suo colorito è l'unica salvezza\ldots{} - rispondeva la Drubeckaja. - Mi sorprendo come mai abbiano fatto il matrimonio di pomeriggio. È da mercanti\ldots{} 

- È più bello. Anch'io mi sono sposata di sera - rispondeva la Korsunskaja e sospirò, ricordando come fosse graziosa quel giorno, come risibilmente innamorato di lei fosse suo marito e come ora tutto fosse diverso. 

- Dicono che non si sposa chi fa da compare d'anello più di dieci volte, e io volevo farlo per la decima volta per acquietarmi, ma il posto era occupato - diceva il conte Sinjavin alla graziosa principessa carskaja, che aveva delle mire su di lui. 

La carskaja gli rispondeva soltanto con un sorriso. Guardava Kitty e pensava a come e quando si sarebbe trovata lei al posto di Kitty, in piedi con il conte Sinjavin, e come allora gli avrebbe ricordato lo scherzo di oggi. 

Šcerbackij diceva alla vecchia damigella d'onore Nikolaevna che aveva intenzione di porre la corona nuziale sullo chignon di Kitty perché fosse felice. 

- Non occorreva mettersi lo chignon - rispondeva la Nikolaevna, la quale aveva da lungo tempo deciso che, se il vecchio vedovo che stava pescando l'avesse sposata, il matrimonio si sarebbe svolto nel modo più semplice. - Io non amo questo fasto. 

Sergej Ivanovic parlava con Dar'ja Dmitrievna, sostenendo per scherzo che l'usanza di partire dopo il matrimonio è diffusa perché gli sposi novelli si vergognano sempre un poco. 

- Vostro fratello può essere orgoglioso. È un miracolo, tanto è carina. Lo invidiate, penso. 

- Sono già passato attraverso questo, Dar'ja Dmitrievna - egli rispose, e il suo viso improvvisamente prese un'espressione seria e triste. 

Stepan Arkad'ic raccontava alla cognata il suo giuoco di parole sul divorzio. 

- Bisogna accomodare la corona - rispondeva lei, senza ascoltarlo. 

- Che peccato che sia così sciupata! - diceva la contessa Nordston alla L'vova. - Eppure lui non vale neanche un suo dito, non è vero? 

- No, mi piace tanto. Non perché sia il mio futuro beau-frère - rispondeva la L'vova - ma come si comporta bene! Ed è così difficile comportarsi bene in questa circostanza, non essere ridicoli. E lui non è ridicolo, non è impacciato, si vede che è commosso. 

- Mi pare che la cosa fosse attesa. 

- Eh, sì. Lei l'ha sempre amato. 

- Via, guardiamo chi dei due mette prima il piede sul tappeto. Io l'ho detto a Kitty. 

- Fa lo stesso - rispondeva la L'vova. - Noi siamo tutte mogli docili, l'abbiamo nel sangue. 

- E io invece mi ci son messa per prima con Vasilij. E voi, Dolly? 

Dolly stava in piedi accanto a loro, le ascoltava, ma non rispondeva. Era commossa. Aveva le lacrime agli occhi e non avrebbe potuto dir nulla senza mettersi a piangere. Era felice per Kitty e per Levin; ritornando col pensiero al suo matrimonio, guardò Stepan Arkad'ic, sempre raggiante, dimenticò tutto il presente e ricordò solo il suo primo amore innocente. Rammentò non solo se stessa, ma tutte le donne a lei vicine e note; le ricordò in quell'unico momento solenne per loro, quando così come ora Kitty, stavano sotto la corona nuziale con l'amore, la speranza e l'ansia nel cuore, rinunciando al passato ed entrando in un futuro misterioso. Fra tutte le spose che le tornarono in mente, ricordò anche Anna, a lei cara, e a proposito della quale, da non molto, aveva sentito parlare di divorzio. Anche lei, egualmente pura, era stata lì, in piedi, con i fiori d'arancio e il velo. E ora? ``Terribilmente strano'' si disse. 

Non solo le sorelle, le amiche e i parenti seguivano tutti i particolari della funzione; ma le donne estranee, le spettatrici, con un'emozione che spezzava loro il respiro, temendo di perdere ogni movimento, seguivano l'espressione del viso dello sposo e della sposa, e con irritazione non rispondevano, e spesso non ascoltavano neppure, i discorsi indifferenti degli uomini, che facevano osservazioni scherzose o estranee. 

- Come mai ha pianto tanto? Si sposa forse controvoglia? 

- E come controvoglia, con un giovane così bello! Un principe, vero? 

- È la sorella questa in raso bianco? Su, senti come urla il diacono: ``che tema suo marito''. 

- Sono di cudovo? 

- Sono del Sinodo. 

- Ho fatto parlare il servitore. Dice che lui la porta subito nelle sue proprietà. È tanto ricco, dicono. Perciò gliel'hanno data. 

- No, la coppia è bella. 

- Ed ecco voi, Mar'ja Vlas'evna, dicevate che le carnaline non si portano staccate. Guarda un po' quella vestita color pulce, dicono che sia un'ambasciatrice, con quel risvolto\ldots{} Così, di nuovo così. 

- E come è carina la sposa, guarnita come un'agnellina! E qualunque cosa diciate, fa sempre pena. 

Così si parlottava fra la folla delle spettatrici che erano riuscite a varcare la porta della chiesa. 

\capitolo{VI}\label{vi-4} 

Quando la funzione terminò, un chierico distese dinanzi al leggio, nel centro della chiesa, un pezzo di seta rosa, il coro si mise a cantare un salmo difficile e complesso, nel quale il basso e il tenore si rispondevano tra di loro, e il sacerdote, voltatosi, indicò agli sposi il pezzo di seta rosa disteso. Per quanto tutti e due avessero sentito parlare spesso della superstizione che, chi per primo mette il piede sul tappeto, quegli diviene il capo della famiglia, né Levin né Kitty riuscirono a ricordarsene, quando fecero quei pochi passi. Non sentirono neppure le osservazioni fatte ad alta voce, né le discussioni sul fatto che, secondo quanto avevano osservato alcuni, lui ci si era messo per primo, e quanto ad altri invece, tutti e due insieme. 

Dopo le solite domande sul desiderio di contrarre matrimonio e di non essere promessi ad altri, e dopo le risposte che risonarono strane a loro stessi, cominciò una nuova funzione. Kitty ascoltava le parole della preghiera, desiderando intenderne il senso, ma non poteva. Un sentimento di festosità e di gioia luminosa, a misura che il rito si compiva, invadeva sempre più l'animo suo e le toglieva la possibilità di raccogliersi. 

Si pregava ``perché fosse loro donata la purezza, e il frutto delle viscere per il loro bene, perché si rallegrassero della vista dei figli e delle figlie''. Si ricordava anche che Iddio aveva creato la donna dalla costola di Adamo, e che ``per questo l'uomo lascerà il padre e la madre e si unirà con la moglie, ed essi saranno due in una sola carne'' e ``che questo è un grande mistero''; si chiedeva ``che Dio concedesse loro fecondità e benedizione, come a Isacco e a Rebecca, a Giuseppe, a Mosè e a Sefora, e che essi giungessero a vedere i figli dei loro figli''. ``Tutto questo è molto bello - pensava Kitty ascoltando le parole - tutto questo non potrebbe essere diversamente'' e un sorriso di gioia, che si comunicava involontariamente a tutti quelli che la guardavano, le splendeva sul viso luminoso. 

- Mettetela per bene! - risonarono i consigli nel momento in cui il sacerdote impose loro le corone, e Šcerbackij, con la mano tremante nel guanto a tre bottoni, tenne la corona in alto, sulla testa di lei. 

- Mettetemela! - mormorò lei, sorridendo. 

Levin si voltò a guardarla, e fu sorpreso dello splendore gioioso del viso di lei; e questo sentimento gli si comunicò. Divenne, come lei, luminoso e allegro. 

Si rallegrarono di ascoltare la lettura dell'epistola dell'apostolo e l'eco della voce del protodiacono all'ultimo versetto, atteso con tanta impazienza dal pubblico estraneo. Si rallegrarono di bere nella tazza dalla forma schiacciata il vino rosso, tiepido, unito all'acqua, ed ancor più si rallegrarono quando il sacerdote, toltasi la pianeta e prese nella sua le loro mani, li condusse, fra gli slanci del basso che emetteva l'``Isaia giubila!'', accanto al leggio. Šcerbackij e cirikov, che sostenevano le corone, impigliandosi nello strascico della sposa, anch'essi sorridenti e come rallegrati da qualcosa, ora si fermavano, ora urtavano contro gli sposi, alle soste del sacerdote. La scintilla di gioia che si era accesa in Kitty sembrava essersi comunicata a tutti quelli che erano in chiesa. A Levin sembrava che pure il sacerdote e il diacono avessero voglia di sorridere, così come lui. 

Dopo aver tolto le corone dalle teste, il sacerdote finì di leggere l'ultima preghiera e si rallegrò con gli sposi. Levin guardò Kitty e fino a quel momento non l'aveva mai vista così. Ella era deliziosa per quella nuova luce di felicità che era nel suo viso. Levin avrebbe voluto dire qualcosa, ma non sapeva se tutto era finito, oppure no. Il sacerdote lo tolse d'impaccio. Sorrise con la sua bocca da buono e disse piano: ``Baciate vostra moglie, e voi baciate vostro marito'' e tolse loro di mano i ceri. 

Levin baciò lievemente le labbra sorridenti di lei, le offrì il braccio e, provando la sensazione di una nuova strana vicinanza, uscì dalla chiesa. Non credeva, non poteva credere che fosse vero. Soltanto quando i loro sguardi timidi e attoniti si incontrarono, credette, perché sentì che ormai erano una cosa sola. 

Dopo il pranzo, quella notte stessa, i giovani sposi partirono per la campagna. 

\capitolo{VII}\label{vii-4} 

Vronskij e Anna viaggiavano già da tre mesi insieme per l'Europa. Avevano visitato Venezia, Roma, Napoli ed erano appena arrivati in una piccola città italiana, dove volevano stabilirsi per un certo tempo. 

Il capocameriere, un bell'uomo, con una scriminatura che incominciava dal collo nei capelli folti impomatati, in frac e grande sparato bianco di batista alla camicia, con una filza di ciondoli sulla pancetta arrotondata, rispondeva, le mani in tasca e gli occhi socchiusi e disdegnosi, qualcosa di arcigno a un signore che s'era fermato. Avendo sentito dall'altra parte dell'ingresso dei passi che salivano la scala, il capocameriere si voltò e, visto il conte russo che da loro occupava le stanze migliori, tolse rispettosamente le mani di tasca e, inchinandosi, riferì che l'inserviente era andato e che la faccenda dell'affitto del palazzo si era conclusa. L'amministratore era pronto a firmare il contratto. 

- Ah! Sono molto contento - disse Vronskij. - E la signora è in casa o no? 

- La signora è uscita a passeggiare, ma è rientrata or ora - rispose il cameriere. 

Vronskij si tolse il cappello floscio dalle falde larghe e asciugò col fazzoletto la fronte sudata e i capelli lunghi fino a metà orecchie e pettinati all'indietro, in modo da nascondere la calvizie. Guardato distrattamente il signore che stava ancora là e che lo contemplava, fece per passare. 

- Questo signore è un russo e ha chiesto di voi - disse il capocameriere. 

Con un senso di irritazione, perché non riusciva a sfuggire in nessun posto ai conoscenti, misto al desiderio di trovare una qualche distrazione alla monotonia della propria vita, Vronskij guardò ancora una volta il signore che si era allontanato e fermato e, nello stesso istante, a tutt'e due si illuminarono gli occhi. 

- Golenišcev! 

- Vronskij! 

Golenišcev era stato, infatti, compagno di Vronskij al corpo dei paggi. Golenišcev, allora, apparteneva al partito liberale; era uscito dal corpo con un grado civile, ma non era mai stato impiegato in alcun posto. Finito il corso, i due compagni si erano completamente perduti di vista e in seguito s'erano incontrati una volta soltanto. 

In quell'incontro Vronskij aveva capito che Golenišcev aveva scelto un'attività del tutto libera e intellettuale e che, perciò, voleva spregiare l'attività e la condizione sociale di Vronskij. Per questo Vronskij, in quell'incontro con Golenišcev, gli aveva opposto quella fredda e orgogliosa resistenza che egli sapeva opporre alla gente, e il cui senso era questo: ``Vi può piacere o non piacere il mio modo di vivere, ma questo per me è assolutamente indifferente: mi dovete stimare se volete conoscermi''. Golenišcev era stato sprezzantemente indifferente al tono di Vronskij. Quell'incontro sembrava avesse dovuto separarli ancor più. Adesso, invece, si erano illuminati e avevano dato un grido di gioia nel riconoscersi. Vronskij non si aspettava in nessun modo di rallegrarsi tanto per Golenišcev, ma probabilmente non sapeva neanche lui quanto si annoiasse. Dimenticò l'impressione spiacevole dell'ultimo incontro e con un viso aperto, gioioso, tese la mano al compagno di un tempo. Un'eguale espressione di gioia tramutò la prima espressione di titubanza del viso di Golenišcev. 

- Come sono contento di incontrarti! - disse Vronskij, mostrando, in un sorriso cordiale, i suoi forti denti bianchi. 

- E io sento dire ``Vronskij'', ma quale non sapevo. Molto, molto contento. 

- Entriamo, allora. Ebbene, cosa fai? 

- Vivo qua, già da due anni. Lavoro. 

- Ah! - disse Vronskij con interesse. - Entriamo allora. 

E per la solita abitudine dei russi, invece di dire proprio in russo quello che voleva nascondere alla servitù, cominciò a parlare in francese. 

- Conosci la Karenina? Viaggiamo insieme. Vado da lei - disse in francese, guardando attento il viso di Golenišcev. 

- Ah! non sapevo - rispose con interesse Golenišcev sebbene lo sapesse. - Sei arrivato da parecchio tempo, qua? - soggiunse. 

- Io, da tre giorni - rispose Vronskij, esaminando ancora una volta con intenzione il viso del compagno. 

``Già, è un uomo per bene e considera la cosa così come va considerata - si disse Vronskij dopo aver capito l'espressione del viso di Golenišcev e la ragione del suo mutamento di discorso. - Gli si può far conoscere Anna; egli considera la cosa così come va considerata''. 

Vronskij, in quei tre mesi che aveva passato con Anna all'estero, nel fare amicizia con gente nuova, si era sempre posta la domanda come ogni nuova persona avrebbe considerato i suoi rapporti con Anna, e, nella maggioranza dei casi, aveva incontrato nella gente una certa comprensione, così ``come si deve''. Ma se avessero chiesto a lui e a quelle persone che intendevano la cosa così ``come si deve'', in che cosa consistesse questo loro intendere, e lui e queste persone si sarebbero trovate in difficoltà. 

In fondo, quelli che, secondo Vronskij, capivano la cosa ``come si deve'', non la intendevano affatto, ma si comportavano, in genere, come si comportano le persone beneducate riguardo a tutte le questioni complesse e insolubili che d'ogni parte circondano la vita; si comportavano secondo le convenienze, evitando allusioni e domande spiacevoli. Facevano finta di capire il senso della situazione, di riconoscerla, perfino di approvarla, ma di considerare fuori posto e superfluo spiegare tutto ciò. 

Vronskij indovinò subito che Golenišcev era uno di questi e perciò fu doppiamente contento di vederlo. Difatti Golenišcev si comportò con la Karenina, quando fu introdotto da lei, così come Vronskij poteva desiderare. Evitava, naturalmente, senza il più piccolo sforzo, tutti i discorsi che potevano portar disagio. 

Egli prima non conosceva Anna e fu stupito della sua bellezza e ancor più della semplicità con cui accettava la sua situazione. Ella arrossì quando Vronskij introdusse Golenišcev, e questo rossore infantile, che si diffuse sul viso di lei, aperto e bello, gli piacque in modo straordinario. E gli piacque in modo particolare, perché subito, come apposta, per non far sorgere equivoci davanti a una persona estranea, ella chiamò Vronskij semplicemente Aleksej e disse che andavano a stare in una casa allora presa in affitto, che la gente del luogo chiamava ``palazzo''. Questo atteggiamento leale e semplice dinanzi alla propria situazione, piacque a Golenišcev. Osservando la maniera cordialmente allegra, decisa, di Anna, conoscendo Aleksej Aleksandrovic e Vronskij, a Golenišcev sembrava di comprenderla in pieno. Gli sembrava di capire quello che lei non riusciva a capire in nessun modo: il fatto che lei dopo aver fatto l'infelicità del marito, dopo aver abbandonato lui e il figlio e dopo aver perduto la propria reputazione, potesse, tuttavia, sentirsi decisamente gaia e felice. 

- C'è nella guida - disse Golenišcev a proposito di quel palazzo che Vronskij aveva preso in affitto. - Là c'è un bellissimo Tintoretto. Dell'ultimo periodo. 

- Sapete cosa? Il tempo è bellissimo, andiamo là, diamoci un'occhiata ancora una volta - disse Vronskij ad Anna. 

- Sono molto contenta, vado subito a mettermi il cappello. Cosa dite, fa caldo? - disse, fermandosi sulla porta e guardando interrogativamente Vronskij. E di nuovo un vivace rossore le coprì il viso. 

Vronskij capì dal suo sguardo ch'ella non sapeva in quali rapporti egli volesse essere con Golenišcev, e che temeva di non essersi comportata come avrebbe voluto lui. Egli la guardò con uno sguardo tenero, prolungato. 

- No, non molto - disse. 

E a lei parve d'aver capito tutto, principalmente ch'egli era contento di lei; e sorridendo uscì con passo veloce dalla porta. 

Gli amici si guardarono l'un l'altro, e sul viso di tutte e due passò un'ombra di disagio, come se Golenišcev, che evidentemente l'aveva ammirata, volesse dire qualcosa di lei e non trovasse le parole, mentre Vronskij desiderava e temeva la stessa cosa. 

- Allora, ecco come va - cominciò Vronskij per cominciare un discorso. - Allora ti sei stabilito qua? - continuò ricordando che gli avevano detto che Golenišcev scriveva qualcosa. 

- Già, scrivo la seconda parte dei Due princìpi - disse Golenišcev, accendendosi di soddisfazione a questa domanda - cioè, per essere precisi, non scrivo ancora, ma vado preparando e raccogliendo il materiale. Sarà molto più ampia e comprenderà quasi tutte le questioni. Da noi, in Russia, non si vuole capire che siamo gli eredi di Bisanzio - e qui cominciò una lunga, calorosa spiegazione. 

Vronskij in principio si sentì a disagio, perché non conosceva neppure il primo capitolo dei Due princìpi, di cui l'autore gli parlava come di cosa nota. Ma poi, quando Golenišcev cominciò a esporre le sue idee e Vronskij poté seguirlo, allora, anche senza conoscere i Due princìpi, l'ascoltò con interesse, perché Golenišcev parlava bene. Ma l'agitata irritazione con la quale Golenišcev parlava dell'argomento che lo interessava, colpì e amareggiò Vronskij. Quanto più s'ingolfava nel discorso, tanto più gli si accendevano gli occhi, tanto più concitatamente ribatteva i suoi presunti avversari e tanto più agitata e offesa diveniva l'espressione del suo viso. Ricordando Golenišcev come un ragazzo magro, vivace, cordiale e nobile, sempre il primo della classe al corpo dei paggi, Vronskij non riusciva a capire in nessun modo le ragioni di quell'irritazione, e non l'approvava. In particolare, non gli piaceva che Golenišcev, uomo della buona società, si mettesse allo stesso livello di quegli scribacchini che lo irritavano, e si arrabbiasse con loro. Ne valeva la pena? Questo non piaceva a Vronskij; malgrado ciò, egli sentiva che Golenišcev non era felice e ne aveva pena. Un'infelicità, quasi un'alienazione mentale, si scopriva in quel viso mobile, abbastanza bello, mentre egli, senza notare neppure l'apparizione di Anna, continuava a esprimere in fretta e con calore le proprie idee. 

Quando Anna apparve in cappello e pellegrina e, giocherellando con l'ombrello con un movimento rapido della mano, si fermò vicino a lui, Vronskij con un senso di sollievo si distolse dagli occhi dolenti di Golenišcev fissi su di lui e guardò con rinnovato amore la sua deliziosa amica, piena di vita e di gioia. Golenišcev tornò in sé con difficoltà e in un primo momento fu triste e cupo; ma Anna, disposta affabilmente verso tutti (così era in quel periodo), lo rianimò presto col suo modo di fare semplice e gaio. Dopo aver tentato vari argomenti di conversazione, l'indusse a parlare di pittura di cui egli parlava molto bene, e prese ad ascoltarlo con attenzione. Giunsero a piedi fino alla casa presa in affitto e la visitarono. 

- Sono molto contenta di una cosa - diceva Anna a Golenišcev sulla via del ritorno. - Aleksej avrà un buon atelier. Prendi assolutamente tu quella stanza - diceva a Vronskij in russo e dandogli del tu, poiché aveva già capito che, nella loro solitudine, Golenišcev sarebbe divenuta una persona intima, e che dinanzi a lui non bisognava fingere. 

- Dipingi, forse? - disse Golenišcev, volgendosi in fretta a Vronskij. 

- Sì, molto tempo fa me ne sono interessato ed ora ho ripreso un poco - disse Vronskij, arrossendo. 

- Ha un gran talento - disse Anna con un sorriso gioioso. - Io, s'intende, non posso giudicare. Ma alcuni competenti hanno detto la stessa cosa. 

\capitolo{VIII}\label{viii-4} 

Anna, in quel periodo di libertà e rapida guarigione, si sentiva imperdonabilmente felice e piena di gioia di vivere. Il ricordo dell'infelicità del marito non avvelenava più la sua felicità. Questo ricordo, da una parte, era troppo terribile per poterci pensare; dall'altra aveva dato a lei una felicità troppo grande per pentirsene. Il ricordo di tutto quello che le era accaduto dopo la malattia, la riconciliazione col marito, la rottura, la notizia della ferita di Vronskij, la sua apparizione, i preparativi per il divorzio, l'abbandono del tetto maritale, l'addio al figlio, tutto questo le sembrava un sogno febbrile dal quale si era svegliata all'estero, sola, con Vronskij. Il ricordo del male causato al marito destava in lei una sensazione simile alla ripugnanza e vicina a quella che proverebbe un uomo che, nell'annegare, abbia strappato via da sé un essere che gli si era aggrappato. Quest'essere era annegato. Era stato male, s'intende, ma era stata l'unica salvezza, ed era meglio non ricordare particolari così paurosi. 

Nel primo momento del distacco le era venuto in mente un solo ragionamento, che la tranquillizzava su quel che aveva fatto; e ora che ricordava tutto il passato, ricordava questo solo ragionamento. ``Io ho fatto inevitabilmente l'infelicità di questo uomo - pensava - ma non voglio profittare di questa infelicità; anch'io soffro e soffrirò: sono privata di quello che prima mi era più caro, sono privata dell'onestà del mio nome e di mio figlio. Ho agito male e perciò non voglio la felicità, non voglio il divorzio e soffrirò la vergogna e il distacco da mio figlio''. Ma, per quanto sinceramente volesse soffrire, Anna non soffriva. Vergogna non ce n'era. Con quel tatto che in così grande misura avevano entrambi, all'estero, evitando le signore russe, non si mettevano mai in una posizione falsa e incontravano ovunque persone che fingevano di capire completamente la loro posizione molto meglio di loro stessi. Il distacco dal figlio che amava, neanche questo la tormentava, nei primi tempi. La bambina, la figlia avuta da lui, era così graziosa che Anna di rado ricordava il figlio. 

Il bisogno di vivere, reso più forte dalla guarigione, era così prepotente, e le condizioni di vita così nuove e piacevoli, che Anna si sentiva imperdonabilmente felice. Quanto più conosceva Vronskij, tanto più l'amava. Lo amava per lui stesso e per il suo amore per lei. Il completo possesso di lui la rendeva continuamente felice. La vicinanza di lui le era sempre piacevole. Tutti i tratti del suo carattere, che veniva a conoscere, le erano sempre più indicibilmente cari. Il suo aspetto, diverso negli abiti borghesi, era affascinante per lei come per una ragazza innamorata. In tutto quello ch'egli diceva, pensava e operava, ella vedeva qualcosa di particolarmente nobile ed elevato. Il proprio entusiasmo dinanzi a lui spesso la sgomentava: cercava, e non le riusciva, di trovar qualcosa in lui che non fosse bello. Non osava mostrargli la consapevolezza della propria nullità di fronte a lui. Le sembrava che, sapendo questo, egli potesse disincantarsi più presto di lei; e in questo momento, pur non avendone alcun motivo, nulla temeva tanto quanto perdere il suo amore. Ma non poteva non essergli riconoscente per il suo comportamento verso di lei, e non poteva non dimostrargli di apprezzarlo. Lui, pur avendo, così le pareva, una spiccata inclinazione per l'attività politica, nella quale doveva sostenere una parte eminente, aveva sacrificato la sua ambizione per lei, senza mai mostrare il più piccolo rimpianto. Era più di prima amorevolmente rispettoso verso di lei, e il pensiero ch'ella non sentisse mai il disagio della propria posizione, non lo abbandonava neppure un attimo. Lui, così virile, nei rapporti con lei non solo non la contrariava mai, ma non aveva una propria volontà e sembrava preoccupato solo dal pensiero di prevenire i desideri di lei. E lei non poteva non apprezzare ciò, sebbene l'intensità delle attenzioni verso di lei, l'atmosfera di premura di cui egli la circondava, a volte le pesassero. 

Vronskij intanto, malgrado il completo appagamento di quello ch'egli aveva così a lungo desiderato, non era pienamente felice. Ben presto sentì che l'appagamento del desiderio gli aveva dato solo un granello di sabbia di quella montagna di felicità che si attendeva. Questo appagamento gli aveva mostrato l'eterno errore che commettono gli uomini che si figurano la felicità nell'appagamento di un desiderio. Nel primo periodo in cui era unito a lei e aveva indossato gli abiti borghesi, aveva sentito tutto l'incanto della libertà che prima non conosceva, e della libertà nell'amore; e ne fu contento, ma non a lungo. Ben presto sentì che nell'animo suo s'era destato il desiderio dei desideri: la malinconia. Indipendentemente dalla propria volontà cominciò ad aggrapparsi ad ogni capriccio passeggero, scambiandolo per un'aspirazione e uno scopo. Sedici ore della giornata bisognava pure occuparle con qualcosa, giacché all'estero vivevano in piena libertà, al di fuori di quella cerchia di condizioni di vita sociale che, a Pietroburgo, assorbiva loro il tempo. Ai piaceri della vita da scapolo, che nei precedenti viaggi all'estero avevano occupato Vronskij, non si poteva neppure pensare, giacché un esperimento di tal genere aveva prodotto in Anna un abbattimento inaspettato e inadeguato in seguito a una cena fatta a tarda ora con amici. Relazioni con la società locale e con quella russa, data l'indeterminatezza della loro posizione, non si potevano avere. La visita ai monumenti più importanti, oltre al fatto che tutto era stato visitato, non aveva per lui, russo e uomo d'ingegno, quell'inspiegabile importanza che le attribuiscono gli inglesi. 

E, come un animale affamato afferra qualsiasi cosa gli capiti, sperando di trovarvi cibo, così pure Vronskij, del tutto inconsapevole, s'aggrappava ora alla politica, ora ai libri nuovi, ora ai quadri. 

In gioventù aveva avuto disposizione alla pittura e, non sapendo come spendere il denaro, aveva cominciato a raccogliere incisioni; si fermò, quindi, sulla pittura, prese ad occuparsene e ripose in essa quella insoddisfatta riserva di desideri che reclamava d'essere appagata. 

Aveva attitudine a intendere l'arte e ad imitare con fedeltà, con gusto, l'opera d'arte; credette così d'avere ciò che occorre all'artista. Dopo un certo tempo d'incertezza sul genere di pittura da scegliere, religioso, storico, di genere o realistico, si mise a dipingere. Intendeva qualsiasi genere, e poteva ispirarsi a questo e a quello; non immaginava che si potesse del tutto ignorare quali generi di pittura esistessero e che ci si potesse ispirare direttamente a quello che c'è nell'anima, senza preoccuparsi se quello che si è dipinto appartiene a un certo determinato genere. Poiché non sapeva questo e non traeva ispirazione direttamente dalla vita, ma mediamente, dalla vita già incarnata nell'arte, egli si ispirava molto alla svelta, e con facilità otteneva che quanto dipingeva fosse molto simile a quel tal genere che voleva imitare. 

Più di tutti gli altri gli piaceva il francese, grazioso e d'effetto, e in questo genere cominciò a fare il ritratto di Anna in costume italiano: questo ritratto, a lui e a tutti quelli che lo vedevano, sembrava molto ben riuscito. 

\capitolo{IX}\label{ix-4} 

Il vecchio palazzo abbandonato, dai soffitti alti, modellati e gli affreschi sui muri, dai pavimenti a mosaico, le pesanti tende di damasco giallo alle finestre alte, e i vasi sulle mensole e sui camini, dalle porte intagliate e le sale oscure con i quadri appesi, questo palazzo, dopo che vi presero alloggio, con lo stesso suo aspetto esteriore, manteneva Vronskij nel piacevole errore ch'egli non fosse tanto il proprietario russo, il gran cacciatore a riposo, quanto un illuminato amatore e protettore di arti, e lui stesso un modesto artista che avesse rinunciato al mondo, alle relazioni, all'ambiente, per la donna amata. 

La parte assunta da Vronskij, col passaggio nel palazzo, riuscì perfettamente, e, fatta la conoscenza di alcune persone interessanti per mezzo di Golenišcev, in un primo tempo egli fu tranquillo. Dipingeva, sotto la guida di un maestro italiano, degli studi dal vero, e si occupava di vita medioevale italiana. La vita medioevale italiana, negli ultimi tempi, aveva tanto affascinato Vronskij che perfino il cappello e lo scialle di lana sulla spalla cominciò a portare alla foggia medioevale, cosa che gli donava molto. 

- E noi viviamo e non sappiamo nulla - disse una volta Vronskij a Golenišcev che era venuto da lui di buon'ora. - Hai veduto il quadro di Michajlov? - disse, tendendogli un giornale russo ricevuto appena quella mattina e mostrandogli un articolo sull'artista russo che viveva nella stessa città e che aveva ultimato un quadro del quale, da lungo tempo, si parlava e che era stato acquistato in anticipo. Nell'articolo c'erano rimproveri al governo e all'accademia perché un artista così notevole era lasciato privo di incoraggiamento e d'aiuti. 

- Ho visto - rispose Golenišcev. - S'intende, egli non è privo di talento, ma è su di una via completamente falsa. Sempre la stessa maniera di trattare il Cristo e la pittura religiosa alla Ivanov-Strauss-Renan. 

- Cosa rappresenta il quadro? - chiese Anna. 

- Cristo dinanzi a Pilato. Cristo è rappresentato come un ebreo, con tutto il realismo della nuova scuola. 

E, portato dalla domanda sul contenuto del quadro a uno dei suoi temi preferiti, Golenišcev cominciò a parlare: 

- Io non capisco come possano sbagliarsi così grossolanamente. Cristo ha già la sua incarnazione definita nell'arte dei grandi\ldots{} Dunque, se non vogliono rappresentare Iddio, ma un rivoluzionario o un saggio, che prendano pure dalla storia Socrate, Franklin, Carlotta Corday, ma Cristo, no. Essi prendono proprio quel personaggio che non si può prendere per l'arte, ma dopo\ldots{} 

- Ebbene, è vero che questo Michajlov si trova in tanta miseria? - chiese Vronskij, pensando che lui, come mecenate russo, avrebbe dovuto aiutare l'artista, bello o brutto che fosse il quadro. 

- È difficile. È un ritrattista famoso. Avete visto il ritratto della Vasil'cikova? Ma sembra che ora non voglia far più ritratti, e perciò è probabile che sia in ristrettezze. Io dico che\ldots{} 

- Non si potrebbe pregarlo di fare il ritratto ad Anna Arkad'evna ? - disse Vronskij. 

- Perché a me? - chiese Anna. - Dopo il tuo, io non voglio altro ritratto. Piuttosto ad Annie - così ella chiamava la bambina. - Eccola - aggiunse, dopo aver dato un'occhiata dalla finestra alla bella balia italiana che aveva portato fuori la bambina in giardino, e voltandosi subito a guardare Vronskij. La bella balia, che serviva da modella a Vronskij per una testa di un suo quadro, era l'unico dolore segreto di Anna. Vronskij, ritraendola, ne ammirava la bellezza e il tipo medioevale, e Anna non aveva il coraggio di confessarsi di temere d'essere gelosa di questa balia, e perciò blandiva e viziava particolarmente lei e il suo bambino. 

Vronskij guardò anche lui dalla finestra e guardò Anna negli occhi, ma poi, voltosi subito a Golenišcev, disse: 

- E tu, lo conosci questo Michajlov? 

- L'ho incontrato. Ma è un originale e non ha nessuna cultura. Sapete, uno di quegli uomini nuovi selvaggi che adesso s'incontrano di frequente; sapete, uno di quei liberi pensatori che sono educati d'embléè nelle idee del materialismo, della negazione, dell'ateismo. Prima succedeva - diceva Golenišcev senza notare o senza voler notare che sia Anna che Vronskij desideravano interloquire - prima succedeva che il libero pensatore fosse un uomo educato nelle idee della religione, della legge, della morale, e che da solo fosse giunto con lotte e stenti al libero pensiero; ma adesso è comparso un nuovo tipo di libero pensatore istintivo, il quale cresce senza neppure sentir dire che ci sono leggi morali, religiose, che ci sono delle autorità; cresce, senz'altro, nell'idea di negar tutto, cioè come un selvaggio. Ecco, lui è così. È figlio, mi pare, di un capocameriere moscovita e non ha ricevuto alcuna istruzione. Quando entrò in accademia e acquistò fama, da un uomo non sciocco qual'era, desiderò di istruirsi. E si rivolse a quella che gli sembrava la fonte della cultura: alle riviste. E voi capite, nei tempi passati, un uomo che avesse voluto istruirsi, mettiamo un francese, avrebbe cominciato con lo studiare tutti i classici, i tragici, gli storici, i filosofi; e voi intendete tutto il lavoro intellettuale che avrebbe avuto dinanzi a sé. Ma da noi, adesso, egli si è imbattuto proprio nella letteratura nichilista: ha fatto sua, molto presto, tutta l'essenza della scienza che nega, ed è bell'e pronto. E ancora questo sarebbe poco: venti anni fa, avrebbe trovato in questa letteratura i segni della lotta con le autorità, con le opinioni secolari, e da questa lotta avrebbe capito che c'era stato qualcosa di diverso; ma adesso si imbatte in una letteratura tale, che, in essa, non vengono degnate neppure di una discussione le opinioni di un tempo, ma si dice francamente: non c'è nulla, évolution, selezione, lotta per l'esistenza, ed è tutto. Io, nel mio articolo\ldots{} 

- Sapete cosa? - disse Anna che già da tempo scambiava, cauta, delle occhiate con Vronskij e sapeva che non la cultura di quell'artista lo interessava, ma solo l'idea di aiutarlo e di ordinargli il ritratto. - Sapete cosa? - disse interrompendo decisa Golenišcev che non la smetteva di parlare. - Andiamo da lui! 

Golenišcev ritornò in sé e acconsentì volentieri. Ma poiché l'artista abitava lontano, stabilirono di prendere una vettura. 

Dopo un'ora Anna, a fianco di Golenišcev, e Vronskij invece seduto nel sedile anteriore della vettura, si avvicinavano a una brutta costruzione moderna di un quartiere periferico. Avendo saputo dalla moglie del portiere, che venne loro incontro, che Michajlov riceveva nel suo studio, ma che in quel momento era a casa, in un quartiere a due passi da lì, la mandarono da lui con i loro biglietti da visita, per chiedere il permesso di vedere i quadri. 

\capitolo{X}\label{x-4} 

Il pittore Michajlov, come sempre del resto, era al lavoro quando gli portarono i biglietti da visita di Vronskij e di Golenišcev. La mattina aveva lavorato nello studio al quadro grande. Venuto a casa, si era arrabbiato con la moglie perché non aveva saputo destreggiarsi con la padrona di casa che pretendeva denaro. 

- Te l'ho detto venti volte, non metterti a dare spiegazioni. Anche così sei sciocca, ma se cominci a spiegarti in italiano, allora diventi tre volte sciocca - le disse dopo una lunga discussione. 

- Allora tu non trascurare, la colpa non è mia. Se avessi denaro\ldots{} 

- Lasciami in pace, per l'amor di Dio! - gridò Michajlov con le lacrime nella voce e, tappatesi le orecchie, andò nella sua stanza di lavoro, di là da un tramezzo, chiudendo la porta dietro di sé. ``Insensata!'' borbottava, sedendo al tavolo e, distesovi un cartone, cominciò subito a lavorare con particolare ardore a un disegno incominciato. 

Non lavorava mai con tanto ardore e successo come quando la vita gli andava male e, soprattutto, quando litigava con la moglie. ``Ah! se potesse sprofondare in qualche parte!'' pensava e continuava a lavorare. Stava facendo il disegno per una figura d'uomo in stato d'ira. Il disegno era già fatto; ma egli non ne era contento. ``No, quell'altro era migliore\ldots{} Dov'è?''. Andò dalla moglie e, accigliato, senza guardarla, domandò alla bambina più grande dov'era quel cartone che aveva dato loro. Il cartone col disegno abbandonato si trovò, ma era sporco e macchiato di stearina. Egli prese, tuttavia, il disegno, lo stese dinanzi a sé sul tavolo e, allontanandolo e socchiudendo gli occhi, cominciò a guardarlo. Improvvisamente sorrise e agitò con gioia le mani. 

- Sì, così - pronunciò, e subito, afferrata una matita, cominciò a disegnare alla svelta. La macchia di stearina aveva dato un atteggiamento nuovo alla figura. 

Disegnava questo nuovo atteggiamento e, a un tratto, gli venne in mente il viso energico, dal mento prominente, del venditore di sigari; e quello stesso viso, quello stesso mento egli dette alla sua figura. Rise di gioia. La figura a un tratto da morta che era, inventata, divenne viva e tale che non si poteva mutarla. Questa figura viveva, ed era definita con chiarezza e precisione. Si poteva correggere il disegno secondo le esigenze della figura, si poteva e si doveva perfino divaricare le gambe in altro modo, si poteva cambiare del tutto la posizione della mano sinistra, buttare all'indietro i capelli. Eppure, facendo queste correzioni, egli non alterava la figura, ma toglieva solo quello che la velava. Era come se togliesse via quei veli che non la rendevano perfettamente visibile; ogni nuovo tratto tendeva solo a maggiormente esprimere tutta la figura nella sua forza, così come gli era emersa d'un tratto per effetto della macchia prodotta dalla stearina. Stava finendo con cautela la figura quando gli portarono i biglietti. 

- Subito, subito! 

E andò dalla moglie. 

- Su, via, Saša, non ti arrabbiare! - disse, sorridendo timido e affettuoso. - Tu non ne hai colpa. La colpa è mia. Accomoderò tutto. - E rappacificatosi con la moglie, si mise il cappotto olivastro dal collo di velluto e il cappello e andò nello studio. Aveva già dimenticato la figura che gli era riuscita. Ora lo rallegrava e lo agitava la visita del suo studio da parte di quei personaggi russi importanti, arrivati in vettura. 

Del proprio quadro, quello che stava attualmente sul cavalletto, nel profondo dell'animo suo, aveva una sola idea, che un quadro simile nessuno mai l'avesse dipinto. Egli non pensava che fosse migliore di tutti quelli di Raffaello, ma sapeva che quanto aveva voluto esprimervi, nessuno mai l'aveva espresso. Questo lo sapeva fermamente e lo sapeva da gran tempo, da quando aveva cominciato a dipingerlo; ma i giudizi degli altri, quali che fossero, avevano tuttavia per lui un'importanza enorme e lo agitavano fino in fondo all'anima. Ogni osservazione, anche la più inconsistente, che mostrasse che i giudici vedevano una sia pur piccola parte di quello che vedeva lui in quel quadro, lo impressionava profondamente. Ai suoi giudici attribuiva sempre una profondità di comprensione maggiore di quella che lui stesso non avesse, e da loro aspettava sempre qualcosa che lui stesso non aveva scorto nel quadro. E spesso, nei giudizi degli osservatori, gli sembrava di trovarlo questo qualcosa. 

Egli si avvicinava con passo svelto alla porta dello studio e, malgrado l'agitazione, la figura di Anna tenuemente illuminata che, dritta nell'ombra del portone, ascoltava Golenišcev che le parlava calorosamente di qualcosa, e che nel medesimo tempo mostrava di voler esaminare l'artista che si avvicinava, lo stupì. Ma non notò neppure che, nell'avvicinarsi a loro, egli aveva afferrato e assorbito questa impressione, così come aveva fatto col mento del venditore di sigari, per nasconderla chi sa dove, e trarla fuori di là quando ce ne sarebbe stato bisogno. I visitatori, delusi già dalle precedenti descrizioni di Golenišcev del pittore, furono ancor più delusi dal suo aspetto esteriore. Di media statura, tarchiato, con un'andatura inquieta, Michajlov col cappello marrone, il cappotto olivastro e i pantaloni stretti, quando già da tempo si portavano larghi, con quel suo viso piatto, ordinario, e con quella espressione di timidezza, mista al desiderio di mantenere la propria dignità, produsse proprio un'impressione sgradevole. 

- Vi prego di voler passare - egli disse, cercando di prendere un'aria indifferente e, entrato nell'ingresso, tirò fuori di tasca la chiave, e aprì la porta. 

\capitolo{XI}\label{xi-4} 

Entrando nello studio, il pittore Michajlov guardò ancora una volta gli ospiti e annotò ancora, nella sua fantasia, l'espressione del viso di Vronskij, in particolare i suoi zigomi. Sebbene il suo senso artistico lavorasse continuamente per raccogliere materiale, sebbene egli sentisse una agitazione sempre maggiore all'avvicinarsi del giudizio sul suo lavoro, tuttavia con rapidità e finezza d'intuito, attraverso segni impercettibili, si andò formando una idea su quelle tre persone. Quello (Golenišcev), era un russo di qua. Michajlov non ne ricordava il nome, né dove l'avesse incontrato, né di che cosa avessero parlato insieme. Ricordava solo il suo viso, come ricordava tutti i visi che vedeva una sola volta o due; ma ricordava anche che era uno di quei visi messi da parte, nella sua immaginazione, nel gran reparto di quelli dall'espressione mutevole e povera. I capelli lunghi e la fronte molto aperta davano un'importanza esteriore a un viso nel quale non c'era che una piccola espressione infantile, concentrata sopra la radice stretta del naso. Vronskij e la Karenina, secondo le considerazioni di Michajlov, dovevano essere russi di gran famiglia e ricchi, che non capivano nulla di arte, come tutti i russi ricchi, del resto, ma che se ne mostravano intenditori ed amatori. ``Probabilmente, ormai, avranno visitato le antichità e ora fanno il giro degli studi moderni, di un ciarlatano tedesco o di uno stupido inglese preraffaellita, e da me sono venuti solo per completare la visita'' pensava. Sapeva molto bene il modo di fare dei dilettanti (e quanto più intelligenti sono, tanto è peggio), di visitare, cioè, gli studi degli artisti contemporanei col solo scopo di avere il diritto di dire che l'arte è in decadenza, e che quanto più si guardano i nuovi, tanto più si vede come siano rimasti inimitabili i grandi maestri d'un tempo. Tutto questo se l'aspettava, tutto questo lo scorgeva nei loro visi, lo vedeva nell'indifferente negligenza con cui parlavano fra di loro, con cui guardavano i manichini e i busti e passeggiavano liberamente, aspettando ch'egli scoprisse il quadro. Malgrado ciò, mentre voltava i suoi studi, sollevava le tendine e toglieva il lenzuolo, egli sentiva una grande agitazione, tanto più che sebbene, secondo lui, tutti i russi di gran famiglia e ricchi dovessero essere bestie e stupidi, Vronskij e in modo particolare Anna gli piacevano. 

- Ecco, volete favorire? - disse, allontanandosi da un lato con la sua andatura irrequieta e indicando il quadro. - È l'esortazione di Pilato. Capitolo XXVII del vangelo di Matteo - disse, sentendo che le labbra cominciavano a tremargli per l'agitazione. Si allontanò e si mise dietro di loro. 

In quei pochi secondi in cui i visitatori guardarono il quadro in silenzio, anche Michajlov lo guardò con occhio indifferente, distaccato. In quei pochi secondi egli credette, in principio, che il giudizio più alto e più giusto sarebbe stato pronunciato da loro, proprio da quei visitatori che aveva tanto disprezzato un momento prima. Aveva dimenticato tutto quello che pensava del suo quadro prima, in quei tre anni in cui l'aveva dipinto, aveva dimenticato tutti i pregi di cui non aveva dubitato; guardava ora il quadro con occhio indifferente, distaccato, nuovo, e non ci vedeva nulla di bello. Vedeva, in primo piano, il viso irritato di Pilato e quello calmo del Cristo e, in secondo piano, le figure dei servi di Pilato e il viso di Giovanni che osservava quanto accadeva. Ogni viso che era sorto in lui dopo tanta ricerca, dopo tanti errori e correzioni, col proprio carattere a sé stante, ogni viso che gli aveva dato tanto tormento e tanta gioia, e tutti quei visi tante volte cambiati di posto per l'insieme, tutte le sfumature di colori e di toni, ottenute con tanto sforzo, adesso tutto quell'insieme, visto con gli occhi loro, gli sembrava una cosa volgare, mille volte ripetuta. Il viso che più gli era caro, quello del Cristo, nel centro del quadro, che tanto entusiasmo gli aveva dato quando lo aveva scoperto, si disperse tutto per lui, ora che guardava il quadro con gli occhi loro. Vedeva una copia ben fatta (anzi neanche ben fatta, adesso ci vedeva un cumulo di difetti), di quegli infiniti Cristi del Tiziano, di Raffaello, del Rubens e di quegli stessi soldati e Pilati. Tutto questo era volgare, povero e risaputo e perfino dipinto male, con troppi colori e con fiacchezza. Essi avrebbero avuto ragione di pronunciare frasi di cortese ipocrisia dinanzi all'artefice per poi compiangerlo e irriderlo quando fossero rimasti soli. 

Quel silenzio gli divenne troppo penoso (sebbene non durasse più di un minuto). Per spezzarlo e per far vedere che non era agitato, fatto uno sforzo su di sé, si rivolse a Golenišcev. 

- Mi pare di aver avuto il piacere di incontrarvi - disse, volgendosi a guardare con inquietudine ora Anna, ora Vronskij per non perdere neppure un tratto dell'espressione dei loro visi. 

- E come! ci siamo visti in casa Rossi, ricordate, quella sera in cui quella signorina italiana, una nuova Rachel, declamava - cominciò a dire con disinvoltura Golenišcev, distaccando senza il minimo rimpianto gli occhi dal quadro e rivolgendosi all'artista. 

Avendo però notato che Michajlov si aspettava un giudizio sul quadro, aggiunse: 

- Il vostro quadro è andato molto avanti da quando l'ho visto l'ultima volta. E, come allora, anche adesso mi colpisce straordinariamente la figura di Pilato. Si capisce così bene che quest'essere, questo buono e bravo giovane, ma burocrate fino in fondo all'anima, non sa quello che fa. Ma mi pare\ldots{} 

Tutto il viso mobile di Michajlov a un tratto si illuminò: gli occhi si accesero. Voleva dire qualcosa, ma non poté pronunciare nulla per l'agitazione, e finse di schiarirsi la voce. Per quanto mediocre egli stimasse la possibilità di intendere l'arte di Golenišcev, per quanto inconsistente fosse quella giusta osservazione sulla vera espressione di Pilato come burocrate, per quanto increscioso dovesse riuscirgli il vedere espressa per prima un'osservazione così inconsistente, mentre non si diceva nulla delle cose importanti, Michajlov fu entusiasta di quella osservazione. Anch'egli pensava della figura di Pilato quello che aveva detto Golenišcev. Che questa osservazione fosse una delle infinite che, Michajlov fermamente sapeva, sarebbero state tutte giuste, non diminuì per lui il significato dell'osservazione di Golenišcev. Prese ad amare Golenišcev per quella osservazione, e da uno stato di abbattimento passò a un tratto all'entusiasmo. Improvvisamente il quadro tornò a vivere dinanzi a lui con tutta l'indicibile complessità di quello che vive. Michajlov di nuovo tentò di esprimere, che egli così intendeva Pilato; ma le labbra gli tremarono indocili, ed egli non poté pronunciarlo. Vronskij ed Anna pure dicevano qualcosa, con quella voce bassa con cui di solito si parla alle mostre dei quadri e per non offendere l'artista e per non dire ad alta voce una sciocchezza, così facile a dirsi in tema d'arte. A Michajlov sembrava che il quadro avesse fatto impressione anche su di loro. Si avvicinò. 

- Com'è sorprendente l'espressione del Cristo! - disse Anna. Di tutto quello che aveva visto questa espressione le era piaciuta maggiormente, e sentiva che questa era il nucleo del quadro, e che perciò la lode avrebbe fatto piacere all'artista. - Si vede che ha pena di Pilato. 

Era di nuovo una di quelle infinite considerazioni giuste che si potevano fare sul quadro e sulla figura di Cristo. Ella aveva detto ch'egli aveva pena di Pilato. Nell'espressione del Cristo ci doveva essere anche un'espressione di pena perché in Lui c'era l'espressione dell'amore, della calma ultraterrena, della preparazione alla morte e della consapevolezza della vanità delle parole. Naturalmente, in Pilato, c'era l'espressione del burocrate, e nel Cristo la pietà, giacché l'uno è la personificazione della vita del corpo, l'altro della vita dello spirito. Tutto ciò e molte altre cose balenarono nella mente di Michajlov. E di nuovo il suo viso si illuminò di entusiasmo. 

- Sì, e come è fatta questa figura, quanto respiro! Si può girarle intorno - disse Golenišcev, mostrando evidentemente, con questa osservazione, che non approvava il contenuto e il pensiero della figura. 

- Sì, d'una maestria eccezionale! - disse Vronskij. - Come risaltano queste figure sullo sfondo! Ecco la tecnica - egli disse rivolto a Golenišcev, alludendo a una conversazione che era corsa fra di loro sul fatto che Vronskij disperava di acquistare questa tecnica. 

- Sì, sì, sorprendente! - confermarono Golenišcev e Anna. 

Malgrado lo stato di eccitamento in cui era, l'osservazione sulla tecnica provocò una fitta dolorosa al cuore di Michajlov, ed egli, dopo aver guardato irritato Vronskij, si accigliò a un tratto. Aveva sentito spesso la parola tecnica, e decisamente non capiva che cosa si intendesse con questa parola. Egli sapeva che con questa parola si intendeva la facoltà meccanica di dipingere e di disegnare, del tutto indipendente dal contenuto. Spesso aveva notato come anche, in un giudizio veritiero, si contrapponesse la tecnica al valore intimo del lavoro, come se fosse possibile dipingere bene quello che era informe. Sapeva che ci voleva molta attenzione e cautela per non danneggiare l'opera stessa nel toglierle un velo o tutti i veli; ma l'arte non aveva nulla a che fare con la tecnica. Se a un fanciullo o alla propria cuoca si fosse dischiuso tutto quello ch'egli aveva visto, allora anche costei avrebbe potuto cavar fuori quello che vedeva. Invece, il più esperto e abile maestro della tecnica, con la sola facoltà meccanica, non può dipingere nulla se non gli si dischiudono prima le possibilità del contenuto. Inoltre egli vedeva che, a parlar di tecnica, non gli si potevano certo fare degli elogi. In tutto quello che dipingeva e aveva dipinto vedeva difetti che gli ferivano gli occhi dovuti all'audacia con la quale toglieva i veli, e che ormai non poteva correggere senza sciupare l'intera opera. E su quasi tutte le figure e i volti egli vedeva ancora i resti dei veli, non completamente tolti, che sciupavano il quadro. 

- Una cosa si potrebbe dire, se mi permettete di fare quest'osservazione\ldots{} - notò Golenišcev. 

- Ah, sono molto contento, e ve ne prego - disse Michajlov, sorridendo con finzione. 

- È che voi avete fatto di lui un uomo-Dio e non un Dio-uomo. Del resto so che volevate proprio questo. 

- Non potevo dipingere quel Cristo che non ho nell'anima - disse Michajlov torvo. 

- Sì, ma in tal caso, se mi permettete di dire la mia idea\ldots{} il vostro quadro, del resto, è così bello che la mia osservazione non lo guasta, e poi è una mia opinione personale. In voi è un'altra cosa. Anche il motivo è un altro. Ma prendiamo, magari, Ivanov. Io credo che, se il Cristo è abbassato al grado di personaggio storico, sarebbe stato meglio per Ivanov scegliere un altro tema storico, fresco, non sfruttato. 

- Ma se questo è il tema più alto che si presenti all'arte? 

- A cercare se ne troveranno altri. Ma l'arte non ammette discussioni e ragionamenti. E davanti al quadro di Ivanov per il credente e per il miscredente si presenta la questione: ``È Dio o non è Dio?'' e ciò distrugge l'unità dell'impressione. 

- E perché? Mi pare che per le persone colte - disse Michajlov - ormai non possa esistere discussione. 

Golenišcev non acconsentì a questo e, attenendosi alla sua prima idea sull'unità dell'impressione, necessaria all'arte, sgominò Michajlov. 

Michajlov si agitava, ma non sapeva dire nulla in difesa della propria idea. 

\capitolo{XII}\label{xii-4} 

Anna e Vronskij già da tempo si scambiavano occhiate, deplorando la concettosa verbosità del loro amico; finalmente Vronskij, senza aspettare il padrone di casa, passò a un altro piccolo quadro. 

- Ah, che incanto! Una meraviglia! Che incanto! - dissero a una voce. 

``Che cosa mai è loro piaciuto tanto?'' pensò Michajlov. Si era perfino dimenticato di quel quadro da lui dipinto tre anni prima. Aveva dimenticato tutte le pene e gli entusiasmi che aveva vissuto per quel quadro, che per vari mesi lo aveva avvinto incessantemente di notte e di giorno; lo aveva dimenticato, come sempre dimenticava i quadri compiuti. Non gli piaceva neppure più guardarlo, e lo aveva esposto solo perché aspettava un inglese che desiderava comprarlo. 

- Già, ecco, un vecchio studio - disse. 

- Com'è bello! - disse Golenišcev, anche lui, evidentemente, preso dalla grazia del quadro. 

Due ragazzi, all'ombra di un canneto, pescavano con la lenza. Uno di loro, il più grande, aveva appena gettato la lenza e faceva uscire con cura il galleggiante di là dal cespuglio, tutto assorto in questa faccenda; l'altro, il più piccolo, stava sdraiato sull'erba, poggiando la testa bionda e scarmigliata sulle braccia, e guardava con gli occhi azzurri e pensosi l'acqua. A cosa pensava? 

L'entusiasmo dinanzi a questo suo quadro provocò in Michajlov la stessa agitazione di prima, ma egli temeva e non amava quell'ozioso sentimento verso ciò che era compiuto, e perciò, pur rallegrato dalle lodi, volle attrarre l'attenzione dei visitatori verso un terzo quadro. 

Ma Vronskij chiese se il quadro era in vendita. In quel momento per Michajlov, agitato dai visitatori, un discorso su di una questione di denaro era molto spiacevole. 

- È esposto per la vendita - rispose, accigliandosi cupo. 

Quando i visitatori se ne furono andati, Michajlov sedette di fronte al quadro di Pilato e di Cristo, e nella sua mente riandò a tutto quello che era stato detto e, anche se non detto, sottinteso dai visitatori. E, cosa strana, quello che aveva avuto tanta importanza per lui mentre essi erano là, quando egli si era trasferito col pensiero nel loro modo di vedere, a un tratto, perse ogni significato. Cominciò a guardare il quadro con uno sguardo pienamente d'artista e giunse ad essere sicuro della sua bellezza e, perciò, della sua importanza, cosa di cui aveva bisogno per giungere a quella tensione che escludeva ogni altro interesse e che sola gli era necessaria per lavorare. 

La gamba del Cristo, di scorcio, non era, tuttavia, come doveva. Prese la tavolozza e si mise a lavorare. Correggendo la gamba, egli osservava continuamente la figura di Giovanni in secondo piano, che i visitatori non avevano neppure notato, ma che era, egli lo sapeva, la cosa più compiuta. Ritoccata la gamba, voleva mettersi a lavorare intorno a quella figura, ma si sentiva troppo agitato per far questo. Non riusciva a lavorare né quand'era troppo calmo né quand'era troppo commosso e vedeva tutto con chiarezza eccessiva. Vi era solo una gradazione, nel passaggio tra l'aridità e l'ispirazione, in cui era possibile lavorare. Ma ora egli era troppo agitato. Voleva coprire il quadro, ma si fermò, e, trattenuto con la mano il lenzuolo, sorridendo beato, si soffermò a lungo a guardare la figura di Giovanni. Infine, quasi distaccandosene con rimpianto, abbassò il lenzuolo e stanco, ma felice, andò a casa. 

Vronskij, Anna e Golenišcev, nel tornare a casa, erano particolarmente animati e allegri. Parlavano di Michajlov e dei suoi quadri. La parola ``talento'', con cui intendevano una qualità innata, quasi fisica, indipendente dalla mente e dal cuore, e con la quale volevano definire tutto quello che era stato vissuto dall'artista, ricorreva particolarmente spesso nella loro conversazione, poiché era loro indispensabile fissare in termini quello di cui non avevano nessuna idea, ma di cui volevano parlare. Dicevano che il talento non gli si poteva negare, ma che il suo talento non si era sviluppato per mancanza di cultura, calamità comune a tutti gli artisti russi. Ma il quadro dei ragazzi era loro rimasto nella memoria e ogni tanto vi tornavano su. 

- Che incanto! Come gli è riuscito bene e come è semplice! E lui non capisce neppure come sia bello. Non bisogna lasciarselo sfuggire e comprarlo - diceva Vronskij. 

\capitolo{XIII}\label{xiii-4} 

Michajlov vendette il quadro a Vronskij e acconsentì a fare il ritratto ad Anna. Nel giorno stabilito venne e cominciò il lavoro. 

Il ritratto, dopo cinque sedute, colpì tutti, e in particolare Vronskij, non soltanto per la somiglianza, ma anche per la sua particolare bellezza. Era strano come Michajlov avesse potuto cogliere la bellezza particolare di lei. ``Bisognava amarla e conoscerla, come l'ho amata io, per cogliere proprio quella sua cara espressione spirituale'' pensava Vronskij, pur avendo, solo da questo ritratto, imparato a conoscere quella sua cara espressione spirituale. Ma l'espressione era così vera, che a lui e agli altri pareva di conoscerla da tempo. 

- Io mi affatico da tempo e non ho concluso nulla - diceva del proprio ritratto - e lui ha guardato e ha dipinto. Ecco, cosa vuol dire la tecnica. 

- Questa verrà - lo consolava Golenišcev, nella cui opinione Vronskij aveva un certo talento e, soprattutto, quella cultura che dà una visione superiore dell'arte. La fiducia di Golenišcev nel talento di Vronskij era sostenuta anche dal fatto che egli aveva bisogno della simpatia e delle lodi di Vronskij per i suoi articoli e per le sue idee, e sentiva che le lodi e l'appoggio dovevano essere scambievoli. 

In casa altrui, e in particolare nel palazzo di Vronskij, Michajlov era tutt'altro uomo che non nel suo studio. Era ostilmente rispettoso, quasi temesse l'amicizia con persone che non stimava. Chiamava Vronskij ``vostra eccellenza'', non rimaneva mai a pranzo, malgrado gli inviti di Anna e di Vronskij, e non veniva che per le sedute. Anna era verso di lui più cordiale che non gli altri e gli era riconoscente per il ritratto. Vronskij era con lui più che cortese e, evidentemente, si interessava del giudizio dell'artista sul proprio quadro. Golenišcev non lasciava sfuggir l'occasione per ispirare a Michajlov le vere idee sull'arte. Ma Michajlov rimaneva egualmente freddo verso tutti. Anna sentiva dal suo sguardo ch'egli si compiaceva di guardarla, ma che sfuggiva le conversazioni con lei. Nei discorsi di Vronskij sulla sua pittura taceva ostinatamente e così pure ostinatamente taceva quando gli facevano vedere il quadro di Vronskij; era evidente che i discorsi di Golenišcev gli pesavano e non li ribatteva. 

In generale Michajlov, col suo modo di trattare sostenuto e freddo, quasi ostile, non piacque loro per nulla, quando lo conobbero più da vicino. E furono contenti allorché, finite le sedute, rimase nelle loro mani un magnifico ritratto ed egli cessò di venire. 

Golenišcev, per primo, espresse l'idea che tutti avevano avuto, che cioè Michajlov fosse semplicemente invidioso di Vronskij. 

- Ammettiamo che non provi invidia, perché ha talento; ma lo irrita il fatto che un uomo di corte e ricco, conte per giunta (perché loro odiano tutto ciò), senza particolare fatica, faccia la stessa cosa, se non pure meglio, di lui che vi ha dedicato tutta la vita. Perché ciò che più conta è la cultura che lui non ha. 

Vronskij difendeva Michajlov, ma in fondo all'animo credeva in questo, perché, secondo lui, un uomo d'un altro mondo inferiore doveva provare invidia. 

Il ritratto di Anna, la stessa cosa dipinta dal vero da lui e da Michajlov, avrebbe dovuto mostrare a Vronskij la differenza che esisteva fra lui e Michajlov; ma egli non la vedeva. Decise soltanto di non lavorare più al ritratto di Anna, ritenendolo ormai superfluo, dopo quello di Michajlov. Continuò, invece, il quadro di ambiente medioevale. E lui stesso e Golenišcev, e in particolare Anna, lo ritenevano molto bello, perché molto più somigliante ai quadri famosi che non il quadro di Michajlov. 

Michajlov intanto, malgrado il ritratto di Anna lo avesse molto appassionato, fu ancora più contento di loro quando le sedute terminarono ed egli non fu più costretto a sentire il vaniloquio di Golenišcev sull'arte, e poté cancellare dalla memoria la pittura di Vronskij. Egli sapeva che non si poteva proibire a Vronskij di divertirsi con la pittura; sapeva che lui e tutti i dilettanti avevano il pieno diritto di dipingere quello che pareva loro, ma questo gli spiaceva. Non si può proibire a un uomo di farsi una gran bambola di cera e di baciarla. Ma se quest'uomo con la bambola venisse a sedere dinanzi a un innamorato e cominciasse ad accarezzare la bambola così come l'innamorato può accarezzare colei che ama, all'innamorato questo spiacerebbe. Un sentimento simile di sgradevolezza provava Michajlov alla vista della pittura di Vronskij; provava scherno e stizza, pena e offesa. 

La passione di Vronskij per la pittura e il medioevo non durò a lungo. Aveva tanto gusto in pittura che non poteva terminare il proprio quadro. Il quadro si fermò. Egli sentiva confusamente che i suoi difetti, poco avvertiti nell'abbozzo, sarebbero apparsi rilevanti, se avesse continuato. Gli accadeva la stessa cosa che accadeva a Golenišcev il quale sentiva di non aver nulla da dire, e ingannava continuamente se stesso col dire che il suo pensiero non era ancora maturo, che lo avrebbe compiuto e che preparava materiali. Ma Golenišcev era irritato e tormentato da tutto ciò, mentre Vronskij non poteva ingannarsi e tormentarsi e, soprattutto non poteva irritarsi. Con la decisione propria del suo carattere, senza spiegar nulla e senza giustificarsi, smise di occuparsi di pittura. 

Ma, senza questa occupazione, la sua vita e quella di Anna, sorpresa della delusione di lui, sembrarono così noiose nella cittadina italiana, così evidentemente vecchio e sudicio parve, a un tratto, il palazzo, così spiacevoli parvero le macchie sulle tende, le crepe sui pavimenti, lo stucco spaccato sui cornicioni e così tedioso parve il fatto di veder sempre lo stesso Golenišcev, il professore italiano e il viaggiatore tedesco, che bisognò cambiar vita. Decisero di andare in Russia, in campagna, a Pietroburgo. Vronskij aveva in mente di fare la spartizione dei beni col fratello e Anna di vedere il figlio. D'estate pensavano poi di andare nella grande tenuta di Vronskij. 

\capitolo{XIV}\label{xiv-4} 

Levin era ammogliato da tre mesi. Era felice, ma in modo del tutto diverso da come si aspettava. A ogni passo, egli provava una delusione per quello che aveva sognato e un incanto nuovo, inaspettato. Levin era felice, ma, entrato nella vita di famiglia, vedeva a ogni passo che era completamente diversa da quella che aveva immaginato. A ogni passo provava quello che prova l'uomo che, dopo aver ammirato il facile, sereno incedere di una barchetta su di un lago, segga egli stesso in quella barchetta. Vede che non basta star seduti senza ondeggiare, ma che bisogna riflettere, senza dimenticare neppure per un attimo la direzione, che sotto i piedi c'è l'acqua e che bisogna remare, e alle braccia non abituate ciò fa male, che soltanto stare a guardare è facile, ma farlo, anche se piacevole, è molto difficile. 

Da scapolo gli era accaduto, nell'osservare la vita matrimoniale altrui, di sorridere con sprezzo nell'animo suo delle preoccupazioni meschine, dei litigi, delle gelosie. Secondo lui, nella sua futura vita coniugale non solo non poteva esserci nulla di simile, ma anche tutte le forme esteriori dovevano essere completamente diverse da quelle degli altri, così almeno gli pareva. E a un tratto, invece, la sua vita coniugale non solo non si svolgeva in modo particolare, ma, al contrario, risultava tutta fatta di quelle stesse insignificanti meschinità, così disprezzate prima, e che adesso, contro il suo volere, acquistavano una straordinaria e incontestabile importanza. E Levin vedeva che la organizzazione di tutte queste piccole cose era tutt'altro che facile come gli sembrava prima. Pur credendo di avere le idee più chiare sulla vita familiare, Levin, come tutti gli uomini, immaginava istintivamente la vita di famiglia come una gioia di amare che nulla deve impedire, e dalla quale le piccole preoccupazioni non devono distogliere. Secondo lui, egli doveva attendere al proprio lavoro e prender riposo da questo nella gioia d'amare. Lei doveva essere amata e basta. Egli dimenticava, infatti, come tutti gli uomini, che anche lei doveva attendere a un suo lavoro. E si sorprendeva come lei, quella poetica e deliziosa Kitty, potesse non solo nelle prime settimane, ma nei primi giorni di vita in comune, pensare, affannarsi e ricordarsi delle tovaglie, dei mobili, delle materasse per gli ospiti, di un vassoio, del cuoco, del pranzo e via di seguito. Già quand'era stato fidanzato, si era sorpreso della sicurezza con cui ella aveva rinunciato al viaggio all'estero e aveva deciso di andare in campagna, quasi avesse avuto in mente qualcosa che si doveva fare, e quasi ella potesse pensare, oltre al suo amore, a un qualcosa che ne fosse al di fuori. Questo lo aveva offeso allora, e anche adesso le cure e le preoccupazioni meschine di lei, lo offendevano. Ma vedeva che questo le era indispensabile. E amandola, pur irridendo a queste preoccupazioni, senza saperne il perché, non poteva non compiacersene. Egli scherzava sul modo di disporre i mobili portati da Mosca, di attaccare le tendine, di adornare modernamente la sua stanza, di predisporre la distribuzione degli ospiti, di Dolly, di sistemare la sua nuova cameriera, di ordinare il pranzo al vecchio cuoco, di entrare in discussione con Agaf'ja Michajlovna, allontanandola dalla dispensa. Egli vedeva che il vecchio cuoco sorrideva, compiacendosi, nell'ascoltare gli ordini di lei, inesperti, assurdi; vedeva che Agaf'ja Michajlovna scoteva il capo pensosa e tenera alla nuove disposizioni della giovane signora nella dispensa; vedeva che Kitty era straordinariamente graziosa, quando, ridendo e piangendo, veniva da lui a dirgli che Maša, la cameriera, era abituata a considerarla signorina e che, perciò, nessuno le obbediva. Tutto questo gli sembrava grazioso, ma strano, e pensava che, senza tutto questo, sarebbero stati meglio. 

Non intendeva quale senso di mutamento ella provasse ora che, dopo aver desiderato, talvolta, a casa i cavoli col kvas o i confetti e non aver avuto né gli uni né gli altri, poteva ordinare tutto quello che voleva, comprare mucchi di confetti, spendere quanto denaro voleva e ordinare tutti i pasticcini che desiderava. 

Adesso sognava con gioia l'arrivo di Dolly con i bambini, proprio perché avrebbe potuto ordinare per i bambini i dolci preferiti da ognuno, e perché Dolly avrebbe apprezzato tutta la sua nuova organizzazione. Lei stessa non sapeva perché, ma le faccende domestiche l'attiravano irresistibilmente. Sentendo per istinto l'avvicinarsi della primavera e sapendo che ci sarebbero state anche le giornate grigie, intesseva, così come poteva, il suo nido, e si affrettava a intesserlo e a imparare a intesserlo nello stesso tempo. 

Quell'affannarsi di Kitty, fatto di piccole cose, così contrario all'ideale di una felicità più alta che Levin aveva nei primi tempi, era una delle delusioni, mentre questo stesso grazioso affannarsi di cui non capiva il senso, ma che non poteva non aver caro, era uno dei nuovi incanti. 

Un'altra delusione e un altro incanto erano i litigi. Levin non avrebbe mai potuto immaginare che tra lui e sua moglie potessero esservi altri rapporti oltre quelli teneri, comprensivi, pieni d'amore; e a un tratto, invece, fin dai primi giorni litigarono, in modo tale ch'ella gli rimproverò di non amarla, di amare solo se stesso e si mise a piangere e ad agitar le mani. 

Questo loro primo litigio avvenne perché Levin era andato alla nuova fattoria, vi si era trattenuto una mezz'ora di più, e, volendo passare per la strada più breve, vi si era smarrito. Andava a casa, pensando solo a lei, al suo amore, alla sua felicità, e quanto più si avvicinava, tanto più si accendeva in lui la tenerezza per lei. Corse in camera con quello stesso sentimento, e ancor più forte, con il quale era andato a casa Šcerbackij a far la sua domanda di matrimonio. E a un tratto, invece, lo accolse un'espressione torva, mai vista in lei. Voleva baciarla, ella lo respinse. 

- Cos'è successo? 

- Tu sei allegro\ldots{} - cominciò lei, cercando d'essere velenosamente calma. 

Ma aveva appena aperto la bocca, che parole di rimprovero, d'insensata gelosia e tutto quello che l'aveva tormentata nella mezz'ora trascorsa immobile accanto alla finestra, le sfuggirono. Soltanto ora, per la prima volta, egli capì con chiarezza quello che non aveva capito quando, dopo il rito, l'aveva condotta fuori della chiesa. Capì che, non solo ella gli era vicina, ma che ora non sapeva più dove finiva lei e dove cominciava lui. Capì ora questo, attraverso il tormentoso senso di sdoppiamento che provava. Si sentì offeso dapprima, ma nello stesso momento sentì che non poteva essere offeso da lei che era lui stesso. Provò in un primo momento, una sensazione simile a quella che prova un uomo che, ricevuto a un tratto un forte colpo alle spalle si volti con rabbia e con desiderio di vendetta per trovare il colpevole, e si convinca che è stato lui stesso a colpirsi involontariamente e non c'è contro chi arrabbiarsi e bisogna sopportare e placare il dolore. 

In seguito non provò mai più con tanta forza una sensazione simile, ma quella prima volta, a lungo, non riuscì a riaversi. Un sentimento istintivo pretendeva la giustificazione e la dimostrazione della colpa di lei; ma mostrare la colpa di lei significava irritarla maggiormente e aumentare quel distacco che era la causa di tutta la pena. Un sentimento consueto lo spingeva a scrollare da sé la colpa e a rigettarla su di lei; un altro sentimento più forte lo spingeva a eliminare presto, il più presto possibile, il distacco avvenuto, senza consentirgli di aumentare. Rimanere sotto un'accusa così ingiusta era tormentoso, ma, dopo essersi giustificato, farle del male, era ancora peggio. Come un uomo affannato dal dolore nel dormiveglia, egli voleva strappare, gettar via da sé il punto dolente e, tornato in sé, sentiva che il punto dolente era lui stesso. Bisognava soltanto aiutare a sopportare il punto dolente, ed egli si sforzò di far questo. 

Si rappacificarono. Lei, riconosciuta la propria colpa, anche senza confessarlo, divenne più tenera verso di lui, ed essi provarono una nuova raddoppiata gioia d'amore. Ma ciò non impedì che quegli urti si ripetessero, e anche con particolare frequenza, per i motivi più inaspettati e inconsistenti. Questi urti provenivano spesso anche dal fatto che essi non sapevano ancora che cosa fosse importante per loro e perché in tutto quel primo tempo fossero spesso di cattivo umore. Quando l'uno era di buon umore e l'altro di cattivo, la pace non veniva turbata ma quando tutti e due erano di cattivo umore, allora gli urti venivano fuori da così incomprensibili cause e così inconsistenti, che dopo non riuscivano in nessun modo a ricordare per quale motivo avessero litigato. È vero che, quando erano tutti e due di buon umore, la gioia della loro vita si raddoppiava; tuttavia questo primo periodo fu difficile per loro. 

Per tutto il primo periodo si sentì, in modo particolarmente vivo, una certa tensione, come il tendersi da una parte e dall'altra di quella catena che li aveva avvinti. In generale, quel mese di luna di miele, il primo mese di matrimonio, dal quale, secondo la tradizione, Levin si aspettava tanto, non solo fu senza miele, ma rimase nel ricordo di entrambi come il più difficile e umiliante periodo della loro vita. Tutti e due, in seguito, cercarono di cancellare dalla memoria le circostanze assurde e umilianti di quel periodo insano, in cui di rado erano stati tutti e due di umore normale, in cui di rado erano stati loro stessi. 

Soltanto nel terzo mese di matrimonio, dopo il ritorno da Mosca, dove avevano passato un mese, la loro vita divenne più piana. 

\capitolo{XV}\label{xv-4} 

Erano da poco arrivati da Mosca, ed erano contenti della loro solitudine. Lui nello studio, accanto al tavolo, scriveva. Lei, in quel vestito lilla scuro che aveva portato nei primi giorni del matrimonio e ora aveva indosso di nuovo perché così particolarmente caro e impresso nella memoria di lui, sedeva sul divano, su quello stesso vecchio divano di pelle, che era sempre stato nello studio, al tempo del nonno e del padre di Levin; ricamava una broderie anglaise. Egli pensava e scriveva, senza cessare di sentire con gioia la presenza di lei. Le sue occupazioni, che riguardavano l'azienda domestica e il libro, nel quale dovevano essere esposte le basi di una nuova economia, non erano state da lui abbandonate; ma come, tempo addietro, queste occupazioni e questi pensieri gli erano sembrati piccoli e insignificanti rispetto alle tenebre che ricoprivano tutta la sua vita, così ora, proprio allo stesso modo, piccoli e insignificanti gli sembravano rispetto alla vita futura inondata di una luce chiara di felicità. Continuava le sue occupazioni, ma sentiva che il centro di gravità della propria attenzione si era spostato su di un'altra cosa e che, in seguito a ciò, egli considerava il lavoro in maniera del tutto diversa e con maggiore chiarezza. Prima, per lui, quest'attività era un mezzo per salvarsi dalla vita. Prima sentiva che, senza quest'attività, la sua vita sarebbe stata troppo buia. Ora, invece, queste occupazioni gli erano indispensabili perché la vita non fosse troppo uniformemente luminosa. Avendo ripreso in mano le proprie carte, rileggendo quello che aveva scritto, constatò con soddisfazione che valeva la pena di occuparsene. Il lavoro era nuovo e utile. Molte idee di un tempo gli parvero superflue ed estremiste, ma molti problemi gli si chiarirono quando ravvivò nella memoria tutta l'opera. In questo momento scriveva un capitolo sulle cause della situazione svantaggiosa dell'agricoltura in Russia. Dimostrava che la povertà della Russia derivava non solo dalla ingiusta distribuzione della proprietà terriera e da una falsa tendenza, ma a ciò avevano cooperato negli ultimi tempi la civilizzazione straniera, introdotta in Russia anormalmente, e soprattutto le vie di comunicazione, le ferrovie, che avevano portato all'accentramento nelle città, al diffondersi del lusso e in seguito a questo, a tutto danno dell'agricoltura, allo sviluppo dell'industria manifatturiera, del credito e del suo satellite, il giuoco di borsa. Gli sembrava che, con uno sviluppo normale della ricchezza dello stato, tutti questi fenomeni si sarebbero fatti avanti solo quando nell'agricoltura si fosse già impiegato un lavoro considerevole, quando questa si fosse posta in condizioni regolari o almeno definite; che la ricchezza del paese dovesse crescere in modo uniforme e tale che gli altri settori della ricchezza non superassero l'agricoltura; che in conformità di una data situazione agricola dovessero essere anche le vie di comunicazione ad essa corrispondenti, e che, con la errata utilizzazione della terra, le ferrovie, volute non da una necessità economica, ma politica, fossero premature e, invece di incrementare l'agricoltura come ci si aspettava, l'avessero arrestata, superando l'agricoltura stessa e incrementando l'industria e il credito; che, perciò, come in un animale lo sviluppo unilaterale e prematuro d'un solo organo ostacolerebbe lo sviluppo generale, così, per lo sviluppo generale della ricchezza in Russia, il credito, le vie di comunicazione, lo sviluppo dell'attività manifatturiera, indubbiamente indispensabili in Europa dove erano state tempestive, avrebbero prodotto solamente danno, allontanando la questione principale, urgente, dell'organizzazione agricola. 

Mentre egli scriveva, ella pensava come suo marito fosse stato poco spontaneamente premuroso verso il giovane principe carskij, il quale, alla vigilia della partenza, con molto poco tatto, era stato galante con lei. ``Perché è geloso - ella pensava. - Dio mio! com'è simpatico e sciocco! È geloso di me! Se sapesse che tutti loro sono per me come Pëtr il cuoco! - ella pensava, guardando con un senso, per lei strano, di proprietà la nuca e il collo rosso di lui. - È peccato staccarlo dalle sue occupazioni, ma ne avrà del tempo, bisogna guardargli il viso; sentirà che lo guardo? Voglio che si volti\ldots{} lo voglio!'' ed ella aprì ancor più gli occhi, desiderando così rafforzare l'effetto del proprio sguardo. 

- Sì, attirano a sé tutte le linfe e dànno un falso splendore - mormorò, fermandosi nello scrivere, e, sentendo ch'ella lo guardava e sorrideva, si voltò. 

- Che c'è? - domandò, sorridendo e alzandosi. 

``S'è voltato'' ella pensò. 

- Niente, volevo che ti voltassi - disse lei, guardandolo e desiderando di indovinare se era irritato o no d'essere stato distratto. 

- Be', come stiamo bene noi due! Io, davvero - disse, accostandosi a lei e splendendo d'un sorriso di felicità. 

- Sto tanto bene, non andrò in nessun posto, specialmente a Mosca no. 

- E a che cosa pensavi? 

- Io? pensavo\ldots{} No, no, va', scrivi, non ti distrarre - disse lei increspando le labbra - anch'io adesso devo tagliare questi buchini, non vedi? 

Prese le forbici e cominciò a tagliare. 

- No, di', allora cosa? - disse, sedendosi accanto a lei e seguendo il movimento circolare delle forbicine. 

- A che cosa pensavo? Pensavo a Mosca, alla tua nuca. 

- Perché proprio a me una simile felicità? Non è naturale. È troppo bello - disse lui, baciandole la mano. 

- Per me, invece, meglio si sta, e più è naturale. 

- E tu hai una trecciolina - disse lui, volgendole il capo con delicatezza. - Una trecciolina\ldots{} Vedi, ecco qua. Sì, sì, noi lavoriamo proprio. 

Il lavoro non continuò, ed essi si scostarono, d'un tratto, come colpevoli, quando Kuz'ma entrò ad annunciare che il tè era pronto. 

- E dalla città sono venuti? - chiese Levin a Kuz'ma. 

- Sono venuti or ora; stanno dividendo le lettere. 

- Vieni presto, allora - ella gli disse, andando via dallo studio - altrimenti leggerò senza di te. E soniamo a quattro mani. 

Rimasto solo e riuniti i suoi quaderni in una cartella nuova, comperata da lei, cominciò a lavarsi le mani in un lavabo nuovo, elegante, con tutto l'occorrente, anche questo apparso con lei. Levin sorrideva ai propri pensieri e scoteva il capo con disapprovazione; un senso simile al rimorso lo tormentava. Qualcosa di vergognoso, di molle, di capuano, com'egli lo definiva, era nella sua vita di ora. ``Vivere così non è bene - pensava. - Ecco, tra poco sono tre mesi, e io non faccio quasi nulla. Quest'oggi, forse per la prima volta, mi son messo al lavoro, ebbene? Ho appena cominciato che ho lasciato andare. Perfino le mie occupazioni solite, anche quelle, ho lasciato andare. Non vado quasi neppure più, né a piedi né a cavallo, in giro per l'azienda. Ora mi fa pena lasciarla, ora vedo che lei si annoia. E io invece, prima del matrimonio, pensavo che la mia vita fosse mediocre, che non valesse e che solo dopo il matrimonio sarebbe cominciata la vera vita. Ma ecco che son quasi tre mesi, e io non ho mai passato il tempo così oziosamente e inutilmente. No, così non può andare, bisogna cominciare. S'intende, la colpa non è sua. A lei non si può rimproverare nulla. Devo essere io più deciso, devo difendere la mia indipendenza d'uomo. Altrimenti potrei io stesso abituarmi e fare abituare lei\ldots{} S'intende, la colpa non è sua'' egli si diceva. 

Ma era difficile per un uomo scontento non dar la colpa ad altri, e proprio alla persona che più di tutti gli era vicina, per quello di cui era scontento. E a Levin veniva confusamente in testa non che ella fosse colpevole (lei non poteva essere colpevole di nulla), ma che colpevole fosse la sua educazione troppo superficiale e frivola (``quello sciocco di carskij; lei, lo so, voleva, ma non sapeva fermarlo''). ``Sì, oltre l'interesse per la casa (questo ce l'ha), oltre il proprio abbigliamento e la broderie anglaise, non ha altri interessi seri. Né interesse per il mio lavoro, né per l'azienda, né per i contadini, né per la musica, in cui è pur esperta, né per la lettura. Non fa nulla ed è pienamente soddisfatta''. Levin fra di sé biasimava ciò, e non capiva ancora che ella si preparava a quel periodo di attività che stava per giungere per lei, quando sarebbe stata nello stesso tempo la moglie, la padrona di casa e avrebbe portati in sé, avrebbe allevato e educato i propri figli. Non capiva ch'ella lo sapeva per istinto e che, preparandosi a questo lavoro pieno di ansie, non si rimproverava i momenti di spensieratezza e di felicità d'amore che aveva adesso, intessendo lieta il proprio nido di domani. 

\capitolo{XVI}\label{xvi-4} 

Quando Levin andò di sopra, sua moglie era seduta vicino a un nuovo samovar d'argento, davanti a un nuovo servizio da tè e, posta a sedere accanto a un tavolino la vecchia Agaf'ja Michajlovna con la tazza di tè versatole, leggeva una lettera di Dolly, con la quale era in continua e frequente corrispondenza. 

- Ecco, la vostra signora m'ha messa a sedere, mi ha ordinato di sedere con lei - disse Agaf'ja Michajlovna, sorridendo benevola verso Kitty. 

In queste parole di Agaf'ja Michajlovna, Levin intuì lo scioglimento di un dramma accaduto in quegli ultimi tempi fra Agaf'ja Michajlovna e Kitty. Egli vedeva che, malgrado tutta l'amarezza provocata ad Agaf'ja Michajlovna dalla nuova padrona, che le aveva tolto le redini della direzione, Kitty l'aveva tuttavia conquistata e l'aveva costretta a volerle bene. 

- Ecco, ho letto anche una tua lettera - disse Kitty, porgendogli una lettera sgrammaticata. - È di quella donna, mi pare, di tuo fratello\ldots{} - ella disse. - Non ho finito di leggere. E questa è dai miei e da Dolly. Figurati! Dolly ha portato Griša e Tanja dai Sarmatskij a un ballo di bambini; Tanja era vestita da marchesa. 

Ma Levin non l'ascoltava; fattosi rosso, aveva preso la lettera di Mar'ja Nikolaevna, l'amante di un tempo di Nikolaj e aveva cominciato a leggerla. Era già la seconda lettera di Mar'ja Nikolaevna. Nella prima scriveva che il fratello l'aveva scacciata senza sua colpa, e con commovente ingenuità aggiungeva che, sebbene ella fosse di nuovo nella miseria, non chiedeva nulla, non desiderava nulla, ma che il pensiero che Nikolaj Dmitrievic si rovinasse senza di lei, la struggeva, e chiedeva al fratello di sorvegliarlo. Adesso scriveva un'altra cosa. Aveva ritrovato Nikolaj Dmitrievic, si era di nuovo unita a lui a Mosca ed era andata con lui in una città capoluogo di governatorato, dove egli aveva ricevuto un posto al servizio dello stato. Ma là egli aveva litigato con il capo ed era tornato a Mosca, e in viaggio si era tanto ammalato che difficilmente si sarebbe riavuto, ella scriveva. ``Ha sempre ricordato voi e anche di denari non ce n'è più''. 

- Leggi, Dolly scrive di te - stava per cominciare Kitty sorridendo, ma si fermò a un tratto, notando l'espressione mutata del viso del marito. 

- Che hai? Cos'è successo? 

- Mi scrive che Nikolaj sta per morire. Io vado. 

Il viso di Kitty si mutò a un tratto. I suoi pensieri su Tanja che faceva da marchesa, su Dolly, tutto scomparve. 

- Quando parti? 

- Domani. 

- E io vengo con te, posso? - ella disse. 

- Kitty, via, cos'è questo? - disse lui con rimprovero. 

- Come cosa? - domandò Kitty, offesa dal fatto ch'egli accogliesse controvoglia e con dispetto la sua proposta. - Perché non posso venire? Non ti darò fastidio. Io\ldots{} 

- Io vado perché mio fratello muore - disse Levin. - Perché tu\ldots{} 

- Perché io? per lo stesso motivo che ci vai tu. 

``E in un momento così grave per me, pensa solo che si annoierà a star sola'' pensò Levin. E questo pretesto, in un fatto così grave, lo irritò. 

- Non è possibile - disse severo. 

Agaf'ja Michajlovna, vedendo che la cosa andava a finire in un litigio, posò silenziosa la tazza e uscì. Kitty non l'aveva neppur notata. Il tono con cui suo marito aveva detto le ultime parole, la offese in modo particolare perché, evidentemente, egli non credeva a quello ch'ella diceva. 

- E io ti dico che se vai tu, verrò con te, verrò assolutamente - cominciò a dire in fretta e con rabbia. - Perché non è possibile? Perché dici che non è possibile? 

- Perché andare Dio sa dove, chi sa per quali strade, in quali alberghi. Tu mi sarai d'intralcio - rispose Levin, cercando di mantenere il suo sangue freddo. 

- Niente affatto. Io non ho bisogno di nulla. Dove puoi star tu, là anch'io\ldots{} 

- Ma via, non fosse altro che per il fatto che là c'è quella donna di cui tu non puoi fare la conoscenza. 

- Io non so nulla e non voglio saper nulla, chi ci sia e che cosa ci sia. Io so che il fratello di mio marito sta per morire e mio marito va da lui e io vado con mio marito per\ldots{} 

- Kitty, non ti arrabbiare. Ma pensa, questa è una cosa grave, mi spiace pensare che tu vi mescoli una sensazione di debolezza, il disappunto di rimanere sola. Via, se proprio ti è così noioso restar sola, vieni allora a Mosca. 

- Ecco, tu sempre mi attribuisci pensieri cattivi e volgari - cominciò a dire lei con lacrime d'offesa e di rabbia. - Nient'affatto, io non per debolezza, niente affatto\ldots{} Sento che il mio dovere è di stare con mio marito, quando egli soffre, ma tu vuoi farmi del male apposta, vuoi non capire apposta\ldots{} 

- No, questo è terribile. Questo significa essere schiavo! - proruppe Levin alzandosi e senza aver più la forza di contenere la propria stizza. Ma proprio in questo momento sentì che colpiva se stesso. 

- Allora perché ti sei sposato? Saresti libero. Perché, se te ne penti? - ella cominciò a dire, si alzò di scatto e corse in salotto. 

Quando egli andò a cercarla, piangeva singhiozzando. 

Egli cominciò a parlare, cercando quelle parole che potessero non dissuaderla, ma calmarla. Ma lei non l'ascoltava e non consentiva in nulla. Egli si chinò verso di lei e le prese la mano che gli faceva resistenza. Le baciò la mano, le baciò i capelli, le baciò ancora la mano; lei taceva sempre. Ma quando egli le prese il viso fra tutte e due le mani e disse: ``Kitty!'' ella a un tratto tornò in sé, pianse ancora e poi fece la pace. 

Fu deciso di partire l'indomani insieme. 

Levin disse alla moglie ch'egli aveva creduto al suo desiderio di andare solo per rendersi utile, acconsentì con lei che la presenza di Mar'ja Nikolaevna accanto al fratello non presentava nulla di sconveniente; ma in fondo all'animo partiva scontento di lei e di se stesso. Era scontento di lei perché non gli aveva permesso di allontanarsi quando era necessario (e come era strano che lui, che, fino a poco tempo addietro, non aveva coraggio di credere ch'ella lo amasse, ora si sentisse infelice perché lo amava troppo!), ed era scontento di sé perché non aveva mostrato carattere. Ancor meno era d'accordo, in fondo all'animo, che a lei non dovesse interessare la donna che era col fratello, e con terrore pensava a tutti gli urti che ne sarebbero potuti derivare. Già il solo fatto che sua moglie, la sua Kitty, sarebbe stata nella stessa stanza con una donna perduta, lo faceva rabbrividire di ribrezzo e di orrore. 

\capitolo{XVII}\label{xvii-4} 

L'albergo della città di provincia, nel quale giaceva a letto Nikolaj Levin, era uno di quegli alberghi di capoluogo che vengono costruiti secondo modelli perfezionati, con le migliori intenzioni di pulizia, comodità e perfino eleganza, ma che per il pubblico che li frequenta si trasformano con straordinaria celerità in locande sudice con pretese di modernità, e diventano, per queste stesse pretese, ancora peggiori dei vecchi alberghi semplicemente sudici. Questo albergo era già in questo stato; e un soldato con una divisa sporca e una sigaretta in bocca, che doveva far da portiere, e la scala di passaggio in ghisa, tetra e sgradevole, il cameriere disinvolto con un frac unto, e la sala centrale con un mazzo di fiori di cera impolverato che adornava la tavola, il sudiciume, la polvere, il disordine disseminati ovunque e, nello stesso tempo, quel certo nuovo presuntuoso affannarsi, modernamente collegato con il movimento ferroviario, dell'albergo, produssero sui Levin, dopo la loro recente vita di sposi, una penosa sensazione, specialmente perché l'impressione equivoca, prodotta dall'albergo, non si confaceva in nessun modo con quello che li aspettava. 

Come sempre, dopo la domanda sul prezzo della camera desiderata, risultò che nessuna camera buona era libera: una camera buona era occupata da un ispettore delle ferrovie, un'altra da un avvocato di Mosca, una terza dalla principessa Astaf'eva venuta dalla campagna. Ne rimaneva una sporca, accanto alla quale promisero di liberarne un'altra per la sera. Irritato con la moglie perché si avverava quello ch'egli aveva immaginato, che cioè al momento dell'arrivo, mentre egli aveva il cuore in ansia al pensiero del fratello, invece di correre subito da lui, avrebbe dovuto preoccuparsi di lei, Levin introdusse la moglie nella camera loro assegnata. 

- Va', va' - gli disse lei, guardandolo con uno sguardo timido, colpevole. 

Egli uscì in silenzio, e proprio là s'imbatté in Mar'ja Nikolaevna che aveva saputo del suo arrivo e non aveva avuto il coraggio di entrare da lui. Era proprio come l'aveva vista a Mosca: lo stesso vestito di lana, le stesse braccia e il collo scoperti e lo stesso viso butterato, benevolmente ottuso, un po' ingrassato. 

- Ebbene, che c'è? Come sta? che c'è? 

- Molto male. Non sta in piedi. Non faceva che aspettar voi. Lei\ldots{} Voi\ldots{} siete con vostra moglie\ldots{} 

Levin nel primo momento non capì quello che l'intimidiva, ma lei glielo spiegò subito. 

- Io andrò via, andrò in cucina - mormorò. - Sarà contento. Ha sentito, la conosce e la ricorda all'estero. 

Levin capì ch'ella intendeva sua moglie, e non sapeva cosa rispondere. 

- Andiamo, andiamo! - disse. 

Ma s'era appena mosso, che la porta della sua camera si aprì, e Kitty si sporse fuori. Levin arrossì di vergogna e di rabbia contro sua moglie, che metteva se stessa e lui in quella situazione penosa, ma Mar'ja Nikolaevna arrossì ancora di più. S'era tutta rattrappita e s'era fatta rossa fino alle lacrime e, afferrate con tutte e due le mani le punte del fazzoletto, le ravvolgeva con le dita rosse, non sapendo che dire e che fare. 

Nel primo attimo Levin vide un'espressione di avida curiosità nello sguardo con cui Kitty osservava quella donna per lei incomprensibile e paurosa; ma questo durò solo un attimo. 

- E dunque come sta? - disse rivolta al marito e subito dopo a lei. 

- Ma non si può discorrere nel corridoio! - disse Levin, guardando con stizza un signore che, tentennando sulle gambe, attraversava in quel momento il corridoio andandosene per i fatti suoi. 

- Su, allora, entrate - disse Kitty, rivolgendosi a Mar'ja Nikolaevna ch'era tornata in sé; ma, avendo notato il viso spaventato del marito: - Oppure andate, andate, andate e mandatemi a chiamare - disse, e rientrò in camera. Levin andò dal fratello. 

Egli non si aspettava per nulla quello che vide e sentì dal fratello. Si aspettava di trovare quello stato di autoinganno che, aveva sentito dire, i tisici hanno spesso, e che così fortemente lo aveva colpito durante il soggiorno autunnale del fratello. Si aspettava di trovare i segni fisici di una morte prossima più definiti, una maggiore debolezza, una maggiore magrezza, ma sempre quella stessa situazione. Si aspettava di provar lui stesso quel sentimento di pietà per la perdita del fratello prediletto e di orrore dinanzi alla morte che aveva provato allora, ma solo in grado più alto. E si preparava a questo; trovò, invece, una cosa affatto diversa. 

In una camera piccola, sudicia, coperta di sputi sui riquadri dipinti dei muri, di là dalla sottile intelaiatura dove si sentiva parlare, in un'aria impura di un lezzo soffocante, su di un letto scostato dal muro, giaceva un corpo sotto una coperta. Un braccio di questo corpo era al di sopra della coperta e la mano enorme, come un rastrello, di questo braccio era incomprensibilmente attaccata a un fuso sottile ed eguale dall'estremità al centro. La testa era adagiata di lato su di un guanciale. Levin poteva vedere i capelli sudati, radi sulle tempie, e la fronte tesa, quasi trasparente. 

``Non può essere che questo corpo orribile sia di mio fratello Nikolaj'' pensò Levin. Ma quando egli si fece dappresso e vide il viso, il dubbio non fu più possibile. Malgrado il pauroso mutamento del viso, a Levin bastò guardare quegli occhi vividi che si erano levati su di lui che entrava, bastò notare il leggero movimento della bocca sotto i baffi sottili, per capire la verità paurosa, che quel corpo morto era suo fratello vivo. 

Gli occhi scintillanti guardavano severi e accusatori il fratello che entrava. E subito, con questo sguardo, si stabilì un rapporto vivo tra vivi. Levin sentì una riprovazione nello sguardo fisso su di lui e provò rimorso per la propria felicità. 

Quando Konstantin lo prese per una mano, Nikolaj sorrise. Il sorriso era debole, appena percettibile, e nonostante il sorriso, l'espressione severa degli occhi non mutò. 

- Non ti aspettavi di trovarmi così - pronunciò a stento. 

- Sì\ldots{} no - diceva Levin confondendosi nelle parole\ldots{} - Come mai non mi hai fatto sapere prima, cioè anche durante il periodo del mio matrimonio? Ho chiesto notizie dappertutto. 

Bisognava parlare per non tacere, ma egli non sapeva che cosa dire, tanto più che il fratello non rispondeva e guardava soltanto, senza abbassare lo sguardo, penetrando, evidentemente, il senso di ogni parola. Levin comunicò al fratello che sua moglie era venuta con lui. Nikolaj mostrò piacere, ma disse che temeva di spaventarla col suo stato. Seguì un silenzio. Improvvisamente Nikolaj si mosse e cominciò a dire qualcosa. Levin si aspettava qualcosa di particolarmente importante e significativo dall'espressione del viso, ma Nikolaj cominciò a parlare della sua salute. Incolpava il dottore, rimpiangeva che non ci fosse il medico famoso di Mosca, e Levin capì ch'egli sperava ancora. 

Al primo momento di silenzio Levin si alzò, desiderando liberarsi, sia pure per un attimo, da quella sensazione tormentosa, e disse che andava a chiamare sua moglie. 

- Sì, sì, va bene, e io intanto dirò di pulire un poco qua. È tutto sporco, e ci deve puzzare, credo. Maša! Metti in ordine - disse a stento il malato. - E quando avrai messo in ordine, vattene - aggiunse, guardando interrogativamente il fratello. 

Levin non rispose nulla. Uscito nel corridoio, si fermò. Aveva detto che avrebbe condotto la moglie, ma ora, rendendosi conto del sentimento che provava, decise che, al contrario, avrebbe cercato di persuaderla a non andare dal malato. ``Perché deve tormentarsi come me?'' pensò. 

- Ebbene? che c'è? come? - chiese Kitty con viso spaventato. 

- Ah, è orribile, orribile! Perché sei venuta? - disse Levin. 

Kitty tacque per qualche secondo, guardando timida e implorante il marito; poi si avvicinò e con tutte e due le mani si afferrò al suo gomito. 

- Kostja! portami da lui, staremo meglio tutti e due. Purché tu mi porti, portami per favore, e poi te ne vai - ella cominciò a dire. - Devi capire che veder te e non veder lui mi è molto più penoso. Là forse io posso essere utile a te e a lui. Ti prego, promettimi! - ella supplicava il marito come se la felicità della sua vita dipendesse da questo. 

Levin dovette acconsentire e, riavutosi, dimentico del tutto di Mar'ja Nikolaevna, andò di nuovo dal fratello con Kitty. 

Con passo leggero, guardando sempre il marito e mostrandogli un viso coraggioso e compassionevole, ella entrò nella stanza del malato e, voltatasi con calma, chiuse l'uscio senza far rumore. A passi leggeri si avvicinò svelta al lettuccio del malato e, accostandosi in modo che egli non avesse da voltare il capo, prese subito nella sua mano fresca, giovane, lo scheletro enorme della mano di lui, la strinse e, con quella sommessa animazione compassionevole, ma non offensiva, propria solo delle donne, cominciò a parlare con lui. 

- Ci siamo incontrati, ma non ci conoscevamo, a Soden - ella disse. - Voi non pensavate che sarei diventata vostra sorella. 

- Non mi avreste riconosciuto? - egli disse con un sorriso che si era illuminato quando ella era entrata. 

- No, vi avrei riconosciuto. Come avete fatto bene a farcelo sapere! Non c'era giorno che Kostja non si ricordasse di voi e non ne fosse inquieto. 

Ma l'animazione del malato non durò a lungo. Ella non aveva finito di parlare che sul viso di lui si formò di nuovo l'espressione severa di rimprovero e di invidia di colui che muore verso chi vive 

- Temo che qui non stiate del tutto bene - ella disse, sottraendosi al suo sguardo fisso ed esaminando la stanza. - Occorre chiedere un'altra stanza al padrone - ella disse al marito - anche per essere più vicini. 

\capitolo{XVIII}\label{xviii-4} 

Levin non poteva guardare tranquillamente il fratello, non poteva essere naturale e calmo in sua presenza. Quando entrava dal malato i suoi occhi e la sua attenzione si velavano incoscientemente, ed egli non vedeva e non distingueva i particolari dello stato del fratello. Sentiva un lezzo tremendo, vedeva la sporcizia, il disordine, la situazione penosa, udiva i lamenti, e aveva la sensazione che non si potesse porre rimedio a questo. Non gli veniva neppure in mente che, esaminando i particolari della condizione del malato, pensando come giacesse, là sotto la coperta, quel corpo, come si disponessero, contraendosi, quelle gambe smagrite, i femori, la schiena, non fosse possibile disporre meglio, fare qualcosa che, sia pure non meglio, almeno fosse meno peggio. Il gelo gli penetrava nella schiena, quando cominciava a pensare a questi particolari. Era assolutamente convinto che non si potesse far nulla, né per prolungargli la vita, né per alleviargli le sofferenze. Ma la consapevolezza del fatto che egli riconosceva impossibile qualsiasi rimedio, era sentita dal malato e lo irritava. E perciò Levin si sentiva ancora più tormentato. Stare nella camera del malato gli era penoso, non starci, più penoso ancora. E continuamente, con vari pretesti, ne usciva e di nuovo vi entrava, senza aver la forza di rimaner solo. 

Kitty, invece, pensava, sentiva e agiva in modo del tutto diverso. Alla vista del malato ne aveva provato pietà. E la pietà, nell'animo suo di donna, aveva prodotto, invece della sensazione di orrore e di disgusto che aveva prodotto nel marito, la necessità di agire, di rendersi conto di tutti i particolari dello stato del malato e di aiutarlo. E poiché in lei non esisteva il più piccolo dubbio ch'ella dovesse portargli aiuto, non dubitava neppure del fatto che ciò fosse possibile, e si era messa subito all'opera. Quegli stessi particolari, il cui pensiero aveva prodotto orrore nel marito, richiamarono subito la sua attenzione. Mandò a chiamare un medico, mandò in farmacia, fece spazzare, spolverare, lavare dalla donna arrivata con lei e da Mar'ja Nikolaevna; lei stessa lavò, bagnò, pose qualcosa sotto la coperta. Per ordine suo portarono dentro e tolsero via qualcosa dalla camera del malato. Lei stessa andò varie volte nella propria camera senza far caso a chi incontrava; tirò fuori e portò lenzuola, federe, asciugamani, camicie. 

Il cameriere, che nella sala comune serviva il pranzo ad alcuni ingegneri, era venuto varie volte al suo richiamo con il viso irritato, ma non aveva potuto non eseguire gli ordini, poiché ella li dava in maniera così insistentemente dolce, che non ci si poteva in nessun modo sottrarre. Levin non approvava tutto questo; non credeva che ne venisse fuori qualcosa di utile per l'ammalato. Più di tutto, poi, temeva che l'ammalato si irritasse. Ma l'ammalato, pur mostrandosi indifferente, non si irritava, ma si vergognava soltanto, e in genere sembrava interessarsi a quello che ella gli andava facendo. Tornato dal dottore, dal quale Kitty l'aveva mandato, Levin, aperta la porta, trovò il malato nel momento in cui gli cambiavano la biancheria per ordine di Kitty. Lo scheletro lungo e bianco della schiena dalle scapole enormi, sporgenti e dalle costole e le vertebre in fuori, era scoperto, e Mar'ja Nikolaevna e il cameriere s'erano confusi in una manica della camicia e non riuscivano a dirigervi il braccio lungo, penzoloni. Kitty, che aveva chiuso in fretta la porta dietro a Levin, non guardava da quella parte; ma il malato si lamentò ed ella si diresse in fretta verso di lui. 

- Presto, via - ella disse. 

- Ma non venite - pronunciò il malato con rabbia - faccio da me\ldots{} 

- Che avete detto? - chiese Mar'ja Nikolaevna. 

Ma Kitty sentì e capì ch'egli si vergognava e che gli spiaceva di mostrarsi nudo davanti a lei. 

- Io non guardo, non guardo! - disse lei, dirigendo il braccio. - Mar'ja Nikolaevna, andate dall'altra parte, mettete a posto - soggiunse. 

- Dammi, per favore, la boccetta che ho nel sacchetto piccolo - ella disse rivolta al marito - sai, nel taschino di lato; portamela, per favore, e intanto qui metteranno in ordine. 

Tornando con la boccetta, Levin trovò il malato già disteso nel letto e tutto intorno completamente diverso. Il lezzo greve si era cambiato in un profumo d'aceto che Kitty, sporgendo le labbra e gonfiando le guance arrossate, andava spruzzando da un tubicino. Polvere non se ne vedeva in nessun posto, ai piedi del letto c'era un tappeto. Sulla tavola erano disposte le boccette, una caraffa, ed erano piegate la biancheria necessaria e il lavoro di Kitty di broderie anglaise. Su di un altro tavolo, accanto al letto del malato, c'erano una bevanda, una candela e delle polverine. Lo stesso infermo, lavato e pettinato, giaceva fra le lenzuola pulite, sui guanciali sollevati, con una camicia pulita dal colletto bianco intorno al collo estremamente esile e, con una nuova espressione di speranza, guardava Kitty senza staccarne gli occhi. Il dottore, trovato al club e condotto da Levin, non era quello che curava Nikolaj Levin e del quale egli era scontento. Il nuovo dottore tirò fuori l'astuccio e ascoltò l'ammalato, scosse il capo, prescrisse una medicina e spiegò con particolare minuzia come prendere la medicina, e quale dieta osservare. Egli consigliava uova crude o appena cotte e acqua di selz con latte fresco a una certa temperatura. Quando il dottore se ne andò, il malato disse qualcosa al fratello, ma Levin sentì solo le ultime parole: ``la tua Katja'' e, dallo sguardo con cui egli la guardò, Levin capì che ne faceva le lodi. Egli volle vicino a sé anche Katja, come la chiamava lui. 

- Sto già molto meglio - disse. - Ecco, con voi sarei guarito da tempo. Come sto bene! - Le prese la mano e l'accostò alle labbra, ma, quasi temendo che questo potesse spiacerle, cambiò idea, la lasciò andare e l'accarezzò soltanto. Kitty prese quella mano con tutte e due le sue e la strinse. 

- Adesso mettetemi sul lato sinistro e andate a dormire - egli pronunciò. 

Nessuno capì quello ch'egli aveva detto, Kitty sola capì. Ella capiva perché non desisteva dal seguire col pensiero quello che gli era necessario. 

- Dall'altro lato - ella disse al marito; - dorme sempre da quella parte. Fagli cambiar posizione, non sta bene chiamare la servitù. Io non posso. E voi non potete? - si rivolse a Mar'ja Nikolaevna. 

- Io ho paura - rispose Mar'ja Nikolaevna. 

Per quanto terribile fosse per Levin circondare con le braccia quel corpo terrificante, afferrare sotto la coperta quelle membra che voleva ignorare, tuttavia, sottomettendosi all'influenza della moglie, Levin fece il viso risoluto che sua moglie conosceva e, ficcate le mani, lo afferrò; tuttavia, malgrado la sua forza, fu sorpreso dallo strano peso di quelle membra sfinite. Mentre lo voltava e sentiva il proprio collo stretto dal braccio enorme, smagrito, Kitty, in fretta, senza far rumore, capovolse il guanciale, lo sprimacciò e vi accomodò la testa del malato e i suoi capelli radi, di nuovo appiccicati alle tempie. 

Il malato trattenne nella propria mano la mano del fratello. Levin sentiva ch'egli voleva fare qualcosa con la sua mano e che la tirava chi sa dove. Levin lasciava fare, sentendosi una stretta al cuore. Sì, egli la tirò verso la propria bocca e la baciò. Levin sussultò per i singhiozzi e, senza aver la forza di dir nulla, uscì dalla camera. 

\capitolo{XIX}\label{xix-4} 

``L'ha nascosto ai saggi e l'ha rivelato ai fanciulli e ai semplici'' così pensava Levin di sua moglie, discorrendo con lei quella sera. Levin pensava al detto del Vangelo non perché si considerasse un saggio. Non si considerava un saggio, ma non poteva non sapere d'essere più intelligente di Agaf'ja Michajlovna e non poteva non sapere che, pensando alla morte, ci pensava con tutte le forze dell'anima. Sapeva pure che molte grandi intelligenze maschili, di cui aveva letto le considerazioni sulla morte, ci pensavano, ma non sapevano neppure la centesima parte di quello che sapevano sua moglie e Agaf'ja Michajlovna. Per quanto diverse fossero queste due donne, Agaf'ja Michajlovna e Katja, come la chiamava suo fratello Nikolaj, e come adesso era particolarmente caro per Levin chiamarla, in questo erano perfettamente simili. Tutte e due sapevano con certezza che cosa fosse la vita e cosa fosse la morte, e pur senza saper rispondere e neppure capire le questioni che si presentavano a Levin, tutte e due non avevano dubbi sull'importanza di questo fenomeno e lo consideravano proprio allo stesso modo, d'accordo non solo fra di loro, ma dividendo questa loro concezione con milioni di uomini. La prova che esse sapessero con certezza cosa fosse la morte consisteva nel fatto che, senza un attimo di esitazione, sapevano come regolarsi con i moribondi, e non ne avevano paura. Levin e gli altri, invece, pur dissertando a lungo sulla morte, evidentemente non sapevano che cosa fare quando la gente muore. Se Levin in quel momento fosse stato solo con il fratello Nikolaj, l'avrebbe guardato con orrore, e con orrore ancora più grande avrebbe atteso, e nulla di più avrebbe saputo fare. 

Ed era ancora poco: egli non sapeva che cosa dire, come guardare, come camminare. Parlare di cose estranee gli sembrava offensivo, impossibile; parlare della morte, di cose tetre, non si poteva. Tacere neppure si poteva. ``Se lo guardo, penserà che lo osservo, che ho paura; se non lo guardo crederà che penso ad altro. Se cammino in punta di piedi, sarà scontento; ma poggiare tutto il piede, c'è da vergognarsi''. Kitty, invece, si vedeva, non pensava e non aveva il tempo di pensare a sé; pensava a lui, perché sapeva quella tale cosa, e tutto andava bene. Raccontava anche qualcosa di sé e del suo matrimonio, e sorrideva e lo compativa e lo carezzava, e parlava di casi di guarigione, e tutto andava bene; dunque ella sapeva. La prova che l'attività sua, come quella di Agaf'ja Michajlovna, non fosse istintiva, animale, irrazionale, consisteva nel fatto che oltre la cura fisica, oltre l'alleviamento delle sofferenze, sia Agaf'ja Michajlovna che Kitty esigevano per il moribondo qualcosa di ancor più importante della cura fisica, qualcosa che non aveva nulla di comune con le condizioni fisiche. Agaf'ja Michajlovna, parlando del vecchio che era morto, aveva detto: ``Ebbene, sia lodato Iddio! l'hanno comunicato, gli hanno dato l'estrema unzione, conceda Iddio a ognuno di morire così''. Katja, proprio alla stessa maniera, oltre tutte le preoccupazioni per la biancheria, per le piaghe, per le bevande, aveva fin dal primo giorno convinto il malato della necessità di comunicarsi e di ricevere l'estrema unzione. 

Rientrato in camera sua, al numero due, Levin sedette, col capo chino, non sapendo che fare. Senza parlare di cena, di sonno, senza riflettere a quello che avrebbe fatto, egli non poteva neanche parlare con sua moglie: si vergognava. Kitty, al contrario, era più attiva del solito. Era perfino più animata del solito. Ordinò di portare la cena, disfece lei stessa le valigie, lei stessa aiutò a fare i letti e non dimenticò di cospargerli di polvere persica. C'erano in lei l'eccitamento e l'intuito che appaiono negli uomini prima di un combattimento, di una lotta, nei momenti decisivi e pericolosi della vita, nei momenti in cui l'uomo mostra una volta per sempre il proprio valore e in cui tutto il suo passato non sembra vano, ma come una preparazione a questi momenti. 

Tutto il lavoro le riusciva, e non erano ancora le dodici, che già tutte le cose erano assestate con pulizia, con cura, come se proprio la camera d'albergo fosse simile alla casa, alla propria camera: i letti fatti, le spazzole, i pettini, gli specchietti messi fuori, i tovagliolini distesi. 

Levin trovava imperdonabile in questo momento il fatto di mangiare, dormire, di parlare perfino, e sentiva che ogni suo movimento era poco adatto. Lei, invece, metteva in ordine le spazzole e faceva tutto ciò senza che vi fosse nulla di offensivo. 

Però non poterono mangiar nulla e per lungo tempo non poterono prender sonno; anzi per un pezzo non riuscirono neppure a sdraiarsi per dormire. 

- Sono molto contenta di averlo convinto a ricevere domani l'estrema unzione - ella diceva, sedendo in vestaglia dinanzi allo specchio pieghevole e pettinando con un pettine fitto i suoi capelli morbidi, profumati. - Io non ho mai visto questa funzione; ma so, mamma me lo diceva, che ci sono delle preghiere per la guarigione. 

- Possibile che tu pensi ch'egli possa guarire? - disse Levin, guardando la sottile scriminatura che si chiudeva continuamente dietro alla piccola testa rotonda, appena ella faceva passare avanti il pettine. 

- L'ho domandato al dottore: ha detto che non potrà vivere più di tre giorni. Ma loro possono mai sapere ciò? Io intanto sono contenta d'averlo convinto - ella disse, guardando di lato il marito di là dai capelli. - Tutto può essere - ella soggiunse con quella espressione particolare, un po' accorta, che aveva sempre in viso quando parlava di religione. 

Dopo il loro colloquio sulla religione, quando erano ancora fidanzati, né lui né lei ne avevano mai più parlato; ma lei osservava le sue pratiche, frequentava la chiesa e pregava sempre con la tranquilla costante consapevolezza che così bisognasse fare. Malgrado le assicurazioni di lui sul contrario, ella era fermamente convinta che egli fosse cristiano come lei e ancora di più, e che tutto quello ch'egli diceva non fosse che una delle sue risibili uscite maschili, simili a quella che diceva a proposito della broderie anglaise, che, cioè, la povera gente rammendava i buchi e lei, invece, li tagliava a bella posta, e via di seguito. 

- Sì, ecco, quella donna, Mar'ja Nikolaevna, non sapeva organizzare tutto questo - disse Levin. - E\ldots{} devo riconoscere che sono molto, molto contento che tu sia venuta. Tu sei una tale purezza\ldots{} - Egli le prese la mano e non la baciò (baciarle la mano in quella prossimità di morte gli sembrava sconveniente), ma la strinse soltanto, guardando con un'espressione colpevole gli occhi di lei che s'erano illuminati. 

- Ti saresti talmente tormentato da solo - ella disse e, sollevando in alto le mani che coprivano le guance arrossite di soddisfazione, avvolse le trecce sulla nuca e le appuntò con le forcine. - No - ella continuò - lei non sapeva\ldots{} Io, per fortuna, ho imparato molte cose a Soden. 

- Possibile che là ci fossero malati simili? 

- C'era di peggio. 

- Per me è orribile non poterlo ritrovare così come era da giovane\ldots{} non puoi credere che giovane delizioso fosse, ma io allora non lo capivo. 

- Ci credo, ci credo. Come sento che saremmo stati amici, io e lui! - ella disse e, spaventata di quello che aveva detto, guardò il marito, e le lacrime le spuntarono negli occhi. 

- Sì, sareste stati - disse lui triste. - Ecco uno di quegli uomini di cui si dice che non sono fatti per questo mondo. 

- Ma abbiamo molti giorni dinanzi a noi, bisogna coricarsi - disse Kitty, dopo aver guardato il suo minuscolo orologio 

\capitolo{XX}\label{xx-4} 

\emph{LA MORTE} 

Il giorno dopo, il malato ricevette la comunione e l'estrema unzione. Durante il rito Nikolaj Levin pregava con ardore. Nei suoi grandi occhi, fissi sull'icona, posta su di un tavolo da giuoco ricoperto di un tovagliolino colorato, c'erano una preghiera e una speranza così appassionate, che Levin provava raccapriccio a guardare. Levin sapeva che questa preghiera appassionata e questa speranza avrebbero reso solo più penoso per lui il distacco dalla vita che amava tanto. Levin conosceva il fratello e il corso dei suoi pensieri; sapeva che la sua mancanza di fede era sorta non perché gli fosse più facile vivere senza una fede, ma perché di volta in volta le spiegazioni modernamente scientifiche dei fenomeni del mondo avevano soppiantato la fede, e perciò sapeva che questo suo ritorno alla fede non era legittimo, non era compiuto attraverso lo stesso pensiero, ma era soltanto momentaneo, interessato, per una folle speranza di guarigione. Levin sapeva pure che Kitty aveva rafforzato questa speranza con il racconto delle guarigioni da lei sentite. Tutto questo Levin lo sapeva, e gli era tormentoso vedere quello sguardo supplichevole, pieno di speranza, e quella mano scheletrica che si sollevava con stento a segnarsi sulla fronte stirata, sulle spalle sporgenti e sul petto vuoto rantolante, incapace di trattenere in sé quella vita che l'ammalato chiedeva. Durante la funzione Levin pregava e faceva proprio quello che lui, miscredente, aveva fatto mille volte. Diceva, rivolgendosi a Dio: ``Fa', se esisti, fa' che quest'uomo guarisca (questo si è pur verificato molte volte), e Tu salverai lui e me''. 

Dopo l'unzione il malato, a un tratto, migliorò molto. Per un'ora intera non tossì neppure una volta, sorrise, baciò la mano a Kitty, ringraziandola fra le lacrime, e disse di star bene, che non aveva dolore in nessun posto e che aveva appetito e si sentiva in forze. Si sollevò perfino da sé, quando gli portarono la minestra, e chiese anche una costoletta. Per quanto le sue condizioni fossero disperate, per quanto fosse evidente, guardandolo, che non poteva guarire, Levin e Kitty in quell'ora furono nella medesima eccitazione felice, ma timida, per la paura di sbagliare. 

``Meglio?''. ``Sì, molto''. ``Sorprendente''. ``Non c'è nulla di straordinario''. ``Tuttavia sta meglio'' essi dicevano sottovoce, sorridendosi l'un l'altro. 

Questa illusione non durò a lungo. Il malato si addormentò tranquillo, ma, dopo mezz'ora la tosse lo svegliò. E a un tratto scomparvero tutte le speranze e in coloro che lo circondavano e in lui stesso. La realtà della sofferenza distrusse, indubitatamente, le speranze di prima, senza lasciarne neppure più il ricordo, in Levin e in Kitty e nel malato stesso 

Senza ricordare neppure quello cui aveva creduto mezz'ora prima, quasi il ricordo fosse vergognoso, egli pretese che gli dessero dello iodio per inalazioni, in una fiala ricoperta da un pezzetto di carta con dei forellini. Levin gli porse il vasetto, e lo stesso sguardo di speranza appassionata, con cui il malato aveva ricevuto l'estrema unzione, si diresse adesso sul fratello, pretendendo da lui la conferma delle parole del dottore sul fatto che le inalazioni di iodio producono miracoli. 

- Katja non c'è? - rantolò lui, guardandosi in giro, quando Levin gli ebbe ripetuto stentatamente le parole del dottore. - No, allora si può dire\ldots{} Per lei ho fatto questa commedia. È così cara, ma noi due ormai non ci possiamo ingannare. Ecco, a questo io credo - disse e, afferrata la fiala con la mano ossuta, cominciò a respirarvi sopra. 

Dopo le sette di sera, Levin e la moglie prendevano il tè nella loro camera, quando Mar'ja Nikolaevna corse da loro trafelata. Era pallida e le labbra le tremavano. 

- Muore! - mormorò. - Ho paura che muoia subito. 

Tutti e due corsero da lui. Sollevatosi, egli stava seduto sul letto, i gomiti appoggiati, con la lunga schiena ricurva e la testa china. 

- Cosa ti senti? - chiese sottovoce Levin dopo un silenzio. 

- Sento che me ne vado - mormorò Nikolaj, articolando lentamente le parole, con fatica, ma con straordinaria precisione. Non sollevava il capo, ma volgeva in su soltanto gli occhi senza raggiungere con lo sguardo il viso del fratello. - Katja, va' via! - sussurrò ancora. 

Levin saltò su e con un mormorio imperioso la obbligò ad uscire. 

- Me ne vado - egli disse di nuovo. 

- Perché pensi questo? - disse Levin, tanto per dire qualcosa. 

- Perché me ne vado - egli ripeté come se avesse preso ad amare questa espressione. - È la fine. 

Mar'ja Nikolaevna si accostò. 

- Se vi sdraiaste, stareste meglio - ella disse. 

- Presto sarò disteso, quieto, morto - disse ironico, con rabbia. - Su, sdraiatemi, come volete. 

Levin adagiò il fratello sulla schiena, sedette accanto a lui e, senza respirare, gli osservava il viso. Il moribondo giaceva, con gli occhi chiusi, ma sulla fronte, sia pure di rado, si movevano dei muscoli, come in un uomo che pensi profondamente, con tensione. Levin istintivamente pensava con lui quello che in lui si compiva, eppure malgrado lo sforzo del pensiero per procedere insieme, vedeva dall'espressione di quel viso severo e calmo e dal movimento del muscolo al di sopra del sopracciglio, che il moribondo sempre più si chiariva quello che rimaneva sempre egualmente oscuro per Levin. 

- Sì, sì, così - pronunciò a intervalli, lentamente, il moribondo. - Aspettate. - Poi tacque di nuovo. - Così! - strascicò a un tratto tranquillamente, come se tutto fosse risolto per lui. - O Signore! - disse, e sospirò pesantemente. 

Mar'ja Nikolaevna gli tastò i piedi. 

- Si freddano - mormorò. 

A lungo, molto a lungo, come parve a Levin, il malato rimase immobile, disteso. Ma era ancora sempre vivo, e di quando in quando sospirava. Levin era già stanco per la tensione del pensiero. Sentiva, nonostante tutta la tensione del pensiero, che non poteva capire quello che era così. Sentiva di essersi già da tempo distaccato dal morente. Egli non poteva già più pensare alla stessa questione della morte, ma istintivamente gli sorgeva il pensiero di quello che avrebbe dovuto fare subito: chiudergli gli occhi, vestirlo, ordinare la bara. E, cosa strana, si sentiva completamente impassibile, e non sentiva né dolore, né distacco, e tanto meno pietà del fratello. Se adesso aveva un sentimento verso il fratello, era piuttosto un senso di invidia per quella conoscenza ch'egli, morente, aveva e che lui non poteva avere. 

Ancora a lungo rimase seduto così, curvo su di lui, aspettando sempre la fine. Ma la fine non veniva. La porta si aprì e apparve Kitty. Levin si alzò per fermarla. Ma mentre si alzava, sentì un movimento del morente. 

- Non te ne andare - disse Nikolaj e tese una mano. Levin gli diede la sua e fece un gesto irato verso la moglie, perché andasse via. 

Con la mano del morente nella propria, rimase lì seduto per mezz'ora, per un'ora, per un'altra ora. Ormai non pensava già più alla morte. Pensava a cosa stesse facendo Kitty, a chi si trovasse nella camera accanto, se il medico avesse, oppure no, una casa di sua proprietà. Aveva voglia di mangiare e di dormire. Liberò la mano con precauzione e tastò i piedi. I piedi erano freddi, ma il malato respirava. Levin voleva di nuovo uscire in punta di piedi, ma il malato si mosse di nuovo e disse: 

- Non te ne andare. 

\begin{center}\rule{3in}{0.4pt}\end{center} 

Si fece giorno; le condizioni del malato erano sempre le stesse. Levin, liberata pian piano la mano e senza guardare il morente, andò in camera sua e si addormentò. Quando si svegliò, invece della notizia della morte del fratello che si aspettava, seppe che il malato era nella condizione di prima. Aveva ripreso a sedersi, a tossire, aveva ricominciato a mangiare, s'era messo a parlare, e aveva di nuovo smesso di parlare di morte, aveva di nuovo cominciato a esprimere la speranza di guarire, e s'era fatto ancora più irritabile e tetro di prima. Nessuno, né il fratello, né Kitty potevano calmarlo. Si irritava con tutti e diceva a tutti cose spiacevoli, incolpava tutti delle proprie sofferenze e pretendeva che gli portassero il medico famoso da Mosca. A tutte le domande che gli facevano su come si sentiva, rispondeva nello stesso modo, con un'espressione di rancore e di rimprovero. 

- Soffro terribilmente, insopportabilmente! 

Il malato soffriva sempre di più e in modo particolare per le piaghe che ormai non riuscivano più a cicatrizzarsi, e si irritava sempre più contro quelli che gli stavano intorno, soprattutto perché non gli portavano il celebre medico di Mosca. Kitty cercava in tutti i modi di venirgli in aiuto, di calmarlo; ma tutto era inutile, e Levin vedeva che lei stessa era sfinita fisicamente e moralmente, sebbene non volesse riconoscerlo. Quel senso di morte che era stato suscitato in tutti dal suo addio alla vita, nella notte in cui aveva chiamato il fratello, si era disperso. Tutti sapevano che inevitabilmente e presto sarebbe morto, che era già morto a metà. Tutti desideravano soltanto una cosa, ch'egli morisse al più presto possibile, e tutti, nascondendolo, gli davano le medicine dalle fiale, cercavano rimedi e dottori, ingannando lui e loro stessi e ingannandosi l'un l'altro. Tutto questo era una menzogna, una disgustosa, offensiva e sacrilega menzogna. E questa menzogna, e per il carattere che gli era proprio, e perché era lui più di tutti che amava quegli che moriva, Levin la sentiva in modo particolarmente doloroso. 

Levin, che da tempo era preoccupato dal pensiero di rappacificare i fratelli almeno dinanzi alla morte, aveva scritto a Sergej Ivanovic e, ricevutane risposta, lesse la lettera al malato. Sergej Ivanovic scriveva che non poteva venire di persona, ma, con espressioni commoventi, chiedeva perdono al fratello. 

Il malato non disse nulla. 

- E cosa gli devo scrivere? - domandò Levin. - Spero che tu non sia arrabbiato contro di lui. 

- No, per nulla! - rispose Nikolaj stizzito da questa domanda. - Scrivigli che mi mandi il dottore. 

Passarono ancora tre giorni tormentosi; il malato era sempre nelle stesse condizioni. Il senso di desiderio della sua morte lo provavano adesso indistintamente tutti quelli che lo vedevano: i camerieri dell'albergo e il padrone, e i clienti, il medico e Mar'ja Nikolaevna, e Levin e Kitty. Soltanto il malato non esprimeva questo sentimento, al contrario, si irritava perché non gli portavano il dottore, e continuava a prendere la medicina e parlava di vivere. Soltanto in rari momenti, quando l'oppio lo costringeva per un attimo a smemorarsi delle ininterrotte sofferenze, egli diceva a volte, nel dormiveglia, quello che nell'anima sua era più forte che in tutti gli altri: ``ah, e fosse almeno la fine!'' oppure: ``quando finirà?''. 

Le sofferenze, crescendo uniformi, compivano l'opera loro e lo preparavano alla morte. Non c'era posizione in cui non soffrisse, non vi era un attimo in cui egli si assopisse, non vi era membro del corpo che non gli dolesse, che non lo tormentasse. Perfino i ricordi, le impressioni, i pensieri di quel corpo eccitavano ora in lui una repulsione. La vista di altre persone, i loro discorsi, i suoi ricordi personali, tutto questo era per lui solo tormento. Quelli che lo circondavano sentivano ciò e inconsciamente non si permettevano dinanzi a lui né movimenti liberi, né conversazioni, né la manifestazione dei loro desideri. Tutta la sua vita si fondeva in un solo senso di pena e nel desiderio di liberarsene. 

Evidentemente si compiva in lui quel rivolgimento che doveva portarlo a guardare alla morte come alla fine dei suoi desideri, come alla felicità. Prima, ogni singolo desiderio, provocato da una sofferenza o da una privazione, come la fame, la stanchezza, la sete, veniva soddisfatto con una funzione del corpo che dava piacere; ma adesso la privazione e la sofferenza non ricevevano soddisfazioni, anzi il tentativo di soddisfazione provocava una nuova sofferenza. E perciò tutti i desideri si fondevano in un unico desiderio: nel desiderio di liberarsi di tutte le sofferenze e della loro fonte, del corpo. Ma per esprimere questo desiderio di liberazione egli non aveva parole, e perciò non parlava, e per abitudine chiedeva il soddisfacimento di quei desideri che non potevano più essere soddisfatti. ``Mettetemi dall'altro lato'' diceva, e subito dopo voleva che lo mettessero così come prima. ``Datemi del brodo. Portate via il brodo. Raccontate qualcosa, perché state zitti?''. E così, appena cominciavano a parlare, chiudeva gli occhi ed esprimeva stanchezza, indifferenza e disgusto. 

Dieci giorni dopo il suo arrivo nella cittadina, Kitty si ammalò. Le venne mal di capo, vomito e per tutta la mattina non poté alzarsi dal letto. 

Il medico spiegò che il malanno derivava dalla stanchezza, dall'agitazione, e prescrisse serenità di spirito. 

Dopo pranzo però Kitty si alzò e andò come sempre, col lavoro, dal malato. Egli la guardò arcigno quand'ella entrò, e sorrise con sprezzo quando ella disse che era stata male. Quel giorno egli si soffiava il naso di continuo e si lamentava penosamente. 

- Come vi sentite? - ella chiese. 

- Peggio - rispose a stento. - Male! 

- Dove vi duole? 

- Dappertutto. 

- Quest'oggi finirà, guardate - disse Mar'ja Nikolaevna, sia pur sottovoce, ma in modo che il malato, che sentiva benissimo, come aveva notato Levin, doveva averla udita. Levin le fece segno di tacere e si voltò a guardare il malato. Nikolaj aveva sentito; ma queste parole non produssero nessuna impressione su di lui. Il suo sguardo era sempre teso e accusatore. 

- Perché pensate questo? - le domandò Levin quando ella uscì dietro di lui nel corridoio. 

- Ha cominciato a spogliarsi - disse Mar'ja Nikolaevna. 

- Come spogliarsi? 

- Ecco, così - ella disse, tirando le pieghe del suo vestito di lana. In realtà, egli aveva notato che tutto quel giorno il malato aveva afferrato quello che aveva addosso e pareva che volesse strappar via qualcosa. 

La previsione di Mar'ja Nikolaevna era giusta. Verso sera il malato non aveva più la forza di sollevare le braccia e guardava soltanto davanti a sé senza mutare l'espressione dello sguardo, attenta e concentrata. Perfino quando il fratello o Kitty si chinavano su di lui in modo ch'egli potesse vederli, egli guardava in quello stesso modo. Kitty mandò a chiamare il prete, per legger la preghiera degli agonizzanti. 

Mentre il prete leggeva la preghiera, il morente non dava alcun segno di vita; gli occhi erano chiusi. Levin, Kitty e Mar'ja Nikolaevna stavano in piedi accanto al letto. Il prete non aveva ancora finito di leggere, che il morente si stirò, sospirò e aprì gli occhi. Il prete, finita la preghiera, appoggiò sulla fronte fredda la croce, poi la ravvolse lentamente nella stola e, dopo aver sostato ancora due minuti in silenzio, toccò la mano enorme, divenuta fredda ed esangue. 

- È finito - disse il prete e voleva andar via; ma improvvisamente i baffi sottili del morente si mossero, e con chiarezza nel silenzio, emessi dal profondo del petto, si sentirono i suoni netti e precisi: 

- Non del tutto\ldots{} Presto. 

Dopo un istante il viso si rischiarò, sotto i baffi apparve un sorriso, e le donne, raccoltesi, si diedero a vestire il morto, affaccendandosi. 

La vista del fratello e la presenza della morte rinnovarono nell'animo di Levin quel senso di paura dinanzi all'inesplicabile inevitabilità della morte, che lo aveva sconvolto quella sera d'autunno, quando il fratello era giunto da lui. Questo senso, adesso, era ancora più forte; ancora meno di prima egli si sentiva in grado di capire il senso della morte, e ancora più terribile gliene appariva l'inevitabilità; ma ora, grazie alla vicinanza della moglie, questo senso non lo gettava nella disperazione: malgrado la morte, egli sentiva la necessità di vivere e di amare. Sentiva che l'amore lo salvava dalla disperazione e che l'amore, sotto la minaccia della disperazione, diveniva ancora più forte e puro. 

Dinanzi ai suoi occhi si era appena compiuto un mistero di morte, rimasto sempre inesplicabile, che ne sorgeva un altro, altrettanto inesplicabile, che richiamava all'amore e alla vita. 

Il medico confermò le sue supposizioni riguardo a Kitty. Il suo malessere era dovuto alla gravidanza. 

\capitolo{XXI}\label{xxi-4} 

Dal momento in cui Aleksej Aleksandrovic capì dalle spiegazioni di Betsy e di Stepan Arkad'ic che da lui si pretendeva non solo ch'egli lasciasse in pace sua moglie, senza affaticarla con la sua presenza, ma che sua moglie stessa desiderava ciò, egli si sentì così smarrito da non poter decidere nulla da solo, non sapendo lui stesso cosa desiderare in quel momento; così, abbandonandosi nelle mani di coloro che con tanto compiacimento si occupavano delle sue faccende, rispondeva acconsentendo a tutto. Solo quando Anna abbandonò la casa e la signorina inglese mandò a chiedere se dovesse pranzare con lui o a parte, egli capì, per la prima volta, con chiarezza, la propria posizione, e n'ebbe orrore. 

La cosa più difficile, in tutto questo, era ch'egli non riusciva in nessun modo a legare e a fondere il suo passato con quello che era avvenuto ora. Non quel passato in cui egli aveva vissuto felice con la moglie lo sconvolgeva. Il passaggio da quello alla conoscenza dell'infedeltà della moglie egli lo aveva già vissuto dolorosamente; e la condizione gli era stata penosa ma comprensibile. Se la moglie, allora, rivelata la propria infedeltà, fosse fuggita, egli si sarebbe addolorato, sarebbe stato infelice, ma non si sarebbe trovato in quella condizione senza via d'uscita, assurda, in cui si trovava ora. Non poteva in nessun modo legare insieme il suo recente perdono, la sua commozione, il suo amore per la moglie malata e per la creatura dell'altro, con quello che accadeva adesso, con il fatto, cioè, che, come a ricompensa di tutto, egli si trovava solo, ricoperto di vergogna, deriso, non utile a nessuno e disprezzato da tutti. 

I primi due giorni dopo la partenza della moglie, Aleksej Aleksandrovic ricevette i clienti, il capo di gabinetto, andò al comitato e pranzò in sala, come al solito. Senza rendersi conto perché facesse questo, tese tutte le forze dell'animo suo per avere un aspetto calmo e perfino indifferente. Rispondendo alle domande su come disporre le cose e le stanze di Anna Arkad'evna, faceva il più grande sforzo su di sé per avere l'aspetto di un uomo per il quale l'avvenimento accaduto non fosse imprevisto e non avesse in sé nulla che uscisse fuori dalla serie degli avvenimenti soliti, e otteneva lo scopo: nessuno poteva scorgere in lui i segni della disperazione. Ma due giorni dopo la partenza, quando Kornej gli presentò il conto di un magazzino di mode, che Anna aveva dimenticato di pagare, e riferì che il commesso stava lì, Aleksej Aleksandrovic fece chiamare il commesso. 

- Scusate, Eccellenza, se oso disturbarvi. Ma se Sua Eccellenza ordina di rivolgersi alla Signora, voglia compiacersi di comunicarne l'indirizzo. 

Aleksej Aleksandrovic si fece pensieroso, come parve al commesso, e a un tratto, voltatosi, sedette al tavolo. Chinata la testa fra le mani, rimase a lungo in questa posizione, più di una volta provò a parlare e si fermò. 

Kornej, che aveva capito i sentimenti del padrone, pregò il commesso di venire un'altra volta. Rimasto di nuovo solo, Aleksej Aleksandrovic capì di non avere più la forza di sostenere la parte dell'uomo fermo e calmo. Ordinò di staccare i cavalli dalla carrozza che aspettava, di non ricevere più nessuno e non uscì in sala a pranzare. 

Sentiva di non poter sostenere quell'attacco generale di sprezzo e di crudeltà che aveva scorto apertamente sul viso di quel commesso, di Kornej e, senza eccezione, di tutti quelli che aveva incontrato in quei due giorni. Sentiva di non poter allontanare da sé l'odio degli uomini, perché quest'odio non derivava dal fatto ch'egli fosse cattivo (in tal caso non avrebbe cercato di essere migliore), ma dal fatto ch'era infelice in maniera vergognosa e ripugnante. Sapeva che proprio perché il suo cuore era lacerato, gli uomini sarebbero stati senza pietà verso di lui. Sentiva che gli uomini l'avrebbero distrutto, così come i cani strozzavano un cane dilaniato che guaisce dal dolore. Sapeva che l'unico modo di salvarsi dagli uomini era nascondere loro le proprie ferite, e questo aveva cercato inconsciamente di fare per due giorni, ma adesso sentiva che già non aveva più la forza di continuare questa lotta impari. 

La sua disperazione aumentava perché sapeva di essere assolutamente solo con il suo dolore. Non soltanto a Pietroburgo non aveva neppure una persona alla quale confidare tutto quello che provava, che lo compatisse non come alto funzionario, non come membro della società, ma semplicemente come uomo che soffre, ma in nessun altro posto aveva una persona simile. 

Aleksej Aleksandrovic era cresciuto orfano. Erano due fratelli. Il padre non se lo ricordavano, la madre era morta quando Aleksej Aleksandrovic aveva dieci anni. Il loro patrimonio non era grande. Uno zio Karenin, funzionario importante e, un tempo, favorito del defunto zar, li aveva educati. 

Terminati i corsi al ginnasio e all'università con medaglie d'onore, Aleksej Aleksandrovic, con l'aiuto dello zio, intraprese subito una carriera burocratica di primo piano e da quel momento si dette tutto all'ambizione del funzionario. Né al ginnasio, né all'università, né più tardi in servizio, Aleksej Aleksandrovic aveva mai stretto relazioni di amicizia con alcuno. Il fratello era la persona spiritualmente più vicina a lui, ma, funzionario del ministero degli esteri, aveva vissuto sempre all'estero, dove poi era morto presto, subito dopo il matrimonio di Aleksej Aleksandrovic. 

Durante il suo governatorato, una zia di Anna, ricca signora di provincia, aveva presentato l'uomo non più giovane, ma giovane governatore, alla nipote e l'aveva messo in una situazione tale per cui egli doveva o dichiararsi o andar via dalla città. Aleksej Aleksandrovic esitò a lungo. In quel momento gli argomenti a favore di questo passo erano tanti quanti quelli contro, e non c'era un motivo determinante che lo costringesse a venir meno alla sua regola di astenersi, nel dubbio. Ma la zia di Anna gli fece giungere all'orecchio, per mezzo di un conoscente, ch'egli si era già compromesso con la ragazza e che un dovere di onore lo obbligava a far la sua domanda di matrimonio. Egli fece la sua richiesta, e dette alla fidanzata e alla moglie tutto quel sentimento di cui era capace. 

L'affetto che provava per Anna aveva escluso dall'animo suo le ultime esigenze di rapporti cordiali con le persone. E ora, fra tutti i suoi conoscenti, non ne aveva alcuno intimo. C'erano molte di quelle che si chiamano relazioni, ma rapporti di amicizia non ce n'erano. Aleksej Aleksandrovic aveva molte persone da poter invitare a pranzo a casa sua, alle quali poteva chiedere la partecipazione in un affare che lo interessava, una raccomandazione per qualche protetto, persone con le quali poteva giudicare le azioni delle alte personalità del governo, ma i rapporti con queste persone erano chiusi in un campo solo, fermamente circoscritto dall'uso e dall'abitudine. C'era un compagno d'università con il quale era stato in dimestichezza più tardi e con il quale avrebbe potuto parlare di un dolore suo personale; ma questo compagno era provveditore in una circoscrizione scolastica lontana. Delle persone poi che erano a Pietroburgo, le più vicine e le più adatte erano il direttore della cancelleria e il medico. 

Michail Vasil'evic Šljudin, capo di gabinetto, era un uomo semplice, intelligente, buono e morale, e in lui Aleksej Aleksandrovic sentiva una certa simpatia verso la propria persona; ma la loro attività di servizio, che durava da cinque anni, aveva frapposto una barriera alle spiegazioni intime. 

Aleksej Aleksandrovic, finita la firma delle carte, tacque a lungo, guardando di tanto in tanto Michail Vasil'evic e varie volte tentò, ma non gli riuscì di parlare. Aveva già preparato la frase: ``avete sentito del mio dolore''; ma finì col dire, come al solito: ``allora mi preparerete questo'' e così lo congedò. 

L'altra persona era il medico, che era anche lui ben disposto nei suoi riguardi; ma tra di loro si era già da tempo tacitamente convenuto che bisognava far presto perché erano entrambi carichi di affari. 

Alle sue amicizie femminili, alla prima fra queste, la contessa Lidija Ivanovna, Aleksej Aleksandrovic non pensava. Tutte le donne, così come donne, erano per lui terribili e insopportabili 

\capitolo{XXII}\label{xxii-4} 

Aleksej Aleksandrovic aveva dimenticato la contessa Lidija Ivanovna, ma lei non lo aveva dimenticato. Proprio in quel momento penoso di disperata solitudine, ella venne da lui e, senza farsi annunciare, entrò nel suo studio. Lo trovò nella stessa posizione nella quale si era seduto, con il capo appoggiato su tutte e due le mani 

- J'ai forcé la consigne - disse, entrando a passi svelti e ansante per l'agitazione e il movimento rapido. - Ho sentito tutto! Aleksej Aleksandrovic! - ella proseguì, stringendo forte, con tutte e due le mani, la mano di lui e guardandolo negli occhi con i suoi bellissimi occhi pensosi. 

Aleksej Aleksandrovic, aggrottando le sopracciglia, si alzò e, liberata la mano da lei, le accostò una sedia. 

- Volete favorire, contessa? Io non ricevo perché sono ammalato - disse e le labbra gli tremarono. 

- Amico mio! - ripeté la contessa Lidija Ivanovna, senza staccare gli occhi da lui, e improvvisamente le sopracciglia le si sollevarono dalla parte interna, formando un triangolo sulla fronte; il suo viso brutto e giallognolo divenne ancora più brutto; ma Aleksej Aleksandrovic sentiva che aveva pena di lui e che era pronta a piangere. E la commozione lo prese: afferrò la mano morbida di lei e cominciò a baciarla. 

- Amico mio! - ella disse con voce rotta dall'agitazione. - Voi non dovete abbandonarvi al dolore. Il vostro dolore è grande, ma dovete pur trovare consolazione. 

- Sono distrutto, ucciso, non sono più un uomo! - disse Aleksej Aleksandrovic, lasciando andare la mano di lei, ma continuando a guardarle gli occhi pieni di lacrime. - La mia situazione è orribile perché non trovo in nessun posto, non trovo in me stesso nessun punto d'appoggio. 

- Voi troverete appoggio, ma non lo cercate in me, sebbene io vi preghi di credere alla mia amicizia - ella disse con un sospiro. - L'appoggio nostro è l'amore, quell'amore che Egli ci ha lasciato. Il Suo peso è leggero - ella disse con quello sguardo entusiastico che Aleksej Aleksandrovic conosceva così bene. - Lui vi sosterrà e vi aiuterà. 

Sebbene in queste parole ci fosse la commozione di fronte ai propri elevati sentimenti e ci fosse quella nuova entusiastica disposizione mistica, recentemente diffusasi a Pietroburgo, che ad Aleksej Aleksandrovic era parsa oziosa, tuttavia in questo momento Aleksej Aleksandrovic provò piacere nel sentirle. 

- Sono debole. Sono annientato. Io non ho previsto nulla e ora non capisco nulla. 

- Amico mio! - ripeteva Lidija Ivanovna. 

- Non già la perdita di quello che adesso non è più, non è questo - continuava Aleksej Aleksandrovic. - Io non ho rimpianti. Ma non posso non vergognarmi davanti agli uomini per questa situazione in cui mi trovo. È male, ma non posso, non posso. 

- Non siete voi che avete compiuto l'alto gesto del perdono, del quale io sono entusiasta e con me tutti, ma Lui, che abitava nel vostro cuore - disse la contessa Lidija Ivanovna, sollevando entusiasticamente gli occhi - e perciò voi non potete vergognarvi della vostra azione. 

Aleksej Aleksandrovic si accigliò e, piegate le mani, cominciò a far scricchiolare le dita. 

- Bisogna conoscere tutti i particolari - egli disse con voce stridula. - Le forze umane hanno dei limiti, contessa, e io ho raggiunto i miei limiti. Oggi, per tutto il giorno, ho dovuto dare ordini, ordini che riguardavano la casa, derivanti - e sottolineò la parola ``derivanti'' - dalla mia nuova situazione di uomo solo. La servitù, la governante, i conti\ldots{} Questo fuoco sottile mi ha bruciato, non ho avuto la forza di sopportare. A pranzo\ldots{} ieri sono quasi scappato via da tavola. Non potevo sopportare come mio figlio mi guardava. Non mi chiedeva il senso di tutto questo, ma voleva chiederlo, e io non potevo sopportare quello sguardo. Lui aveva paura di guardarmi, ma questo è poco\ldots{} - Aleksej Aleksandrovic voleva dire del conto che gli avevano portato, ma la voce cominciò a tremare ed egli si fermò. Quel conto su carta azzurra per un cappello e dei nastri non poteva ricordarlo senza provar pena di se stesso. 

- Capisco, amico mio! - disse la contessa Lidija Ivanovna. - Io capisco tutto. L'aiuto e la consolazione voi dovete trovarli non in me, anche se non sono venuta che per aiutarvi, se posso. Se potessi alleviarvi tutte queste piccole preoccupazioni umilianti\ldots{} Io capisco che c'è bisogno della parola di una donna, di un ordine di donna. Date l'incarico a me? 

Aleksej Aleksandrovic le strinse la mano in silenzio, con gratitudine. 

- Ci occuperemo insieme di Serëza. Io non sono forte nelle cose pratiche. Ma mi ci proverò, sarò la vostra governante. Non mi ringraziate. Io non sono sola a farlo\ldots{} 

- Non posso non esservi grato. 

- Ma, amico mio, non vi abbandonate al sentimento cui avete accennato, vergognarsi di quanto c'è di più alto per il cristiano: ``chi si umilia sarà innalzato''. E ringraziare me non potete. Bisogna ringraziare Lui e chiedere aiuto a Lui. In Lui solo troveremo calma, consolazione, salvezza e amore - disse lei e, sollevati gli occhi al cielo, cominciò a pregare, come parve ad Aleksej Aleksandrovic dal suo silenzio. 

Aleksej Aleksandrovic adesso l'ascoltava, e quelle espressioni che prima, anche senza riuscirgli sgradite, gli parevano oziose, adesso gli parvero naturali e consolanti. Ad Aleksej Aleksandrovic non piaceva quel nuovo spirito entusiastico. Era un credente che si interessava alla religione soprattutto in senso politico, e la nuova dottrina che permetteva alcune interpretazioni, proprio perché apriva le porte alla discussione e all'analisi, gli spiaceva come principio. Egli prima si era mostrato freddo e persino ostile a questa nuova dottrina, e con la contessa Lidija Ivanovna, che ne era appassionata, non ne aveva discusso mai, e aveva evitato con cura, tacendo, le sue sfide. Ora invece, per la prima volta, egli ascoltava con piacere le sue parole, e intimamente non le avversava. 

- Vi sono molto, molto grato per quanto avete fatto e per quanto mi avete detto - disse, quando ella ebbe finito di pregare. 

La contessa Lidija Ivanovna strinse ancora una volta le mani del suo amico. 

- Ora mi accingo all'opera - ella disse con un sorriso, dopo aver taciuto un po' e asciugando sul viso i resti delle lacrime. - Vado da Serëza. Solo in casi estremi mi rivolgerò a voi. - Si alzò e uscì. 

La contessa Lidija Ivanovna stette una mezz'ora da Serëza e là, bagnando di lacrime le guance del ragazzo spaurito, gli disse che suo padre era un santo e che sua madre era morta. 

La contessa Lidija Ivanovna mantenne la promessa. Ella prese su di sé tutte le cure riguardanti l'organizzazione e la direzione della casa di Aleksej Aleksandrovic, ma non aveva esagerato nel dire che non era forte nelle cose pratiche. Tutto quello che veniva ordinato da lei doveva esser cambiato, perché non era eseguibile, e lo cambiava Kornej, il maggiordomo di Aleksej Aleksandrovic, che, senza farsi notare da nessuno, conduceva tutta la casa di Karenin con calma e prudenza, riferendo l'indispensabile mentre il padrone si vestiva. L'aiuto di Lidija Ivanovna fu, tuttavia, molto efficace: ella dette un appoggio morale ad Aleksej Aleksandrovic con la coscienza del suo amore e della sua considerazione per lui, e soprattutto col riportarlo al cristianesimo, pensiero per lei consolante; lo trasformò cioè da indifferente e pigro credente in un appassionato e convinto seguace di quella nuova interpretazione della dottrina cristiana che negli ultimi tempi si era diffusa a Pietroburgo. Aleksej Aleksandrovic poteva facilmente persuadersi di questa interpretazione. Aleksej Aleksandrovic, così come Lidija Ivanovna e le altre persone che condividevano le loro opinioni, era privo di una certa profondità di immaginazione, di quella facoltà, cioè, dell'anima, grazie alla quale le immagini suscitate dalla fantasia divengono così reali da pretendere la rispondenza con altre immagini e con la realtà. Egli non vedeva nulla di impossibile e di poco coerente nell'immaginare che la morte, esistente per coloro che non credono, per lui non esistesse, e che, possedendo egli la fede più piena, della cui intensità era giudice lui stesso, non ci fosse neanche più colpa nell'anima sua, e che qui sulla terra egli provasse già la salvezza completa. 

È vero che la vanità e l'errore di questa rappresentazione della propria fede erano confusamente avvertite da Aleksej Aleksandrovic, ed egli sapeva pure che, quando si era abbandonato a un sentimento immediato, senza pensare affatto che il proprio perdono fosse l'effetto di una forza superiore, aveva provato una felicità ben più grande di quella che provava ora pensando ogni momento che nell'anima sua c'era il Cristo e che egli, firmando le carte, eseguiva la Sua volontà; tuttavia, per Aleksej Aleksandrovic era indispensabile pensare in tal modo, gli era così indispensabile, nella sua umiliazione, avere quella superiorità spirituale, sia pure immaginaria, dalla quale lui, disprezzato da tutti, poteva disprezzare gli altri, che vi si teneva aggrappato come a una vera salvezza, a una sua salvezza immaginaria. 

\capitolo{XXIII}\label{xxiii-4} 

La contessa Lidija Ivanovna, giovanissima, piena di entusiasmi, era stata maritata a un ricco, nobile, cordiale e dissoluto gaudente. Dopo un mese, il marito l'aveva abbandonata, e alle entusiastiche assicurazioni di tenerezza di lei aveva opposto un'irrisione e perfino un'ostilità che le persone, cui era noto il buon cuore del conte e che non vedevano nessun difetto nell'entusiastica Lidija, non riuscivano a spiegarsi in nessun modo. Da quel momento, anche se non divorziati, vivevano divisi e quando il marito si incontrava con la moglie la trattava sempre con quella immutata, velenosa irrisione, di cui non si riusciva a capire la causa. 

La contessa Lidija Ivanovna aveva cessato da tempo di essere innamorata del marito, ma da allora non aveva mai smesso di essere innamorata di qualcuno. Le accadeva di innamorarsi di varie persone nello stesso tempo, di uomini e donne; le accadeva di innamorarsi di quasi tutte le persone che eccellevano in un qualche modo. Era stata innamorata di tutte le nuove principesse e dei principi che entravano a far parte della famiglia imperiale, di un metropolita, di un vicario e di un prete. Era stata innamorata di un giornalista, di tre slavi e di Komisarov, d'un ministro, d'un dottore, d'un missionario inglese e di Karenin. Tutti questi amori, ora più deboli, ora più forti, non le avevano impedito di estendere i più larghi e complicati rapporti nella corte e nel gran mondo. Ma dal momento in cui la sventura aveva colpito Karenin, ed ella l'aveva preso sotto la sua personale protezione, dal momento in cui s'era affaccendata in casa Karenin prendendosi cura del suo buon andamento, ella aveva sentito che tutti i precedenti amori non erano mai stati veri, e che adesso era veramente innamorata del solo Karenin. Il sentimento ch'ella provava in questo momento verso di lui, le sembrava più forte di tutti i precedenti amori. Analizzando il proprio sentimento e paragonandolo con i precedenti, vedeva con chiarezza che non si sarebbe mai innamorata di Komisarov se costui non avesse salvato la vita all'imperatore, che non si sarebbe innamorata di Ristic-Kudzickij se non ci fosse stata la questione slava, ma che Karenin ella lo amava per lui stesso, per la sua alta anima incompresa, per il tono sottile a lei caro della voce, dalle intonazioni strascicate, per il suo sguardo stanco, per il suo carattere, per le sue mani morbide dalle vene gonfie. Ella non solo gioiva di un incontro con lui, ma cercava sul viso di lui i segni dell'impressione che lei stessa produceva. Voleva piacergli non solo con i discorsi, ma con tutta la persona. Adesso per lui si occupava del proprio abbigliamento molto più di prima. Si sorprendeva a far sogni su quello che sarebbe stato se lei non fosse stata maritata e lui libero. Arrossiva di agitazione quando egli entrava nella stanza; non poteva trattenere un sorriso d'entusiasmo quando egli le diceva qualcosa di piacevole. 

Già da qualche giorno la contessa Lidija Ivanovna si trovava in grande agitazione. Aveva saputo che Anna e Vronskij erano a Pietroburgo. Bisognava salvare Aleksej Aleksandrovic da un incontro con lei, bisognava salvarlo perfino dalla tormentosa notizia che quella donna terribile era nella stessa città in cui era lui, e che ad ogni momento egli poteva incontrarla. 

Lidija Ivanovna, per mezzo dei suoi conoscenti, si informava di quello che avevano intenzione di fare quelle ``persone ripugnanti'', come ella chiamava Anna e Vronskij, e cercava di guidare, in quei giorni, tutti i movimenti del suo amico, affinché non avesse a incontrarli. Il giovane aiutante di campo, amico di Vronskij, attraverso il quale ella riceveva le notizie e che dalla contessa Lidija Ivanovna sperava di ricevere una concessione, le disse che avevano ultimato i loro affari e che l'indomani sarebbero partiti. Lidija Ivanovna aveva già cominciato a rasserenarsi quando, proprio la mattina dopo, le portarono un biglietto di cui ella riconobbe con orrore la scrittura. Era la scrittura di Anna Karenina. La busta era di carta doppia come scorza d'albero; sul foglio giallo, oblungo, c'era un enorme monogramma, e la lettera emanava un ottimo profumo. 

- Chi l'ha portato? 

- Un inserviente d'albergo. 

La contessa Lidija Ivanovna per un pezzo non riuscì a mettersi a sedere per leggere la lettera. L'agitazione le provocò un attacco d'asma, cui andava soggetta. Quando si fu calmata, lesse la seguente lettera in francese. 

\begin{quote}
``Madame la Comtesse, 

i sentimenti cristiani che riempiono il vostro cuore mi dànno, lo sento, l'imperdonabile ardire di scrivervi. Io sono infelice per il distacco da mio figlio. Supplico voi per ottenere il permesso di vederlo una volta prima della mia partenza. Perdonatemi se mi rivolgo a voi. Mi rivolgo a voi e non ad Aleksej Aleksandrovic, perché non voglio far soffrire quest'uomo generoso col ricordarmi a lui. Conoscendo la vostra amicizia per lui, mi comprenderete. Manderete Serëza da me, o devo venire io in casa a una determinata ora, o mi farete sapere quando e dove posso vederlo fuori? Non suppongo un rifiuto, conoscendo la generosità di colui dal quale ciò dipende. Voi non potete immaginare come sia ansiosa di vedere mio figlio, e quanto vi sia grata per il vostro aiuto. 

Anna
\end{quote}

Tutto in questa lettera irritò la contessa Lidija Ivanovna: e il contenuto, e l'accenno alla generosità e, soprattutto, il tono, come le sembrava, disinvolto. 

- Di' che non ci sarà risposta - disse la contessa Lidija Ivanovna e subito, aperta una cartella, scrisse ad Aleksej Aleksandrovic che sperava vederlo dopo mezzogiorno, durante gli auguri a corte. 

``Ho bisogno di parlare con voi per una faccenda triste e importante. Là stabiliremo dove. Meglio di tutto da me, dove farò preparare il vostro tè. È indispensabile. Egli ci impone la croce, ma Egli ci dà la forza'' aggiunse per prepararlo. 

La contessa Lidija Ivanovna scriveva, di solito, due o tre biglietti al giorno ad Aleksej Aleksandrovic. Per comunicare con lui amava questo sistema, che aveva quella certa raffinata misteriosità che non avevano i suoi rapporti personali. 

\capitolo{XXIV}\label{xxiv-3} 

Gli auguri erano finiti. Quelli che andavano via, incontrandosi, parlavano dell'ultima novità del giorno, delle ricompense appena ricevute e del cambiamento degli alti funzionari. 

- Se dessero il ministero della guerra alla contessa Mar'ja Borisova e facessero capo di Stato Maggiore la principessa Vatkovskaja - diceva rivolto a una damigella d'onore, bella e slanciata, che gli aveva chiesto del movimento, un vecchietto canuto con l'uniforme ricamata in oro. 

- E me aiutante di campo - rispondeva la damigella, sorridendo. 

- Voi l'avete già la nomina. A voi per l'amministrazione ecclesiastica e, per aiuto, Karenin. 

- Buon giorno, principe! - disse il vecchietto, stringendo la mano a colui che si era avvicinato. 

- Cosa dicevate di Karenin? - disse il principe. 

- Che lui e Putjatov hanno ricevuto l'Aleksandr Nevskij. 

- Pensavo che l'avesse già. 

- No, guardatelo - disse il vecchietto, indicando con il cappello ricamato Karenin in uniforme di corte e con la nuova fascia rossa a tracolla, fermo sulla porta della sala con uno dei membri influenti del consiglio di stato. - È felice e contento, come un soldone di rame - aggiunse, fermandosi per stringere la mano a un bel ciambellano dalla statura atletica. 

- No, è invecchiato - disse il ciambellano. 

- Per le preoccupazioni. Adesso non fa che scrivere progetti. Non lascerà il disgraziato finché non gli avrà esposto tutto. 

- Come invecchiato? Il fait des passions. Penso che ora la contessa Lidija Ivanovna sarà gelosa della moglie per lui. 

- Ma cosa! Per favore, non parlate della contessa Lidija Ivanovna. 

- Ma è forse male che sia innamorata di Karenin? 

- Ma è vero che la Karenina è qui? 

- Cioè, non qui a corte, ma a Pietroburgo. L'ho incontrata ieri, con Aleksej Vronskij, bras dessus bras dessous, per la Morskaja. 

- C'est un homme qui n'a pas\ldots{} - cominciò il ciambellano, ma si fermò cedendo il passo e salutando un personaggio della famiglia imperiale. 

Così si parlava, senza posa, di Aleksej Aleksandrovic, giudicandolo e irridendolo, mentre lui, sbarrata la strada al membro del consiglio di stato che aveva afferrato, e senza smettere neppure per un attimo la propria esposizione per non lasciarselo sfuggire, gli esponeva punto per punto un suo progetto finanziario. 

Quasi nello stesso tempo in cui la moglie aveva abbandonato Aleksej Aleksandrovic, gli era accaduto l'avvenimento più doloroso per un funzionario: la fine del movimento di ascesa nella carriera. Questa fine si era verificata e tutti vedevano ciò chiaramente, ma lui, Aleksej Aleksandrovic, non riconosceva ancora che la sua carriera era finita. Fosse l'urto con Stremov, fosse la disgrazia con la moglie, o fosse che Aleksej Aleksandrovic era giunto al limite che gli era stato assegnato, certo è che quell'anno era evidente che la sua carriera di funzionario era finita. Occupava ancora un posto importante, era membro di molte commissioni e comitati; ma era ormai l'uomo che s'era esaurito tutto e dal quale non ci si attendeva più nulla. Qualunque cosa egli dicesse, qualunque cosa proponesse, lo si ascoltava come se quello che proponeva fosse da gran tempo noto e fosse proprio quello che non era necessario. 

Ma Aleksej Aleksandrovic non lo sentiva; al contrario, allontanatosi dalla partecipazione diretta all'attività governativa, vedeva ora più chiaramente di prima i difetti e gli errori dell'attività degli altri, e stimava suo dovere indicare i mezzi per correggerli. Ben presto, dopo la sua separazione dalla moglie, cominciò a scrivere i suoi primi appunti sul nuovo tribunale, primi dell'innumere serie di appunti inutili su tutti i rami dell'amministrazione che era destinato a scrivere. 

Aleksej Aleksandrovic non solo non avvertiva la sua posizione senza speranza nel mondo burocratico e non solo non se ne amareggiava, ma era più che mai contento della propria attività. 

``Chi ha moglie si cura delle cose mondane, si preoccupa di compiacere la moglie; chi non è ammogliato si preoccupa delle cose del Signore, si preoccupa di compiacere il Signore'' dice l'apostolo Paolo, e Aleksej Aleksandrovic, che adesso si lasciava guidare in tutte le sue cose dalla Sacra Scrittura, ricordava spesso questo testo. Gli pareva, dal momento in cui era rimasto senza moglie, di servire, con quei progetti, il Signore più di prima. 

L'evidente impazienza del membro del consiglio di stato, che voleva sfuggirgli, non scomponeva Aleksej Aleksandrovic; egli smise di esporre soltanto quando il membro, profittando del passaggio di un personaggio di casa imperiale, scivolò via. 

Rimasto solo, Aleksej Aleksandrovic abbassò la testa, raccogliendo le idee, poi si voltò distrattamente e si diresse verso una porta presso la quale sperava di incontrare la contessa Lidija Ivanovna. 

``E come sono tutti fisicamente forti e sani! - pensò Aleksej Aleksandrovic, guardando le fedine profumate, ben in ordine, di un aitante ciambellano e il collo rosso di un principe stretto nell'uniforme, accanto al quale doveva passare. - È giusto dire che nel mondo tutto è male'' pensò guardando ancora una volta di traverso i polpacci del ciambellano. 

Movendo le gambe lentamente, Aleksej Aleksandrovic, con la solita aria di stanchezza e di dignità, si inchinò a quei signori che parlavano di lui, e, guardando la porta, cercò con gli occhi la contessa Lidija Ivanovna. 

- Ah! Aleksej Aleksandrovic! - disse il vecchietto, con gli occhi che scintillavano maligni, nel momento in cui Karenin lo raggiungeva e chinava il capo con gesto freddo. - Non mi sono ancora congratulato con voi - disse, indicando la fascia appena ricevuta. 

- Vi ringrazio - rispose Aleksej Aleksandrovic. - Che bella giornata! - soggiunse, secondo la sua abitudine, sottolineando in modo particolare la parola ``bella''. Che lo irridessero lo sapeva, ma da loro non si aspettava null'altro che ostilità; vi si era già abituato. 

Viste le spalle giallognole, sporgenti dal busto, della contessa Lidija Ivanovna e gli occhi pensosi di lei, belli e invitanti, Aleksej Aleksandrovic sorrise, scoprendo i suoi denti bianchi, inalterabili, e le si avvicinò. 

La toletta di Lidija Ivanovna le era costata molta fatica, come, del resto, sempre il suo abbigliamento in questi ultimi tempi. Adesso, lo scopo del suo abbigliamento era del tutto opposto a quello che ricercava trent'anni prima. Allora le piaceva abbellirsi per qualche cosa, e quanto più tanto meglio. Adesso, invece, era immancabilmente adornata in modo così poco conveniente alla sua età e alla sua figura, che si preoccupava solo di non rendere troppo stridente il contrasto fra gli ornamenti e il suo aspetto esteriore. E nei riguardi di Aleksej Aleksandrovic aveva raggiunto lo scopo, ché a lui ella sembrava attraente. Per lui ella era l'unica isola non solo di simpatia, ma di amore, in mezzo al mare di ostilità e di irrisione che lo circondava. 

Passando attraverso la fila degli sguardi di scherno, egli tendeva istintivamente verso lo sguardo innamorato di lei, come una pianta verso la luce. 

- Mi rallegro - gli disse, indicando la fascia con gli occhi. 

Trattenendo un sorriso di compiacimento, egli alzò le spalle e socchiuse gli occhi, come a dire che questo non lo poteva rallegrare. La contessa Lidija Ivanovna sapeva bene che questa era una delle sue più grandi gioie, anche se egli non l'avrebbe mai confessato. 

- Come va il nostro angelo? - disse la contessa Lidija Ivanovna, intendendo Serëza. 

- Non posso dire d'essere pienamente contento di lui - disse Aleksej Aleksandrovic sollevando le sopracciglia e aprendo gli occhi - e anche Sitnikov non è contento. - Sitnikov era il precettore al quale era stata affidata l'educazione mondana di Serëza. - Come vi dicevo, c'è in lui una certa freddezza per le questioni importanti che devono commuovere l'animo di ogni uomo e di ogni fanciullo - e Aleksej Aleksandrovic cominciò a esporre le proprie idee sull'unica questione che lo interessava oltre l'ufficio, l'educazione del figlio. 

Quando Aleksej Aleksandrovic, con l'aiuto di Lidija Ivanovna, era tornato alla vita e all'attività, aveva sentito il dovere di occuparsi dell'educazione del figlio che era rimasto con lui. Non essendosi mai occupato prima di questioni di educazione, Aleksej Aleksandrovic aveva dedicato un po' di tempo allo studio teorico della materia. Letti alcuni libri di antropologia, di pedagogia e di didattica, Aleksej Aleksandrovic si era formato un piano di educazione e, fatto venire il miglior pedagogo di Pietroburgo per la direzione, si era accinto all'opera. E questa opera lo teneva continuamente occupato. 

- Sì, ma il cuore? Io vedo in lui il cuore del padre e, con un cuore simile, un bambino non può essere cattivo - disse Lidija Ivanovna con entusiasmo. 

- Sì, può darsi\ldots{} Per quel che riguarda me, io compio il mio dovere. È tutto quello che posso fare. 

- Venite da me - disse la contessa Lidija Ivanovna, dopo un certo silenzio; - dobbiamo parlare di una cosa triste per voi. Io darei tutto per liberarvi di alcuni ricordi, ma gli altri non la pensano così. Ho ricevuto una lettera da lei. Lei è qui, a Pietroburgo. 

Aleksej Aleksandrovic rabbrividì al ricordo della moglie, e immediatamente sul suo viso apparve quella fissità di morte che esprimeva un completo abbandono in questa faccenda. 

- Me l'aspettavo - disse. 

La contessa Lidija Ivanovna lo guardò con entusiasmo, e lacrime di estasi le vennero agli occhi dinanzi alla grandezza dell'anima di lui. 

\capitolo{XXV}\label{xxv-3} 

Quando Aleksej Aleksandrovic entrò nello studio piccolo, accogliente, pieno di porcellane antiche e di ritratti, della contessa Lidija Ivanovna, la padrona non c'era ancora. Si stava cambiando. 

Sulla tavola rotonda era distesa una tovaglia e c'era un servizio cinese e una teiera d'argento a spirito. Aleksej Aleksandrovic guardò distrattamente, in giro, gli innumerevoli ritratti noti che adornavano lo studio e, sedutosi presso la tavola, aprì il Vangelo che vi stava sopra. Il fruscio del vestito di seta della contessa lo distrasse. 

- Su, ecco, adesso ci metteremo a sedere tranquillamente - disse la contessa Lidija Ivanovna, con un sorriso agitato, insinuandosi in fretta fra il divano e la tavola - e parleremo prendendo il tè. 

Dopo alcune parole di preparazione, la contessa Lidija Ivanovna, respirando pesantemente e diventando rossa, consegnò nelle mani di Aleksej Aleksandrovic la lettera da lei ricevuta. 

Finita di leggere la lettera, egli tacque a lungo. 

- Io non credo di avere il diritto di rifiutare ciò - egli disse timido, sollevando gli occhi. 

- Amico mio, voi non vedete in nessuno il male! 

- Io, al contrario, vedo che tutto è male. Ma è giusto questo? 

Sul suo viso c'erano l'incertezza e la ricerca di consiglio, di appoggio e di guida in una faccenda per lui incomprensibile. 

- No - lo interruppe la contessa Lidija Ivanovna. - C'è un limite a tutto. Io capisco l'immoralità - disse, non del tutto sinceramente, poiché non aveva mai potuto capire che cosa conducesse le donne all'immoralità; - ma non capisco la crudeltà, e con chi? con voi! Come soggiornare nella stessa città in cui siete voi? No, davvero, più si vive, più si impara. E io imparo a capire la vostra elevatezza e la sua meschinità. 

- E chi getterà la prima pietra? - disse Aleksej Aleksandrovic, evidentemente soddisfatto della propria parte. - Io ho perdonato tutto e perciò non posso privarla di quello che è un'esigenza d'amore per lei, d'amore per il figlio. 

- Ma è amore questo, amico mio? È sincero? Ammettiamolo, voi avete perdonato, voi perdonate\ldots{} ma abbiamo noi il diritto di agire sull'anima di quell'angelo? Egli prega per lei e chiede a Dio di perdonarle i peccati\ldots{} E così è meglio. Ma ora, cosa penserà? 

- Non avevo pensato a questo - disse Aleksej Aleksandrovic, evidentemente consentendo. 

La contessa Lidija Ivanovna si coprì il viso con le mani e tacque un po'. Pregava. 

- Se chiedete il mio consiglio - ella disse, dopo aver pregato e scoprendo il viso - io non vi consiglio di far questo. Non vedo forse come soffrite, come ciò ha riaperto tutte le vostre ferite? Ma del resto voi, come sempre, dimenticate voi stesso. Ma a che cosa mai può condurre ciò? A nuove pene da parte vostra, a tormenti per il bambino. Se in lei è rimasto qualcosa di umano, ella stessa non deve desiderare ciò. No, io non lo consiglio, non ho dubbi, e, se me ne autorizzate, le scriverò. 

Aleksej Aleksandrovic acconsentì, e la contessa Lidija Ivanovna scrisse la seguente lettera in francese. 

\begin{quote}
``Gentile Signora,

il ricordarvi a vostro figlio può produrre da parte sua domande alle quali non è possibile rispondere senza introdurre, nell'animo del bambino, lo spirito del giudizio su quello che deve essere sacro per lui, e perciò vi prego di comprendere il rifiuto di vostro marito nello spirito dell'amore cristiano. Chiedo per voi misericordia dall'Altissimo.

Contessa Lidija''.
\end{quote} 

Questa lettera raggiunse lo scopo segreto che la contessa Lidija Ivanovna nascondeva a se stessa. Offese Anna nel profondo dell'anima. 

Da parte sua Aleksej Aleksandrovic, tornando da Lidija Ivanovna a casa sua, non poté, quel giorno, darsi alle sue solite occupazioni, né trovare quella calma spirituale di uomo credente e salvato che aveva provato prima. 

Il ricordo della moglie, che era tanto colpevole dinanzi a lui e di fronte alla quale egli era così santo, come gli diceva giustamente la contessa Lidija Ivanovna, non avrebbe dovuto turbarlo, ma egli non era tranquillo: non riusciva a capire il libro che leggeva, non riusciva a scacciare i ricordi tormentosi dei suoi rapporti con lei, gli errori che gli sembrava aver commesso verso di lei. Il ricordo di come aveva accolto, ritornando dalle corse, la sua confessione di infedeltà (in particolare il fatto che aveva preteso da lei solo le convenzioni esteriori, senza sfidare lui a duello) lo tormentava come un rimorso. Lo tormentava pure il ricordo della lettera che aveva scritto a lei; in particolare il proprio perdono, non necessario a nessuno, e le sue preoccupazioni per una creatura non sua gli bruciavano il cuore di vergogna e di rimorso. 

E proprio un sentimento di vergogna e di rimorso provava adesso, esaminando tutto il suo passato con lei e ricordando le parole impacciate con le quali egli, dopo lunghe esitazioni, le aveva fatto la sua domanda di matrimonio. 

``Ma di che cosa sono colpevole io?'' diceva a se stesso. E questa domanda suscitava sempre in lui un'altra domanda: ``Sentivano forse diversamente, amavano forse diversamente, si sposavano forse diversamente quelle altre persone, quei Vronskij, quegli Oblonskij\ldots{} quei ciambellani dai polpacci grassi?''. E gli si presentava tutta una serie di persone dense di umori, forti, che non avevano dubbi, le quali involontariamente attiravano, sempre e dovunque, la sua attenzione curiosa. Scacciava da sé questi pensieri, cercava di convincersi che non viveva per la vita di questo mondo, ma per quella eterna; che nell'animo suo c'erano pace e amore. Ma l'aver commesso in questa vita terrena, per lui insignificante, degli errori insignificanti lo tormentava come se non ci fosse neppure mai stata quella salvezza eterna in cui credeva. Ma questa tentazione non durò a lungo, e ben presto nell'animo di Aleksej Aleksandrovic si ristabilirono la calma e la elevatezza, grazie alle quali egli poteva dimenticare quello che non voleva ricordare. 

\capitolo{XXVI}\label{xxvi-3} 

- E allora, Kapitonyc? - disse Serëza, rosso in viso e allegro, tornando dalla passeggiata, alla vigilia del suo compleanno e dando il cappotto a pieghe al vecchio portiere che sorrideva al piccolo uomo dall'alto della sua statura. - Be', è venuto oggi l'impiegato col braccio al collo? L'ha ricevuto papà? 

- L'ha ricevuto. Era appena uscito il capo gabinetto che io l'ho annunciato - disse il portiere, ammiccando. - Vi prego, lo levo io. 

- Serëza! - disse lo slavo istitutore fermandosi sulla porta che dava nelle stanze interne - levatelo da solo. 

Ma Serëza, pur avendo sentito la voce fiacca dell'istitutore, non vi fece attenzione. Stava fermo, tenendosi con la mano alla cintura del portiere, e lo guardava in viso. 

- Be', e papà ha fatto per lui quello che occorreva? 

Il portiere fece un cenno affermativo col capo. 

L'impiegato col braccio al collo, che era già andato sette volte a chiedere qualcosa ad Aleksej Aleksandrovic, interessava Serëza e il portiere. Serëza l'aveva trovato una volta nell'ingresso e aveva sentito come chiedeva pietosamente al portiere di annunciarlo, dicendo che gli toccava di morire insieme ai figliuoli. 

Da quel momento Serëza, dopo averlo incontrato ancora una volta nell'ingresso, se ne interessò. 

- Ebbene, era contento? - chiese. 

- E come non esser contento! È scappato via di qua, saltando quasi. 

- E hanno portato qualcosa? - chiese Serëza, dopo un attimo di silenzio. 

- Eh, signorino - disse in un bisbiglio il portiere, scotendo il capo - da parte della contessa. 

Serëza capì subito che quello di cui parlava il portiere, era il regalo della contessa Lidija per il suo compleanno. 

- Cosa dici? Dove? 

- Kornej l'ha portato dentro da papà. Deve essere una bella cosa! 

- Com'è grande? Così? 

- Più piccola, ma bella. 

- Un libro? 

- No, una cosa. Andate, andate, Vasilij Lukic chiama - disse il portiere, che aveva sentito i passi dell'istitutore che si avvicinavano e, raddrizzando accortamente la manina nel guanto tolto a mezzo che lo teneva per la cintura, accennò col capo verso Lukic. 

- Vasilij Lukic, un minutino! - disse Serëza con quel sorriso allegro e amorevole, che conquistava sempre l'imperioso Vasilij Lukic. 

Serëza era troppo allegro, era troppo felice perché potesse non far partecipe anche il suo amico portiere della gioia di famiglia appresa alla passeggiata al Giardino d'Estate dalla nipote della contessa Lidija Ivanovna. Questa gioia gli sembrava particolarmente importante perché coincideva con la gioia dell'impiegato e con la propria, per i giocattoli che gli avevano portato. A Serëza sembrava che quello fosse un giorno in cui tutti dovessero essere contenti e allegri. 

- Lo sai? Papà ha ricevuto l'Aleksandr Nevskij? 

- E come se non lo so! Sono già venuti a rallegrarsi. 

- Be', è contento? 

- E come non esser contento del favore dello zar? Vuol dire che l'ha meritato - disse, severo e compreso, il portiere. 

Serëza si fece pensieroso, guardando il viso del portiere, studiato fin nei minimi particolari, soprattutto il mento che pendeva fra le fedine canute, e che nessuno vedeva, tranne Serëza che non lo guardava altrimenti che di sotto in su. 

- Be', e tua figlia è un pezzo che è stata da te? 

La figlia del portiere era una ballerina del balletto. 

- E quando si può venire nei giorni di lavoro? Anche loro devono studiare. Anche per voi c'è lo studio; andate signorino. 

Giunto nella stanza, Serëza, invece di mettersi a sedere per fare le lezioni, confidò al maestro la sua supposizione che quello che avevano portato doveva essere una macchina. 

- Che ne pensate? - chiese. 

Ma Vasilij Lukic pensava solo che bisognava studiare la lezione di grammatica per il maestro che sarebbe venuto alle due. 

- No, ditemi solo, Vasilij Lukic - domandò a un tratto, già seduto dietro al tavolo di studio e trattenendo nelle mani il libro - che cosa c'è di più dell'Aleksandr Nevskij? Lo sapete? papà ha ricevuto l'Aleksandr Nevskij. 

Vasilij Lukic rispose che al di sopra dell'Aleksandr Nevskij c'era l'ordine di Vladimir. 

- E più su? 

- Più su di tutto l'Andrej Pervozvannyj. 

- E più su dell'Andrej? 

- Non lo so. 

- E come, neanche voi lo sapete? - Serëza, appoggiatosi sui gomiti, si sprofondò in meditazioni. 

Le sue meditazioni erano più complesse e varie. Egli si figurava come suo padre avrebbe ricevuto a un tratto l'ordine di Vladimir e quello di Andrej, e come, in seguito a ciò, quel giorno alla lezione sarebbe stato molto più indulgente, e come lui stesso, divenuto grande, avrebbe ricevuto tutti gli ordini, anche quello che avrebbero creato più su di quello di Andrej. Appena creato, egli lo avrebbe meritato. Ne avrebbero creato uno ancora più in alto e lui, subito, ecco, se lo merita. 

In queste meditazioni il tempo passò e quando venne il maestro, la lezione sui complementi di tempo, di luogo e di modo non era preparata, e il maestro non solo non fu contento, ma si dispiacque. Il dolore del maestro commosse Serëza. Si sentiva colpevole perché non aveva studiato la lezione; per quanto si sforzasse non ci riusciva in nessun modo. Finché il maestro spiegava, ci credeva e sembrava capire, ma non appena rimaneva solo, non poteva assolutamente capire e ricordarsi come un'espressione così corta e facile a intendersi quale ``a un tratto'' fosse un complemento di modo; gli dispiacque di avere addolorato il maestro, e voleva consolarlo. 

Scelse un momento in cui il maestro taceva, guardando il libro. 

- Michail Ivanyc, quando è il vostro onomastico? - chiese a un tratto. 

- Sarebbe meglio che pensaste al vostro lavoro, ché l'onomastico non ha nessuna importanza per un essere ragionevole. È un giorno come un altro, nel quale bisogna lavorare. 

Serëza guardò il maestro, la sua barbetta rada, gli occhiali che erano scesi più giù del taglio ch'era sul naso, e si fece così pensieroso che non sentì più nulla di quello che gli spiegava il maestro. Egli capiva che il maestro non pensava a quello che diceva; lo sentiva dal tono con cui le parole erano dette. ``Ma perché si sono messi tutti d'accordo nel dire queste cose sempre alla stessa maniera, tutte le più noiose e inutili cose? Perché mi allontana da sé, perché non mi vuol bene?'' si chiedeva con tristezza e non riusciva a immaginare una risposta. 

\capitolo{XXVII}\label{xxvii-3} 

Dopo il maestro, c'era la lezione del padre. Finché non giunse il padre, Serëza sedette al tavolo, gingillandosi con un coltellino, e cominciò a pensare. Nel numero delle occupazioni preferite da Serëza c'era la ricerca della madre durante la passeggiata. Egli non credeva alla morte in genere, e in particolare alla morte di lei, sebbene Lidija Ivanovna glielo avesse detto e il padre riconfermato, e perciò anche dopo che gli avevano detto che era morta, egli, durante la passeggiata, continuava a cercarla. Ogni donna bella e piacente, con i capelli scuri era sua madre. Quando vedeva una figura di donna simile, nell'animo suo si sollevava un tale sentimento di tenerezza, ch'egli si sentiva soffocare e gli venivano le lacrime agli occhi. E aspettava da un momento all'altro ch'ella si avvicinasse a lui e sollevasse il velo. Si sarebbe, allora, visto tutto il viso, ella avrebbe sorriso, l'avrebbe abbracciato, lui avrebbe sentito il profumo di lei, avrebbe sentito la tenerezza della sua mano, e si sarebbe messo a piangere felice come una volta, di sera, che le si era coricato ai piedi e lei gli aveva fatto il solletico, ed egli aveva riso e morso la mano bianca con gli anelli. Quando poi aveva saputo, per caso, dalla njanja che sua madre non era morta, e quando il padre e Lidija Ivanovna gli spiegarono che era morta per lui perché era cattiva (alla quale cosa non credeva in nessun modo perché l'amava), egli la cercò e l'aspettò sempre allo stesso modo. Quel giorno, al Giardino d'Estate, c'era una signora col velo lilla, ch'egli aveva seguito con un gran batticuore credendo che fosse lei, mentre si avvicinava a loro per un sentiero. Ma quella signora non era giunta fino a loro e si era nascosta chi sa dove. Quel giorno Serëza aveva sentito più forte che mai uno slancio di amore per lei e adesso, mentre aspettava il padre, aveva tagliuzzato l'orlo del tavolo con il temperino, guardando dinanzi a sé con gli occhi scintillanti, smarrito nel pensiero di lei. 

- Viene papà - lo richiamò Vasilij Lukic. 

Serëza saltò su, si accostò al padre e, baciatagli la mano, lo guardò attento, cercando dei segni di gioia per aver ricevuto l'Aleksandr Nevskij. 

- Hai passeggiato bene? - chiese Aleksej Aleksandrovic, sedendosi nella sua poltrona, attirando a sé il libro dell'Antico Testamento e aprendolo. Sebbene Aleksej Aleksandrovic avesse detto più di una volta a Serëza che ogni buon cristiano deve conoscere con sicurezza la storia sacra, egli stesso controllava spesso l'Antico Testamento, e Serëza l'aveva notato. 

- Sì, è stato molto divertente, papà - disse Serëza, sedendosi di sbieco sulla sedia e dondolandosi, il che era proibito. - Ho visto Naden'ka - Naden'ka era la nipote di Lidija Ivanovna che veniva educata presso di lei. - Mi ha detto che vi hanno dato una croce nuova. Siete contento, papà? 

- In primo luogo, non dondolarti, per piacere - disse Aleksej Aleksandrovic. - E in secondo luogo, non è cara la ricompensa, ma il lavoro. E vorrei che tu lo capissi. Ecco, se tu lavorerai, se studierai per avere una ricompensa, il lavoro ti sembrerà faticoso; ma se lavorerai - Aleksej Aleksandrovic parlava, ricordando come si era sostenuto con la coscienza del dovere durante il noioso lavoro della mattinata, consistente nella firma di centodiciotto carte - amando il lavoro, vi troverai una ricompensa per te. 

Gli occhi scintillanti di tenerezza e di felicità di Serëza si spensero e si abbassarono sotto lo sguardo del padre. Era quello stesso tono che da tempo il padre usava con lui e che da tempo Serëza aveva preso a secondare. Il padre gli parlava sempre come se si rivolgesse a un ragazzo immaginato da lui, uno di quelli che esistono nei libri, per nulla affatto simile a Serëza. E Serëza, col padre, si sforzava sempre di fingersi proprio questo tale ragazzo da libro. 

- Tu capisci questo, spero? - disse il padre. 

- Sì, papà - rispose Serëza, fingendosi il ragazzo immaginario. 

La lezione consisteva nell'imparare a memoria alcuni versetti del Vangelo e nel ripetere il principio dell'Antico Testamento. I versetti del Vangelo Serëza li sapeva discretamente, ma nel momento in cui prese a recitarli si mise a fissare l'osso della fronte del padre che si piegava così rapidamente verso la tempia, si confuse, e traspose la fine di un versetto con una parola eguale al principio di un altro. Per Aleksej Aleksandrovic era evidente ch'egli non capiva quello che diceva, e si irritò. 

Aggrottò le sopracciglia e cominciò a spiegare quello che Serëza aveva sentito già molte volte e che non ricordava mai perché lo capiva troppo chiaramente, così come, press'a poco, il fatto che ``a un tratto'' era complemento di modo. Serëza guardava il padre con uno sguardo spaventato e pensava solo a un'unica cosa: l'avrebbe costretto o no a ripetere quello che aveva detto, come a volte accadeva? E questo pensiero spaventò tanto Serëza che non capì più nulla. Ma il padre non lo costrinse a ripetere e passò alla lezione sull'Antico Testamento. Serëza raccontò bene gli avvenimenti per se stessi, ma quando bisognò rispondere alle domande su quello che significavano allegoricamente alcuni di essi, non seppe niente, pur essendo stato punito per questa lezione. Il passo, poi, dove non sapeva dire proprio nulla ed esitava e tagliuzzava il tavolo e si dondolava sulla seggiola era quello in cui bisognava parlare dei patriarchi antidiluviani. Fra questi non ne conosceva nessuno all'infuori di Enoch, assunto vivo in cielo. Prima ricordava dei nomi, ma adesso li aveva dimenticati completamente, proprio perché Enoch era il suo personaggio preferito di tutto l'Antico Testamento, e all'assunzione al cielo di Enoch vivo si collegava tutto un lungo ragionamento, a cui egli, adesso, si abbandonò guardando con gli occhi fissi la catena dell'orologio del padre e un bottone del panciotto abbottonato a metà. 

Alla morte, di cui così spesso gli parlavano, Serëza non credeva per nulla. Non credeva che le persone da lui amate potessero morire, e in particolare ch'egli sarebbe morto. Questo era per lui assolutamente impossibile e incomprensibile. Ma gli dicevano che tutti sarebbero morti; egli interrogava perfino persone alle quali credeva e quelle lo confermavano: anche la njanja lo diceva, suo malgrado. Ma Enoch non era morto, perciò non tutti morivano. ``E perché mai, non può ognuno essere così degno dinanzi a Dio da essere assunto vivo in cielo?'' pensava Serëza. I cattivi, cioè quelli a cui Serëza non voleva bene, quelli potevano morire, ma i buoni potevano essere tutti come Enoch. 

- Su, allora, quali sono i patriarchi? 

- Enoch, Enos. 

- Ma questi li hai già detti. Male, Serëza, molto male. Se non cerchi di conoscere quello che è più necessario di tutto per un cristiano - disse il padre, alzandosi - allora cosa mai può interessarti? Io sono scontento di te, anche Pëtr Ignat'ic - era costui l'istitutore più importante - è scontento di te\ldots{} Io devo punirti. 

Il padre e l'istitutore erano scontenti di Serëza, e realmente egli studiava molto male. Ma non si poteva dire in nessun modo che fosse un ragazzo senza capacità. Al contrario era molto più versato di quei ragazzi che l'istitutore portava come esempio a Serëza. Secondo il padre, egli non voleva apprendere quello che gli insegnavano. In realtà invece non poteva studiarlo. Non poteva perché nell'animo suo c'erano esigenze più imperiose di quelle che presentavano il padre e l'istitutore. Queste esigenze erano in contrasto, ed egli lottava proprio con i suoi educatori. 

Aveva nove anni, era un bambino; ma l'anima sua la conosceva, gli era cara, la proteggeva come la palpebra protegge l'occhio, e senza la chiave dell'amore non lasciava entrare nessuno nella sua anima. I suoi educatori si lamentavano che non voleva studiare, e la sua anima era piena di sete di sapere. E apprendeva da Kapitonyc, dalla njanja, da Naden'ka, da Vasilij Lukic e non dai maestri. Quell'acqua che i padre e l'istitutore attendevano per le loro ruote, correva già da tempo e lavorava in altro luogo. 

Il padre punì Serëza col proibirgli di andare da Naden'ka, la nipote di Lidija Ivanovna; ma questa punizione si dimostrò favorevole per Serëza. Vasilij Lukic era di buon umore e gli mostrò come si fanno i mulini a vento. Passò tutta la sera lavorando e vagheggiando un mulino a vento fatto in modo che si potesse girarci sopra; afferrarsi con le mani alle ali o legarsi e girare. Alla madre Serëza non pensò per tutta la sera; ma, messosi a letto, se ne ricordò, a un tratto, e pregò con parole sue perché la mamma l'indomani, per il suo compleanno, smettesse di nascondersi e venisse da lui. 

- Vasilij Lukic, sapete per che cosa ho pregato in più, fuori del conto? 

- Per andare meglio nello studio? 

- No. 

- Per i giocattoli? 

- No. Non indovinerete. Per una cosa bellissima, ma è un segreto! Quando accadrà ve lo dirò. Non l'avete indovinato? 

- No, non l'indovinerò. Ditemelo voi - disse Vasilij Lukic sorridendo, il che accadeva di rado. - Via, coricatevi, spengo la candela. 

- Ma senza la candela per me è più chiaro quello che vedo e quello per cui ho pregato. Ecco che stavo quasi per dirvi il segreto - disse Serëza, ridendo allegramente. 

E come ebbero portato via la candela, Serëza udì e sentì sua madre. Era china su di lui e lo carezzava con uno sguardo pieno d'amore. Ma poi apparvero i mulini a vento, il coltellino, tutto si confuse ed egli si addormentò. 

\capitolo{XXVIII}\label{xxviii-3} 

Arrivati a Pietroburgo, Vronskij e Anna si erano fermati in uno dei migliori alberghi: Vronskij, a parte, nel piano di sotto, Anna sopra con la bambina, la balia e la cameriera, in un grande appartamento composto di quattro stanze. 

Fin dal primo giorno dell'arrivo Vronskij andò dal fratello. Là trovò sua madre che era venuta da Mosca per affari. La madre e la cognata lo accolsero come se nulla fosse; lo interrogarono sul suo viaggio all'estero, parlarono dei conoscenti comuni, ma neppure con una parola accennarono alla sua relazione con Anna. Il fratello poi, venuto il giorno dopo da Vronskij, di mattina, domandò di lei, e Aleksej Vronskij gli disse francamente che considerava la propria relazione con la Karenina come un matrimonio, che sperava di render possibile il divorzio e di sposarla, e che fino a quel momento la considerava come sua moglie, e lo pregava di riferire ciò alla madre e a sua moglie. 

- Se il mondo non approva ciò, a me è indifferente - disse Vronskij - ma se i miei parenti vogliono rimanere in rapporti di parentela con me, devono essere in questi stessi rapporti con mia moglie! 

Il fratello maggiore, che rispettava sempre le idee del minore, non sapeva bene se egli avesse ragione o no, finché il mondo non avesse risolto tale questione; lui stesso poi, da parte sua, non aveva nulla in contrario, e insieme con Aleksej andò da Anna. 

Vronskij, in presenza del fratello, come del resto in presenza di tutti, dava del voi ad Anna e la trattava come una conoscente intima, ma era sottinteso che il fratello conoscesse i loro rapporti, e si parlava del fatto che Anna andava nella proprietà di Vronskij. 

Malgrado tutta la sua esperienza mondana, Vronskij, in seguito alla nuova situazione in cui si trovava, era in uno strano errore. Sembrava che egli avrebbe dovuto capire come la società fosse preclusa a lui e ad Anna; adesso invece nella sua testa erano sorte certe nuove confuse considerazioni, che, cioè, così era soltanto nei tempi passati, ma che ora, col rapido progresso (senza accorgersene era divenuto seguace d'ogni progresso), il giudizio della società era mutato e che la questione, se essi sarebbero stati ricevuti in società, non era ancora risolta. ``Si intende - pensava - il mondo della corte non la riceverà, ma le persone intime possono e devono intendere ciò così come va inteso''. 

Si può restar seduti parecchie ore, incrociando le gambe, nella stessa posizione, quando si sa che nulla impedisce di cambiar posizione; ma se una persona sa che deve rimaner seduto così, con le gambe incrociate, gli verranno i crampi, le gambe si stireranno e si stringeranno in quel punto dove egli vorrebbe allungarle. Vronskij sperimentava questa stessa cosa rispetto alla società. Pur sapendo, in fondo all'animo, che la società era preclusa, voleva provare se questa si fosse mutata oppure no e se li avrebbe ricevuti. Si accorse, invece, subito, che la società era aperta a lui personalmente ed era preclusa ad Anna. Come nel giuoco del gatto e del topo, le braccia sollevate per lui, si abbassavano immediatamente dinanzi ad Anna. 

Una delle prime signore del gran mondo di Pietroburgo che Vronskij vide fu sua cugina Betsy. 

- Finalmente! - ella lo accolse con gioia. - E Anna? Come sono contenta! Dove siete? Immagino come vi sembri orribile la nostra Pietroburgo dopo il vostro delizioso viaggio; immagino la vostra luna di miele a Roma. E il divorzio? È stato fatto tutto? 

Vronskij notò che l'entusiasmo di Betsy diminuì quando seppe che il divorzio non c'era ancora. 

- Mi scaglieranno la pietra addosso, lo so - ella disse - ma verrò da Anna; sì, verrò assolutamente. Non vi fermerete molto qui? 

E realmente quello stesso giorno ella andò da Anna; ma il suo tono era ormai ben diverso da quello di prima. Era palesemente orgogliosa del proprio coraggio, e desiderava che Anna apprezzasse la fedeltà della sua amicizia. Rimase non più di dieci minuti parlando delle novità mondane, e nell'andarsene disse: 

- Non m'avete detto quand'è il divorzio. Del resto, io non mi faccio scrupoli, ma gli altri, quelli che alzano il bavero, quelli vi colpiranno con il loro gelo, finché non sarete sposati. E questo è così semplice adesso. Ça se fait. Così, allora, partite venerdì? Peccato che non ci vedremo più. 

Dal tono di Betsy, Vronskij avrebbe dovuto capire quello che doveva aspettarsi dalla società, ma fece ancora un tentativo nella propria famiglia. In sua madre non sperava. Sapeva che sua madre, entusiasta di Anna quando l'aveva conosciuta la prima volta, adesso era implacabile contro di lei perché era stata causa del dissesto nella carriera del figlio. Ma egli riponeva grandi speranza in Varja, la moglie del fratello. Gli sembrava che non avrebbe scagliato lei la prima pietra, e con semplicità e franchezza sarebbe andata da Anna e l'avrebbe ricevuta. 

Fin dal giorno dopo il suo arrivo, Vronskij andò da lei e, trovatala sola, espresse francamente il suo desiderio. 

- Tu sai, Aleksej - ella disse, dopo averlo ascoltato - come io ti voglia bene e come sia pronta a fare tutto per te; ma ho taciuto, perché sapevo che non posso essere utile a te e ad Anna Arkad'evna - ella disse, pronunciando con particolare sforzo ``Anna Arkad'evna''. - Ti prego di non credere che io voglia biasimare. Mai; forse io, al posto suo, avrei fatto lo stesso. Non entro e non voglio entrare in particolari - ella diceva, guardando timidamente il viso di lui. - Ma bisogna chiamar le cose col loro nome. Tu vuoi che io vada da lei, che la riceva e, con questo, la riabiliti in società; ma tu devi capire che io non posso far questo. Le figlie mi crescono, io devo vivere nella società per mio marito. Ammettiamo che io venga da Anna Arkad'evna; lei capirà che non posso invitarla a casa mia o che devo farlo in modo ch'ella non incontri le persone che pensano diversamente; questo offenderà lei stessa. Io non posso risollevarla\ldots{} 

- Ma io non considero ch'ella sia caduta più in basso di quanto non lo siano centinaia di donne che voi ricevete! - la interruppe ancora più cupo Vronskij, e si alzò in silenzio, avendo capito che la decisione della cognata era irremovibile. 

- Aleksej! non arrabbiarti con me. Ti prego, comprendimi, io non ne ho colpa - prese a dire Varja, guardandolo con un sorriso timido. 

- Io non sono arrabbiato con te - egli disse sempre cupo - ma ciò mi fa doppiamente male. Mi fa male anche perché rompe la nostra amicizia. Ammettiamo che non la rompa, ma la indebolisce. Tu pure comprenderai come, per me, questo non possa essere diversamente. - E con questo se ne andò via. 

Vronskij capì che ulteriori tentativi erano inutili e che bisognava passare quei pochi giorni a Pietroburgo come in un paese straniero, sfuggendo ogni rapporto col mondo di prima per non sottomettersi a dispiaceri e affronti, così tormentosi per lui. Una delle cose più spiacevoli della situazione di Pietroburgo era che Aleksej Aleksandrovic e il suo nome sembrava fossero dappertutto. Non si poteva cominciare a parlare di nulla, senza che il discorso non si aggirasse intorno ad Aleksej Aleksandrovic; non si poteva andare in nessun posto senza incontrar lui. Così almeno pareva a Vronskij, così come a un uomo che abbia un dito malato pare di urtar sempre, contro tutto, proprio con quel dito. 

La permanenza a Pietroburgo sembrava ancor più incresciosa a Vronskij perché in tutto quel tempo egli vedeva in Anna un umore nuovo, per lui incomprensibile. Ella, un momento, era innamorata di lui, un momento, diventava fredda, irritabile e impenetrabile. Qualcosa la tormentava, e qualcosa ella gli nascondeva e sembrava non notare neppure quelle offese che a lui avvelenavano l'esistenza e che per lei, per la sua finezza d'intuito, dovevano essere più tormentose ancora. 

\capitolo{XXIX}\label{xxix-3} 

Uno degli scopi del viaggio in Russia era per Anna l'incontro con il figlio. Dal giorno in cui era partita dall'Italia, questo pensiero dell'incontro non aveva cessato d'agitarla. E quanto più si avvicinava a Pietroburgo, tanto più grande le appariva la gioia e l'importanza di quest'incontro. Non si domandava neppure come sarebbe avvenuto l'incontro. Le sembrava naturale e semplice vedere il figlio, quando sarebbe stata nella stessa città dove era lui; ma giunta a Pietroburgo, a un tratto le si presentò chiara la sua posizione nella società, e capì che combinare l'incontro era difficile. 

Era a Pietroburgo già da due giorni. Il pensiero del figlio non l'aveva abbandonata neanche un attimo, eppure ancora non l'aveva visto. Andare direttamente in casa, dove poteva incontrarsi con Aleksej Aleksandrovic, sentiva di non averne il diritto. Avrebbero potuto impedirle di entrare e offenderla. Scrivere ed entrare in rapporti col marito, le era tormentoso anche il solo pensarlo: le riusciva d'essere calma solo quando non pensava al marito. Vedere il figlio alla passeggiata, dopo aver saputo dove e quando andava, le sembrava ben poca cosa: si era tanto preparata a quest'incontro, doveva dirgli tante cose, voleva tanto abbracciarlo e baciarlo. La vecchia njanja di Serëza poteva aiutarla e istruirla, ma non era più in casa di Aleksej Aleksandrovic. In queste esitazioni e nella ricerca della njanja erano passati questi due giorni. 

Avendo saputo degli intimi rapporti di Aleksej Aleksandrovic con la contessa Lidija Ivanovna, Anna, il terzo giorno, si decise a scriverle la lettera che le era costata tanta fatica, nella quale diceva, intenzionalmente, che il permesso di vedere il figlio doveva dipendere dalla generosità del marito. Ella sapeva che, se avessero mostrato la lettera al marito, egli, per sostenere la parte di uomo generoso, non glielo avrebbe rifiutato. 

L'inserviente che aveva portato la lettera, consegnò la risposta più crudele e inaspettata per lei, che, cioè, non ci sarebbe stata risposta. Non s'era mai sentita così umiliata come nel momento in cui, chiamato il fattorino, aveva sentito da lui il racconto particolareggiato di come egli avesse aspettato a lungo e di come poi gli avessero detto: ``Non c'è risposta''. Anna si sentiva umiliata, offesa, ma vedeva che, dal suo punto di vista, la contessa Lidija Ivanovna aveva ragione. Il suo dolore era tanto più forte in quanto era solo. Non poteva e non voleva farne partecipe Vronskij. Sapeva che per lui, anche se questa era la causa principale dell'infelicità di lei, la questione dell'incontro col figlio sarebbe apparsa come la cosa più insignificante. Sapeva che mai egli avrebbe avuto la forza di capire tutta la profondità della sua pena; sapeva che per quel tono freddo, col quale egli accennava alla cosa, avrebbe preso a odiarlo. E aveva paura di questo, più di qualunque altra cosa al mondo, e perciò gli nascondeva tutto quello che riguardava il figlio. 

Rimasta in casa tutto il giorno, aveva escogitato vari mezzi per vedersi col figlio e si era soffermata sulla decisione di rivolgersi al marito. Stava già scrivendo, quando le portarono la lettera di Lidija Ivanovna. Il silenzio della contessa l'aveva domata e sottomessa, ma la lettera, tutto quello ch'ella scorse fra le righe, la irritò a tal punto e così disgustoso le parve quel rancore di fronte alla sua appassionata tenerezza verso il figlio, che si rivoltò contro gli altri e cessò d'incolpare se stessa. 

``Questa freddezza è la finzione di un sentimento! - diceva a se stessa. - Hanno solo bisogno di offendere me e di tormentare il bambino, e io dovrei sottostare a loro! Per nulla al mondo! Lei è peggiore di me. Io almeno non mento''. E subito decise che l'indomani, giorno del compleanno di Serëza, sarebbe andata direttamente in casa del marito, avrebbe corrotto i servitori, avrebbe ingannato, ma a qualunque costo avrebbe visto il figlio, avrebbe spezzato l'inganno crudele nel quale avevano ravvolto il disgraziato fanciullo. 

Andò in un negozio di giocattoli, comprò tanti giocattoli e preparò il suo piano d'azione. Sarebbe andata la mattina presto, alle otto, quando Aleksej Aleksandrovic probabilmente ancora non s'era alzato. Avrebbe avuto in mano del denaro per il portiere e per il servo che l'avrebbero lasciata passare, e, senza sollevare il velo, avrebbe detto che veniva da parte del padrino di Serëza e che aveva l'incarico di lasciare i giocattoli accanto al letto del bambino. Solo le parole da dire al figlio non aveva preparato. Per quanto ci pensasse non riusciva a immaginare nulla. 

Il giorno dopo, alle otto del mattino, Anna discendeva sola da una carrozza da nolo e sonava all'ingresso principale della sua casa di un tempo. 

- Va' a vedere cosa vogliono. C'è una signora - disse Kapitonyc, non ancora vestito, in cappotto e soprascarpe, dopo aver guardato dalla finestra la signora, avvolta in un velo, che stava dritta proprio accanto alla porta. 

L'aiutante del portiere, un ragazzo che Anna non conosceva, le aveva appena aperto la porta che lei vi si era già infilata e, cacciato fuori dal manicotto un biglietto da tre rubli, glielo ficcò frettolosamente in mano. 

- Serëza\ldots{} Sergej Alekseic - ella pronunciò e voleva andare avanti. Dopo aver guardato il biglietto, l'aiutante del portiere la fermò all'altra porta di vetro. 

- Ma chi volete? - chiese. 

Ella non sentiva le sue parole e non rispondeva nulla. 

Avendo notato la confusione della sconosciuta, lo stesso Kapitonyc le venne incontro, la lasciò passare dalla porta e domandò cosa desiderasse. 

- Da parte del principe Skorodumov a Sergej Alekseic - ella disse. 

- Non è ancora alzato - disse il portiere, osservandola attentamente. 

Anna non si aspettava che l'atmosfera immutata dell'anticamera di quella casa, dove aveva vissuto nove anni, l'avrebbe impressionata con tanta violenza. Uno dietro l'altro i ricordi, felici e tormentosi, si sollevarono nell'animo suo, e per un attimo ella dimenticò perché si trovava là. 

- Volete aspettare? - disse Kapitonyc, togliendole la pelliccia. 

Tolta la pelliccia, Kapitonyc la guardò in viso, la riconobbe e in silenzio si inchinò profondamente. 

- Prego, eccellenza - disse. 

Ella voleva dire qualcosa, ma la sua voce si rifiutò di emettere un suono qualsiasi; dopo aver guardato il vecchio con l'implorazione di chi è colpevole, si avviò su per la scala a passi leggeri e svelti. 

Kapitonyc, tutto piegato in avanti e inciampando con le soprascarpe negli scalini, le correva dietro, cercando di oltrepassarla. 

- C'è il maestro là, forse non è vestito. Vado ad annunciare. 

Anna continuava ad andare per la scala nota, senza capire quello che il vecchio diceva. 

- Di qua, favorite a sinistra. Perdonate se non è pulito. Adesso è nella stanza dei divani di prima - diceva il portiere, riprendendo fiato. - Permettete, eccellenza, abbiate pazienza, do un'occhiata - disse e, sorpassatala, socchiuse una porta grande e scomparve dietro di essa. Anna rimase in attesa. - S'è appena svegliato - disse il portiere, uscendo di nuovo dalla porta. 

E in quel momento, mentre il portiere diceva questo, Anna sentì il suono di uno sbadiglio infantile. Dal solo suono di questo sbadiglio, Anna riconobbe Serëza e lo vide vivo dinanzi a sé. 

- Lascia, lascia, va' - cominciò a dire, e varcò la grande porta. A destra della porta c'era un letto e sul letto era seduto, sollevato, il fanciullo con la sola camicina sbottonata, e, col corpicino piegato, si stiracchiava, mentre finiva uno sbadiglio. Nel momento in cui le labbra si unirono, si disposero a un sorriso beato e sonnolento, e con questo sorriso egli di nuovo ricadde all'indietro, lentamente e con dolcezza. 

- Serëza! - ella mormorò, avvicinandosi a lui senza farsi sentire. 

Durante il distacco da lui e in quell'ondata d'amore che aveva provato in tutto quell'ultimo tempo, ella se lo era sempre figurato bambino, di quattro anni, così come più di tutto l'aveva amato. Ora egli non era neppure più come lo aveva lasciato; era ancor più lontano da quello di quattro anni, era cresciuto ancora e smagrito. Il viso era magro, i capelli erano corti. E come erano lunghe le braccia! Come era cambiato da quando l'aveva lasciato! Ma era lui, con la forma della testa, con le sue labbra, con il piccolo collo morbido e le spallucce larghe. 

- Serëza!- ella ripeté proprio sull'orecchio del bambino. 

Egli si sollevò sul gomito, girò la testa arruffata da tutte e due le parti, come cercando qualcosa e aprì gli occhi. Per alcuni secondi guardò in silenzio e interrogativamente la madre immobile su di lui; poi, d'un tratto, sorrise beato e, chiusi di nuovo gli occhi che non volevano stare aperti, si gettò giù, riverso, e non all'indietro, ma verso di lei, verso le sue braccia. 

- Serëza! bambino caro! - ella esclamò, soffocando e abbracciandogli il corpo morbido. 

- Mamma! - egli disse, movendosi sotto le sue braccia per sentirne il contatto con le varie parti del corpo. 

Sorridendo assonnato, sempre con gli occhi chiusi, l'afferrò, attraverso la spalliera del letto, per le spalle, con le manine paffute, si appoggiò a lei, inondandola di quel caro odore di sonno e tepore che hanno solo i bambini, e cominciò a fregarsi col viso al collo e alle spalle di lei. 

- Lo sapevo - disse, aprendo gli occhi. - Oggi è il mio compleanno. Lo sapevo che saresti venuta. Mi alzo subito. 

E dicendo questo, riprendeva sonno. 

Anna lo guardava avidamente, vedeva come era cresciuto e come si era cambiato nella sua assenza. 

Riconosceva e non riconosceva le sue gambe nude, così grandi ora che si erano liberate della coperta, riconosceva quelle guance smagrite, quei ricci tagliati, corti sulla nuca, dove ella lo baciava tanto spesso. Lo palpava tutto e non poteva dir nulla: le lacrime la soffocavano. 

- E perché piangi, mamma? - disse egli, svegliandosi completamente. - Mamma, perché piangi? - gridò con voce piagnucolosa. 

- Io? non piangerò più\ldots{} Piango di gioia. Non ti vedevo da tanto. Non lo farò più, non lo farò più - disse lei, inghiottendo le lacrime e voltandosi dall'altra parte. - Su via, è ora di vestirsi - aggiunse, rimessasi, e dopo un po' di silenzio e senza lasciare la mano di lui, sedette accanto al letto sulla sedia dove erano disposti i vestiti. 

- Come ti vesti senza di me? come\ldots{} - voleva cominciare a parlare semplicemente e allegramente, ma non poté, e di nuovo si voltò dall'altra parte. 

- Non mi lavo più con l'acqua fredda: papà non ha voluto. E Vasilij Lukic l'hai visto? verrà. Ma tu ti siedi sui miei vestiti! 

E Serëza scoppiò a ridere. Ella lo guardò e sorrise. 

- Mamma, mammina mia bella! - egli gridò, gettandosi di nuovo verso di lei e abbracciandola. Pareva che soltanto ora, dopo aver visto il sorriso di lei, avesse capito chiaramente quello che era successo. - Questo non ci vuole - disse, togliendole il cappello. E come se l'avesse scoperta di nuovo, senza cappello, si slanciò di nuovo ad abbracciarla. 

- Ma cosa pensavi mai di me? Pensavi che ero morta? 

- Non ci ho mai creduto. 

- Non ci credevi, piccolo mio? 

- Lo sapevo, lo sapevo! - ripeteva lui la sua frase preferita e, presale la mano che gli carezzava i capelli, cominciò a premerne la palma sulla bocca e a baciarla. 

\capitolo{XXX}\label{xxx-3} 

Vasilij Lukic, intanto, senza capire in principio chi fosse quella signora e capito poi, dai discorsi, che era la madre che aveva abbandonato il marito e che lui non conosceva perché assunto quando lei era andata via, era nel dubbio se entrare o no, o se comunicare la cosa a Aleksej Aleksandrovic. Considerato infine che il suo dovere consisteva nel fare alzare a una determinata ora Serëza, e che perciò non aveva ragione di distinguere chi fosse seduto là, la madre o altri, ma che pur bisognava compiere il proprio dovere, si vestì, si accostò alla porta e l'aprì. 

Ma le carezze della madre e del figlio, il suono delle loro voci e quello che dicevano, tutto questo gli fece cambiar idea. Scosse il capo e, dopo aver sospirato, chiuse la porta. ``Aspetterò ancora dieci minuti'' si disse, tossendo e asciugandosi le lacrime. 

Nello stesso tempo, fra la servitù di casa, era sorta grande agitazione. Tutti avevano saputo che era venuta la signora e che Kapitonyc l'aveva lasciata entrare, e che adesso era nella camera del bambino, e intanto, dopo le otto, il signore entrava lui nella camera del bambino, e tutti capivano che l'incontro dei coniugi era impossibile e che bisognava impedirlo. Kornej, il maggiordomo, sceso in portineria, aveva domandato chi l'avesse lasciata passare e come, e saputo che Kapitonyc l'aveva ricevuta e accompagnata, rimproverava il vecchio. Il portiere taceva ostinato, ma quando Kornej gli disse che per questo sarebbe stato scacciato di casa, Kapitonyc gli balzò addosso e, agitando le mani sotto al viso di Kornej, cominciò a dire: 

- E già, ecco, tu non l'avresti lasciata entrare! In dieci anni di servizio non hai visto altro che bontà e ora andresti a dirle: ``Prego, vada fuori''. Tu, solo la politica conosci bene! Proprio così! Pensa bene a te, che rubi al padrone e porti via le pellicce di vaio. 

- Soldataccio! - disse con sprezzo Kornej e si voltò verso la njanja che entrava. - Ecco, dite voi, Mar'ja Efimovna: l'ha lasciata entrare, non l'ha detto a nessuno - disse Kornej rivolto a lei. - Aleksej Aleksandrovic verrà fuori subito, andrà nella camera del bambino. 

- Dio mio, Dio mio! - diceva la njanja. - Voi, Kornej Vasil'evic, dovreste trattenerlo, il padrone, e io corro su; in qualche modo la porterò via. Dio mio, Dio mio! 

Quando la njanja entrò nella camera del bambino, Serëza stava raccontando alla madre come fosse caduto insieme a Naden'ka, rotolando dall'alto, e come avesse fatto tre capriole. Ella ascoltava il suono della sua voce, vedeva il viso e il giuoco d'espressione, sentiva la sua mano, ma non capiva quello che egli diceva. Doveva andar via, doveva lasciarlo, pensava e sentiva solo questo. Aveva sentito i passi di Vasilij Lukic che si era avvicinato alla porta e aveva tossito; aveva sentito i passi della njanja che s'era accostata, ma lei era rimasta a sedere, come impietrita, senza aver la forza di parlare, né di alzarsi. 

- Signora cara! - cominciò a dire la njanja, avvicinandosi ad Anna e baciandole le mani e le spalle. - Ecco che Dio ha portato una gioia al nostro festeggiato. Non siete cambiata per nulla. 

- Ah, njanja mia cara, non sapevo che foste in casa - disse Anna, riavutasi per un momento. 

- Non ci sto io. Sto con una mia figliuola, ma sono venuta a fare gli auguri. Anna Arkad'evna, amore mio! 

La njanja a un tratto si mise a piangere e cominciò di nuovo a baciarle la mano. 

Serëza, splendente negli occhi e nel sorriso e aggrappandosi con una mano alla madre e con l'altra alla njanja, pestava il tappeto con le gambette tornite. La tenerezza della njanja, ch'egli amava, verso la madre, lo mandava in visibilio. 

- Mamma, lei viene spesso da me, e quando verrà\ldots{} - cominciò a dire, ma si fermò avendo notato che la njanja aveva sussurrato qualcosa alla madre e che sul viso della madre s'erano espressi lo spavento e qualcosa di simile alla vergogna che le stava così male. 

Ella gli si avvicinò. 

- Mio caro! - disse. 

Non poteva dire ``addio'', ma l'espressione del suo viso lo disse ed egli capì. 

- Caro, caro Kutik - ella disse il nomignolo con cui lo chiamava da piccolo - non ti scorderai di me? Tu\ldots{} - non poté dire altro. 

Quante parole trovò poi, che avrebbe potuto dirgli! E adesso non sapeva e non poteva dire nulla. Egli capì ch'ella era infelice e che lo amava. Capì perfino quello che diceva la njanja sottovoce. Sentì le parole: ``dopo le otto'', e capì che questo era detto del padre, e che la madre non poteva incontrarsi con lui. Questo lo capiva, ma una cosa non poteva capire: perché sul suo viso apparivano lo spavento e la vergogna?\ldots{} Lei non era colpevole, ma aveva paura di lui e si vergognava di qualcosa. Egli voleva fare una domanda, che gli avrebbe chiarito questo dubbio, ma non osava farla: vedeva ch'ella soffriva e aveva pena di lei. Le si strinse in silenzio e disse sottovoce: 

- Non te ne andare ancora. Non verrà presto. 

La madre lo allontanò da sé, per capire s'egli pensava quello che diceva, e nell'espressione spaventata del viso di lui, lesse che non solo parlava del padre, ma le chiedeva cosa avrebbe dovuto pensare di lui. 

- Serëza, piccolo mio - ella disse - amalo, è più buono di me, e io sono colpevole davanti a lui. Quando sarai grande, giudicherai. 

- Più buono di te non c'è nessuno!\ldots{} - egli gridò disperato attraverso le lacrime e, presala per le spalle, cominciò a stringerla a sé con tutta la forza, tremando nelle braccia per lo sforzo. 

- Amore mio, piccolo mio! - sussurrò Anna e si mise a piangere anche lei, debolmente, infantilmente, come piangeva lui. 

Intanto una porta si aprì, entrò Vasilij Lukic. All'altra porta si sentirono dei passi e la njanja, con un sussulto spaventato, disse:``Viene'' e porse il cappello ad Anna. 

Serëza si lasciò andare giù sul letto e si mise a singhiozzare, coprendosi il viso con le mani. Anna tolse via quelle mani, baciò ancora una volta il viso bagnato e a passi rapidi si avviò verso la porta. Aleksej Aleksandrovic le veniva incontro. Vistala, egli si fermò e abbassò il capo. 

Malgrado ella avesse allora allora detto che era migliore di lei, nel rapido sguardo che gli lanciò, cogliendone la figura in tutti i particolari, un senso di repulsione, di rancore verso di lui e di invidia per il figlio l'afferrò. Con un movimento rapido ella abbassò il velo e, affrettato il passo, uscì quasi correndo dalla stanza. 

Non aveva fatto neanche in tempo a tirar fuori i giocattoli, e li riportò a casa così com'erano, quei giocattoli che il giorno prima aveva scelto nel negozio con tanto amore e tanta tristezza. 

\capitolo{XXXI}\label{xxxi-3} 

Per quanto Anna avesse fortemente desiderato l'incontro col figlio, per quanto da tempo ci avesse pensato e ci si fosse preparata non si aspettava in nessun modo che quest'incontro potesse sconvolgerla tanto. Rientrata nel suo solitario appartamento d'albergo, a lungo non riuscì a capire perché si trovasse là. ``Tutto è finito, sono di nuovo sola'' si disse e, senza togliersi il cappello, sedette in una poltrona accanto al camino. Posando gli occhi immobili sull'orologio di bronzo che stava sulla tavola in mezzo alle finestre, cominciò a pensare. 

La cameriera francese, assunta all'estero, entrò per proporle di vestirsi. Ella la guardò con sorpresa e disse: 

- Dopo. 

Un cameriere le offrì il caffè. 

- Dopo - ella disse. 

La nutrice italiana, vestita la bambina, entrò con lei e la porse ad Anna. La bambina paffuta, ben nutrita, come sempre, vista la madre, girò le manine nude protese, piccole, con le palme all'ingiù e, sorridendo con la piccola bocca sdentata, cominciò ad annaspare con le manine, come un pesce con le pinne, frusciando sulle pieghe inamidate della sottanina ricamata. Non si poteva non sorridere, non baciare la piccola; non si poteva non tenderle un dito al quale ella si aggrappò stringendo e sussultando in tutto il corpo; non si poteva non tenderle il labbro ch'ella afferrò nella piccola bocca a mo' di bacio. E tutto questo Anna lo fece: e la prese in braccio e la fece saltare e le baciò la gota fresca e i piccoli gomiti scoperti; ma, nel vedere quella bambina, le si chiariva ancor più che il sentimento che provava per lei non era neppure amore in confronto di quello che provava per Serëza. Tutto in quella bambina era grazioso, eppure, chi sa perché, non le prendeva il cuore. Nel primo bambino, sia pure avuto da un uomo non amato, erano state riposte tutte le forze dell'amore insoddisfatto; la bambina era stata messa al mondo nelle condizioni più difficili, e in lei non era stata riposta neppure la centesima parte delle cure che si erano prodigate al primo. Inoltre, nella bambina tutto era ancora attesa, mentre Serëza era già una personcina, una persona amata; in lui lottavano già il pensiero, il sentimento, egli capiva, amava, la giudicava - ella pensava - ricordando le parole e lo sguardo di lui. Ed era divisa per sempre non solo fisicamente, ma spiritualmente da lui, e a questo non si poteva porre rimedio. 

Ella diede la bambina alla nutrice, la congedò e aprì un medaglione in cui era il ritratto di Serëza quando aveva la stessa età della bambina. Si alzò e, toltasi il cappello, prese da sopra un tavolo un album in cui c'erano le fotografie del figlio in altre età. Voleva fare il confronto e cominciò a tirar fuori dall'album le fotografie. Le tirò fuori tutte. Ne rimase una, l'ultima, la migliore. Egli era a cavallo di una seggiola con una camicia bianca, aveva gli occhi accigliati e con la bocca sorrideva. Era un'espressione tutta sua, la migliore. Con le mani piccole e agili, che, proprio quel giorno si movevano con particolare tensione nelle dita bianche, sottili, ella afferrò varie volte l'angolo della fotografia, ma la fotografia si lacerava ed ella non riusciva a staccarla. Il tagliacarte non era sul tavolo, ed ella, con una fotografia che prese lì accanto (una fotografia di Vronskij in cappello tondo e con i capelli lunghi fatta a Roma), tirò fuori la fotografia del figlio. ``Sì, eccolo!'' disse, guardando la fotografia di Vronskij, e a un tratto ricordò chi era la causa del suo presente dolore. Non si era ricordata di lui neppure una volta in tutta quella mattinata. Ma adesso, a un tratto, visto quel volto virile, nobile, a lei noto e caro, sentì un inatteso afflusso d'amore per lui. 

``Ma dov'è mai? Come mai mi lascia sola con le mie pene?'' pensò a un tratto con un sentimento di rimprovero, dimenticando che lei stessa gli nascondeva tutto quello che riguardava il figlio. Mandò da lui a chiedergli di venire subito; lo aspettava col cuore che veniva meno, cercando le parole con cui avrebbe detto tutto, e le espressioni d'amore di lui che l'avrebbero consolata. La persona mandata tornò con la risposta che egli aveva un ospite, ma che sarebbe andato subito da lei; aveva ordinato di chiederle se poteva riceverlo col principe Jašvin che era arrivato a Pietroburgo. ``Non verrà solo e da ieri a pranzo non mi ha visto; non verrà in modo che io possa dirgli tutto, ma verrà con Jašvin''. E a un tratto le venne uno strano pensiero: e s'egli si fosse disincantato di lei? 

E, riandando agli avvenimenti di quegli ultimi giorni, le sembrava di vedere in tutto una conferma di questo pensiero spaventoso: nel fatto ch'egli il giorno prima avesse pranzato fuori casa, che avesse insistito perché a Pietroburgo prendessero dimora separatamente, e che perfino adesso non venisse da lei solo, quasi evitando un incontro a quattr'occhi. 

``Ma deve dirmelo. Mi occorre saperlo. Se lo saprò, allora saprò cosa fare'' diceva a se stessa senza aver la forza di figurarsi la situazione in cui si sarebbe trovata, dopo essersi convinta della indifferenza di lui. Pensava ch'egli si fosse disincantato di lei, si sentiva vicina alla disperazione, ma gliene veniva una particolare forma di eccitamento. Sonò per la cameriera e andò nel bagno. Nel vestirsi si occupò più che in tutti quei giorni del suo abbigliamento, come se egli, dopo essersi disincantato di lei, potesse di nuovo amarla per quel vestito o per quella pettinatura che più le si addicevano. 

Sentì suonare il campanello prima di essere pronta. 

Quando uscì in salotto, non lui, ma Jašvin le venne incontro con lo sguardo. Quanto a lui, stava osservando le fotografie del figlio ch'ella aveva dimenticato sul tavolo e non aveva premura di guardarla. 

- Ci conosciamo - ella disse ponendo la sua piccola mano nella mano enorme di Jašvin che si era turbato (cosa strana data la sua natura gigantesca e il suo viso rude). - Ci conosciamo dall'anno passato, alle corse. Date qua - ella disse prendendo a Vronskij, con un movimento rapido, le fotografie del figlio, ch'egli stava guardando e fissandolo significativamente con gli occhi scintillanti. - E quest'anno sono state belle le corse? Invece di queste, io ho visto le gare sul Corso a Roma. Ma a voi non piace la vita all'estero - ella disse, sorridendo affabile. - Io vi conosco e conosco tutti i vostri gusti, pur avendovi visto poco. 

- Questo mi spiace, perché i miei gusti sono per lo più cattivi - disse Jašvin, mordicchiandosi il baffo sinistro. 

Dopo aver parlato un po' e notato che Vronskij guardava l'orologio, Jašvin le chiese se sarebbe rimasta ancora a lungo a Pietroburgo e, raddrizzata la figura enorme, prese il berretto. 

- Non credo a lungo - ella disse con impaccio, dopo aver guardato Vronskij. 

- Così non ci vedremo più? - disse Jašvin, alzandosi e rivolgendosi a Vronskij. - Dove pranzi? 

- Venite da me - disse Anna decisa, quasi irritata con se stessa della propria agitazione, e arrossendo come sempre quando mostrava dinanzi a una persona nuova la propria posizione. - Qui non si mangia bene, ma almeno starete insieme. Fra tutti i compagni di reggimento, a nessuno Aleksej vuol tanto bene quanto a voi. 

- Ne sono molto contento - disse Jašvin con un sorriso, dal quale Vronskij capì che Anna gli era piaciuta molto. 

Jašvin salutò e uscì, Vronskij rimase indietro. 

- Vai anche tu? - ella disse. 

- Sono già in ritardo - egli rispose. - Avviati! Ti raggiungo subito! - gridò a Jašvin. 

Ella lo prese per mano e, senza abbassare gli occhi, lo guardò, cercando cosa dire per trattenerlo. 

- Aspetta, devo dirti qualcosa - e, presa la mano corta di lui, se la premette contro il collo. - Non fa nulla che l'ho invitato a pranzo? 

- Hai fatto benissimo - egli disse con un sorriso calmo, scoprendo i denti serrati e baciandole la mano. 

- Aleksej, non sei cambiato verso di me? - ella disse, stringendogli la mano con tutte e due le sue. - Aleksej, io mi tormento qui. Quando andiamo via? 

- Presto, presto. Non puoi credere come sia penosa anche per me la nostra vita qui - egli disse, e ritrasse la mano. 

- Sì, va', va' - ella disse offesa, e s'allontanò rapida da lui. 

\capitolo{XXXII}\label{xxxii-3} 

Quando Vronskij tornò a casa, Anna non c'era ancora. Gli dissero che, subito dopo di lui, era venuta una signora ed ella era uscita con lei. Il fatto che fosse uscita senza lasciar detto dove fosse andata e che finora non fosse rientrata, che la mattina fosse andata ancora in un altro posto senza dirgli dove, tutto questo, insieme all'espressione stranamente eccitata del viso e il ricordo del tono ostile con cui, dinanzi a Jašvin, gli aveva quasi strappato di mano le fotografie del figlio, lo costrinse a riflettere. Decise che era indispensabile spiegarsi con lei. E l'aspettò in salotto. Anna però non rientrò sola, ma portò con sé una sua zia, una vecchia zitella, la principessa Oblonskaja. Era lei che era venuta la mattina e con lei Anna era andata a far delle spese. Anna pareva non notare l'espressione del viso di Vronskij, preoccupato e interrogativo, e gli raccontava allegramente quello che aveva comprato la mattina. Egli vedeva che in lei avveniva qualcosa di particolare: negli occhi scintillanti, quando, come un lampo, si fermavano su di lui, c'era un'attenzione tesa, e nel discorso e nei movimenti vi erano quella velocità e grazia nervosa che, nel primo tempo della loro unione, lo avevano affascinato tanto e che ora lo rendevano inquieto e timoroso. 

Il pranzo era apparecchiato per quattro. Tutti si erano riuniti per passare nella piccola sala da pranzo quando giunse Tuškevic con un'ambasciata per Anna da parte della principessa Betsy. La principessa Betsy pregava di scusarla se non era venuta a salutare; non stava bene, e pregava Anna di andare da lei fra le sei e mezzo e le nove. Vronskij guardò Anna nel sentire questa delimitazione di tempo che mostrava ch'erano state prese le misure necessarie perché ella non incontrasse nessuno; ma Anna sembrava non averlo notato. 

- È un gran peccato che io non possa proprio fra le sei e mezzo e le nove - disse, sorridendo appena. 

- La principessa se ne rammaricherà molto. 

- Anche io. 

- Forse andate a sentire la Patti? - disse Tuškevic. 

- La Patti? Mi date un'idea. Ci andrei se potessi avere un palco. 

- Posso pensarci io - si offrì Tuškevic. 

- Vi sarei molto, molto grata - disse Anna. - Ma non volete pranzare con noi? 

Vronskij alzò le spalle in modo appena percettibile. Non capiva in nessun modo quel che stesse facendo Anna. Perché s'era portata dietro quella vecchia principessa, perché faceva rimanere a pranzo Tuškevic e, cosa più stupefacente di tutto, perché mandava costui a prendere un palco? Era mai possibile pensare di andare, nella sua posizione, ad una serata d'abbonamento in cui cantava la Patti, in cui ci sarebbe stata tutta la società che lei conosceva? La guardò con uno sguardo serio, ma lei gli rispose con quello stesso sguardo provocante, fra l'allegro e il disperato, di cui egli non riusciva ad afferrare il senso. A pranzo, Anna fu animata e gaia; pareva che civettasse con Tuškevic e con Jašvin. Quando si alzarono da tavola e Tuškevic andò a prendere il palco e Jašvin a fumare, Vronskij scese con lui nelle proprie stanze. Dopo essere rimasto un po', corse di sopra. Anna era già vestita d'un abito chiaro di seta con velluto, fatto fare a Parigi, col petto scoperto e un prezioso merletto bianco in testa che le inquadrava il viso e faceva risaltare in modo particolare la sua bellezza luminosa. 

- Andate proprio a teatro? - disse lui, cercando di non guardarla. 

- Perché me lo domandate con tanto sgomento? - ella disse, ancor più offesa ch'egli non la guardasse. - Perché non dovrei andarci? 

Pareva non capire il significato delle parole di lui. 

- S'intende, non c'è nessuna ragione - egli disse, accigliandosi. 

- Ecco, questo proprio dico anch'io - confermò lei, non raccogliendo, con intenzione, l'ironia del tono di lui e rimboccando tranquillamente il lungo guanto profumato. 

- Anna, in nome di Dio! cosa v'è successo? - chiese lui, cercando di scuoterla, proprio come un tempo aveva fatto il marito. 

- Non capisco cosa mi domandiate. 

- Lo sapete che non si può andare. 

- Perché? Non andrò sola. La principessa Varvara è andata a vestirsi, verrà con me. 

Egli si strinse nelle spalle con aria perplessa e desolata. 

- Ma non sapete forse\ldots{} - prese a dire. 

- Ma io non voglio saperlo! - ella gridò quasi. - Non voglio. Mi pento forse di quello che ho fatto? No, no e no. E se si ricominciasse daccapo, sarebbe lo stesso. Per noi, per me e per voi, una cosa sola è importante: se ci amiamo l'un l'altro. E altre considerazioni non ci sono. Perché viviamo qui separatamente e non ci vediamo? perché non posso andare? Io ti amo e per me tutto il resto è indifferente - ella disse in russo, guardandolo con uno scintillio particolare, per lui incomprensibile - se tu non sei cambiato. Perché non mi guardi? 

Egli la guardò. Vedeva tutta la bellezza del suo viso e dell'abbigliamento che sempre le si adattava così bene. Ma proprio quella sua bellezza ed eleganza lo irritavano, adesso. 

- Il mio sentimento non può cambiare, lo sapete, ma vi prego di non andare, vi supplico - egli disse di nuovo in francese con tenera preghiera nella voce, ma con freddezza nello sguardo. 

Ella non sentiva le parole, ma vedeva la freddezza dello sguardo, e rispose con irritazione: 

- E io vi prego di dirmi perché non devo andare. 

- Perché questo vi può arrecare quel\ldots{} - si confuse. 

- Non capisco niente. Jašvin n'est pas compromettant e la principessa Varvara non è per nulla peggiore delle altre. Ma ecco anche lei. 

\capitolo{XXXIII}\label{xxxiii-2} 

Vronskij provava, per la prima volta, contro Anna, un sentimento di collera, quasi di rancore per la sua meditata incomprensione della propria situazione. Questo sentimento era rafforzato dal fatto ch'egli non poteva esprimergliene la causa. S'egli le avesse detto francamente quello che pensava, le avrebbe detto: ``Apparire in teatro in questo abbigliamento con la principessa che tutti conoscono, significa non solo riconoscere la propria situazione di donna perduta, ma anche gettare una sfida al mondo, cioè rinunciarvi per sempre''. 

Egli non poteva dirle questo. ``Ma come può non capire ciò? ma cosa accade in lei?'' si diceva. Sentiva come, in uno stesso tempo, diminuisse la propria stima verso di lei e aumentasse la coscienza della sua bellezza. 

Tornò accigliato in camera sua e, sedutosi vicino a Jašvin, che aveva disteso le sue lunghe gambe su di una sedia e beveva cognac con acqua di seltz, si fece portare la stessa cosa. 

- Tu dici Mogucij di Lankovskij. È un buon cavallo e ti consiglio di comprarlo - disse Jašvin dopo aver guardato il viso scuro del compagno. - Ha il posteriore che pende, ma per le gambe e la testa non si può desiderare nulla di meglio. 

- Credo che lo comprerò - rispose Vronskij. 

Il discorso sui cavalli lo interessava, ma non dimenticava neppure per un attimo Anna, e prestava involontariamente orecchio al suono dei passi per il corridoio, guardando di tanto in tanto l'orologio sul camino. 

- Anna Arkad'evna ha ordinato di riferire che è andata a teatro. 

Jašvin, versato un altro bicchierino di cognac nell'acqua di seltz, bevve e si alzò, abbottonandosi. 

- Ebbene, andiamo? - egli disse, sorridendo appena sotto i baffi e facendo vedere con questo sorriso che capiva la ragione dell'umor nero di Vronskij, ma che non ci dava importanza. 

- Io non vado - rispose, torvo, Vronskij. 

- E io devo, ho promesso. Su, arrivederci. Ma tu vieni in poltrona, prendi la poltrona di Krasinskij - aggiunse Jašvin uscendo. 

- No, ho da fare. 

``Con la moglie preoccupazioni, con chi non è moglie, peggio ancora'' pensò Jašvin, uscendo dall'albergo. 

Vronskij, rimasto solo, si alzò dalla sedia e si mise a camminare per la stanza. 

``Ma oggi cos'è? La quarta in abbonamento\ldots{} C'è Egor con la moglie e la madre, probabilmente. Significa che tutta Pietroburgo è là. Adesso sarà entrata, si sarà tolta la pelliccia, sarà apparsa alla luce. Tuškevic, Jašvin, la principessa Varvara\ldots{} - egli si immaginava. - E io che faccio mai? Ho paura forse e ho affidato a Tuškevic di proteggerla? Comunque si consideri la cosa, ciò è sciocco, è sciocco\ldots{} Ma perché mi mette in questa posizione?'' disse facendo un gesto con la mano. 

Con questo movimento si impigliò nel tavolino su cui erano l'acqua di seltz e la caraffa con il cognac e per poco non lo buttò giù. Volle afferrarlo, lo fece cadere e, per la stizza, diede un calcio al tavolo e sonò. 

- Se vuoi rimanere al mio servizio - disse al cameriere che era entrato - ricordati il tuo dovere. Che questo non accada più. Porta via. 

Il cameriere, sentendosi innocente, voleva giustificarsi, ma, guardato il signore, capì dal viso che bisognava tacere e, scusandosi in fretta, si chinò sul tappeto e cominciò a separare le bottiglie e i bicchieri interi da quelli rotti. 

- Questo non è affar tuo, manda il servo e preparami il frac. 

Vronskij entrò in teatro alle otto e mezzo. Lo spettacolo era in pieno fervore. La maschera, un vecchietto, tolse la pelliccia a Vronskij e, riconosciutolo, lo chiamò ``eccellenza'' e gli propose di non prendere il biglietto, ma di chiamare semplicemente Fëdor. Nel corridoio illuminato non c'era nessuno, all'infuori della maschera e di due inservienti con le pellicce sul braccio, che ascoltavano vicino alla porta. Di là dalla porta socchiusa si sentivano i suoni precisi d'un accompagnamento staccato dell'orchestra e una voce femminile che pronunciava chiaramente una frase musicale. La porta si aprì, lasciando passare una maschera che scivolò via, e la frase, che si avviava alla fine, colpì chiaramente l'udito di Vronskij. Ma la porta si chiuse subito e Vronskij non sentì la fine della frase della cavatina, ma capì, dal tuono di applausi di là dalla porta, che la cavatina era finita. Quando egli entrò nella sala tutta illuminata dai lampadari e dai becchi a gas di bronzo, il fragore continuava ancora. Sul palcoscenico la cantante, con le spalle nude, splendenti di brillanti, chinandosi e sorridendo, raccoglieva, con l'aiuto del tenore che la teneva per mano, i mazzi di fiori che volavano goffamente di là dalla ribalta, e si avvicinava a un signore con la scriminatura in mezzo ai capelli lucidi di pomata che protendeva, con le braccia lunghe attraverso la ribalta, un qualche cosa. Tutto il pubblico in platea, come anche nei palchi, si agitava, si spingeva in avanti, gridava e applaudiva. Il direttore d'orchestra, sul podio, aiutava e faceva da tramite e si riaccomodava la cravatta bianca. Vronskij entrò nella corsia della platea e, fermatosi, cominciò a guardare in giro. Quel giorno meno che mai fece attenzione all'ambiente abituale, al brusio, a tutto quel noto, poco interessante, variopinto gregge di spettatori nel teatro pieno zeppo. 

Nei palchi c'erano determinate signore, sempre le stesse, con determinati ufficiali in fondo al palco; le stesse, Dio sa quali, variopinte donne, e poi divise, soprabiti e la stessa folla sporca in loggione; e in tutta quella folla, tra palchi e prime file, c'erano una quarantina di uomini e di donne veri. E su questa oasi Vronskij rivolse subito la sua attenzione e prese contatto con essa. 

L'atto finiva quand'egli era entrato, e perciò, senza passare nel palco del fratello, andò fino alla prima fila di poltrone e si fermò accanto al proscenio con Serpuchovskoj il quale, piegando un ginocchio e battendo col tacco sulla ribalta, l'aveva visto da lontano, e se l'era chiamato a sé con un sorriso. 

Vronskij non aveva visto ancora Anna, e deliberatamente non guardava dalla parte sua. Ma sapeva, dalla direzione degli sguardi, dove si trovava. Senza farsi notare, guardava in giro, ma non la cercava; aspettandosi il peggio, cercava con gli occhi Aleksej Aleksandrovic. Per sua fortuna questa volta Aleksej Aleksandrovic non era in teatro. 

- Quanto poco di militare è rimasto in te! - disse Serpuchovskoj. - Un diplomatico, un artista, ecco, qualcosa di simile. 

- Già, appena sono tornato a casa, ho messo il frac - rispose Vronskij, sorridendo e tirando fuori il binocolo con lentezza. 

- Ecco, in questo, lo riconosco, ti invidio. Quando torno dall'estero e metto queste - e toccò le cordelline dell'uniforme - rimpiango la libertà. 

Serpuchovskoj aveva rinunciato già da tempo a far rientrare in servizio Vronskij, ma gli voleva bene come prima, e adesso era particolarmente gentile con lui. 

- Peccato, sei arrivato tardi per il primo atto. 

Vronskij, ascoltando con un orecchio solo, portava il binocolo dal primo ordine di palchi al secondo. Accanto a una signora in turbante e a un vecchietto calvo, che ammiccava rabbiosamente nella lente del binocolo, Vronskij, a un tratto, vide la testa di Anna, superba, meravigliosamente bella e sorridente nella cornice dei merletti. Era nel quinto palco della prima fila a venti passi da lui. Era seduta e, voltandosi leggermente, diceva qualcosa a Jašvin. L'attaccatura della testa sulle spalle belle e larghe e lo splendore contenuto ed eccitato dei suoi occhi e di tutto il viso, gli ricordarono proprio l'Anna ch'egli aveva visto al ballo di Mosca. Ma, adesso, egli sentiva in tutt'altro modo quella bellezza. Nel suo sentimento per lei non c'era adesso più nulla di misterioso, e perciò la bellezza di lei, pur attraendolo più fortemente di prima, lo offendeva a un tempo. Ella non guardava dalla sua parte, ma Vronskij sentiva che lo aveva già visto. 

Quando Vronskij diresse di nuovo il binocolo verso il palco, notò che la principessa Varvara era particolarmente arrossata, rideva con affettazione e guardava di continuo nel palco vicino; Anna, invece, chiuso il ventaglio e battendo con esso sul velluto rosso, guardava chi sa dove, ma non vedeva, ed evidentemente non voleva vedere quello che avveniva nel palco accanto. Sul viso di Jašvin c'era l'espressione che soleva avere quando perdeva al giuoco. Accigliatosi, ficcava sempre più profondamente in bocca il baffo sinistro e guardava di sbieco il palco vicino. 

In quel palco, a sinistra, c'erano i Kartasov. Vronskij li conosceva e sapeva che Anna era in rapporti amichevoli con loro. La Kartasova, una donna magra, piccola, stava ritta nel palco e, voltando la schiena ad Anna, si metteva la mantella che il marito le tendeva. Aveva il viso pallido e irritato, e diceva qualcosa con agitazione. Kartasov, un uomo grasso, calvo, volgendosi continuamente a guardare Anna, cercava di calmare la moglie. Quando la moglie uscì, il marito si attardò a lungo, cercando con gli occhi lo sguardo di Anna, con l'evidente desiderio di salutarla. Ma Anna, mostrando ostentatamente di non accorgersi di lui, voltandosi indietro, diceva qualcosa a Jašvin, che si chinava su di lei con la testa dai capelli corti. Kartasov uscì senz'aver potuto salutare, e il palco rimase vuoto. 

Vronskij non capì cosa fosse precisamente accaduto fra i Kartasov e Anna, ma capì che era accaduto qualcosa di umiliante per Anna. Lo capì e da quello che aveva visto e più ancora dal viso di Anna la quale, egli lo sapeva, aveva raccolto le sue ultime forze per sostenere la parte intrapresa. E questa parte di calma esteriore le riusciva perfettamente. Quelli che non conoscevano lei e il suo ambiente, e che non avevano sentito tutte le espressioni di compassione, di indignazione e di sorpresa da parte delle signore, perché ella si era permessa di farsi vedere in società, e in modo così appariscente, in quella sua acconciatura di merletto e in tutta la sua bellezza, quelli ne ammiravano la calma e la bellezza, e non sospettavano ch'ella provasse la sensazione d'una persona esposta al palo dell'infamia. 

Sapendo che qualcosa era accaduto, ma non sapendo precisamente cosa, Vronskij provava un'ansia assillante e, sperando di riuscire a informarsi, andò nel palco del fratello. Scelto il passaggio proprio di fronte al palco di Anna, nell'uscire si scontrò con l'antico comandante di reggimento che parlava con due amici. Vronskij sentì che era stato pronunciato forte il nome di Karenin, e notò che il comandante del reggimento si era affrettato a chiamare ad alta voce Vronskij, dopo aver guardato significativamente quelli che parlavano. 

- Ah, Vronskij! E quando al reggimento? Noi non ti lasciamo andare senza un banchetto. Tu hai radici più antiche di tutti noi - disse il comandante. 

- Non farò in tempo, mi spiace, sarà per un'altra volta - disse Vronskij, e corse su per la scala, nel palco del fratello. La vecchia contessa, madre di Vronskij, era nel palco del fratello, coi suoi riccioli d'acciaio. Varja, con la principessina Sorokina, gli vennero incontro nel corridoio del primo ordine. 

Accompagnata la principessina Sorokina dalla madre, Varja diede il braccio al cognato e subito cominciò a parlare con lui di quello che lo interessava. Era così agitata come egli non l'aveva vista mai. 

- Io ritengo ciò basso e disgustoso, e m.me Kartasova non aveva alcun diritto. M.me Karenina\ldots{} - cominciò. 

- Ma cosa? Io non so. 

- Come, non hai sentito? 

- Capisci, io sarò l'ultimo a sentirlo. 

- C'è un essere più perfido di questa Kartasova? 

- Ma che cosa ha fatto? 

- Me l'ha raccontato mio marito\ldots{} Ha offeso la Karenina. Suo marito aveva cominciato a parlare con lei attraverso il palco, e la Kartasova gli ha fatto una scenata. Dicono che abbia detto qualcosa di offensivo e che sia uscita. 

- Conte, vostra maman vi chiama - disse la principessina Sorokina affacciandosi alla porta del palco. 

- E io non faccio che aspettarti - gli disse la madre, sorridendo con irrisione. - Non ti si vede mai. 

Il figlio vedeva ch'ella non riusciva a trattenere un sorriso di compiacimento. 

- Buona sera, maman. Venivo da voi - disse freddo. 

- Come mai non vai faire la cour à madame Karenine? - soggiunse, quando la principessina Sorokina si fu allontanata. - Elle fait sensation. On oublie la Patti pour elle. 

- Maman, vi prego di non parlarmi di questo - egli rispose accigliato. 

- Io dico quello che dicono tutti. 

Vronskij non rispose nulla e, dette poche parole alla principessina Sorokina, uscì. Sulla porta incontrò il fratello. 

- Ah, Aleksej! - disse il fratello. - Che bassezza! Una stupida e niente più\ldots{} Volevo andar da lei adesso. Andiamo insieme. 

Vronskij non lo ascoltava. Andò giù a passi svelti: sentiva di dover agire, ma non sapeva come. La stizza contro di lei, perché poneva se stessa e lui in una situazione così falsa, mista alla pietà per le sue pene, lo agitava. Scese in platea e si avviò verso il palco di prim'ordine di Anna. Vicino al palco stava in piedi Stremov e parlava con lei. 

- Tenori non ce ne sono più. Le moule en est brisé. 

Vronskij le fece un inchino e si fermò a salutare Stremov. 

- Mi sembra che siate venuto in ritardo e che non abbiate ascoltato l'aria più bella - disse Anna a Vronskij guardandolo ironicamente, così almeno gli parve. 

- Sono un cattivo intenditore - disse lui, guardandola severo. 

- Come il principe Jašvin - disse lei, sorridendo - il quale trova che la Patti canta troppo forte. Vi ringrazio - ella disse, dopo aver preso nella piccola mano dal guanto lungo il programma sollevato da Vronskij, e in quell'attimo, il suo bel viso ebbe un brivido. Ella si alzò e andò in fondo al palco. 

Avendo notato che per l'atto seguente il palco di lei era rimasto vuoto, Vronskij, suscitando gli zittii del teatro che aveva fatto silenzio ai suoni di una cavatina, uscì dalla platea e se ne andò a casa. 

Anna era già a casa. Quando Vronskij entrò da lei, ella era sola e nella stessa acconciatura che aveva a teatro. Sedeva sulla prima poltrona vicino al muro e guardava davanti a sé. Lo guardò e immediatamente riprese la posizione di prima. 

- Anna - egli disse. 

- Tu, tu hai colpa di tutto! - ella gridò con lacrime di disperazione e rancore nella voce, alzandosi. 

- Ti ho pregato, ti ho supplicato di non andare, lo sapevo che sarebbe stato spiacevole per te. 

- Spiacevole! - gridò lei. - Terribile! Per quanto vivrò non dimenticherò. Ha detto che è vergognoso sedere accanto a me. 

- Parole di una donna sciocca! - egli disse - ma perché rischiare, provocare?\ldots{} 

- Io odio la tua calma. Tu non avresti dovuto ridurmi a questo. Se tu mi amassi\ldots{} 

- Anna! che c'entra la questione del mio amore?\ldots{} 

- Sì, se tu mi amassi come ti amo io, se tu ti tormentassi come me\ldots{} - disse lei, guardandolo con un'espressione di spavento. 

Egli sentiva pietà per lei, e tuttavia stizza. La rassicurava del proprio amore perché vedeva che adesso soltanto quest'unica cosa poteva calmarla, e non la rimproverava a parole, ma nell'animo suo la rimproverava. 

E quelle assicurazioni d'amore, che a lui sembravano così volgari tanto da provar vergogna nel pronunciarle, lei le sorbiva e a poco a poco si calmava. Il giorno dopo, completamente rappacificati, partirono per la campagna. 

\parte{PARTE SESTA}\label{parte-sesta} 

\capitolo{I}\label{i-5} 

Dar'ja Aleksandrovna passava l'estate coi bambini a Pokrovskoe da sua sorella Kitty Levina. Nella sua proprietà la casa era crollata del tutto. Levin e sua moglie l'avevano convinta a passare l'estate da loro. Stepan Arkad'ic aveva approvato molto questa organizzazione. Diceva che gli spiaceva che l'ufficio gli impedisse di passare l'estate in campagna con la famiglia, cosa che per lui rappresentava la più grande felicità; intanto, rimanendo a Mosca, veniva di rado in campagna per un giorno o due. Oltre agli Oblonskij, con i bambini e la governante, era ospite quell'estate dei Levin anche la vecchia principessa, che riteneva suo dovere sorvegliare la figlia inesperta, in stato interessante. Inoltre Varen'ka, l'amica di Kitty a Soden, aveva mantenuto la promessa di recarsi da lei appena Kitty si fosse sposata, ed era ospite dell'amica. Questi erano tutti parenti e amici della moglie di Levin. Ed egli, pur volendo bene a tutti, si rammaricava un po' e per il suo mondo e per l'ordine delle cose ``leviniano'' che veniva soffocato da quell'inondazione dell'``elemento šcerbackiano'', come diceva lui. Dei suoi parenti era loro ospite, quell'estate, il solo Sergej Ivanovic, e anche costui non era un uomo dalla struttura leviniana, ma koznyševiana, così che lo spirito leviniano veniva completamente annientato. 

Nella casa dei Levin, così a lungo vuota, c'era, adesso, tanta gente che quasi tutte le stanze erano occupate, e quasi ogni giorno la vecchia principessa, sedendosi a tavola, doveva contar tutti e isolare il tredicesimo nipotino su di un tavolino a parte. E per Kitty, che si occupava accuratamente della casa, c'era non poco lavoro a procurarsi galline e tacchini e anitre, di cui, con l'appetito estivo degli ospiti e dei bambini, occorreva gran numero. 

Tutta la famiglia era seduta a tavola. I bambini di Dolly con la governante e Varen'ka, facevano progetti per andare a cogliere funghi. Sergej Ivanovic, che fra gli ospiti, per intelligenza e cultura, godeva di un prestigio che giungeva quasi all'adorazione, sorprese tutti con l'intervenire nel discorso sui funghi. 

- Prendete anche me con voi. Mi piace molto andare a cercar funghi. - disse, guardando Varen'ka; - penso che sia una bella occupazione. 

- Ma certo, siamo molto contenti - rispose Varen'ka, facendosi rossa. Kitty scambiò un'occhiata significativa con Dolly. La proposta di Sergej Ivanovic, persona di cultura e d'ingegno, di andare a cercar funghi con Varen'ka, confermava alcune supposizioni di Kitty, che negli ultimi tempi l'avevano molto interessata. Si affrettò a parlare con la madre, per non far notare la sua occhiata. Dopo pranzo Sergej Ivanovic sedette con la sua tazza di caffè accanto alla finestra nel salotto, continuando una conversazione incominciata col fratello e guardando di tanto in tanto la porta dalla quale dovevano venir fuori i bambini, pronti per andare a cercar funghi. Levin si accoccolò sulla finestra accanto al fratello. 

Kitty stava in piedi vicino al marito, evidentemente aspettando la fine della conversazione che non era interessante, per dirgli qualcosa. 

- Sei cambiato in molte cose da che sei sposato, e in meglio - disse, sorridendo a Kitty, Sergej Ivanovic, poco interessato evidentemente alla conversazione avviata - ma sei rimasto fedele alla tua passione di difendere temi paradossali. 

- Katja, non ti fa bene restare in piedi - le disse il marito, accostandole una seggiola e guardandola significativamente. 

- Eh sì, il resto a dopo, non c'è neanche tempo - soggiunse Sergej Ivanovic, vedendo i bambini venir fuori correndo. 

Innanzi a tutti, di fianco, a galoppo, con le sue calze tese, agitando un cestino e il cappello di Sergej Ivanovic, proprio verso di lui correva Tanja. 

Correndo verso Sergej Ivanovic, con gli occhi scintillanti, tanto simili ai bellissimi occhi del padre, ella porse a Sergej Ivanovic il cappello e fece finta di volerglielo mettere, mitigando la libertà che s'era presa con un sorriso timido e delicato. 

- Varen'ka aspetta - disse, mettendogli cauta il cappello, dopo aver visto dal sorriso di Sergej Ivanovic che si poteva. 

Varen'ka stava in piedi sulla porta, in abito di seta gialla, con un fazzoletto bianco annodato in capo. 

- Vengo, vengo, Varvara Andreevna - diceva Sergej Ivanovic, finendo di bere la tazza di caffè e distribuendo per le tasche il fazzoletto e il portasigari. 

- Ma che incanto è la mia Varen'ka! Eh? - disse Kitty al marito, non appena Sergej Ivanovic si fu alzato. Lo disse in modo che Sergej Ivanovic potesse sentirla, cosa che evidentemente desiderava. - E com'è bella, nobilmente bella! Varen'ka! - gridò Kitty - andate nel bosco, al mulino? Verremo anche noi da voi. 

- Tu decisamente dimentichi il tuo stato - prese a dire la vecchia principessa, uscendo in fretta sulla porta. - Non devi gridare così. 

Varen'ka, udita la voce di Kitty e il rimprovero della madre, si accostò a Kitty a passi rapidi e leggeri. La velocità del movimento, il colorito che le copriva il viso animato, tutto faceva vedere che in lei accadeva qualcosa di eccezionale. Kitty sapeva cos'era questo qualcosa di eccezionale, e la osservava attenta. Adesso, aveva chiamato Varen'ka solo per benedirla mentalmente, in vista di un avvenimento così importante che, secondo Kitty, doveva compiersi in quel giorno, dopo pranzo, nel bosco. 

- Varen'ka, sarò molto contenta se accadrà una certa cosa - disse sottovoce, baciandola. 

- E voi verrete con noi? - disse Varen'ka confusa a Levin, facendo finta di non aver sentito quello che le era stato detto. 

- Verrò solo fino all'aia e rimarrò là. 

- Ma che gusto ci provi? - disse Kitty. 

- Bisogna guardare e verificare i carri nuovi - disse Levin. - E tu dove resterai? 

- Sulla terrazza. 

\capitolo{II}\label{ii-5} 

Sulla terrazza s'era riunita tutta la compagnia delle donne di casa. In genere piaceva loro sedere là dopo pranzo, ma quel giorno c'era anche da fare. Oltre la confezione delle camicine e delle fasce a maglia, di cui tutte si occupavano, quel giorno si confezionava la marmellata secondo un metodo nuovo per Agaf'ja Michajlovna, senza, cioè, aggiungervi acqua. Kitty introduceva questo metodo nuovo usato in casa sua. Agaf'ja Michajlovna, alla quale in un primo momento era stato affidato questo lavoro, ritenendo che tutto quello che si faceva in casa Levin non potesse essere mal fatto, aveva, nonostante tutto, versato dell'acqua sulle fragole del giardino e su quelle selvatiche, convinta che era impossibile fare altrimenti; ma era stata sorpresa in questo, e ora si cuocevano i lamponi alla presenza di tutti. Agaf'ja Michajlovna avrebbe dovuto convincersi che, anche senza acqua, la marmellata sarebbe riuscita bene. 

Agaf'ja Michajlovna, col viso rosso e crucciato, i capelli arruffati e le braccia magre nude fino al gomito, faceva dondolare circolarmente la casseruolina sul braciere e guardava torva i lamponi, desiderando con tutta l'anima che si rappigliassero e che non finissero di cuocere. La principessa, sentendo che l'ira di Agaf'ja Michajlovna era diretta contro di lei, quale consulente più esperta nella cottura dei lamponi, cercava di aver l'aria di attendere ad altro e di non interessarsi affatto dei lamponi; parlava di cose estranee ma di tanto in tanto sbirciava il braciere. 

- Io compro sempre da me al mercato i vestiti per le ragazze - diceva la principessa, continuando un discorso incominciato.- Non si deve levare adesso la schiuma, amica mia! - soggiungeva rivolta ad Agaf'ja Michajlovna. - Ma non lo devi fare tu, fa caldo - disse, trattenendo Kitty. 

- Lo farò io - disse Dolly e, alzatasi, cominciò a passare col cucchiaio su per lo zucchero schiumoso, battendolo, di tanto in tanto, per staccarne quel che vi s'era appiccicato, sul piatto già coperto di schiuma variopinta giallorosa, con lo sciroppo sanguigno che colava sotto. ``Come se lo leccheranno col tè!'' pensava dei suoi bambini, ricordando come lei stessa, nell'infanzia, si stupiva che i grandi non mangiassero il meglio, la schiuma. 

- Stiva dice che è preferibile dare in denaro - continuava intanto Dolly l'argomento interessante incominciato sul modo migliore di far regali alla servitù - ma\ldots{} 

- Come si può dar denaro! - cominciarono a dire a una voce la principessa e Kitty. - Esse apprezzano questo. 

- Io, per esempio, l'anno scorso, ho comprato per la nostra Matrëna Semënovna non popeline, ma una specie - disse la principessa. 

- Mi ricordo, ce l'aveva al vostro onomastico. 

- Un disegno molto grazioso; così semplice e distinto. Me lo sarei fatto anch'io, se non l'avesse avuto lei. Una specie di quello di Varen'ka. Così carino e a buon prezzo. 

- Su, adesso mi pare che sia pronta - disse Dolly, facendo colar giù lo sciroppo dal cucchiaio. 

- Quando è caramellato, allora è pronto. Cuocete ancora, Agaf'ja Michajlovna. 

- Quante mosche! - disse, rabbiosa, Agaf'ja Michajlovna. - Sarà sempre lo stesso - aggiunse. 

- Ah, com'è carino, non lo spaventate! - disse inaspettatamente Kitty, guardando un passerotto che s'era posato sulla balaustrata e che, rigirato un rametto di lampone, aveva preso a beccarlo. 

- Sì, ma sta' lontana dal braciere - disse la madre. 

- A propos de Varen'ka - disse Kitty in francese, come del resto parlavano sempre perché Agaf'ja Michajlovna non capisse. - Voi sapete, maman, che oggi, chissà perché, mi aspetto una decisione. Voi capite quale. Come sarebbe bello! 

- Ma che abile pronuba! - disse Dolly. - Con che accortezza e con che abilità li mette insieme\ldots{} 

- No, ditemi, maman, cosa ne pensate? 

- Ma cosa pensare? Lui - ``lui'' era naturalmente Sergej Ivanovic - ha sempre potuto rappresentare il miglior partito della Russia; ora non è più tanto giovane, tuttavia, molte lo sposerebbero\ldots{} Lei è molto buona, ma egli potrebbe\ldots{} 

- No, mamma, capitemi, per lui e per lei non si può pensare nulla di meglio. Per prima cosa - disse Kitty piegando un dito - lei è una delizia! 

- Lei gli piace molto, questo è vero - confermò Dolly. 

- Poi, egli occupa una posizione tale in società che non ha assolutamente bisogno né del patrimonio né della posizione sociale della moglie. Ha bisogno di una cosa sola: di una moglie buona, simpatica, tranquilla. 

- Sì, con lei davvero si può esser tranquilli - confermò Dolly. 

- Terza cosa, che lei lo ami. E questo c'è\ldots{} Sarebbe così bello!\ldots{} Ecco, io aspetto che spuntino fuori dal bosco e che tutto si decida. Me ne accorgerò subito dagli occhi. Sarei così contenta. Che ne pensi, Dolly? 

- Ma tu non agitarti. Tu non devi agitarti per nulla - disse la madre. 

- Ma io non mi agito, mamma. Mi pare solo che quest'oggi egli farà la sua proposta. 

- Ah, è così strano, quando un uomo fa la sua dichiarazione\ldots{} C'è una certa barriera e a un tratto questa si rompe - disse Dolly, sorridendo pensierosa e ricordando il proprio passato con Stepan Arkad'ic. 

- Mamma, come vi ha fatto papà la sua domanda? - chiese a un tratto Kitty. 

- Non c'è stato nulla di straordinario, molto semplicemente - disse la principessa, ma il suo viso s'illuminò tutto al ricordo. 

- No, ma come? Lo amavate, però, prima che vi permettessero di parlare? 

Kitty provava un incanto particolare nel poter parlare adesso con la madre come ad una sua pari, di queste importanti questioni della vita di una donna. 

- S'intende che l'amavo; veniva da noi in campagna. 

- Ma come si decise la cosa, mamma? 

- Credete forse di aver inventato qualcosa di nuovo? È sempre la stessa cosa: si risolve con gli occhi, con il sorriso\ldots{} 

- Come l'avete detto bene, mamma! Proprio con gli occhi, con il sorriso - ripeté Dolly. 

- Ma quali parole diceva? 

- Quali ti diceva Kostja? 

- Lui le scriveva col gesso. Che cosa strana\ldots{} Come mi sembra lontano tutto ciò! - ella disse. 

E le tre donne cominciarono a pensare alla stessa cosa. Kitty ruppe il silenzio per prima. Le era venuto in mente l'ultimo inverno prima del suo matrimonio e la sua esaltazione per Vronskij. 

- Una cosa sola\ldots{} la passione precedente di Varen'ka - disse, ricordandosi di questo per naturale associazione di idee. - Volevo dirlo, in qualche modo, a Sergej Ivanovic, prepararlo. Loro, tutti gli uomini - aggiunse - sono terribilmente gelosi del nostro passato. 

- Non tutti - disse Dolly. - Tu giudichi da tuo marito. Ancora adesso lui si tormenta al ricordo di Vronskij. Non è vero, forse? 

- È vero - rispose Kitty sorridendo, pensosa, con gli occhi. 

- Ma io non so - disse la principessa in difesa della sua protezione materna sulla figlia - quale tuo passato potesse mai agitarlo! Forse il fatto che Vronskij ti avesse fatto la corte? Questo succede a ogni ragazza. 

- Ma non parliamo di questo - disse Kitty, arrossendo. 

- No, permetti - continuò la madre - e poi tu stessa non volevi che io parlassi a Vronskij. Ricordi? 

- Ah, mamma! - disse Kitty con un'espressione di sofferenza. 

- Adesso, non si riesce a trattenervi\ldots{} I tuoi rapporti non potevano andare oltre quel che si conviene; io stessa l'avrei richiamato. Inoltre a te, anima mia, non fa bene agitarti; non dimenticarlo, ti prego e sta' tranquilla. 

- Sono perfettamente calma, Ma. 

- Come fu bene per Kitty che venisse Anna! - disse Dolly - e come è stato male per lei! Ecco, proprio tutto al contrario! - aggiunse, colpita dalla propria idea. - Allora Anna era così felice e Kitty si considerava infelice! Come è ora tutto al contrario! Io penso spesso a lei. 

- Non hai a che pensare! una donna odiosa, ripugnante, senza cuore - disse la madre, che non poteva dimenticare che Kitty non aveva sposato Vronskij ma Levin. 

- Che gusto c'è a parlar di questo? - disse Kitty con dispetto - io non ci penso e non ci voglio pensare\ldots{} E non ci voglio pensare - disse, prestando orecchio ai passi noti del marito per la scala della terrazza. 

- E che cosa significa questo ``e non ci voglio pensare''? - chiese Levin, entrando sulla terrazza. 

Ma nessuno gli rispose, ed egli non ripeté la domanda. 

- Mi dispiace di aver disturbato il vostro regno femminile - disse, dopo aver osservato tutte, contrariato, e dopo aver capito che si stava parlando di una cosa che non avrebbero ripetuto a lui. 

Per un attimo sentì che condivideva il sentimento di Agaf'ja Michajlovna, lo scontento perché i lamponi si cuocevano senza acqua, e in genere l'influenza estranea degli Šcerbackij. Sorrise, tuttavia, e si accostò a Kitty. 

- Be', come va? - chiese, guardandola con quella stessa espressione con cui adesso tutti si volgevano a lei. 

- Niente, sto benissimo - disse sorridendo Kitty - e da te come si va? 

- Ne portano tre volte di più di un carro. Allora devo andare a prendere i bambini? Ho detto di attaccare. 

- Ma come! vuoi portare Kitty in calesse? - chiese la madre in tono di rimprovero. 

- Ma se va al passo, principessa. 

Levin non chiamava mai la principessa maman, come fanno i generi, e questo dispiaceva alla principessa. Levin, pur amando e rispettando la principessa, non poteva chiamarla così senza offendere il proprio sentimento verso la madre morta. 

- Venite con noi, maman - disse Kitty. 

- Non voglio assistere a queste sconsideratezze. 

- Via, andrò a piedi. Mi fa bene del resto. - Kitty si alzò, si avvicinò al marito e lo prese per mano. 

- Fa bene, ma tutto con misura - disse la principessa. 

- Be', Agaf'ja Michajlovna, è pronta la marmellata? - disse Levin, sorridendo ad Agaf'ja Michajlovna e desiderando rallegrarla. - Va bene col nuovo metodo? 

- Deve andar bene. Per noi è troppo cotta. 

- È anche meglio, Agaf'ja Michajlovna, non andrà a male, e qui da noi ormai il ghiaccio si è già sciolto e non si sa dove conservarla - disse Kitty, dopo aver capito subito l'intenzione del marito, e rivolgendosi alla vecchia col medesimo intento. - In compenso la vostra salamoia è così buona che la mamma dice di non averne mangiata in nessun posto una simile - soggiunse, sorridendo e accomodandole addosso lo scialletto. 

Agaf'ja Michajlovna guardò Kitty con rabbia. 

- Non mi consolate, signora. Ecco, vi guardo con lui e mi rallegro - ella disse, e questo rozzo modo di esprimersi commosse Kitty. 

- Venite con noi a cercar funghi, ci farete vedere i posti. - Agaf'ja Michajlovna sorrise, scosse il capo, quasi a dire: ``Sarei anche contenta d'arrabbiarmi con voi, ma non si può''. 

- Fate secondo il mio consiglio, per favore - disse la vecchia principessa; - sopra la marmellata mettete un pezzetto di carta e bagnatelo di rum: anche quando non sarà più freddo, non si formerà la muffa. 

\capitolo{III}\label{iii-5} 

Kitty fu particolarmente felice di trovarsi sola con suo marito, perché aveva notato come un'ombra di disappunto fosse passata sul suo viso, che rispecchiava tutto così vivacemente, nel momento in cui era entrato sulla terrazza e aveva domandato di che si parlava e non gli era stato risposto. 

Quando si avviarono a piedi, avanti agli altri, e sparvero dalla vista della casa sulla strada battuta, polverosa e cosparsa di spighe e di granelli di segala, ella si appoggiò fortemente al braccio di lui e lo strinse a sé. Egli aveva già dimenticato l'impressione spiacevole di un attimo prima e ora, solo con lei, mentre il pensiero della sua gravidanza non lo lasciava neppure per un momento, provava quella soddisfazione, ancora nuova e gioiosa per lui, del tutto scevra di sensualità, della vicinanza della donna amata. Non c'era nulla da dire, ma egli voleva ascoltare la voce di lei, come pure vederne lo sguardo, cambiatosi ora con la gravidanza. Nella voce, come pure nello sguardo c'erano una dolcezza e una serietà simili a quelli che hanno le persone continuamente concentrate in un'unica opera amata. 

- Allora non ti stancherai? Appoggiati di più - disse. 

- No, sono così contenta d'esser rimasta per caso con te; ti confesso che, per quanto bene stia con loro, rimpiango le nostre serate invernali a due. 

- Allora era bene, e adesso è ancora meglio. Tutte e due le cose sono buone - disse lui, stringendole il braccio. 

- Sai di che parlavamo quando sei entrato? 

- Della marmellata? 

- Sì, anche della marmellata; ma poi della maniera di far la dichiarazione. 

- Ah! - disse Levin, ascoltando più il suono della voce di lei che non le parole, badando tutto il tempo alla strada, che in quel punto attraversava un bosco, ed evitando di passare nei punti dove ella avrebbe potuto mettere un piede in fallo. 

- E di Sergej Ivanovic e Varen'ka. Hai fatto caso? Io lo desidero molto - ella proseguì: - Cosa ne pensi? - E lo guardò in viso. 

- Non so cosa pensare - rispose, sorridendo, Levin. - Sergej Ivanovic, sotto questo riguardo, è molto strano per me. Perché t'ho raccontato\ldots{} 

- Sì, che era innamorato di quella ragazza che è morta\ldots{} 

- Successe quando ero bambino; lo so per quel che m'hanno detto. Me lo ricordo allora. Era straordinariamente simpatico. Ma da allora lo osservo nei suoi rapporti con le donne: è gentile, alcune gli piacciono, ma senti che per lui sono semplicemente degli esseri, non delle donne. 

- Sì, ma adesso con Varen'ka\ldots{} pare che qualcosa ci sia\ldots{} 

- Può darsi che ci sia\ldots{} Ma bisogna conoscerlo\ldots{} È un uomo speciale, straordinario. Vive di sola vita spirituale. È un essere troppo puro e di animo elevato. 

- Come? Questo forse lo abbasserà? 

- No, ma è così abituato a vivere di sola vita spirituale, che non può riconciliarsi con la realtà, e Varen'ka, tuttavia, è la realtà. 

Levin, adesso, s'era già abituato a dire coraggiosamente il proprio pensiero, senza darsi la pena di rivestirlo di parole precise; sapeva che la moglie, in momenti così teneri come quello, avrebbe capito ciò ch'egli voleva dire da un accenno, ed ella lo capì. 

- Sì, ma in lei non c'è tanta realtà quanta in me; capisco come egli non mi amerebbe mai. Lei è tutta spirito. 

- Eh no, lui ti vuol così bene, e a me fa sempre tanto piacere che i miei te ne vogliano\ldots{} 

- Sì, egli è buono con me, ma\ldots{} 

- Ma non è così come col povero Nikolen'ka\ldots{} vi volevate bene l'un l'altro - terminò Levin. - Perché non parlarne? - aggiunse. - A volte, me lo rimprovero: si finisce col dimenticare. Ah, che uomo terribile e delizioso che era\ldots{} Sì, allora, di che stavamo parlando? - proseguì Levin, dopo un momento di silenzio. 

- Tu pensi dunque ch'egli non possa innamorarsi? - disse Kitty, traducendo nel suo linguaggio. 

- Non ch'egli non possa innamorarsi - disse Levin, sorridendo - ma non ha quella debolezza necessaria\ldots{} Io l'ho sempre invidiato e anche adesso, pur essendo così felice, tuttavia lo invidio. 

- Lo invidi perché non riesce a innamorarsi? 

- Lo invidio perché è migliore di me - disse Levin, sorridendo. - Egli non vive per sé. Tuttavia la sua vita è sottoposta al dovere. E lo invidio perché lui sì, che può essere tranquillo e soddisfatto. 

- E tu? - disse Kitty con un sorriso amabile, scherzoso. 

Ella non avrebbe potuto esprimere in nessun modo il corso dei pensieri che la faceva sorridere, ma l'ultima deduzione era questa, che suo marito esaltando il fratello e abbassando se stesso, non era sincero. Kitty sapeva che questa mancanza di sincerità derivava dall'amore per il fratello, da un senso di vergogna della propria felicità e, in particolare, dal desiderio, che non l'abbandonava mai, di essere migliore, e questo le piaceva in lui; perciò sorrideva. 

- E tu? Di che cosa mai sei scontento? - chiese lei, sempre con quel sorriso. 

L'incredulità di lei verso la sua insoddisfazione lo rallegrava ed egli la spingeva inconsciamente a manifestare le ragioni della propria incredulità. 

- Io sono felice, ma scontento di me\ldots{} - egli disse. 

- Allora come puoi mai esser scontento, se sei felice? 

- Cioè, come dirti? Io sinceramente non desidero nulla di più, tranne che tu non inciampi. Ah, ma non si può mica saltare così - egli interruppe il suo dire con un rimprovero per il movimento troppo rapido ch'ella aveva fatto nel saltare un ramo che si trovava sul viottolo. - Ma quando mi giudico e mi paragono agli altri, in particolare a mio fratello, allora sento di essere cattivo. 

- Ma in che cosa? - continuò Kitty con un sorriso. - Non lavori anche tu per gli altri? E la tua azienda, la tua fattoria, e il tuo libro? 

- No, lo sento in modo particolare adesso: tu ne hai colpa - egli disse, stringendole il braccio - che le cose non vadano bene. Lavoro così, alla leggera. Se io potessi amare tutto questo lavoro come amo te\ldots{} invece in quest'ultimo tempo, lo faccio come una lezione assegnata. 

- E allora cosa dirai mai di papà? - chiese Kitty. - Che anche lui è cattivo, perché non ha fatto nulla per il lavoro comune? 

- Lui? no. Bisogna avere quella semplicità, quella chiarezza, quella bontà di tuo padre e io le posseggo forse? Io non faccio nulla e mi tormento. Tutto questo l'hai fatto succedere tu. Quando tu non c'eri e non c'era codesto - egli disse guardando i fianchi di lei in modo ch'ella capisse - io mettevo tutte le mie energie nel lavoro; invece ora non posso e me ne vergogno; lo faccio proprio come una lezione assegnata, fingo\ldots{} 

- Su, via, e vorresti fare il cambio con Sergej Ivanyc - disse Kitty. - Vorresti compiere questo lavoro comune e amare questa lezione assegnata come lui, e basta? 

- S'intende che no - disse Levin. - Del resto, io sono così felice che non capisco nulla. Allora tu credi che oggi egli faccia addirittura la sua proposta di matrimonio? - egli soggiunse dopo aver taciuto un po'. 

- Lo credo e non lo credo. Lo desidero tanto. Ecco, aspetta. - Ella si chinò e strappò dal ciglio della strada una margherita selvatica. - Su, conta: sarà o non sarà - ella disse, dandogli il fiore. 

- Sarà, non sarà - diceva Levin, strappando i petali bianchi, stretti e scanalati. 

- No, no! - lo fermò, afferrandolo per la mano, Kitty che aveva seguito con agitazione il movimento delle dita. - Ne hai strappati due insieme. 

- Su, in compenso, questo piccolo qui non conta - disse Levin, strappando un petalo corto non cresciuto. - Ecco, anche il calesse ci ha raggiunto. 

- Non sei stanca, Kitty? - gridò la principessa. 

- Per nulla. 

- Altrimenti sali su, i cavalli sono tranquilli e vanno al passo. 

Ma non valeva la pena di salir su, erano quasi arrivati e tutti andarono a piedi. 

\capitolo{IV}\label{iv-5} 

Varen'ka, con un fazzoletto bianco sui capelli neri, circondata dai bambini di cui si occupava con allegra cordialità, visibilmente agitata dalla possibilità di una spiegazione con un uomo che le piaceva, era molto attraente. Sergej Ivanovic le camminava a fianco e non cessava d'ammirarla. Guardandola, ricordava tutti i discorsi di lode che aveva sentito su di lei, tutto quello che di buono sapeva di lei, e riconosceva sempre più che il sentimento ch'egli provava per lei era qualcosa di particolare, da lui provato tanto e tanto tempo fa e solamente una volta, nella prima giovinezza. Il senso di gioia per la vicinanza di lei, divenendo sempre più forte, giunse al punto che, porgendole nel cestino un enorme fungo prugnolo dalle estremità accartocciate, col gambo sottile, la guardò negli occhi e, notando che un rossore di gioiosa e spaventata agitazione le ricopriva il viso, si confuse egli stesso e le sorrise in silenzio con un sorriso fin troppo eloquente. 

``Se è così - egli disse - devo riflettere e decidere, e non abbandonarmi, come un ragazzo, all'esaltazione d'un momento''. 

- Adesso andrò a cercar funghi da solo, altrimenti le mie conquiste non si notano - disse, e si avviò, solo, dall'estremità del bosco, dove camminavano sull'erba bassa e morbida, fra le betulle vecchie e rade, verso il folto del bosco, dove, in mezzo ai tronchi bianchi delle betulle, apparivano i fusti grigi delle tremule e i cespugli scuri dei nocciuoli. Allontanatosi di una quarantina di passi e oltrepassato un cespuglio di fusaro in pieno fiore, con le pannocchie color rosa, Sergej Ivanovic, sapendo di non essere visto, si fermò. Intorno a lui c'era un silenzio assoluto. Solo in cima alle betulle sotto alle quali si trovava, ronzavano incessantemente le mosche come uno sciame di api, e di tanto in tanto giungevano le voci dei bambini. Ad un tratto, non lontano dal limitare del bosco, risonò la voce di contralto di Varen'ka, che chiamava Griša, e un sorriso gioioso apparve sul viso di Sergej Ivanovic. Avuta la percezione di questo sorriso, Sergej Ivanovic scosse il capo, disapprovando la propria situazione e, tirato fuori un sigaro, prese ad accenderlo. Per un pezzo non riuscì ad accendere il fiammifero contro il tronco di una betulla. Lo strato sottile della corteccia bianca aderiva al fosforo e la fiamma si spegneva. Alla fine uno di questi fiammiferi si accese, e il fumo odoroso del sigaro, come un largo drappo ondeggiante, si allungò in forme precise, al di sopra del cespuglio e oltre, sotto i rami pendenti della betulla. Seguendo con gli occhi la striscia di fumo, Sergej Ivanovic si avviò con passo tranquillo, riflettendo alla propria posizione. 

``E perché no? - egli pensava. - Se questo fosse un capriccio o una passione, se provassi solo una simpatia, una simpatia reciproca (posso ben dire reciproca), e sentissi che non va d'accordo con tutto il mio ordine di vita, se sentissi, abbandonandomi a questa simpatia, di tradire la mia vocazione e il mio dovere\ldots{} ma questo non è. L'unica cosa che posso dire in contrario è che, avendo perduto Marie, mi dicevo che sarei rimasto fedele alla sua memoria. Questa sola cosa posso dire contro il mio sentimento\ldots{} È importante - diceva Sergej Ivanovic, sentendo nello stesso tempo che tale considerazione, per lui personalmente, non poteva avere nessuna importanza, salvo forse a sciupare agli occhi degli altri la propria parte poetica. - Ma al di fuori di questo, per quanto io cerchi, non trovo nulla da poter dire contro il mio sentimento. Se avessi scelto con la sola ragione, non avrei potuto trovar di meglio''. 

Per quante donne e fanciulle note ricordasse, non poteva ricordare una ragazza che riunisse fino a tal punto tutte le qualità ch'egli, ragionando freddamente, desiderava di vedere in sua moglie. Ella aveva tutto l'incanto e la freschezza della giovinezza, ma non era una fanciulla, e se lo amava, lo amava coscientemente, come deve amare una donna: e questa era già una cosa. Un'altra: ella era non solo lontana dalla mondanità, ma, evidentemente, aveva il disprezzo del mondo, e nello stesso tempo lo conosceva, e aveva tutte le maniere della donna di buona società, senza le quali per Sergej Ivanovic era inconcepibile la compagna della propria vita. Terza cosa: ella era religiosa, ma non religiosa e buona senza rendersene conto, come può esserlo un bambino, come, per esempio, era Kitty; la sua vita era fondata su convinzioni religiose. Perfino nei particolari Sergej Ivanovic trovava in lei tutto quello che avrebbe desiderato in una moglie: era povera e sola, sicché non avrebbe portato con sé un nugolo di parenti e la loro influenza in casa del marito, come egli vedeva nel caso di Kitty; ma avrebbe dovuto tutto al marito, cosa ch'egli aveva sempre desiderato per la propria futura vita familiare. E questa ragazza che riuniva in sé queste qualità, lo amava. Egli era modesto, ma non poteva non accorgersene. Anch'egli l'amava. Sola considerazione sfavorevole era la propria età. Ma la sua razza era longeva, egli non aveva neppure un capello bianco, nessuno gli dava quarant'anni, e ricordava che Varen'ka aveva detto che soltanto in Russia gli uomini a cinquant'anni si considerano vecchi, mentre in Francia un uomo di cinquant'anni si considera dans la force de l'âge, e uno di quaranta, un jeune homme. Ma che significava il conto degli anni, quand'egli si sentiva giovane d'animo, come vent'anni prima? Non era forse giovinezza il sentimento che provava ora, quando, uscito di nuovo sul limitare del bosco, dall'altra parte, aveva visto, nella luce viva dei raggi obliqui del sole, la figura graziosa di Varen'ka, col vestito giallo e il cestello, camminare con passo leggero accanto al tronco di una vecchia betulla, e quando questa impressione della vista di Varen'ka si era fusa in uno con la vista d'un campo di avena gialla, splendido, inondato dai raggi obliqui, e, di là del campo, un vecchio bosco lontano, screziato d'oro, che si perdeva nella lontananza azzurra? Il cuore gli si strinse per la gioia. Un senso di commozione lo afferrò. Sentì che si era deciso. Varen'ka, che si era appena abbassata per cogliere un fungo, con un movimento agile si levò e si voltò a guardarlo. Gettato via il sigaro, Sergej Ivanovic si diresse verso di lei a passi decisi. 

\capitolo{V}\label{v-5} 

``Varvara Andreevna, quando ero ancora molto giovane mi sono formato l'ideale della donna che avrei voluto amare ed essere felice di chiamare mia moglie. Ho vissuto una lunga vita e, adesso, per la prima volta ho incontrato in voi quello che cercavo. Vi amo e vi offro la mia mano''. 

Sergej Ivanovic si diceva questo, mentre era già a dieci passi da Varen'ka. In ginocchio e difendendo con le mani un fungo da Griša, Varen'ka chiamava la piccola Maša. 

- Qua, qua piccoli! ce ne sono tanti! - ella diceva con la sua simpatica voce di petto. 

Scorto Sergej Ivanovic che si avvicinava, non si levò e non cambiò posizione; ma tutto a lui diceva ch'ella sentiva il suo avvicinarsi e che ne era felice. 

- Ebbene, avete trovato qualcosa? - ella chiese, volgendo verso di lui, di là dal fazzoletto bianco, il suo bel viso dolcemente sorridente. 

- Neppure uno - disse Sergej Ivanovic. - E voi? 

Ella non gli rispose, occupata con i bambini che la circondavano. 

- Ancora questo, accanto al ramo - ed ella mostrò alla piccola Maša una minuscola rossola tagliata per traverso, nel suo piccolo, morbido cappello rosa, da un filo d'erba secco, da sotto al quale si liberava. Varen'ka si alzò quando Maša, spezzatala in due metà bianche, ebbe tirata su la rossola. - Questo mi ricorda l'infanzia - ella soggiunse, allontanandosi dai bambini a fianco di Sergej Ivanovic. 

Fecero alcuni passi in silenzio. Varen'ka vedeva ch'egli voleva parlare; indovinava di che cosa e veniva meno per l'agitazione, dovuta alla felicità e al timore. Erano ormai così lontani che nessuno avrebbe potuto sentirli; egli tuttavia non cominciava a parlare. Per Varen'ka era meglio tacere. Dopo un silenzio era più facile dire quello ch'essi volevano, che non dopo i discorsi sui funghi; ma contro la sua volontà, come per caso, Varen'ka disse: 

- Allora non avete trovato nulla? Del resto, in mezzo al bosco, ce n'è sempre meno. 

Sergej Ivanovic sospirò e non rispose nulla. Era urtato che si fosse messa a parlare di funghi. Voleva ricondurla alle prime parole ch'ella aveva dette sulla sua infanzia, ma come contro la sua volontà, dopo un po' di silenzio, egli stesso fece un'osservazione sulle ultime parole di lei. 

- Ho sentito dire che solo gli ovoli sono di preferenza ai margini, sebbene io non sappia distinguere un ovolo. 

Passarono ancora alcuni minuti, essi erano andati ancora più lontani dai bambini ed erano completamente soli. Il cuore di Varen'ka batteva tanto ch'ella ne udiva i colpi e sentiva di arrossire, d'impallidire e di arrossire di nuovo. 

Essere la moglie di un uomo come Koznyšev, dopo la sua posizione presso la signora Stahl, le appariva il colmo della felicità. Inoltre era quasi sicura d'esserne innamorata. E ora questo avrebbe dovuto decidersi. Ella ne aveva timore. Era terribile e quello ch'egli avrebbe detto e quello che non avrebbe detto. 

Bisognava spiegarsi adesso o mai più: questo Sergej Ivanovic lo sentiva. Tutto, nello sguardo, nel colorito acceso, negli occhi bassi di Varen'ka tutto rivelava un'aspettazione morbosa. Sergej Ivanovic lo vedeva e aveva pena per lei. Sentiva perfino che non dir nulla, adesso, significava offenderla. Si ripeteva anche le parole con cui voleva esprimerle la sua proposta; ma, invece di queste parole, per un certo pensiero inatteso che gli venne in mente, chiese a un tratto. 

- E che differenza c'è tra un ovolo e un prugnolo? 

Le labbra di Varen'ka tremarono per l'agitazione quando rispose: 

- Nel cappello non c'è quasi differenza, ma nel gambo sì. 

E non appena queste parole furono dette, lui e lei capirono che tutto era finito, che quello che si doveva dire non si sarebbe detto, e la loro agitazione, che prima di questo aveva raggiunto l'acme, cominciò a placarsi. 

- Il fungo prugnolo ricorda nel gambo la barba non rasata di due giorni di un uomo bruno - disse, ormai calmo, Sergej Ivanovic. 

- Già, è vero - rispose sorridendo Varen'ka, e involontariamente la direzione della loro passeggiata cambiò. Cominciarono ad avvicinarsi ai bambini. Varen'ka provava pena e vergogna, ma nello stesso tempo anche un senso di sollievo. 

Ritornando a casa ed esaminando tutti gli argomenti, Sergej Ivanovic scoprì che non aveva ragionato in modo giusto. Non poteva tradire la memoria di Marie. 

- Piano, bambini, piano! - gridò perfino sgarbato Levin, ponendosi davanti alla moglie per difenderla, quando la folla dei bambini volò loro incontro con un grido di gioia. 

Dopo i bambini, uscirono dal bosco Sergej Ivanovic e Varen'ka. Kitty non ebbe bisogno di interrogare Varen'ka: dall'espressione calma e un po' vergognosa di tutti e due i visi capì che i suoi programmi non si erano avverati. 

- Su, ebbene? - le domandò il marito nel tornare di nuovo a casa. 

- Non attacca - disse Kitty, rassomigliando al padre nel sorriso e nel modo di parlare, cosa che Levin notava spesso in lei con piacere. 

- Come non attacca? 

- Ecco, così - disse lei, prendendo la mano del marito, portandola alla bocca e toccandola con le labbra chiuse. - Come si bacia la mano al vescovo. 

- E chi è che non attacca? - disse lui, ridendo. 

- Tutti e due. E deve esser così\ldots{} 

- Vengono i contadini\ldots{} 

- No, non hanno visto. 

\capitolo{VI}\label{vi-5} 

Durante il tè dei bambini, i grandi erano seduti sul balcone e conversavano come se nulla fosse accaduto, sebbene tutti, e in particolare Sergej Ivanovic e Varen'ka, sapessero molto bene che era accaduto un avvenimento sia pure negativo, ma molto importante. Tutti e due provavano una sensazione identica, simile a quella che prova uno scolaro dopo un esame andato male, per cui o deve ripetere o viene per sempre escluso dall'istituto. Tutti i presenti, sentendo pure che era successo qualcosa, parlavano animatamente di argomenti estranei. Levin e Kitty si sentivano particolarmente felici e pieni d'amore, quella sera. E il fatto che fossero felici del loro amore, racchiudeva in sé un'allusione spiacevole per coloro che sentivano la stessa cosa e non vi erano riusciti, e se ne vergognavano. 

- Ricordate le mie parole: Alexandre non verrà - disse la vecchia principessa. 

Quella sera si aspettava l'arrivo col treno di Stepan Arkad'ic e il vecchio principe scriveva che forse anche lui sarebbe venuto. 

- E io lo so perché - continuò la principessa - egli dice che i giovani sposi devono essere lasciati soli, nei primi tempi. 

- Ma anche così papà ci ha lasciato. Non l'abbiamo più visto - disse Kitty. - E che giovani siamo mai? Siamo già così vecchi! 

- Se lui non verrà, vi saluterò anch'io, ragazzi - disse la principessa, dopo aver sospirato con tristezza. 

- Ma cosa avete, mamma! - l'investirono le due figlie insieme. 

- Ma pensa, chi sa come sta? Del resto ora\ldots{} 

E a un tratto, del tutto inaspettatamente, la voce della vecchia principessa tremò. Le figlie tacquero e si guardarono: ``Mamma ha sempre qualcosa di triste'' si dissero con lo sguardo. Non sapevano che, per quanto bene si trovasse la principessa dalla figlia, per quanto necessaria vi si sentisse, provava una tormentosa tristezza e per sé e per il marito da che avevano maritato l'ultima figlia prediletta e il nido familiare era rimasto vuoto. 

- Di che cosa avete bisogno, Agaf'ja Michajlovna? - chiese a un tratto Kitty ad Agaf'ja Michajlovna che si era fermata con un'aria misteriosa e un viso significativo. 

- Per la cena. 

- Sì, ecco, va benissimo - disse Dolly - tu va' a dare gli ordini, e io andrò con Griša a risentirgli la lezione. Se no, quest'oggi, non ha fatto nulla. 

- Ma questa lezione è per me! No, Dolly, andrò io - disse Levin, scattando. 

Griša, che era entrato al ginnasio, d'estate doveva ripetere le lezioni. Dar'ja Aleksandrovna, che già a Mosca aveva studiato insieme col figlio il latino, venuta dai Levin s'era imposta come regola di ripetere con lui, almeno una volta al giorno, le materie più difficili, aritmetica e latino. Levin si era offerto di sostituirla, ma la madre, ascoltando una volta la lezione di Levin e notando che non era simile alla ripetizione dell'insegnante di Mosca, confusamente e cercando di non offendere Levin, gli aveva detto con risolutezza che bisognava studiare secondo il testo, come faceva l'insegnante, e che piuttosto l'avrebbe fatto di nuovo lei. Levin era urtato e contro Stepan Arkad'ic, perché per la sua incuria non lui, ma la madre si occupava di sorvegliare gli studi di cui non capiva nulla, e contro i professori che insegnavano così male ai ragazzi; ma aveva promesso alla cognata di far lezione così come voleva lei. E aveva continuato a studiare con Griša non più secondo il proprio metodo, ma secondo il testo, e perciò senza voglia, dimenticando spesso l'ora della lezione. Così era stato anche quel giorno. 

- No, vado io, Dolly, e tu resta a sedere - disse Levin. - Faremo tutto bene, secondo il libro. Soltanto quando arriverà Stiva, andremo a caccia, e allora salteremo le lezioni. - E Levin andò da Griša. 

Lo stesso disse Varen'ka a Kitty. Varen'ka anche nella casa felice, ben organizzata dei Levin, aveva saputo rendersi utile. 

- Ordinerò io la cena, e voi restate a sedere - disse e si alzò per andare da Agaf'ja Michajlovna. 

- Già, forse non hanno trovato i polli. Allora dei nostri\ldots{} - disse Kitty. 

- Ne ragioneremo con Agaf'ja Michajlovna - e Varen'ka sparve con lei. 

- Che cara ragazza! - disse la principessa. 

- Non cara, maman, ma una tale delizia, come non ce n'è al mondo. 

- Allora, quest'oggi aspettate Stepan Arkad'ic - disse Sergej Ivanovic, non desiderando evidentemente continuare il discorso su Varen'ka. - È difficile trovare due cognati meno somiglianti fra di loro - disse con un sorriso fine: - uno, mobile, che vive soltanto in società come un pesce nell'acqua; l'altro, il nostro Kostja, vivace, svelto, sensibile a tutto, ma che appena è in società, si gela o si dibatte insensatamente, come un pesce tirato a secco. 

- Già, è proprio insensato - disse la principessa, rivolgendosi a Sergej Ivanovic. - Proprio di questo volevo pregarvi, di dirgli che per lei - e indicò Kitty - è impossibile restare qui, è assolutamente necessario andare a Mosca, invece. Lui dice che farà venire il dottore\ldots{} 

- Maman, egli farà tutto, è d'accordo su tutto - disse Kitty, con stizza verso la madre che aveva chiamato a giudice, in questa faccenda, Sergej Ivanovic. 

Nel più bello della loro conversazione si sentì per il viale uno sbuffar di cavalli e un rumore di ruote sulla ghiaia. 

Dolly non aveva ancora fatto in tempo ad alzarsi per andare incontro al marito, che di sotto, dalla finestra dove studiava Griša, saltò fuori Levin e fece scendere Griša. 

- È Stiva! - gridò Levin da sotto al balcone. - Abbiamo finito, Dolly, non aver paura! - soggiunse e si mise a correre incontro alla carrozza come un ragazzo. 

- Is, ea, id, eius, eius, eius! - gridava Griša, saltellando per il viale. 

- E qualcun altro ancora. Forse papà! - gridò Levin fermandosi all'ingresso del viale. - Kitty, non andare per la scala ripida, fa' il giro. 

Ma Levin s'era sbagliato nello scambiare la persona seduta in carrozza per il vecchio principe. Quando si fu accostato alla carrozza, vide, accanto a Stepan Arkad'ic, non già il principe, ma un bel giovane robusto, con un berretto irlandese con due lunghi nastri dietro. Era Vasen'ka Veslovskij, cugino in terzo grado degli Šcerbackij, giovane elegante di Pietroburgo e di Mosca, ``ottimo ragazzo e appassionato cacciatore'', come lo presentò Stepan Arkad'ic. 

Per nulla sconcertato dalla delusione provocata per essersi sostituito al vecchio principe, Veslovskij salutò allegramente Levin, ricordando la sua conoscenza di un tempo, e preso Griša in carrozza, lo fece passare di là dal pointer che Stepan Arkad'ic aveva portato con sé. 

Levin non montò in carrozza, ma tenne dietro. Era urtato perché non era venuto il vecchio principe, cui voleva tanto più bene quanto più lo imparava a conoscere, e perché era apparso Vasen'ka Veslovskij, persona del tutto estranea e superflua. Gli apparve ancor più estraneo e superfluo quando, avvicinatosi alla scalinata, dove s'era riunita tutta l'animata compagnia dei grandi e dei piccoli, vide che Vasen'ka Veslovskij baciava la mano a Kitty con un'aria particolarmente affabile e galante. 

- E noi siamo cousins con vostra moglie, e anche vecchi amici - disse Vasen'ka Veslovskij, stringendo di nuovo fortemente la mano di Levin. 

- Be', com'è la caccia? - chiese a Levin Stepan Arkad'ic che aveva appena fatto in tempo a salutare tutti. - Ecco, io e lui abbiamo le intenzioni più feroci. Ma come, maman, da allora non sono ancora andati a Mosca! Ecco, Tanja, questo è per te! Tiralo fuori dalla vettura, di dietro - diceva da tutte le parti. 

- Come ti sei ringiovanita, Dollen'ka! - diceva alla moglie, baciandole ancora una volta la mano, trattenendola nella propria, e dandovi sopra dei colpetti con l'altra. 

Levin, che un minuto prima era nella più amena disposizione d'animo, adesso guardava tutti torvo, e niente più gli piaceva. 

``Chi sa chi ha baciato ieri con queste labbra!'' pensava, guardando le tenerezze di Stepan Arkad'ic con la moglie. Guardò Dolly e neanche lei gli piacque. 

``Lei non crede mica al suo amore. Allora perché mai è così contenta? È disgustosa!'' pensava Levin. 

Guardò la principessa, che un minuto prima era stata così cara, e non gli piacque il modo con cui salutava e accoglieva in casa sua questo Vasen'ka con i suoi nastri. 

Perfino Sergej Ivanovic, uscito anche lui sulla scalinata, non gli riuscì gradito per quella finta benevolenza con cui accolse Stepan Arkad'ic, quando egli sapeva che suo fratello non amava Oblonskij e non lo stimava. 

E Varen'ka, anche lei, gli era antipatica per il suo modo di far conoscenza, con quell'aria di sainte nitouche, con quel signore, quando invece pensava soltanto al modo di prendere marito. 

E più antipatica di tutti gli era Kitty, per la maniera con la quale si era sottomessa al tono di allegria con cui quel signore considerava il proprio arrivo in campagna come una festa per sé e per gli altri, e gli era particolarmente sgradita per quello speciale sorriso con cui rispondeva ai sorrisi di lui. 

Discorrendo rumorosamente, entrarono in casa; ma non appena tutti si furono seduti, Levin si voltò e uscì. 

Kitty si accorse che qualcosa era accaduto al marito. Voleva trovare un momento per parlargli a solo, ma egli si affrettò ad allontanarsi da lei, dicendo che doveva passare in amministrazione. Da tempo gli affari dell'azienda non gli apparivano così importanti come quel giorno. ``Per loro è sempre festa - pensava - ma qui ci sono gli affari tutt'altro che festivi, che non aspettano indugio e senza i quali non si può vivere''. 

\capitolo{VII}\label{vii-5} 

Levin tornò a casa solo quando lo mandarono a chiamare per la cena. Sulla scala stavano Kitty e Agaf'ja Michajlovna che si consigliavano sui vini per la cena. 

- Ma perché fate un tale fuss? Servite quello d'ogni giorno. 

- No, Stiva non lo beve\ldots{} Kostja, aspetta, che t'è successo? - disse Kitty, seguendolo, ma egli se ne andò senza aspettarla, inesorabile, a grandi passi, in sala da pranzo ed entrò subito nell'animata conversazione generale sostenuta da Vasen'ka Veslovskij e Stepan Arkad'ic. 

- Su, dunque, domani andiamo a caccia? - disse Stepan Arkad'ic. 

- Andiamo, vi prego - disse Veslovskij, mettendosi a sedere di fianco e ripiegando sotto di sé una delle sue gambe grasse. 

- Io sono contentissimo, andiamo. E voi siete già andato a caccia quest'anno? - chiese Levin a Veslovskij, esaminando attentamente la gamba di lui, ma con quella falsa affabilità che Kitty conosceva così bene e che gli si addiceva così poco. - Beccacce bottaie non so se ne troveremo, ma di beccaccini ce n'è tanti. Solo che bisogna mettersi in cammino presto. Non vi stancherete? Non sei stanco, Stiva? 

- Io stanco? Non sono ancora mai stato stanco. Vogliamo non dormire tutta la notte? Andiamo a passeggio. 

- Davvero vogliamo non dormire? benissimo! - confermò Veslovskij. 

- Oh, di questo siamo sicuri, che tu possa non dormire e non lasciar dormire gli altri - disse Dolly al marito con quell'ironia appena percettibile con cui adesso trattava quasi sempre il marito. - E secondo me, adesso, è già ora\ldots{} io vado, non ceno. 

- No, rimani a sedere, Dollen'ka - disse Stepan Arkad'ic, passando dalla sua parte alla tavola grande alla quale cenavano. - Ti racconterò ancora tante cose. 

- Probabilmente, nulla. 

- E sai, Veslovskij è stato da Anna. E va di nuovo da loro. Perché sono a settanta verste da voi, e io pure ci andrò certamente. Veslovskij, vieni qua. 

Vasen'ka passò accanto alle signore e si sedette vicino a Kitty 

- Ah, raccontate per favore, siete stato da lei? come sta? - gli si rivolse Dar'ja Aleksandrovna. 

Levin era rimasto all'altra estremità della tavola e, senza smettere di parlare con la principessa e con Varen'ka, vedeva che fra Stepan Arkad'ic, Dolly, Kitty e Veslovskij si svolgeva un'animata, misteriosa conversazione. Non solo c'era una conversazione misteriosa, ma egli scorgeva sul viso di sua moglie l'espressione di un sentimento serio, mentre guardava senza abbassare gli occhi, il bel viso di Vasen'ka che raccontava con animazione qualcosa. 

- Da loro si sta proprio bene - raccontava Veslovskij di Vronskij e Anna. - Io, naturalmente, non voglio giudicare, ma in casa loro ci si sente come in famiglia. 

- Che cosa pensano di fare? 

- Pare che per l'inverno vogliano andare a Mosca. 

- Come sarebbe bene andare insieme da loro! Tu quando vai? - chiese Stepan Arkad'ic a Vasen'ka. 

- Passerò da loro il mese di luglio. 

- E tu andrai? - si rivolse Stepan Arkad'ic alla moglie. 

- È tanto tempo che voglio andare e ci andrò certamente - disse Dolly. - Lei mi fa pena, io la conosco. È una carissima donna. Andrò da sola, quando tu sarai già andato via, e così non darò fastidio a nessuno. Anzi, è meglio senza di te. 

- E va bene - disse Stepan Arkad'ic. - E tu Kitty? 

- Io? e perché dovrei andare? - disse Kitty, infiammandosi tutta. E si voltò a guardare il marito. 

- Ma voi conoscete Anna Arkad'evna? È una donna molto interessante. 

- Sì - rispose a Veslovskij, arrossendo ancora di più; poi si alzò e si accostò al marito. 

- Allora domani vai a caccia? - ella disse. 

La gelosia di Levin in quei pochi momenti, in particolare a causa del rossore che aveva ricoperto le guance di Kitty mentre parlava con Veslovskij, era già arrivata lontano. Adesso, sentendo le parole di lei, le intendeva a modo suo. Per quanto in seguito gli fosse strano ricordare questo, ora gli appariva chiaro che, nel chiedergli se andava a caccia, il fatto interessava lei solo per sapere se egli avrebbe procurato questo piacere a Vasen'ka Veslovskij, del quale, secondo lui, ella era già innamorata. 

- Sì, andrò - rispose con voce non naturale, antipatica a lui stesso. 

- Ma è meglio che domani restiate qui per un giorno, altrimenti Dolly non lo vede proprio suo marito; e che andiate dopodomani - disse Kitty. 

Ma il senso delle parole di Kitty si era già trasformato così in Levin: ``Non mi separare da lui. Che tu parta per me è indifferente, ma lasciami godere la compagnia di questo delizioso giovane'' 

- Ah, se vuoi domani rimarremo - rispose Levin, con gentilezza accentuata. 

Vasen'ka intanto, non sospettando per nulla la sofferenza che causava la sua presenza, si alzò da tavola dopo Kitty e, seguendola con uno sguardo sorridente, affabile, le tenne dietro. 

Levin aveva visto quello sguardo. Impallidì e per un attimo non riuscì a respirare. ``Ma come si permette di guardare mia moglie così!'' gli ribolliva dentro. 

- Allora domani? Andiamo, per favore? - disse Vasen'ka, sedendosi su di una sedia e piegando di nuovo una gamba secondo la sua abitudine. 

La gelosia di Levin si spinse ancora oltre. Si vedeva già un marito ingannato, di cui la moglie e l'amante avevano bisogno solo perché procurasse loro gli agi e gli svaghi della vita. 

Tuttavia, malgrado questo, interrogava con cortese ospitalità Vasen'ka sulle sue cacce, sul fucile, sugli stivali e acconsentì a partire l'indomani. 

Per fortuna di Levin, la vecchia principessa pose fine alle sue sofferenze con l'alzarsi lei stessa e consigliare a Kitty di andare a letto. Ma anche qui la cosa non fu senza pena per Levin. Nel salutare la padrona di casa, Vasen'ka voleva di nuovo baciarle la mano, ma Kitty, arrossendo, con un'ingenua scortesia di cui la principessa la rimproverò, disse, sottraendo la mano: 

- Questo da noi non si usa. 

Agli occhi di Levin ella era colpevole di aver ammesso simili rapporti, e ancor più colpevole di aver mostrato, così goffamente, che non le piacevano. 

- Via, che gusto c'è a dormire! - disse Stepan Arkad'ic, che, dopo parecchi bicchieri di vino bevuti a tavola era adesso dell'umore più cordiale e poetico. - Guarda, Kitty - diceva, indicando la luna che si levava di là dai tigli - che incanto! Veslovskij, ecco, ora ci vorrebbe una serenata. Sai, ha una bella voce, io e lui abbiamo cantato insieme in viaggio. Ha portato con sé delle bellissime romanze, due nuove. Sarebbe bene cantare con Varvara Andreevna. 

Quando tutti si lasciarono, Stepan Arkad'ic passeggiò ancora a lungo con Veslovskij per il viale, e si sentivano le loro voci che intonavano la nuova romanza. 

Levin, accigliato, stava seduto, ascoltando queste voci, su di una poltrona nella camera della moglie, e taceva ostinatamente alle domande di lei su cosa gli fosse accaduto; ma quando alla fine lei stessa, sorridendo timida, gli chiese: ``Forse non t'è piaciuto qualcosa in Veslovskij?'' egli proruppe e disse tutto. Ma quello che diceva lo umiliava e perciò ancor più si irritava. Stava in piedi dinanzi a lei, con gli occhi paurosamente scintillanti di sotto alle ciglia aggrottate, e stringeva al petto le mani forti, come volesse tendere tutte le proprie forze per contenersi. L'espressione del viso sarebbe stata severa e perfino crudele, se non avesse espresso nello stesso tempo una sofferenza che commoveva lei. Gli zigomi gli tremavano e la voce si spezzava. 

- Tu devi capire che io non sono geloso: è una parola disprezzabile. Non posso esser geloso e credere che\ldots{} Non posso dire quello che sento, ma ciò è terribile\ldots{} Non sono geloso, ma sono umiliato, offeso dal fatto che qualcuno osi guardarti con occhi così\ldots{} 

- Ma quali occhi? - diceva Kitty, cercando di ricordare, il più scrupolosamente possibile, tutti i discorsi e i gesti di quella serata e tutte le sfumature. 

In fondo all'anima ella riteneva che ci fosse stato qualcosa proprio nel momento in cui Levin era andato a sedersi dietro di lei all'altra estremità della tavola, ma non osava confessarlo neanche a se stessa, e tanto meno dirlo a lui, rendendogli più forte la sofferenza. 

- E che cosa può esserci di attraente in me, così come sono? 

- Ah - gridò lui, mettendosi le mani nei capelli. - Sarebbe meglio che non parlassi!\ldots{} Vuol dire che se fossi attraente\ldots{} 

- Ma no, Kostja, aspetta, ascolta! - diceva lei, guardandolo con un'espressione tormentata e pietosa. - Su, cosa mai puoi pensare? Quando per me non c'è nessuno, nessuno, nessuno\ldots{} Su, vuoi che io non veda nessuno? 

Nel primo momento la gelosia di lui le era parsa offensiva; era irritata che la più piccola distrazione, la più innocente, le venisse preclusa; ma ora, in quel momento, avrebbe sacrificato volentieri non solo sciocchezze simili, ma qualsiasi cosa per la tranquillità di lui, per liberarlo dalla sofferenza che provava. 

- Tu devi capire l'orrore e la comicità della mia situazione - continuava lui con un sussurro disperato - che lui è in casa mia, che in particolare non ha fatto nulla di sconveniente, oltre quella sua disinvoltura e quell'accavallare le gambe\ldots{} Lui ritiene questo del miglior tono, e perciò devo essere cortese con lui. 

- Ma, Kostja, tu esageri! - diceva Kitty, rallegrandosi in fondo all'animo di quella forza d'amore verso di lei che adesso si esprimeva nella gelosia. 

- La cosa più orribile è che tu sei come sempre, e adesso, quando tu sei una cosa tanto sacra per me, e noi siamo così felici, a un tratto un simile fango\ldots{} Ma non fango, perché lo insulto? Lui non mi riguarda. Ma perché la mia, la tua felicità\ldots{} 

- Sai, capisco come ciò sia avvenuto - cominciò Kitty. 

- Perché, perché? 

- Ho visto come guardavi quando parlavamo a cena. 

- Sì, sì - disse Levin spaventato. 

Ella gli raccontò di che cosa avevano parlato. E, raccontando questo, le veniva meno il respiro per l'agitazione. Levin stette un po' in silenzio, poi le esaminò il viso pallido, spaventato, e a un tratto si prese la testa fra le mani. 

- Katja, ti ho tormentata! Amore mio, perdonami! È una follia! Katja, sono proprio colpevole. E si poteva tormentarsi tanto per una sciocchezza simile? 

- No, mi fai pena tu. 

- Io? Io? Cosa sono io, un pazzo!\ldots{} Ma tu perché? È orribile pensare che una qualsiasi persona estranea possa turbare la nostra felicità. 

- Certo, questo è proprio offensivo. 

- No, allora io, al contrario, lo lascerò stare apposta da noi tutta l'estate e lo colmerò di gentilezze - diceva Levin, baciandole le mani. - Ecco, vedrai. Domani\ldots{} Sì, è vero, domani andremo via. 

\capitolo{VIII}\label{viii-5} 

Il giorno dopo, le signore non s'erano ancora alzate, e gli equipaggi per la caccia, un calesse e un carro, erano fermi dinanzi all'ingresso. Laska, che fin dal mattino aveva capito che si sarebbe usciti a caccia, dopo aver mugolato e saltato a sazietà, sedeva a cassetta accanto al cocchiere, guardando, agitata e scontenta del ritardo, la porta dalla quale dovevano ancora uscire i cacciatori. Per primo uscì Vasen'ka Veslovskij, con grandi stivali nuovi che gli arrivavano fino a metà delle cosce grasse, un camiciotto verde, la cartuccera nuova che odorava di pelle, il berrettino coi nastri e un fucile inglese nuovo fiammante, senza ganci e senza cinghia. Laska gli andò incontro, lo salutò, saltando, gli chiese a modo suo se sarebbero usciti presto gli altri, ma, non avendone ricevuta risposta, s'accucciò al suo posto d'attesa e di nuovo trattenne il respiro con la testa girata da un lato e un orecchio teso. Alla fine la porta si aprì con fracasso, ne corse fuori, girando e voltando il muso all'aria, Krak, il pointer pezzato di giallo di Stepan Arkad'ic, e uscì lo stesso Stepan Arkad'ic col fucile in mano e un sigaro in bocca. ``Tout beau, tout beau Krak!'' egli gridava, carezzando il cane che gli poneva le zampe sul ventre e sul petto, impigliandosi con esse nel carniere. Stepan Arkad'ic aveva calzature di cuoio d'un sol pezzo, fasce, pantaloni sdruciti e cappotto corto. In testa aveva l'avanzo di un cappello, ma il fucile di nuovo tipo era un gioiello e il carniere e la cartuccera, benché consunti, erano di ottima qualità. 

Vasen'ka Veslovskij non aveva conosciuto, fino a quel momento, una simile vera eleganza venatoria, essere, cioè, rivestito di cenci, ma possedere gli attrezzi di caccia della qualità migliore. Lo capì ora, guardando Stepan Arkad'ic che, fra quei cenci, splendeva nella figura elegante, signorile, ben curata e allegra, e decise che per la prossima caccia si sarebbe assolutamente equipaggiato in una maniera simile. 

- Be', e il nostro padron di casa che fa? - gli domandò. 

- La moglie giovane - disse sorridendo, Stepan Arkad'ic. 

- Già, e così deliziosa. 

- Era già vestito. Probabilmente sarà corso di nuovo da lei. 

Stepan Arkad'ic aveva indovinato. Levin era corso di nuovo dalla moglie a domandarle ancora una volta se lo aveva perdonato per la sciocchezza del giorno prima, e ancora per pregarla d'essere prudente, in nome di Dio. Soprattutto che stesse lontana dai bambini: loro potevano sempre darle un urtone. Poi bisognò ancora una volta ricevere da lei conferma che non era arrabbiata con lui perché se ne andava via per due gironi, e ancora pregarla di mandargli assolutamente un biglietto l'indomani mattina a mezzo di un uomo a cavallo, di scriver magari solo due parole tanto perché egli potesse sapere che stava bene. 

Per Kitty, come sempre, era doloroso separarsi dal marito per due giorni, ma, vista la figura di lui animata, che sembrava ancor più grande e forte con gli stivaloni e il camiciotto bianco, e una certa luce negli occhi, per lei incomprensibile, dovuta alla eccitazione della caccia, per questa sua gioia dimenticò il proprio cruccio e lo congedò allegra. 

- Perdonate, signori! - egli disse, venendo di corsa sulla scala. - La colazione l'hanno messa dentro? Perché il sauro a destra? Be', fa lo stesso. Laska, lascia, a cuccia! 

- Lasciali andare nell'armento giovane - disse rivolto al bovaro che lo aveva aspettato accanto alla scalinata per chiedergli dei torelli. - Perdonate, ecco che arriva un altro sciagurato. 

Levin saltò giù dal calesse, dove già aveva preso posto, incontro al legnaiuolo imprenditore, che veniva verso la scala di ingresso con una sazen' in mano. 

- Ecco, ieri non sei venuto in amministrazione e ora mi fai perder tempo. Che c'è? 

- Ordinate di fare ancora un giro. Si devono aggiungere tre scalini. E ce li metteremo proprio bene. Sarà molto più comoda. 

- Avresti dovuto ascoltarmi - riprese Levin con stizza. - Ti avevo detto: metti a posto le pareti di sostegno e poi incastra gli scalini. Adesso non potrai più correggere. Fa' come t'ho detto; tagliane una nuova. 

Il fatto era che, nell'ala del fabbricato che si stava costruendo, l'imprenditore aveva sciupato la scala, tagliandola a parte e senza calcolare la pendenza, in modo che tutti gli scalini erano risultati inclinati, quando l'avevano messa a posto. Adesso, l'imprenditore voleva lasciar la stessa scala e aggiungervi tre scalini. 

- Sarà molto meglio. 

- Ma dove ti uscirà mai coi tre scalini? 

- Di grazia, signore - disse l'imprenditore con un sorriso sprezzante\ldots{} - Uscirà proprio a misura. Così, cioè, verrà fuori di giù - disse con un gesto convincente - e andrà, andrà su e arriverà. 

- Ma tre scalini sposteranno la lunghezza\ldots{} E dove arriverà? 

- Ma, come andrà giù, così pure arriverà, s'intende - diceva con insistenza e persuasione l'imprenditore. 

- Già, sotto al soffitto arriverà, e nel muro. 

- Vi prego. Ecco, comincerà di giù. Andrà, andrà su e arriverà. 

Levin tirò fuori la bacchetta del fucile e cominciò a disegnare una scala sulla polvere. 

- Su, vedi? 

- Come ordinate voi - disse il falegname mentre a un tratto gli si rischiaravano gli occhi perché, evidentemente, aveva capito. - Si vede che bisogna tagliarne un'altra. 

- E già, fa' proprio come t'è stato ordinato - gridò Levin, sedendosi in calesse. - Via! Tieni i cani, Filipp. 

Levin adesso, lasciate dietro di sé tutte le preoccupazioni familiari e amministrative, sentiva un così forte senso di gioia di vivere e di attesa, che non aveva voglia di parlare. Inoltre provava quel senso di agitazione che prova ogni cacciatore, avvicinandosi al luogo della caccia Se qualcosa ancora lo occupava, adesso, erano solo dei particolari: se avrebbero trovato qualcosa nella palude di Kolpen, come si sarebbe comportata Laska in confronto di Krak, e come lui stesso quel giorno sarebbe riuscito a tirare. E se avesse fatto una brutta figura dinanzi a una persona nuova? E se Oblonskij l'avesse superato nel tiro? pure questo gli veniva in mente. 

Oblonskij provava una sensazione simile, ed era poco loquace anche lui. Solo Vasen'ka Veslovskij parlava allegramente, senza interruzione. Adesso, ascoltandolo, Levin si vergognava di ricordare come fosse stato ingiusto con lui il giorno innanzi. Vasen'ka era veramente un buon ragazzo, semplice, cordiale e allegro. Se Levin l'avesse conosciuto da scapolo, si sarebbe certo legato a lui. A Levin spiaceva solo un po' quel suo modo ozioso di veder la vita e quella sua certa disinvolta eleganza. Come se si riconoscesse un alto, indiscusso valore, perché aveva le unghie lunghe e il berrettino e tutto il resto bene assortito; ma questo gli si poteva perdonare in cambio della cordialità e della finezza. Piaceva a Levin per la sua buona educazione, per l'ottima pronuncia del suo francese e del suo inglese e perché era un uomo del suo mondo. 

A Vasen'ka piaceva straordinariamente il cavallo di steppa del Don che era al bilancino sinistro. Continuamente se ne estasiava. 

- Com'è bello galoppare per la steppa su di un cavallo da steppa. Eh? Non è vero? - egli diceva. 

Nel montare un cavallo da steppa, s'immaginava qualcosa di selvaggio, di poetico, da cui non veniva fuori nulla; ma la sua ingenuità, unita alla sua bellezza, al sorriso cordiale e alla grazia dei suoi movimenti, era molto attraente. Forse perché la sua natura era simpatica a Levin, o perché Levin cercava di espiare la colpa del giorno prima giudicando tutto buono in lui, certo è che gli faceva piacere stare con lui. 

Allontanatisi di tre verste, Veslovskij a un tratto s'accorse che non aveva i sigari e il portafoglio, e non sapeva se li avesse perduti o lasciati sul tavolo. Nel portafoglio c'erano trecentosettanta rubli e perciò non si poteva lasciar perdere. 

- Sapete cosa, Levin? Faccio una galoppata fino a casa su questo cavallo del Don del bilancino. Sarà un'ottima cosa. Eh? - egli diceva, preparandosi già a montare. 

- Ma no, perché - rispose Levin che calcolava che Vasen'ka non doveva pesar certo meno di sei pudy. - Manderò il cocchiere. 

Il cocchiere montò sul cavallo del bilancino e Levin si mise a guidar lui stesso la pariglia. 

\capitolo{IX}\label{ix-5} 

- Be', qual'è il nostro itinerario? Diccelo ben bene - disse Stepan Arkad'ic . 

- Il programma è il seguente: per ora andiamo fino a Gvozdëv. A Gvozdëv c'è una palude da beccacce bottaie, da questa parte, e dietro a Gvozdëv ci sono delle meravigliose paludi da beccacce, e ci sono anche le bottaie. Adesso fa caldo, e noi verso sera (sono venti verste) arriveremo e faremo la caccia della sera: poi pernotteremo e domani andremo alle paludi grandi. 

- E per strada non c'è nulla? 

- C'è qualcosa; ma perderemmo del tempo, e poi fa caldo. Ci sono due bei piccoli passi ma è difficile che ci sia qualcosa. 

A Levin stesso era venuta voglia di fermarsi ai due passi; ma questi erano vicini a casa e ci poteva sempre andare, ed erano tanto stretti che per tre non c'era spazio per tirare. E perciò mancava di sincerità nel dire che era difficile che ci fosse qualcosa. Giunti all'altezza di una piccola palude, Levin voleva passare oltre, senza fermarsi, ma l'occhio esperto di Stepan Arkad'ic vide subito l'acquitrino che si scorgeva dalla strada. 

- Non ci passiamo? - disse, indicando la piccola palude. 

- Levin, per favore, ma è una cosa magnifica! - cominciò a pregare Vasen'ka Veslovskij, e Levin non poté non acconsentire. 

Non fecero in tempo a fermarsi che i cani, sorpassandosi l'un l'altro, volavano già verso la palude. 

- Krak! Laska! 

I cani tornarono. 

- In tre staremo stretti. Io rimarrò qui - disse Levin, sperando che non avrebbero trovato niente all'infuori delle pavoncelle che, levatesi a causa dei cani e dondolandosi in volo, piangevano lamentosamente sopra la palude, 

- No, andiamo; Levin, andiamo insieme! - chiamò Veslovskij. 

- Davvero staremo stretti. Laska, indietro! Laska! Non avete bisogno di un altro cane? 

Levin rimase accanto al calesse e guardava con invidia i cacciatori che attraversavano tutta la piccola palude. All'infuori di una gallinella e delle pavoncelle, di cui Vasen'ka ne uccise una, non c'era altro. 

- Su, ecco, come vedete, non ne valeva la pena - disse Levin - si perde solo tempo. 

- Tuttavia è piacevole. Avete visto? - diceva Vasen'ka Veslovskij, montando poco destramente sul calesse col fucile e la pavoncella in mano. - Come l'ho ammazzata bene questa! Non è vero? Via, arriveremo presto al passo vero? 

D'un tratto i cavalli si lanciarono in avanti, Levin batté col capo contro la canna del fucile di qualcuno ed echeggiò uno sparo. Lo sparo era echeggiato prima, almeno così parve a Levin. Il fatto era che Vasen'ka Veslovskij, nell'abbassare i cani, aveva premuto un grilletto e trattenuto l'altro. La cartuccia si conficcò nel terreno, senza far male a nessuno. Stepan Arkad'ic scosse il capo e rise con aria di rimprovero verso Veslovskij, ma Levin non aveva il coraggio di sgridarlo. In primo luogo, qualsiasi rimprovero sarebbe sembrato provocato dallo scampato pericolo e dal bernoccolo che gli era spuntato sulla fronte; in secondo luogo, Veslovskij fu prima così ingenuamente addolorato e poi così cordialmente e piacevolmente esilarato dalla loro comune confusione, che non poté non riderne lo stesso Levin. 

Quando si avvicinarono alla seconda palude, che era abbastanza grande e doveva prender molto tempo, Levin li esortò a non scendere, ma Veslovskij di nuovo lo pregò. Di nuovo da padrone ospitale, giacché la palude era stretta, Levin rimase accanto agli equipaggi. 

Appena arrivati, Krak, si lanciò verso le montagnole. Vasen'ka Veslovskij per primo corse dietro al cane. E Stepan Arkad'ic non fece in tempo ad avvicinarsi che era già volata fuori una beccaccia bottaia. Veslovskij fece padella, e la bottaia si posò su di un prato non falciato. La beccaccia fu lasciata a Veslovskij. Krak la scovò di nuovo, la puntò e Veslovskij l'uccise e tornò verso gli equipaggi. 

- Adesso andate voi e io resterò coi cavalli - egli disse. 

Levin cominciava ad essere tormentato dalla gelosia venatoria. Consegnò le redini a Veslovskij e si avviò nella palude. Laska, che già da tempo guaiva, protestando lamentosamente per l'ingiustizia, si lanciò in avanti, verso un gruppo di montagnole, che era noto a Levin e pieno di possibilità, perché Krak non vi era andato. 

- Come mai non la fermi? - gridò Stepan Arkad'ic . 

- Non la spaventerà - rispose Levin, compiacendosi del cane e affrettandosi a tenergli dietro. 

Nella ricerca, Laska, quanto più si avvicinava alle note montagnole, tanto più diventava seria. Un piccolo uccello di palude non la distrasse che per un attimo. Aveva fatto un giro intorno alle montagnole, ne cominciò un altro e a un tratto tremò tutta e s'irrigidì. 

- Va', va', Stiva! - gridò Levin, sentendo che il cuore gli cominciava a battere forte e che a un tratto, come se un chiavistello avesse aperto il suo udito teso, tutti i suoni, perduta la misura della distanza, cominciavano a colpirlo disordinatamente, ma con chiarezza. Sentiva i passi di Stepan Arkad'ic e gli parevano un lontano calpestio di cavalli; sentiva il suono dell'estremità della montagnola su cui era salito, che cedeva con le radici, e questo suono gli pareva il volo della beccaccia. Udiva pure dietro, non lontano, un certo ciangottio su per l'acqua di cui non riusciva a rendersi conto. 

Scegliendo il posto per poggiare il piede, si avvicinò al cane. 

- Pille! 

Non una beccaccia, ma una bottaia era sfuggita di sotto al cane. Levin imbracciò il fucile, ma nel momento in cui mirava, quello stesso ciangottio su per l'acqua si fece più forte, più vicino e si unì alla voce di Veslovskij che gridava in maniera strana e forte qualche cosa. Levin si accorse di mirar la beccaccia di dietro, e tuttavia tirò. Convinto d'aver fatto padella, Levin si voltò e vide che i cavalli col calesse non erano più sulla strada, ma nella palude. 

Veslovskij, per osservare il tiro, era entrato nella palude e aveva fatto impantanare i cavalli. 

``Che il diavolo se lo pigli!'' sbottò Levin fra sé, tornando verso il veicolo impantanato. 

- Perché vi siete mosso? - disse secco e, chiamato il cocchiere, si diede a liberare i cavalli. 

Levin era irritato che gli avessero disturbato il tiro, che gli avessero fatto impantanare i cavalli e, soprattutto, che, per liberare i cavalli, per staccarli, né Stepan Arkad'ic, né Veslovskij dessero una mano a lui e al cocchiere, poiché né l'uno né l'altro avevano la più piccola idea di cosa fosse il tiro a due. Senza rispondere neppure una parola a Vasen'ka che lo rassicurava che là era completamente asciutto, Levin lavorava in silenzio col cocchiere per liberare i cavalli. Ma poi, riscaldatosi nel lavoro e avendo visto con quanta buona volontà Veslovskij tirasse il calesse per un parafango, fin quasi a staccarlo, Levin si rimproverò di essere stato troppo freddo verso di lui, sotto l'influenza del sentimento del giorno innanzi, e cercò di riparare alla propria freddezza con una particolare affabilità. Quanto tutto fu messo in ordine e gli equipaggi furono ricondotti sulla strada, Levin ordinò di tirar fuori la colazione. 

- Bon appétit, bonne conscience! Ce poulet va tomber jusqu'au fond de mes bottes! - diceva con un motto francese Vasen'ka che, di nuovo allegro, mangiava un secondo pollastrino. - Via, adesso i nostri guai sono finiti; ora tutto andrà bene. Solo che io, in sconto del mio peccato, ho il dovere di sedere a cassetta. Non è vero? Eh no, no, io sono l'automedonte. Vedrete come vi porterò! - rispondeva, senza lasciare le redini, a Levin che lo pregava di lasciare fare al cocchiere. - no, io devo espiare la mia colpa, e sto benissimo a cassetta. - E partì. 

Levin temeva ch'egli avrebbe stancato i cavalli, specialmente quello di sinistra, il sauro, che non riusciva a tener bene; ma senza volere si sottometteva all'allegria di lui, ascoltava le romanze che Veslovskij, seduto a cassetta, cantò per tutta la strada, e i racconti e le rappresentazioni dialogate, e ancora su come bisognava guidare all'inglese, four in hand; e tutti, dopo colazione, nella più amena disposizione d'animo, giunsero alla palude di Gvozdëv. 

\capitolo{X}\label{x-5} 

Vasen'ka guidò i cavalli così in fretta che giunsero alla palude troppo presto, sì che faceva ancora caldo. 

Avvicinandosi alla palude più grande, mèta prima del viaggio, Levin pensò involontariamente come liberarsi di Vasen'ka e procedere senza impedimenti. Stepan Arkad'ic, evidentemente desiderava anche lui questo, e sul suo viso Levin scorse l'espressione ansiosa che ha il cacciatore prima di cominciare la caccia, mista a una certa bonaria furberia che gli era propria. 

- E come andremo? La palude è magnifica, vedo, e ci sono gli avvoltoi - disse Stepan Arkad'ic, mostrando due grandi uccelli che volteggiavano sopra la càrice palustre. - Dove ci sono gli avvoltoi, c'è anche la selvaggina. 

- Su, ecco, vedete, signori - disse Levin con un'espressione alquanto torva, stringendosi gli stivali ed esaminando i pistoni del fucile. - Vedete questa càrice? - Egli indicò un isolotto, che sembrava scuro per le erbe nereggianti in un enorme prato bagnato, falciato a metà, che si stendeva sul lato destro del fiume. - La palude comincia qui, dritto dinanzi a voi, guardate, là dove è più verde. Di qua volta a destra, dove camminano i cavalli; là ci sono delle montagnole, di solito ci sono le bottaie, intorno a quella càrice fino a quegli ontani e fin proprio al mulino. Ecco, là, vedi, dove c'è l'ansa. È il luogo migliore. Là una volta ho ucciso diciassette beccacce. Ci divideremo coi due cani in direzione opposta e là accanto al mulino ci riuniremo. 

- Be', chi va a destra e chi a sinistra? - domandò Stepan Arkad'ic. - A destra è più largo, andate voi due, mentre io andrò a sinistra - disse con noncuranza. 

- Bene! Lo vinceremo nel tiro! Su, andiamo, andiamo, andiamo! - rincalzò Vasen'ka. 

Levin non poteva non acconsentire, ed essi si divisero. 

Erano appena entrati nella palude, che tutti e due i cani cominciarono a cercare insieme e si spinsero verso l'acqua rugginosa. Levin conosceva questo braccare di Laska, cauto e vago; conosceva anche il luogo e aspettava un piccolo stormo di beccacce. 

- Veslovskij, venitemi a fianco, a fianco! - disse, con voce smorzata, al compagno che sguazzava dietro nell'acqua e del quale gli interessava la direzione del fucile, dopo lo sparo casuale nella palude di Kolpen. 

- No, non voglio darvi fastidio, non vi curate di me. 

Ma Levin, senza volere, pensava e ricordava le parole di Kitty, quando lo aveva lasciato partire: ``Attenti a non ammazzarvi l'un l'altro''. I cani si avvicinavano sempre più evitandosi e seguendo ognuno la propria traccia; l'attesa di una beccaccia era così grande che il cigolar del proprio tacco, tirato fuori dall'acqua rugginosa, sembrava a Levin lo zirlio della beccaccia, ed egli afferrava e stringeva il calcio del fucile. 

- Pum, pum - gli echeggiò sopra l'orecchio. Vasen'ka aveva tirato a uno stormo di anatre che volteggiavano sopra la palude e che in quel momento erano volate fin troppo sopra ai cacciatori. Levin non fece in tempo a guardare, che una beccaccia zirlò, poi un'altra e una terza, e ancora otto se ne levarono l'una dietro l'altra. 

Stepan Arkad'ic ne uccise una proprio nel momento in cui si metteva a fare le sue volute e l'uccello cadde come una palla nel terreno paludoso. Oblonskij, senza perder tempo, ne mirò un'altra che volava ancora in basso verso la càrice e, contemporaneamente al suono dello sparo, anche questo uccello cadde, e lo si vide saltellar fuori dalla càrice falciata, sbattendo l'ala rimasta intatta, bianca di sotto. 

Levin non fu così fortunato: tirò alla prima beccaccia troppo dappresso e fece padella, la mirò quando aveva già cominciato a sollevarsi, ma in quel momento ne volò fuori un'altra di sotto ai suoi piedi e lo distrasse, ed egli fece padella un'altra volta. 

Mentre caricavano i fucili, ancora una beccaccia si alzò e Veslovskij, che aveva fatto in tempo a caricare un'altra volta, lasciò andare in acqua altre due cariche a pallini piccoli. Stepan Arkad'ic raccolse le sue beccacce e guardò Levin con occhi splendenti. 

- Su, ora ci separiamo - disse Stepan Arkad'ic e, zoppicando un po' con la gamba sinistra e tenendo il fucile pronto e fischiando al cane, andò da una parte. Levin e Veslovskij andarono dall'altra. 

A Levin capitava sempre che, se falliva i primi colpi, si accalorava, si irritava e sparava male per tutta la giornata. Così fu anche quel giorno. Di beccacce evidentemente ce n'erano molte. Di sotto al cane, di sotto ai piedi dei cacciatori ne volavano continuamente, e Levin avrebbe potuto riprendersi; ma quanto più sparava tanto più faceva una brutta figura dinanzi a Veslovskij, che tirava allegramente bene o male, senza colpir nulla e senza per nulla confondersi. Levin si affannava, non resisteva, si scalmanava sempre più ed era giunto ormai al punto che, tirando, non sperava quasi più di colpire. Anche Laska sembrava capire questo. Aveva cominciato a cercare più pigra e si voltava a guardare i cacciatori, perplessa e contrariata. Gli spari seguivano agli spari. Il fumo della polvere era intorno ai cacciatori e nel grande, spazioso carniere c'erano soltanto tre beccacce leggere, piccole. Di queste, una era stata uccisa da Veslovskij e un'altra da tutti e due insieme. Intanto, dall'altra parte della palude, si sentivano gli spari non frequenti ma, così pareva a Levin, sostanziosi di Stepan Arkad'ic e quasi dopo ogni sparo si sentiva: ``Krak, Krak porta qua!''. 

Questo agitava ancor più Levin. Le beccacce volavano senza posa nell'aria sopra la càrice. Lo zirlio raso terra e il gracidare alto si sentivano da ogni parte continuamente; le beccacce, fatte alzar prima e poi sospese nell'aria, si posavano davanti ai cacciatori. Invece di due avvoltoi, adesso, ne volavano decine, stridendo al di sopra della palude. 

Oltrepassata la parte più grande della palude, Levin e Veslovskij si spinsero là dove un prato di contadini era diviso a lunghe strisce fiancheggiate dalla càrice, segnato qua da strisce calpestate, là da un piccolo tratto falciato. Una metà di queste strisce era già falciata. 

Sebbene per il tratto non falciato ci fosse poca speranza di trovare altrettanta caccia quanto per il tratto falciato, Levin aveva promesso a Stepan Arkad'ic di riunirsi con lui e andò innanzi col compagno, per le strisce falciate e per quelle non falciate. 

- Ehi, cacciatori - gridò loro uno dei contadini seduto presso un carro staccato - venite a far la siesta con noi! c'è da bere il vino! 

Levin si voltò a guardare. 

- Venite, via! - gridò un allegro contadino con la barba, dal viso simpatico, mostrando i denti bianchi e sollevando una bottiglia quadrata, verdastra, che splendeva al sole. 

- Qu'est ce qu'ils disent? - domandò Veslovskij. 

- Invitano a bere la vodka. Probabilmente hanno diviso i campi. Io andrei a bere - disse Levin non senza malizia, sperando che Veslovskij, sedotto dalla vodka, andasse accanto a loro. 

- E perché offrono? 

- Così, se ne rallegrano. Davvero, avvicinatevi a loro. Per voi ciò sarà interessante. 

- Allons, c'est curieux. 

- Andate, andate, troverete la strada per il mulino! - gridò Levin e, voltandosi, vide con piacere che Veslovskij, curvo e incespicante, con le gambe stanche e il fucile nella mano tesa, si tirava fuori dalla palude verso i contadini. 

- Vieni anche tu! - gridava il contadino a Levin. - Non aver paura! Mangerai il pirog! 

Levin aveva una gran voglia di bere la vodka e di mangiare un pezzo di pane. Era fiacco e sentiva che tirava a stento fuori dal terreno melmoso le gambe che s'impigliavano, e per un attimo fu in dubbio. Ma il cane si fermò. E subito tutta la stanchezza scomparve, ed egli andò spedito attraverso il terreno melmoso verso il cane. Di sotto alle sue gambe volò fuori una beccaccia, egli sparò e la colpì; il cane continuava a star fermo. ``Pille!''. Di sotto al cane se ne sollevò un'altra. Levin tirò. Ma la giornata era cattiva; fece padella e, quando andò a cercare l'uccello ucciso, non riuscì a trovarlo. Si trascinò per tutta la càrice, ma Laska non credeva ch'egli avesse colpito, e quando egli la mandava a cercare, fingeva di cercare e non cercava. 

Anche senza Vasen'ka, al quale Levin attribuiva la propria sfortuna, le cose non andarono meglio. Anche qua di beccacce ce n'erano molte, ma Levin faceva una padella dopo l'altra. 

I raggi obliqui del sole erano ancora caldi; il vestito, passato da parte a parte dal sudore, si appiccicava al corpo; lo stivale sinistro, pieno d'acqua, era pesante e ciangottava; il sudore gli colava a gocce per il viso sporco del sedimento della polvere; in bocca aveva un certo sapore amaro, nel naso odor di polvere e d'acqua rugginosa, negli orecchi l'incessante zirlio delle beccacce; le canne del fucile non si potevano toccare tanto erano roventi, il cuore aveva battiti forti e rapidi, le mani gli tremavano per l'agitazione, e le gambe stanche inciampavano e si intersecavano nelle anfrattuosità del terreno melmoso; ma egli continuava a camminare e a sparare. Finalmente, fatta una padella vergognosa, buttò a terra il fucile e il cappello. 

``No, bisogna rientrare in sé'' si disse. Riprese il fucile e il cappello, chiamò ai suoi piedi Laska e uscì dalla palude. Uscito all'asciutto, sedette su di un monticello, si tolse gli stivali, ne versò fuori l'acqua; poi si avvicinò alla palude, bevve un po' d'acqua dal sapor di ruggine, bagnò le mani infocate e si lavò il viso e le mani. Rinfrescatosi, si avviò di nuovo verso il luogo dove era andata la beccaccia, con la ferma intenzione di non scalmanarsi. 

Voleva essere tranquillo, ma fu sempre lo stesso. Il suo dito premeva sul grilletto prima ch'egli prendesse la mira dell'uccello. Tutto andava di male in peggio. 

Aveva soltanto cinque pezzi nel carniere quando uscì dalla palude, dirigendosi verso gli ontani dove si doveva incontrare con Stepan Arkad'ic. 

Prima di vedere Stepan Arkad'ic, ne scorse il cane. Di sotto alla radice capovolta di un ontano, saltò fuori Krak, tutto nero di fango maleodorante della palude e, con aria di vincitore, scambiò un'annusata con Laska. Dietro a Krak si fece vedere, all'ombra degli ontani, anche la figura ben fatta di Stepan Arkad'ic. Veniva incontro rosso, sudato, col colletto sbottonato, zoppicando sempre un po' alla stessa maniera. 

- Be', avete sparato molto! - disse, sorridendo allegramente. 

- E tu? - domandò Levin. Ma era inutile domandare, perché aveva già visto il carniere pieno. 

- Be', non c'è male. 

Aveva quattordici pezzi. 

- Una bella palude! A te, probabilmente, ha dato fastidio Veslovskij. In due con un cane solo si va male - disse Stepan Arkad'ic, per attenuare il proprio trionfo. 

\capitolo{XI}\label{xi-5} 

Quando Levin e Stepan Arkad'ic giunsero all'izba del contadino dal quale si fermava sempre Levin, Veslovskij era già là. Era seduto nel centro della capanna e, tenendosi con tutte e due le mani alla panca dalla quale lo tirava via un soldato fratello della padrona, per cavargli gli stivali spruzzati di melma, rideva del suo riso contagiosamente allegro. 

- Sono arrivato or ora. Ils ont été charmants. Figuratevi, mi han dato da bere, da mangiare. Che pane! Una meraviglia! Délicieux! E la vodka, io non ne ho mai bevuta di più gustosa! E a nessun costo hanno voluto prendere del denaro. E non facevano che dire: ``non discutere'', così press'a poco. 

- E perché pigliar denari? Si vede che ve l'hanno offerta. Che forse hanno la vodka per venderla? - disse il soldato, dopo aver tirato via, finalmente, insieme con la calza annerita, lo stivale bagnato. 

Malgrado la sporcizia dell'izba, insudiciata dagli stivali dei cacciatori e dai cani sporchi che si leccavano, malgrado l'odore di polvere e di palude di cui s'era impregnata e l'assenza di coltelli e forchette, i cacciatori bevvero il tè, e cenarono con un gusto tale, quale solo a caccia si prova. Lavati e ripuliti, andarono in un fienile che avevano notato e dove i cocchieri avevano preparato i giacigli per i signori. 

Sebbene cominciasse già ad annottare, nessuno dei cacciatori aveva voglia di dormire. Dopo aver ondeggiato fra i ricordi e i racconti sul tiro, sui cani, sulle cacce precedenti, il discorso cadde su un tema che interessò tutti. Pigliando spunto dalle espressioni di entusiasmo, più volte ripetute da Vasen'ka, sul fascino di quel ricovero notturno, dell'odore del fieno e del carro rotto (a lui sembrava rotto perché era stato staccato dalla parte anteriore), sulla cordialità dei contadini che gli avevano dato da bere la vodka, sui cani che giacevano ciascuno ai piedi del proprio padrone, Oblonskij parlò dell'incanto della caccia da Malthus, alla quale aveva partecipato l'anno prima. Stepan Arkad'ic descriveva quali paludi avesse comprato questo Malthus, nel governatorato di Tver', e come fossero mantenute, e quali equipaggi avessero trasportato i cacciatori, quali mute di cani e quale tenda con colazione fosse stata piantata accanto alla palude. 

- Non capisco - disse Levin, sollevandosi sul fieno - come non ti siano antipatiche quelle persone. Capisco che una colazione con del Lafite possa esser piacevole, ma possibile che non ti sia odioso tutto quello sfarzo? Tutte queste persone, come un tempo i nostri appaltatori, guadagnano il denaro in un modo tale che, mentre lo guadagnano, meritano il disprezzo della gente, ed essi non si curano di questo disprezzo e dopo, con il loro profitto disonesto, si riscattano dal precedente disprezzo. 

- Perfettamente giusto! - rispose Vasen'ka Veslovskij. - Perfettamente! S'intende, Oblonskij lo fa per bonhomie, ma gli altri dicono: ``Oblonskij ci va\ldots{}''. 

- Per nulla - e Levin sentiva che Oblonskij sorrideva dicendo questo; - io non lo stimo più disonesto di qualunque altro mercante e nobile ricco. E quelli e questi hanno guadagnato egualmente col lavoro e con l'ingegno. 

- Già, ma con quale lavoro? È forse lavoro ottenere una concessione e rivenderla? 

- Certo che è un lavoro. Un lavoro nel senso che se non ci fosse lui o altri simili a lui, non ci sarebbero neanche le strade. 

- Ma non un lavoro come il lavoro del contadino o dello scienziato. 

- Ammettiamolo, ma è lavoro nel senso che la sua attività dà dei risultati: le strade. Ma già, tu pensi che le strade ferrate siano inutili. 

- No, questa è un'altra questione; sono pronto a riconoscere che sono utili. Ma qualsiasi acquisto non corrispondente al lavoro impiegatovi non è onesto. 

- Ma chi determinerà la corrispondenza? 

- L'acquisto per via disonesta, per mezzo dell'astuzia - disse Levin, sentendo di non saper definire chiaramente la linea di separazione fra l'onesto e il disonesto - così come l'accaparramento degli uffici bancari - continuò. - Questo male, l'acquisto di enormi sostanze senza lavoro, esiste come al tempo degli appalti, solo che ha cambiato forma. Le roi est mort, vive le roi! Hanno appena fatto in tempo a distruggere gli appalti, che sono apparse le strade ferrate, le banche: è lo stesso: un lucro senza lavoro. 

- Sì, tutto questo è forse giusto e acuto\ldots{} A cuccia, Krak! - gridò al cane Stepan Arkad'ic, evidentemente sicuro della esattezza del proprio tema e perciò calmo e posato. - Ma tu non hai definito la linea di separazione fra il lavoro onesto e quello disonesto. Il fatto che io riceva uno stipendio maggiore del mio capo-ufficio, anche s'egli conosce il lavoro meglio di me, è disonesto? 

- Non lo so. 

- Su, allora ti dirò: il fatto che tu per il tuo lavoro nell'azienda ricavi, ammettiamo, cinquemila rubli, mentre il contadino che ci ospita, per quanto si affatichi non ricavi più di cinquanta rubli, è disonesto nello stesso preciso modo che io riceva più del capo-ufficio e che Malthus riceva più di un ispettore delle ferrovie. Ma, al contrario, c'è un certo atteggiamento ostile, basato su nulla, della società verso queste persone, e mi sembra che ci sia dell'invidia\ldots{} 

- No, questo è ingiusto - disse Veslovskij - invidia non può esserci, piuttosto qualcosa di poco pulito in questo lavoro. 

- No, permetti - proseguì Levin. - Tu dici che è ingiusto che io ricavi cinquemila rubli, e il contadino cinquanta: è vero. È ingiusto, lo sento, ma\ldots{} 

- È proprio così. Come mai noi mangiamo, beviamo, andiamo a caccia, non facciamo nulla, e lui è eternamente, eternamente al lavoro? - disse Vasen'ka Veslovskij, dopo aver pensato, evidentemente per la prima volta in vita sua, in modo chiaro e, perciò con piena sincerità. 

- Sì, tu lo senti, ma non gli daresti il tuo podere - disse Stepan Arkad'ic, che pareva proprio voler stuzzicare Levin. 

Negli ultimi tempi fra i due cognati si erano stabiliti dei segreti rapporti ostili; come se, da quando avevano sposato le sorelle, fosse sorta fra di loro una certa rivalità in chi avesse assestato meglio la propria vita, e quella ostilità, ora, si esprimeva nella conversazione che cominciava ad assumere un tono personale. 

- Non lo do, perché nessuno lo pretende da me, e se volessi, non potrei darlo - rispondeva Levin - e non ci sarebbe a chi darlo. 

- Dàllo pure a questo contadino; non rifiuterà. 

- Sì, come glielo darò? Andrò da lui e concluderò un contratto d'acquisto? 

- Non so, ma se sei convinto di non averne diritto\ldots{} 

- Non sono del tutto convinto. Al contrario sento di non avere il diritto di alienare, sento di avere dei doveri verso la terra e verso la famiglia. 

- No, permetti; ma se tu consideri questa disuguaglianza ingiusta, allora perché non agisci in maniera da\ldots{} 

- Ma io agisco appunto, soltanto negativamente, nel senso che non cerco di aumentare quella differenza di condizione che esiste fra me e lui. 

- No, perdonami, questo è un paradosso. 

- Sì, è una spiegazione in un certo modo sofistica - confermò Veslovskij. - Oh, padrone - disse al contadino che, facendo scricchiolare la porta, era entrato nel fienile. - Non dormi ancora? 

- Macché dormire! Pensavo che lor signori dormissero e invece sento che discutono. Devo prendere un forcone qua. Non morde? - soggiunse, camminando cauto a piedi nudi. 

- E tu dove dormi? 

- Noi facciamo la guardia. 

- Ah, che notte - disse Veslovskij, guardando il limitare dell'izba e la vettura staccata che si intravedeva alla debole luce del crepuscolo, nella grande cornice del portone spalancato. - Ascoltate, sono voci di donne che cantano, e davvero, mica male. Chi è che canta, padrone? 

- Sono le ragazze di casa, qui accanto. 

- Andiamo a divertirci! Tanto non dormiremo. Oblonskij, andiamo! 

- Come sarebbe bello starsene sdraiati e camminare nello stesso tempo! - rispose Oblonskij, stiracchiandosi. - Sdraiàti si sta benissimo. 

- Be', andrò da solo - disse Veslovskij, alzandosi con vivacità e calzandosi. - Arrivederci, signori. Se si sta allegri, vi vengo a chiamare. Mi avete offerto della caccia e io non mi scorderò di voi. 

- Non è vero che è un bravo ragazzo? - disse Oblonskij, quando Veslovskij fu uscito e il contadino ebbe chiuso il portone dietro di lui. 

- Sì, bravo - rispose Levin, continuando a pensare all'argomento della conversazione che c'era stata or ora. Gli pareva d'aver chiaramente espresso, per quanto gli era stato possibile, i suoi pensieri e i suoi sentimenti, e invece tutti e due loro, persone intelligenti e sincere, avevano detto ch'egli si consolava con dei sofismi. Questo lo tormentava. 

- È proprio così, amico mio. Ci vuole una delle due: o confessare che la presente organizzazione della società è giusta, e allora difendere i propri diritti; o confessare che si usufruisce di privilegi ingiusti e, come faccio io, usufruirne con piacere. 

- No, se questo fosse ingiusto, tu non potresti usufruire con piacere di questi beni, almeno io non potrei. Io, soprattutto, ho bisogno di non sentirmi in colpa. 

- Ma davvero non dobbiamo andarci? - disse Stepan Arkad'ic, evidentemente stanco per la tensione del pensiero. - Tanto non dormiremo. Via, andiamo! 

Levin non rispondeva. La frase da lui detta nella conversazione, che egli agiva giustamente solo in senso negativo, lo teneva preoccupato. ``Possibile che solo in senso negativo si possa essere giusti?'' si domandava. 

- Ma come odora forte il fieno fresco! - disse Stepan Arkad'ic, cercando il berretto nel buio. 

- Non per principio, ma perché dovrei venirci? 

- Ma, credimi, ti creerai dei guai - disse Stepan Arkad'ic, che, trovato il berretto, si stava alzando. 

- Perché? 

- E forse non vedo in quali rapporti ti sei messo con tua moglie? Ho sentito che tra voi è questione di primaria importanza che tu vada o no a caccia per due giorni. Tutto questo va bene come idillio, ma per tutta la vita non soddisfa. L'uomo deve essere indipendente, egli ha i suoi interessi virili. L'uomo deve essere uomo - disse Oblonskij, aprendo il portone. 

- Cioè? Andare a far la corte alle ragazze di casa? - domandò Levin. 

- E perché non andarci, se è una cosa allegra? Ça ne tire pas à conséquence. Mia moglie non per questo starà peggio, e io me la spasserò. La cosa principale consiste nel conservare il sacrario della casa. Che in casa non ci sia nulla. Ma le mani non te le legare. 

- Può darsi - disse asciutto Levin, e si voltò su di un fianco. - Domani bisogna andar via presto, e io non sveglio nessuno, e vado via all'alba. 

- Messieurs, venez vite! - si sentì la voce di Veslovskij ch'era tornato. - Charmante! L'ho scoperta proprio io. Charmante, proprio una Gretchen, e abbiamo già fatto amicizia io e lei. Davvero carina assai! - egli raccontava con aria d'approvazione, come se fosse stata fatta carina proprio per lui, ed egli si mostrasse contento di chi gliela aveva preparata. 

Levin finse di dormire, e Oblonskij, infilate le pantofole e acceso un sigaro, uscì dal fienile e subito le loro voci si spensero. 

Levin a lungo non poté prender sonno. Sentiva i cavalli che masticavano il fieno, poi il padrone che col ragazzo più grande si preparava e andava via per far la guardia; sentiva poi che il soldato si metteva a letto, dall'altra parte del fienile, col nipote, un figlioletto del padrone; e il bimbo, con una vocina sottile, comunicava allo zio la propria impressione sui cani che gli erano sembrati terribili ed enormi; il ragazzo poi domandava chi avrebbe acchiappato quei cani, e il soldato con voce sorda e assonnata gli diceva che domani i cacciatori sarebbero andati nella palude e avrebbero sparato coi fucili, e infine, per liberarsi dalle domande del bambino, diceva: ``Dormi, Vas'ka, dormi, se no guarda''; e presto si mise a russare lui stesso, e tutto si chetò; si sentiva solo il nitrito di un cavallo e lo zirlio di una beccaccia. ``Possibile che io sia soltanto negativo? - egli si ripeteva. - Be', e allora? non sono colpevole!''. E si mise a pensare all'indomani. 

``Domani andrò di buon mattino e m'impegno a non scalmanarmi. Di beccacce ce n'è un'infinità. E anche di bottaie. E tornerò all'alloggio e troverò un biglietto di Kitty. E che Stiva abbia ragione, magari: non sono virile con lei, divento una femminuccia. Ma che fare! Sono di nuovo negativo!''. Nel sonno sentì il riso e l'allegro parlottare di Veslovskij e di Stepan Arkad'ic. Per un attimo aprì gli occhi: la luna era spuntata, ed essi stavano discorrendo nel vano del portone aperto, illuminato in pieno dalla luce lunare. Stepan Arkad'ic diceva qualcosa sulla freschezza della ragazza, paragonandola a una fresca nocciuola appena schiusa, e Veslovskij, ridendo del suo riso comunicativo, ripeteva le parole dettegli probabilmente da un contadino: ``Sbrigatela come puoi con una moglie tua''. Levin disse nel sonno: 

- Signori, a domani, appena si fa giorno! - e s'addormentò. 

\capitolo{XII}\label{xii-5} 

Svegliatosi sul far dell'alba, Levin provò a svegliare i compagni. Vasen'ka, sdraiato sul ventre e con una gamba allungata avvolta nella calza, dormiva così profondamente che non si poteva ottenere risposta da lui. Oblonskij, nel sonno, si rifiutava di andare via così presto. Perfino Laska, che dormiva acciambellata sull'orlo del fieno, si alzò di malavoglia, allungando e raddrizzando pigramente, una dopo l'altra, le zampe posteriori. Calzatosi, preso il fucile e aperta cautamente la porta del fienile che scricchiolava, Levin uscì sulla strada. I cocchieri dormivano accanto alle vetture, i cavalli sonnecchiavano. Uno di essi mangiava l'avena, spargendola con il muso per il trogolo. Nel cortile era ancora grigio 

- Com'è che ti sei alzato così presto, giaggiolo mio? - gli si rivolse con cordialità, come a un buon vecchio amico, la padrona di casa uscitagli incontro dalla izba. 

- Ma per andare a caccia, zia. Si va di qua alla palude? 

- Là di dietro, diritto; per le aie e per la canapa, uomo caro, c'è un viottolo. 

Camminando cauta coi piedi nudi abbronzati, la vecchia accompagnò Levin e levò la chiusura presso l'aia. 

- Diritto così e arriverai alla palude. I ragazzi iersera hanno spinto là le bestie. 

Laska correva avanti allegramente per il sentiero; Levin le teneva dietro con passo veloce, leggero, guardando continuamente il cielo. Voleva che il sole non sorgesse prima ch'egli fosse giunto alla palude, ma il sole non indugiava. La luna, che splendeva ancora quand'egli era uscito, adesso brillava opaca come un pezzo di mercurio; il lampeggiare mattutino, che prima non si poteva non distinguere, ora bisognava cercarlo; le macchie, prima indefinite nella campagna lontana, erano adesso già chiaramente visibili. Erano mucchi di segala. La rugiada, non ancora visibile senza la luce del sole sulla canapa alta, profumata, dalla quale era stata tolta quella secca, bagnava le gambe e il camiciotto di Levin più su della cintola. Nel silenzio trasparente del mattino si udivano i suoni più sottili. Un'ape piccola, col sibilo di una palla, volò accanto all'orecchio di Levin. Egli la guardò attento e ne scorse un'altra, poi una terza. Volavano tutte fuori dal graticcio di un'arnia e scomparivano sopra la canapa in direzione della palude. Il viottolo lo portò diritto alla palude, che si poteva riconoscere per i vapori che ne esalavano dove più densi, dove più radi, così che la càrice e i cespugli di citiso, come isolette, si cullavano su quel vapore. Al limite della palude e della strada i ragazzi e i contadini, che avevano fatto la guardia, erano sdraiati, e, innanzi l'alba, dormivano tutti sotto i gabbani. Non lontano da loro andavano tre cavalli impastoiati. Uno di essi faceva rumore coi ferri. Laska camminava accanto al padrone, chiedendo di andare avanti e guardandosi in giro. Oltrepassati i contadini che dormivano e giunto all'altezza del primo tratto paludoso, Levin esaminò i percussori e lasciò andare il cane. Uno dei cavalli, quello bruno, ben pasciuto, di un tre anni, visto il cane, fece uno scarto e, sollevata la coda, sbuffò. Gli altri cavalli si spaventarono anch'essi e, sguazzando per l'acquitrino con le zampe impastoiate e producendo con gli zoccoli tirati su dall'argilla spessa un suono simile a uno schiocco, si misero a saltar fuori della palude. Laska si fermò, guardando con irrisione i cavalli e interrogativamente Levin. Levin l'accarezzò e fischiò in segno che si poteva cominciare. 

Laska si mise a correre allegra e intenta per la melma che tremolava sotto di lei. Entrata di corsa nella palude, distinse subito, fra gli odori a lei noti della palude, della ruggine e l'odore estraneo dello sterco dei cavalli, l'odore degli uccelli, sparso per tutto il luogo, di quegli stessi uccelli odorosi che più degli altri l'agitavano. Qua e là, tra il muschio e le bardane della palude, quest'odore era molto forte, ma non si poteva stabilire, da quale parte venisse più forte e da quale più debole. Per trovare la direzione bisognava andare più lontano, sottovento. Senza avvertire il movimento delle proprie zampe, Laska a galoppo trattenuto, in modo da potersi fermare a ogni salto, se ve ne fosse stata la necessità, corse a destra, lontano dal venticello antelucano che spirava da oriente e si voltò verso il vento. Aspirata l'aria con le narici dilatate, sentì subito che non solo c'erano le orme, ma che essi erano là, davanti a lei, e non uno, ma molti. Laska diminuì la velocità della corsa. Essi erano là, ma dove esattamente, essa non poteva ancora precisarlo. Per trovare proprio il posto, aveva cominciato un giro, quando improvvisamente la voce del padrone la distrasse. ``Laska, qua!'' egli disse, indicandole l'altra parte. Essa si fermò un po', quasi a chiedergli se non fosse meglio fare così come aveva cominciato. Ma egli ripeté l'ordine con voce irritata, indicando un ammasso di montagnole ricoperto di acqua, dove non poteva esserci nulla. Essa obbedì, fingendo di cercare, per fargli piacere, rovistò le montagnole e tornò al posto di prima, e subito sentì di nuovo. Adesso, quando egli non la disturbava, essa sapeva che cosa fare e, senza guardare sotto le zampe, impigliandosi con stizza nelle montagnole ripide e cadendo nell'acqua, ma raddrizzandosi con le zampe agili, forti, cominciò un giro che doveva spiegarle tutto. L'odore degli uccelli la colpiva in maniera sempre più forte, sempre più precisa, e improvvisamente tutto le si fece chiaro, che, cioè, uno di quelli era là, dietro a un monticello, a cinque passi davanti a lei. Si fermò e si irrigidì in tutto il corpo. Sulle zampe basse non poteva veder nulla dinanzi a sé, ma dall'odore sapeva che esso si era posato non più lontano di cinque passi. Stava ritta, percependo sempre di più e pregustando l'attesa. La coda diritta era allungata, e tremava soltanto alla punta. La bocca era leggermente aperta, gli orecchi sollevati. Un orecchio s'era voltato ancora durante la corsa, ed essa respirava faticosamente, ma con cautela, e con cautela ancora maggiore si voltava a guardare, più con gli occhi che con la testa, il padrone. Costui, col suo solito viso, ma sempre con gli occhi minacciosi, camminava, inciampando, su per le montagnole e straordinariamente lento, come a lei sembrava. Le sembrava che camminasse lentamente, mentre egli correva. Notato questo speciale braccare di Laska (essa si stringeva tutta al terreno, come se avanzasse a grandi passi raccogliendo con le zampe posteriori, e apriva leggermente la bocca), Levin capì che fiutava le beccacce e, pregando Iddio per il successo, in particolare per il primo uccello, corse verso di essa. Accostatosi, cominciò a guardare dietro di sé dalla propria altezza, e vide con gli occhi quello ch'essa vedeva con il naso. Nel viottolo, fra le montagnole, sopra una di queste si vedeva una beccaccia bottaia. Col capo voltato stava in ascolto. Poi, assestate e richiuse le ali, dimenata goffamente la coda, scomparve dietro un angolo. 

- Pille, pille! - gridò Levin, spingendo Laska nel sedere. 

``Ma io non posso andare - pensava Laska. - Dove vado? Di qua li sento, ma se mi muovo in avanti non sentirò più nulla, né dove sono né chi sono''. Ma ecco ch'egli la spinse col ginocchio e con un mormorio agitato proferì: ``Pille, Lasocka, pille!''. 

``Ebbene, se lui lo vuole, lo farò, ma non rispondo più di me'' essa pensò e si lanciò in avanti fra le montagnole a gambe levate. Adesso non fiutava più e vedeva e sentiva soltanto, senza capir nulla. 

A dieci passi dal posto di prima, con lo zirlio chioccio e il rumore particolare delle beccacce, se ne alzò una. E subito, dopo lo sparo, cadde pesantemente, battendo il petto bianco contro la melma bagnata. Un'altra non aspettò e si levò dietro a Levin, senza bisogno del cane. 

Quando Levin si voltò, era già lontana, ma il colpo la raggiunse. Dopo aver volato per una ventina di passi, la seconda beccaccia si levò in alto come una palla lanciata e precipitò pesante sull'asciutto. 

``Ecco, ce ne saranno molte!'' pensava Levin, riponendo nel carniere le beccacce tiepide e grasse. 

- Ehi, Lasocka, ce ne saranno molte? 

Quando Levin, caricato il fucile, si mosse per andare avanti, il sole, non ancora visibile dietro le piccole nuvole, s'era già levato. La luna, perduto tutto il suo splendore, sbiancava nel cielo come una piccola nube; di stelle non se ne vedeva neanche più una. I tratti acquitrinosi, che prima si inargentavano di rugiada, ora si facevano d'oro. La ruggine era tutta ambrata. L'azzurro delle erbe tramutava in un verde giallastro. Gli uccelli di palude brulicavano sui piccoli cespugli che, presso il ruscello, brillavano di rugiada e proiettavano un'ombra lunga. Un avvoltoio s'era svegliato e s'era andato a posare sopra una bica, voltando il capo da un lato all'altro guardando scontento la palude. Le cornacchie volavano sui campi e un ragazzetto scalzo spingeva già innanzi a sé i cavalli verso un vecchio, che s'era levato di sotto al gabbano e si grattava. Il fumo degli spari biancheggiava come latte su per il verde dell'erba. Uno dei ragazzi venne, correndo, da Levin. 

- Zio, iersera qua c'erano le anitre! - gli gridò, e gli tenne dietro a distanza. 

E a Levin fece molto piacere colpire ancora, proprio là, una dietro l'altra, tre beccacce, alla presenza di quel ragazzetto che esprimeva la sua approvazione. 

\capitolo{XIII}\label{xiii-5} 

Il presagio venatorio: se non ti lasci scappare la prima bestia o il primo uccello, la caccia sarà buona, si mostrò veritiero. 

Stanco, affamato, felice, Levin, verso le dieci del mattino, dopo aver camminato per trenta verste, con diciannove capi di selvaggina reale e un'anatra, che aveva legata alla cintura, giacché non entrava più nel carniere, tornò al rifugio. I compagni s'erano svegliati da un pezzo e avevano avuto il tempo di farsi venir fame e di far colazione. 

- Aspettate, aspettate, lo so che sono diciannove - diceva Levin, ricontando per la seconda volta le beccacce e le beccacce bottaie, che non avevano più quell'aspetto importante che avevano quando volavano, così ritorte e rinsecchite com'erano, col sangue coagulato e i capini voltati da un lato. 

Il conto era giusto, e l'invidia di Stepan Arkad'ic fece piacere a Levin. Gli fece anche piacere trovare, appena giunto all'alloggio, l'inserviente inviato da Kitty, già là con un biglietto. 

``Sto perfettamente bene e di ottimo umore. E se temi per me, puoi star più tranquillo di prima. Ho una nuova guardia del corpo, Mar'ja Vlas'evna - era la levatrice, un personaggio nuovo, importante nella vita familiare di Levin. - È venuta a farmi visita. M'ha trovato in ottima salute e noi l'abbiamo trattenuta fino al tuo arrivo. Stiamo tutti di ottimo umore, bene in salute, e tu, per favore, non affrettarti, e se la caccia è buona, rimani ancora un giorno''. 

Queste due gioie, la caccia fortunata e il biglietto della moglie, erano tanto grandi che i due piccoli dispiaceri, seguìti alla caccia, passarono con facilità per Levin. Uno consisteva nel fatto che il bilancino, il sauro che il giorno prima evidentemente aveva faticato troppo, non mangiava la biada e s'era fatto triste. Il cocchiere diceva che era sfiancato. 

- Ieri l'avete fiaccato, Konstantin Dmitric - diceva. - E come l'avete spinto per dieci verste fuori di strada! 

L'altro disappunto che distrusse sulle prime la sua buona disposizione d'animo, ma che poi lo fece rider molto, consistette nel fatto che di tutte le provviste, date da Kitty con un'abbondanza tale che sembrava non si potessero finire in una settimana, non era rimasto più nulla. Ritornando stanco ed affamato dalla caccia, Levin sognava in modo così preciso gli sfogliantini che, accostandosi all'alloggio, ne pregustava l'odore e il sapore in bocca, così come Laska fiutava la selvaggina, e ordinò subito a Filipp di portarglieli. Risultò che non solo di sfogliantini, ma neanche di pollastri ce n'erano più. 

- Uhm, che appetito! - disse Stepan Arkad'ic, indicando Vasen'ka Veslovskij. - Io non soffro di inappetenza, ma questo qui è sorprendente\ldots{} 

- E che fare? - disse Levin, guardando torvo Veslovskij. - Filipp, dammi del manzo. 

- Il manzo l'hanno mangiato e l'osso l'hanno dato ai cani - disse Filipp. 

Levin provò un così gran dispetto che disse con stizza: 

- E mi avessero almeno lasciato qualcosa! - e gli venne voglia di piangere. - Allora, sventra un po' di caccia - disse con voce tremante a Filipp, cercando di non guardare Vasen'ka - e mettici sopra l'ortica. E per me, chiedi almeno del latte. 

Dopo, quando si fu saziato, si vergognò di aver mostrato la propria furia a una persona estranea, e prese a ridere del proprio affamato risentimento. 

La sera fecero ancora una battuta, nella quale anche Veslovskij ammazzò alcuni capi, e nella notte tornarono a casa. 

La via del ritorno fu altrettanto amena come era stata quella dell'andata. Veslovskij ora cantava, ora ricordava con piacere le proprie avventure con i contadini che gli avevano offerto la vodka, e gli avevano detto: ``non discutere'', ora le proprie avventure notturne con le nocciuole e la ragazza di casa e col contadino che gli aveva chiesto se era ammogliato e, avendo sentito che non era ammogliato, gli aveva detto: ``E tu non desiderare le mogli degli altri, ma piuttosto sbrigati a procurartene una per te''. Queste parole in particolare facevano ridere Veslovskij. 

- In generale, sono straordinariamente soddisfatto della nostra gita. E voi Levin? 

- Contentissimo - disse con sincerità Levin che era in particolar modo felice, non soltanto di non sentire più quell'ostilità che a casa aveva provato verso Vasen'ka, ma di sentire, al contrario, verso di lui la più cordiale disposizione d'animo. 

\capitolo{XIV}\label{xiv-5} 

Il giorno dopo alle dieci, Levin, dopo aver fatto il giro dell'azienda, bussò alla camera dove aveva dormito Vasen'ka. 

- Entrez! - gli gridò Vasen'ka. - Perdonatemi, non ho ancora finito le mie ablutions - disse sorridendo, ritto dinanzi a lui, con la sola biancheria indosso. 

- Non vi preoccupate, vi prego. - Levin sedette accanto alla finestra. - Avete dormito bene? 

- Come un morto. E oggi che giornata è per la caccia? 

- Cosa bevete: tè o caffè? 

- Né l'uno né l'altro. Faccio colazione. Mi vergogno proprio. Le signore, penso, si saranno alzate. Adesso è bellissimo passeggiare. Fatemi vedere i cavalli. 

Dopo aver passeggiato per il giardino, dopo esser stati nella scuderia, e aver perfino fatto insieme la ginnastica alle sbarre, Levin tornò a casa con l'ospite ed entrò nel salotto con lui. 

- Abbiamo cacciato magnificamente, e quante impressioni! - disse Veslovskij avvicinandosi a Kitty, che sedeva presso il samovar. - Che peccato che le signore siano private di questi piaceri! 

``Be', che c'è di strano, bisogna pure ch'egli parli in qualche modo con la padrona di casa'' si diceva Levin. Gli era parso di veder di nuovo qualcosa nel sorriso, in quell'espressione trionfante con cui l'ospite si era rivolto a Kitty\ldots{} 

La principessa, seduta dall'altra parte della tavola con Mar'ja Vlas'evna e Stepan Arkad'ic, chiamò a sé Levin e avviò una conversazione con lui sul trasferimento a Mosca per il parto di Kitty e sulla preparazione della casa. Per Levin, come al tempo del matrimonio, era spiacevole qualsiasi preparativo che offendesse, con la sua piccineria, la grandiosità di quello che si compiva; così ancora più offensivi gli sembravano i preparativi per il prossimo parto, il cui tempo veniva calcolato in un certo modo, sulle dita. Aveva cercato sempre di non ascoltare questi discorsi sul metodo di fasciare il futuro bambino, aveva cercato di voltarsi dall'altra parte e di non vedere certe misteriose infinite strisce a maglia, certi triangoli di tela, ai quali Dolly attribuiva una particolare importanza, e cose simili. L'avvenimento della nascita di un figlio (era sicuro che sarebbe stato un figlio), che gli avevano promesso, ma al quale tuttavia non poteva credere, tanto gli sembrava straordinario, gli appariva da un lato una felicità così immensa e perciò impossibile, dall'altro lato un avvenimento così misterioso, che quella pretesa conoscenza di ciò che sarebbe stato, e di conseguenza la preparazione come a qualcosa di ordinario, opera degli uomini stessi, gli appariva disgustosa e umiliante. 

Ma la principessa non capiva i suoi sentimenti e interpretava la sua pigrizia a pensare e a parlare di questo per leggerezza e indifferenza; perciò non gli dava tregua. Aveva dato l'incarico a Stepan Arkad'ic di andare a vedere un appartamento e ora aveva chiamato a sé Levin. 

- Io non so nulla, principessa. Fate come volete - egli diceva. 

- Bisogna decidere a quando il trasferimento. 

- Io, davvero, non lo so. So che nascono milioni di bambini senza Mosca e senza medici\ldots{} perché mai\ldots{} 

- Ma se è così\ldots{} 

- Ma no, come vuole Kitty. 

- Con Kitty non se ne può parlare! Che vuoi, che la spaventi? Ecco, questa primavera Natalie Golicyna è morta per colpa d'un cattivo ostetrico. 

- Come direte voi, così farò - disse lui cupo. 

La principessa cominciò a parlargli, ma egli non l'ascoltava. Sebbene questa conversazione lo sconvolgesse, s'era fatto cupo non per la conversazione, ma per quello che vedeva presso il samovar. 

``No, non è possibile'' pensava guardando Vasen'ka che s'era chinato verso Kitty, dicendole qualcosa col suo bonario sorriso e lei che arrossiva e si agitava\ldots{} 

C'era qualcosa d'impuro nell'atteggiamento di Vasen'ka, nel suo sguardo, nel sorriso. Levin vedeva perfino qualcosa d'impuro nell'atteggiamento e nello sguardo di Kitty. E di nuovo la luce si oscurò dinanzi ai suoi occhi. Di nuovo, come il giorno prima, a un tratto, senza il più piccolo passaggio, si sentì gettato giù dall'altezza della propria felicità, della propria calma e dignità in un abisso di disperazione, di rancore, di umiliazione. Di nuovo tutto e tutti gli divennero disgustosi. 

- Allora, principessa, fate come volete - disse, voltandosi di nuovo a guardare. 

- ``Tiara del Monomaco, sei pesante!'' gli disse, scherzando, Stepan Arkad'ic, alludendo, evidentemente, non alla sola conversazione con la principessa, ma all'agitazione di Levin che egli aveva notata. - Come hai fatto tardi, oggi Dolly! 

Tutti si alzarono per accogliere Dar'ja Aleksandrovna. Vasen'ka si alzò per un attimo solo e, con quella mancanza di cortesia verso le signore, propria dei giovanotti moderni, s'inchinò appena e riprese il discorso, mettendosi a ridere per qualcosa. 

- Maša mi ha tormentato. Ha dormito male e oggi è terribilmente capricciosa - disse Dolly. 

La conversazione avviata da Vasen'ka con Kitty riguardava di nuovo l'argomento del giorno precedente, cioè Anna e la questione se l'amore possa sovrapporsi alle convenienze sociali. Per Kitty questa conversazione non era gradita, l'agitava e per la sostanza stessa e per il tono col quale era condotta, e soprattutto perché sapeva già come avrebbe agito su suo marito. Ma era troppo semplice e innocente per saper porre termine alla conversazione e finanche per nascondere il piacere esteriore che le procurava l'evidente premura di quel giovane. Voleva porre termine alla conversazione, ma non sapeva cosa dovesse fare. Qualunque cosa avesse fatto, sarebbe stata notata da suo marito e, lo sapeva, interpretata in senso cattivo. E realmente quando domandò a Dolly cosa avesse Maša mentre Vasen'ka, aspettando la fine di questa interruzione per lui noiosa, guardava con indifferenza Dolly, questa domanda sembrò a Levin un'astuzia poco naturale, odiosa. 

- Ebbene, andiamo a cercar funghi, oggi? - domandò Dolly. 

- Andiamo, vi prego, verrò anch'io - disse Kitty e arrossì. Voleva domandare a Vasen'ka per cortesia se sarebbe andato, e non lo domandò. - Tu dove vai, Kostja? - ella domandò al marito con aria colpevole, mentre egli, con fare deciso, le passava accanto. Quest'espressione colpevole confermò tutti i dubbi di lui. 

- Quando non c'ero è arrivato il meccanico, non l'ho ancora visto - disse, senza guardarla. 

Egli scese giù, ma non aveva ancora fatto in tempo a uscire dallo studio che sentì i noti passi della moglie, che veniva verso di lui cauta e frettolosa. 

- Che vuoi? - le disse asciutto. - Siamo occupati. 

- Scusatemi - ella disse rivolta al meccanico tedesco: - devo dire due parole a mio marito. 

Il tedesco voleva andarsene, ma Levin gli disse: 

- Non vi incomodate. 

- Il treno è alle tre? - domandò il tedesco - che non abbia a giungere in ritardo. 

Levin non gli rispose e uscì lui stesso con la moglie. 

- Be', cosa avete da dirmi? - egli disse in francese. 

Egli non la guardava e non voleva accorgersi che lei, nelle sue condizioni, tremava tutta in viso e aveva un'aria pietosa, annientata. 

- Io\ldots{} io voglio dire che così non si può vivere, che è un tormento\ldots{} - ella pronunciò. 

- Qui nella dispensa c'è la servitù - egli disse irritato: - non fate scene. 

- Su, andiamo di qua! 

Essi stavano in piedi, in una stanza di passaggio. Kitty voleva entrare in quella accanto, ma là l'inglese dava lezione a Tanja. 

- Su, andiamo in giardino! 

In giardino s'imbatterono in un contadino che ripuliva un viottolo. E, senza pensare che il contadino vedeva il viso lacrimoso di lei e quello agitato di lui, senza pensare che avevano l'aspetto di persone che venissero fuori da una qualche sventura, andavano avanti a passi svelti, sentendo che dovevano dirsi tutto e disingannarsi a vicenda, stare un po' soli insieme e con questo liberarsi del tormento che provavano tutti e due. 

- Così non si può vivere! È un tormento! Io soffro, tu soffri. Per cosa! - disse lei quando finalmente raggiunsero una panca solitaria all'angolo del viale dei tigli. 

- Ma tu dimmi una cosa sola: c'era nel suo tono qualcosa di sconveniente, di impuro, di umiliante e pauroso? - egli diceva, ponendosi dinanzi a lei di nuovo in quella stessa posa, coi pugni davanti al petto, così come s'era messo dinanzi a lei quella notte. 

- C'era - diceva lei con voce tremante. - Ma, Kostja, non vedi forse che io non sono colpevole? Io fin dalla mattina volevo prendere un tono diverso, ma queste persone\ldots{} Perché è venuto? Come eravamo felici! - ella diceva, soffocando per i singhiozzi che sollevavano tutto il suo corpo ingrossato. 

Il giardiniere si accorse con stupore, che, sebbene nulla li inseguisse e non ci fosse motivo alcuno di fuggire e sebbene nulla di particolarmente gioioso avessero potuto trovare sulla panchina - il giardiniere si accorse che tornavano a casa, passandogli accanto, con il viso rasserenato, luminoso. 

\capitolo{XV}\label{xv-5} 

Accompagnata la moglie di sopra, Levin andò nell'appartamento di Dolly. Dar'ja Aleksandrovna, per conto suo, era molto amareggiata quel giorno. Camminava per la stanza e diceva arrabbiata alla bambina che stava in piedi in un angolo e piangeva: 

- E starai nell'angolo tutto il giorno, e pranzerai sola, e non vedrai neanche una bambola, e il vestito nuovo non te lo farò - diceva, non sapendo più come punirla. - No, è una bambina cattiva! - disse rivolta a Levin. - Di dove vengono queste inclinazioni disgustose? 

- Ma che ha fatto mai? - chiese Levin alquanto indifferente, e, desideroso di trovar consiglio per la cosa sua, s'irritò d'esser capitato fuor di proposito. 

- Lei e Griša sono andati dove ci sono i lamponi e là\ldots{} non posso neppur dire quello che lei ha fatto. Rimpiango mille volte miss Elliot. Questa qui non bada a nulla. Figurez vous, que la petite\ldots{} - E Dar'ja Aleksandrovna raccontò il delitto di Maša. 

- Ma questo non dimostra nulla, queste sono tutt'altro che inclinazioni cattive, questa è semplice birichineria - la tranquillizzava Levin. 

- Ma tu sei un po' sconvolto! Perché sei venuto? - chiese Dolly. - Che succede là? 

E dal tono di questa domanda, Levin sentì che gli sarebbe stato facile metter fuori quello che aveva in mente di dire. 

- Non sono stato là, sono stato in giardino con Kitty. Abbiamo litigato per la seconda volta da che\ldots{} Stiva è arrivato. 

Dolly lo guardava con gli occhi intelligenti che intendevano. 

- Su, di', con una mano sul cuore, c'era\ldots{} non in Kitty, ma in quel signore, un tono tale da essere spiacevole, non solo spiacevole, ma spaventoso, offensivo per un marito? 

- Be', come dirti\ldots{} Resta, resta nell'angolo! - si rivolse a Maša che, visto un sorriso appena percettibile sul viso della madre, stava per voltarsi. - L'opinione del mondo sarebbe che egli si comporta come si comportano tutti i giovanotti. Il fait la cour à une jeune et jolie femme. E un marito mondano deve esserne semplicemente lusingato. 

- Sì, sì - disse cupo Levin - ma tu l'hai notato? 

- Non solo io, ma Stiva l'ha notato. M'ha proprio detto dopo il tè: ``Je crois que Veslovskij fait un petit brin de cour à Kitty''. 

- E allora benissimo, ora sono tranquillo. Lo caccerò fuori - disse Levin. 

- Ma sei impazzito? - gridò Dolly con orrore. - Che hai, Kostja, ritorna in te - disse ella, ridendo. - Su, adesso puoi andare da Fanny - ella disse a Maša. - No, se vuoi proprio, allora lo dirò a Stiva. Lui lo porterà via. Si può dire che tu aspetti ospiti. In fondo, è di troppo in casa nostra. 

- No, no, faccio da me. 

- Ma ti bisticcerai?\ldots{} 

- Niente affatto. Per me sarà molto divertente, realmente divertente - disse Levin con gli occhi che gli brillavano. - Su, perdonala, Dolly! Non lo farà più - disse egli della piccola delinquente, che non andava da Fanny e stava lì in piedi, indecisa dinanzi alla madre, aspettando e cercando il suo sguardo di sotto in su. 

La madre la guardò. La bambina scoppiò in singhiozzi, si sprofondò col viso nelle ginocchia della mamma, e Dolly le mise sul capo la sua mano magra, tenera. 

``E che c'è di comune fra noi e lui?'' pensò Levin e andò in cerca di Veslovskij. 

Passando per l'anticamera, ordinò di far attaccare il calesse, per andare alla stazione. 

- Ieri s'è rotta una molla - rispose il servitore. 

- Su, allora la carretta, ma presto. Dov'è l'ospite? 

- Il signore è andato in camera sua. 

Levin trovò Vasen'ka che, tolta dalla valigia tutta la roba, e sparpagliate le nuove romanze, provava delle ghette per andare a cavallo. 

O che ci fosse qualcosa di particolare nel viso di Levin o che lo stesso Vasen'ka avesse sentito che ce petit brin de cour, ch'egli aveva vagheggiato, era fuor di posto in quella famiglia, certo egli fu un poco (quanto può esserlo un uomo di mondo) confuso dall'entrata di Levin. 

- Montate a cavallo con le ghette? 

- Sì, è molto più igienico - disse Vasen'ka, ponendo la gamba grassa su di una sedia, agganciando la fibbia di sotto e sorridendo allegramente, di cuore. 

Era senza dubbio un buon ragazzo, e Levin provò pena per lui e vergogna per sé, padrone di casa, quando notò la timidezza nello sguardo di Vasen'ka. 

Sulla tavola giaceva un pezzo di bastone che la mattina avevano spezzato insieme nel far ginnastica, provando a sollevare le sbarre rigonfie per l'umidità. Levin prese in mano questo pezzo di legno e cominciò a romperne l'estremità già spaccata, senza sapere da che parte cominciare a parlare. 

- Volevo\ldots{} - Stava già per tacere, ma ad un tratto, ricordatosi di Kitty e di tutto quello che c'era stato, disse guardandolo risoluto negli occhi: - Ho ordinato che vi attaccassero i cavalli. 

- Sarebbe a dire, come? - cominciò Vasen'ka con sorpresa. - E dove si deve andare? 

- Per voi, alla stazione - disse torvo Levin, tormentando il bastone. 

- Forse voi partite, o è accaduto qualcosa? 

- È accaduto che aspetto ospiti - disse Levin, rompendo sempre più in fretta con le dita forti le estremità del bastone che s'era spaccato. - Anzi, non aspetto ospiti e non è accaduto nulla, ma vi prego di partire. Potete spiegare come volete la mia sgarberia. 

Vasen'ka si raddrizzò. 

- Io prego voi di spiegarmi\ldots{} - disse con dignità, avendo finalmente capito. 

- Non posso spiegarvelo - cominciò a dire piano e lento Levin, cercando di nascondere il tremito dei propri zigomi. - Ed è meglio che non domandiate. 

E siccome le estremità scheggiate erano state già tolte, Levin s'attaccò con le dita alle estremità grosse, spaccò il bastone e raccolse con cura la punta che era caduta. 

Probabilmente la vista di quelle mani tese, di quegli stessi muscoli che la mattina aveva palpato facendo ginnastica, e degli occhi scintillanti, la voce piana e gli zigomi tremanti convinsero Vasen'ka più delle parole. Dopo aver alzato le spalle e sorriso con sprezzo, egli s'inchinò. 

- Non posso vedere Oblonskij? 

L'alzata di spalle e il sorriso non irritarono Levin. ``Che gli rimane da fare?'' egli pensò. 

- Ve lo mando subito. 

- Che cosa insensata! - diceva Stepan Arkad'ic, dopo aver saputo dall'amico che lo si cacciava di casa e avendo trovato Levin in giardino, dove passeggiava, in attesa della partenza dell'ospite. - Mais c'est ridicule! Ma che mosca t'ha mai pizzicato? Mais c'est du dernier ridicule! Cosa mai ti è sembrato, se un giovanotto\ldots{} 

Ma il punto in cui la mosca aveva pizzicato Levin doleva ancora, si vedeva, perché egli di nuovo impallidì, quando Stepan Arkad'ic cercò di chieder ragione, e lo interruppe in fretta. 

- Per favore, non chiedermi la ragione! non posso fare altrimenti. Me ne vergogno molto dinanzi a te e dinanzi a lui. Ma per lui, penso, non sarà un gran dolore partire e per me e per mia moglie la sua presenza è spiacevole. 

- Ma per lui è offensivo! Et puis c'est ridicule! 

- E per me è offensivo e tormentoso! E io non sono colpevole di nulla e non ho ragione di soffrire. 

- Ma questo non me l'aspettavo da te! On peut être jaloux, mais à ce point c'est du dernier ridicule! 

Levin si voltò in fretta e andò via in fondo al viale, continuando a camminare solo, avanti e indietro. Presto sentì il rumore della carretta e vide di là dagli alberi Vasen'ka col berrettino alla scozzese seduto sul fieno (per colmo di sventura sulla carretta non c'era il sedile), mentre passava per il viale, sobbalzando alle scosse. 

``Che altro c'è?'' pensò Levin quando un servitore, uscito di corsa dalla casa, fermò la carretta. Era il meccanico, di cui Levin si era del tutto dimenticato. Il meccanico, salutando, disse qualcosa a Veslovskij, poi montò sulla carretta e partirono insieme. 

Stepan Arkad'ic e la principessa erano indignati dell'azione di Levin. E lui stesso si sentiva non solo ridicule al sommo grado, ma anche completamente colpevole e coperto di vergogna; ma, ricordando quello che lui e sua moglie avevano sofferto, e domandandosi come avrebbe agito una seconda volta, si rispondeva che avrebbe agito proprio allo stesso modo. 

Malgrado tutto questo, alla fine di quel giorno, tutti, tranne la principessa, che non perdonava quell'azione a Levin, divennero straordinariamente allegri e animati, come bambini dopo una punizione o come grandi dopo un faticoso ricevimento ufficiale. Così la sera, in assenza della principessa, si parlò della cacciata di Vasen'ka, come di un avvenimento lontano. E Dolly, che aveva ricevuto dal padre l'arte di raccontare le cose con spirito, faceva morir dal ridere Varen'ka, quando per la terza o quarta volta, sempre con nuove aggiunte umoristiche, raccontava come, proprio mentre lei si accingeva a mettersi dei nuovi nastri in omaggio all'ospite, e stava per uscire in salotto, ecco che, all'improvviso, aveva sentito il rumore della carretta. E chi c'era mai nella carretta? Vasen'ka in persona, col berrettino scozzese, le romanze e le ghette, era seduto sul fieno. 

- Avesse almeno fatto attaccar la carrozza! No, e poi sento: ``Aspettate!''. Be', penso, avranno avuto compassione. Guardo: gli hanno messo a sedere accanto quel grassone di tedesco e l'hanno portato via\ldots{} E i miei nastri sono andati perduti. 

\capitolo{XVI}\label{xvi-5} 

Dar'ja Aleksandrovna mise in atto la sua intenzione e andò da Anna. Le spiaceva molto addolorare la sorella e far cosa sgradita al marito di lei; capiva come avessero ragione i Levin a non desiderare d'avere nessun rapporto con i Vronskij; ma riteneva suo dovere stare un po' da Anna e dimostrarle che i suoi sentimenti non potevano essere cambiati, malgrado la mutata situazione di lei. 

Per non dipendere dai Levin per quel viaggio, Dar'ja Aleksandrovna mandò in paese a noleggiare i cavalli; ma Levin, saputolo, venne da lei a fare le sue rimostranze. 

- Ma perché pensi che mi spiaccia il tuo viaggio? E se anche ciò fosse mi spiace ancor più se tu non prendi i miei cavalli - egli diceva. - Non m'hai detto neppure una volta che eri decisa ad andare. E poi, prenderli in affitto al paese, in primo luogo mi rincresce, e poi anche se prenderanno l'incarico, non ti porteranno fin là. Io li ho i cavalli. E se non vuoi darmi un dispiacere, prendi i miei. 

Dar'ja Aleksandrovna aveva dovuto acconsentire, e, il giorno fissato, Levin preparò per la cognata un tiro a quattro e dei cavalli di cambio, scegliendoli fra quelli da lavoro e da sella, molto brutti, ma che potevano portare a destinazione Dar'ja Aleksandrovna in un solo giorno. In quel momento, in cui i cavalli servivano e per la principessa che partiva e per la levatrice, la cosa era stata difficile per Levin, ma per dovere di ospitalità egli non poteva permettere a Dar'ja Aleksandrovna di noleggiare i cavalli, inoltre, sapeva che i venti rubli che le avrebbero chiesto per quel viaggio avevano grande peso per lei, e le faccende finanziarie di Dar'ja Aleksandrovna, così dissestate, erano sentite dai Levin come loro proprie. 

Dar'ja Aleksandrovna, per consiglio di Levin, partì prima dell'alba. La strada era buona, la vettura ottima, i cavalli correvano allegramente, e a cassetta, oltre il cocchiere, sedeva, invece del servitore, lo scrivano, mandato da Levin come scorta. Dar'ja Aleksandrovna sonnecchiava e si svegliò solo all'avvicinarsi della locanda, dove bisognava dare il cambio ai cavalli. 

Dopo aver bevuto il tè da quello stesso ricco contadino-proprietario dal quale s'era fermato Levin nel recarsi da Svijazskij, e dopo aver chiacchierato con le donne dei bambini e col vecchio del conte Vronskij, ch'egli lodava molto, Dar'ja Aleksandrovna, alle dieci, proseguì. A casa, per le preoccupazioni dei bambini, non aveva mai il tempo di riflettere. In compenso adesso, in quel lungo spazio di quattro ore, tutti i pensieri prima contenuti, le si affollarono a un tratto nella mente. Ed ella ripensò a tutta la sua vita come non mai prima, e dai lati più diversi. Erano strani anche per lei i suoi pensieri. Pensò dapprima ai bambini, per i quali, benché la principessa e soprattutto Kitty (in lei aveva più fiducia) avessero promesso di sorvegliarli, era tuttavia inquieta. ``Che Maša non incominci di nuovo a far birichinate, che il cavallo non faccia male a Griša, e che lo stomaco di Lily non si disturbi di più''. Ma poi le questioni del presente cominciarono a tramutarsi nelle questioni dell'immediato futuro. Si mise a pensare che a Mosca, per quell'inverno, bisognava prendere in affitto un appartamento nuovo; cambiare la mobilia in salotto e fare una pelliccia nuova alla figlia maggiore. Poi cominciarono ad apparire le questioni d'un futuro più lontano: come avrebbe sistemato in società i figliuoli. ``Per le bambine, ancora non è nulla - pensava - ma i ragazzi?''. 

``Va bene, io ora mi occupo di Griša, ma è solo perché ora sono libera e non devo partorire. Su Stiva, naturalmente, non c'è da fare nessun assegnamento. E io con l'aiuto di persone buone, li sistemerò nel mondo; ma se c'è di nuovo un parto\ldots{}''. E le venne l'idea che era stato detto ingiustamente che una maledizione gravava sulla donna, perché generasse i figli fra i tormenti: ``Partorire non è niente, ma essere incinta, ecco quello ch'è tormento'' ella pensò ricordandosi della sua ultima gravidanza e della morte dell'ultimo bambino. E le tornò in mente la conversazione con una giovane là alla locanda. Alla domanda se avesse bambini, la bella sposa aveva risposto allegra: 

- Ho avuto una bambina, ma Dio mi ha liberata, l'ho sotterrata a quaresima. 

- N'hai sofferto molto, vero? - chiese Dar'ja Aleksandrovna. 

- E perché soffrire? Il vecchio, anche così, di nipoti ne ha tanti. È soltanto una preoccupazione. Non puoi lavorare, né fare altro. Non è che un legame. 

Questa risposta era parsa ripugnante a Dar'ja Aleksandrovna, malgrado l'aspetto buono della giovane donna; ma ora ella si ricordò involontariamente di quelle parole. In quelle ciniche parole c'era anche una parte di verità. 

``Già, così sempre - pensava Dar'ja Aleksandrovna, dopo aver dato uno sguardo a tutta la sua vita di quei quindici anni di matrimonio: - gravidanze, nausee, confusione mentale, indifferenza e, soprattutto, la deformità. Kitty, anche Kitty così giovane e carina, è diventata brutta e io, incinta, divento mostruosa, lo so. Il parto, le sofferenze mostruose, quegli ultimi momenti\ldots{} poi l'allattamento, quelle notti insonni, quei dolori terribili\ldots{}''. 

Dar'ja Aleksandrovna rabbrividì al solo ricordo del dolore ai capezzoli screpolati, che provava quasi ad ogni bambino. ``Poi le malattie dei bambini, quest'eterno terrore; poi l'educazione, le inclinazioni cattive - ricordò il delitto della piccola Maša fra i lamponi - lo studio, il latino, tutto questo è così incomprensibile e difficile. E al di sopra di tutto, la morte di questi stessi bambini''. E di nuovo nella sua mente si sollevò il ricordo crudele che sempre opprimeva il suo cuore di mamma, la morte dell'ultima bambina lattante, morta di crup, il funerale, l'indifferenza di tutti per la piccola bara rosa e il proprio dolore, solo, che le spezzava il cuore, dinanzi a quella minuta e pallida fronte con le tempie dai capelli ondulati, e la piccola bocca aperta e stupita, che appariva dalla bara nel momento in cui la coprivano d'un coperchio rosa con una croce di gallone. 

``E tutto questo perché? Che cosa risulterà mai da tutto questo? Che io passerò la mia vita senza un momento di pace, ora incinta, ora in periodo di allattamento, eternamente irritata, scontenta, tormentata io stessa e tormentatrice degli altri, invisa a mio marito, e che cresceranno dei figli infelici, male educati e poveri. E ora, se non avessimo trascorso questa estate dai Levin, non so come ce la saremmo passata. S'intende, Kostja e Kitty sono così delicati da non farcene accorgere; ma questo non può durare. Cominceranno ad avere anche loro dei bambini, non potranno più aiutarci e anche adesso sono a disagio. Che forse, ci aiuterà papà, cui non è rimasto quasi nulla? E, per dare una sistemazione ai miei figliuoli, non posso farlo da sola, ma soltanto con l'aiuto degli altri, attraverso umiliazioni. Ma via, pensiamo a qualcosa di più felice: di bambini non ne moriranno più e io in qualche modo li educherò. Nel caso migliore non saranno dei furfanti, ecco. Ecco tutto quello che posso desiderare. Per tutto questo, quanti tormenti, quante fatiche\ldots{} Tutta una vita rovinata!''. Le tornò di nuovo in mente quello che aveva detto la giovane donna, e di nuovo provò ribrezzo a rammentarsene; ma non poteva non convenire che in quelle parole c'era anche una parte di verità brutale. 

- È lontano, Michajla? - domandò Dar'ja Aleksandrovna allo scrivano, per distrarsi dai pensieri che la spaventavano. 

- Da questo villaggio, dicono, sette verste. 

La vettura, per la strada del villaggio, scendeva verso un ponticello. Per il ponte, discorrendo ad alta voce e allegramente, una folla di donne andava colle funi dei covoni legate dietro alle spalle. Le donne si fermarono sul ponte esaminando con curiosità la vettura. Tutti i visi rivolti verso di lei sembrarono a Dar'ja Aleksandrovna sani, allegri, quasi eccitanti in lei la gioia di vivere: ``Tutti vivono, tutti godono la vita - seguitò a pensare Dar'ja Aleksandrovna, oltrepassando le donne, e uscendo su di un poggio, cullata di nuovo piacevolmente sulle morbide molle del vecchio carrozzino al trotto - e io, liberata, come da una prigione, da un mondo che mi uccide di preoccupazioni, soltanto adesso, per un attimo, sono tornata in me. Tutti vivono: e queste donne e Natalie mia sorella, e Varen'ka e Anna dalla quale vado, soltanto io no''. 

``E si scagliano contro Anna. Perché mai? Io forse sono migliore? Io almeno ho un marito che amo. Non così come vorrei amarlo, ma lo amo e Anna non amava il suo. In che cosa è mai colpevole? Vuol vivere. Iddio, questo, ce l'ha messo dentro l'anima. Può darsi benissimo che anch'io avrei fatto lo stesso. E finora non so se ho fatto bene ad ascoltarla in quel momento terribile quando venne a Mosca. Allora io dovevo abbandonare mio marito e cominciare a vivere di nuovo. Potevo amare ed essere amata davvero. E adesso, è forse meglio? Non lo stimo. Mi è necessario - ella pensava del marito - e lo sopporto. È forse meglio? Allora potevo ancora piacere, mi rimaneva la bellezza'' seguitò a pensare Dar'ja Aleksandrovna, e le venne voglia di guardarsi nello specchio. Aveva uno specchietto da viaggio nella borsa e avrebbe voluto tirarlo fuori; ma guardando la schiena del cocchiere e quella dello scrivano che si dondolava, sentì che si sarebbe vergognata se uno di loro si fosse voltato, e non tirò fuori lo specchio. 

Ma pur senza guardarsi nello specchio, pensava che, adesso, era troppo tardi; e ricordò Sergej Ivanovic, particolarmente gentile con lei, un amico di Stiva, il buon Turovcyn che insieme con lei aveva curato i bambini durante la scarlattina e s'era innamorato di lei. E c'era stato anche un uomo proprio giovane che, come aveva detto scherzando il marito, riteneva ch'ella fosse la più bella delle sorelle. E i romanzi più appassionati e impossibili si presentavano alla mente di Dar'ja Aleksandrovna. ``Anna ha agito benissimo, e io, poi, non starò certo a rimproverarla. È felice, fa la felicità d'un'altra persona e non è avvilita come me, ma probabilmente fresca, sveglia, aperta a tutto, sempre allo stesso modo'' pensava Dar'ja Aleksandrovna, e un sorriso malizioso le increspava le labbra, proprio perché, pensando al romanzo di Anna, parallelamente ad esso, Dar'ja Aleksandrovna si figurava un suo romanzo, quasi simile, con un personaggio anonimo e collettivo che fosse innamorato di lei. Come Anna, ella avrebbe confessato tutto al marito. E lo stupore e la confusione di Stepan Arkad'ic, a questa notizia, la facevano sorridere. 

In queste fantasticherie giunse alla svolta che, dalla strada maestra, portava a Vozdvizenskoe. 

\capitolo{XVII}\label{xvii-5} 

Il cocchiere fermò il tiro a quattro e si voltò a destra verso un campo di segala, dove stavano seduti, accanto a un carro, dei contadini. Lo scrivano parve voler scendere, ma poi cambiò idea e chiamò imperiosamente un contadino, facendogli segno di accostarsi. Quel po' di vento che c'era stato durante il cammino, era cessato quando si fermarono; i tafani si attaccavano ai cavalli sudati che se ne liberavano con rabbia. Il suono metallico di un martello sulla falce, che veniva da un carro, cessò. Uno dei contadini, alzatosi, si avviò verso la carrozza. 

- Guarda, non si regge in piedi! - gridò irritato lo scrivano al contadino che camminava piano coi piedi scalzi sulla strada asciutta, non battuta. - Vieni, via! 

Il vecchio riccioluto, coi capelli legati da uno stelo di tiglio, la schiena ricurva e scurita dal sudore, affrettato il passo, s'accostò alla vettura e afferrò con la mano abbronzata un parafango della carrozza. 

- Vozdvizenskoe? alla casa dei signori? dal conte? - ripeté. - Ecco, appena sbocca il viottolo. A sinistra c'è una svolta. Dritto per il viale e ci vai a finir dentro. Ma voi chi volete, proprio lui? 

- Sono in casa, amico mio? - disse senza precisare Dar'ja Aleksandrovna, non sapendo come domandare di Anna nemmeno a un contadino. 

- Devono essere in casa - disse il contadino, muovendo lentamente i piedi scalzi e lasciando nella polvere l'impronta chiara delle cinque dita del piede. - Devono essere in casa - egli ripeté, desiderando evidentemente di attaccar discorso. - Anche ieri sono arrivati degli ospiti. Di ospiti ce n'è sempre tanti\ldots{} Che vuoi? - disse, rivolto a un giovanotto che gli gridava qualcosa dal carro. - Anche quello! Son passati or ora tutti a cavallo per vedere la mietitrice. Adesso devono essere a casa. E voi di dove sareste? 

- Noi siamo di lontano - disse il cocchiere, salendo a cassetta. - Allora è vicino? 

- Vi dico che è proprio qui. Appena vai in là - egli disse, toccando con la mano il parafango della vettura. 

Un giovanotto sano, tarchiato, si avvicinò pure. 

- Be', non c'è lavoro per il raccolto? - domandò. 

- Non lo so, amico. 

- Allora, prendi a sinistra, troverai subito - diceva il contadino, lasciando evidentemente andar via malvolentieri i passanti, perché desiderava chiacchierare un po'. 

Il cocchiere si mosse, ma avevano appena voltato che il contadino si mise a gridare: 

- Ferma, ohi, amico! Fermati! - gridavano due voci. Il cocchiere si fermò. 

- Vengono! Eccoli! - gridò il contadino. - E guarda come montano bene! - disse, indicando le quattro persone a cavallo e le altre due in uno char à bancs che venivano per la strada. 

Erano Vronskij con un fantino, Veslovskij e Anna a cavallo e la principessa Varvara con Svijazskij nello char à bancs. Erano andati a fare una passeggiata e a vedere funzionare le mietitrici meccaniche fatte venire da poco. 

Quando la carrozza si fermò, le persone a cavallo si misero al passo. Avanti andava Anna, accanto a Veslovskij. Anna andava a passo tranquillo su di un piccolo cavallo inglese, basso e tozzo, con la criniera tagliata e la coda corta. La sua bella testa, con i capelli neri sfuggenti di sotto il cappello alto, le spalle piene, la vita sottile nell'amazzone nera, e la calma, aggraziata posizione in sella colpirono Dolly. 

Nel primo momento le parve sconveniente che Anna andasse a cavallo. Alla figura di una signora che cavalcava si univa, nella concezione di Dar'ja Aleksandrovna, l'immagine di una giovanile, leggera civetteria, che, secondo lei, non si addiceva alla situazione di Anna; ma quando l'ebbe esaminata da vicino, si riconciliò subito con la sua equitazione. Malgrado l'eleganza, tutto era così semplice, tranquillo e dignitoso, e nell'atteggiamento e nell'abito e nei movimenti di Anna, che non ci poteva essere nulla di più spontaneo. 

Accanto ad Anna, su di un accaldato cavallo grigio di cavalleria, allungando in avanti le gambe grasse ed evidentemente compiaciuto, cavalcava Vasen'ka Veslovskij col berrettino scozzese dai nastri svolazzanti. Dar'ja Aleksandrovna non poté trattenere un sorriso d'ilarità, dopo averlo riconosciuto. Dietro di loro veniva Vronskij. Cavalcava un cavallo baio scuro, purosangue, evidentemente accaldatosi nel galoppo. Egli, nel trattenerlo, lavorava di redini. 

Dietro di lui cavalcava un ometto in costume di fantino. Svijazskij e la principessa, in uno char à bancs nuovo fiammante, tirato da un grosso morello trottatore, tenevano dietro ai cavalieri. 

Il viso di Anna, nel momento in cui, nella piccola figura che si stringeva contro l'angolo della vecchia vettura, riconobbe Dolly, s'illuminò d'un tratto di un sorriso di gioia. Ella diede un grido, ebbe un sussulto sulla sella e mise il cavallo al galoppo. Avvicinatasi alla vettura, saltò giù senza aiuto e, sollevando l'amazzone, corse verso Dolly. 

- Lo pensavo e non osavo pensarlo. Che gioia! Non puoi immaginare la mia gioia!- diceva, ora stringendosi col viso a Dolly e baciandola, ora allontanandosi ed esaminandola con un sorriso. - Ecco, questa è una gioia, Aleksej! - ella disse, volgendosi a Vronskij che era sceso da cavallo e si avvicinava a loro. 

Vronskij, levatosi il cappello grigio alto, si accostò a Dolly. 

- Non potete credere come siamo contenti del vostro arrivo - disse, dando un significato particolare alle parole pronunciate e scoprendo nel sorriso i suoi forti denti bianchi. 

Vasen'ka Veslovskij, senza smontar da cavallo, si levò il berrettino e salutò l'ospite, agitando gioiosamente i nastri sopra il capo. 

- È la principessa Varvara - rispose Anna a uno sguardo interrogativo di Dolly, quando lo char à bancs si fu accostato. 

- Ah! - disse Dar'ja Aleksandrovna, e il suo viso espresse lo scontento. 

La principessa Varvara era una zia di suo marito, e lei da tempo la conosceva e non la stimava. Sapeva che la principessa Varvara aveva passato tutta la sua vita da parassita presso parenti ricchi; ma che adesso vivesse da Vronskij, persona per lei estranea, la offendeva per la parentela del marito. Anna notò l'espressione del viso di Dolly e si confuse, arrossì, lasciò sfuggire dalle mani l'amazzone e vi inciampò. 

Dar'ja Aleksandrovna si avvicinò allo char à bancs che s'era fermato e salutò freddamente la principessa Varvara. Anche Svijazskij era a lei noto. Egli domandò come stava quell'originale del suo amico con la giovane moglie e, esaminati con uno sguardo fuggevole i cavalli non appaiati e la vettura dai parafanghi rappezzati, offrì alle signore di montare nello char à bancs. 

- E io andrò in codesto veicolo - egli disse. - Il cavallo è tranquillo e la principessa guida benissimo. 

- No, restate così come siete - disse Anna che s'era avvicinata - e noi andremo nella vettura - e, presa Dolly sotto braccio, la condusse via. 

Dar'ja Aleksandrovna era abbagliata da quella vettura elegante non mai vista da lei, da quei cavalli bellissimi, da quelle persone mondane, raffinate che la circondavano. Ma più di tutto la stupiva il cambiamento avvenuto nell'Anna che conosceva e amava. Un'altra donna, meno attenta, che non avesse conosciuto Anna prima e che, soprattutto, non avesse avuto quei pensieri che Dar'ja Aleksandrovna aveva avuto durante il viaggio, non avrebbe neppure notato nulla di particolare in Anna. Ma adesso, Dolly era colpita da quella bellezza momentanea che le donne sogliono avere quando amano e che lei scopriva nel viso di Anna. Tutto, infatti, era tale nel viso di Anna: le fossette ben definite delle guance e del mento, la piega delle labbra, il riso, ch'era come se le errasse sul volto, lo scintillio degli occhi, la grazia e la rapidità dei movimenti, la pienezza dei suoni della voce, perfino il modo carezzevolmente irato col quale ella rispondeva a Veslovskij che le chiedeva il permesso di montare il suo cavallo inglese per insegnargli l'ambio della zampa destra, tutto era particolarmente attraente; e sembrava che ella ne fosse consapevole e ne gioisse. 

Quando le due donne furono sedute nella vettura, furono prese a un tratto da un certo disagio. Anna, per quello sguardo attentamente interrogativo con cui la guardava Dolly, si era confusa; Dolly, per le parole di Svijazskij sul veicolo, aveva involontariamente cominciato a vergognarsi della vecchia vettura sdrucita nella quale Anna si era seduta con lei. Il vecchio Filipp e lo scrivano provavano lo stesso sentimento. Lo scrivano, per nascondere il proprio disagio, si dava da fare, mettendo a sedere le signore, ma Filipp il cocchiere si era fatto scuro e s'era preparato fin d'ora a non sottostare a quella superiorità esteriore. Sorrise ironicamente, dopo aver guardato il trottatore morello e dopo aver già stabilito in cuor suo che il morello dello char à bancs era buono solo per la prominaz e che non avrebbe fatto quaranta verste, col caldo, in una sola tirata. 

I contadini s'erano alzati tutti dal carro e guardavano curiosi e allegri l'incontro degli ospiti, facendo le loro osservazioni. 

- Anche loro sono contente, è un pezzo che non si son viste - diceva il vecchio riccioluto, quello con il legaccio di tiglio. 

- Ecco, zio Gerasim, lo stallone morello sì che andrebbe alla svelta a portare i covoni! 

- Guarda un po'! Questa qui in pantaloni è una donna? - disse uno di loro indicando Vasen'ka Veslovskij che sedeva sulla sella da signora. 

- No, è un uomo. Guarda come è saltato su svelto! 

- Be', figliuoli, non dormiremo, vero? 

- Ma che dormire oggi! - disse il vecchio, e guardò il sole di traverso. - Mezzogiorno è passato, guarda! Prendi il forcone, comincia! 

\capitolo{XVIII}\label{xviii-5} 

Anna guardava il viso magro, sfinito, cosparso di polvere nelle piccole rughe, di Dolly e voleva dire quello che pensava, che cioè Dolly era sciupata; ma, ricordando che lei era invece divenuta più bella e che lo sguardo di Dolly glielo diceva, sospirò e cominciò a parlare di sé. 

- Tu mi guardi - ella disse - e pensi come mai io possa essere felice nella mia situazione. Ma, non so! È vergognoso confessarlo; ma io\ldots{} sono imperdonabilmente felice. M'è accaduto qualcosa di magico, come quando in un sogno si prova spavento, impressione e a un tratto ci si sveglia e si sente che tutti quegli spaventi non esistono. Mi sono svegliata. Ho vissuto il tormento e la paura, e ora, già da tempo, ma in particolare da che siamo qui, sono felice!\ldots{} - ella disse, guardando Dolly con un timido sorriso di domanda. 

- Come sono contenta! - disse, sorridendo, Dolly, involontariamente più fredda di quanto volesse. - Sono molto contenta per te. Perché non m'hai scritto? 

- Perché?\ldots{} perché non osavo\ldots{} tu dimentichi la mia situazione\ldots{} 

- A me? non osavi? Se sapessi come io\ldots{} Io credo\ldots{} 

Dar'ja Aleksandrovna voleva dire i suoi pensieri della mattina, ma, chissà perché, questo le parve fuor di posto. 

- Del resto di questo parleremo poi. Cosa sono tutte queste costruzioni? - domandò, desiderando cambiar discorso e indicando i tetti rossi e verdi che si scorgevano di là dal verde vivo delle siepi di acacia e di serenelle. - Pare una piccola città. 

Ma Anna non rispondeva. 

- No! no! Cosa pensi mai della mia situazione, cosa pensi, cosa? - ella domandò. 

- Io suppongo\ldots{} - cominciò a dire Dar'ja Aleksandrovna, ma in quel momento Vasen'ka Veslovskij, messo al galoppo sul piede destro il piccolo cavallo inglese e battendo con la sua giacchetta corta contro la sella scamosciata da signora, passò loro accanto di galoppo. 

- Va bene, Anna Arkad'evna! - gridò. 

Anna non lo guardò neppure; ma di nuovo a Dar'ja Aleksandrovna parve che là nella vettura non fosse opportuno incominciare quel lungo discorso, e perciò abbreviò il suo pensiero. 

- Io non penso nulla - ella disse - ma t'ho sempre voluto bene, e quando si vuol bene, si vuol bene a tutta la persona così com'è, e non come si vuole che sia. 

Anna, allontanando gli occhi dal viso dell'amica e socchiudendoli (era questo un nuovo tratto che Dolly non le conosceva), si fece pensierosa, desiderando di afferrare in pieno il senso di queste parole. E avendole evidentemente capite così come voleva lei, guardò Dolly. 

- Se tu avessi delle colpe - ella disse - ti sarebbero tutte perdonate per la tua venuta e per codeste tue parole. 

E Dolly vide che le eran venute le lacrime agli occhi. Ella strinse in silenzio la mano ad Anna. 

- E allora che cosa sono queste costruzioni? Quante! - ella ripeté la sua domanda dopo un attimo di silenzio. 

- Sono le case degli impiegati, l'allevamento, le scuderie - rispose Anna. - E qui comincia il parco. Tutto ciò era abbandonato, ma Aleksej ha rinnovato tutto. Egli ama molto questa tenuta e, cosa che non m'aspettavo in nessun modo, è stato preso dalla passione per l'organizzazione dell'azienda. Del resto è una natura così ricca! Qualunque cosa intraprenda, fa tutto così bene. Non solo non si annoia, ma si occupa con passione. È diventato, così come lo conosco io, un padrone calcolatore, perfetto, perfino avaro nell'azienda. Ma solo nell'azienda. Là dove si tratta di decine di migliaia di rubli, non sta a calcolare - ella diceva con quel sorriso gioiosamente sagace con cui spesso le donne parlano delle particolarità misteriose, a loro sole rivelate, della persona amata. - Ecco, vedi questa grande costruzione? È un nuovo ospedale. Io penso che costerà più di centomila rubli. È il suo dada adesso. E sai perché è venuto fuori, questo? I contadini gli avevano chiesto di ceder loro più a buon mercato i prati, mi pare, ma lui aveva rifiutato, e io gli ho rimproverato la sua avarizia. Certo, non solo per questo, ma per tutto un insieme di cose, ha cominciato questo ospedale per mostrare, capisci, come non sia avaro. Se vuoi, c'est une petitesse; ma io l'amo ancora di più per questo. Ma ecco che vedrai subito la casa. È ancora la casa del nonno e non è affatto cambiata all'esterno. 

- Com'è bella! - disse Dolly, guardando con involontario stupore la bella casa con le colonne che sovrastava il verde variegato dei vecchi alberi del giardino. 

- Non è vero che è bella? E dalla casa, di sopra, c'è una veduta meravigliosa. 

Esse entrarono in un cortile cosparso di ghiaia e accomodato a giardino, in cui due operai circondavano di pietre porose, grezze, un'aiuola di fiori rimossa, e si fermarono nell'ingresso coperto. 

- Ah, sono già arrivati! - disse Anna, guardando i cavalli da sella che stavano portando via proprio allora dall'ingresso. - Vero che è bello questo cavallo? È un cob. È il mio preferito. Portatelo qua e datemi dello zucchero. Dov'è il conte? - ella chiese ai due servitori in livrea che erano sbucati fuori. - Ah, ecco anche lui! - ella disse, vedendo Vronskij che, con Veslovskij, le veniva incontro. 

- Dove metterete la principessa? - disse Vronskij in francese, rivolto ad Anna, e, senza aspettare la risposta, salutò ancora una volta Dar'ja Aleksandrovna e adesso le baciò la mano. - Io penso, nella camera grande col balcone. 

- Oh, no, è lontano! È meglio in quella d'angolo, ci vedremo di più. Su, andiamo - disse Anna che dava lo zucchero, portatole dal servitore, al cavallo preferito. 

- Et vous oubliez votre devoir - ella disse a Veslovskij che era uscito anche lui sulla scalinata. 

- Pardon, j'en ai tout plein les poches - rispose lui sorridendo, sprofondando le dita nelle tasche del panciotto. 

- Mais vous venez trop tard - ella disse, asciugando col fazzoletto la mano che le aveva bagnato il cavallo nel prender lo zucchero. Anna si voltò a Dolly. 

- Sei venuta per restare a lungo? Un giorno solo? È impossibile! 

- Così ho promesso, e i bambini\ldots{} - disse Dolly impacciata e perché doveva prendere la borsa dalla vettura e perché sapeva di avere il viso molto impolverato. 

- No, Dolly, amica mia\ldots{} Su, vedremo. Andiamo, andiamo! - e Anna condusse Dolly nella sua camera. 

Questa stanza non era quella di lusso, che aveva proposto Vronskij, ma una stanza per cui Anna disse a Dolly di scusarla. Ma anche questa stanza, della quale bisognava scusarsi, era di uno sfarzo nel quale Dolly non aveva mai vissuto e che le ricordò i migliori alberghi all'estero. 

- Ma come sono contenta, anima mia! - disse Anna, sedendosi per un attimo accanto a Dolly nella sua amazzone. - Raccontami dunque dei tuoi. Stiva l'ho visto di sfuggita. Ma lui non può raccontare dei bambini. Come sta Tanja, la mia preferita? È una bambina grande, penso. 

- Sì, molto grande - rispose breve Dar'ja Aleksandrovna, meravigliandosi lei stessa di rispondere così freddamente sui suoi figliuoli. - Stiamo benissimo dai Levin - ella aggiunse. 

- Ecco, se avessi saputo - disse Anna - che non mi disprezzi\ldots{} Sareste venuti tutti da noi. Perché Stiva è un vecchio e grande amico di Aleksej - ella aggiunse, e a un tratto arrossì. 

- Sì, ma ci stiamo così bene\ldots{} - rispose Dolly, confondendosi. 

- Ma del resto, è per la gioia che dico delle sciocchezze. Una cosa sola, amica mia, come sono contenta di vederti! - disse Anna, baciandola di nuovo. - Tu non mi hai ancora detto come e cosa pensi di me e io voglio sapere tutto. Ma sono contenta che tu mi vedrai così come sono. Io, soprattutto, non voglio che pensino che io tenda a dimostrare qualcosa. Io non voglio dimostrare nulla, voglio semplicemente vivere, non far del male a nessuno, tranne che a me. Di questo ho il diritto, vero? Ma questo è un discorso lungo e noi parleremo ancora di tutto. Ora andrò a vestirmi e ti manderò la donna. 

\capitolo{XIX}\label{xix-5} 

Rimasta sola, Dar'ja Aleksandrovna esaminò la camera con lo sguardo della padrona di casa. Tutto quello che aveva visto, avvicinandosi alla villa e attraversandola, e quello che ora vedeva nella propria camera, tutto produceva in lei un'impressione di ricchezza ed eleganza, e di lusso moderno, di cui aveva letto soltanto nei romanzi inglesi, ma che non aveva ancora visto in Russia e per di più in campagna. Tutto era moderno, dalle tappezzerie francesi fino al tappeto che ora era disteso per tutta la stanza. Il letto era a molle con un piccolo materasso e un capezzale speciale, le federe di seta cruda sopra i piccoli guanciali. Il lavabo di marmo, la specchiera, la sedia a sdraio, le tavole, l'orologio di bronzo sul camino, le tende e i tendaggi, tutto questo era costoso e nuovo. 

L'elegante cameriera ch'era venuta a offrire i suoi servigi, con un'acconciatura e un vestito più alla moda di quelli di Dolly, era egualmente costosa e alla moda quanto la stanza. A Dar'ja Aleksandrovna piaceva quella sua cortesia, quel lindore e ossequio, ma si sentiva a disagio con lei; si vergognava dinanzi a lei della propria camicetta rammendata che le era stata messa dentro per sbaglio, come per disgrazia. Si vergognava di quegli stessi rammendi e di quelle stesse toppe di cui a casa andava orgogliosa. A casa era chiaro che per sei camicette occorrevano ventiquattro aršiny di batista a sessantacinque copeche, il che formava più di quindici rubli, all'infuori delle guarnizioni e della manifattura, e che questi quindici rubli erano stati risparmiati. Ma dinanzi alla cameriera non già che si vergognasse, ma non era a suo agio. 

Dar'ja Aleksandrovna si sentì molto sollevata quando nella camera entrò Annuška, che ella conosceva da tempo. La cameriera elegante era richiesta dalla signora e Annuška rimase con Dar'ja Aleksandrovna. 

Annuška era evidentemente molto contenta dell'arrivo della signora e discorreva senza posa. Dolly notò che aveva voglia di esprimere la propria opinione sulla posizione della padrona, soprattutto sull'amore e sulla devozione del conte per Anna Arkad'evna, ma Dolly la trattenne con garbo, non appena cominciò a parlare. 

- Io sono cresciuta con Anna Arkad'evna, m'è più cara di tutto il mondo. E poi non dobbiamo essere noi a giudicare. E poi l'ama tanto, mi pare. 

- Allora, per favore, date a lavare, se è possibile - la interruppe Dar'ja Aleksandrovna. 

- Sissignora. Da noi, due donne sono addette alla lavatura, ma la biancheria si fa tutta a macchina. Il conte pensa a tutto. Altro che marito\ldots{} 

Dolly fu contenta quando Anna entrò da lei e con la sua presenza fece cessare il chiacchierio di Annuška. 

Anna aveva cambiato abito, e aveva indossato un abito di batista molto semplice. Dolly esaminò attentamente questo vestito semplice. Sapeva cosa significasse e a qual prezzo si acquistasse una simile semplicità. 

- Una vecchia amica - disse Anna di Annuška. 

Anna adesso non si vergognava più. Era del tutto libera e calma. Dolly vedeva che adesso s'era già riavuta in pieno dall'impressione che aveva prodotto in lei il suo arrivo e aveva già preso quel tono superficiale, indifferente, per cui pareva che la porta di quel reparto, dove si trovavano i suoi sentimenti e i suoi pensieri intimi, fosse preclusa. 

- E la tua bambina, Anna? - domandò Dolly. 

- Annie? - così ella chiamava sua figlia Anna. - Sta bene. Si è molto rimessa. Vuoi vederla? Andiamo, te la farò vedere. C'è stato un gran trambusto con le bambinaie - ella cominciò a raccontare. - Abbiamo avuto un'italiana per balia. Buona, ma così sciocca! La volevamo mandar via, ma la bambina s'era tanto abituata a lei, che la teniamo ancora. 

- E come vi siete accomodati?\ldots{} - cominciò Dolly e voleva chiedere che nome avrebbe portato la bambina; ma avendo notato il viso di Anna, divenuto improvvisamente cupo, cambiò il senso della domanda. - E come vi siete accomodati? L'avete già svezzata? 

Ma Anna aveva capito. 

- Non è questo che mi volevi chiedere. Non volevi chiedermi del suo nome? Vero? Questo tormenta Aleksej. Non ha nome. Cioè è una Karenina - disse Anna, socchiudendo gli occhi così che le si vedevano soltanto le ciglia unite. - Del resto - disse, rischiarandosi d'un tratto in viso - di questo diremo tutto dopo. Andiamo, te la farò vedere. Elle est très gentille. Si trascina già per terra. 

Nella camera della bambina, lo sfarzo, che in tutta la casa stupiva Dar'ja Aleksandrovna, la colpì ancora di più. C'erano carrettini, fatti venire dall'Inghilterra, strumenti per imparare a camminare, un divano fatto apposta a guisa di biliardo per trascinarsi, e sedie a dondolo e vasche speciali, moderne. Tutto era inglese, solido e di buona qualità e, evidentemente, molto costoso. La camera era grande, molto alta e luminosa. 

Quando esse entrarono, la bambina, con la sola camicina, stava seduta in una piccola poltrona accanto al tavolo e sorbiva un brodo che s'era versato addosso su tutto il piccolo petto. Dava da mangiare alla bambina, e visibilmente mangiava lei stessa insieme alla piccola, una ragazza russa che faceva il servizio nella camera della bambina. Non c'erano né la balia, né la njanja; erano nella camera attigua, e di là si sentiva il loro chiacchierio in uno strano francese, unica lingua nella quale riuscivano a intendersi. 

Udita la voce di Anna, l'inglese tutta adorna, alta, con un viso antipatico e una espressione disonesta, entrò per la porta, agitando i riccioli biondi in fretta, e cominciò subito a giustificarsi, benché Anna non l'accusasse di nulla. A ogni parola di Anna, l'inglese aggiungeva in fretta, varie volte: ``yes, my lady''. 

La bambina, dalle sopracciglia e i capelli neri, colorita, con un corpicino forte e colorito e dalla pelle aggricciata, malgrado l'espressione severa con cui guardò la persona nuova, piacque molto a Dar'ja Aleksandrovna; ella invidiò perfino il suo aspetto sano. E le piacque pure come si trascinava, quella piccola. Neanche uno dei suoi bambini s'era trascinato così. Quando la misero a sedere sul tappeto e le ficcarono da dietro il vestitino, fu straordinariamente graziosa. Voltandosi a guardare come una piccola belva, con i grandi occhi neri scintillanti, evidentemente rallegrata dall'ammirazione, sorridendo e tenendo le gambette di traverso, s'appoggiava con forza sulle mani e traeva a sé tutto il sederino e poi di nuovo s'aggrappava avanti con le manine. 

Ma tutto l'insieme della camera della bambina e in particolare l'inglese, non piacquero per nulla a Dar'ja Aleksandrovna. Solo col fatto che, in una famiglia irregolare come quella di Anna, una buona donna non ci sarebbe andata, Dar'ja Aleksandrovna si spiegò come proprio Anna, con la sua conoscenza delle persone, avesse potuto assumere per la propria creatura una inglese così poco simpatica, così poco rispettabile. Inoltre, da alcune parole, Dar'ja Aleksandrovna capì che Anna, la balia, la njanja e la bambina non erano affiatate per nulla e che una visita da parte della madre era una cosa insolita. Anna voleva tirar fuori alla bambina un giocattolo e non riuscì a trovarlo. 

La cosa, poi, più sorprendente di tutto fu che, alla domanda su quanti denti avesse, Anna si sbagliasse e non sapesse affatto degli ultimi due. 

- A volte è penoso per me essere come superflua qui - disse Anna, uscendo dalla camera della bambina e sollevando il proprio strascico per evitare i giocattoli che si trovavano accanto alla porta. - Non era così col primo. 

- Io pensavo al contrario - disse timida Dar'ja Aleksandrovna. 

- Oh, no! Perché lo sai, ho visto Serëza - disse Anna, socchiudendo gli occhi, come se osservasse qualcosa lontano. - Del resto questo lo diremo dopo. Tu non ci crederai, ma io sono proprio come un'affamata alla quale, d'un tratto, abbiano messo dinanzi un intero pranzo, e che non sa da che cosa cominciare. Il pranzo completo sei tu e i miei discorsi con te, che non potevo avere con altri; e non so da quale di questi incominciare. Mais je ne vous ferai grâce de rien. Già, devo farti un quadro della compagnia che troverai da noi - ella cominciò. - Comincio dalle signore. La principessa Varvara. Tu la conosci e io conosco l'opinione tua e di Stiva su di lei. Stiva dice che tutto lo scopo della sua vita consiste nel dimostrare la propria superiorità sulla zia Katerina Pavlovna; questo è vero, ma è buona, e io le sono riconoscente. A Pietroburgo c'è stato un momento in cui mi era indispensabile avere un chaperon. In quel momento è capitata lei. Ma davvero è buona. M'ha alleviato molto la situazione. M'accorgo che non ti rendi conto di tutta la difficoltà della mia posizione\ldots{} là a Pietroburgo - aggiunse. - Qui sono completamente tranquilla e felice. Ma di questo, dopo. Facciamo l'elenco. Dopo c'è Svijazskij, maresciallo della nobiltà e persona molto a modo, ma ha bisogno di qualcosa da Aleksej. Capirai, adesso che ci siamo stabiliti in campagna, Aleksej, con il suo patrimonio, può avere una grande influenza. Poi Tuškevic, l'hai visto, era addetto a Betsy. Ora l'hanno licenziato, e lui è venuto da noi. Come dice Aleksej, è una di quelle persone che vanno prese per quel che vogliono parere, et puis, il est comme il faut, come dice la principessa Varvara. Dopo c'è Veslovskij\ldots{} questo lo conosci. Un ragazzo molto simpatico - disse, e un sorriso malizioso le increspò le labbra. - Cos'è mai quella selvaggia storia con Levin? Veslovskij l'ha raccontata ad Aleksej e noi non ci crediamo. Il est très gentil et näif - disse, di nuovo con lo stesso sorriso. - Gli uomini hanno bisogno di distrazione, ed Aleksej ha bisogno di pubblico, perciò io mi procuro tutta questa compagnia. Bisogna che da noi ci sia animazione e allegria, e che Aleksej non desideri nulla di nuovo. Poi vedrai l'amministratore. Un tedesco molto bravo che sa il suo mestiere. Aleksej lo apprezza molto. Dopo, il dottore, un giovanotto, non è proprio un nichilista, ma sai, mangia col coltello\ldots{} eppure è un bravo medico. Dopo c'è l'architetto\ldots{} Une petit cour. 

\capitolo{XX}\label{xx-5} 

- Eccovi anche Dolly, principessa, volevate tanto vederla - disse Anna, uscendo insieme con Dar'ja Aleksandrovna su di una grande terrazza in muratura, sulla quale, all'ombra, ricamando una poltrona per il conte Aleksej Kirillovic, sedeva al telaio la principessa Varvara. - Lei dice di non voler prendere nulla prima di pranzo, ma voi fate portar la colazione, e io andrò a cercare Aleksej e li condurrò tutti qua. 

La principessa Varvara accolse Dolly affabilmente e con aria di protezione e subito cominciò a spiegarle che s'era stabilita da Anna perché le aveva voluto sempre bene, più di sua sorella Katerina Pavlovna, quella stessa che aveva allevato Anna, e che adesso, quando tutti l'avevano abbandonata, lei aveva ritenuto suo dovere aiutarla, in questo periodo transitorio che era il più difficile. 

- Il marito le darà il divorzio e allora io rientrerò di nuovo nella mia solitudine, ma ora posso essere utile e compirò il mio dovere, per quanto mi sia pesante, e non farò come gli altri. E come sei cara, come hai fatto bene a venire! Essi vivono proprio come i migliori coniugi; li giudicherà Iddio, non noi. E che forse Birjuzovskij e la Aveneva\ldots{} E lo stesso Nikandrov, e Vasil'ev con la Mamonova, e Liza Neptunova\ldots{} Del resto chi diceva nulla? E andò a finire che tutti li ricevevano. E dopo, c'est un interieur si joli, si comme il faut. Tout-à-fait à l'anglaise. On se réunit le matin au breakfast et puis on se sépare. Ognuno fa quello che vuole fino a pranzo. Il pranzo è alle sette. Stiva ha fatto bene a mandarti. Egli deve tenersi unito a loro. Sai, lui, per mezzo di sua madre e del fratello, può ottenere tutto. Poi fanno molta beneficenza. Non t'ha parlato del suo ospedale? Ce sera admirable, viene tutto da Parigi. 

La loro conversazione fu interrotta da Anna, che aveva trovato la compagnia degli uomini nella stanza del biliardo e ritornava insieme con loro sulla terrazza. Per il pranzo ci voleva ancora molto; il tempo era splendido e perciò furono proposti vari modi di passare quelle due ore che rimanevano. Modi di passare il tempo ce ne erano molti a Vozdvizenskoe, ed erano del tutto diversi da quelli che erano in uso a Pokrovskoe. 

- Une partie de lawn tennis - propose Veslovskij, sorridendo col suo bel sorriso. - Di nuovo con voi, Anna Arkad'evna. 

- No, fa caldo; meglio passare per il giardino e andare a fare una passeggiata in barca, far vedere a Dar'ja Aleksandrovna le rive - propose Vronskij. 

- Io sono d'accordo su tutto - disse Svijazskij. 

- Io penso che per Dolly la cosa più piacevole sia passeggiare un po', non è vero? E poi in barca - disse Anna. 

Fu stabilito proprio così. Veslovskij e Tuškevic andarono al bagno e promisero di preparare là una barca e di aspettare. 

Si misero in cammino per un sentiero in due coppie: Anna con Svijazskij e Dolly con Vronskij. Dolly era un po' turbata e preoccupata dall'ambiente per lei del tutto nuovo in cui era venuta a trovarsi. In astratto, in teoria, non solo giustificava, ma approvava perfino l'atto di Anna. Come in generale le donne moralmente irreprensibili, non di rado stanche dell'uniformità della loro vita morale, così ella, da lontano, non solo scusava l'amore colpevole, ma l'invidiava perfino. Inoltre, voleva bene di cuore ad Anna. Ma in realtà, vistala in quell'ambiente di persone per lei estranee, di gran tono, inusitato per Dar'ja Aleksandrovna, si sentiva a disagio. Spiacevole soprattutto era stato per lei trovarvi la principessa Varvara, che perdonava tutto per quelle comodità di cui profittava. 

In generale, Dolly approvava in astratto il gesto di Anna, ma veder l'uomo per cui era stato compiuto quel gesto le era spiacevole. Inoltre, Vronskij non le era mai piaciuto. Lo riteneva molto orgoglioso e non vedeva in lui nulla di cui potesse inorgoglirsi, all'infuori della ricchezza. Ma, contro la sua volontà, là, a casa sua, egli le incuteva ancor più soggezione, e non riusciva a sentirsi disinvolta con lui. Provava una sensazione simile a quella che aveva provato con la cameriera per la camicetta. Come dinanzi alla cameriera, non che si vergognasse, ma non si sentiva a suo agio per i rammendi, così di continuo, anche con lui, non che si vergognasse, ma si sentiva a disagio per se stessa. 

Dolly s'era confusa, e cercava un argomento di conversazione. Pur ritenendo che, col suo orgoglio, gli dovessero spiacere le lodi della casa e del giardino, ella, non trovando altro argomento di conversazione, gli disse tuttavia che le era piaciuta molto la casa. 

- Sì, è una costruzione molto bella e in buono stile antico - egli disse. 

- M'è piaciuto molto il cortile davanti alla scalinata. Era così? 

- Oh, no! - egli disse, e il viso gli s'illuminò di piacere. - Se aveste veduto quel cortile, questa primavera! 

Ed egli cominciò, prima cauto, poi accalorandosi sempre più, ad attrarre l'attenzione di lei sui vari particolari della casa e del giardino. Si vedeva che, avendo dedicato molto lavoro al miglioramento e all'abbellimento della sua casa signorile, Vronskij sentiva la necessità di vantarsene dinanzi a una persona nuova, e si rallegrava con tutta l'anima delle lodi di Dar'ja Aleksandrovna. 

- Se volete dare un'occhiata all'ospedale e non siete stanca, non è lontano. Andiamo - egli disse, dopo averla guardata in viso per convincersi che ella proprio non s'annoiasse. - Tu vieni, Anna? - si rivolse a lei. 

- Andiamo, non è vero? - ella disse rivolta a Svijazskij - Mais il ne faut pas laisser le pauvre Veslovskij et Tuškevic se morfondre là dans le bateau. Bisogna mandare qualcuno ad avvisarli. 

- Sì, questo è un monumento ch'egli pone qui - disse Anna, rivolgendosi a Dolly con quello stesso sorriso malizioso, consapevole, col quale prima ella aveva parlato dell'ospedale. 

- Oh, una cosa monumentale! - disse Svijazskij. Ma poi per non mostrarsi ammiratore incondizionato di Vronskij, aggiunse subito un'osservazione lievemente critica. - Ma mi sorprendo, conte - disse - come voi, che fate tanto per la salute del popolo, siate così indifferente alle scuole. 

- C'est devenu tellement commun, les écoles - disse Vronskij. - Voi capite, non è certo per questo, ma così, mi ci sono appassionato. Allora bisogna passare di qua per andare all'ospedale - disse rivolto a Dar'ja Aleksandrovna, indicando un'uscita laterale del viale. 

Le signore aprirono gli ombrelli e uscirono sul sentiero laterale. Dopo aver passato alcune svolte e dopo essere entrate attraverso una porticina, Dar'ja Aleksandrovna vide dinanzi a sé, su di un luogo alto, una costruzione grande, rossa, di forma ingegnosa, già quasi finita. Il tetto di ferro, non ancora verniciato, scintillava in modo accecante al sole vivo. Accanto alla costruzione finita, ne spuntava un'altra circondata di impalcature, e sopra i ponti gli operai in grembiule ponevano i mattoni, versavano dai secchi la calcina e la spianavano con le cazzuole. 

- Come va in fretta da voi il lavoro! - disse Svijazskij. - Quando ci sono stato l'ultima volta non c'era ancora il tetto. 

- In autunno sarà tutto pronto. Di dentro è già quasi tutto rifinito - disse Anna. 

- E questo nuovo edificio cos'è mai? 

- È il locale per il dottore e la farmacia - rispose Vronskij, avendo visto l'architetto in cappotto corto che si avvicinava loro, e, dopo aver chiesto scusa alle signore, gli andò incontro. 

Fatto il giro della buca dalla quale gli operai tiravano fuori la calce ammonticchiandola, si fermò con l'architetto e prese a parlare di qualcosa con calore. 

- Il frontone riesce sempre più basso - rispose ad Anna che aveva domandato di che cosa si trattava. 

- Lo dicevo che bisognava sollevare le fondamenta - disse Anna. 

- Già, s'intende, sarebbe stato meglio, Anna Arkad'evna - disse l'architetto - ma ormai è stato trascurato. 

- Sì, me ne interesso molto - rispose Anna a Svijazskij, che aveva espresso meraviglia per le sue nozioni di architettura. - Bisogna che la nuova costruzione corrisponda all'ospedale. Ma è stata concepita dopo e cominciata senza progetto. 

Terminata la conversazione con l'architetto, Vronskij si unì alle signore e le condusse nell'interno dell'ospedale. 

Mentre, all'esterno, finivano ancora di fare i cornicioni e al piano di sotto verniciavano, di sopra quasi tutto era già rifinito. Dopo essere passati per una scala larga di ghisa, su di un pianerottolo, entrarono nella prima stanza grande. I muri erano intonacati di gesso, uso marmo, le finestre enormi interne erano già incastrate, soltanto il pavimento di legno non era ancora finito e i falegnami che piallavano un riquadro sollevato sospesero il lavoro per salutare i signori, togliendosi i legacci che trattenevano loro i capelli. 

- Questa è la sala da ricevimento - disse Vronskij; - qui ci saranno un banco, un tavolo, un armadio e nient'altro. 

- Di qua, andiamo di qua. Non accostarti alla finestra - disse Anna, provando se la vernice s'era asciugata. - Aleksej, la vernice è già asciutta - aggiunse. 

Dalla sala di ricevimento passarono in un corridoio. Qui Vronskij mostrò loro come funzionava la ventilazione, con un sistema nuovo. Poi mostrò le vasche da bagno di marmo, i letti con delle molle straordinarie. Dopo mostrò uno dopo l'altro i reparti, la dispensa, la stanza per la biancheria, poi le stufe di nuova foggia, le carriole fatte in modo da non produrre rumore nel trasportare la roba per il corridoio, e molte altre cose. Svijazskij apprezzava tutto come persona che conosce tutti i nuovi perfezionamenti. Dolly si meravigliava semplicemente di quello che finora non aveva mai veduto e, desiderando di capir tutto, chiedeva spiegazioni, il che faceva evidente piacere a Vronskij. 

- Sì, io penso che sarà in Russia l'unico ospedale attrezzato in modo pienamente adeguato - disse Svijazskij. 

- E non avrete un reparto di maternità? - domandò Dolly. - È così necessario in campagna. Io spesso\ldots{} 

Malgrado la sua cortesia, Vronskij la interruppe. 

- Questa non è una clinica ostetrica, ma un ospedale, ed è destinato a tutte le malattie, tranne le contagiose - egli disse. - E questo, guardate un po'\ldots{} - ed egli fece rotolare verso Dar'ja Aleksandrovna una poltrona per convalescenti fatta venire da poco. - Guardate. - Egli si sedette sulla poltrona e cominciò a muoverla. - Uno non può camminare, è ancora debole o ha una malattia alle gambe, ma ha bisogno d'aria, e va, passeggia\ldots{} 

Dar'ja Aleksandrovna si interessava di tutto, tutto le piaceva molto, ma ancor più le piaceva lo stesso Vronskij con quella spontanea, ingenua esaltazione. ``Sì, è una persona molto simpatica, buona'' ella pensava a volte, senza ascoltarlo, e guardandolo e penetrandone l'espressione, si trasportava col pensiero in Anna. Egli le piaceva talmente, adesso, nella sua animazione, che capiva come Anna avesse potuto innamorarsene. 

\capitolo{XXI}\label{xxi-5} 

- No, la principessa è stanca, penso, e i cavalli non la interessano - disse Vronskij ad Anna che aveva proposto di andare fino all'allevamento, dove Svijazskij voleva vedere uno stallone nuovo. - Voi andate, e io accompagnerò a casa la principessa, e discorreremo un po' - egli disse - se vi fa piacere - aggiunse rivolto a lei. 

- Di cavalli non capisco nulla, e sono molto contenta - disse Dar'ja Aleksandrovna alquanto sorpresa. 

Vedeva dal viso di Vronskij ch'egli aveva bisogno di qualcosa da lei. Non s'era sbagliata. Non appena entrarono, per la porticina, di nuovo nel giardino, egli guardò dalla parte dove era Anna, e, rassicuratosi ch'ella non potesse né sentirli né vederli, cominciò: 

- Avete indovinato che desideravo parlare un po' con voi - disse, guardandola con gli occhi ridenti. - Non mi sbaglio ritenendovi amica di Anna. - Si tolse il cappello e, tirato fuori il fazzoletto, s'asciugò la testa, che andava diventando calva. 

Dar'ja Aleksandrovna non rispose nulla e lo guardò soltanto come spaurita. Rimasta sola con lui, aveva avuto paura tutto ad un tratto: gli occhi ridenti e l'espressione severa del viso la spaventavano. 

Le più varie supposizioni su quello ch'egli stava per dirle, le balenarono in mente; prenderà a dirmi di voler venire a trovar noi e i bambini, e io dovrò dirgli di no; oppure di predisporre un certo ambiente per Anna a Mosca\ldots{} O che non sia Vasen'ka Veslovskij e dei suoi rapporti con Anna? o forse di Kitty, del fatto ch'egli si senta colpevole? Non prevedeva se non cose spiacevoli, ma non indovinò quello di cui egli voleva parlarle. 

- Voi avete un tale potere su di Anna, vi vuole tanto bene - disse - aiutatemi. 

Dar'ja Aleksandrovna guardava intimidita e interrogativa il viso energico di lui che, tutto o in parte, ora usciva in una striscia di sole fra l'ombra dei tigli, ora si oscurava di nuovo di ombra: e aspettava quello ch'egli avrebbe detto ancora, ma lui, urtando con un bastone nei ciottoli, camminava in silenzio accanto a lei. 

- Se siete venuta da noi, unica donna fra le vecchie amicizie di Anna (non conto la principessa Varvara) capisco che l'avete fatto non perché consideriate regolare la nostra posizione, ma perché, comprendendo tutta la gravità di questa posizione, le volete bene egualmente e desiderate aiutarla. È così, vi ho capita? - chiese voltandosi a guardarla. 

- Oh, sì - rispose Dar'ja Aleksandrovna, chiudendo l'ombrellino, - ma\ldots{} 

- No - egli l'interruppe e, senza volere, dimenticando di porre la propria interlocutrice in una posizione disagevole, si fermò così che anche lei dovette fermarsi. - Nessuno sente più fortemente di me tutta la gravità della posizione di Anna. E questo è comprensibile, se voi mi fate l'onore di considerarmi un uomo di cuore. Io sono la causa di questa situazione e perciò la sento. 

- Capisco - disse Dar'ja Aleksandrovna, involontariamente ammirandolo per la sua sincerità e la fermezza con cui aveva pronunciato ciò. - Ma proprio perché sentite di esserne la causa, temo che, esageriate. La sua posizione in società è penosa, capisco. 

- Nella società è un inferno! - egli pronunciò in fretta, accigliandosi. - Non si può immaginare tormento morale peggiore di quello che lei ha sofferto a Pietroburgo in due settimane\ldots{} e vi prego di crederlo. 

- Sì, ma qui, fino a che né Anna\ldots{} né voi sentiate il bisogno della società\ldots{} 

- La società! - egli disse con disprezzo - che bisogno posso avere io della società? 

- Fino ad allora, e potrebbe essere sempre, voi sarete felici e tranquilli. Io vedo che Anna è felice, completamente felice, ha già avuto il tempo di comunicarmelo - disse Dar'ja Aleksandrovna, sorridendo; e senza volere, proprio nel dire questo, ella dubitò che Anna fosse totalmente felice. 

Ma Vronskij sembrava non dubitarne. 

- Sì, sì - egli disse. - Lo so che s'è ripresa dopo tutte le sue sofferenze, è felice. È felice del presente. Ma io?\ldots{} io ho paura di quello che ci attende\ldots{} Perdonate, volete camminare? 

- No, è lo stesso. 

- Su, allora sediamoci qui. 

Dar'ja Aleksandrovna sedette su una panchina del giardino, in un angolo del viale. Egli rimase in piedi dinanzi a lei. 

- Vedo che lei è felice - egli ripeté, e il dubbio ch'ella fosse felice colpì ancor di più Dar'ja Aleksandrovna. - Ma può mai durare così? Che noi abbiamo agito bene o male è un'altra questione; ma il dado è tratto - egli disse, passando dal russo al francese - e noi siamo legati per tutta la vita. Siamo uniti con il vincolo per noi più santo dell'amore. Abbiamo una creatura, possiamo averne ancora. Ma la legge e tutte le circostanze della nostra situazione sono tali che si presentano migliaia di complicazioni, che lei adesso, distendendo l'animo dopo tante prove e sofferenze, non vede e non vuol vedere. E questo è comprensibile. Ma io non voglio non vedere. Mia figlia, secondo la legge, non è figlia mia, ma è una Karenina. Io non voglio questo inganno, no! - egli disse con un gesto energico di rifiuto e guardò con cupa interrogazione Dar'ja Aleksandrovna. 

Ella non rispondeva nulla e lo guardava soltanto. Egli continuò. 

- Domani nascerà un figlio, il figlio mio, e secondo la legge è un Karenin, non è l'erede del mio nome, né del mio patrimonio, e per quanto felici noi potremo essere in famiglia, e per quanti figli potremo avere, fra me e loro non vi sarà legame. Sarebbero dei Karenin. Voi vi rendete conto della gravità e dell'orrore di questa situazione? Ho provato a parlarne ad Anna. Ciò la irrita. Non capisce e io non posso dir tutto a lei. Adesso guardate la cosa da un altro lato. Io sono felice del suo amore, ma devo avere un'occupazione. Ho trovato questa occupazione, e ne sono orgoglioso, e la considero più nobile delle occupazioni dei miei colleghi di un tempo a corte e in servizio. E senza dubbio, ormai, non cambierò questo lavoro con il loro. Io lavoro qui, sempre allo stesso posto, e sono felice, soddisfatto e non ci occorre altro per essere felici. Io amo questa attività. Cela n'est pas un pis-aller, al contrario\ldots{} 

Dar'ja Aleksandrovna notò che in questo punto della spiegazione egli si confondeva, e a lei non era possibile capire bene la digressione, ma sentiva che, messosi una volta a parlare dei suoi rapporti intimi, di cui non poteva parlare con Anna, egli adesso rivelava tutto, e la questione della sua attività in campagna rientrava nello stesso reparto di pensieri intimi, così come la questione dei suoi rapporti con Anna. 

- E così, continuo - egli disse, dopo essersi ripreso. - La cosa principale poi è che, lavorando, mi è indispensabile avere la certezza che ciò che si fa non morirà con me, che avrò degli eredi; e questa certezza io non l'ho. Figuratevi la situazione di un uomo che sa in precedenza che i figli suoi e della donna che ama non saranno suoi, ma di un essere che li odia e non vuol saperne di loro. Questo è tremendo! 

Tacque visibilmente agitato. 

- Sì, s'intende, capisco. Ma cosa può mai Anna? - domandò Dar'ja Aleksandrovna. 

- Ecco, questo mi porta allo scopo del mio discorso - disse lui, dominandosi con sforzo. - Anna può, ciò dipende da lei\ldots{} Perfino per chiedere la legittimazione allo zar, è necessario il divorzio. E ciò dipende da Anna. Suo marito in qualunque momento è stato consenziente al divorzio, e vostro marito stava proprio per predisporlo. E ora, lo so, egli non si rifiuterebbe. Basterebbe solo scrivergli. Allora aveva risposto apertamente che, se lei ne avesse espresso il desiderio, egli non avrebbe opposto un rifiuto. S'intende - disse cupo - è una di quelle crudeltà farisaiche di cui sono capaci solo uomini senza cuore. Egli sa quale tormento le costi ogni ricordo suo e, conoscendola, esige una lettera da lei. Capisco che per lei ciò sia tormentoso. Ma le ragioni sono così gravi che bisogna passer pardessus toutes ces finesses de sentiment. Il y va du bonheur et de l'existence d'Anne et de ses enfants. Io non parlo di me, pur essendo in uno stato penoso, molto penoso - disse con una espressione di minaccia verso qualcuno per il fatto che era in uno stato penoso. - E così, principessa, io mi aggrappo senza riguardi a voi, come all'ancora di salvezza. Aiutatemi a convincerla a scrivergli e ad esigere il divorzio! 

\begin{itemize} \itemsep1pt\parskip0pt\parsep0pt \item Sì, s'intende - disse pensierosa Dar'ja Aleksandrovna, ricordando chiaramente l'ultimo suo incontro con Aleksej Aleksandrovic. - Sì, s'intende - ella ripeté decisa, ricordandosi di Anna. \end{itemize} 

- Adoperate tutto il vostro ascendente su di lei, fate in modo che ella scriva. Io non voglio e non posso quasi parlare di questo con lei. 

- Va bene, parlerò. Ma come, non ci pensa lei stessa? - disse Dar'ja Aleksandrovna, ricordando a un tratto la nuova strana abitudine di Anna di socchiudere gli occhi. E ricordò che Anna socchiudeva gli occhi, proprio quando si trattava delle questioni intime della sua vita. ``Proprio come se li socchiudesse dinanzi alla sua vita, per non vedere tutto'' pensò Dolly. - Assolutamente, e per me e per lei le parlerò - rispose Dar'ja Aleksandrovna, all'espressione grata di lui. 

Si alzarono e andarono verso casa. 

\capitolo{XXII}\label{xxii-5} 

Trovata Dolly già di ritorno, Anna la guardò attenta negli occhi, come a interrogarla di quella conversazione ch'ella aveva avuto con Vronskij, ma non chiese nulla. 

- Mi pare che sia già tempo di andare a pranzo - disse. - Non ci siamo ancora viste per nulla. Ma io conto sulla serata. Adesso bisogna andare a cambiarsi. Anche tu, penso. Ci siamo tutti inzaccherati là, sulla costruzione. 

Dolly andò in camera sua, e le venne da ridere. Per cambiarsi non aveva nulla, perché aveva già indossato il vestito migliore; ma per segnalare in qualche modo la propria acconciatura per il pranzo, pregò la cameriera di pulirle il vestito, cambiò i manichini e il nastro e mise dei pizzi in testa. 

- Ecco tutto quello che ho potuto fare - disse, sorridendo ad Anna che venne da lei in un terzo abito, di nuovo straordinariamente semplice. 

- Sì, noi qui siamo molto ricercati - ella disse, quasi a scusarsi della propria eleganza. - Aleksej è veramente contento del tuo arrivo come raramente gli accade di esserlo di qualcosa. È proprio innamorato di te - soggiunse. - E tu non sei stanca? 

Prima di pranzo, non c'era tempo di parlare di qualcosa. Entrate nel salotto, vi trovarono già la principessa Varvara e gli uomini in finanziera nera. L'architetto era in frac. Vronskij presentò all'ospite il dottore e l'amministratore. L'architetto era stato già presentato all'ospedale. 

Un maggiordomo grasso che risplendeva in un viso tondo e rasato, con un nodo inamidato alla cravatta bianca, riferì che le pietanze erano pronte, e le signore si alzarono. Vronskij pregò Svijazskij di dare il braccio ad Anna Arkad'evna, e lui stesso si avvicinò a Dolly. Veslovskij diede il braccio alla principessa Varvara prima di Tuškevic, così che Tuškevic con l'amministratore e il dottore si avviarono soli. 

Il pranzo e la sala da pranzo, le stoviglie, la servitù, i vini e i cibi non solo corrispondevano al tono generale di rinnovato sfarzo della casa, ma sembrava che fossero ancora più eleganti e moderni. Dar'ja Aleksandrovna osservava questo sfarzo per lei inusitato e, come padrona che mandava avanti una casa, senza speranza alcuna di poter mai applicare alla propria nulla di tutto quello che vedeva, tanto tutto questo era lontano per lusso dal suo tenore di vita, ne osservava senza volere tutti i particolari e si chiedeva chi avesse predisposto tutto ciò e come. Vasen'ka Veslovskij, suo marito e perfino Svijazskij e molte persone che lei conosceva, non ci pensavano proprio e credevano realmente quello che ogni bravo padrone di casa aspira a far sentire ai propri ospiti, che cioè tutto quello che da lui è così bene organizzato non sia costato alcun lavoro a lui, padrone di casa, ma si sia compiuto da sé. Dar'ja Aleksandrovna invece sapeva che da sé non si fa neanche la farinata per la colazione dei bambini e che, perciò, per un'organizzazione così complessa e perfetta, doveva essere stata impiegata la cura intensa di qualcuno. E dallo sguardo di Aleksej Kirillovic, da come egli esaminò la tavola e fece un cenno col capo al maggiordomo, da come offrì a Dar'ja Aleksandrovna la scelta tra la botvin'ja e la soupe, ella capì che tutto era stato disposto e diretto dalle cure dello stesso padrone di casa. Era evidente che tutto ciò non dipendeva da Anna più che da Veslovskij. Ella, Svijazskij, la principessa e Veslovskij erano egualmente ospiti, che profittavano allegramente di quello che veniva loro preparato. 

Anna si rivelava padrona di casa soltanto nel dirigere la conversazione. E questa conversazione, molto difficile per una padrona di casa con una tavola non grande, con persone come l'amministratore e l'architetto, di tutt'altro mondo, che cercavano di non intimidirsi dello sfarzo inusitato, ma che non potevano partecipare a lungo alla conversazione generale, questa difficile conversazione, ella la dirigeva con il suo tatto abituale, con naturalezza e quasi con soddisfazione, come notava Dar'ja Aleksandrovna. 

La conversazione si aggirò su come Tuškevic e Veslovskij fossero andati soli in barca, e Tuškevic allora prese a raccontare delle ultime regate allo Yacht-club a Pietroburgo. Ma Anna, aspettata un'interruzione, si rivolse subito all'architetto per trarlo fuori dal suo silenzio. 

- Nikolaj Ivanyc è rimasto sorpreso - disse a proposito di Svijazskij - come sia andata avanti la nuova costruzione da che egli è stato qui l'ultima volta; ma io stessa che ci vado ogni giorno, ogni giorno mi sorprendo di come proceda alla svelta. 

- Col signor conte si lavora bene - disse con un sorriso l'architetto (era un uomo cosciente del proprio merito, rispettoso e calmo). - Non è come avere a che fare con le autorità del governatorato. Là dove si riempirebbe una risma di carta, io riferisco al conte, si discute e tutto è bell'e fatto in tre parole. 

- All'americana - disse Svijazskij sorridendo. 

- Già, là gli edifici si elevano razionalmente\ldots{} 

La conversazione passò agli abusi dei poteri negli Stati Uniti, ma Anna la portò subito su di un altro tema per tirar fuori dal silenzio l'amministratore. 

- Hai mai visto una mietitrice americana? - disse rivolta a Dar'ja Aleksandrovna. - Eravamo andati a vederla quando t'abbiamo incontrato. Io stessa la vedevo per la prima volta. 

- E come funzionano? - chiese Dolly. 

- Proprio come forbici. Una tavola e molte piccole forbici. Ecco, così. 

Anna prese con le sue belle mani coperte di anelli un coltello e una forchetta e cominciò a far vedere. Ella, evidentemente, vedeva che dalla sua spiegazione non si sarebbe capito nulla; ma sapendo di parlare piacevolmente e di avere le mani belle, continuò la spiegazione. 

- Piuttosto come dei temperini - disse provocante Veslovskij, che non staccava gli occhi da lei. 

Anna sorrise in modo appena percettibile, ma non gli rispose. 

- Non è vero, Karl Fëdorovic, che sono come delle forbicine? - disse rivolta all'amministratore. 

- Oh, ja - rispose il tedesco. - Es ist ein ganz einfaches Ding - e cominciò a spiegare il congegno della macchina. 

- Peccato che non leghi. Ne ho vista una all'esposizione di Vienna che lega col filo di ferro - disse Svijazskij: - quelle sarebbero più convenienti. 

- Es kommt drauf an\ldots{} Der Preis vom Draht muss ausgerechnet werden. - E il tedesco, tratto fuori dal suo silenzio, si rivolse a Vronskij. - Dass lasst sich ausrechnen, Erlaucht. - Il tedesco stava già per metter la mano in tasca, dove aveva una matita infilata in un libretto in cui calcolava tutto, ma, ricordatosi ch'era a tavola, e notato lo sguardo freddo di Vronskij, si trattenne: - Zu compliziert, macht zu viel Klopot - concluse. 

- Wünscht man Dochots, so hat man auch Klopots - disse Vasen'ka Veslovskij, prendendo in giro il tedesco. - J'adore l'allemand - disse rivolto ad Anna con lo stesso sorriso. 

- Cessez - ella disse con scherzosa severità. 

- E noi credevamo di trovarvi nei campi, Vasilij Semënyc - disse rivolta al dottore, persona alquanto malaticcia - eravate là? 

- Ero là, ma mi sono volatilizzato - rispose, cupamente scherzoso, il dottore. 

- Perciò avete fatto un bel po' di moto. 

- Magnifico! 

- Su, e come va la salute della vecchietta? Spero che non sia tifo. 

- Per essere tifo non è tifo. Ma con questo non si trova in vantaggio. 

- Peccato! - disse Anna e, dato così un tributo di cortesia alle persone di casa, si rivolse ai suoi. 

- E tuttavia costruire una macchina secondo il vostro racconto sarebbe difficile, Anna Arkad'evna - disse scherzando Svijazskij. 

- No, e perché? - disse Anna con un sorriso che diceva com'ella sapesse che nella sua spiegazione del congegno della macchina c'era qualcosa di grazioso, notato anche da Svijazskij. Questa nuova civetteria giovanile colpì spiacevolmente Dolly. 

- Ma in compenso, in architettura, le conoscenze di Anna sono sorprendenti - disse Tuškevic. 

- E come, ieri ho sentito che Anna Arkad'evna diceva: nella corrente e i plinti - disse Veslovskij. - Ho detto giusto? 

- Non c'è nulla di sorprendente, quando si vedono e si sentono tante cose - disse Anna. - E voi, probabilmente, non sapete neppure con che cosa si fanno le case. 

Dar'ja Aleksandrovna vedeva che Anna era scontenta di quel tono di scherzosità che c'era fra lei e Veslovskij, ma lei stessa vi cadeva involontariamente. 

Vronskij, in questo caso, agiva in modo del tutto opposto a Levin. Evidentemente, egli non attribuiva alcuna importanza al chiacchierio di Veslovskij e, al contrario, incoraggiava questi scherzi. 

- Sì, dite un po', Veslovskij, con che cosa si uniscono le pietre? 

- S'intende, col cemento. 

- Bravo! E che cos'è il cemento? 

- Ma così, una specie di polenta d'orzo\ldots{} no, del mastice - disse Veslovskij, suscitando il riso generale. 

La conversazione fra quelli che pranzavano, a eccezione del dottore, l'architetto e l'amministratore immersi in un cupo silenzio, non languiva mai, ora scivolando, ora impigliandosi, ora toccando nel vivo qualcuno. Una volta Dar'ja Aleksandrovna fu toccata nel vivo e si accalorò tanto da arrossire perfino, e, dopo, cercò di ricordarsi se non avesse fatto qualcosa di superfluo o di spiacevole. Svijazskij si era messo a parlare di Levin, raccontando i suoi strani giudizi sulle macchine che, secondo lui, erano dannose in un'azienda russa. 

- Io non ho il piacere di conoscere codesto signor Levin - disse, sorridendo, Vronskij - ma probabilmente egli non ha mai visto quelle macchine che condanna. E se ne avrà vista e sperimentata qualcuna, sarà stata una qualunque, e non straniera ma una qualche macchina russa. E che idee ci possono mai essere qui? 

- In genere idee turche - disse Veslovskij con un sorriso, rivolto ad Anna. 

- Io non posso difendere le sue idee - disse, accalorandosi, Dar'ja Aleksandrovna - ma posso dire che è un uomo colto e che se fosse qui, saprebbe cosa rispondervi; io invece non so. 

- Io gli voglio molto bene, e io e lui siamo grandi amici - disse Svijazskij, sorridendo cordialmente. - Mais, pardon, il est un petit toqué; ecco ad esempio, egli sostiene che il consiglio distrettuale e i giudici di pace, tutto questo, insomma, non sia necessario, e non vuol far parte di nulla. 

- È la nostra apatia russa - disse Vronskij, versando l'acqua da una caraffa ghiacciata in un bicchiere sottile a calice - non sentire gli obblighi che i nostri diritti ci impongono, e perciò negare questi obblighi. 

- Io non conosco una persona più austera nel compimento dei propri doveri - disse Dar'ja Aleksandrovna, irritata da questo tono di superiorità di Vronskij. 

- Io, al contrario - continuò Vronskij evidentemente, chissà perché, toccato nel vivo da questa conversazione - io al contrario, come vedete, sono molto riconoscente dell'onore che mi hanno fatto, ecco, grazie a Nikolaj Ivanyic - ed egli indicò Svijazskij - eleggendomi giudice di pace onorario. Io considero che per me l'obbligo di andare a un congresso, di giudicare la causa di un contadino per un cavallo sia egualmente importante come tutto quello che riesco a fare. E lo riterrò un onore se mi eleggeranno consigliere distrettuale. Soltanto con questo posso ripagare tutti quei vantaggi di cui usufruisco come proprietario di terre. Per nostra disgrazia, i grossi proprietari di terre non capiscono l'importanza che devono avere nello stato. 

Per Dar'ja Aleksandrovna era strano ascoltare come egli fosse sicuro d'aver ragione alla propria tavola. Ricordò come Levin, che pensava il contrario, fosse altrettanto risoluto nei suoi giudizi a casa sua, a tavola. Ma ella voleva bene a Levin e perciò era dalla parte sua. 

- Allora possiamo contare su di voi, conte, per la prossima assemblea? - disse Svijazskij. - Ma occorre andare prima, per essere là il giorno 8. Mi farete l'onore di venire da me? 

- E io sono un po' d'accordo col tuo beau-frère - disse Anna. - Soltanto non così come lui - ella aggiunse con un sorriso. - Io temo che, nell'ultimo tempo, noi abbiamo troppi di questi obblighi pubblici. Prima c'erano tanti funzionari e per ogni affare era necessario un funzionario; adesso non ci sono che uomini pubblici. Aleksej è qui da sei mesi, ed è già membro di cinque o sei, mi pare, istituzioni diverse: è curatore, giudice, consigliere distrettuale, giurato, qualcosa di ippico. Du train que cela va, tutto il tempo andrà via per questo. E io temo che, con una tal massa d'affari, tutto si riduca a formalità. Voi di quanti uffici siete membro, Nikolaj Ivanyc? - si rivolse a Svijazskij - sembra, più di venti. 

Anna parlava scherzosamente, ma nel suo tono si sentiva l'irritazione. Dar'ja Aleksandrovna, che osservava attenta Anna e Vronskij, notò questo immediatamente. Notò anche come il viso di Vronskij, a questo discorso, prendesse un'espressione dura e ostinata. Avendo notato ciò e il fatto che la principessa Varvara, per cambiar discorso, s'era affrettata a parlare di conoscenti di Pietroburgo, Dolly ricordò quello che Vronskij in giardino aveva detto inconsideratamente sulla propria attività, e capì che alla questione della pubblica attività era collegata una certa questione intima fra Anna e Vronskij. 

Il pranzo, i vini, il servizio, tutto era stato molto bello, ma era tutto come Dar'ja Aleksandrovna aveva veduto in pranzi e balli di gala, che ormai non frequentava più e che avevano quello stesso carattere impersonale e poco disteso; e perciò, in un giorno qualunque e in una piccola compagnia, tutto questo le fece un'impressione spiacevole. 

Dopo pranzo rimasero un po' a sedere sulla terrazza. Poi, si misero a giocare al lawn-tennis. I giocatori, dopo essersi divisi in due campi, si disposero su un croquet-ground accuratamente spianato e battuto, dalle due parti di una rete tesa con colonnine dorate. Dar'ja Aleksandrovna tentò, ma a lungo non le riuscì di afferrare il giuoco, e quando lo capì era così stanca che sedette accanto alla principessa Varvara e rimase solo a guardare i giocatori. Il suo compagno, Tuškevic, s'era ritirato anch'esso, ma gli altri continuarono a lungo il giuoco. Svijazskij e Vronskij giocavano tutti e due molto bene e con impegno. Sorvegliavano con occhio vigile la palla lanciata loro, senza affrettarsi e senza indugiare si avvicinavano ad essa, ne aspettavano il rimbalzo e, colpendola di sotto in modo preciso e giusto con la racchetta, la rilanciavano al di là della rete. Veslovskij giocava peggio degli altri. Si accalorava troppo, ma in compenso con la sua allegria animava i giocatori. Il suo riso e le sue grida non cessavano. S'era tolto come gli altri, col permesso delle signore, la giacca, e la sua robusta e bella figura in maniche di camicia bianche, il viso sudato e rosso, e i suoi movimenti a scatti s'imprimevano proprio nella memoria. 

Quando Dar'ja Aleksandrovna andò a dormire, quella notte, appena chiudeva gli occhi vedeva Vasen'ka Veslovskij che si agitava su per il croquet-ground. 

Durante il giuoco Dar'ja Aleksandrovna non si divertì. Non le piacevano i rapporti che intercorrevano fra Vasen'ka Veslovskij e Anna e quella generale innaturalezza dei grandi quando, da soli, senza bambini, fanno un giuoco infantile. Ma per non turbare gli altri e per passare in qualche modo il tempo, dopo essersi riposata, si unì di nuovo al giuoco e finse di divertirsi. Tutto quel giorno le sembrò sempre di recitare con attori migliori di lei e di sciupare tutto con la sua cattiva recitazione. 

Era arrivata con l'intenzione di restare due giorni, se ci si fosse trovata bene. Ma la sera, durante il giuoco, decise di partire l'indomani. Le tormentose preoccupazioni materne che durante il viaggio le erano apparse così detestabili, ora, dopo un giorno passato senza di esse, le si presentavano già in altra luce, e l'attiravano di nuovo. 

Quando, dopo il tè della sera e una passeggiata notturna in barca, Dar'ja Aleksandrovna entrò sola nella sua stanza da letto e, toltosi il vestito, sedette ad accomodare per la notte i suoi capelli radi, sentì un gran sollievo. 

Le spiaceva quasi pensare che Anna sarebbe venuta subito da lei. Voleva restare un po' sola con i suoi pensieri. 

\capitolo{XXIII}\label{xxiii-5} 

Dolly si disponeva già a coricarsi, quando Anna, in veste da notte, entrò da lei. 

Durante il giorno, diverse volte, Anna aveva avviato il discorso su cose intime, e ogni volta, dette alcune parole, si era fermata. ``Dopo, a quattr'occhi diremo tutto. Ho tante cose da dirti'' aveva detto. 

Ora erano sole, e Anna non sapeva di che cosa parlare. Stava seduta accanto alla finestra, guardava Dolly e, passando in rassegna nella mente tutte quelle riserve che le erano parse inesauribili di discorsi intimi, non trovava nulla. Le pareva in quel momento che tutto fosse stato detto. 

- Be', come sta Kitty? - ella disse dopo aver sospirato profondamente e guardato Dolly con aria colpevole. - Dimmi la verità, Dolly, non è arrabbiata con me? 

- Arrabbiata? No - disse, sorridendo, Dar'ja Aleksandrovna. 

- Ma mi odia, mi disprezza? 

- Oh no! ma tu lo sai, questo non si perdona. 

- Sì, sì - disse Anna, voltandosi a guardare dalla finestra aperta. - Ma io non ne ho colpa. E di chi è la colpa? E quale colpa? Poteva forse essere diversamente? Dimmi, come credi tu? Era possibile che tu non fossi la moglie di Stiva? 

- Davvero non so. Ma, ecco, cosa mi devi dire\ldots{} 

- Sì, sì, ma non abbiamo finito di Kitty. È felice? Egli è un'ottima persona, dicono. 

- È poco dire ottima. Non conosco uomo migliore. 

- Ah, come sono contenta! Sono molto contenta! È poco dire ottima - ella ripeté. 

Dolly sorrise. 

- Ma tu dimmi di te. Io con te devo fare un lungo discorso. Abbiamo parlato con\ldots{} - Dolly non sapeva come chiamarlo. Era imbarazzata a chiamarlo conte e a chiamarlo Aleksej Kirillovic. 

- Con Aleksej - disse Anna - lo so che avete parlato. Ma io volevo chiederti apertamente, che cosa pensi di me, della mia vita. 

- Come dire così, a un tratto? Io davvero non so. 

- Ma dimmi comunque\ldots{} Tu vedi la mia vita. Ma non dimenticare che ci vedi ora d'estate, ora che sei arrivata tu e non siamo soli\ldots{} Ma siamo arrivati al principio della primavera, abbiamo vissuto completamente soli e vivremo soli, e io non desidero nulla di meglio. Ma figurati quando io dovrò viver sola, senza di lui, sola, e ciò avverrà\ldots{} Io, da tutto, vedo che questo si ripeterà spesso, che la metà del tempo egli sarà fuori casa - ella disse, alzandosi e avvicinandosi a Dolly. - S'intende - ella riprese, interrompendo Dolly che voleva obiettare - s'intende, con la forza non lo tratterrò. Non ci riuscirei nemmeno. Oggi sono le corse, i suoi cavalli corrono, egli va via. Sono molto contenta. Ma tu pensa a me, figurati la mia situazione\ldots{} Ma perché parlare di questo? - Sorrise. - Allora di che cosa ha parlato con te? 

- Ha parlato di quello di cui io stessa voglio parlare e mi è facile essere il suo avvocato: se non ci sia la possibilità e non si possa\ldots{} - Dar'ja Aleksandrovna esitò - accomodare, migliorare la tua situazione\ldots{} Tu sai come io consideri\ldots{} Tuttavia, se è possibile, occorre sposarsi\ldots{} 

- Cioè, il divorzio? - disse Anna. - Tu sai, l'unica donna che sia venuta da me a Pietroburgo è stata Betsy Tverskaja. La conosci, è vero? Au fond c'est la femme la plus dépravée qui existe. Aveva una relazione con Tuškevic, ingannando il marito nel modo più disgustoso. Eppure mi disse che non voleva saperne di me, finché la mia posizione non fosse regolare. Non pensare che io faccia confronti\ldots{} Ti conosco, anima mia. Ma io mi son ricordata senza volere\ldots{} Allora, cosa mai ti ha detto? - ripeté. 

- Ha detto che soffre per te e per sé. Forse tu dirai che è egoismo, ma è un egoismo così legittimo e nobile! Egli vuole, in primo luogo, legittimare sua figlia ed essere tuo marito, aver diritto su di te. 

- Quale moglie, quale schiava, può essere a tal punto schiava come me, nella mia situazione? - ella interruppe torva. 

- La cosa principale poi, che egli vuole, è che tu non soffra. 

- Questo è impossibile! E poi? 

- E poi, la cosa più legittima: vuole che i vostri figli abbiano un nome. 

- E quali figli? - disse Anna, senza guardare Dolly e socchiudendo gli occhi. 

- Annie e quelli che potranno venire. 

- Per questo può star tranquillo: io non avrò più figli. 

- E come puoi dire che non ne avrai? 

- Non ne avrò, perché non ne voglio. 

E, malgrado tutta la sua agitazione, Anna sorrise avendo notato un'espressione ingenua di curiosità, di stupore e di orrore sul viso di Dolly. 

- Il dottore m'ha detto, dopo la mia malattia\ldots{} 

- Non può essere! - disse Dolly, spalancando gli occhi. Per lei questa era una di quelle rivelazioni, le cui conseguenze e deduzioni sono così enormi che, nel primo momento, si sente soltanto che considerar tutto non si può, che bisogna pensarci molto e poi molto. 

Questa rivelazione, che le spiegava a un tratto tutte quelle famiglie per lei prima incomprensibili in cui vi erano soltanto uno o due bambini, suscitò in lei tanti pensieri, considerazioni e sentimenti contraddittori, ch'ella non aveva più nulla da dire e guardava soltanto Anna, stupita, con gli occhi spalancati. Era quella stessa cosa ch'ella aveva sognato durante il viaggio, ma ora, venuta a sapere che era possibile, inorridiva. Sentiva che era la soluzione troppo semplicistica di una troppo complessa questione. 

- N'est-ce-pas immoral? - ella disse soltanto dopo un po' di silenzio. 

- Perché? Pensa, io ho la scelta fra le due: o essere incinta, cioè malata, o essere l'amica, la compagna di mio marito - disse Anna con un tono cosciente di superficialità e leggerezza. 

- Eh già, eh già - diceva Dar'ja Aleksandrovna, ascoltando quegli identici argomenti ch'ella pure portava a se stessa, e senza più trovarvi la forza di persuasione di una volta. 

- Per te, per gli altri - diceva Anna, quasi indovinando i suoi pensieri - ci può ancora essere un dubbio, ma per me\ldots{} Devi capire, io non sono sua moglie; egli mi ama finché mi ama. E con che posso trattenere il suo amore? Con questo? 

E allungò le braccia bianche dinanzi al ventre. 

Con una rapidità straordinaria, come accade in un momento di agitazione, i pensieri e i ricordi si affollarono nella mente di Dar'ja Aleksandrovna. ``Io - pensava - non ho attirato a me Stiva; s'è allontanato da me verso altre, e quella prima per la quale mi ha tradito, non l'ha trattenuto con l'esser sempre bella e allegra. Egli l'ha buttata e ne ha presa un'altra. È possibile che Anna con questo attiri e trattenga il conte Vronskij? S'egli cercherà questo, troverà toilettes e maniere ancora più attraenti di queste. E per quanto siano bianche e per quanto siano splendide le sue braccia nude, per quanto sia bella la sua figura morbida, il suo viso acceso, e questi suoi capelli neri, egli troverà ancora di meglio, come cerca e trova il mio disgustoso, compassionevole e caro marito''. 

Dolly non rispose nulla e sospirò soltanto. Anna notò questo sospiro che esprimeva dissenso, e continuò. In riserva aveva argomenti ancora così forti, ai quali non si poteva rispondere nulla. 

- Tu dici che questo non è bene? Ma bisogna ragionare - ella proseguì. - Tu dimentichi la mia situazione. Come posso desiderare dei figli? Io non parlo delle sofferenze, non le temo. Pensa, cosa saranno i miei figli? disgraziati che dovranno portare un cognome estraneo. Per la stessa nascita dovranno vergognarsi della madre, del padre, della propria venuta al mondo. 

- Ma dunque, proprio per questo è indispensabile il divorzio 

Ma Anna non l'ascoltava. Voleva esporre fino in fondo quegli argomenti con cui si era persuasa tante volte. 

- E perché mi è data la ragione, se non l'adopero per non mettere al mondo dei disgraziati? 

Guardò per un attimo Dolly, ma senza aspettare risposta, continuò. 

- Io mi sentirei sempre colpevole dinanzi a questi disgraziati figli - disse. - Se non ci sono, almeno non sono disgraziati; invece se sono disgraziati, la colpa è mia soltanto. 

Erano quegli identici argomenti che Dar'ja Aleksandrovna aveva portato a se stessa; ma adesso li ascoltava e non li intendeva: ``Come essere colpevole dinanzi a esseri non esistenti?'' pensava. E a un tratto le venne un pensiero, se avrebbe potuto, in un qualche caso, esser meglio per il suo Griša, il beniamino, di non esistere. E questo le parve così strano, così bizzarro, che scosse un po' la testa, quasi a cacciarne via quella confusione di pensieri che vi turbinava. 

- No, non so, non è bene - ella disse soltanto, con una espressione di ripugnanza in viso. 

- Sì, ma tu non devi dimenticare cosa sei tu e cosa sono io\ldots{} E poi - soggiunse Anna che, malgrado la ricchezza dei propri argomenti dinanzi alla povertà di quelli di Dolly, aveva tuttavia l'aria di confessare che ciò non era bene - tu non dimenticare la cosa principale, che adesso io mi trovo in una posizione non simile alla tua. Per te la questione è se devi o no desiderare di non avere più figli, e per me se devo o no desiderare di averne. C'è una grande differenza. Capirai come io non possa desiderarne nella mia posizione. 

Dar'ja Aleksandrovna non obiettava nulla. Aveva sentito a un tratto di essere già così lontana da Anna, che fra di loro esistevano questioni sulle quali non si sarebbero mai intese e delle quali era meglio non parlare. 

\capitolo{XXIV}\label{xxiv-4} 

- Allora tanto più devi accomodare la tua situazione, se è possibile - disse Dolly. 

- Sì, se è possibile - disse Anna, a un tratto, con voce completamente diversa, piana e triste. 

- Non è forse possibile il divorzio? M'hanno detto che tuo marito acconsente. 

- Dolly! Non voglio parlare di questo. 

- Via, non ne parleremo - si affrettò a dire Dar'ja Aleksandrovna dopo aver notato l'espressione di sofferenza nel viso di Anna. - Io vedo soltanto che tu guardi la cosa troppo cupamente. 

- Io? per nulla. Io sono allegra e soddisfatta. Hai visto. Je fais des passions. Veslovskij\ldots{} 

- Sì, a dire la verità, non mi è piaciuto il tono di Veslovskij - disse Dar'ja Aleksandrovna, desiderando cambiare discorso. 

- Ah! per nulla! Ciò eccita Aleksej e niente altro; ma lui è un ragazzo, ed è tutto nelle mie mani; capisci, lo dirigo come voglio. È proprio come il tuo Griša\ldots{} Dolly! - ella cambiò discorso a un tratto - tu dici che io guardo cupamente. Tu non puoi capire. È troppo orribile. Io cerco di non guardare proprio. 

- Ma bisogna, mi pare. Bisogna fare quello che è possibile. 

- Ma cosa mai è possibile? Nulla. Tu dici che devo sposare Aleksej e che non ci penso. Io non ci penso! - ripeté e il rossore le apparve sul viso. Si alzò, raddrizzò il busto, sospirò penosamente, e si mise a camminare col suo passo leggero avanti e indietro per la stanza, fermandosi di tanto in tanto. - Non ci penso? Non c'è giorno, non c'è ora in cui non pensi e non mi rimproveri perché penso\ldots{} perché questi pensieri possono fare impazzire. Fare impazzire - ella ripeté. - Quando ci penso, non mi addormento mai senza morfina. Ma via. Parliamo con calma. Mi dicono: il divorzio. In primo luogo, lui non me lo darà. Lui adesso è sotto l'influenza della contessa Lidija Ivanovna. 

Dar'ja Aleksandrovna, allungata sulla sedia, volgendo il capo, seguiva, con viso di sofferente compassione, Anna che camminava. 

- Bisogna tentare - ella disse piano. 

- Ammettiamolo, tentare. Che cosa significa? - disse, esprimendo evidentemente un pensiero mille volte ripensato e imparato a memoria. - Significa che io, che lo detesto, eppure mi riconosco colpevole dinanzi a lui, e lo stimo magnanimo, io mi umilio a scrivergli\ldots{} Ma ammettiamo pure; farò uno sforzo, lo farò. O riceverò una risposta offensiva o il consenso. Va bene, ricevo il consenso\ldots{} - Anna in quel momento era in un angolo lontano della stanza e si fermò là, facendo qualcosa con la tenda d'una finestra. - Riceverò il consenso, e mio fi\ldots{} figlio? Perché non me lo renderanno. Perché lui cresce disprezzandomi, accanto a un padre che io ho abbandonato. Tu devi capire che io amo, mi pare, allo stesso modo, ma tutti e due più di me, due esseri: Serëza e Aleksej. 

Ella venne in mezzo alla stanza e si fermò dinanzi a Dolly, premendosi il petto con le mani. Nell'accappatoio bianco la sua figura sembrava particolarmente grande e larga. Aveva chinato il capo e guardava di sotto in su con gli occhi scintillanti e umidi la figura magra, piccola e meschina di Dolly, con la sua camicetta rammendata e la cuffia da notte, tutta tremante per l'agitazione. 

- Soltanto questi due esseri io amo, e l'uno esclude l'altro. Io non posso unirli, eppure questa sola cosa mi è necessaria. E se questo non è, allora è tutto lo stesso. Tutto, tutto è lo stesso. E in qualche modo finirà, e perciò non posso, non amo parlare di questo. Così tu non giudicarmi, non rimproverarmi nulla. Tu, con la tua purezza, non puoi capire tutto quello che io soffro. 

Si avvicinò, sedette accanto a Dolly e, esaminando il suo viso con un'espressione di colpa, la prese per mano. 

- Cosa pensi? cosa pensi di me? Non mi disprezzare. Io non merito il disprezzo. Sono proprio disgraziata. Se c'è un essere disgraziato, questo sono io - disse, e, voltando le spalle, si mise a piangere. 

Rimasta sola, Dolly pregò Iddio e si coricò. Aveva provato pena di Anna, mentre aveva parlato con lei; ma ora non poteva costringersi a pensare a lei. I ricordi della casa e dei bambini sorgevano nella sua immaginazione con un fascino particolare, nuovo per lei, in una luce nuova. Questo suo mondo adesso le era apparso così caro e gentile, che non voleva per nessuna ragione trascorrere una giornata di più fuori di esso e decise di partire assolutamente l'indomani. 

Anna, intanto, tornata nel suo studio, prese un bicchierino e vi versò alcune gocce di un medicinale, in cui predominava la morfina, e, dopo aver bevuto ed essere rimasta a sedere immobile qualche tempo, andò in camera sua con lo spirito rasserenato e gaio. 

Quando entrò in camera Vronskij la guardò con attenzione. Cercava i segni di quel colloquio che, per essere rimasta così a lungo nella stanza di Dolly, egli sapeva doveva aver avuto con lei. Ma nella sua espressione eccitata e contenta, ma che nascondeva qualcosa, egli non trovò nulla, oltre la bellezza, per lui abituale, e che sempre ancora lo seduceva, la consapevolezza di essa e il desiderio che agisse su di lui. Egli non voleva chiederle cosa avessero detto, ma sperava ch'ella stessa ne riferisse qualcosa. Ma ella disse solo: 

- Sono contenta che ti sia piaciuta Dolly. Non è vero? 

- Ma io la conosco da tempo. È molto buona, mi pare, mais excessivement terre-à-terre. Ma sono stato molto contento di vederla. 

Egli prese le mani di Anna e la guardò negli occhi interrogativamente. 

Ella, capito in altro modo quello sguardo, gli sorrise. 

La mattina dopo, malgrado le insistenze dei padroni di casa, Dar'ja Aleksandrovna si preparò a partire. Il cocchiere di Levin, col gabbano non più nuovo e il cappello quasi a foggia di postiglione, coi cavalli di colore diverso, la vettura dai parafanghi rappezzati, entrò torvo e risoluto nell'androne coperto, cosparso di sabbia. 

Il congedo dalla principessa Varvara e dagli uomini era poco piacevole per Dar'ja Aleksandrovna. Dopo essere rimasta un giorno, sia lei che i padroni avevano sentito che non si confacevano reciprocamente, e che era meglio non trovarsi insieme. Solo Anna era triste. Sentiva che con la partenza di Dolly nessuno più avrebbe agitato nell'animo suo quei sentimenti che s'erano sollevati in lei in quell'incontro. Agitare questi sentimenti era per lei doloroso; tuttavia sapeva che era la parte migliore dell'animo suo e che questa parte dell'animo suo si sarebbe cicatrizzata rapidamente nella vita che conduceva. 

Quando furono usciti nei campi, Dar'ja Aleksandrovna provò un piacevole senso di sollievo e voleva domandare agli uomini come s'erano trovati da Vronskij quando subito il cocchiere Filipp prese a parlare: 

- Ricchi, ricconi, ma d'avena ne hanno date tre misure in tutto. Prima che cantassero i galli, gli animali avevano già tutto ripulito. Che cosa sono tre misure? solo per l'antipasto. Al giorno d'oggi l'avena, dai portieri, sta a quarantacinque copeche. Da noi, grazie al cielo, a quelli che arrivano, quanta ne mangiano tanta ne danno. 

- Un signore avaro - confermò lo scrivano. 

- Be', e i cavalli ti son piaciuti? - domandò Dolly. 

- Son cavalli, in una parola. E il mangiare è buono. Ma a me è parso un po' d'annoiarmi, Dar'ja Aleksandrovna, non so, se a voi - disse, volgendo verso di lei il suo viso bello e buono. 

- Ma anche a me. Che dici, arriveremo verso sera? 

- Bisogna arrivare. 

Tornata a casa e trovati tutti completamente felici e particolarmente simpatici, Dar'ja Aleksandrovna raccontò con grande animazione del suo viaggio, di come l'avevano accolta bene, dello sfarzo e del buon gusto della vita dei Vronskij, degli svaghi, e non permise a nessuno di dire una parola contro di loro. 

- Bisogna conoscere Anna e Vronskij; io adesso li ho conosciuti meglio, per capire come siano simpatici e commoventi - ella diceva, con piena sincerità, dimenticando quel vago senso di scontento e di disagio che aveva provato in casa loro. 

\capitolo{XXV}\label{xxv-4} 

Vronskij e Anna, sempre nelle stesse condizioni, sempre senza prendere alcun provvedimento per il divorzio, passarono tutta l'estate e parte dell'autunno in campagna. Era stabilito tra di loro che non sarebbero andati in nessun posto; ma tutti e due sentivano che, quanto più avrebbero vissuto soli, specie di autunno e senza ospiti, tanto più non avrebbero sopportato quella vita e sarebbe stato necessario cambiarla. 

La vita pareva tale, quale migliore non si poteva desiderare: c'era ogni agio, c'era la salute, c'era la bambina, e tutti e due avevano delle occupazioni. Anna, senza ospiti, si occupava di sé sempre allo stesso modo e leggeva moltissimo: romanzi e libri seri, alla moda. Ordinava tutti i libri che erano recensiti favorevolmente nei giornali e nelle riviste che riceveva dall'estero, e li leggeva con quell'attenzione che si pone nella lettura soltanto nella solitudine. Inoltre, tutte le materie di cui si occupava Vronskij, lei le studiava sui libri e sulle riviste specializzate, così che spesso egli si rivolgeva direttamente a lei con domande di agricoltura, di architettura, perfino di allevamento equino e di sport. Egli si sorprendeva delle conoscenza, della memoria di lei, e in principio dubbioso, desiderava una conferma; e lei trovava nei libri quello di cui egli l'aveva richiesta e glielo mostrava. 

Anche la costruzione dell'ospedale la occupava. Non soltanto aiutava, ma molte cose le organizzava e le pensava lei. Ma la sua preoccupazione maggiore era tuttavia la propria persona, lei stessa, in quanto era cara a Vronskij, in quanto poteva sostituire per lui tutto quello ch'egli aveva abbandonato. Vronskij apprezzava questo desiderio, che era divenuto l'unico scopo della vita di lei, non solo di piacergli, ma di essergli utile; tuttavia nello stesso tempo, sentiva anche il peso di quelle reti amorose in cui ella cercava di avvolgerlo. Quanto più il tempo passava, quanto più spesso egli si vedeva avvolto in queste reti, tanto più gli veniva il desiderio non di uscirne, ma di provare se queste non intralciassero la sua libertà. Se non fosse stato questo desiderio sempre più forte di sentirsi libero, di non avere scenate ogni volta che doveva andare in città per un'assemblea, alle corse, Vronskij sarebbe stato del tutto soddisfatto della propria vita. La parte ch'egli aveva scelto, la parte di uno di quei ricchi proprietari di terre, che dovevano costituire il nerbo dell'aristocrazia russa, non solo lo soddisfaceva in pieno, ma adesso, dopo sei mesi di questa vita, gli procurava una soddisfazione sempre maggiore. E il suo lavoro, occupandolo e assorbendolo sempre di più procedeva benissimo. Malgrado le enormi spese sostenute per l'ospedale, le macchine, le mucche fatte venire dalla Svizzera e molte altre cose, era sicuro di non dissestare, ma di valorizzare la propria sostanza. Là dove si trattava di entrate, di vendita di boschi, di grano, di lana, del fitto delle terre, Vronskij era duro come la pietra, e sapeva mantenere il prezzo. Negli affari economici in grande, in questo e negli altri possedimenti, si atteneva ai sistemi più semplici e meno arrischiati ed era estremamente economo e calcolatore nelle piccole cose dell'azienda. Malgrado tutta la furberia e l'abilità del tedesco, che lo spingeva agli acquisti ed esponeva ogni calcolo in modo che in principio ci voleva molto di più, ma dopo aver riflettuto, si poteva far la stessa cosa più a buon mercato e ricavarne subito un utile, Vronskij non gli si sottometteva mai. Ascoltava l'amministratore, lo interrogava e consentiva con lui soltanto quando quello che si ordinava o si organizzava era la cosa più nuova, ancora sconosciuta in Russia, che poteva suscitare la meraviglia. Inoltre si decideva a una grande spesa solo quando c'era denaro superfluo e, facendo tale spesa, entrava in tutti i dettagli e insisteva per avere la cosa migliore in corrispettivo dei suoi denari. Così che, dal modo come egli conduceva i propri affari era chiaro che non dissestava, ma aumentava il patrimonio. 

Nel mese di ottobre c'erano le elezioni nobiliari nel governatorato di Kašin, dove erano i possedimenti di Vronskij, di Svijazskij, di Koznyšev, di Oblonskij e una piccola parte di quelli di Levin. 

Queste elezioni, per molte circostanze e per le persone che vi partecipavano, attiravano l'attenzione generale. Se ne parlava molto e ci si preparava. Da Mosca, da Pietroburgo e dall'estero, persone che non andavano mai alle elezioni, convenivano per quelle. 

Vronskij aveva promesso a Svijazskij di andarci, già da tempo. 

Prima delle elezioni, Svijazskij, che visitava spesso Vozdvizenskoe, venne a prendere Vronskij. 

Fin dalla vigilia di questo giorno, fra Vronskij e Anna c'era stata una discussione per il viaggio progettato. Era il periodo autunnale più noioso, più penoso in campagna, e perciò Vronskij, preparandosi alla lotta, con un'espressione severa e fredda, come non aveva mai parlato ad Anna, le annunciò la sua partenza. Ma, con suo stupore, Anna accolse questa notizia con molta calma e domandò solo quando sarebbe tornato. Egli la guardò attento, senza spiegarsi questa calma. Lei sorrise al suo sguardo. Egli conosceva in lei la facoltà di ritrarsi in se stessa, e sapeva che ciò accadeva solo quando aveva deciso qualcosa fra di sé, senza comunicargli i suoi piani. Egli ne aveva paura; ma aveva un così grande desiderio di evitare una scenata, che finse di credere, e in parte sinceramente credette, a quello che voleva credere, alla ragionevolezza di lei. 

- Spero che non ti annoierai. 

- Spero - disse Anna. - Ieri ho ricevuto una cassa di libri da Gauthier. No, non mi annoierò. 

``Lei vuol prender questo tono, e tanto meglio - egli pensò - e se no, fa lo stesso''. 

E così, senza averla spinta a una spiegazione sincera, partì per le elezioni. Era la prima volta dal principio della loro relazione, ch'egli si separava da lei senza che si fossero spiegati fino in fondo. Da una parte ciò lo agitava, dall'altra giudicava che fosse meglio. ``Da principio ci sarà, come ora, qualcosa di poco chiaro, di nascosto, ma dopo ci si abituerà. In ogni caso io posso darle tutto, ma non la mia indipendenza di uomo'' pensò. 

\capitolo{XXVI}\label{xxvi-4} 

A settembre Levin si trasferì a Mosca per il parto di Kitty. Viveva già da un mese a Mosca senza far nulla, quando Sergej Ivanovic, che aveva un podere nel governatorato di Kašin e prendeva parte attiva alle imminenti elezioni, si preparò a recarvisi. Egli invitava ad andare con lui anche il fratello, che aveva un voto per il distretto di Seleznevsk. Oltre a ciò, Levin aveva a Kašin un affare di estrema urgenza, per sua sorella che viveva all'estero, circa una tutela e l'esazione del denaro di un riscatto. 

Levin era ancora incerto, ma Kitty, vedendo ch'egli si annoiava a Mosca, gli consigliò di andare, e gli ordinò, a sua insaputa, una divisa nobiliare che costava ottanta rubli. E questi ottanta rubli, pagati per la divisa, furono la ragione principale che spinse Levin ad andare. Andò a Kašin. 

Levin era a Kašin già da sei giorni, frequentando tutti i giorni l'assemblea e affannandosi per l'affare della sorella che non riusciva a sistemare. I marescialli della nobiltà erano tutti presi dalle elezioni, e non si riusciva a portare a termine quel semplicissimo affare che dipendeva dalla tutela. L'altro affare, poi, la riscossione del denaro, incontrava gli stessi ostacoli. Dopo lunghe sollecitazioni perché fosse tolto il divieto, il denaro era pronto per il pagamento; ma il notaio, uomo servizievolissimo, non poteva consegnare il mandato, perché era necessaria la firma del presidente, e il presidente, senza aver fatto la consegna dell'ufficio, era alla sessione. Tutte queste preoccupazioni, l'andare da una parte all'altra, le conversazioni con quelle buone, brave persone, che si rendevano pienamente conto della spiacevolezza della situazione del sollecitatore, ma non potevano aiutarlo, tutta quella tensione, che non portava a nulla, producevano in Levin un senso di molestia, simile a quella stizzosa impotenza che si prova in sogno quando si vuole usare della forza fisica. Egli provava spesso ciò, discorrendo con quel brav'uomo del suo procuratore. Questo procuratore sembrava far tutto il possibile e tendere tutte le sue forze intellettive per togliere Levin dalle difficoltà. ``Ecco cosa dovete provare - diceva più di una volta - andate prima là e poi là'' e il procuratore faceva tutto un piano, per superare quel fatale principio che intralciava tutto. Ma subito aggiungeva: ``Tuttavia non lo rilasceranno; provate, comunque''. E Levin provava, andava a piedi, andava in carrozza. Tutti erano buoni e cortesi, ma veniva in chiaro che ciò che si era evitato spuntava di nuovo alla fine e di nuovo intralciava lo svolgimento della pratica. Era particolarmente increscioso che Levin non potesse in nessun modo capire contro chi lottava, né chi traesse vantaggio dal fatto che il suo affare non giungesse a termine. Questo, nessuno sembrava saperlo; neppure il procuratore lo sapeva. Se Levin avesse potuto capire, così come capiva perché alla stazione non ci si può avvicinare alla biglietteria senza mettersi in fila, non avrebbe provato stizza e vergogna; ma degli ostacoli che incontrava nella sua faccenda nessuno gli poteva dire perché esistessero. 

Ma Levin era molto cambiato dal tempo del suo matrimonio; era paziente, e, se non capiva perché tutto questo fosse disposto così, si diceva che, non sapendo tutto, non poteva giudicare, che, probabilmente, doveva essere così e cercava di non indignarsene. 

Ora, assistendo alle elezioni e partecipandovi, cercava egualmente di non giudicare, di non discutere, ma per quanto era possibile cercava di capire quella faccenda di cui si occupavano con tanta serietà e passione persone oneste e brave, da lui stimate. Da quando s'era sposato, gli si erano rivelati tanti lati nuovi e profondi della vita che prima, per la leggerezza con cui li trattava, gli sembravano insignificanti; così anche nell'affare delle elezioni egli supponeva e ricercava un senso profondo. 

Sergej Ivanovic gli spiegò il senso e il significato del cambiamento di rotta che si prevedeva nelle elezioni. Il maresciallo della nobiltà del governatorato, nelle cui mani, secondo la legge, si trovavano tanti pubblici affari (le tutele, quelle stesse per cui Levin adesso soffriva, e somme enormi di denaro nobiliare, e i ginnasi, quello femminile, quello maschile e quello militare, e l'istruzione popolare, secondo il nuovo regolamento, e infine il consiglio distrettuale), il maresciallo della nobiltà del governatorato, Snetkov, era un uomo d'antico stampo nobiliare, che aveva sperperato un enorme patrimonio, un uomo buono, onesto nel suo genere, ma che non capiva affatto le esigenze dei tempi nuovi. Sosteneva sempre, in tutto, le parti della nobiltà, contrastava apertamente la diffusione dell'istruzione popolare, e dava al consiglio distrettuale, che doveva avere una così vasta importanza, un carattere di classe. Bisognava mettere al suo posto un uomo nuovo, moderno, attivo, completamente diverso, e bisognava condurre la cosa in modo da ricavare da tutti i diritti concessi alla nobiltà, non come nobiltà, ma come elemento del consiglio distrettuale, tutti quei vantaggi d'autonomia che potevano essere ricavati. Nel ricco governatorato di Kašin, che era sempre in testa, s'erano raccolte adesso tali forze che l'azione condotta là, come andava condotta, poteva servire di modello per gli altri governatorati, per tutta la Russia. E per questo tutta la faccenda aveva un grande significato. Come maresciallo al posto di Snetkov, si pensava di mettere o Svijazskij o, ancora meglio, Nevedovskij, che era docente universitario, uomo di straordinaria intelligenza e grande amico di Sergej Ivanovic. 

L'assemblea l'aprì il governatore, che fece un discorso ai nobili, perché eleggessero funzionari non secondo le simpatie personali, ma secondo i meriti e il bene della patria, e disse di sperare che la degna nobiltà di Kašin, come nelle elezioni precedenti, avrebbe compiuto santamente il proprio dovere e avrebbe giustificato l'alta fiducia del monarca. 

Finito il discorso, il governatore andò via dalla sala, e i nobili lo seguirono con rumorosa animazione, mentre egli infilava la pelliccia e discorreva cordialmente con il maresciallo del governatorato. Levin, desiderando di penetrare il senso di tutto e di non lasciarsi sfuggire nulla, stava dritto proprio lì nella folla e sentiva come il governatore diceva: ``Per favore dite a Mar'ja Ivanovna che a mia moglie spiace molto di non poter andare all'asilo''. E dopo di questo, i nobili presero allegramente le pellicce e andarono tutti alla cattedrale. 

Nella cattedrale Levin, sollevando il braccio e ripetendo le parole dell'arciprete insieme con gli altri, giurò con i giuramenti più terribili di compiere tutto quello che sperava il governatore. Il servizio divino aveva sempre un influsso su Levin, e quando egli pronunciò le parole: ``bacio la croce'' e si voltò a guardare quella folla di persone giovani e vecchie, che ripetevano la stessa cosa, si sentì commosso. 

Il secondo e il terzo giorno si discussero gli affari delle somme di denaro nobiliare e del ginnasio femminile, che non avevano, come spiegò Sergej Ivanovic, alcuna importanza, e Levin, preso dal corso dei suoi affari, non li seguì. Il quarto giorno, intorno al tavolo del governatorato, si svolse la verifica delle somme di denaro del governatorato stesso. E qui, per la prima volta, avvenne un urto fra il partito nuovo e il vecchio. La commissione, incaricata di verificare le somme, riferì che le somme erano intatte. Il maresciallo del governatorato si alzò, ringraziando la nobiltà per la fiducia, e sparse qualche lacrima. I nobili lo acclamarono a gran voce e gli strinsero la mano. Ma in quel punto, un nobile del partito di Sergej Ivanovic disse di aver sentito che la commissione non aveva verificato le somme, considerando la verifica un'offesa per il maresciallo del governatorato. Uno dei membri della commissione, incautamente, confermò la cosa. Allora un signore molto piccolo, molto giovane all'aspetto, ma velenoso, prese a dire che al maresciallo del governatorato avrebbe, probabilmente, fatto piacere dare un rendiconto delle somme, e che la eccessiva delicatezza dei membri della commissione lo privava di questa soddisfazione morale. Allora i membri della commissione rinunciarono alla propria dichiarazione, e Sergej Ivanovic cominciò a dimostrare, a fil di logica, che bisognava o riconoscere che le somme erano state da loro verificate o che non lo erano state, e svolse minuziosamente questo dilemma. A Sergej Ivanovic ribatté l'oratore del partito opposto. Poi parlo Svijazskij e di nuovo il signore velenoso. Le discussioni durarono a lungo e finirono senza concludere nulla. Levin era sorpreso che si discutesse così a lungo di questo, soprattutto perché quando aveva chiesto a Sergej Ivanovic se egli supponeva che le somme fossero state malversate, Sergej Ivanovic aveva risposto: 

- Oh, no! È una persona onesta. Ma questo sistema antiquato di amministrazione familiare, paterna, degli affari nobiliari bisogna scrollarlo. 

Il quinto giorno ci furono le elezioni dei marescialli distrettuali. Questa giornata fu abbastanza burrascosa in alcuni distretti. Nel distretto di Seleznevsk, Svijazskij fu eletto, all'unanimità senza ballottaggio, e quel giorno ci fu un pranzo per lui. 

\capitolo{XXVII}\label{xxvii-4} 

Per il sesto giorno erano fissate le elezioni del governatorato. Le sale grandi e le piccole erano piene di nobili in varie divise. Molti erano arrivati solo quel giorno. Amici che non si vedevano da tempo, chi dalla Crimea chi da Pietroburgo, chi dall'estero, s'incontravano nelle sale. Alla tavola del governatorato, sotto il ritratto dello zar, si svolgevano i dibattiti. 

I nobili, nella sala grande e nella piccola, si raggruppavano in campi opposti, e dall'ostilità e dalla diffidenza, dal discorso che taceva all'avvicinarsi di gente estranea, dal fatto che alcuni, parlando sottovoce, si spingevano perfino in un corridoio lontano, si vedeva che ciascuna parte aveva dei misteri da nascondere all'altra. Dall'aspetto esteriore i nobili si dividevano decisamente in due specie: nei vecchi e nei nuovi. I vecchi erano per la maggior parte o in vecchie divise nobiliari abbottonate, con spade e cappelli, o nelle loro particolari divise, che s'erano conquistate, di marina, di cavalleria, di fanteria. Le divise dei vecchi nobili erano cucite all'antica, con gli sbuffi sulle spalle; erano, evidentemente, piccole, corte di vita e strette come se coloro che le portavano ne fossero cresciuti fuori. I giovani erano, invece, con le divise nobiliari sbottonate, con la vita bassa e le spalle larghe, coi panciotti bianchi, o con le divise dal colletto nero e i fregi a foglie di lauro, emblema del ministero della giustizia. Pure ai giovani appartenevano alcune divise di corte, che abbellivano qua e là la massa. 

Ma la divisione in giovani e vecchi non corrispondeva alla divisione dei partiti. Alcuni fra i giovani, come aveva osservato Levin, appartenevano al partito vecchio, e, al contrario, alcuni dei nobili più vecchi parlavano sottovoce con Svijazskij, ed erano, evidentemente, caldi sostenitori del partito nuovo. 

Levin stava in piedi nella sala piccola, dove si fumava e si mangiava, accanto a un gruppo dei suoi, prestando orecchio a quello che dicevano e tendendo invano le forze del suo ingegno per capire cosa dicessero. Sergej Ivanovic era il centro intorno al quale si raggruppavano gli altri. Egli ora ascoltava Svijazskij e Chljustov, maresciallo di un altro distretto, che apparteneva al loro partito. Chljustov non consentiva ad andare da Snetkov col suo distretto e pregarlo di mettersi in ballottaggio, ma Svijazskij lo esortava a farlo e Sergej Ivanovic approvava questo piano. Levin non capiva perché il partito contrario dovesse pregare di mettere in ballottaggio quel maresciallo ch'essi non volevano eleggere. 

Stepan Arkad'ic, che proprio allora aveva finito di mangiare e di bere, asciugandosi la bocca con un fazzoletto di batista, ricamato e profumato, si accostò a loro nella sua divisa di ciambellano. 

- Occupiamo la posizione - disse, lisciandosi tutte e due le fedine - Sergej Ivanyc! 

E prestato ascolto alla conversazione, confermò l'opinione di Svijazskij. 

- Basta un distretto, e Svijazskij significa già, evidentemente, l'opposizione - disse con parole comprensibili a tutti fuor che a Levin. 

- Be', Kostja, anche tu, a quanto pare, ci hai preso gusto? - aggiunse rivolto a Levin, e lo prese sotto braccio. Levin sarebbe stato contento di prenderci gusto, ma non riusciva a capire di che cosa si trattava e, allontanandosi di alcuni passi da quelli che parlavano, espresse a Stepan Arkad'ic la propria meraviglia, per il fatto che bisognasse pregare il maresciallo del governatorato. 

- O Sancta simplicitas! - disse Stepan Arkad'ic e, in breve e con chiarezza, spiegò a Levin di che si trattava. 

Se tutti i distretti, come nelle elezioni passate, avessero pregato il maresciallo del governatorato, l'avrebbero eletto con tutte palle bianche. Questo non doveva accadere. Ora, otto distretti consentivano a pregarlo; se invece due si rifiutavano di pregarlo, Snetkov avrebbe potuto rinunciare al ballottaggio. E allora il vecchio partito poteva eleggere un altro dei suoi, poiché tutto il calcolo sarebbe andato perduto. Ma se il solo distretto di Svijazskij non lo avesse pregato, Snetkov sarebbe stato messo in ballottaggio. Lo avrebbero perfino eletto, riservando dei voti per lui, così che il partito contrario si sarebbe confuso nei calcoli e quando avessero proposto un candidato dei nostri, avrebbero portato i voti su di lui. 

Levin capì, ma non perfettamente, e voleva ancora fare delle domande, quando a un tratto tutti presero a parlare, a rumoreggiare e a muoversi verso la sala grande. 

- Che c'è? cosa? chi? - La procura? a chi? perché? - La ricusano? - Non la procura. - Non ammettono Flerov. - Ma che cosa, se è sotto giudizio? - Così non ammetteranno nessuno. Ciò è vile. - La legge! - sentiva Levin da varie parti e, insieme con gli altri che si affrettavano chi sa dove, e temevano di perdere qualcosa, si diresse nella sala grande e, stretto dai nobili, si accostò al tavolo del governatorato, presso al quale discutevano con calore il maresciallo del governatorato, Svijazskij e altri rappresentanti. 

\capitolo{XXVIII}\label{xxviii-4} 

Levin era in piedi, abbastanza lontano. Un nobile che accanto a lui respirava greve, con l'affanno, e un altro che scricchiolava con le suole doppie, gli impedivano di ascoltare chiaramente. Sentiva soltanto la voce morbida, suadente del maresciallo, poi quella stridula del nobile velenoso e poi la voce di Svijazskij. Discutevano, a quanto egli poteva capire, sul significato degli articoli di legge o sul significato delle parole: ``trovantesi sotto inchiesta''. 

La folla si divise per lasciar passare Sergej Ivanovic che si accostava al tavolo. Sergej Ivanovic, dopo aver ascoltato la fine del discorso del nobile velenoso disse che gli sembrava che la cosa più sicura da fare sarebbe stata la consultazione dell'articolo di legge, e pregò il segretario di trovare l'articolo. Nell'articolo era detto che, in caso di dissenso, bisognava passare alla votazione. 

Sergej Ivanovic lesse l'articolo e cominciò a spiegare il senso, ma a questo punto, un proprietario alto e grasso, un po' curvo, con i baffi tinti, in una divisa stretta, con un bavero che gli sosteneva il collo da dietro, lo interruppe. Si accostò al tavolo e, picchiandovi sopra con un anello, cominciò a gridare a gran voce: 

- Bisogna votare! Alle urne! Senza chiacchiere! Alle urne! 

Allora, improvvisamente, presero a levarsi parecchie voci, e il nobile alto con l'anello, irritandosi sempre più, gridava sempre più forte. Non si poteva distinguere quello che diceva. 

Diceva la stessa cosa che aveva proposto Sergej Ivanovic, ma evidentemente odiava lui e tutto il suo partito, e questo senso di odio si comunicava a tutto il partito e suscitava un'eguale animosità in risposta, anche se più corretta, dall'altra parte. Si levarono grida e per un momento tutto si fece confuso, tanto che il maresciallo del governatorato dovette invocare l'ordine. 

- Bisogna votare, votare! Chi è nobile capisce\ldots{} Noi versiamo il sangue\ldots{} La fiducia del monarca\ldots{} Non fare i conti col maresciallo\ldots{} non è mica un amministratore\ldots{} Ma non si tratta di questo\ldots{} Permettete che si vada alla votazione!\ldots{} - si udiva gridare animosamente, con violenza, da tutte le parti. Gli sguardi e le espressioni erano ancora più irritati e più violenti delle grida. Esprimevano un odio irriconciliabile. Levin non riusciva in nessun modo a capire di che si trattasse e si stupiva della passionalità con cui si esaminava la questione se mettere o no ai voti la opinione su Flerov. Dimenticava, come poi gli spiegò Sergej Ivanovic, questo sillogismo: che per il bene comune bisognava far cadere il maresciallo del governatorato; per far cadere il maresciallo era necessaria la maggioranza dei voti, per la maggioranza dei voti bisognava dare a Flerov il diritto di voto; per riconoscere Flerov idoneo, bisognava spiegare l'articolo della legge. 

- E un voto può decidere tutto l'affare, bisogna essere seri e coerenti, se si vuol servire la causa pubblica - concluse Sergej Ivanovic. 

Ma Levin se l'era dimenticato, e gli era penoso vedere quelle brave persone, da lui stimate, in un'eccitazione così spiacevole e perversa. Per liberarsi da questa sensazione penosa, senz'attendere la fine del dibattito, se ne andò in una sala dove non c'era nessuno, tranne i servitori vicino a una credenza. Nel vedere i camerieri, affaccendati a rasciugar stoviglie e a disporre piatti e bicchieri, nel vedere i loro visi calmi, animati, Levin provò un inaspettato senso di sollievo, come se da una stanza maleodorante fosse uscito all'aria aperta. Si mise a camminare avanti e indietro, guardando con soddisfazione i camerieri. Gli piaceva molto che un cameriere con le fedine grige, disprezzando gli altri giovani che lo prendevano in giro, insegnasse loro come piegare i tovaglioli. Levin stava proprio per intavolare una conversazione col vecchio cameriere, quando il segretario della tutela nobiliare, un vecchietto che aveva la specialità di conoscere tutti i nobili del governatorato per nome e patronimico, lo distrasse. 

- Prego, Konstantin Dmitric - gli disse - vostro fratello vi cerca. Si vota. 

Levin entrò nella sala, ricevette una pallina bianca e dietro a suo fratello Sergej Ivanovic si avviò al tavolo, presso al quale stava in piedi, con un viso espressivo e ironico, Svijazskij, raccogliendo nel pugno la barba e annusandola. Sergej Ivanovic mise la mano nella cassetta, mise chi sa dove la pallina e, fatto posto a Levin, si fermò proprio lì. Levin si avvicinò ma, avendo completamente dimenticato di che cosa si trattava, ed essendosi confuso, si voltò verso Sergej Ivanovic a chiedergli: ``dove la metto?''. Egli aveva domandato sottovoce, mentre là vicino si parlava, sperando che la sua domanda non fosse udita. Ma quelli che parlavano tacquero, e la sua domanda sconveniente fu udita. Sergej Ivanovic si accigliò: 

- È una questione che riguarda la convinzione del singolo - disse severo. 

Alcuni sorrisero. Levin arrossì, ficcò in fretta la mano sotto il panno e la mise a destra, poiché la palla era nella mano destra. Dopo averla messa ricordò che bisognava metter dentro anche la sinistra, e la mise, ma era tropo tardi, e, confusosi ancor più, si ritirò subito nelle ultime file. 

- Centoventisei favorevoli! Novantotto sfavorevoli! - risonò la voce del segretario che non pronunciava l'erre. Poi si sentirono delle risate: si eran trovati nell'urna un bottone e due noci. Il nobile era ammesso e il partito nuovo aveva vinto. 

Ma il partito vecchio non si considerava vinto. Levin sentì che pregavano Snetkov di presentarsi al ballottaggio, e vide che una folla di nobili circondava il maresciallo del governatorato che diceva qualcosa. Levin si fece dappresso. Rispondendo ai nobili, Snetkov parlava della fiducia della nobiltà, dell'amore per lui, che non meritava, poiché tutto il suo merito consisteva nella dedizione alla nobiltà, alla quale egli aveva dedicato dodici anni di servizio. Parecchie volte egli ripeté le parole: ``ho servito con quante forze avevo, con fede e verità, apprezzo e ringrazio'' e improvvisamente si fermò per le lacrime che lo soffocavano, e uscì dalla sala. Derivassero queste lacrime dalla coscienza d'una ingiustizia fattagli, o dall'amore per la nobiltà, o dalla tensione dovuta allo stato in cui si trovava nel sentirsi circondato da nemici, fatto sta che l'agitazione si comunicò, e la maggior parte dei nobili erano commossi, e Levin provò tenerezza per Snetkov. 

Sulla porta il maresciallo del governatorato si scontrò con Levin. 

- Chiedo scusa, perdonatemi, vi prego - disse egli come a un estraneo; ma, riconosciuto Levin, sorrise timidamente. A Levin sembrò ch'egli volesse dire qualcosa, ma che non potesse per l'agitazione. La espressione del viso e di tutta la sua figura in divisa, con le croci e i pantaloni bianchi gallonati, mentre camminava in fretta, ricordò a Levin una bestia inseguita che vede la propria situazione farsi cattiva. Questa espressione sul viso del maresciallo era in particolar modo emozionante per Levin, perché, soltanto il giorno prima, era stato a casa sua per l'affare della tutela e l'aveva visto in tutta la sua grandezza di uomo buono e casalingo. Una grande casa con vecchi mobili di famiglia; vecchi camerieri senza eleganza, un po' trasandati, ma rispettosi, che evidentemente provenivano ancora da servi della gleba che non avevano mai cambiato padrone; una moglie grassa e cordiale, in cuffia con pizzi e scialle turco che carezzava una graziosa nipotina, figlia della figlia; un bel figliuolo, allievo della sesta classe del ginnasio, che tornava dalla scuola e che, salutando il padre, gli aveva baciato la mano grossa; le parole e i gesti solenni, affabili del padrone di casa, tutto questo il giorno prima aveva destato in Levin un involontario rispetto e una forte simpatia. Per Levin adesso quel vecchio era commovente e pietoso, e voleva dirgli qualcosa di cordiale. 

- Siete dunque di nuovo il nostro maresciallo - disse. 

- È difficile - disse il maresciallo, voltandosi a guardare spaventato. - Io sono stanco, già vecchio. Ce n'è di più degni e di più giovani di me; che questi prestino servizio. 

E il maresciallo sparve per una porta laterale. 

Venne il momento più solenne. Si doveva procedere alle elezioni. I capi dell'uno e dell'altro partito calcolavano sulle dita le palline bianche e le nere. 

Le discussioni su Flerov avevano dato al partito nuovo non solo il voto di Flerov, ma anche il tempo per portare a votare tre nobili che erano stati privati, dagli intrighi del partito vecchio, della possibilità di partecipare alle elezioni. Due di essi, che avevano un debole per il vino, erano stati ubriacati dai fautori di Snetkov, e a un terzo era stata tolta la divisa. 

Saputo ciò, il partito nuovo, durante le discussioni su Flerov, aveva avuto il tempo di mandare con una vettura qualcuno dei suoi a procurare una divisa al nobile e a portare uno dei due ubriachi all'assemblea. 

- Uno l'ho portato, gli ho versato dell'acqua addosso - proferì il proprietario ch'era andato a prelevarlo, avvicinandosi a Svijazskij. - Non c'è male, può andare. 

- Non è ubriaco fradicio, non cascherà? - stava chiedendo Svijazskij. 

- No, sta su da bravo, purché qui non gli diano da bere\ldots{} Ho detto al cameriere che non gliene dia per nessuna ragione al mondo. 

\capitolo{XXIX}\label{xxix-4} 

La sala stretta nella quale si fumava e si mangiucchiava, era piena di nobili. L'agitazione aumentava sempre più e su tutti i visi si notava l'inquietudine. Erano agitati specialmente i capi che conoscevano tutti i particolari e lo scrutinio di tutti i voti. Erano gli organizzatori del combattimento imminente. Gli altri invece, come i soldati prima della battaglia, anche se pronti a battersi, cercavano di distrarsi. Alcuni mangiavano qualcosa in piedi o seduti al tavolo; altri camminavano, fumando sigarette, avanti e indietro per la stanza lunga e discorrevano con amici che non vedevano da tempo. 

Levin non aveva voglia di mangiare e non fumava; unirsi ai suoi, cioè a Sergej Ivanovic, Stepan Arkad'ic, Svijazskij e gli altri, non voleva, perché stava con loro in animato colloquio Vronskij, in divisa da scudiere. Anche il giorno prima Levin l'aveva visto alle elezioni e lo aveva evitato con cura, non desiderando di incontrarsi con lui. Si avvicinò alla finestra e sedette, osservando i gruppi e prestando orecchio a quel che si diceva intorno a lui. Era triste in particolar modo perché tutti, come vedeva, erano animati, preoccupati, e soltanto lui con un vecchietto sdentato, vecchio decrepito, in divisa di marina, che biascicava con le labbra e s'era venuto a sedere accanto a lui, non prendeva interesse e parte a nulla. 

- È un volpone quello! io glielo dicevo e lui, no. Come! In tre anni non poteva prepararsi - diceva energicamente un proprietario un po' curvo, non alto, coi capelli impomatati sopra il colletto ricamato della divisa, battendo con forza i tacchi degli stivali nuovi, messi per le elezioni. E il proprietario, dopo aver lanciato uno sguardo di insoddisfazione su Levin, si voltò bruscamente. 

- Sì, è un affare poco pulito, non c'è che dire - pronunciò con voce sottile un proprietario di piccola statura. 

Dietro a questi, tutta una folla di proprietari, che circondava un generale grasso, si avvicinò frettolosa a Levin. I proprietari cercavano evidentemente un luogo dove poter parlare senza essere uditi. 

- Come osa dire che io gli ho fatto rubare i calzoni! Se li è bevuti, io penso. Io gli posso sputare addosso, a lui e al suo principato! Che non osi dirlo, roba da porci! 

- Ma permettete dunque! Essi si basano sull'articolo - dicevano in un altro gruppo - la moglie deve essere iscritta come nobile. 

- E al diavolo l'articolo! Io parlo sinceramente. Per questo siamo nobili galantuomini. Abbi fiducia. 

- Eccellenza, andiamo a bere, fine champagne! 

Un'altra folla andava dietro a un nobile che gridava forte qualcosa; era uno dei tre ubriachi. 

- Io ho sempre consigliato a Mar'ja Semënovna di dare in affitto, perché così ne ricaverà dell'utile - diceva con voce piacevole un proprietario con i baffi grigi, in divisa di colonnello del vecchio Stato Maggiore generale. Era quello stesso proprietario che Levin aveva incontrato da Svijazskij. Lo riconobbe subito. Anche il proprietario guardò Levin con attenzione e si salutarono. 

- Con molto piacere. E come! Ricordo benissimo. L'anno scorso, da Nikolaj Ivanovic, il maresciallo. 

- E come va la vostra azienda? - chiese Levin. 

- Sempre allo stesso modo, in perdita - rispose il proprietario fermandoglisi accanto con un sorriso rassegnato, ma con un'espressione di calma e di convinzione che la cosa così dovesse andare. 

- E voi come mai siete venuto nel nostro governatorato? - domandò. - Siete venuto per prendere parte al nostro coup d'état? - disse, pronunciando con fermezza, anche se male, le parole francesi. 

- Tutta la Russia vi è convenuta: i ciambellani e, per poco, anche i ministri. - Egli indicò la figura rappresentativa di Stepan Arkad'ic in pantaloni bianchi e divisa da ciambellano, che camminava con un generale. 

- Vi devo confessare che capisco molto male il significato delle elezioni nobiliari - disse Levin. 

Il proprietario lo guardò. 

- Ma che cosa c'è da capire qui? Non c'è nessun significato. Un'istituzione tramontata che continua il suo movimento per forza d'inerzia soltanto. Guardate le divise, anche quelle vi parlano: questa è un'assemblea di giudici di pace, di consiglieri effettivi e via di seguito, ma non di nobili. 

- Allora perché ci venite? - chiese Levin. 

- Per un'abitudine, in primo luogo. Poi, bisogna mantenere le relazioni. È un dovere morale, in un certo modo. E poi, a dire il vero, c'è l'interesse personale. Mio cognato vuol presentarsi al ballottaggio dei membri effettivi; sono persone modeste, bisogna farlo passare. Ecco, questi signori perché ci vengono? - egli disse, indicando quel signore velenoso che aveva parlato al tavolo del governatorato. 

- È la nuova generazione di nobili. 

- Nuova per modo di dire. Ma non è nobiltà. Essi sono possessori di terre e noi siamo proprietari. Loro, come nobili, vanno contro i loro interessi. 

- Allora voi dite che è un'istituzione che ha fatto il suo tempo? 

- Per fare il suo tempo l'ha fatto, tuttavia bisognerebbe considerarla con maggior rispetto. Almeno Snetkov\ldots{} Che siamo buoni o no, siamo pur cresciuti su per mille anni. Sapete, anche se dobbiamo fare un giardinetto davanti alla casa, farne il disegno, e se là cresce un albero di cent'anni\ldots{} Anche se è contorto e vecchio, voi tuttavia per delle aiuole di fiori non taglierete un albero, ma disporrete le aiuole in modo da profittarne. In un anno non lo cresci - egli disse cauto e cambiò immediatamente discorso. - Be', e la vostra azienda come va? 

- Ma, non bene. Un cinque per cento. 

- Sì, ma voi non vi contate. Perché anche voi valete qualcosa. Ecco, vi dirò di me. Finché non ho preso a dirigere l'azienda, in servizio guadagnavo tremila rubli. Ora lavoro più di quanto lavoravo in servizio, e proprio come voi, ne ricavo il cinque per cento, e ancora quello che Dio mi dà. E il mio lavoro è inutile. 

- E allora perché lo fate, se c'è una perdita diretta? 

- Ma ecco, si fa! Che volete? L'abitudine, e poi si sa che così bisogna fare. Vi dirò di più - continuò il proprietario, poggiandosi con i gomiti alla finestra e parlando senza interruzione; - mio figlio non ha nessuna passione per l'azienda. Evidentemente sarà uno studioso. Così che non ci sarà nessuno a continuarla. Eppure si fa sempre, ancora. Ecco, ora ho piantato un giardino. 

- Sì, sì - disse Levin - è proprio vero. Io sento sempre che non c'è un vero tornaconto nella mia azienda, eppure si fa\ldots{} Senti un certo dovere verso la terra. 

- Ma ecco, vi dirò - continuò il proprietario. - È stato da me un vicino che fa il mercante. Abbiamo passeggiato per il podere, per il giardino. ``No, dice, Stepan Vasil'evic, tutto da voi è in ordine, ma il giardino è in abbandono. Da me è in ordine. A criterio mio, codesto bosco di tigli, lo taglierei. Basta farlo quando è in succhio. Perché son mille tigli, da ognuno ne vengon fuori due buone tavole. E oggi giorno le tavole di tiglio sono pregiate, e io ci taglierei tante intelaiature''. 

- E con questi denari lui comprerebbe del bestiame e un po' di terra per quattro soldi e la darebbe in affitto ai contadini - concluse con un sorriso Levin, che evidentemente più di una volta si era imbattuto in calcoli simili. - E si formerà un patrimonio. Invece, voi ed io\ldots{} basta che Dio ci conceda di conservare il nostro e di lasciarlo ai figli. 

- Vi siete ammogliato, ho sentito - disse il proprietario. 

- Sì - rispose Levin con soddisfazione orgogliosa. - Già, è un po' strano - continuò. - Noi viviamo proprio così, senza ricavar nulla, proprio come se fossimo addetti, come le antiche vestali, a custodire una certa fiamma. 

Il proprietario sorrise sotto i baffi bianchi. 

- Ce ne sono anche fra i nostri, ecco, magari il nostro amico Nikolaj Ivanovic o adesso il conte Vronskij che è venuto qui a stabilirsi, i quali vogliono condurre un'azienda agricola; ma finora tutto questo, fuor che distruggere il capitale, non ha portato a niente. 

- Ma perché non facciamo come i mercanti? Perché non tagliamo il giardino per farne tavole? - disse Levin, tornando a un pensiero che l'aveva colpito. 

- Ma ecco, come avete detto voi, per custodire la fiamma. Altrimenti non è lavoro da nobili. E il nostro lavoro da nobili non è qui alle elezioni, ma là, nel nostro angolo. C'è pur sempre il nostro istinto di classe, per quel che si deve e quello che non si deve fare. Ecco, anche i contadini io li osservo bene: appena c'è un bravo contadino, prende in affitto quanta più terra può. Per quanto scadente possa essere la terra, lui la ara sempre. Anche senza tornaconto. Proprio in perdita. 

- Così anche noi - disse Levin. - Sono stato molto, molto contento d'avervi incontrato - aggiunse, avendo visto Svijazskij che si avvicinava. 

- È la prima volta che ci siamo incontrati dopo esserci visti da voi - disse il proprietario - e ci siamo messi a discorrere. 

- Be', avrete detto male degli ordinamenti nuovi? - disse con un sorriso Svijazskij. 

- Non se ne poteva fare a meno. 

- Ci siamo alleviato l'animo. 

\capitolo{XXX}\label{xxx-4} 

Svijazskij prese Levin sotto braccio e andò con lui verso i suoi. 

Ormai non si poteva evitare Vronskij. Egli stava con Stepan Arkad'ic e Sergej Ivanovic e guardava Levin che si avvicinava. 

- Molto lieto. Mi pare d'aver avuto il piacere di incontrarvi\ldots{} dalla principessa Šcerbackaja - disse, dando la mano a Levin. 

- Sì, mi ricordo bene del nostro incontro - disse Levin e, fattosi rosso di fuoco, si voltò subito dall'altra parte e si mise a parlare con suo fratello. 

Sorridendo lievemente, Vronskij riprese a parlare con Svijazskij, non avendo, evidentemente, nessun desiderio di intavolare una nuova conversazione con Levin; ma Levin, parlando con il fratello, si voltava di continuo a guardare Vronskij, pensando cosa potergli dire, per cancellare la propria scontrosità. 

- Di che si tratta adesso? - domandò Levin, volgendosi a guardare Svijazskij e Vronskij. 

- Di Snetkov. Bisogna che rifiuti o accetti - rispose Svijazskij. 

- Ma lui che ha fatto, ha consentito o no? 

- Qui sta la faccenda, che non ha fatto né l'una né l'altra cosa - disse Vronskij. 

- E se rifiuterà, chi entrerà in ballottaggio? - domandò Levin, guardando Vronskij. 

- Chi vorrà - disse Svijazskij. 

- Voi lo farete? - domandò Levin. 

- Io no di certo - disse Svijazskij, confondendosi e gettando uno sguardo spaventato al signore velenoso che stava dritto lì accanto, insieme a Sergej Ivanovic. 

- E allora chi? Nevedovskij? - disse Levin, sentendo di confondersi. 

Ma era ancora peggio, Nevedovskij e Svijazskij erano due candidati. 

- Io, poi, in nessun caso - disse il signore velenoso. 

Era Nevedovskij in persona. Svijazskij gli presentò Levin. 

- Be', anche tu ci sei entrato in pieno! - disse Stepan Arkad'ic, strizzando l'occhio a Vronskij. - È un po' come alle corse. Ci si può scommettere. 

- Già, prende in pieno - disse Vronskij. - E, una volta che si è dentro, si vuol arrivare fino in fondo. La lotta! - disse, accigliandosi e stringendo gli zigomi forti. 

- Che uomo d'affari quel Svijazskij! Tutto è chiaro per lui. 

- Oh, sì - disse distratto Vronskij. 

Seguì un silenzio durante il quale Vronskij, tanto per guardare qualcosa, osservò Levin, le sue gambe, la divisa, poi il viso, e notando gli occhi cupi rivolti su di lui, per dir qualcosa, chiese: 

- Come mai voi che siete un abitatore fisso della campagna, non siete giudice di pace? Non ne portate la divisa. 

- Perché ritengo che il tribunale di pace sia un'istituzione sciocca - rispose Levin, che aveva sempre aspettato l'occasione per mettersi a parlare con Vronskij e dissipare così la scontrosità del primo incontro. 

- Io non lo credo, al contrario - disse Vronskij con calma sorpresa. 

- È un giuoco - l'interruppe Levin. - I giudici di pace non sono necessari. Io in otto anni non ho avuto neanche una causa. E quelle che ho avuto, sono state giudicate alla rovescia. Il giudice di pace è a quaranta verste da me. Io, per una causa che vale due rubli, devo mandare un procuratore che ne costa quindici. 

E raccontò come un contadino avesse rubato la farina al mugnaio, e come, quando il mugnaio glielo aveva detto, il contadino l'avesse citato per calunnia. Tutto ciò era fuor di proposito e sciocco, e Levin, mentre parlava, lo sentiva lui stesso. 

- Oh, è un tal originale! - disse Stepan Arkad'ic col suo sorriso più dolce. - Andiamo però, mi pare che si voti\ldots{} 

E si separarono. 

- Io non capisco - disse Sergej Ivanovic, che aveva notato l'uscita spiacevole del fratello - io non capisco come si possa esser privi fino a tal punto di qualsiasi tatto politico. Ecco quello che noi russi non abbiamo. Il maresciallo del governatorato è nostro avversario, tu sei ami cochon con lui e lo preghi di entrare in ballottaggio. E il conte Vronskij\ldots{} io non me ne farò un amico; m'ha invitato a pranzo, io non andrò da lui, ma è uno dei nostri, perché mai farsene un nemico? Poi tu domandi a Nevedovskij se entrerà in ballottaggio. Questo non si fa. 

- Ah, io non capisco nulla, e poi tutte queste cose sono sciocchezze - rispose torvo Levin. 

- Ecco, tu mi dici che tutte queste cose sono sciocchezze, ma se ti ci metti dentro, allora imbrogli tutto. 

Levin tacque, ed entrarono insieme nella sala grande. 

Il maresciallo del governatorato, malgrado sentisse nell'aria l'inganno preparatogli, e malgrado non tutti l'avessero pregato, decise tuttavia di entrare in ballottaggio. Si fece silenzio nella sala, il segretario annunciò a voce alta che veniva messo ai voti come maresciallo del governatorato il capitano di cavalleria della Guardia Michail Stepanovic Snetkov. 

I marescialli distrettuali passavano, con dei vassoi in cui erano le palle, dalle proprie tavole a quella del governatorato e le elezioni cominciarono. 

- Metti a destra - sussurrò Stepan Arkad'ic a Levin, quando egli, insieme col fratello, si accostò alla tavola dietro al maresciallo. Ma Levin aveva dimenticato il calcolo che gli avevano spiegato, temeva che Stepan Arkad'ic si fosse sbagliato, dicendo: ``a destra''. Snetkov infatti era un nemico. Avvicinandosi alla cassetta egli teneva la palla nella destra, ma pensando d'essersi sbagliato, proprio dinanzi all'urna, passò la palla nella mano sinistra e, in maniera evidente, la mise successivamente a sinistra. Un conoscitore della faccenda, che stava ritto presso l'urna, e che dal solo movimento del gomito capiva dove ciascuno avrebbe messo la palla, fece una smorfia di scontento. Non aveva su che cosa esercitare la propria penetrazione. 

Si fece silenzio e si sentì il conto delle palle. Dopo, una voce isolata, proclamò il numero delle favorevoli e delle sfavorevoli. 

Il maresciallo era eletto con una maggioranza considerevole. Tutti presero a far rumore e si lanciarono impetuosamente verso la porta. Snetkov entrò e la nobiltà lo circondò, congratulandosi. 

- Be', adesso è finita? - domandò Levin a Sergej Ivanovic. 

- Comincia solo adesso - disse, sorridendo, Svijazskij per Sergej Ivanovic. - Il candidato a maresciallo può ricevere più voti. 

Levin l'aveva di nuovo dimenticato. Si ricordò soltanto adesso che lì c'era una certa sottigliezza, ma per lui era noioso ricordarsi in che cosa consistesse. Si sentiva abbattuto, e voleva uscire da quella folla. 

Poiché nessuno faceva attenzione a lui, e lui, a quel che sembrava, non serviva a nessuno, si diresse pian piano nella sala piccola dove si mangiava e provò un gran sollievo nel veder di nuovo i camerieri. Il vecchio gli offrì da mangiare e Levin acconsentì. Dopo aver mangiato una costoletta con fagioli e dopo aver parlato con i camerieri dei signori di una volta, Levin, che non desiderava rientrare nella sala dove provava un'impressione così spiacevole, andò a passeggiare sulla tribuna. 

La tribuna era piena di signore eleganti, che si curvavano sulla balaustrata e cercavano di non perdere neanche una parola di quello che veniva detto giù. Vicino alle signore stavano a sedere avvocati eleganti, professori di ginnasio con gli occhiali e ufficiali. Dovunque si parlava delle elezioni e di come erano state belle le discussioni; in un gruppo Levin sentì una lode rivolta a suo fratello. Una signora diceva a un avvocato: 

- Come sono contenta d'aver sentito Koznyšev! Vale la pena di soffrire un po' la fame. Un incanto! Com'è chiaro e come si sente tutto! Ecco, da voi, in tribunale nessuno parla così. Non c'è che Meidel, e anche quello è lontano dall'essere così eloquente. 

Trovato un posto libero sulla balaustrata, Levin si appoggiò e cominciò a guardare e ad ascoltare. 

Tutti i nobili erano seduti dietro a tramezzi, divisi per distretti. In mezzo alla sala stava in piedi un uomo in divisa e con voce stridula e alta proclamava: 

- È proposto come candidato alla carica di maresciallo del governatorato il capitano in seconda di cavalleria Evgenij Ivanovic Apuchtin! 

Seguì un silenzio di morte e si sentì una debole voce di vecchio: 

- Rinuncia! 

- È proposto il consigliere di corte Pëtr Petrovic Bol' - riprese di nuovo la voce. 

- Rinuncia! - risonò una voce giovane e stridula. 

Di nuovo cominciò la stessa cosa e di nuovo ``rinuncia''. Così durò per quasi un'ora. Levin, appoggiandosi coi gomiti alla balaustrata, guardava e ascoltava. Da principio si sorprendeva e voleva capire che cosa significasse ciò; poi, convintosi di non poterlo capire, cominciò a sentir noia. Poi, ricordando l'agitazione e la cattiveria scorte nelle facce di tutti, gliene venne tristezza: decise di andar via e scese giù. Passando per il vestibolo della tribuna, si imbatté in uno studente, che aveva gli occhi gonfi, e camminava triste avanti e indietro. Sulla scala poi, gli venne incontro una coppia: una signora che correva veloce sui tacchetti, e un agile sostituto procuratore. 

- Ve lo dicevo che non sareste arrivata in ritardo - disse il procuratore, mentre Levin si faceva da parte per cedere il passo alla signora. 

Levin era già sulla scala d'uscita e tirava fuori dalla tasca del panciotto il numero della pelliccia quando il segretario lo afferrò: 

- Favorite, Konstantin Dmitric, si vota. 

Veniva messo in ballottaggio, come candidato, Nevedovskij che aveva rifiutato così recisamente. 

Levin si avvicinò alla porta della sala: era chiusa. Il segretario picchiò: la porta si aprì e incontro a Levin scivolarono via due proprietari, rossi in viso. 

- Non ne posso più - diceva uno dei proprietari tutto rosso in viso. 

Dietro il proprietario spuntò il viso del maresciallo del governatorato. Il suo viso era spaventoso per l'abbattimento e il terrore. 

- Io t'ho detto: non far uscire! - gridò al custode. 

- Ho fatto entrare, eccellenza! 

- Signore! - e, dopo aver sospirato penosamente, il maresciallo del governatorato, sgattaiolando stanco nei suoi pantaloni bianchi, con il capo chino, andò in mezzo alla sala, verso la tavola grande. 

Su Nevedovskij avevano portato i voti, come del resto era stato calcolato, ed egli era maresciallo del governatorato. Molti erano allegri, molti contenti, felici, entusiasti; molti scontenti e infelici. Il maresciallo del governatorato era in una disperazione che non riusciva a nascondere. Quando Nevedovskij uscì dalla sala, la folla lo circondò, e lo seguì entusiasticamente, proprio così come il primo giorno aveva seguito il governatore che aveva aperto le elezioni, e così come aveva seguito Snetkov quando era stato eletto. 

\capitolo{XXXI}\label{xxxi-4} 

Il maresciallo del governatorato appena eletto e molti del partito trionfante dei giovani erano, in quel giorno, a pranzo da Vronskij. 

Vronskij era venuto alle elezioni perché si annoiava in campagna e perché aveva bisogno di affermare i propri diritti di libertà di fronte ad Anna, e per disobbligarsi con Svijazskij, appoggiandolo nelle elezioni, di tutte le beghe che s'era prese per Vronskij alle elezioni del consiglio; ma più di tutto per adempiere rigorosamente tutti gli obblighi di quella posizione di nobile e di proprietario di terre che s'era scelta. Ma non s'aspettava in alcun modo che quella faccenda delle elezioni lo potesse interessare tanto e che potesse riuscirgli così bene. Era una persona completamente nuova nell'ambiente nobiliare, ma evidentemente aveva avuto successo e non si sbagliava, pensando d'aver già acquistato influenza fra i nobili. A questa influenza aveva contribuito la sua ricchezza e notorietà, il bellissimo appartamento in città cedutogli dal vecchio amico Širkov, che si occupava di affari finanziari e che aveva organizzato una fiorente banca a Kašin, l'ottimo cuoco di Vronskij, portato dalla campagna, l'amicizia col governatore che era stato suo compagno, ed era ancora un compagno protetto da Vronskij, e, più di tutto, i suoi modi semplici, eguali verso tutti, che avevano costretto molto presto i nobili a mutar giudizio sulla sua presunta superbia. Egli stesso sentiva che, tranne quello stravagante signore che aveva sposato Kitty Šcerbackaja, che, à propos des bottes, gli aveva detto con furiosa irritazione un mucchio di sciocchezze che non c'entravano per niente, ogni nobile di cui aveva fatto la conoscenza era divenuto suo partigiano. Vedeva chiaramente, e gli altri lo riconoscevano, che al successo di Nevedovskij aveva cooperato molto lui. E adesso, alla propria tavola, nel festeggiare l'elezione di Nevedovskij, egli provava una piacevole sensazione di trionfo per il proprio eletto. Le stesse elezioni lo avevano talmente preso che, se fosse riuscito a regolare la sua posizione di marito, pensava, per il futuro triennio, di entrare lui stesso in ballottaggio, quasi come, dopo aver vinto un premio per mezzo di un fantino, gli fosse venuta la voglia di correre lui stesso. 

Adesso, invece, si festeggiava la vittoria del fantino. Vronskij sedeva a capotavola, alla sua destra sedeva il governatore, generale di corte. Per tutti gli altri questi era il padrone del governatorato, colui che aveva aperto solennemente le elezioni, pronunciato un discorso e suscitato in molti, come aveva visto Vronskij, stima e servilità; per Vronskij, invece, era Maslov Kat'ka, così era soprannominato al corpo dei paggi, che dinanzi a lui si confondeva e che egli cercava di mettre à son aise. A sinistra sedeva Nevedovskij col suo viso giovane, imperturbabile e velenoso. Con lui Vronskij era semplice e rispettoso. 

Svijazskij sopportava allegramente il proprio smacco. Non era neppure uno smacco per lui, come egli stesso disse, rivolgendosi con la coppa a Nevedovskij: non si poteva trovare un rappresentante migliore di quella nuova tendenza che la nobiltà doveva seguire. E perciò tutti gli onesti, com'egli disse, erano dalla parte del successo di oggi e lo celebravano solennemente. 

Stepan Arkad'ic pure era felice e perché aveva passato allegramente il tempo e perché tutti erano di buon umore. Durante l'ottimo pranzo si rievocarono gli episodi delle elezioni. Svijazskij riferì comicamente il lacrimoso discorso del maresciallo e notò, rivolto a Nevedovskij, che sua eccellenza avrebbe dovuto scegliere un controllo diverso e più complesso delle somme che non le lacrime. Un altro nobile scherzoso raccontò come fossero stati fatti venire i camerieri con le calze lunghe per il ballo del maresciallo del governatorato e come adesso si sarebbe dovuti rimandarli indietro, se il nuovo maresciallo del governatorato non avesse dato il ballo con i camerieri in calze lunghe. 

Durante il pranzo, quando le persone si rivolgevano a Nevedovskij, dicevano continuamente: ``il nostro maresciallo del governatorato'' e ``vostra eccellenza''. 

Questo era pronunciato col medesimo piacere con cui si chiama una giovane donna ``madame'' e con il cognome del marito. Nevedovskij fingeva non solo d'essere indifferente, ma anche di spregiare quel titolo; ma era evidente che ne era felice e che si conteneva per non mostrare un entusiasmo poco adatto a quell'ambiente nuovo, liberale, in cui si trovavano tutti. 

Durante il pranzo, furono spediti alcuni telegrammi a persone che si interessavano dell'andamento delle elezioni. E Stepan Arkad'ic, che era molto allegro, mandò a Dar'ja Aleksandrovna un telegramma così concepito: ``Nevedovskij eletto con venti voti. Sono felice. Comunica la notizia''. Lo dettò ad alta voce, notando: ``bisogna farli contenti''. Dar'ja Aleksandrovna, invece, ricevuto il telegramma, sospirò soltanto per il rublo del telegramma e capì che la cosa era avvenuta alla fine d'un pranzo. Ella sapeva che Stiva aveva la debolezza, alla fine dei pranzi, di faire jouer le télégraphe. 

Tutto, compreso l'ottimo pranzo e i vini, acquistati non certo da rivenditori di vino russo, ma direttamente da produttori esteri, fu molto nobile, semplice, allegro. Quel gruppetto di venti persone era stato scelto da Svijazskij fra uomini pubblici delle stesse idee liberali, giovani e nello stesso tempo intelligenti e selezionati. Si fecero dei brindisi, anch'essi semiseri, alla salute del nuovo maresciallo del governatorato, e del governatore, e del direttore della banca e del ``nostro gentile padron di casa''. 

Vronskij era soddisfatto. Non si aspettava per nulla un tono così simpatico in provincia. 

Alla fine del pranzo ci fu ancor più allegria. Il governatore pregò Vronskij di andare a un concerto di beneficenza a favore dei ``fratelli'', che aveva organizzato sua moglie, la quale desiderava conoscerlo. 

- Ci sarà un ballo, e vedrai la nostra bellezza. Davvero, è una cosa straordinaria. 

- Not in my line - rispose Vronskij, al quale piaceva questa espressione, ma sorrise e promise di andare. 

Ancora prima che si alzassero da tavola, quando tutti avevano cominciato a fumare, il cameriere di Vronskij gli si avvicinò con una lettera su di un vassoio. 

- Da Vozdvizenskoe, con un corriere espresso - disse con aria significativa. 

- È sorprendente come assomigli al sostituto procuratore Sventickij - disse in francese, del cameriere, uno degli ospiti, mentre Vronskij leggeva la lettera, accigliato. 

La lettera era di Anna. Ancora prima di leggerla, egli ne sapeva bene il contenuto. Supponendo che le elezioni sarebbero finite entro cinque giorni, aveva promesso di tornare venerdì. Si era al sabato, ed egli sapeva che il contenuto della lettera consisteva nei rimproveri per il ritardo. La lettera da lui spedita il giorno innanzi probabilmente non era ancora giunta. 

Il contenuto era quello che si aspettava, ma la forma era inaspettata e particolarmente spiacevole per lui. ``Annie è molto malata. Il dottore dice che può trattarsi di infiammazione. Io sola perdo la testa. La principessa Varvara non è un aiuto, ma un intralcio. T'ho aspettato ieri l'altro, ieri, e adesso mando a domandare dove sei e che fai. Volevo venire io stessa, ma ho cambiato idea, pensando che ti sarebbe spiaciuto. Dammi una risposta qualsiasi, purché sappia che cosa fare''. 

La bimba ammalata e lei stessa che voleva venire. La figlia ammalata e questo tono ostile. 

L'innocente allegria delle elezioni e quel cupo, pesante amore a cui egli doveva tornare colpirono Vronskij col loro contrasto. Ma bisognava andare, e, col primo treno della notte, egli partì per tornare a casa. 

\capitolo{XXXII}\label{xxxii-4} 

Prima della partenza di Vronskij per le elezioni, considerando che quelle scenate che si ripetevano fra di loro a ogni partenza potevano soltanto raffreddarlo e non legarlo, Anna si era decisa a fare su di sé tutti gli sforzi possibili per sopportare con calma la separazione da lui. Ma quello sguardo freddo, severo, con cui egli l'aveva guardata quando era venuto ad annunciare la sua partenza, l'aveva offesa, ed egli non era ancora partito, che la calma di lei era già distrutta. 

Ripensando poi in solitudine a quello sguardo, che aveva espresso il diritto alla libertà, ella venne, come sempre a un'unica conclusione: alla coscienza della propria umiliazione. ``Lui ha tutti i diritti, io non ne ho nessuno. Eppure, sapendolo, non doveva far ciò. Ma cosa mai ha fatto? Mi ha guardato con un'espressione fredda, severa. S'intende, è una cosa indefinita, impalpabile, ma prima non c'era, e questo sguardo significa molte cose - ella pensava. - Questo sguardo dimostra che comincia il raffreddamento''. 

E, pur convinta che cominciava il raffreddamento, non poteva far nulla, non poteva cambiar in nulla i suoi rapporti con lui. E sempre, come prima, poteva trattenerlo solo con l'amore e il proprio fascino. E sempre allo stesso modo di prima, con le occupazioni di giorno e con la morfina di notte riusciva a soffocare i pensieri tremendi su quello che sarebbe stato s'egli si fosse disincantato di lei. In verità, c'era ancora un mezzo: non trattenerlo, poiché ella non voleva altro che l'amore di lui, ma legarglisi, essere in una situazione tale ch'egli non la lasciasse. Questo mezzo era il divorzio e il matrimonio. Ed ella cominciò a desiderare questo e si decise a consentire, la prima volta che lui o Stiva avessero preso a parlargliene. 

In tali pensieri ella passò senza di lui i cinque giorni, quegli stessi in cui egli doveva essere assente. 

Le passeggiate, le conversazioni con la principessa Varvara, le visite all'ospedale e soprattutto la lettura, la lettura di un libro dietro l'altro, occuparono il suo tempo. Ma il sesto giorno, quando il cocchiere ritornò senza di lui, ella sentì che non aveva più la forza di soffocare, in nessun modo, il pensiero di lui e di quel ch'egli facesse là. In quello stesso giorno sua figlia s'ammalò. Anna cominciò a curarla, ma anche questo non la distrasse, tanto più che la malattia non era grave. Per quanto si sforzasse, ella non amava quella bambina, e fingere di volerle bene non poteva. Verso la sera di quel giorno, rimasta sola, sentì un tale terrore per lui, che fu sul punto di andare in città; ma dopo aver esitato un po' scrisse quella lettera contraddittoria che Vronskij aveva ricevuta, e, senza rileggerla, la mandò per espresso. La mattina dopo ricevette la lettera di lui e si pentì della propria. Aspettava con terrore il ripetersi di quello sguardo severo ch'egli le aveva lanciato nel partire, specie quando sarebbe venuto a sapere che la bambina non era stata gravemente malata. Tuttavia era contenta d'avergli scritto. Adesso Anna riconosceva già ch'egli sentiva il peso di lei, che lasciava con rammarico la propria libertà per tornare da lei, e, malgrado questo, era contenta che egli sarebbe tornato. Che ne sentisse pure il peso, ma fosse là con lei. In ogni modo ch'ella lo vedesse, sapesse ogni suo movimento. 

Era seduta in salotto, sotto la lampada, con un nuovo libro del Taine e leggeva, prestando orecchio al suono del vento fuori e aspettando da un momento all'altro l'arrivo della carrozza. Parecchie volte le era parso di sentire il suono delle ruote, ma si era sbagliata; finalmente si udì non soltanto il suono delle ruote, ma anche il vociare del cocchiere e un suono sordo nell'ingresso coperto. Perfino la principessa Varvara, che faceva un solitario, lo confermò, e Anna si alzò, rossa di fiamma, ma invece di andare giù, come aveva fatto due volte prima, si fermò. A un tratto provò vergogna del proprio inganno, ma ancor più terrore di come egli l'avrebbe accolta. Il sentimento d'offesa era già passato; ella aveva soltanto paura dell'espressione del suo scontento. Si ricordò che la figlia, già da due giorni, stava perfettamente bene. Le venne perfino rabbia contro di lei che si era rimessa proprio quando aveva mandato la lettera. Poi si ricordò che lui era là. Ne sentì la voce. E, dimentica di tutto, gli corse incontro. 

- Be', come sta Annie? - disse lui, con ansia, di sotto, guardando Anna che gli era corsa incontro. 

Era seduto su di una sedia, e un cameriere gli tirava via uno stivale caldo. 

- Non c'è male, sta meglio. 

- E tu? - egli chiese scotendosi la roba addosso. 

Ella con tutte e due le mani prese un suo braccio e se lo passò intorno alla vita, senza levargli gli occhi di dosso. 

- Be', sono molto contento - egli disse, guardando freddo la pettinatura, il vestito di lei, ch'egli sapeva preparati per lui. 

Tutto ciò gli piaceva, ma quante volte gli era piaciuto! E quell'espressione severamente impietrita ch'ella paventava, si fissò sul viso di lui. 

- Sono molto contento. E tu stai bene? - disse lui, dopo aver asciugato col fazzoletto la barba bagnata e baciandole la mano. 

``È sempre lo stesso - ella pensava - basta ch'egli sia qui e quando è qui, non può, non può, non oserà non amarmi''. 

La serata passò serena e allegra in presenza della principessa Varvara che si lamentava con lui che Anna in sua assenza prendeva la morfina. 

- E che fare? Non potevo dormire\ldots{} I pensieri me lo impedivano. Quando c'è lui non la prendo mai. Quasi mai. 

Egli raccontò delle elezioni, e Anna seppe eccitarlo con domande a parlare proprio di quello che lo rallegrava: del suo successo. Ella gli raccontò tutto quello che gli interessava della casa. E tutte le informazioni di lei erano allegre. 

Ma la sera tardi, quando rimasero soli, Anna, vedendo che era di nuovo pienamente padrona di lui, volle cancellare quella penosa impressione dello sguardo, a causa della lettera. Disse: 

- Di' la verità, ti sei arrabbiato quando hai ricevuto la lettera, e non m'hai creduto? 

Appena detto ciò, capì che, per quanto egli fosse pieno d'amore verso di lei, quella non gliel'aveva perdonata. 

- Sì - disse lui. - La lettera era così strana: un momento Annie era malata, un momento tu stessa volevi venire. 

- Tutto questo era vero. 

- Ma io non ne dubito neppure. 

- No, ne dubiti. Sei scontento, lo vedo. 

- Neppure un attimo. Soltanto sono scontento, è vero, che tu non voglia ammettere che ci siano degli obblighi\ldots{} 

- Gli obblighi di andare al concerto\ldots{} 

- Ma non ne parliamo - egli disse. 

- E perché non parlarne? - disse lei. 

- Voglio dire soltanto che possono capitare degli affari inderogabili. Ecco, adesso, dovrò andare a Mosca per l'affare della casa\ldots{} Ah, Anna, perché sei così suscettibile? Ma non sai ch'io non posso vivere senza di te? 

- Ma se è così - disse Anna, cambiata a un tratto nella voce - allora tu senti il peso di questa vita\ldots{} Sì, verrai per un giorno e partirai, come fanno\ldots{} 

- Anna, questo è crudele. Io sono pronto a dar tutta la vita\ldots{} 

Ma ella non l'ascoltava. 

- Se tu andrai a Mosca, ci verrò anch'io. Non rimarrò qui. O ci dobbiamo separare, o vivere insieme. 

- Lo sai che è l'unico mio desiderio. Ma per questo\ldots{} 

- Ci vuole il divorzio? Gli scriverò. Vedo che non posso vivere così\ldots{} Ma verrò con te a Mosca. 

- Come se minacciassi. Ma io non desidero altro che non separarmi mai da te - disse Vronskij, sorridendo. 

Ma non solo uno sguardo freddo, anche uno sguardo cattivo di uomo perseguitato e accanito balenò nei suoi occhi, mentr'egli diceva queste tenere parole. 

Ella aveva visto questo sguardo e ne aveva intuito con precisione il senso. 

``Se è così è una disgrazia!'' diceva lo sguardo di lui. Fu l'impressione di un attimo ma ella non lo dimenticò più. 

Anna scrisse una lettera al marito, chiedendogli il divorzio, e alla fine di novembre, separatasi dalla principessa Varvara che aveva bisogno di recarsi a Pietroburgo, andò a stabilirsi a Mosca con Vronskij. Aspettando da un giorno all'altro la risposta di Aleksej Aleksandrovic e poi il divorzio, essi si stabilirono, come coniugi, insieme. 

\parte{PARTE SETTIMA}\label{parte-settima} 

\capitolo{I}\label{i-6} 

Era il terzo mese che i Levin erano a Mosca. Era già passato da tempo il termine, in cui, secondo i calcoli più sicuri delle persone che conoscono tali cose, Kitty doveva partorire; e lei era tuttora in stato interessante, e da nulla si vedeva che il tempo si era fatto più vicino adesso che non due mesi prima. E il dottore e la levatrice e Dolly e la madre, e in particolare Levin, che non poteva pensare senza agitarsi a quello che si avvicinava, cominciavano a essere impazienti e inquieti; soltanto Kitty si sentiva pienamente tranquilla e serena. 

Adesso, aveva chiara la coscienza ch'era sorto in lei un sentimento nuovo per la creatura che sarebbe nata, già in parte presente per lei, e prestava ascolto con gioia a questo sentimento. Adesso non era più semplicemente una parte di lei, ma viveva a volte anche di vita sua propria, indipendente da lei. Spesso le accadeva di sentirsi male per questo, ma nello stesso tempo aveva voglia di ridere per una strana nuova gioia. 

Tutti quelli che ella amava erano con lei, e tutti erano così buoni, la curavano tanto, le era offerto di ogni cosa solo il lato piacevole, così che, se non avesse saputo e sentito che ciò sarebbe finito presto, non avrebbe neppur desiderato una vita migliore e più piacevole. Soltanto una cosa le sciupava l'incanto di questa vita, ed era il fatto che suo marito non era così com'ella lo amava e come era stato in campagna. 

Le piaceva il tono calmo, carezzevole, e ospitale ch'egli aveva avuto in campagna. In città, invece, sembrava continuamente inquieto e guardingo, come se temesse che qualcuno non offendesse lui, ma soprattutto lei. Là in campagna, egli evidentemente sapendosi al suo posto, non si affannava e non era mai inoperoso. Qui, in città, si affannava continuamente, come se non volesse lasciarsi sfuggire nulla, eppure non aveva nulla da fare. E a lei faceva pena. Agli altri, lo sapeva, non appariva pietoso; al contrario, quando Kitty lo guardava in società, come si guarda a volte la persona che si ama, cercando di giudicarla da estranei, per definire l'impressione che produce sugli altri, ella vedeva, quasi con apprensione per la propria gelosia, che egli non solo non era pietoso, ma era molto attraente con la sua onestà, con la sua cortesia un po' all'antica, timida con le signore, con la sua figura forte e il volto particolarmente espressivo. Ma ella l'osservava non dal di fuori, ma nell'intimo; vedeva che non era spontaneo; non altrimenti poteva definire lo stato di lui. A volte lo accusava fra sé e sé di non saper vivere in città; a volte, invece, riconosceva che realmente gli era difficile organizzarsi una vita in modo da esserne soddisfatto. 

E che cosa doveva fare, infatti? Giocare a carte non gli piaceva. Al club non andava. Frequentare dei buontemponi sul genere di Oblonskij, ella sapeva ormai che cosa significasse\ldots{} significava bere, e, dopo aver bevuto, andare chi sa dove. Non poteva pensare senza orrore dove andavano gli uomini in simili casi. Frequentare il gran mondo? Ma ella sapeva che per questo bisognava trovar piacere nell'avvicinare donne giovani, e lei questo non lo poteva desiderare. Rimanere a casa con lei, con la madre e le sorelle? Ma per quanto piacevoli e allegri fossero per lei gli stessi discorsi, le ``Aline-Nadine'' come il vecchio principe chiamava questi discorsi fra le sorelle, sapeva bene che questo lo doveva annoiare. Allora che cosa gli rimaneva da fare? Continuare a scrivere il suo libro? Aveva anche tentato di farlo, e da principio andava in biblioteca a consultare lavori monografici e d'informazione per il suo libro; ma, come egli diceva, quanto più non faceva nulla, tanto meno tempo gli rimaneva. Inoltre egli stesso si lamentava che aveva finito col discorrere troppo del proprio libro e che perciò tutte le idee gli si erano confuse e avevano perso interesse. 

L'unico vantaggio di questa vita cittadina era che, in città, litigi fra di loro non ne avvenivano. Forse perché le condizioni di vita cittadina erano diverse, o perché tutti e due s'erano fatti più accorti e più ragionevoli a questo riguardo, certo è che a Mosca non ebbero le questioni di gelosia che tanto avevano temuto, andando in città. 

A proposito di questo, accadde un avvenimento molto importante per tutti e due, l'incontro cioè di Kitty con Vronskij. 

La vecchia principessa Mar'ja Borisovna, madrina di Kitty, che sempre le aveva voluto molto bene, desiderò assolutamente di vederla. Kitty, che per il suo stato non andava in nessun posto, andò col padre dalla veneranda signora, e da lei incontrò Vronskij. 

In quell'incontro, Kitty poté soltanto rimproverarsi che, per un attimo, quando riconobbe nell'abito borghese i tratti a lei un tempo così noti, le venne meno il respiro, il sangue le affluì al cuore, e un colorito vivace, lo sentì, le apparve sul viso. Ma questo durò soltanto alcuni attimi. Il padre, che apposta aveva cominciato a parlare ad alta voce con Vronskij, non aveva ancora finito la sua conversazione, che lei era già del tutto pronta a guardar Vronskij, a parlare con lui, se ce n'era bisogno, proprio così come parlava con la principessa Mar'ja Borisovna, e, soprattutto, in modo che ogni cosa, fino all'ultima intonazione e all'ultimo sorriso, fosse approvata dal marito, la cui presenza invisibile era come sentita da lei su di sé, in quel momento. 

Disse alcune parole con lui, sorrise perfino tranquilla all'arguzia sulle elezioni, che egli chiamava ``il nostro parlamento''. (Bisognava sorridere per mostrare che se n'era capito lo spirito). Ma subito si voltò verso Mar'ja Borisovna e non lo guardò neanche più una volta, finché egli non si alzò per salutare; allora lo guardò, ma evidentemente solo perché è scortese non guardare una persona quando saluta. 

Fu riconoscente al padre che non le disse nulla sull'incontro con Vronskij; ma vedeva dalla particolare tenerezza di lui dopo la visita, durante la solita passeggiata, ch'egli era contento di lei. Lei stessa era contenta. Non si aspettava in nessun modo di trovare in sé la forza di trattenere chi sa dove, nel profondo dell'anima, tutti i ricordi del sentimento provato un tempo per Vronskij e non solo di apparire, ma anche di essere del tutto indifferente e calma verso di lui. 

Levin arrossì molto più di lei, quand'ella gli disse d'aver incontrato Vronskij dalla principessa Mar'ja Borisovna. Le fu molto difficile dirglielo, ma ancor più difficile continuare a parlare dei particolari dell'incontro, poiché egli non la interrogava, ma la guardava soltanto accigliato. 

- Mi dispiace molto, che tu non ci sia stato - ella disse. - Non che tu non ci fossi nella stanza\ldots{} non sarei stata così naturale in tua presenza. Io adesso arrossisco molto di più, molto, molto di più - diceva lei, arrossendo fino alle lacrime. - Ma che tu non potessi vedere attraverso una fessura. 

Gli occhi sinceri dissero a Levin ch'ella era contenta di sé, e lui, malgrado ella arrossisse, si calmò immediatamente e cominciò a farle delle domande, il che era proprio quello che lei voleva. Quando egli seppe tutto, fino al particolare che solo al primo momento non aveva potuto non arrossire, ma che poi s'era comportata in modo semplice e spontaneo come col primo venuto, Levin si rallegrò completamente e disse che ne era molto contento e che adesso non avrebbe agito in modo così sciocco come alle elezioni, ma che avrebbe cercato, al primo incontro con Vronskij, di essere cordiale, per quanto possibile. 

- È un tale tormento pensare che c'è un essere quasi nemico col quale è penoso incontrarsi - disse Levin. - Sono molto, molto contento. 

\capitolo{II}\label{ii-6} 

- Su via, ti prego, passa dai Bol' - disse Kitty al marito, quando egli, alle undici, prima di uscire di casa, andò da lei. - So che pranzi al club; papà ti ha iscritto. E stamattina che fai? 

- Vado soltanto da Katavasov - rispose Levin. 

- Come mai così presto? 

- Mi ha promesso di farmi conoscere Metrov. Desideravo parlare con lui del mio lavoro, è un noto studioso di Pietroburgo - disse Levin. 

- Sì, è suo quell'articolo che hai lodato tanto? Be', e poi? - disse Kitty. 

- Può darsi che vada ancora in tribunale per l'affare di mia sorella. 

- E al concerto? - ella domandò. 

- Ma perché devo andare solo? 

- No, vacci: là dànno cose nuove\ldots{} Ti interessava tanto. Io ci andrei assolutamente. 

- Be', in ogni modo passerò a casa prima di pranzo - egli disse, guardando l'orologio. 

- Mettiti la finanziera per poter andare direttamente dalla contessa Bol'. 

- È proprio necessario? 

- Ah, assolutamente! Lui è stato da noi. Via, che ti costa? Arrivi, ti siedi, parli per cinque minuti, ti alzi e te ne vai. 

- Non ci crederai, ma sono talmente disabituato a questo, che quasi me ne vergogno. Cos'è questo? Viene una persona estranea, si siede, rimane lì a sedere senza far niente, dà loro noia, si disorienta e se ne va. 

Kitty rise. 

- Ma facevi pur delle visite, da scapolo! - disse. 

- Le facevo, ma mi vergognavo sempre, e ora sono così disabituato che, lo sa Iddio, meglio non pranzare per due giorni anzi che far questa visita. Mi vergogno tanto! Mi pare sempre debbano offendersi, che debbano dire: perché sei venuto senza una ragione seria? 

- No, non si offenderanno. Per questo ne rispondo io - disse Kitty, mentre guardava ridendo il viso di lui. Lo prese per una mano. - Su, addio\ldots{} Va', per favore. 

Stava già per uscire, dopo aver baciato la mano alla moglie, quando ella lo fermò. 

- Kostja, lo sai che mi son rimasti soltanto cinquanta rubli? 

- Va bene, passerò a prenderne in banca. Quanto? - egli disse con un'espressione di scontento a lei nota. 

- No, aspetta. - Ella lo trattenne per mano. - Parliamo un po', ciò mi preoccupa. Io, mi pare, non pago nulla più caro di quel che dovrei, e i denari se ne vanno a fiumi. C'è qualcosa che non va. 

- Per nulla - disse lui, tossendo e guardandola di sotto in su. 

Questo tossicchiare ella lo conosceva. Era un segno di forte scontento, non verso di lei, ma verso se stesso. Realmente egli era contrariato, non che se ne andassero molti denari, ma che gli si ricordasse quello che lui, sapendo che v'era qualcosa che non andava, voleva dimenticare. 

- Ho ordinato a Sokolov di vendere il frumento e di prendere il denaro in anticipo per il mulino. I denari ci saranno in ogni modo. 

- No, ma io ho paura che in generale, sì, molto\ldots{} 

- Per nulla, per nulla - egli ripeteva. - Be', addio, cara. 

- No, davvero, a volte, rimpiango di avere ascoltato la mamma. Come sarebbe stato bello in campagna! Invece vi ho tormentato tutti e sperperiamo denaro\ldots{} 

- Per nulla, per nulla. Non è avvenuto neppure una volta, da che sono sposato, che io abbia detto che sarebbe stato meglio altrimenti di quello che non sia\ldots{} 

- Davvero? - disse lei, guardandolo negli occhi. 

Egli l'aveva detto senza pensare, solo per consolarla. Ma quando, guardandola, vide che quei sinceri, cari occhi lo fissavano interrogativamente, ripeté la stessa cosa con tutta l'anima. ``Decisamente la trascuro'' pensò. E ricordò quello che li attendeva tra poco. 

- Sarà presto? Come ti senti? - sussurrò lui, prendendola per tutt'e due le mani. 

- Ci ho pensato tante volte, che ora non ci penso più e non so nulla. 

- E non hai paura? 

Ella sorrise con disprezzo. 

- Neppure un briciolo - ella disse. 

- Allora se succede qualcosa, sono da Katavasov. 

- No, non succederà niente, non ci pensare. Io andrò a passeggio con papà sul viale. Passeremo da Dolly. Prima di pranzo ti aspetto. Ah, già, lo sai che la situazione di Dolly diventa proprio impossibile? È piena di debiti, denari non ne ha. Ieri abbiamo parlato, mamma ed io, con Arsenij - così ella chiamava il marito della sorella L'vova - e abbiamo deciso di lanciare te e lui contro Stiva. È proprio una cosa impossibile. Con papà non se ne può parlare\ldots{} Ma se tu e lui\ldots{} 

- Ma che possiamo fare mai? - disse Levin. 

- Comunque, va' da Arsenij, parla un po' con lui; ti dirà quello che abbiamo deciso. 

- Ma con Arsenij sono fin d'ora d'accordo su tutto. Allora passerò da casa sua. A proposito, se devo andare al concerto, ci andrò con Natalie. Be', addio. 

Sulle scale il vecchio servo Kuz'ma, ch'egli aveva ancora da quando era scapolo, e che sorvegliava la casa di città, fermò Levin. 

- Krasavcik - era un cavallo, timoniere di sinistra, portato dalla campagna - è stato ferrato di nuovo, ma zoppica sempre - disse. - Cosa comandate? 

I primi tempi a Mosca, Levin si occupava dei cavalli portati dalla campagna. Desiderava organizzare questo servizio nel modo migliore e più conveniente; ma era successo che i cavalli propri venivano a costare più cari di quelli di fitto, e le vetture da nolo venivano prese ugualmente. 

- Ordina che si mandi a chiamare il veterinario, può darsi che sia un'ammaccatura. 

- E per Katerina Aleksandrovna? - domandò Kuz'ma. 

Ormai Levin non si stupiva più, come nei primi tempi della sua vita a Mosca, che per andare dalla Vozdvizen'ka al Sivcev Vrazëk bisognasse attaccare una pariglia di cavalli forti a una carrozza pesante, condurre questa carrozza per un quarto di versta su per il fango nevoso e star lì fermi quattro ore, pagando cinque rubli per questo. Questo ormai gli sembrava naturale. 

- Ordina al vetturino di portare una pariglia per la nostra carrozza - disse. 

- Sissignore. 

E risolta così, con semplicità e disinvoltura, grazie alle condizioni di vita cittadina, una difficoltà che in campagna avrebbe richiesto tanto lavoro personale e tanta cura, Levin uscì sulla scalinata e, chiamata una vettura da nolo, vi salì e si diresse alla Nikitskaja. Per strada non pensava già più ai denari, ma rifletteva su come avrebbe fatto conoscenza con lo studioso di Pietroburgo che si occupava di sociologia, e come avrebbe parlato con lui del proprio libro. 

Soltanto nei primissimi tempi a Mosca quelle spese inusitate per un abitatore della campagna, improduttive, ma inevitabili, che da ogni parte lo circuivano, stupirono Levin. Ma adesso vi si era già abituato. A questo riguardo gli accadde quello che, si dice, accade agli ubriaconi: il primo bicchierino va giù come un palo, il secondo come un falco e, dopo il terzo, gli altri volano via come uccelletti di nido. Quando Levin aveva cambiato il primo biglietto da cento rubli per comprare le livree al cameriere e al portiere, involontariamente aveva considerato che queste livree non erano necessarie a nessuno, ma erano inevitabilmente indispensabili, a giudicare dal modo con cui si erano sorprese la principessa e Kitty a un suo accenno che, senza livree, si poteva vivere lo stesso, che queste livree sarebbero costate come due operai estivi, cioè quasi trecento giorni lavorativi, dalla settimana di Pasqua fino all'ultimo giorno di carnevale, e ogni giorno di lavoro pesante, dalla mattina presto fino alla sera tardi; e questo biglietto da cento rubli gli era andato giù come un palo. Ma il seguente, cambiato per comprar cibarie che erano costate ventotto rubli, per un pranzo ai parenti, pur destando in Levin il pensiero che ventotto rubli erano nove stai di avena, che si falciavano, accovonavano, battevano, vagliavano, stacciavano e versavano, sudando e sbuffando, questo secondo biglietto, tuttavia, se ne era andato con maggiore facilità. E adesso i biglietti cambiati non suscitavano più da lungo tempo queste considerazioni e volavano via come uccelli di nido. La considerazione se al lavoro impiegato nel procurarsi il denaro avesse corrisposto il piacere che procurava ciò che veniva comprato con esso, era sfumata già da lungo tempo. Il calcolo economico che c'era un certo prezzo al di sotto del quale non si poteva vendere una certa qualità di grano, anch'esso era stato dimenticato. La segala, il cui prezzo egli aveva tenuto su per tanto tempo, era stata venduta a cinquanta copeche lo staio in meno di quello che si dava per essa un mese prima. Perfino il calcolo che con simili spese non sarebbe stato possibile vivere tutto l'anno senza debiti, anche questo calcolo non aveva più nessuna importanza. Si voleva soltanto una cosa: aver denari in banca senza domandare donde venissero, in modo da saper sempre come comprar carne l'indomani. E questo calcolo finora era stato mantenuto; egli aveva sempre avuto denari in banca, ma adesso erano stati spesi, ed egli non sapeva bene da qual parte prenderli. E questo, proprio quando Kitty gli aveva ricordato la questione dei denari, l'aveva sconvolto; ma non aveva avuto il tempo di pensarci su. Mentre andava in vettura, ripensava a Katavasov e all'imminente incontro con Metrov. 

\capitolo{III}\label{iii-6} 

Levin, in quella sua permanenza, era di nuovo in rapporti intimi con il suo compagno di università, il professore Katavasov, che non aveva più visto dal tempo del suo matrimonio. Katavasov gli piaceva per la chiarezza e la semplicità della sua concezione del mondo. Levin pensava che la chiarezza della concezione del mondo di Katavasov derivasse dalla povertà della sua natura; Katavasov invece pensava che la mancanza di coerenza nel pensiero di Levin derivasse dalla mancanza di disciplina del suo ingegno; ma la chiarezza di Katavasov piaceva a Levin e l'abbondanza di pensieri di Levin piaceva a Katavasov, ed essi amavano incontrarsi e discutere. 

Levin aveva letto a Katavasov alcuni punti della propria opera, e questi gli erano piaciuti. Il giorno avanti, incontrando Levin a una conferenza, Katavasov gli aveva detto che il famoso Metrov, il cui articolo era tanto piaciuto a Levin, si trovava a Mosca e si era molto interessato, a dire di Katavasov, al lavoro di Levin; l'indomani alle undici sarebbe andato da lui e sarebbe stato felice di conoscerlo. 

- State decisamente migliorando, amico mio, e ciò mi fa piacere - disse Katavasov, accogliendo Levin in un piccolo salotto. - Sento una scampanellata e penso: è impossibile che arrivi in orario\ldots{} Be', come sono i montenegrini? Di razza sono guerrieri. 

- E allora? 

Katavasov in poche parole gli riferì l'ultima notizia e, entrando nello studio, presentò Levin a un uomo non alto, tarchiato, di aspetto molto simpatico. Era Metrov. La conversazione si fermò per un poco sulla politica e su come a Pietroburgo, nelle alte sfere, si considerassero gli ultimi avvenimenti. Metrov riferì le parole dette a quel proposito dall'imperatore e da uno dei ministri, apprese da fonte sicura. Katavasov invece aveva sentito, pure da fonte sicura, che l'imperatore aveva detto tutt'altra cosa. Levin cercò di immaginare una situazione in cui potessero essere state dette e le une e le altre parole, ma la conversazione su questo argomentò cessò. 

- Sì, ecco che ha quasi finito di scrivere un libro sulle condizioni naturali del lavoratore in rapporto alla terra - disse Katavasov; - io non sono uno specialista, ma m'è piaciuto, come naturalista, ch'egli non consideri l'umanità qualcosa al di fuori delle leggi zoologiche, ma, al contrario, ne veda la dipendenza dall'ambiente e in questa dipendenza ricerchi le leggi dell'evoluzione. 

- È molto interessante - disse Metrov. 

- Io veramente avevo cominciato a scrivere un libro di economia rurale, ma senza volere, occupandomi dello strumento principale dell'economia rurale, del lavoratore - disse Levin, arrossendo - sono pervenuto a risultati del tutto inaspettati. 

E Levin cominciò a esporre prudentemente, quasi a tastare il terreno, il proprio punto di vista. Sapeva che Metrov aveva scritto un articolo contro la dottrina di economia politica generalmente accettata, ma fino a che punto poteva sperare di trovar simpatia presso di lui per i suoi nuovi criteri, non lo sapeva e non poteva indovinarlo dall'intelligente e calmo volto dello studioso. 

- Ma in che cosa vedete le peculiarità del lavoratore russo? - disse Metrov. - Nelle sue caratteristiche, per così dire, zoologiche o nelle condizioni in cui esso si trova? 

Levin vedeva che in questa domanda si rivelava già un'idea alla quale egli non accedeva; ma continuò a esporre il proprio pensiero, sostenendo che il popolo russo ha un modo di considerare la terra assolutamente particolare rispetto agli altri popoli. E, per dimostrare questa idea, si affrettò ad aggiungere che, secondo la sua opinione, questa visione del popolo russo derivava dalla consapevolezza della propria vocazione di popolare enormi spazi a oriente, non occupati. 

- È facile essere indotti in errore, facendo una conclusione sulla vocazione generale di un popolo - disse Metrov, interrompendo Levin. - Lo stato del lavoratore tuttavia dipenderà sempre dai suoi rapporti con la terra e il capitale. 

E, senza consentire a Levin di esporre a fondo il proprio pensiero, Metrov cominciò a esporgli la particolarità della propria dottrina. 

In che cosa consistesse la particolarità della sua dottrina, Levin non lo capì perché non si sforzava neppure di capirlo: vedeva che Metrov, alla stregua degli altri, sebbene nel proprio articolo avesse smentito la dottrina degli economisti, tuttavia guardava la situazione del lavoratore russo soltanto rispetto al capitale, al salario e alla rendita. Pur riconoscendo che nella parte orientale, la più grande della Russia, la rendita era ancora nulla, che il salario per i nove decimi della popolazione russa di ottanta milioni consisteva soltanto nel nutrire se stessi, e che il capitale non esisteva ancora che sotto l'aspetto di strumenti primitivi, tuttavia soltanto sotto questo aspetto considerava qualsiasi lavoratore, pur senza concordare in molte cose con gli economisti e avendo una sua nuova teoria sul salario, che espose a Levin. 

Levin ascoltava svogliato, e in principio fece delle obiezioni. Voleva interrompere Metrov per dire il proprio pensiero che, secondo lui, avrebbe resa superflua un'ulteriore esposizione. Ma poi, convintosi che consideravano la cosa in modo così diverso, che mai si sarebbero capiti, non contraddisse neppure più e ascoltò soltanto. Pur senza interessarsi affatto a quello che diceva Metrov, provava un certo gusto ad ascoltarlo. Il suo amor proprio era lusingato dal fatto che un uomo così colto gli esponesse le proprie idee così volentieri, con una tale premura e fiducia nella conoscenza della materia da parte di Levin, indicando a volte tutto un aspetto della cosa con una sola allusione. Egli attribuiva ciò al proprio merito, e non sapeva che Metrov, avendone parlato con tutti gli intimi, parlava particolarmente volentieri di questa materia con ogni persona nuova, e che in generale parlava volentieri con tutti della materia che l'occupava, ancora poco chiara a lui stesso. 

- Però arriveremo in ritardo - disse Katavasov, guardando l'orologio, appena Metrov ebbe finito la sua esposizione. 

- Già, oggi c'è una seduta alla Società degli amatori per il cinquantenario di Svintic - disse Katavasov a una domanda di Levin. - Io e Pëtr Ivanovic siamo disposti ad andare. Ho promesso di parlare dei suoi lavori sulla zoologia. Venite con noi, è molto interessante. 

- Sì, è ora - disse Metrov. - Venite con noi e di là, se volete, a casa mia. Desidererei parlare del vostro lavoro. 

- Ma no. È ancora così poco finito. Ma alla seduta sì, sono molto contento. 

- Ebbene, amico mio, avete sentito? Ho presentato una soluzione a parte - disse Katavasov che indossava il frac in un'altra stanza. 

E cominciò una conversazione sulla questione universitaria. 

La questione universitaria di quell'inverno, a Mosca, era molto importante. Tre vecchi professori, in consiglio, non avevano accettato la soluzione dei docenti giovani; i giovani avevano presentato una soluzione separata. Questa soluzione, secondo il giudizio degli uni, era orribile; secondo il giudizio degli altri, era la più semplice e giusta, e i professori si erano divisi in due partiti. Gli uni, a cui apparteneva Katavasov, vedevano nella parte contraria una denuncia vile e un inganno; gli altri, una ragazzata e una mancanza di rispetto verso le autorità. Levin, pur senza appartenere all'università, nella sua permanenza a Mosca aveva già sentito e parlato parecchie volte di questa faccenda e aveva una propria opinione a questo riguardo; prese quindi parte alla conversazione che continuò anche per la strada, finché tutti e tre giunsero alla vecchia università. 

La seduta era già cominciata. Intorno a una tavola coperta d'un panno, a cui sedettero Katavasov e Metrov, sedevano sei persone, e una di loro, curva su di un manoscritto, leggeva qualcosa. Levin sedette su di una delle sedie vuote che stavano intorno alla tavola, e domandò sottovoce a uno studente, ch'era là seduto, cosa leggessero. Lo studente, dopo aver squadrato Levin con aria scontenta, disse: 

- La biografia. 

Sebbene Levin non s'interessasse della biografia dello studioso, pure, involontariamente, prestò ascolto e venne a sapere qualcosa d'interessante e di nuovo sulla vita del famoso scienziato. 

Quando il lettore ebbe finito, il presidente lo ringraziò, lesse i versi del poeta Ment, mandatigli per quella celebrazione, e alcune parole di ringraziamento al poeta. Dopo, Katavasov, con voce forte, stridente, lesse la sua nota sui lavori scientifici di Svintic. Quando Katavasov ebbe finito, Levin guardò l'orologio, vide che erano già più dell'una, e pensò che non avrebbe fatto in tempo a leggere il proprio lavoro a Metrov prima del concerto, e adesso, ormai, non lo desiderava neppure più. Durante la lettura aveva pensato alla conversazione passata. Adesso per lui era chiaro che, ammessa una certa importanza alle idee di Metrov, anche le sue erano importanti; e queste idee potevano chiarirsi e portare a qualcosa solo quando ognuno avesse lavorato separatamente sulla via scelta, ma dalla comunicazione di queste idee non poteva venir fuori nulla. Decisosi quindi a declinare l'invito di Metrov, Levin, alla fine della seduta, gli si avvicinò. Metrov lo presentò al presidente, con cui parlava di novità politiche. Allora Metrov raccontò al presidente la stessa cosa che aveva raccontato a Levin, e Levin fece le stesse osservazioni che aveva già fatto quella mattina e, per variare, espresse una nuova opinione che gli era venuta in mente proprio in quel punto. Dopo si riprese la conversazione sulla questione universitaria. Poiché Levin aveva già sentito tutto questo, si affrettò a dire a Metrov che gli rincresceva di non poter profittare dell'invito, salutò e andò da L'vov. 

\capitolo{IV}\label{iv-6} 

L'vov, che aveva sposato Natalie, la sorella di Kitty, aveva passato tutta la sua vita nelle capitali e all'estero, dove si era formato e dove era stato come diplomatico. 

Da un anno aveva lasciato la carriera diplomatica, non per dispiaceri (non aveva mai avuto dispiaceri con nessuno), ed era passato a un impiego amministrativo della casa imperiale a Mosca, per dare un'educazione migliore ai suoi due ragazzi. 

Malgrado la grande diversità di abitudini e di opinioni e malgrado L'vov fosse più anziano di Levin, quell'inverno avevano stretto grande amicizia e avevano preso a volersi bene. 

L'vov era in casa, e Levin entrò da lui senza farsi annunciare. 

L'vov, in veste da camera con cintura e scarpe scamosciate, era seduto in una poltrona e, con un pince-nez dalle lenti turchine, leggeva un libro che stava su di un leggio, tenendo con accortezza, un po' discosto, con la bella mano un sigaro incenerito a metà. 

Il suo bel viso, fine e ancora giovane, al quale i capelli d'argento lucidi e inanellati davano un'espressione ancor più nobile, s'illuminò d'un sorriso quando scorse Levin. 

- Benissimo! E io volevo mandar da voi. Be', come va Kitty? Sedetevi qua: stiamo più tranquilli\ldots{} Avete letto l'ultima circolare nel ``Journal de St.~Pétersbourg''? Io penso che vada benissimo - egli disse con un accento un po' francese. 

Levin raccontò ciò che aveva sentito da Katavasov e da Metrov su quello che si diceva a Pietroburgo, e, dopo aver parlato un po' di politica, raccontò d'aver conosciuto Metrov e di essere andato alla seduta. Questo interessò molto L'vov. 

- Ecco, vi invidio il libero ingresso in quell'interessante mondo scientifico - egli disse e, preso a parlare, passò, come al solito, immediatamente al francese, per lui più comodo. 

- È vero che io non ne ho neppure il tempo. E il mio impegno e la cura dei ragazzi me ne privano; e poi, non mi vergogno di dire che la mia cultura è insufficiente. 

- Questo non lo credo - disse Levin con un sorriso, ammirando, come sempre, quella modesta opinione di sé, non assunta per il desiderio di apparire o anche di essere modesto, ma assolutamente sincera. 

- Eh, come! Lo sento, adesso, come sono poco istruito. Perfino per l'educazione dei ragazzi devo rinfrescare molte cose nella memoria o impararle del tutto. Perché non basta che ci siano dei maestri, bisogna che ci sia chi sorveglia, così come nella vostra azienda ci vogliono i lavoratori e un sorvegliante. Ecco, sto leggendo - egli indicò la grammatica di Buslaev, che stava sul leggio; - pretendono questo da Miša, ed è così difficile\ldots{} Su, ecco, spiegatemi. Qui egli dice\ldots{} 

Levin voleva spiegargli che non si poteva capire, ma che bisognava impararlo; ma L'vov non era d'accordo con lui. 

- Sì, ecco, voi ci ridete su! 

- Al contrario, non potete immaginare come, guardando voi, io impari sempre quello che per me è imminente, l'educazione dei bambini. 

- Ma da imparare non c'è nulla - disse L'vov. 

- Io so soltanto - disse Levin - che non ho veduto ragazzi più educati dei vostri, e non potrei desiderarne migliori dei vostri. 

L'vov, evidentemente, voleva contenersi, per non esprimere la propria gioia, ma s'illuminò tutto in un sorriso. 

- Basta che siano migliori di me. Ecco quello che desidero. Voi non sapete ancora tutta la fatica - egli cominciò - con dei ragazzi che, come i miei, sono stati trascurati per causa di questa vita all'estero. 

- Tutto questo lo riguadagnerete. Sono ragazzi di grande talento. La cosa più importante è l'educazione morale. Ecco ciò che imparo, guardando i vostri figli. 

- Voi dite: l'educazione morale. Non ci si può immaginare come sia difficile! Avete appena vinto un'inclinazione, che ne vengono fuori altre, e di nuovo bisogna lottare. Se non si ha un appoggio nella religione (ricordate che ne parlavamo con voi?), nessun padre, con le sole sue forze, potrebbe educare senza questo aiuto. 

Questa conversazione, che interessava sempre Levin, fu interrotta dalla bella Natalie Aleksandrovna ch'era entrata, già pronta per uscire. 

- E io non sapevo che foste qui - ella disse, evidentemente non solo non rimpiangendo, ma rallegrandosi di aver interrotto quella conversazione a lei nota da tempo e venutale a noia. - Ebbene, come va Kitty? Pranzo da voi, oggi. Ecco, Arsenij - ella disse rivolta al marito - ti prenderai la carrozza\ldots{} 

E tra marito e moglie si cominciò a decidere come avrebbero passato la giornata. Poiché lui doveva andare a incontrare qualcuno per dovere d'ufficio e lei al concerto e alla seduta pubblica del comitato sud-orientale, bisognava risolvere e considerare molte cose. Levin, come persona di casa, doveva prender parte a questi conciliaboli. Fu deciso che Levin sarebbe andato al concerto e alla seduta pubblica con Natalie, e di là avrebbero mandato la carrozza all'ufficio a prendere Arsenij, e lui sarebbe passato a rilevarla per portarla da Kitty; oppure, s'egli non avesse sbrigato gli affari, avrebbe mandato la carrozza, e Levin sarebbe andato con lei. 

- Ecco, egli mi vizia - disse L'vov alla moglie - mi assicura che i nostri ragazzi sono buonissimi, quando io so che in loro c'è tanto di cattivo. 

- Arsenij giunge agli estremi, io dico sempre - disse la moglie. - A cercar la perfezione, non si sarà mai contenti. E dice bene papà, che quando educavano noi, si esagerava, ci tenevano nei mezzanini, e i genitori vivevano al piano nobile; adesso al contrario, i genitori in uno stambugio e i ragazzi al piano nobile. I genitori adesso non devono più vivere, ma tutto deve essere per i figli. 

- Ebbene, se questo fa più piacere? - disse L'vov, sorridendo col suo bel sorriso e toccandole il braccio. - Chi non ti conosce penserà che tu non sia un madre, ma una matrigna. 

- No, l'esagerazione non va bene in nulla - disse tranquilla Natalie, mettendo al suo posto il tagliacarte sul tavolo. 

- Su, ecco, venite qua, ragazzi perfetti - disse L'vov ai bei ragazzi che entravano, i quali, dopo aver salutato Levin, si avvicinarono al padre, desiderando evidentemente di chiedergli qualcosa. 

Levin aveva voglia di parlare un po' con loro, d'ascoltare quello che avrebbero detto al padre, ma Natalie si mise a parlare con lui, e proprio in quel momento entrò nella stanza un compagno di ufficio di L'vov, Machotin, in uniforme di corte, per andare insieme ad incontrare quella tale persona, e cominciò una conversazione senza fine sull'Erzegovina, sulla principessa Korzinskaja, sull'assemblea, sulla morte improvvisa dell'Apraksina. 

Levin s'era perfino dimenticato dell'incarico che gli avevano dato. Se ne ricordò quando era già per uscire, in anticamera. 

- Ah, Kitty mi ha incaricato di parlare un po' con voi degli Oblonskij - disse quando L'vov si fermò sulla scala, accompagnando la moglie e lui. 

- Sì, sì, maman vuole che noi, les beaux-frères, lo assaliamo - disse egli, arrossendo. - E poi, perché mai io? 

- E allora, lo assalirò io - disse, sorridendo, la L'vova, che aspettava la fine della conversazione nella sua bianca rotonde di pelo di cane. - Su, andiamo. 

\capitolo{V}\label{v-6} 

Al concerto pomeridiano venivano date due cose molto interessanti. 

Una era la fantasia Re Lear nella steppa, l'altra era un quartetto dedicato alla memoria di Bach. Tutte e due le cose erano nuove e di gusto moderno, e Levin desiderava di farsene un'opinione. Dopo aver accompagnato la cognata alla sua poltrona, si pose in piedi vicino a una colonna e decise di ascoltare nel modo più attento e coscienzioso possibile. Cercava di non distrarsi e di non sciupare l'impressione guardando il gesticolare delle mani del direttore d'orchestra dalla cravatta bianca, che sempre così spiacevolmente distrae l'attenzione musicale, le signore in cappello, che per il concerto s'erano fasciate con cura le orecchie con nastri, e tutti quei visi oziosi, o presi dagli interessi più vari, tranne quello della musica. Cercava di evitare incontri con intenditori di musica e ciarlatori, guardando in giù davanti a sé, e ascoltando. 

Ma quanto più egli ascoltava la fantasia di Re Lear, tanto più si sentiva lontano dalla possibilità di formarsene una opinione definitiva. Senza posa cominciava, come se si preparasse l'espressione musicale d'un sentimento, ma subito si scomponeva in nuovi frammenti di frasi musicali, accennate, e a volte in suoni per null'altro legati se non che per capriccio del compositore, e straordinariamente complessi. Ma anche gli stessi frammenti di queste frasi musicali, a volte buone, erano spiacevoli, perché del tutto inaspettati e non predisposti. L'allegria e la tristezza, la disperazione e la tenerezza e il trionfo comparivano senza alcun diritto, come i sentimenti di un folle. E, nello stesso tempo, come nel folle, questi sentimenti passavano inaspettatamente. 

Levin, per tutto il tempo della esecuzione, provò la sensazione d'un sordo che guardi dei danzatori. Era in un'assoluta perplessità quando il brano fu finito e sentiva una grande stanchezza per l'attenzione tesa e non ricompensata da nulla. Da tutte le parti si udirono forti applausi. Tutti si alzarono, presero a camminare, a parlare. Desiderando di chiarire la propria perplessità con le opinioni degli altri, Levin andò in giro, cercando gli intenditori, e fu contento di scorgerne uno dei più noti in colloquio con Pescov ch'egli conosceva. 

- Sorprendente! - diceva la voce piena, di basso, di Pescov. - Buon giorno, Konstantin Dmitric. È in particolar modo immaginoso e scultoreo, per così dire, ricco di colori quel punto dove si sente l'avvicinarsi di Cordelia, dove la donna, das ewig Weibliche, entra in lotta col fato. Non è vero? 

- Come dite, perché mai Cordelia qui? - domandò timido Levin, avendo completamente dimenticato che la fantasia rappresentava re Lear nella steppa. 

- Appare Cordelia\ldots{} ecco! - disse Pescov, battendo col dito sul programma di raso che teneva in mano e passandolo a Levin. 

Soltanto allora Levin si ricordò del titolo della fantasia e si affrettò a leggere i versi di Shakespeare nella traduzione russa, stampati a tergo nel programma. 

- Senza questo non si può seguire - disse Pescov, rivolgendosi a Levin, poiché il suo interlocutore se n'era andato ed egli non aveva più con chi parlare. 

Nell'intervallo si intavolò fra Levin e Pescov una discussione sui pregi e sui difetti dell'indirizzo wagneriano della musica. Levin dimostrava che l'errore di Wagner e di tutti i suoi discepoli consisteva nel fatto che la musica voleva passare nel campo di un'arte non sua, che nello stesso modo vien meno la poesia quando descrive i tratti di un volto, cosa che spetta alla pittura, e come esempio di un tale errore, citò uno scultore cui era venuto in mente di tagliare nel marmo le ombre delle immagini poetiche sorgenti intorno alla figura d'un poeta su di un piedistallo. 

- Queste ombre sono così poco ombre nell'opera dello scultore, che devono reggersi a una scala - disse Levin. 

Questa frase gli piacque, ma non ricordava se questa stessa frase non l'avesse detta già prima, e proprio a Pescov, così che, detto questo, si confuse. 

Pescov, invece, dimostrava che l'arte era una e che poteva raggiungere le sue più alte manifestazioni soltanto nell'unione di tutti i generi. 

Il secondo numero del concerto Levin non poté più ascoltarlo. Pescov, fermatoglisi accanto, parlò quasi tutto il tempo, riprovando quel pezzo per la sua eccessiva, sdolcinata, voluta, semplicità, paragonandola alla semplicità dei preraffaelliti in pittura. All'uscita, Levin incontrò ancora molti conoscenti coi quali parlò di politica, di musica e di amici comuni; fra l'altro incontrò il conte Bol', dal quale aveva completamente dimenticato di andare. 

- Su, allora andate subito - gli disse la L'vova, alla quale egli riferì questo - forse non vi riceveranno, e poi venite a prendermi alla seduta. Mi troverete ancora. 

\capitolo{VI}\label{vi-6} 

- Forse non ricevono? - disse Levin, entrando nell'ingresso della casa della contessa Bol'. 

- Ricevono, prego - disse il portiere, togliendogli decisamente la pelliccia di dosso. 

``Che rabbia! - pensava Levin, togliendosi con un sospiro un guanto e accomodando il cappello. - Ma perché ci vengo? che ragione ho mai di parlare con loro?''. 

Passando per il primo salotto, Levin incontrò sulla porta la contessa Bol' che, con viso preoccupato e severo, ordinava qualcosa a un cameriere. Visto Levin, sorrise e lo fece entrare nel piccolo salotto attiguo, in cui si sentivano delle voci. In questo salotto sedevano, in poltrona, le due figlie della contessa e un colonnello moscovita che Levin conosceva. Levin si avvicinò loro, salutò e sedette accanto al divano, tenendo il cappello sulle ginocchia. 

- Come va la salute di vostra moglie? Siete stato al concerto? Noi non abbiamo potuto. La mamma doveva andare alla messa di requiem. 

- Sì, ho sentito\ldots{} Che morte improvvisa! - disse Levin. 

Venne la contessa, sedette sul divano e domandò anche lei della moglie e del concerto. 

Levin rispose e ripeté la frase sulla morte improvvisa della Apraksina. 

- Del resto era sempre stata di salute debole. 

- Siete stato all'opera ieri? 

- Sì, ci sono stato. 

- Come è andata bene la Lucca! 

- Sì, molto bene - disse egli e, siccome gli era del tutto indifferente quel che avrebbero pensato di lui, cominciò a ripetere quel che aveva sentito centinaia di volte sulla particolarità del talento della cantante. La contessa Bol' fingeva di ascoltare. Poi, quando egli ebbe parlato abbastanza e tacque, il colonnello che fino allora aveva taciuto, cominciò a parlare. Il colonnello prese a parlare anche lui dell'opera e dell'illuminazione. Infine, dopo aver parlato della sua programmata folle journée da Tjurin, si mise a ridere e a far chiasso, poi si alzò e uscì. Levin si alzò pure, ma dal viso della contessa si accorse che per lui non era ancora tempo d'andarsene. Ci volevano ancora due minuti. Sedette. 

Ma poiché pensava di continuo come fosse sciocco tutto ciò, non trovava neppure un argomento di conversazione e taceva. 

- Non andate alla seduta pubblica? Dicono che sia molto interessante - disse la contessa. 

- No, ho promesso alla mia belle-soeur d'andarla a prendere - disse Levin. 

Seguì un silenzio. La madre e la figlia si guardarono ancora una volta. 

``Su, adesso pare che sia ora'' pensò Levin e si alzò. Le signore gli strinsero la mano e pregarono di dire mille choses alla moglie. 

Il portiere gli domandò, tendendogli la pelliccia: 

- Dove abitate, di grazia? - e lo annotò immediatamente in un gran libro ben rilegato. 

``S'intende, per me è lo stesso, tuttavia è vergognoso e orribilmente sciocco'' pensò Levin, consolandosi col dirsi che lo facevano tutti, e andò alla seduta pubblica del comitato, dove doveva trovar la cognata per andare a casa con lei. 

Alla seduta pubblica del comitato c'era molta gente e quasi tutto il gran mondo. Levin giunse ancora in tempo per sentire una relazione che, come dicevano tutti, era molto interessante. Quando questa lettura ebbe termine, la società si riunì e lì Levin incontrò Svijazskij che lo invitò assolutamente per quella sera alla società d'economia rurale, dove si sarebbe letta una famosa relazione, incontrò Stepan Arkad'ic, che era appena arrivato dalle corse e ancora molti altri conoscenti; e Levin parlò e ascoltò i giudizi più svariati sulla seduta, su una commedia nuova e su di un processo. Ma, probabilmente per la stanchezza dovuta all'attenzione, che cominciava a provare, si sbagliò parlando del processo, e questo sbaglio poi gli venne in mente varie volte con suo dispetto. Parlando della prossima condanna d'uno straniero giudicato in Russia, e di come sarebbe stato ingiusto punirlo con l'estradizione, Levin aveva ripetuto quel che aveva sentito il giorno innanzi in una conversazione da un conoscente. 

- Io penso che mandarlo all'estero è lo stesso che punire un luccio lasciandolo andare in acqua - disse Levin. Soltanto dopo si ricordò che questa idea, sentita da un conoscente e data per propria da costui, era di una favola di Krylov, e che il conoscente l'aveva ripetuta ricavandola dall'articolo di un giornale. 

Rientrato a casa con la cognata e trovatavi Kitty allegra e felice, Levin andò al club. 

\capitolo{VII}\label{vii-6} 

Levin arrivò al club all'ora giusta. Insieme con lui giungevano ospiti e soci. Levin non era stato al club da molto tempo, fin da quando, finiti gli studi universitari, viveva a Mosca e andava in società. Ricordava il club, i particolari esteriori della sua organizzazione, ma aveva completamente dimenticato l'impressione che provava prima al club. Ma appena entrato nel cortile largo, semicircolare, e appena sceso dalla vettura, salì la scalinata e incontro a lui un portiere con la bandoliera aprì la porta senza rumore e s'inchinò; appena vide nella portineria le soprascarpe e le pellicce dei soci, i quali avevano considerato che costava minor fatica togliere le soprascarpe giù che non portarle su; appena sentì la scampanellata misteriosa che lo precedeva e, salendo per la scala a dolce pendio, coperta d'un tappeto, vide la statua sul pianerottolo e, sulla porta di sopra, nella livrea del club, il terzo portiere a lui noto, invecchiato, il quale senza affrettarsi e senza indugiare apriva la porta ed esaminava l'ospite, Levin fu preso dall'antica impressione del club: un'impressione di distensione, di benessere e di decoro. 

- Favorite il cappello - disse il portiere a Levin, che aveva dimenticato la regola del club di lasciare i cappelli in portineria. - È un pezzo che non siete venuto. Il principe vi ha iscritto proprio ieri. Il principe Stepan Arkad'ic non c'è ancora. 

Il portiere conosceva non solo Levin, ma anche tutte le relazioni e la parentela, e aveva ricordato immediatamente le persone che gli erano prossime. 

Attraversata la prima sala di passaggio coi paraventi e a destra la stanza chiusa da un tramezzo dove sedeva il dispensiere della frutta, Levin, sorpassato un vecchio che camminava adagio, andò nella sala da pranzo che rumoreggiava di gente. 

Passò lungo le tavole già quasi occupate, osservando gli ospiti. Ora qua, ora là gli capitavano dinanzi le persone più disparate, e vecchie e giovani, appena conosciute e intime. Non c'era neppure un viso irritato o preoccupato. Sembrava che tutti avessero lasciato in portineria, insieme con i berretti, le agitazioni e le preoccupazioni, e si preparassero a usar senza fretta dei beni materiali della vita. Qui c'erano Svijazskij e Šcerbackij e Nevedovskij e il vecchio principe e Vronskij e Sergej Ivanovic. 

- Ah, come, sei in ritardo? - disse, sorridendo, il principe, dandogli la mano al di sopra della spalla. - Come va Kitty? - aggiunse, accomodando il tovagliolo che s'era messo dietro il bottone del panciotto. 

- Sta bene; pranzano in tre a casa. 

- Ah, le Aline-Nadine! Eh, da noi non c'è posto. Ma va' a quella tavola e occupa presto un posto - disse il principe e voltatosi, accolse con prudenza un piatto con la zuppa di pesce. 

- Levin, qua! - gridò un po' più lontano una voce bonaria. Era Turovcyn. Sedeva con un giovane militare, e accanto a loro c'erano due sedie girate. Levin si accostò a loro con gioia. Egli voleva sempre bene a quel bonaccione e gozzovigliatore di Turovcyn, al quale si univa il ricordo della spiegazione con Kitty, ma quel giorno, dopo tutte le conversazioni intellettuali che avevano richiesto un certo sforzo, l'aspetto bonario di Turovcyn gli era particolarmente gradito. 

- Questo è per voi e per Oblonskij. Verrà subito. 

Il militare dagli occhi allegri, sempre ridenti, che si teneva molto dritto, era Gagin di Pietroburgo. Turovcyn li presentò. 

- Oblonskij è certamente in ritardo. 

- Eh, ecco anche lui. 

- Sei appena arrivato? - disse Oblonskij, avvicinandosi svelto a loro. - Salve. Hai bevuto la vodka? Su, andiamo. 

Levin si alzò e andò con lui verso una grande tavola piena di bottiglie di vodka e dei più svariati antipasti. Sembrava che si potesse scegliere quello ch'era di proprio gusto fra una ventina di antipasti, ma Stepan Arkad'ic ne volle uno speciale, e uno dei camerieri in livrea, che stava lì in piedi, portò subito quello che era richiesto. Bevvero un bicchierino per uno e tornarono alla tavola. Immediatamente, mentre ancora mangiavano la zuppa di pesce, a Gagin fu servito dello champagne ed egli ordinò di versarlo in quattro bicchieri. Levin non rifiutò il vino offertogli e chiese un'altra bottiglia. Gli era venuta fame e mangiava e beveva con gran piacere, e con piacere ancora maggiore prendeva parte agli allegri e semplici discorsi degli intervenuti. Gagin, abbassando la voce, raccontò una nuova storiella di Pietroburgo, e la storiella, benché indecente e sciocca, era così comica che Levin scoppiò a ridere tanto forte da far voltare i vicini. 

- È dello stesso stampo di: ``io questo proprio non lo posso sopportare''. La sai? - chiese Stepan Arkad'ic. - Ah, è un incanto! Dammi un'altra bottiglia - disse al cameriere e prese a raccontare. 

- Pëtr Il'ic Vinovskij offre - lo interruppe un cameriere vecchio, avvicinando due bicchieri sottili pieni di champagne spumeggiante, e rivolgendosi a Stepan Arkad'ic e a Levin. Stepan Arkad'ic prese il bicchiere e, scambiato uno sguardo con un uomo rosso, calvo e baffuto, all'altro capo della tavola, gli fece un cenno col capo, sorridendo. 

- Chi è? - disse Levin. 

- L'hai incontrato da me una volta, ricordi? Un bravo ragazzo\ldots{} 

Levin imitò Stepan Arkad'ic e prese il bicchiere. 

La storiella di Stepan Arkad'ic era pure molto divertente. Levin raccontò la sua che pure piacque. Dopo, il discorso cadde sui cavalli, sulle corse di quel giorno e su come audacemente aveva vinto il primo premio Atlasnyj di Vronskij. Levin non si accorse come fosse passato il tempo del pranzo. 

- Ah, ecco anche loro! - disse alla fine del pranzo Stepan Arkad'ic, piegandosi di là dalla spalliera della sedia e tendendo la mano a Vronskij che veniva verso di lui con un colonnello alto della Guardia. Nel viso di Vronskij splendeva la stessa generale allegra bonomia del club. Egli si appoggiò allegramente col gomito alla spalla di Stepan Arkad'ic, mormorandogli qualcosa, e con lo stesso allegro sorriso tese la mano a Levin. 

- Sono molto contento d'incontrarvi - disse. - Vi avevo cercato allora alle elezioni, ma mi dissero che eravate andato via - gli disse. 

- Sì, partii il giorno stesso. Or ora parlavamo del vostro cavallo. Mi congratulo con voi - disse Levin. - È veramente una velocità notevole. 

- Ma già, anche voi avete dei cavalli. 

- No, mio padre ne aveva; ma io me ne ricordo e me ne intendo. 

- Dove hai pranzato? - domandò Stepan Arkad'ic. 

- Noi, alla seconda tavola, dietro le colonne. 

- Gli han fatto i complimenti - disse il colonnello alto. - Il secondo premio imperiale; avessi io tanta fortuna alle carte, quanta ce n'ha lui coi cavalli. 

- Ma perché perdere del tempo d'oro? Vado nell'infernale - disse il colonnello e si allontanò dalla tavola. 

- È Jašvin - rispose Vronskij a Turovcyn, e sedette a un posto che s'era fatto libero accanto a loro. Bevuta la coppa offertagli, chiese una bottiglia. Fosse sotto l'influsso del club, o del vino bevuto, Levin si mise a parlare con Vronskij della razza migliore di bestiame e fu molto contento di non sentir nessuna avversione per quell'uomo. Gli disse perfino, tra l'altro, di aver udito dalla moglie che lo aveva incontrato dalla principessa Mar'ja Borisovna. 

- Ah, la principessa Mar'ja Borisovna, che delizia! - disse Stepan Arkad'ic e raccontò su di lei una storiella che fece ridere tutti. In particolare, Vronskij scoppiò a ridere così di cuore, che Levin si sentì del tutto rappacificato con lui. 

- Ebbene, avete finito? - disse Stepan Arkad'ic, alzandosi e sorridendo. - Andiamo! 

\capitolo{VIII}\label{viii-6} 

Levin, alzatosi da tavola, sentendo nel camminare i movimenti delle braccia particolarmente regolari e liberi, andò con Gagin, attraverso le stanze alte, al biliardo. Passando per il salone, si scontrò col suocero. 

- Be', che te ne pare del nostro tempio dell'ozio? - chiese il principe, prendendolo sotto braccio. - Andiamo a fare un giro. 

- Proprio questo volevo fare, andare a guardare un po'. È interessante. 

- Già, per te è interessante. Ma per me l'interesse è un altro. Tu, ecco, guardi questi vecchietti - disse indicando un socio ingobbito, con un labbro penzoloni che, movendo appena le gambe negli stivali flosci, veniva loro incontro - e pensi che siano nati proprio šljupiki. 

- Come? 

- Ecco, non conosci neppure questa parola. È un nostro termine del club. Sai, quando fanno rotolare le uova, se le fanno rotolare molto, diventano šljupiki. Così anche noi: vai, vai al club e diventi šljupiki. Ma ecco che tu ridi, anche noi abbiamo riso un tempo come te; ora invece guardiamo già come diventeremo šljupiki. Lo conosci il principe cecenskij? - domandò il principe, e Levin vedeva dalla faccia ch'egli si accingeva a raccontar qualcosa di ameno. 

- No, non lo conosco. 

- E come mai? Via, il principe cecenskij, quello famoso. Ma fa lo stesso. Lui, ecco, giuoca sempre a biliardo. Fino a tre anni fa non era ancora tra gli šljupiki e faceva il gradasso. E dava dello šljupik agli altri. Ma ecco, una volta, arriva, e il nostro portiere\ldots{} sai, Vasilij? quello grasso. È un bello spirito. Il principe cecenskij gli domanda: ``Be', Vasilij, chi c'è e chi è venuto? E di šljupiki ce ne sono?''. E quello: ``Ecco, con voi sono tre''. E già, amico mio, proprio così. 

Discorrendo e salutando gli amici che incontravano, Levin e il principe passarono tutte le sale; quella grande dove c'erano già i tavoli e dove i soliti compagni di giuoco facevano la partita; la sala dei divani dove si giocava a scacchi e dove era seduto Sergej Ivanovic il quale discorreva con un tale; quella del biliardo, dove in un gomito della stanza, presso un divano, si era formata un'allegra compagnia con champagne, alla quale prendeva parte Gagin. Dettero un'occhiata anche in quella infernale, dove vicino a un tavolo, al quale era già seduto Jašvin, si affollavano molti sostenitori. Cercando di non far rumore, entrarono nella sala oscura di lettura, dove, sotto le lampade coi paralumi, sedevano un giovane dal viso torvo, che afferrava un giornale dietro l'altro, e un generale calvo, sprofondato nella lettura. Poi, entrarono anche nella stanza che il principe chiamava intellettuale. In questa stanza tre signori parlavano con calore dell'ultima novità politica. 

- Principe, prego, è pronto - disse uno dei suoi compagni di giuoco, trovandolo lì, e il principe andò via. Levin rimase un po' a sedere, ad ascoltare, ma ricordando tutti i discorsi della mattina, gli venne a un tratto una malinconia terribile. Si alzò in fretta e andò a cercare Oblonskij e Turovcyn, coi quali si stava allegri. 

Turovcyn era seduto nella sala del biliardo su di un divano alto con una coppa in mano, e Stepan Arkad'ic con Vronskij parlava di qualcosa, vicino alla porta, in un angolo lontano della stanza. 

- Non è che si annoi, ma questa situazione indefinita, incerta - sentì Levin e voleva allontanarsi in fretta; ma Stepan Arkad'ic lo chiamò. 

- Levin! - disse Stepan Arkad'ic, e Levin notò che negli occhi non aveva lacrime, ma un certo umidore, come gli accadeva sempre quando aveva bevuto o quando era commosso. Quel giorno era l'una e l'altra cosa. - Levin, non te ne andare - disse e gli strinse forte il braccio per il gomito, evidentemente desiderando di non lasciarlo andare per nessuna ragione al mondo. - È il mio sincero, forse il mio migliore amico - disse a Vronskij. - Tu per me sei ancora più prossimo e caro. E io voglio e so che voi dovete essere amici, molto amici, perché siete tutti e due brave persone. 

- E allora, non ci rimane che abbracciarci - disse, scherzando, bonariamente, Vronskij, mentre dava la mano. 

Egli prese rapido la mano tesa e la strinse forte. 

- Sono molto, molto contento - disse Levin, stringendogli la mano. 

- Cameriere, una bottiglia di champagne - disse Stepan Arkad'ic. 

- Anch'io sono molto contento - replicò Vronskij. 

Tuttavia, malgrado il desiderio di Stepan Arkad'ic e il loro reciproco desiderio, non avevano nulla da dirsi, e lo sentivano tutti e due. 

- Sai che lui non conosce Anna? - disse Stepan Arkad'ic a Vronskij. - E io voglio assolutamente portarlo da lei. Andiamo, Levin. 

- Davvero? - disse Vronskij. - Lei ne sarà molto lieta. Io andrei subito a casa - egli soggiunse - ma Jašvin mi preoccupa, e voglio rimanere qua finché non smette. 

- Perché, va male? 

- Sta perdendo tutto, e solo io riesco a trattenerlo. 

- E allora, una carambola? Levin, giuochi? Benissimo - disse Stepan Arkad'ic. - Metti una carambola - disse rivolto al marcatore. 

- È pronta da un pezzo - rispose il marcatore che aveva già messo a triangolo le palle e per distrarsi faceva rotolar la palla rossa. 

- Su, via. 

Dopo la partita, Vronskij e Levin sedettero vicino al tavolo di Gagin, e Levin, su proposta di Stepan Arkad'ic, si mise a puntare sugli assi. Vronskij ora sedeva presso il tavolo circondato da amici che gli si avvicinavano continuamente, ora andava nell'infernale a trovare Jašvin. Levin provava un piacevole riposo dalla stanchezza intellettuale del mattino. Lo rallegrava la fine dell'avversione verso Vronskij, e il senso di distensione, benessere e decoro non lo lasciava. 

Quando la partita fu finita, Stepan Arkad'ic prese Levin sotto braccio. 

- Su, allora, andiamo da Anna. Subito? eh? È in casa. Da tempo le ho promesso di condurti da lei. Tu, dove pensavi di andare a passare la serata? 

- Ma in nessun posto particolare. Ho promesso a Svijazskij di andare alla società di economia rurale. Ma andiamo - disse Levin. 

- Benissimo, andiamo! Informati se è venuta la mia carrozza - disse Stepan Arkad'ic a un cameriere. 

Levin si avvicinò al tavolo e pagò i quaranta rubli da lui perduti sugli assi, pagò le spese del club, note, in un certo modo misterioso, al vecchio cameriere che stava in piedi vicino alla porta, e agitando con forza le braccia attraversò le sale dirigendosi verso l'uscita. 

\capitolo{IX}\label{ix-6} 

- La carrozza di Oblonskij! - gridò il portiere con voce di basso irritata. La carrozza si accostò, e ci salirono tutti e due. Solo nel primo momento, mentre la carrozza usciva dal portone, Levin continuò a provare l'impressione di distensione del club, di benessere e di sicuro decoro di quello che lo circondava; ma non appena la carrozza uscì sulla via ed egli sentì il traballìo del veicolo sulla strada ineguale, udì il grido rabbioso di un vetturale, e vide nella luce smorta l'insegna rossa d'una bettola e di una botteguccia, quell'impressione svanì ed egli cominciò a riflettere se faceva bene, oppure no ad andare da Anna. Che avrebbe detto Kitty? Ma Stepan Arkad'ic non gli consentì di riflettere e, quasi indovinando i suoi dubbi, li disperse. 

- Come sono contento - disse - che tu la conosca! Lo sai, Dolly lo desiderava da tempo. Anche L'vov è già stato da lei e ci va. Sebbene mi sia sorella - seguitò Stepan Arkad'ic - posso dire senza timore che è una donna particolare. Ecco, vedrai. La sua situazione è molto penosa, specialmente adesso. 

- E perché proprio adesso? 

- Abbiamo in corso trattative con suo marito per il divorzio. E lui acconsente; ma ora sorgono delle difficoltà per il figlio, e questa faccenda, che doveva finire da un pezzo, si trascina da tre mesi. Non appena verrà il divorzio, ella sposerà Vronskij. Com'è sciocca quella vecchia usanza, a cui nessuno crede, di fare il giro cantando ``Isaia esulta'' che intralcia la felicità della gente! - intercalò Stepan Arkad'ic. - Bene, prima o poi la loro situazione sarà definita, come la mia e la tua. 

- E in che consiste la difficoltà? - disse Levin. 

- Ah, è una storia lunga e noiosa! Tutto questo è così mal definito da noi. Ma il fatto è che lei vive da tre mesi, aspettando questo divorzio, a Mosca, dove tutti conoscono lui e lei; non va in nessun posto; di donne non vede che Dolly, perché, capirai, non vuole che vadano da lei per compassione; quella stupida principessa Varvara, anche quella è andata via, ritenendo ciò poco conveniente. E così, in questa situazione un'altra donna non avrebbe potuto trovare risorse in se stessa. Lei invece, ecco, vedrai come ha organizzato la propria vita, come è calma, dignitosa. A sinistra, nel vicolo di fronte alla chiesa! - gridò Stepan Arkad'ic, piegandosi verso il finestrino della carrozza. - Uff! che caldo! - disse, aprendo ancor più, malgrado i dodici gradi sotto zero, la sua pelliccia già aperta. 

- Ma poiché ha una figlia, probabilmente, si occupa di lei - disse Levin. 

- Tu, a quanto pare, immagini ogni donna come una femmina, come une couveuse - disse Stepan Arkad'ic. - Se è occupata, è di sicuro coi bambini. No, la educa benissimo, a quanto pare, ma non se ne sente parlare. È occupata, in primo luogo, a scrivere. Vedo già che tu sorridi ironicamente, ma a torto. Scrive un libro per ragazzi e non ne parla a nessuno, ma a me l'ha letto, e io ho dato il manoscritto a Vorkuev\ldots{} sai quell'editore\ldots{} e anche lui è scrittore, mi pare. Lui se ne intende, e dice che è una cosa notevole. Ma tu pensi che sia una scrittrice? Per nulla. Prima di tutto è una donna di cuore, del resto lo vedrai. Adesso ha una bambina inglese e tutta una famiglia di cui si occupa. 

- Be', qualcosa di filantropico? 

- Ecco, tu vuoi subito vedere il male. Non di filantropico, ma di cuore. Loro, cioè Vronskij, aveva un allenatore inglese, maestro dell'arte sua, ma ubriacone. Costui s'è proprio dato al bere, delirium tremens; e i familiari sono abbandonati. Lei li ha visti, li ha aiutati, ci s'è affezionata, e adesso tutta la famiglia è sulle sue spalle, e non così, dall'alto in basso, a denari, ma lei stessa prepara i bambini per il russo per l'ammissione al ginnasio, e la bambina l'ha presa con sé. Ma ecco, la vedrai. 

La carrozza entrò nel cortile, e Stepan Arkad'ic sonò forte a un ingresso dove era ferma una slitta. 

E senza chiedere all'inserviente che aveva aperto la porta se erano in casa, Stepan Arkad'ic entrò nell'ingresso. Levin lo seguiva, sempre più in dubbio se faceva bene o male. 

Guardatosi nello specchio, Levin notò che era divenuto rosso; ma era sicuro di non essere ubriaco e andò su per la scala, coperta di un tappeto, dietro a Stepan Arkad'ic. Di sopra, al cameriere che s'era inchinato come a persona intima, Stepan Arkad'ic chiese chi c'era da Anna Arkad'evna e ricevette la risposta che c'era il signor Vorkuev. 

- Dove sono? 

- Nello studio. 

Attraversata una piccola sala da pranzo con pareti scure di legno, Stepan Arkad'ic e Levin, su un morbido tappeto, entrarono in uno studio semibuio, illuminato da una sola lampada con un grosso paralume scuro. Un'altra lampada a riflettere era accesa sulla parete e illuminava un gran ritratto di donna in piedi, su cui Levin rivolse involontariamente l'attenzione. Era il ritratto di Anna, fatto in Italia da Michajlov. Mentre Stepan Arkad'ic entrava di là da una grata e una voce maschile che parlava tacque, Levin guardò il ritratto, che nella luce scintillante risaltava fuori dalla cornice, e non riuscì a staccarne gli occhi. Aveva perfino dimenticato dove si trovava, e, senza ascoltare quello che si diceva, non abbassava gli occhi dal ritratto meraviglioso. Non era un ritratto, ma una deliziosa donna viva, coi capelli neri ondulati, le spalle e le braccia nude e un pensoso, appena accennato sorriso sulle labbra coperte di sottile peluria, che lo guardava trionfante e tenera con occhi che intimidivano. Non era viva solo perché era più bella di quel che possa essere una donna viva. 

- Sono molto contenta - egli sentì a un tratto accanto a sé una voce evidentemente rivolta a lui, la voce di quella stessa donna che aveva ammirato nel quadro. Anna gli era uscita incontro di là dalla grata e Levin vide, nella penombra dello studio, quella stessa donna del ritratto, in abito scuro d'un turchino cangiante, non nella posa, non con l'espressione, ma della stessa bellezza con cui era stata colta dall'artista nel ritratto. Era meno splendente nella realtà, ma in compenso in lei viva c'era un fascino nuovo che mancava nel ritratto. 

\capitolo{X}\label{x-6} 

Ella gli era andata incontro, senza nascondere la propria gioia nel vederlo. E nella calma con cui ella tese la mano piccola ed energica, e con cui lo presentò a Vorkuev e indicò la graziosa bambina rossastra che sedeva là, intenta a un lavoro, chiamandola sua allieva, Levin riconobbe le maniere a lui note e gradite della donna del gran mondo, sempre calma e naturale. 

- Molto, molto contenta - ella ripeté e sulle sue labbra, chissà perché, quelle parole acquistarono per Levin un significato particolare. - Vi conosco e vi voglio bene da lungo tempo e per l'amicizia con Stiva e per vostra moglie\ldots{} l'ho conosciuta per poco tempo, ma ha lasciato in me l'impressione di un fiore delizioso, proprio di un fiore. E sarà presto mamma! 

Ella parlava liberamente e senza fretta, qualche rara volta portando il suo sguardo da Levin al fratello, e Levin sentiva che l'impressione da lui prodotta era buona, e subito provò una sensazione lieve, semplice a star con lei, come se l'avesse conosciuta dall'infanzia. 

- Ivan Petrovic ed io ci siamo messi nello studio di Aleksej - disse ella, rispondendo a Stepan Arkad'ic alla domanda se poteva fumare - appunto per fumare - e, dopo aver guardato Levin, invece di chiedergli se fumava, avvicinò a sé un portasigari di tartaruga e ne tirò fuori una sigaretta. 

- Come va la tua salute oggi? - domandò il fratello. 

- Così. I nervi sono come sempre. 

- Non è vero che è straordinariamente bello? - disse Stepan Arkad'ic, notando che Levin guardava di tanto in tanto il ritratto. 

- Non ho veduto ritratto più bello. 

- È straordinariamente somigliante, vero? - disse Vorkuev. 

Levin dal ritratto passò a guardare l'originale. Uno splendore particolare illuminò il viso di Anna nell'attimo in cui ella sentì su di sé il suo sguardo. Levin arrossì, e, per nascondere la propria confusione, voleva domandare se era da molto che non vedeva Dolly ma nello stesso tempo Anna cominciò a parlare. 

- Si parlava or ora con Ivan Petrovic degli ultimi quadri di Vašcenkov. Li avete visti? 

- Sì, li ho visti - rispose Levin. 

- Ma perdonate, vi ho interrotto, volevate dire\ldots{} 

Levin domandò se era molto che ella non vedeva Dolly. 

- È stata ieri da me, è molto arrabbiata per il ginnasio di Griša. Il professore di latino mi pare sia ingiusto con lui. 

- Già, li ho visti questi quadri. Non mi sono piaciuti molto - disse Levin, tornando al discorso cominciato da lei. 

Levin, adesso, non parlava assolutamente più con quel modo tecnico di trattar l'argomento, che aveva usato la mattina. Ogni parola, nella conversazione con lei, acquistava un significato particolare. E parlare con lei era piacevole, ma ancor più piacevole era ascoltarla. 

Anna parlava non solo con naturalezza e intelligenza, ma con intelligenza e noncuranza, senza attribuire nessun peso alle proprie idee, e dando grande importanza alle idee dell'interlocutore. 

Si venne a parlare della nuova tendenza dell'arte, della nuova Bibbia illustrata da un pittore francese. Vorkuev accusava il pittore di un realismo spinto fino alla volgarità. Levin disse che i francesi avevano spinto il convenzionale nell'arte come nessun altro popolo e che perciò vedevano un merito particolare nel ritorno al realismo. Nel fatto di non mentire più, vedevano la poesia. 

Mai ancora nessuna cosa intelligente detta da Levin le aveva fatto tanto piacere come questa. Il viso di Anna si illuminò quando, a un tratto, apprezzò questo pensiero. Ella si mise a ridere. 

- Rido - disse - come si ride quando si vede un ritratto molto somigliante. Quello che avete detto caratterizza perfettamente l'arte francese di adesso, e la pittura e perfino la letteratura: Zola, Daudet. Ma forse, accade sempre così: si costruiscono le proprie conceptions con figure convenzionali inventate, e poi, quando tutte le combinaisons sono tentate, e le figure inventate son venute a noia, allora si cominciano a inventare figure più vicine alla natura, al vero. 

- Ma è proprio giusto! - disse Vorkuev. 

- E allora siete stati al club? - disse lei al fratello. 

``Sì, ecco una donna!'' pensava Levin, dimentico di sé e guardando ostinatamente il viso bello, mobile di lei che adesso s'era d'un tratto del tutto mutato. Levin non sentiva di cosa ella parlasse, dopo essersi tutta piegata verso il fratello, ma fu sorpreso del mutamento della sua espressione. Il viso di lei, prima tanto bello nella calma, espresse a un tratto una strana curiosità, una rabbia e un certo orgoglio. Ma durò solo un momento. Ella socchiuse gli occhi, come per ricordare qualcosa. 

- Eh sì, del resto, questo non interessa nessuno - disse e si rivolse alla inglese: - Please order the tea in the drawing-room. 

La bambina si alzò e uscì. 

- Be', l'ha superato l'esame? - domandò Stepan Arkad'ic. 

- Benissimo. È una bambina di grande ingegno e di carattere simpatico. 

- Andrai a finire che l'amerai più della tua. 

- Ecco un uomo che parla. Nell'amore non c'è più e meno. Amo mia figlia d'un amore, lei d'un altro. 

- Io, ecco, stavo dicendo ad Anna Arkad'evna - disse Vorkuev - che, qualora dedicasse sia pure una centesima parte dell'energia, che adopera per questa inglese, alla causa comune dell'educazione dei bambini russi, Anna Arkad'evna compirebbe un'opera grande e utile. 

- Sì, ecco, che volete, non potevo. Il conte Aleksej Kirillovic mi spingeva molto - pronunciando ``il conte Aleksej Kirillovic'' ella guardò interrogativamente Levin ed egli le rispose involontariamente con uno sguardo rispettoso e affermativo - mi incitava a occuparmi della scuola in campagna. Ci sono andata varie volte. Sono molto carine le bambine, ma non sono riuscita ad affezionarmi a quest'opera. Voi dite: energia. L'energia è fondata sull'amore. E donde prenderlo l'amore? non si può comandarlo. Ecco, ho preso a voler bene a questa bambina, io stessa non so perché. 

Ed ella guardò di nuovo Levin. E il sorriso, e lo sguardo di lei, tutto gli diceva che, soltanto a lui, ella rivolgeva il proprio discorso, prendendone in considerazione l'opinione e nello stesso tempo sapendo già che si capivano scambievolmente. 

- Lo capisco benissimo - rispose Levin. - Alla scuola, e in genere a simili istituzioni, non si può dare il cuore, e penso che, appunto per questo, tali istituzioni filantropiche dànno sempre così scarsi risultati. 

Ella tacque un po', poi sorrise. 

- Sì, sì - confermò. - Io non ho mai potuto. Je n'ai pas le coeur assez large, per mettermi a voler bene a tutto un asilo con delle bambine sporche. Cela ne m'a jamais réussi. Ci sono tante donne che se ne sono fatte una position sociale. E adesso tanto più - disse, rivolgendosi con una triste, confidente espressione, in apparenza al fratello ma evidentemente solo a Levin. - Anche adesso, quando ho tanto bisogno di un'occupazione, non posso. - E accigliatasi a un tratto (Levin capì che s'era accigliata verso se stessa perché parlava di sé), ella cambiò discorso. - So di voi - disse a Levin - che siete un cattivo cittadino, e vi ho difeso come potevo. 

- E come mi avete difeso? 

- Secondo gli assalti. Vi fa piacere un po' di tè? - Ella si alzò e prese in mano un libro rilegato in marocchino. 

- Datemelo, Anna Arkad'evna - disse Vorkuev, indicando il libro. - Ne vale molto la pena. 

- Oh no, è tutto così poco rifinito. 

- L'ho detto a lui - disse Stepan Arkad'ic rivolto alla sorella e indicando Levin. 

- Hai fatto male. Il mio scritto è sul genere di quei cestini intagliati che talvolta mi vendeva Liza Merkalova dalle carceri. Ella si occupava dei carcerati in quella società - disse rivolta a Levin. - E quei disgraziati facevano miracoli di pazienza. 

E Levin scoprì ancora un nuovo tratto di quella donna, che gli era piaciuta in modo così straordinario. Oltre la grazia, l'intelligenza, la bellezza, in lei c'era la sincerità. Ella non voleva nascondergli tutta la difficoltà della propria situazione. Detto questo, ella sospirò, e il suo viso, presa a un tratto una espressione severa, si fece come di pietra. Con un'espressione simile sul viso ella era ancora più bella di prima; quest'espressione era nuova, era al di fuori di quella sfera di espressioni, che splendeva di felicità ed effondeva felicità e che era stata colta dal pittore nel quadro. Levin guardò ancora una volta il quadro e la figura di lei, quando, al braccio del fratello, ella passò con lui attraverso la porta alta, e sentì per lei una tenerezza e una compassione che lo stupirono. 

Ella pregò Levin e Vorkuev di passare in salotto, e lei stessa rimase a parlare di qualcosa col fratello. ``Del divorzio, di Vronskij, di quel ch'egli fa al club, di me'' pensò Levin. E lo agitava tanto la questione di che cosa parlasse con Stepan Arkad'ic, che quasi non ascoltava quello che gli andava dicendo Vorkuev sui pregi del romanzo per ragazzi scritto da Anna Arkad'evna. 

Durante il tè, continuò quella piacevole conversazione, densa di contenuti. Non solo non c'era un attimo in cui bisognasse cercare l'argomento per discorrere, ma, al contrario, si sentiva che non c'era il tempo di dire quello che si voleva, e che volentieri ci si tratteneva ad ascoltare quello che l'altro diceva. E tutto quello che dicevano, non soltanto lei, ma Vorkuev, Stepan Arkad'ic, tutto acquistava un'importanza particolare, come pareva a Levin, grazie all'attenzione e alle osservazioni fatte da lei. 

Seguendo la conversazione interessante, Levin tutto il tempo ammirava lei e la sua bellezza e intelligenza e la cultura e insieme la semplicità e la cordialità. Egli ascoltava, parlava e tutto il tempo pensava a lei, alla sua vita interiore, cercando di indovinare i suoi sentimenti. E lui che prima la rimproverava così duramente, adesso, per un certo strano corso di pensieri, la giustificava e insieme la compiangeva, e temeva che Vronskij non la capisse in pieno. Dopo le dieci, quando Stepan Arkad'ic si alzò per andar via (Vorkuev era andato via prima), a Levin parve d'esser giunto allora. E si alzò anche lui, con rammarico. 

- Addio - ella disse, trattenendolo per la mano e guardandolo negli occhi con uno sguardo pieno di fascino. - Sono molto contenta que la glace est rompue. 

Lasciò andare la mano di lui e socchiuse gli occhi. 

- Dite a vostra moglie che le voglio bene come prima e che, se ella non mi può perdonare la mia situazione, allora le auguro di non perdonarmi mai. Per perdonare bisogna passare quello che ho passato io, e da questo la salvi Iddio. 

- Certamente, sì, lo dirò\ldots{} - diceva Levin, arrossendo. 

\capitolo{XI}\label{xi-6} 

``Che donna straordinaria, simpatica, degna di compassione!'' egli pensava, uscendo con Stepan Arkad'ic all'aria gelida. 

- Be', via, te l'avevo detto - gli disse Stepan Arkad'ic, vedendo che Levin era completamente conquistato. 

- Sì, rispose Levin - una donna straordinaria. Non solo intelligente, ma tanto cordiale! Fa una gran pena! 

- Adesso, con l'aiuto di Dio, presto s'accomoderà tutto. Eh sì, non giudichiamo prima del tempo - disse Stepan Arkad'ic, aprendo lo sportello della carrozza. - Addio, non facciamo la stessa strada. 

Senza smettere di pensare ad Anna, a tutti i discorsi più semplici che c'erano stati con lei, e ricordando intanto tutti i particolari dell'espressione del suo viso, sempre più immedesimandosi nella posizione di lei e provandone pena, Levin giunse a casa. 

A casa, Kuz'ma riferì a Levin che Katerina Aleksandrovna stava bene, e che solo da poco erano andate vie le sorelle dalla signora e gli diede due lettere. Levin, proprio lì in anticamera, per non dimenticarsene poi, le lesse. Una era di Sokolov, l'amministratore. Sokolov scriveva che il frumento non si poteva vendere, offrivano soltanto cinque rubli e mezzo, ma il denaro non si sapeva più dove prenderlo. L'altra lettera era della sorella. Lo rimproverava che il suo affare non si fosse ancora risolto. 

``E vendiamo per cinque rubli e mezzo, se non ne dànno di più - decise immediatamente Levin, con straordinaria facilità, la prima questione, che in altri tempi gli sarebbe parsa così difficile. - È sorprendente come io qui sia sempre occupato'' pensò a proposito della seconda lettera. Si sentiva colpevole di fronte alla sorella perché finora non aveva fatto quello di cui ella lo aveva pregato. ``Oggi di nuovo non sono andato in tribunale, ma oggi poi non c'era proprio tempo''. E, stabilito che l'avrebbe certamente fatto l'indomani, andò da sua moglie. Andando da lei, Levin percorse rapidamente col pensiero tutto quello che era passato in quel giorno. Tutta la giornata era passata in discorsi: discorsi che aveva ascoltato o ai quali aveva partecipato. Erano tutti su argomenti tali che, se lui fosse stato solo e in campagna, non se ne sarebbe mai occupato, ma qui erano molto interessanti. E tutti i discorsi andavano bene; solo in due punti non andavano bene. L'uno era quello in cui lui aveva parlato del luccio, l'altro quello in cui qualcosa non andava nella tenera pietà ch'egli aveva provato per Anna. 

Levin trovò la moglie triste e annoiata. Il pranzo con le sorelle sarebbe riuscito allegro, ma avevano aspettato lui, l'avevano aspettato tanto, poi tutte avevano preso ad annoiarsi, le sorelle se n'erano andate e lei era rimasta sola. 

- Be', e tu cosa hai fatto? - ella domandò, guardandolo negli occhi che brillavano in maniera particolarmente sospetta. Ma, per non impedirgli di raccontare tutto, ella nascose la propria osservazione e ascoltò con un sorriso d'approvazione il racconto di com'egli aveva passato la serata. 

- Sai, sono stato molto contento d'avere incontrato Vronskij. Mi sono comportato con disinvoltura e semplicità con lui. Capisci, ormai cercherò di non incontrarmi mai più con lui, ma perché questo disagio finisse - disse, ma, ricordatosi che egli, cercando di non incontrarsi mai più, era andato immediatamente da Anna, arrossì. - Ecco, noi diciamo che il popolo beve; non so chi beva di più, il popolo o la nostra classe; il popolo almeno alla festa, ma\ldots{} 

Ma a Kitty non interessava affatto ragionar su come bevesse il popolo. Ella vedeva ch'egli arrossiva e desiderava sapere perché. 

- E poi dove sei stato mai? 

- Stiva mi ha supplicato con straordinaria insistenza di andare da Anna Arkad'evna . 

E, detto questo, Levin arrossì ancora di più, e i suoi dubbi sul fatto se avesse fatto bene o male ad andare da Anna furono definitivamente risolti. Adesso sapeva che non avrebbe dovuto farlo. 

Gli occhi di Kitty si dischiusero e scintillarono in modo particolare al nome di Anna, ma fatto uno sforzo su di sé, ella nascose la propria agitazione e simulò. 

- Ah! - disse soltanto. 

- Tu, non è vero, non ti dispiacerai ch'io sia andato. Stiva mi ha pregato e Dolly lo desiderava - continuò Levin\ldots{} 

- Oh no - disse, ma nei suoi occhi egli vedeva uno sforzo su di sé che non gli prometteva nulla di buono. 

- È una donna molto simpatica, molto, molto da compatire, buona - egli diceva, parlando di Anna, delle sue occupazioni e riferendo quello di cui lo aveva incaricato. 

- Già, s'intende, è molto da compatire - disse Kitty, quando egli ebbe finito. - Da chi mai hai ricevuto una lettera? 

Egli glielo disse e, credendo al suo tono calmo, andò a spogliarsi. 

Tornato, trovò Kitty sulla stessa poltrona. Quando le si avvicinò, ella lo guardò e scoppiò in singhiozzi. 

- Cosa, cosa? - egli domandava, sapendo già da prima cosa. 

- Tu ti sei innamorato di quella donna disgustosa, ti ha affascinato. L'ho visto nei tuoi occhi. Sì, sì! E che ne può venir fuori? Al club hai bevuto, bevuto, hai giocato e poi sei andato\ldots{} da chi? No, partiamo\ldots{} domani io parto. 

A lungo Levin non poté calmare la moglie. Finalmente la calmò quando le confessò che il senso di pena unito al vino l'avevano messo fuor di strada e ch'egli aveva ceduto all'abile influenza di Anna, e che l'avrebbe evitata. La cosa che egli confessava più sinceramente era che, vivendo così a lungo a Mosca, passando il tempo soltanto a far discorsi, a mangiare e a bere, era divenuto un insensato. Parlarono fino alle tre di notte. Soltanto alle tre si rappacificarono e riuscirono ad addormentarsi. 

\capitolo{XII}\label{xii-6} 

Accompagnati gli ospiti, Anna, invece di sedersi, si mise ad andare avanti e indietro per la stanza. Sebbene inconsciamente (come sempre in quegli ultimi tempi nei riguardi di tutti gli uomini giovani), per tutta la sera avesse cercato, con ogni mezzo, di destare in Levin un sentimento d'amore verso di lei, e sebbene sapesse d'averlo ottenuto per quanto è possibile da parte di un uomo onesto ammogliato e in una sola serata, e sebbene egli le fosse piaciuto molto (malgrado la netta differenza, dal punto di vista di un uomo, fra Vronskij e Levin, lei, come donna, vedeva in loro quello stesso lato comune, per cui Kitty aveva amato e Vronskij e Levin), non appena egli fu uscito dalla stanza, cessò di pensare a lui. 

Un unico pensiero, sotto vari aspetti, la perseguitava con insistenza. ``Se io agisco così sugli altri, su quest'uomo che ha famiglia, che ama, perché mai lui è così freddo verso di me?\ldots{} Non che sia freddo, mi ama, lo so. Ma qualcosa di nuovo adesso ci divide. Come mai non viene per tutta la sera? Ha fatto dire da Stiva che non poteva lasciare Jašvin e che doveva sorvegliare il suo gioco. È forse un bambino Jašvin? Ma ammettiamo che sia vero. Egli non dice mai una cosa non vera. Ma in questa verità c'è qualcos'altro. È contento dell'occasione per dimostrami che lui ha altri doveri. Lo so, sono d'accordo su questo. Ma perché dimostrarmelo? Egli vuole dimostrarmi che il suo amore per me non deve intralciare la sua libertà. Ma io non ho bisogno di dimostrazioni, ho bisogno d'amore. Egli dovrebbe capire tutta la difficoltà della mia vita qui, a Mosca. Vivo io forse? Non vivo, ma aspetto lo scioglimento che si trascina e si trascina sempre. Di nuovo non c'è risposta. E Stiva dice che lui non può andare da Aleksej Aleksandrovic. Ma io non posso scrivere ancora. Io non posso fare nulla, non posso cominciar nulla, mutar nulla; mi trattengo, aspetto, mi invento dei passatempi, con la famiglia dell'inglese, con lo scrivere, e il leggere, ma tutto questo è soltanto un inganno, tutto questo è lo stesso della morfina. Egli dovrebbe aver pena di me'' ella diceva, sentendosi venire agli occhi lacrime di compassione verso se stessa. 

Sentì la scampanellata violenta di Vronskij e asciugò in fretta queste lacrime; non solo asciugò le lacrime, ma sedette vicino alla lampada e aprì un libro, fingendosi calma. Bisognava mostrargli ch'era scontenta ch'egli non fosse tornato così come aveva promesso, soltanto scontenta, ma non fargli vedere in nessun modo il proprio dolore, e, soprattutto, la compassione verso se stessa. Lei poteva avere pena di se stessa, ma non lui di lei. Ella non voleva la lotta, lo rimproverava perché egli voleva lottare, e senza volere si poneva lei stessa in posizione di lotta. 

- Be', non ti sei annoiata? - disse egli con animazione e allegramente, avvicinandosi a lei. - Che passione tremenda è il giuoco! 

- No, non mi sono annoiata e già da tempo ho imparato a non annoiarmi. Sono stati qui Stiva e Levin. 

- Già, volevano venire da te. Be', t'è piaciuto Levin? - disse, sedendosi accanto a lei. 

- Molto. Sono andati via da poco. E che ha fatto Jašvin? 

- Era in vincita, diciassettemila rubli. Io lo chiamavo, lui era proprio sul punto d'andar via. Ma è tornato di nuovo, e adesso è in perdita. 

- E allora perché sei rimasto? - ella domandò, levando a un tratto gli occhi su di lui. L'espressione del viso di lei era fredda e ostile. - Hai detto a Stiva che saresti rimasto per portar via Jašvin. E l'hai pure lasciato. 

La stessa espressione di freddezza, di fronte alla lotta, apparve anche sul viso di lui. 

- In primo luogo non gli ho chiesto di dirti niente; in secondo luogo, io non dico mai quello che non è vero. E soprattutto, volevo rimanere e son rimasto - disse egli, aggrottando le sopracciglia. - Anna, perché, perché? - diss'egli dopo un istante di silenzio, piegandosi verso di lei, e aprì la mano, sperando ch'ella vi mettesse la sua. 

Ella era contenta di questo invito alla tenerezza. Ma una certa strana forza maligna non le permetteva di abbandonarsi alla sua inclinazione, come se le condizioni della lotta non le permettessero di sottomettersi. 

- S'intende, tu volevi rimanere e sei rimasto. Tu fai tutto quello che vuoi. Ma perché non lo dici? Per che cosa? - ella diceva, accalorandosi sempre più. - Qualcuno contesta forse i tuoi diritti? Ma tu vuoi avere ragione e abbi pure ragione. 

La mano di lui si chiuse, egli si allontanò, e il suo viso prese un'espressione ancora più caparbia di prima. 

- Per te è questione di ostinazione - diss'ella, dopo averlo guardato fisso e trovando, a un tratto, un nome a quest'espressione del viso che l'irritava - proprio di ostinazione. Per te la questione è se la vincerai con me, ma per me\ldots{} - Di nuovo le venne pena di sé, e si mise quasi a piangere. - Se tu sapessi di che si tratta per me! Quando io sento, come adesso, che tu mi tratti con ostilità, sì proprio con ostilità, se tu sapessi cosa significa questo per me! Se sapessi come sono vicina a una sciagura in questi momenti, come ho paura, come ho paura di me! - e si voltò dall'altra parte per nascondere i singhiozzi. 

- Per quale ragione fai così? - disse lui, spaventato dinanzi a quell'espressione disperata, dopo essersi di nuovo piegato verso di lei e averle preso e baciato la mano. - Perché? Cerco forse delle distrazioni fuori casa? Non evito forse la compagnia delle donne? 

- Ci mancherebbe altro! - disse lei. 

- Dimmelo, via: che devo fare perché tu sia tranquilla? Io sono pronto a tutto purché tu sia felice - egli diceva, commosso dalla disperazione di lei - e che cosa non farei per liberarti da qualsiasi dolore, come adesso, Anna! - diss'egli. 

- Nulla, nulla - ella disse. - Io stessa non so: forse la vita solitaria, i nervi\ldots{} Ma non ne parliamo più. Come sono andate le corse? non me ne hai parlato - domandò cercando di nascondere il trionfo della vittoria che, comunque, era dalla parte sua. 

Egli chiese di cenare e cominciò a raccontarle i particolari delle corse; ma nel tono, negli sguardi di lui, che si facevano sempre più freddi, ella vedeva ch'egli non le aveva perdonato la vittoria, che quel sentimento di ostinazione, contro cui ella aveva lottato, si stabiliva di nuovo in lui. Egli era più freddo di prima con lei, come pentito d'essersi sottomesso. E lei, ricordando le parole che le avevano dato la vittoria, e precisamente ``io son vicina a un'orribile sciagura e ho paura di me stessa'', capì che quest'arma era pericolosa e che non si sarebbe potuto adoperarla una seconda volta. Ma sentiva che, insieme all'amore che li legava, s'era stabilito fra di loro lo spirito perverso di una certa lotta ch'ella non poteva scacciare né dal cuore di lui, né tanto meno dal proprio. 

\capitolo{XIII}\label{xiii-6} 

Non ci sono condizioni tali a cui l'uomo non possa abituarsi, in particolare se vede che tutti quelli che lo circondano vivono allo stesso modo. Levin tre mesi prima non avrebbe creduto di potersi addormentare tranquillamente nelle condizioni in cui era quel giorno; vivendo una vita senza scopo, senza senso, inoltre una vita al di sopra delle proprie possibilità, dopo una ubriacatura (non poteva chiamare diversamente quello che c'era stato al club), i rapporti stranamente amichevoli con un uomo di cui una volta era stata innamorata sua moglie, e la visita ancor più strana a una donna che non si poteva chiamare altrimenti che perduta, e dopo il proprio entusiasmo per questa donna e il dolore della moglie, non sarebbe stato possibile addormentarsi tranquillamente in queste condizioni. Eppure, preso dalla stanchezza, dalla notte insonne e dal vino bevuto, si addormentò profondamente, tranquillo. 

Alle cinque lo scricchiolio di una porta che si apriva lo svegliò. Saltò su e si girò. Kitty non era nel letto accanto a lui. Ma di là dal tramezzo c'era una luce che si moveva, ed egli sentì i passi di lei. 

- Che c'è, che c'è? - egli pronunciò nel sonno. - Kitty, cosa c'è? 

- Nulla - ella disse, uscendo fuori dal tramezzo con una candela in mano. - Non mi sentivo bene - disse, sorridendo di un sorriso particolarmente grazioso e significativo. 

- Cos'è? è cominciato? cominciato? - egli disse con spavento - bisogna mandare a chiamare - e prese a vestirsi in fretta. 

- No, no - ella disse, sorridendo e trattenendolo con la mano. - Probabilmente non è nulla. Mi sentivo indisposta, ma solo un poco. Adesso è passato. 

E, avvicinatasi al letto, spense la candela, si coricò e si calmò. Sebbene lo tenessero in ansia il silenzio del respiro di lei, quasi trattenuto e, più di tutto, l'espressione di particolare tenerezza ed eccitazione con cui, uscendo dal tramezzo, ella gli aveva detto: ``nulla'', egli aveva tanto sonno che si riaddormentò immediatamente. Soltanto dopo ricordò la sospensione del respiro di lei e capì tutto quello che era accaduto nella cara, gentile anima sua quando, senza muoversi, nell'attesa dell'avvenimento più grande nella vita d'una donna, ella si era coricata accanto a lui. Alle sette, lo svegliò il contatto della mano di lei sulla spalla e un sussurro sommesso. Era come s'ella lottasse fra il dispiacere di svegliarlo e il desiderio di parlare con lui. 

- Kostja, non spaventarti. Non è nulla. Ma mi pare\ldots{} Bisogna andare a chiamare Elizaveta Petrovna. 

La candela era accesa di nuovo. Ella era seduta sul letto e teneva in mano il lavoro a maglia di cui si occupava negli ultimi giorni. 

- Per favore, non spaventarti, non è nulla. Io non ho paura per niente - disse, vedendo il viso spaventato di lui, e premette la mano di lui al proprio petto, poi alle labbra. 

Egli saltò su in fretta, senza aver coscienza di sé e, senza levarle gli occhi di dosso, infilò la vestaglia e si fermò, guardandola sempre. Bisognava andare, ma non poteva staccare lo sguardo da lei. O che non gli piacesse quel suo viso, o che non ne conoscesse l'espressione, lo sguardo, certo non l'aveva mai vista così. Come si vedeva disgustoso e detestabile nel ricordare il dolore da poco arrecatole, dinanzi a lei così com'era adesso! Il suo viso, divenuto vermiglio, circondato dai capelli morbidi, di sotto alla cuffia da notte, splendeva di gioia e di risolutezza. 

Per quanto poca fosse la finzione e la convenzionalità nel carattere di Kitty, Levin tuttavia fu sorpreso da quello che ora gli si metteva a nudo, ora che, tolti a un tratto tutti i veli, l'essenza stessa dell'anima di lei le splendeva negli occhi. E in quella semplicità e nudità, lei, proprio quella che egli amava, si vedeva ancora meglio. Lo guardava sorridendo; ma a un tratto le tremarono le sopracciglia, levò il capo, e, avvicinatasi rapida, lo prese per mano e si strinse tutta a lui, inondandolo del proprio respiro caldo. Ella soffriva ed era come se si lamentasse con lui delle proprie sofferenze. E a lui, nel primo momento, per abitudine, parve d'essere colpevole. Ma nello sguardo di lei c'era una tenerezza la quale diceva ch'ella non solo non lo rimproverava, ma lo amava per quelle sofferenze. ``Se non sono io, chi mai è colpevole di questo?'' egli pensò senza volere, cercando il colpevole di quelle sofferenze per punirlo; ma un colpevole non c'era. Ella soffriva, si lamentava ed esultava di quelle sofferenze, ne gioiva e le amava. Egli vedeva che nell'anima di lei si compiva qualcosa di splendido, ma cosa, egli non poteva capirlo. Era al di sopra della sua comprensione. 

- Io mando a chiamar la mamma. E tu va' in fretta a prendere Lizaveta Petrovna\ldots{} Kostja!\ldots{} Non è nulla, è passato. 

Si allontanò da lui e sonò. 

- Su, ecco, adesso va'; viene Paša. Io mi sento abbastanza bene. 

E Levin vide, con stupore, ch'ella riprendeva il lavoro a maglia che aveva portato la notte e si metteva di nuovo a sferruzzare. 

Mentre Levin usciva da una porta, sentì che dall'altra entrava una donna. Si fermò vicino alla porta e sentì Kitty che dava ordini particolareggiati alla donna e che si metteva a spostare il letto con lei. 

Egli si vestì e, mentre attaccavano i cavalli, poiché vetture da nolo non ce n'erano ancora, rientrò di corsa nella stanza da letto, non in punta di piedi, ma sulle ali, come gli sembrava. Due donne cambiavano qualcosa di posto, con precauzione, nella stanza; Kitty camminava e sferruzzava, mettendo rapidamente le maglie sui ferri e dando ordini. 

- Io vado subito dal dottore. A chiamare Lizaveta Petrovna sono già andati, ma io ci passerò ancora. Non c'è bisogno di nulla? Sì, da Dolly? 

Ella lo guardò, evidentemente, senza ascoltare quello ch'egli diceva. 

- Sì, sì, va', va' - disse in fretta, accigliandosi e facendogli un gesto con la mano. 

Egli usciva già in salotto, quando a un tratto un gemito penoso, immediatamente chetatosi, echeggiò dalla stanza da letto. Egli si fermò, e a lungo non poté capire. 

``Sì, è lei'' si disse e, messosi la testa fra le mani, corse giù. 

``Signore abbi pietà, perdona, aiuta!'' egli ripeteva le parole che gli venivano, chi sa come, a un tratto, alle labbra. E lui, persona scettica, ripeteva quelle parole non con le labbra sole. Adesso, in questo momento, egli sapeva che non solo tutti i suoi dubbi, ma quell'impossibilità di credere secondo ragione che conosceva in sé, non gli impedivano per nulla di rivolgersi a Dio. Tutto questo, adesso, era volato via dall'anima sua come polvere. A chi mai rivolgersi, se non a Colui nelle cui mani egli sentiva se stesso, l'anima sua e il suo amore? 

Il cavallo non era ancora pronto, ma, sentendo in sé un particolare tendersi delle forze fisiche e dell'attenzione verso quello che bisognava fare, per non perdere neanche un minuto, senza aspettare il cavallo, uscì a piedi e ordinò a Kuz'ma di raggiungerlo. 

All'angolo incontrò una vettura da nolo notturna che andava in fretta. Nella piccola slitta, in cappa di velluto, con la testa ravvolta in un fazzoletto, sedeva Lizaveta Petrovna. ``Sia lodato Iddio, sia lodato Iddio!'' egli pronunciò, riconoscendo con entusiasmo il piccolo viso biondo di lei, che adesso aveva un'espressione particolarmente seria, perfino severa. Senza fermare la vettura, egli le corse dietro. 

- Allora, un due ore? Non di più? - ella domandò. - Troverete Pëtr Dmitric, ma non lo sollecitate. E prendete dell'oppio in farmacia. 

- Allora, voi pensate che tutto possa andar bene? Signore, abbi pietà e aiutami! - esclamò Levin, dopo aver visto che il suo cavallo usciva dal portone. Saltato nella slitta a fianco di Kuz'ma, ordinò di andare dal dottore. 

\capitolo{XIV}\label{xiv-6} 

Il dottore ancora non s'era alzato, e il cameriere disse che ``il signore s'era coricato tardi e aveva ordinato di non svegliarlo, ma che presto si sarebbe alzato''. Il cameriere puliva i vetri di una lampada e sembrava molto occupato in questa faccenda. Questa attenzione del cameriere per i vetri e l'indifferenza per quello che avveniva in lui, Levin, lo sorpresero dapprima; ma subito, riflettendo, capì che nessuno sapeva né era obbligato a sapere i suoi sentimenti e che tanto più bisognava agire con calma, con riflessione e risolutezza, per sfondare quel muro di indifferenza e raggiungere il proprio scopo. ``Non aver fretta e non trascurare nulla'' si diceva Levin, sentendo un sollevarsi sempre crescente di forze fisiche e di tensione per tutto quello che c'era da fare. 

Saputo che il dottore non s'era ancora alzato, Levin, fra i vari piani che gli si presentarono, si fermò sul seguente: che Kuz'ma andasse con un biglietto da un altro dottore mentre lui sarebbe andato in farmacia a prendere l'oppio, e se, al suo ritorno, il dottore non si fosse alzato ancora, o corrompendo il cameriere o con la forza, se non avesse consentito, avrebbe svegliato il dottore a qualunque costo. 

In farmacia uno sparuto aiutante farmacista, con la stessa indifferenza con cui il cameriere puliva i vetri, suggellava con un'ostia le polverine per un cocchiere che aspettava, e rifiutò l'oppio. Cercando di non avere fretta e di non accalorarsi, fatto il nome del dottore e quello della levatrice, e spiegato perché serviva l'oppio, Levin cominciò a persuaderlo. L'aiutante chiese consiglio in tedesco se dovesse darlo, e, ricevutane da dietro il tramezzo l'approvazione, tirò fuori una fiala e un imbuto; versò lentamente dal recipiente grande nel piccolo, incollò l'etichetta, suggellò, malgrado la preghiera di Levin di lasciar stare e voleva ancora avvolgerla. Questo Levin non poté più sopportarlo: gli strappò decisamente dalle mani la fiala e corse via per la grande porta a vetri. Il dottore non s'era ancora alzato e il cameriere, occupato adesso a stendere un tappeto, si rifiutò di svegliarlo. Levin, senza affrettarsi, tirò fuori un biglietto da dieci rubli, e, pronunciando piano le parole, ma anche senza perder tempo, gli tese il biglietto e spiegò che Pëtr Dmitric (come sembrava a Levin grande e importante Pëtr Dmitric, finora così trascurabile!) aveva promesso di venire in qualunque momento, che certo non si sarebbe arrabbiato e che perciò lo svegliasse subito. 

Il cameriere accettò, andò di sopra e fece entrare Levin nella sala di ricevimento. 

Levin sentiva, dietro la porta, il dottore che tossiva, camminava, si lavava e diceva qualcosa. Passarono circa tre minuti; a Levin parve che fosse passata un'ora. Non poteva più aspettare. 

- Pëtr Dmitric, Pëtr Dmitric - cominciò a dire, attraverso la porta aperta, con voce supplichevole. - In nome di Dio, scusatemi. Ricevetemi così come siete. Son già più di due ore. 

- Subito, subito! - rispondeva la voce, e Levin sentiva con meraviglia che il dottore diceva questo sorridendo. 

- Per un momento solo\ldots{} 

- Subito. 

Passarono ancora due minuti, prima che il dottore avesse infilato gli stivali e altri due minuti, prima che il dottore avesse messo il vestito e si fosse pettinato il capo. 

- Pëtr Dmitric! - cominciò di nuovo la voce pietosa di Levin, ma intanto venne fuori il dottore, vestito e pettinato. ``Non hanno coscienza queste persone - pensò Levin. - Pettinarsi, mentre noi moriamo''. 

- Buon giorno! - gli disse il dottore, dandogli la mano e quasi stuzzicandolo con la propria calma. - Non abbiate fretta. Be'? 

Cercando d'essere il più preciso possibile, Levin cominciò a raccontare i particolari inutili sullo stato della moglie, interrompendo di continuo il racconto con preghiere, perché il dottore andasse immediatamente con lui. 

- Ma non abbiate fretta. Voi non sapete nulla. Io certamente non sono necessario, ma ho promesso, e, vi assicuro, verrò. Ma fretta non ce n'è. Sedetevi per favore; non desiderate forse del caffè? 

Levin lo guardò, domandando con lo sguardo s'egli volesse prendersi giuoco di lui. Ma il dottore non pensava neppure a scherzare. 

- Lo so, lo so - disse il dottore, sorridendo - io stesso ho famiglia; ma noi mariti, in questi momenti, siamo le persone più pietose. Io ho una paziente che ha un marito il quale, quando comincia questo, se ne scappa nella scuderia. 

- Ma voi cosa pensate, Pëtr Dmitric? Credete che possa andar tutto bene? 

- Tutti i sintomi sono per un esito felice. 

- E allora, verrete subito? - disse Levin, guardando ostilmente il cameriere che entrava portando il caffè. 

- Fra un'oretta. 

- No, in nome di Dio! 

- Be', almeno lasciatemi bere il caffè. 

Il dottore si mise a sorbire il caffè. Tutti e due tacquero alquanto. 

- Però i turchi si battono risolutamente. Avete letto il comunicato di ieri? - disse il dottore, masticando un panino. 

- No, non posso! - disse Levin, saltando su. - Allora fra un quarto d'ora sarete da noi? 

- Fra mezz'ora. 

- Parola d'onore? 

Quando Levin tornò a casa, s'incontrò con la principessa e si accostarono insieme alla porta della stanza da letto. La principessa aveva le lacrime agli occhi, e le tremavano le mani. Visto Levin, lo abbracciò e si mise a piangere. 

- Ebbene, carissima Lizaveta Petrovna? - disse, afferrando per un braccio Lizaveta Petrovna ch'era uscita loro incontro con un viso luminoso e preoccupato. 

- Va bene - ella disse - persuadetela a coricarsi. Si sentirà meglio. 

Dal momento in cui s'era svegliato e aveva capito di che cosa si trattava, Levin s'era disposto a sopportare tutto quello cui andava incontro, senza riflettere, senza prevedere nulla, fermando tutti i pensieri e i sentimenti, con forza, senza agitare la moglie, al contrario, calmandola e sostenendone il coraggio. Senza concedersi neppure di pensare quello che sarebbe avvenuto, ma come sarebbe finito, giudicando dall'inchiesta che aveva svolto su quanto durava di solito la cosa, Levin, nella sua immaginazione, s'era preparato a pazientare e a tenere il cuore in mano per circa cinque ore, e questo gli sembrava possibile. Ma quando tornò dal dottore e vide di nuovo le sofferenze di lei, si mise a ripetere più spesso: ``Signore, perdona, aiuta'', a sospirare e a levare la testa in alto; e provò il terrore di non poter sopportare, d'essere costretto a piangere o a fuggire; tanto tormento provava. Ed era passata soltanto un'ora. Ma dopo quest'ora, ne passò ancora un'altra, poi ne passarono due, tre, tutte e cinque le ore, e le cose erano sempre allo stesso punto; e lui sopportava ancora, perché non c'era più niente da fare se non pazientare, pensando, ogni momento, d'essere giunto al limite della sopportazione e che il cuore, subito, da un momento all'altro, si sarebbe spezzato dalla pena. 

Ma passavano ancora minuti, ore e ancora ore, e i suoi sentimenti di pena e di angoscia crescevano e si tendevano ancora di più. 

Tutte quelle solite condizioni di vita, senza le quali non ci si può immaginare nulla, non esistevano più per Levin. Egli aveva perduto la nozione del tempo. A tratti i minuti, quei minuti in cui lei lo chiamava accanto a sé ed egli le teneva la mano sudata, che ora stringeva con una forza straordinaria, ora respingeva, gli sembravano ore; a tratti invece le ore gli sembravano minuti. Fu stupito quando Lizaveta Petrovna lo pregò di accendere una candela dietro il paravento e venne a sapere che erano le cinque di sera. Se gli avessero detto che erano soltanto le dieci del mattino, non si sarebbe stupito maggiormente. Dove fosse in quel momento, lo sapeva così poco come poco sapeva quando ciò sarebbe avvenuto. Vedeva il viso infiammato di lei, che a momenti era ansioso e sofferente, a momenti gli sorrideva, rasserenandolo. Vedeva anche la principessa, rossa, indaffarata, coi riccioli di capelli bianchi disfatti e con le lacrime ch'ella si affrettava a inghiottire, mordendosi le labbra; vedeva Dolly e il dottore che fumava delle grosse sigarette, e Lizaveta Petrovna dal viso deciso e forte e rasserenante, e il vecchio principe che passeggiava per la sala accigliato. Ma come essi venissero e uscissero, dove fossero, non lo sapeva. La principessa era ora col dottore nella stanza da letto, ora nello studio, dove si trovava una tavola imbandita; ora non c'era lei, ma Dolly. Poi Levin ricordava che l'avevano mandato chi sa dove. Una volta a trasportare un divano e una tavola. L'aveva fatto con cura, pensando che questo fosse necessario per lei, e soltanto dopo seppe che era stato preparato un letto per lui. Poi lo avevano mandato nello studio, dal dottore, a chiedere qualcosa. Il dottore aveva risposto, ma poi s'era messo a parlare sui disordini del consiglio di stato. Poi l'avevano mandato nella camera della principessa a prendere un'icona dalla cornice d'argento, e lui, insieme con la vecchia cameriera della principessa, s'era arrampicato su di un armadio per raggiungerla e aveva rotto la lampada, e la cameriera della principessa l'aveva tranquillizzato per la moglie e per la lampada, e lui aveva portato l'icona e l'aveva posta al capezzale di Kitty, ficcandola con cura dietro i guanciali. Ma dove, quando e perché avvenisse tutto questo, non lo sapeva. Non capiva neppure perché la principessa lo prendesse per mano, e, guardandolo pietosamente, lo pregasse di calmarsi, e Dolly lo persuadesse di mangiare un po' e lo portasse via dalla stanza, e perfino il dottore lo guardasse seriamente e con compassione, mentre gli offriva delle gocce. 

Egli sapeva e sentiva soltanto che quello che avveniva era simile a quello che s'era compiuto un anno prima nell'albergo della città di provincia, sul letto di morte di suo fratello Nikolaj. Ma quello era un dolore, questa una gioia. Ma sia quel dolore che questa gioia erano egualmente al di fuori di tutte le solite contingenze della vita ed erano, nella consuetudine della vita, come uno spiraglio attraverso il quale appariva qualcosa di ultraterreno. Ed egualmente con pena, con tormento avanzava quello che si compiva, ed egualmente in modo impenetrabile dinanzi a questo avvenimento di ordine superiore, l'anima si sollevava a un'altezza quale egli mai prima aveva neppure concepito e dove la ragione ormai non poteva tenerle dietro. 

``Signore, perdona e aiuta'' egli ripeteva continuamente, sentendo, malgrado il suo lungo e, in apparenza, completo allontanamento, di rivolgersi a Dio con la stessa confidenza e semplicità del tempo della fanciullezza e della prima gioventù. 

In tutto questo periodo ebbe due stati d'animo distinti. Uno, fuori della presenza di lei, col dottore, che fumava, una dopo l'altra, grosse sigarette e le spegneva contro l'orlo del portacenere pieno, con Dolly e col principe quando, mentre parlava del pranzo, di politica, della malattia di Mar'ja Petrovna, Levin a un tratto dimenticava completamente, per un poco, quello che accadeva, e si sentiva come risvegliato; e l'altro stato d'animo, in presenza di lei, al suo capezzale, quando il cuore voleva spezzarsi, eppure non si spezzava dalla pietà, ed egli pregava Dio continuamente. E ogni volta che un grido, giungendo dalla camera, lo tirava fuori da un momento di oblio, egli cadeva sempre in quello strano smarrimento che gli era piombato addosso il primo momento: ogni volta, sentito il grido, saltava su, correva a scolparsi, si ricordava per via che non era colpevole, e voleva proteggere, aiutare. Ma guardando lei, di nuovo vedeva che non si poteva darle aiuto, e cadeva nel terrore e diceva: ``Signore, perdona e aiuta''. E quanto più tempo passava, più forti si facevano tutti e due gli stati d'animo: tanto più calmo egli diveniva, dimenticandola completamente, lontano da lei, e tanto più tormentose diventavano e le sofferenze di lei e la sensazione d'impotenza di fronte ad esse. Egli saltava su, voleva correre via, in qualche parte, e correva da lei. 

A volte, quand'ella lo chiamava ancora e poi ancora, egli accusava lei. Ma, vedendo quel viso docile e sorridente e sentendo le parole ``Ti sto tormentando'', egli ne accusava Dio, e, ricordandosi di Dio, lo pregava subito di perdonare e di avere pietà. 

\capitolo{XV}\label{xv-6} 

Non sapeva se era tardi, se era presto. Le candele stavano tutte per spegnersi. Dolly era stata allora allora nello studio e aveva proposto al dottore di coricarsi un po'. Levin sedeva, ascoltando i racconti del dottore su di un ciarlatano ipnotizzatore e guardava la cenere della sua sigaretta. Era in un momento di tregua, come smemorato. Aveva completamente dimenticato quel che accadeva, adesso. Ascoltava il racconto del dottore e lo seguiva. A un tratto echeggiò un grido, che non era simile a nulla. Il grido era così terribile che Levin non saltò neanche su, ma, senza respirare, guardò il dottore con uno spavento interrogativo. Il dottore piegò il capo da un lato, ascoltando, e sorrise con approvazione. Tutto era così fuori dell'ordinario, che Levin non si stupiva più di nulla. ``Probabilmente deve essere così'' pensò e continuò a stare seduto. Di chi era quel grido? Saltò su, entrò di corsa in punta di piedi nella stanza da letto, sorpassò Lizaveta Petrovna, la principessa e si pose al suo posto, al capezzale. Il grido era finito, ma adesso qualcosa era cambiato. Cosa, non lo vedeva e non lo capiva e non voleva capirlo. Ma lo vedeva dal viso di Lizaveta Petrovna: il viso di Lizaveta Petrovna era severo e pallido e sempre egualmente deciso, sebbene le mascelle le tremassero un poco e i suoi occhi fossero diretti con fissità su Kitty. Il volto di Kitty, infiammato e sfinito, con una ciocca di capelli appiccicata al viso sudato, era rivolto verso di lui e cercava il suo sguardo. Le mani alzate chiedevano le sue mani. Afferrate con le mani sudate le mani fredde di lui, ella si mise a premerle contro il proprio viso. 

- Non te ne andare, non te ne andare! Io non ho paura, non ho paura! - diceva in fretta. - Mamma, toglietemi gli orecchini. Mi dànno noia. Tu non hai paura. Presto, presto, Lizaveta Petrovna. 

Ella parlava in fretta, molto in fretta e voleva sorridere. Ma, a un tratto, il suo viso si alterò, ella lo respinse da sé. 

- No, è tremendo! Morirò, morirò. Va', va'! - ella gridò e sentì di nuovo quello stesso grido che non era simile a nulla. 

Levin si afferrò la testa tra le mani e fuggì dalla camera. 

- Nulla, nulla, va tutto bene - gli disse dietro Dolly. 

Ma, qualunque cosa dicessero, egli sapeva che ormai tutto era perduto. Col capo contro lo stipite della porta, stava in piedi nella stanza attigua e sentiva uno stridio, un mugghio da lui non mai prima sentito, e sapeva che gridava quella cosa informe che prima era Kitty. Il bambino non lo desiderava più, già da tempo. Adesso odiava quell'essere. Adesso non desiderava neanche più la vita di lei, desiderava solo la fine di quelle orribili sofferenze. 

- Dottore, cos'è mai questo? cos'è mai questo? Dio mio! - disse, afferrando per il braccio il dottore che era entrato. 

- Finisce - disse il dottore. E il viso del dottore era così serio, mentre diceva questo, che Levin capì ``finisce'' nel senso di ``muore''. 

Fuori di sé, entrò di corsa nella stanza da letto. La prima cosa che vide fu il viso di Lizaveta Petrovna. Esso era ancora più agitato e più severo. Il viso di Kitty non c'era più. Nel posto dov'era prima, c'era qualcosa di mostruoso e per l'aspetto di tensione e per il suono che ne usciva. Egli cadde con la testa sul legno del letto, sentendo che il cuore gli si spezzava. L'orribile grido non finiva, s'era fatto ancora più orribile, ma poi, come se fosse giunto al limite estremo dell'orrore, si calmò a un tratto. Levin non credeva al proprio udito, ma non si poteva dubitare: il grido s'era calmato e si sentiva un silenzioso affaccendarsi, un fruscio, un respirare ansioso, e la voce di lei felice e affannata, viva e tenera che pronunciava piano: ``È finito''. 

Egli sollevò il capo. Abbassate sulla coperta le braccia senza forza, straordinariamente bella e calma, ella lo guardava senza parole e voleva, ma non poteva sorridere. 

E a un tratto da quel mondo misterioso e orribile, estraneo, in cui aveva vissuto in quelle ventidue ore, Levin si sentì trasportato in un attimo nel mondo solito di prima, ma splendente, adesso, d'una tale luce nuova di felicità, ch'egli non la sopportò. Le corde tese si strapparono tutte. Singhiozzi e lacrime di gioia, ch'egli non aveva in nessun modo preveduto, si sollevarono in lui con una forza tale, agitando tutto il suo corpo, che per lungo tempo gli impedirono di parlare. 

Caduto in ginocchio davanti al letto, egli teneva dinanzi alle labbra la mano della moglie e la baciava, e questa mano con un debole movimento delle dita rispondeva ai suoi baci. E intanto, là, ai piedi del letto, nelle abili mani di Lizaveta Petrovna, come la fiammella d'una lampada, oscillava la vita d'un essere umano che prima non c'era mai stato e che avrebbe vissuto e creato degli altri esseri nello stesso modo, con lo stesso diritto, con la stessa importanza di sé. 

- Vivo! vivo! È pure un maschio! Non vi agitate! - Levin sentì la voce di Lizaveta Petrovna, che batteva con la mano tremante la schiena del bambino. 

- Mamma, è vero? - disse la voce di Kitty. Le risposero i singhiozzi della principessa. 

E in mezzo al silenzio, come una risposta indubitabile alla domanda della madre, si sentì una voce affatto diversa da tutte le voci che parlavano nella camera. Era il grido ardito, temerario, che non voleva considerare nulla, d'un nuovo essere umano, che non si capiva donde fosse venuto fuori. 

Prima, se avessero detto a Levin che Kitty era morta e che lui era morto insieme con lei, e che avevano per bambini gli angeli, e che Dio era lì dinanzi a loro, non si sarebbe stupito di nulla; ma adesso, tornato nel mondo della realtà, faceva grandi sforzi per capire ch'ella era viva, sana e che l'essere che strideva in modo così disperato era suo figlio. Kitty era viva, le sofferenze erano finite. Ed egli era inesprimibilmente felice. Questo lo capiva e ne era pienamente soddisfatto. Ma il bambino? Donde veniva, perché, chi era? Non poteva in nessun modo abituarsi a questo pensiero. Gli sembrava qualcosa di superfluo, una sovrabbondanza a cui per lungo tempo non poté abituarsi. 

\capitolo{XVI}\label{xvi-6} 

Dopo le nove il vecchio principe, Sergej Ivanovic e Stepan Arkad'ic erano dai Levin e, dopo aver parlato della puerpera, si misero a discorrere anche di argomenti estranei. Levin li ascoltava, ricollegando involontariamente a questi discorsi il passato, cioè quello che era stato fatto fino a quella mattina, ricordava anche se stesso, così come era stato fino a quell'avvenimento. Erano passati proprio cent'anni da allora. Si sentiva ad un'altezza irraggiungibile, dalla quale discendeva con sforzo, per non offendere quelli con cui parlava. Parlava e pensava continuamente alla moglie, ai particolari del suo stato presente e al figlio, all'esistenza del quale cercava di abituare il proprio pensiero. Tutto il mondo femminile, che per lui da quando s'era sposato aveva acquistato un'importanza nuova, sconosciuta, ora si sollevava così in alto nella sua concezione, ch'egli non poteva abbracciarlo con la mente. Sentiva una conversazione sul pranzo del giorno prima al club e pensava: ``Cosa starà facendo adesso? si sarà addormentata? Come sta? cosa pensa? Grida mio figlio Dmitrij?''. E nel bel mezzo della conversazione, nel mezzo di una frase saltò su e uscì dalla stanza. 

- Mandami a dire se si può andare da lei - disse il principe. 

- Va bene, subito - rispose Levin e, senza fermarsi, andò da lei. 

Ella non dormiva e discorreva piano con la madre, facendo progetti per il prossimo battesimo. 

Tutta in ordine, pettinata, con una cuffietta elegante con qualcosa d'azzurro, le braccia stese sulla coperta, giaceva supina, e, incontrato lui con lo sguardo, con lo sguardo l'attirò a sé. I suoi occhi, già luminosi, divennero sempre più splendenti a misura ch'egli si avvicinava a lei. Sul viso c'era quello stesso passaggio dal terreno all'ultraterreno che c'è nel viso dei morenti; ma là c'è un distacco, qui era come un ritrovarsi. Di nuovo un'agitazione, simile a quella ch'egli aveva provato al momento del parto, gli venne al cuore. Ella gli prese la mano e chiese se avesse dormito. Lui non poteva rispondere e si voltava dall'altra parte, convinto della propria debolezza. 

- Io invece mi sono assopita, Kostja! - ella disse. - E adesso sto così bene. 

Lo guardava, ma a un tratto la sua espressione cambiò. 

- Datemelo - ella disse, sentendo il vagito del bambino. - Date qua, Lizaveta Petrovna, che anche lui lo veda. 

- Su, ecco, che il papà lo veda - disse Lizaveta Petrovna, sollevando e tendendo qualcosa di rosso, strano e oscillante. - Aspettate, prima ci sistemiamo - e Lizaveta Petrovna poggiò quella cosa rossa e oscillante sul letto, e si mise a sfasciare e a fasciare il bambino, sollevandolo e rivoltolandolo con un dito e cospargendolo di qualcosa. 

Levin, guardando quel minuscolo essere pietoso, faceva sforzi vani per trovare nell'animo suo il segno di un qualche sentimento paterno verso di lui. Sentiva per lui soltanto disgusto. Ma quando lo spogliarono e guizzarono i braccini sottili, i piedini color zafferano, anch'essi coi ditini e perfino con l'alluce che si distingueva dagli altri, e quando vide come Lizaveta Petrovna premeva, quasi fossero morbide molle, quei braccini che si protendevano, rinchiudendoli in panni di tela, gli venne una tale pietà per quell'essere e un tale terrore ch'ella gli facesse male, che la trattenne per un braccio. 

Lizaveta Petrovna si mise a ridere. 

- Non abbiate paura, non abbiate paura! 

Quando il bambino fu sistemato e trasformato in una pupattola dura, Lizaveta Petrovna lo dondolò, come inorgoglita del proprio lavoro, e si fece da parte, perché Levin potesse vedere il figlio in tutta la sua bellezza. 

Kitty, senza staccar gli occhi, guardava là, di sbieco. 

- Date, date! - disse, e si sollevò perfino. 

- Che fate, Katerina Aleksandrovna? Non si possono fare dei movimenti simili! Aspettate, ve lo darò io. Ecco che faremo vedere a papà, che bel giovanotto siamo! 

E Lizaveta Petrovna sollevò verso Levin, su di una sola mano (l'altra sosteneva solo con le dita la nuca oscillante), quello strano essere rosso che ciondolava e nascondeva il capo tra gli orli delle fasce. Ma c'era anche il naso, e c'eran gli occhi che guardavano di traverso e le labbra che succhiavano. 

- Un bellissimo bambino! - disse Lizaveta Petrovna. 

Levin sospirò con pena. Quel bellissimo bambino gli ispirava soltanto disgusto e pena. Era tutt'altro il sentimento che si aspettava. 

Egli si voltò mentre Lizaveta Petrovna lo accomodò al petto inesperto. 

A un tratto un riso gli fece sollevare il capo. Era Kitty che s'era messa a ridere. Il bambino s'era attaccato al petto. 

- Su, basta, basta! - diceva Lizaveta Petrovna, ma Kitty non lo lasciò andare. Egli si addormentò fra le sue braccia. 

- Guarda, adesso - disse Kitty, voltando verso di lui il bambino perché egli potesse vederlo. Il visino da vecchio, a un tratto, si corrugò ancora di più e il bimbo starnutì. 

Sorridendo e trattenendo appena le lacrime per l'emozione, Levin baciò la moglie e uscì dalla camera buia. 

Quello ch'egli provava per quel piccolo essere era proprio tutt'altra cosa da quello che si aspettava. Non c'era nulla di allegro e di gioioso in questo sentimento; al contrario, un nuovo senso di paura. Era la coscienza di un nuovo campo di vulnerabilità. E questa coscienza era così tormentosa nei primi tempi, il terrore che quell'essere impotente soffrisse era così forte, che proprio per questo non avvertiva lo strano sentimento di spensierata gioia e perfino di orgoglio ch'egli aveva provato proprio nel momento in cui il bambino aveva starnutito. 

\capitolo{XVII}\label{xvii-6} 

Gli affari di Stepan Arkad'ic erano in pessimo stato. 

Il denaro del legname per i due terzi era stato speso, e, detrattone il dieci per cento, egli aveva preso dal compratore, quasi tutto in anticipo, l'ultimo terzo. Il compratore non dava più denaro tanto più che quell'inverno Dar'ja Aleksandrovna, dichiarando per la prima volta in modo esplicito i propri diritti sul patrimonio, aveva rifiutato di quietanzare sul contratto la ricevuta del denaro per l'ultimo terzo del legname. Tutto lo stipendio se ne andava per le spese di casa e per il pagamento dei piccoli debiti insopportabili. Denaro non ce n'era affatto. 

Questo era spiacevole, disgustoso e non doveva prolungarsi tanto, secondo Stepan Arkad'ic. La ragione di questo, secondo il suo criterio, consisteva nel fatto ch'egli riceveva uno stipendio troppo basso. Il posto che occupava era stato, evidentemente, buono cinque anni prima, ma adesso non era più la stessa cosa. Petrov, come direttore di banca, riceveva 12.000 rubli; Sventickij, membro d'una società, ne riceveva 17.000; Mitin, dopo aver fondato una banca, ne ricavava 50.000. ``Evidentemente io mi sono addormentato e gli altri mi hanno dimenticato'' pensava di sé Stepan Arkad'ic. E cominciò a prestare ascolto, a guardarsi in giro, e, verso la fine dell'inverno, notò un posto molto buono e vi mosse all'attacco, prima da Mosca, per mezzo delle zie, degli zii, degli amici e, poi, quando la faccenda fu a buon punto, in primavera andò lui stesso a Pietroburgo. Era uno di quei posti di cui ce n'è adesso, di tutte le misure, dai 1.000 ai 50.000 rubli all'anno di stipendio, che erano disponibili più di quanti non ce ne fossero prima, posti comodi, venali: il posto di membro della commissione dell'agenzia unita del bilancio mutuo-credito delle ferrovie e degli istituti bancari meridionali. Questo posto, come tutti i posti simili, esigeva così enormi conoscenze e attività, che era difficile riunirle in una sola persona. E poiché l'uomo che riunisse queste qualità non c'era, era meglio allora che il posto lo occupasse un uomo onesto piuttosto che un disonesto. E Stepan Arkad'ic era non solo un onest'uomo (senza accento), ma era veramente un uomo onesto (con l'accento), con quel significato particolare che questa parola assume a Mosca, quando si dice uomo politico onesto, scrittore onesto, rivista onesta, istituto onesto, tendenza onesta, e che significa non soltanto che la persona o l'istituto non sono disonesti, ma che sono anche capaci, all'occasione, di scagliare una frecciata al governo. Stepan Arkad'ic frequentava a Mosca gli ambienti dove questa parola era introdotta, era considerato proprio là un uomo onesto e perciò aveva più diritti degli altri a quel posto. 

Quel posto dava dai sette ai diecimila rubli all'anno e Oblonskij poteva occuparlo senza lasciare il proprio posto statale. Esso dipendeva da due ministeri, da una signora e da due ebrei, e tutte queste persone, benché fossero già ben disposte, Stepan Arkad'ic doveva vederle a Pietroburgo. Inoltre, Stepan Arkad'ic aveva promesso a sua sorella Anna di ottenere da Karenin una risposta decisiva sul divorzio. E, ottenuti cinquanta rubli da Dolly, partì per Pietroburgo. 

Sedendo nello studio di Karenin e ascoltando il piano per lo studio delle cause del cattivo stato delle finanze russe, Stepan Arkad'ic aspettava soltanto ch'egli finisse, per cominciare a parlare del proprio affare personale e di Anna. 

- Sì, è molto giusto - egli disse, quando Aleksej Aleksandrovic, toltosi il pince-nez, senza il quale adesso non poteva leggere, guardò interrogativamente l'ex-cognato - è molto giusto nei particolari, ma tuttavia il principio del nostro tempo è la libertà. 

- Sì, ma io espongo un altro principio, che abbraccia il principio della libertà - disse Aleksej Aleksandrovic, accentuando le parole ``che abbraccia'' e mettendo di nuovo il pince-nez, per leggere di nuovo quel passo in cui ciò era scritto. 

E, sfogliato il manoscritto steso in bella grafia, dai margini enormi, Aleksej Aleksandrovic lesse di nuovo il passo convincente. 

- Io non voglio il sistema protezionistico per l'utilità dei privati, ma per il bene comune, per le classi inferiori e per quelle superiori allo stesso modo - egli diceva, guardando Oblonskij al di sopra del pince-nez. - Ma loro non possono capirlo, loro sono occupati soltanto dagli interessi personali e corrono dietro alle frasi. 

Stepan Arkad'ic sapeva che quando Karenin cominciava a parlare di quello che pensano e fanno loro, quegli stessi che non volevano accogliere i suoi piani ed erano tutto il male della Russia, allora si era vicini alla fine, e perciò adesso rinunciò volentieri al principio di libertà e consentì pienamente. Aleksej Aleksandrovic tacque, sfogliando pensieroso il manoscritto. 

- Ah, a proposito - disse Stepan Arkad'ic - volevo pregarti, se ne hai l'occasione, quando vedrai Pomorskoj, di dirgli una parola nel senso che io desidererei molto di occupare il posto che si è reso vacante di membro della commissione dell'agenzia unita del bilancio mutuo-credito delle ferrovie meridionali. 

Per Stepan Arkad'ic il nome di questo posto, tanto vicino al suo cuore, era ormai abituale, ed egli lo pronunciava in fretta senza sbagliarsi. 

Aleksej Aleksandrovic chiese in che consisteva l'attività di questa nuova commissione, e si fece pensieroso. Esaminava se nell'attività di questa commissione non ci fosse qualcosa di contrario ai propri piani. Ma siccome l'attività della nuova istituzione era molto complessa e i suoi piani abbracciavano un campo molto vasto, non poteva esaminarlo subito e, togliendo il pince-nez, disse: 

- Senza dubbio, posso dirglielo; ma perché ci tieni proprio a occupare questo posto? 

- Lo stipendio è buono, fino a novemila rubli, e le mie condizioni\ldots{} 

- Novemila - ripeté Aleksej Aleksandrovic e aggrottò le sopracciglia. 

L'alta cifra di questo stipendio gli ricordava che, da questo lato, l'attività eventuale di Stepan Arkad'ic era contraria al senso principale dei propri piani che tendevano sempre all'economia. 

- Io credo, e ci ho scritto su un memoriale, che nel nostro tempo questi stipendi enormi siano l'indice della falsa assiette economica della nostra amministrazione. 

- Ma come vorresti dire? - disse Stepan Arkad'ic. - Su, ammettiamo che un direttore di banca riceva diecimila rubli; significa che li vale. Oppure che un ingegnere ne riceva ventimila. È lavoro vivo, comunque tu la pensi. 

- Io considero che lo stipendio sia il pagamento di una merce e che esso debba sottostare alla legge della domanda e dell'offerta. Se invece il criterio dello stipendio si allontana da questa legge, come, ad esempio, nel caso di due ingegneri laureati dallo stesso istituto, tutti e due con le stesse nozioni e capacità, uno riceve quattromila rubli e l'altro si accontenta di duemila; o di direttori di banca che assumono con uno stipendio enorme degli studentelli di legge, degli ussari che non hanno nessuna nozione particolare, allora concludo che lo stipendio si fissa non secondo la legge della domanda e dell'offerta, ma direttamente, per compiacere le persone. E qui c'è un abuso, grave per se stesso e che si ripercuote dannosamente sul servizio dello stato. Io considero\ldots{} 

Stepan Arkad'ic si affrettò ad interrompere il cognato. 

- Sì, ma convieni che si apre un'istituzione nuova indubbiamente utile. Checché tu dica è un lavoro vivo! Apprezzano in modo particolare che il lavoro sia condotto con onestà - disse Stepan Arkad'ic con intenzione. 

Ma il significato moscovita di ``onestà'' era incomprensibile ad Aleksej Aleksandrovic. 

- L'onesta è soltanto una qualità negativa - disse. 

- Tuttavia mi farai gran piacere - disse Stepan Arkad'ic - se vorrai dire una parola a Pomorskoj. Così nel discorso\ldots{} 

- Eppure questo dipende maggiormente da Bolgarinov, sembra - disse Aleksej Aleksandrovic. 

- Bolgarinov per parte sua è del tutto consenziente - disse Stepan Arkad'ic, arrossendo. 

Stepan Arkad'ic arrossiva al ricordo di Bolgarinov, perché quello stesso giorno di mattina era stato dall'ebreo Bolgarinov e questa visita gli aveva lasciato una impressione spiacevole. Stepan Arkad'ic sapeva bene che il lavoro che voleva intraprendere era un lavoro nuovo, vivo e onesto; ma quando quella mattina Bolgarinov, evidentemente apposta, lo aveva fatto attendere due ore con gli altri sollecitatori nella sala di ricevimento, a un tratto, si era sentito a disagio. 

Ch'egli si sentisse a disagio perché lui, principe Oblonskij, discendente di Rjurik, aveva aspettato due ore nella sala d'aspetto di un giudeo, o perché, per la prima volta nella vita, non aveva seguito l'esempio degli antenati col servire il governo, e aveva voluto entrare in un'amministrazione nuova, certo è che si era sentito molto a disagio. In quelle due ore di attesa da Bolgarinov Stepan Arkad'ic, passeggiando agile per la sala d'aspetto, accomodandosi le fedine, mettendosi a discorrere con gli altri sollecitatori ed escogitando un giuoco di parole col quale poter dire come egli avesse aspettato dal giudeo, aveva nascosto con cura agli altri e a se stesso la sensazione provata. 

Ma tutto quel tempo s'era sentito a disagio e irritato, senza saper lui stesso perché: forse perché dal giuoco di parole non veniva fuori nulla o per qualcos'altro. Quando poi alla fine Bolgarinov l'aveva ricevuto con eccessiva cortesia, evidentemente trionfando dell'umiliazione inflittagli, e gli aveva quasi detto di no, Stepan Arkad'ic s'era affrettato a dimenticarlo il più presto possibile. E soltanto adesso, ricordandosene, s'era fatto rosso. 

\capitolo{XVIII}\label{xviii-6} 

- Adesso ho ancora una cosa da dirti, e tu sai quale\ldots{} a proposito di Anna - disse Stepan Arkad'ic dopo un certo silenzio e dopo avere scosso via da sé quell'impressione spiacevole. 

Non appena Oblonskij ebbe pronunciato il nome di Anna il viso di Aleksej Aleksandrovic cambiò completamente: in luogo dell'animazione di prima espresse una stanchezza mortale. 

- Che cosa volete da me? - disse, voltandosi sulla poltrona e chiudendo il pince-nez. 

- Una decisione, una qualsiasi decisione, Aleksej Aleksandrovic. Mi rivolgo a te adesso - ``non come al marito offeso'' voleva dire Stepan Arkad'ic, ma, temendo di rovinare la faccenda, cambiò - non come all'uomo di stato - il che risultò fuor di proposito - ma semplicemente all'uomo, all'uomo buono, al cristiano. Tu devi aver pietà di lei - egli disse. 

- Cioè, in che cosa propriamente - disse piano Karenin. 

- Sì, aver pietà di lei. Se tu la vedessi come l'ho vista io (ho passato tutto l'inverno con lei), ne avresti pena. La sua situazione è orribile, proprio orribile. 

- Mi sembrava - rispose Aleksej Aleksandrovic, con voce più sottile, quasi stridente - che Anna Arkad'evna avesse tutto quello che lei proprio aveva voluto. 

- Ah, Aleksej Aleksandrovic in nome di Dio, niente recriminazioni! Quel ch'è passato è passato, e tu sai quello che lei desidera e attende: il divorzio. 

- Ma io ritenevo che Anna Arkad'evna rinunciasse al divorzio nel caso che io esigessi l'obbligo di lasciarmi il figlio. Io ho risposto proprio così e pensavo che questa faccenda fosse finita. Io la ritengo finita - stridette Aleksej Aleksandrovic. 

- Ma in nome di Dio, non ti accalorare - disse Stepan Arkad'ic, toccando il cognato su un ginocchio. - La faccenda non è finita. Se tu mi permetti di ricapitolare, le cose stavano così: quando vi siete separati, tu sei stato grande, generoso come più non si poteva essere; le davi tutto, la libertà, perfino il divorzio. Lei ha apprezzato ciò. No, non credere. L'ha proprio apprezzato. Fino al punto che, in quel primo momento, sentendo la propria colpa dinanzi a te, non ha riflettuto e non poteva riflettere. Ha rifiutato tutto. Ma la realtà, il tempo hanno dimostrato che la situazione è tormentosa e impossibile. 

- La vita di Anna Arkad'evna non può interessarmi - interruppe Aleksej Aleksandrovic, sollevando le sopracciglia. 

- Permettimi di non crederci - obiettò dolcemente Stepan Arkad'ic. - La sua situazione è tormentosa per lei e di nessun vantaggio per gli altri. Ella l'ha meritata, tu dirai. Lei lo sa e non ti chiede nulla; dice apertamente che non osa chieder nulla. Ma io, noi tutti parenti, tutti quelli che le vogliono bene, ti preghiamo, ti supplichiamo. Perché si tormenta? Chi se ne avvantaggia? 

- Ma scusate, voi, a quanto pare, mi mettete nella situazione di un accusato - pronunciò Aleksej Aleksandrovic. 

- Ma no, ma no, per nulla affatto, comprendimi - disse Stepan Arkad'ic, toccandolo di nuovo, nel braccio, come se fosse convinto che questo contatto avrebbe raddolcito il cognato. - Io dico solo una cosa: la sua situazione è tormentosa e può essere alleviata da te, e tu non ci perderai nulla. Io ti accomoderò tutto in modo che non te ne accorgerai. Perché l'hai promesso. 

- La promessa era stata fatta prima. E io credevo che la questione del figlio decidesse la cosa. Inoltre speravo che Anna Arkad'evna avrebbe avuto sufficiente generosità\ldots{} - pronunciò a stento, pallido, con le labbra che gli tremavano, Aleksej Aleksandrovic. 

- Lei affida tutto alla tua generosità. Ti prega, ti supplica d'una cosa sola: di trarla da questa situazione impossibile in cui si trova. Ella ormai non chiede il figlio. Aleksej Aleksandrovic, tu sei un uomo buono. Mettiti al posto suo per un attimo. La questione del divorzio per lei, nella sua situazione, è una questione di vita o di morte. Se tu non avessi promesso prima, lei si sarebbe adattata alla sua situazione, avrebbe vissuto in campagna. Ma tu hai promesso, lei ti ha scritto ed è andata a vivere a Mosca. Ed ecco a Mosca, dove ogni incontro per lei è una coltellata al cuore, vive da sei mesi, aspettando la decisione da un giorno all'altro. E questo è come tenere un condannato a morte per sei mesi col laccio al collo, promettendo ora la morte, ora la grazia. Abbi pietà di lei, e io mi assumo di accomodar tutto così. Vos scrupules\ldots{} 

- Io non parlo di questo\ldots{} - lo interruppe con disgusto Aleksej Aleksandrovic. - Ma forse, io ho promesso quello che non avevo il diritto di promettere. 

- Allora tu rifiuti quello che hai promesso? 

- Io non ho mai respinto l'effettuazione di ciò che è possibile, ma desidero avere il tempo di riflettere fino a che punto sia possibile quello che è stato promesso. 

- No, Aleksej Aleksandrovic - cominciò a dire Oblonskij, saltando su - io non voglio crederci. Lei è così infelice, come solo può essere infelice una donna, e tu non puoi rifiutare una tale\ldots{} 

- Per quanto è possibile quello che è stato promesso. Vous professez d'être un libre penseur. Ma io, come persona credente, non posso agire contro la legge cristiana in una cosa tanto importante. 

- Ma nelle comunità cristiane e da noi, a quanto io sappia, il divorzio è ammesso - disse Stepan Arkad'ic. - Il divorzio è ammesso anche dalla nostra Chiesa. E noi vediamo\ldots{} 

- Ammesso, ma non in questo senso. 

- Aleksej Aleksandrovic, io non ti riconosco - disse Oblonskij, dopo aver taciuto un po'. - Non sei tu (e forse noi non l'abbiamo apprezzato?) che hai perdonato tutto e, mosso proprio dal sentimento cristiano, eri pronto a sacrificare tutto? Tu stesso hai detto: dare la tunica, quando ti han preso la camicia, e adesso\ldots{} 

- Io vi prego - cominciò a dire a un tratto Aleksej Aleksandrovic con voce stridula, alzandosi in piedi, pallido e con la mascella che gli tremava - vi prego di smettere, di smettere\ldots{} questo discorso. 

- Ah, no! Via, scusa, scusami se ti ho addolorato - cominciò a dire, sorridendo confuso Stepan Arkad'ic, tendendo la mano - ma tuttavia, come ambasciatore, ho riferito solo la mia ambasciata. 

Aleksej Aleksandrovic dette la mano, si fece pensieroso e pronunciò: 

- Devo riflettere e informarmi. Domani l'altro vi darò la risposta decisiva - disse, dopo aver riflettuto. 

\capitolo{XIX}\label{xix-6} 

Stepan Arkad'ic voleva andarsene, quando Kornej venne ad annunciare: 

- Sergej Alekseevic! 

``Chi è Sergej Alekseevic?'' stava per chiedere Stepan Arkad'ic, ma se ne ricordò immediatamente. 

- Ah, Serëza! - egli disse. ``Sergej Alekseevic. Io credevo fosse un capodivisione. Anna m'aveva appunto pregato di vederlo'' ricordò. 

E ricordò l'espressione timida, pietosa con cui, lasciandolo andar via, aveva detto: ``Tuttavia lo vedrai. Cerca di sapere dettagliatamente dove si trova, chi lo cura. E, Stiva\ldots{} se è possibile! Perché è possibile?''. Stepan Arkad'ic aveva capito quel che significava questo: ``se fosse possibile!'': se fosse possibile fare il divorzio in modo da darle il figlio\ldots{} Adesso Stepan Arkad'ic vedeva che non c'era neanche da pensarci, tuttavia fu contento di vedere il nipote. 

Aleksej Aleksandrovic ricordò al cognato che al ragazzo non parlavano mai della madre e che lo pregava di non accennarvi neanche con una parola. 

- È stato molto male dopo quell'incontro con la madre, che non avevamo preveduto - disse Aleksej Aleksandrovic. - Abbiamo perfino temuto per la sua vita. Ma una cura appropriata e i bagni di mare d'estate gli han ridato la salute, e ora, per consiglio del dottore, l'ho messo a scuola. In effetti l'influsso dei compagni ha avuto una buona azione su di lui, ed egli è in perfetta salute e studia bene. 

- Che bel giovane che è diventato! E non è Serëza, ma un vero e proprio Sergej Alekseevic! - disse, sorridendo, Stepan Arkad'ic, guardando il bel ragazzo dalle spalle larghe che entrava ardito e franco con una giacchetta turchina e i pantaloni lunghi. Il ragazzo aveva un aspetto sano e allegro. S'inchinò allo zio come a un estraneo, ma poi, riconosciutolo, arrossì e, come offeso e urtato da qualcosa, si voltò frettolosamente in là. Il ragazzo si avvicinò al padre e gli diede un biglietto in cui erano segnati i voti ricevuti a scuola. 

- Be', è passabile - disse il padre - puoi andare. 

- È dimagrito e s'è slanciato, ha finito d'essere un bambino, è diventato un monello; mi piace - disse Stepan Arkad'ic. - Ma ti ricordi di me? 

Il ragazzo si voltò rapido a guardare il padre. 

- Me ne ricordo, mon oncle - rispose, dopo aver guardato lo zio, e di nuovo ficcò gli occhi a terra. 

Lo zio chiamò a sé il ragazzo e lo prese per mano. 

- Be', come vanno le cose? - disse, desiderando chiacchierare, pur senza sapere cosa dire. 

Il ragazzo, arrossendo senza rispondere, tirava fuori dalla mano dello zio la propria. Non appena Stepan Arkad'ic lasciò andare la mano, come un uccello messo in libertà, dopo aver guardato il padre, uscì dalla stanza a passo svelto. 

Era passato un anno da che Serëza aveva visto sua madre per l'ultima volta. Da quel tempo non aveva mai più sentito parlare di lei. E in quello stesso anno era stato messo a scuola e aveva imparato a conoscere e a voler bene ai compagni. Quei sogni e il ricordo della madre, che, dopo l'incontro con lei, lo avevano fatto ammalare, adesso non l'occupavano più. Quando venivano, egli li scacciava con cura da sé, ritenendoli vergognosi, propri soltanto delle bambine, non di un ragazzo e d'un ragazzo che va a scuola. Sapeva che fra il padre e la madre c'era stato un litigio che li aveva separati, sapeva di esser destinato a rimanere col padre, e cercava di abituarsi a questo pensiero. 

Nel vedere lo zio che assomigliava alla madre, aveva provato una sensazione spiacevole, perché questo suscitava in lui quegli stessi ricordi ch'egli riteneva vergognosi. La cosa era stata tanto più spiacevole, in quanto da alcune parole che aveva sentito, aspettando, vicino alla porta dello studio, e in particolare dall'espressione del viso del padre e dello zio, indovinava che fra di loro si era dovuto parlare della madre. E per non giudicare il padre con il quale viveva e da cui dipendeva, e soprattutto per non abbandonarsi alla sensibilità che egli riteneva così umiliante, Serëza aveva cercato di non guardare quello zio, ch'era venuto a sconvolgere la sua calma, e di non pensare a quello ch'egli ricordava. 

Ma quando Stepan Arkad'ic, ch'era uscito dopo di lui, vistolo sulla scala, lo chiamò accanto a sé e domandò come passasse il tempo fra la scuola e le lezioni, Serëza, fuori della presenza del padre, si mise a parlare con lui. 

- Adesso qui abbiamo la ferrovia - disse, rispondendo alla sua domanda. - È così, vedete: due si siedono su di una panca. Sono i passeggeri. E uno si mette in piedi sempre sulla panca. E tutti si attaccano. Si può fare con le braccia, si può fare con le cinture, e si fanno andare per tutte le sale. Le porte si aprono già da prima. Eh, qui è molto difficile fare il conduttore! 

- È quello che sta in piedi? - domandò Stepan Arkad'ic, sorridendo. 

- Sì, qui ci vuole coraggio e sveltezza, specialmente quando si fermano tutt'a un tratto o quando qualcuno cade. 

- Già, non è mica uno scherzo - disse Stepan Arkad'ic, esaminando con tristezza quegli occhi animati che ricordavano la madre, adesso non più infantili, non più del tutto innocenti. E, sebbene avesse promesso ad Aleksej Aleksandrovic di non parlare di Anna, non resistette. 

- E ti ricordi di tua madre? - domandò a un tratto. 

- No, non me ne ricordo - pronunciò svelto Serëza e, fattosi rosso di fuoco, chinò gli occhi a terra. E lo zio non poté cavarne più nulla. 

L'istitutore slavo, dopo mezz'ora, trovò il suo allievo sulla scala, e per lungo tempo non poté capire se fosse irritato o piangesse. 

- Ma vi siete forse fatto male quando siete caduto? - disse l'istitutore. - Lo dicevo io ch'era un giuoco pericoloso. E bisogna dirlo al direttore. 

- Se mi fossi anche fatto male, nessuno se ne sarebbe accorto. Questo è sicuro. 

- E allora cosa mai? 

- Lasciatemi!\ldots{} Mi ricordo, non mi ricordo\ldots{} Che gliene importa? Perché devo ricordarmene? Lasciatemi in pace! - disse rivolto ormai non più all'istitutore, ma a tutto il mondo. 

\capitolo{XX}\label{xx-6} 

Stepan Arkad'ic, come sempre del resto, non passava oziosamente il tempo a Pietroburgo. A Pietroburgo, oltre agli affari, il divorzio della sorella e il posto, aveva bisogno, come sempre, di rinfrescarsi, come diceva, dopo l'odor di chiuso moscovita. 

Mosca, malgrado i suoi cafés chantants e gli omnibus, era pur sempre una palude stagnante. Stepan Arkad'ic lo avvertiva sempre. Dopo aver vissuto un po' a Mosca, specialmente vicino alla famiglia, sentiva che andava giù di umore. Vivendo a lungo a Mosca, senza allontanarsene mai, giungeva al punto da cominciare a irritarsi con la moglie per il cattivo umore e i rimproveri di lei, per la salute, per l'educazione dei figli, per i piccoli interessi del suo ufficio; perfino i debiti l'inquietavano. Ma gli bastava soltanto arrivare e restare un poco a Pietroburgo, nell'ambiente che frequentava, dove si viveva, si viveva proprio e non ci si congelava, come a Mosca, e subito quei pensieri sparivano e si dissolvevano, come cera al fuoco. 

La moglie?\ldots{} Proprio quel giorno aveva parlato col principe cecenskij. Il principe cecenskij aveva una moglie e una famiglia, e dei figlioli che erano paggi; e aveva un'altra famiglia, illegittima, dalla quale pure aveva dei figli. Sebbene anche la prima famiglia fosse buona, il principe cecenskij si sentiva più felice nella seconda famiglia. E portava il primogenito nella seconda famiglia e raccontava a Stepan Arkad'ic che riteneva ciò utile e adatto a sviluppare il figlio. Che ne avrebbero detto a Mosca? 

I figli? A Pietroburgo i figli non impedivano di vivere ai padri. I figli venivano educati negli istituti e non c'era quel barbaro concetto che si diffondeva a Mosca (L'vov ne era un esempio), che i figli dovessero avere tutto il bello della vita, e i genitori solo il lavoro e le preoccupazioni. Qui si capiva che un uomo aveva il dovere di vivere per sé, così come deve vivere un uomo colto. 

L'impiego? Ma anche l'impiego qui non era quel tirar la carretta ostinato, senza speranze, come a Mosca; qui c'era interesse nell'impiego. Un incontro, un favore, una parola incisiva, il saper rappresentare in dialogo vari scherzi, e un uomo, a un tratto, faceva carriera, come Brjancev che Stepan Arkad'ic aveva incontrato il giorno prima e che adesso era primo dignitario. Questo impiego aveva un interesse. 

Il modo poi di Pietroburgo di considerare gli affari pecuniari agiva in maniera particolarmente rasserenante su Stepan Arkad'ic. Bartnjanskij, che spendeva almeno cinquantamila rubli per il train che conduceva, il giorno innanzi gli aveva detto a questo proposito una parola significativa. 

Prima di pranzo, messisi a parlare, Stepan Arkad'ic aveva detto a Bartnjanskij: 

- Tu, mi pare, sei intimo di Mordvinskij; mi puoi fare un favore, digli una parola per me, ti prego. C'è un posto che vorrei occupare. Membro dell'agenzia\ldots{} 

- Via, tanto non me lo ricordo\ldots{} Soltanto, che gusto c'è ad andare in queste imprese ferroviarie insieme ai giudei? Sia come vuoi, ma è un obbrobrio! 

Stepan Arkad'ic non gli disse che era un lavoro vivo: Bartnjanskij non l'avrebbe capito. 

- C'è bisogno di denaro per vivere. 

- E non vivi? 

- Vivo, ma ho dei debiti. 

- Davvero? per molto?- disse Bartnjanskij, compassionevole. 

- Per moltissimo, per ventimila rubli. 

Bartnjanskij scoppiò a ridere allegramente. 

- Oh, un uomo fortunato! - disse. - Io ne ho per mezzo milione e non ho nulla, e, come vedi, si può ancora campare. 

E Stepan Arkad'ic non soltanto a parole, ma nei fatti vedeva la giustezza di questo. zivachov aveva trecentomila rubli di debiti e neanche una copeca di suo, e pure viveva e ancora come! Il conte Krivcov, da tempo, tutti gli avevano già fatto il funerale, e lui ne manteneva due. Petrovskij aveva speso cinque milioni e viveva sempre allo stesso modo e soprintendeva alle finanze e riceveva ventimila rubli di stipendio. Ma, oltre a questo, Pietroburgo agiva su Stepan Arkad'ic in maniera fisicamente piacevole. Lo ringiovaniva. A Mosca qualche volta si guardava i capelli bianchi, si addormentava dopo pranzo, si stirava, saliva la scala al passo, respirando con affanno, si annoiava con le donne giovani, non ballava nelle feste. A Pietroburgo invece sentiva sempre che gli andavano via dieci anni dal groppone. 

Egli provava a Pietroburgo la stessa cosa che gli aveva detto, ancora il giorno prima, il sessantenne principe Pëtr Oblonskij, appena tornato dall'estero. 

- Noi qui non sappiamo vivere - diceva Pëtr Oblonskij. - Ci credi, io ho passato l'estate a Baden; ebbene, mi son sentito proprio un giovanotto. Vedo una donna giovane, e i pensieri\ldots{} Pranzi, bevi un poco e si ha forza, coraggio. Sono arrivato in Russia; dovevo andare da mia moglie e poi in campagna; non ci crederai, dopo due settimane, mi son messo la vestaglia indosso, ho smesso di vestirmi per il pranzo. Altro che pensare alle donne giovani! Son diventato proprio un vecchio. Mi restava solo da pensare all'anima. Sono andato a Parigi e mi son ripreso di nuovo. 

Stepan Arkad'ic sentiva proprio la stessa cosa, come Pëtr Oblonskij. A Mosca egli si lasciava andare così che davvero, a viverci molto, sarebbe giunto perfino a qualcosa di buono, quasi alla salvezza dell'anima; a Pietroburgo invece si sentiva di nuovo una persona in gamba. 

Fra la principessa Betsy Tverskaja e Stepan Arkad'ic esistevano vecchi rapporti, molto strani. Stepan Arkad'ic le faceva sempre scherzosamente la corte e le diceva, sia pure scherzosamente, le cose più sconvenienti, sapendo che questo le piaceva più di tutto. Il giorno dopo la sua conversazione con Karenin, Stepan Arkad'ic, essendo passato da lei, si sentiva tanto giovane, che nel farle la corte e nel celiare, era andato, senza volerlo, così lontano, da non saper più come tornare indietro, perché, per sua disgrazia, non solo non gli piaceva, ma gli era disgustosa. Questo tono s'era stabilito perché lui piaceva molto a lei. Quindi egli fu molto contento dell'arrivo della principessa Mjagkaja che interruppe la loro solitudine a due. 

- Ah, anche voi siete qui - ella disse nel vederlo. - Be', come va la vostra povera sorella? Non mi guardate così - soggiunse. - Da che le si sono scagliati contro tutti quelli che sono centomila volte peggiori di lei, io penso che ha fatto benissimo. Non posso perdonare a Vronskij di non avermi fatto sapere quando erano a Pietroburgo. Sarei andata da lei e con lei dovunque. Per favore, ditele da parte mia il mio affetto. Ma raccontatemi di lei. 

- Sì, la sua situazione è penosa, ella\ldots{} - voleva cominciare a raccontare Stepan Arkad'ic, prendendo ingenuamente per moneta contante le parole della Mjagkaja: ``raccontate di vostra sorella''. La principessa Mjagkaja subito, secondo la sua abitudine, lo interruppe e si mise a raccontare lei stessa. 

- Lei ha fatto quello che tutte, tranne me, fanno, ma nascondono; ma lei non voleva ingannare e ha fatto benissimo. E ha fatto ancora meglio, perché ha abbandonato quel pazzo di vostro cognato. Voi mi scusate. Tutti dicevano ch'era intelligente, io sola dicevo ch'era scemo. Adesso, da che s'è messo insieme con Lidija Ivanovna e con Landau, tutti dicono che è un pazzo, e io sarei magari contenta di non esser d'accordo con tutti, ma questa volta non posso. 

- Ma spiegatemi, per favore - disse Stepan Arkad'ic - che cosa significa questo? Ieri sono stato da lui per l'affare di mia sorella e ho chiesto una risposta decisiva. Lui non m'ha dato una risposta e mi ha detto che ci avrebbe pensato, e questa mattina invece della risposta, ho ricevuto un invito per questa sera dalla contessa Lidija Ivanovna. 

- Eh già, è così! - cominciò a dire con gioia la principessa Mjagkaja. - Chiederanno a Landau che cosa ne dice. 

- Come a Landau? Perché? Chi è mai Landau? 

- Come, non conoscete Jules Landau? Le fameux Jules Landau, le clairvoyant? Anche lui è un pazzo, ma la sorte di vostra sorella dipende da lui. Ecco le conseguenze della vita in provincia, voi non sapete niente. Landau, vedete, era commis in un magazzino di Parigi ed era andato dal dottore. Dal dottore, nella sala d'aspetto, si addormentò e nel sonno cominciò a dare consigli a tutti i malati. E consigli sorprendenti. Poi la moglie di Jurij Meldinskij, sapete, quello malato, è venuta a sapere di questo Landau e lo ha messo accanto al marito. Egli cura il marito. E, secondo me, non gli ha portato nessun miglioramento, perché è sempre debole allo stesso modo, ma loro credono in lui e se lo portano dietro. E l'hanno portato in Russia. Qui tutti si sono gettati su di lui, e lui s'è messo a curar tutti. Ha guarito la contessa Bezzubova, e lei ha preso a volergli tanto bene che l'ha adottato. 

- Come adottato? 

- Sì, adottato. Adesso non è più Landau, ma il conte Bezzubov. Ma la questione non sta qui. Lidija Ivanovna, io le voglio molto bene, ma non ha la testa a posto e naturalmente adesso s'è gettata su questo Landau, così che senza di lui né lei né Aleksej Aleksandrovic decidono nulla, e perciò la sorte di vostra sorella è ora nelle mani di Landau, o conte Bezzubov. 

\capitolo{XXI}\label{xxi-6} 

Dopo un ottimo pranzo e una grande quantità di cognac bevuto da Bartnjanskij, Stepan Arkad'ic, solo un poco in ritardo rispetto all'ora fissata, entrava dalla contessa Lidija Ivanovna. 

- Chi c'è ancora dalla contessa, il francese? - domandò Stepan Arkad'ic al portiere, osservando il noto cappotto di Aleksej Aleksandrovic e uno strano, primitivo cappotto con le fibbie. 

- Aleksej Aleksandrovic Karenin e il conte Bezzubov - rispose sostenuto il portiere. 

``La principessa Mjagkaja ha indovinato - pensò Stepan Arkad'ic salendo la scala. - Strano! Però sarebbe bene farsi intimo con lei. Ha un'influenza enorme. Se dirà una parola a Pomorskoj, allora la cosa è sicura''. 

Fuori era ancora giorno, ma nel piccolo salotto della contessa Lidija Ivanovna dalle tende abbassate erano già accese le luci. Presso una tavola rotonda, sotto la lampada, sedevano la contessa e Aleksej Aleksandrovic, e discorrevano sottovoce di qualcosa. Un uomo non alto, magro, con un bacino da donna, le gambe ripiegate in dentro all'altezza delle ginocchia, molto pallido, bello, con gli occhi luminosi, bellissimi, e i capelli lunghi, spioventi sul colletto del soprabito, stava in piedi all'altra estremità, osservando la parete con i ritratti. Dopo aver salutato la padrona di casa e Aleksej Aleksandrovic, Stepan Arkad'ic, involontariamente, guardò un'altra volta lo sconosciuto. 

- Monsieur Landau! - disse, rivolta a costui, la contessa con una dolcezza e una precauzione che colpirono Oblonskij. E li presentò. 

Landau si voltò in fretta, si avvicinò e, dopo aver sorriso, mise nella mano di Stepan Arkad'ic la sua immobile mano sudata e subito si allontanò di nuovo e si mise a guardare i ritratti. La contessa e Aleksej Aleksandrovic si scambiarono uno sguardo significativo. 

- Sono molto contenta di vedervi, in particolar modo oggi - disse la contessa Lidija Ivanovna, indicando a Stepan Arkad'ic un posto accanto a Karenin. 

- Vi ho presentato a lui come a Landau - disse lei con voce sommessa, dopo aver dato un'occhiata al francese e poi subito ad Aleksej Aleksandrovic - ma invero egli è il conte Bezzubov, come forse saprete. Ma egli non ama questo titolo. 

- Già, ho sentito - rispose Stepan Arkad'ic. - Dicono che abbia del tutto guarito la contessa Bezzubova. 

- Quest'oggi è stata da me, fa proprio pena! - disse rivolta ad Aleksej Aleksandrovic. - Questa separazione è tremenda per lei. Per lei è un tale colpo! 

- E lui va via definitivamente? - domandò Aleksej Aleksandrovic. 

- Sì, va a Parigi. Ieri ha sentito una voce - disse la contessa Lidija Ivanovna, guardando Stepan Arkad'ic. 

- Ah, una voce! - ripeté Oblonskij, sentendo che bisognava essere quanto più possibile prudenti in quella compagnia, dove accadeva o stava per accadere qualcosa di speciale, di cui non possedeva ancora la chiave. 

Seguì un momento di silenzio, dopo il quale la contessa Lidija Ivanovna, come avvicinandosi all'argomento principale della conversazione, disse a Oblonskij con un sorriso sottile: 

- Io vi conosco da tempo e sono molto contenta di conoscervi più da vicino. Les amis de nos amis sont nos amis. Ma per essere amici bisogna penetrare col pensiero lo stato d'animo dell'amico, e io ho paura che voi non lo facciate riguardo ad Aleksej Aleksandrovic. Voi capite di che cosa parlo - disse, alzando i suoi bellissimi occhi pensosi. 

- In parte, contessa, capisco che la posizione di Aleksej Aleksandrovic\ldots{} - disse Oblonskij senza ben capire di che cosa si trattasse, e desiderando perciò di restare sulle generali. 

- Il cambiamento non è nella situazione esterna - disse severa la contessa Lidija Ivanovna, seguendo nello stesso tempo con lo sguardo innamorato Aleksej Aleksandrovic, che s'era alzato e s'era portato presso Landau - il suo cuore s'è cambiato, gli è stato dato un cuore nuovo, e io temo che voi non abbiate pienamente penetrato il cambiamento che è avvenuto in lui. 

- Ma nelle linee generali posso immaginarmi questo cambiamento. Noi siamo sempre stati amici, e adesso\ldots{} - disse, rispondendo con uno sguardo mellifluo allo sguardo della contessa, Stepan Arkad'ic, mentre andava considerando quale dei due ministri ella conoscesse più intimamente, per sapere a quale dei due avrebbe dovuto pregarla di parlare. 

- Quel cambiamento che è avvenuto in lui non può indebolire i suoi sentimenti d'amore verso il prossimo; al contrario, il cambiamento che è avvenuto in lui deve aumentare l'amore. Ma io ho paura che voi non mi comprendiate. Volete del tè? - disse, indicando con gli occhi un cameriere che serviva il tè su di un vassoio. 

- No, grazie contessa. S'intende, la sua sventura\ldots{} 

- Sì, la sventura, che è divenuta la più grande fortuna poiché il cuore s'è fatto nuovo, si è riempito di Lui - diss'ella guardando con trasporto Stepan Arkad'ic. 

``Io penso che si potrà chiederle di dirlo a tutti e due'' pensava Stepan Arkad'ic. 

- Oh, certamente, contessa - diss'egli - ma io penso che questi cambiamenti sono così intimi, che nessuno, neppure la persona più vicina, ama parlarne. 

- Al contrario! Dobbiamo parlarne, e aiutarci l'un l'altro. 

- Sì, senza dubbio, ma può esserci una tale differenza di opinioni, e inoltre\ldots{} - disse Oblonskij con un sorriso mellifluo. 

- Non ci può essere differenza nell'opera della santa verità. 

- Oh sì, certamente, ma\ldots{} - e, essendosi confuso, Stepan Arkad'ic tacque. Aveva capito che si trattava di religione. 

- Mi pare che stia per addormentarsi - pronunciò con un mormorio significativo Aleksej Aleksandrovic, avvicinandosi a Lidija Ivanovna. 

Stepan Arkad'ic si guardò in giro. Landau era seduto vicino a una finestra, appoggiato al bracciuolo della spalliera di una poltrona, col capo chino. Avendo notato gli sguardi rivolti su di lui, alzò il capo e sorrise d'un sorriso infantilmente ingenuo. 

- Non ci badate - disse Lidija Ivanovna, e con un movimento leggero accostò una sedia ad Aleksej Aleksandrovic. - Ho notato\ldots{} - cominciò a dire qualcosa, quando nella stanza entrò un cameriere con una lettera. Lidija Ivanovna scorse rapidamente il biglietto e, chiesto scusa, con una straordinaria sveltezza scrisse, consegnò la risposta e tornò accanto alla tavola. - Ho notato - continuò il discorso incominciato - che i moscoviti, soprattutto gli uomini, sono le persone più indifferenti verso la religione. 

- Oh no, contessa, mi pare che i moscoviti abbiano la fama d'essere i più tenaci - rispose Stepan Arkad'ic. 

- Ma per quanto sappia, voi, disgraziatamente, siete fra gli indifferenti - disse con un sorriso stanco Aleksej Aleksandrovic, rivolgendosi a lui. 

- Come si può essere indifferenti! - disse Lidija Ivanovna. 

- Io, riguardo a questo, non che sia indifferente, ma sono in attesa - disse Stepan Arkad'ic con il più dolce dei suoi sorrisi. - Io penso che per me non sia giunto ancora il tempo per codeste questioni. 

Aleksej Aleksandrovic e Lidija Ivanovna si scambiarono un'occhiata. 

- Noi non possiamo mai sapere se sia venuto o no il tempo per noi - disse severo Aleksej Aleksandrovic. - Noi non dobbiamo pensare al fatto se siamo pronti o non siamo pronti: la grazia non può essere guidata da considerazioni umane; a volte non scende su chi opera e scende sugli impreparati, come su Saul. 

- No, mi pare, non ancora adesso - disse Lidija Ivanovna, che intanto aveva sorvegliato i movimenti del francese. Landau si alzò e si avvicinò a loro. 

- Mi permettete di ascoltare? - chiese. 

- Oh sì, non volevo disturbarvi - disse Lidija Ivanovna, guardandolo con tenerezza - sedetevi con noi. 

- Bisogna soltanto non chiudere gli occhi per non rimanere privi della luce - continuò Aleksej Aleksandrovic. 

- Ah, se voi sapeste la felicità che noi proviamo, sentendo la Sua eterna presenza nell'anima nostra! - disse la contessa Lidija Ivanovna, sorridendo beata. 

- Ma l'uomo a volte può sentirsi incapace di elevarsi a quest'altezza - disse Stepan Arkad'ic, sentendo di mancar di lealtà nel riconoscer l'altezza della religione, ma nello stesso tempo senza decidersi a confessare la propria libertà di pensiero dinanzi a una persona che, con una sola parola a Pomorskoj, poteva fargli avere il posto desiderato. 

- Con questo volete dire che il peccato glielo impedisce? - disse Lidija Ivanovna. - Ma è un'opinione falsa. Non c'è peccato per i credenti, il peccato è già riscattato. Pardon - ella soggiunse, guardando il cameriere che era entrato di nuovo con un altro biglietto. Lesse e rispose a voce: ``dite domani, dalla granduchessa\ldots{}''. - Per il credente non c'è peccato - ella proseguì il discorso. 

- Sì, ma la fede senza opere è morta - disse Stepan Arkad'ic, ricordandosi questa frase del catechismo e difendendo ormai la propria indipendenza col solo sorriso. 

- Eccolo, è dell'epistola dell'apostolo Giacomo - disse Aleksej Aleksandrovic, rivolgendosi con un certo rimprovero a Lidija Ivanovna, evidentemente come per una cosa di cui avevano già parlato più di una volta. - Quanto danno ha fatto la falsa interpretazione di questo passo! Nulla allontana tanto dalla fede come questa interpretazione. ``Io non ho opere, non posso credere'', mentre questo non è detto in nessun posto. È detto il contrario. 

- Lavorare per Dio, salvar l'anima con le fatiche, col digiuno - disse la contessa Lidija Ivanovna con disprezzo e disgusto - sono le idee barbare dei nostri monaci\ldots{} Mentre questo non è detto in nessun passo. È molto più semplice e facile - ella soggiunse, guardando Oblonskij con quello stesso sorriso d'approvazione con cui a corte incoraggiava le giovani damigelle d'onore, confuse dall'ambiente nuovo. 

- Noi siamo salvati dal Cristo che ha sofferto per noi. Siamo salvati dalla fede - confermò Aleksej Aleksandrovic, approvando con lo sguardo le parole di lei. 

- Vous comprenez l'anglais? - chiese Lidija Ivanovna e, ricevutane risposta affermativa, si alzò e cominciò a scegliere dei libri su di uno scaffaletto. 

- Voglio leggere Safe and Happy o Under the wing - ella disse, guardando interrogativamente Karenin. E trovato il libro e sedutasi di nuovo al proprio posto, lo aprì. - È molto breve. Qui è descritta la via attraverso la quale si acquista la fede, e la felicità, al di sopra di ogni cosa terrestre, che allora riempie l'anima. Il credente non può essere infelice, perché non è solo. Ma ecco, vedrete. - Ella s'era già disposta a leggere, quando entrò di nuovo il cameriere. - La Borozdina? Dite domani alle due. Sì - ella disse, segnando con un dito un punto nel libro e guardando con un sospiro dinanzi a sé coi bellissimi occhi pensosi. - Ecco come agisce la fede vera. Conoscete la Sanina? Marie? Sapete la sua sventura? Ha perduto il suo unico bambino. Era disperata. Ebbene? Ha trovato quest'amico, e adesso ringrazia Dio per la morte del bambino. Ecco la felicità che dà la fede! 

- Oh sì, è molto\ldots{} - disse Stepan Arkad'ic, contento del fatto che avrebbero letto e gli avrebbero dato il tempo di riaversi un poco. ``No, ormai si vede che è meglio non chiedere nulla per oggi - pensava - basta uscire di qua senza avere ingarbugliato le cose''. 

- Vi annoierete - disse la contessa Lidija Ivanovna, rivolta a Landau - voi non sapete l'inglese, ma è una cosa breve. 

- Oh, capirò - disse Landau con lo stesso sorriso e chiuse gli occhi. 

Aleksej Aleksandrovic e Lidija Ivanovna si scambiarono un'occhiata significativa e la lettura cominciò. 

\capitolo{XXII}\label{xxii-6} 

Stepan Arkad'ic si sentiva completamente confuso nell'ascoltare quegli strani discorsi, per lui nuovi. La complessità della vita pietroburghese agiva, in genere, da eccitante su di lui, traendolo dal ristagno moscovita; ma quella complessità egli l'amava e capiva nei campi a lui vicini e noti; in quell'ambiente estraneo si sentiva confuso, stordito e non riusciva ad abbracciare tutto. Ascoltando la contessa Lidija Ivanovna e sentendo fissi su di sé gli occhi belli, ingenui o furbi, non lo sapeva lui stesso, di Landau, Stepan Arkad'ic cominciava a provare una certa pesantezza di testa. 

I pensieri più vaghi gli si confondevano nella testa. ``Marie Sanina è contenta che le sia morto il bambino\ldots{} Sarebbe bene fumare adesso\ldots{} Per salvarsi bisogna soltanto credere, e i monaci non sanno come bisogna fare; lo sa la contessa Lidija Ivanovna\ldots{} E perché ho un tal peso sulla testa? Per il cognac o perché tutto questo è molto strano? Tuttavia finora non ho fatto nulla di poco conveniente, mi pare, comunque, non posso più chiederglielo. Dicono che essi obblighino a pregare. Purché non obblighino me. Questo poi sarebbe troppo sciocco. E quali assurdità sta leggendo, ma pronuncia bene. Landau è Bezzubov, perché è Bezzubov?''. A un tratto Stepan Arkad'ic sentì che la mascella inferiore cominciava a torcerglisi in uno sbadiglio irrefrenabile. Accomodò le fedine, nascondendo lo sbadiglio, e si scosse. Ritornò in sé nel momento in cui la voce della contessa Lidija Ivanovna diceva: ``dorme''. 

Stepan Arkad'ic ritornò in sé con spavento, sentendosi colpevole e colto in fallo. Ma si consolò subito vedendo che la parola ``dorme'' non riguardava lui, ma Landau. Il francese s'era addormentato proprio come Stepan Arkad'ic. Ma il sonno di Stepan Arkad'ic, come egli pensava, li avrebbe offesi (ormai non pensava neanche più questo, tanto ormai gli sembrava tutto strano), e il sonno di Landau, invece, li rallegrò straordinariamente, in particolare rallegrò la contessa Lidija Ivanovna. 

- Mon ami - disse Lidija Ivanovna, sollevando con precauzione, per non far rumore, le pieghe del suo vestito di seta e chiamando, ormai eccitata, Karenin non Aleksej Aleksandrovic ma mon ami - donnez lui la main. Vous voyez? Sst! - ella zittì il cameriere che era entrato di nuovo. - Non si riceve. 

Il francese dormiva o fingeva di dormire, appoggiando la testa alla spalliera della seggiola, e con la mano sudata, che teneva su di un ginocchio, faceva dei movimenti deboli come se afferrasse qualcosa. Aleksej Aleksandrovic si alzò, voleva proceder cauto, ma inciampò nella tavola, si avvicinò e mise la mano nella mano del francese. Stepan Arkad'ic si alzò anche lui e, spalancando gli occhi, desiderando di svegliarsi nel caso si fosse addormentato, guardava ora l'uno ora l'altro. Tutto ciò accadeva nella realtà. Stepan Arkad'ic sentiva che nella sua testa le cose andavano sempre peggio. 

- Que la personne qui est arrivée la dernière, celle qui demande, qu'elle sorte! Qu'elle sorte! - pronunciò il francese senza aprire gli occhi. 

- Vous m'excuserez, mais vous voyez\ldots{} Revenez vers dix heures, encore mieux demain. 

- Qu'elle sorte! - ripeté il francese con impazienza. 

- C'est moi, n'est-pas? - E, ricevutane risposta affermativa, Stepan Arkad'ic, dimentico anche di quello che voleva chiedere a Lidija Ivanovna, dimentico dell'affare della sorella, col solo desiderio di andar via di là al più presto, uscì in punta di piedi e corse fuori in istrada come da una casa infestata, e discorse e scherzò a lungo con i vetturini, desiderando di riprendere i sensi al più presto. 

Al Teatro Francese, dove giunse alla fine, e poi dai tartari dove bevve lo champagne, Stepan Arkad'ic respirò un poco nell'atmosfera che gli era propria. Tuttavia quella sera non si sentiva bene affatto. 

Tornato a casa da Pëtr Oblonskij, presso il quale abitava a Pietroburgo, Stepan Arkad'ic trovò un biglietto di Betsy. Ella gli scriveva che desiderava molto terminare la conversazione cominciata, e lo pregava di passare l'indomani. Fece appena in tempo a leggere questo biglietto e a farci su una smorfia, che di sotto si sentirono i passi strascicati di persone che trasportavano qualcosa di pesante. 

Stepan Arkad'ic uscì a vedere. Era il ringiovanito Pëtr Oblonskij. Era così ubriaco da non poter salire le scale; ma, nel vedere Stepan Arkad'ic, ordinò che lo mettessero in piedi e, avvinghiatosi a lui, andò in camera sua e là cominciò a raccontargli come aveva passato la serata, e proprio lì si addormentò. 

Stepan Arkad'ic era avvilito, il che gli accadeva di rado, e per lungo tempo non poté prender sonno. Qualunque cosa ricordasse era disgustosa, e la cosa più disgustosa di tutte, quasi vergognosa, era il ricordo della serata dalla contessa Lidija Ivanovna. 

Il giorno dopo ricevette da Aleksej Aleksandrovic un rifiuto definitivo per il divorzio di Anna e capì che questa decisione era stata basata su quello che aveva detto il giorno prima il francese nel suo sonno vero o finto. 

\capitolo{XXIII}\label{xxiii-6} 

Per intraprendere qualcosa nella vita familiare, sono indispensabili o un completo dissidio fra i coniugi o un amorevole accordo. Quando invece i rapporti fra i coniugi sono indefiniti e non c'è né l'uno né l'altro, nulla può essere intrapreso. 

Molte famiglie rimangono per anni in vecchi luoghi, uggiosi ormai per entrambi i coniugi, solo perché non c'è un assoluto dissidio, né un pieno accordo. 

E per Vronskij e per Anna la vita moscovita con il caldo e con la polvere, quando il sole splendeva ormai non più primaverile, ma estivo, e tutti gli alberi dei viali erano già da tempo con le foglie, e le foglie erano ricoperte di polvere, era insopportabile; ma essi, senza trasferirsi a Vozdvizenskoe, come era stato deciso da tempo, continuavano a stare a Mosca, ormai uggiosa a entrambi, perché negli ultimi tempi non c'era accordo fra di loro. 

L'irritazione che li divideva non aveva nessuna causa esteriore, e tutti i tentativi di spiegazione non solo non la eliminavano, ma l'accrescevano. Era un'irritazione interna, che per lei aveva come base l'affievolirsi dell'amore di lui, per lui, il pentimento d'essersi messo per amore di lei in una posizione penosa, ch'ella, invece di alleviare, rendeva sempre più penosa. Né l'una né l'altro dichiaravano le cause della propria irritazione, ma si consideravano scambievolmente in fallo e ad ogni pretesto cercavano di dimostrarselo a vicenda. 

Per Anna, lui, con le sue abitudini, coi suoi pensieri, coi suoi desideri, col suo complesso spirituale e fisico, significava una cosa sola: l'amore per le donne che, secondo il suo sentimento, doveva concentrarsi unicamente su di lei. Ma questo amore era diminuito e, di conseguenza, pensava che egli aveva dovuto riversarlo su altre o su di un'altra donna e ciò la rendeva gelosa. Non era però gelosa di una determinata donna; lo era solamente perché l'amore per lei si affievoliva. Non avendo ancora trovato un oggetto per la propria gelosia, lo andava cercando. Ogni minima allusione era buona; ora erano quelle donne volgari con le quali egli per la sua vita di scapolo poteva venire in rapporti così facilmente; ora erano le donne che incontrava nell'alta società; ora era una ragazza immaginaria che egli avrebbe voluto sposare, rompendo i rapporti con lei. Quest'ultima gelosia la tormentava più di tutte, in particolare perché egli stesso, in un momento di sincerità, le aveva detto, imprudentemente, che sua madre lo capiva tanto poco da permettersi di esortarlo a sposare la principessina Sorokina. 

E, essendone gelosa, Anna era indignata contro di lui e cercava in tutto i motivi per indignarsi. Di tutto quello che c'era di penoso nella propria situazione ella accusava lui. Il tormentoso stato di attesa che aveva vissuto a Mosca, fra cielo e terra, la lentezza e l'indecisione di Aleksej Aleksandrovic, la propria solitudine, tutto ella attribuiva a lui. Se egli l'avesse amata, avrebbe capito tutta la difficoltà della sua situazione, e l'avrebbe tratta fuori da questa. Del fatto che ella stesse a Mosca e non in campagna, anche lui era colpevole. Egli non poteva vivere sepolto in campagna, come voleva lei. Gli era indispensabile la società, e aveva messo lei in quella orribile posizione, la cui difficoltà non voleva intendere. E, di nuovo, era anche lui colpevole ch'ella fosse divisa per sempre dal figlio. 

Perfino quei pochi momenti di effusione che avvenivano fra di loro non la calmavano più: nell'amore di lui ella intravedeva una certa calma, un tono di sicurezza che prima non c'erano e che la irritavano. 

Era già il crepuscolo. Anna, sola, aspettava il ritorno di lui da un pranzo di scapoli a cui era andato, camminava avanti e indietro per il suo studio (la stanza dove si sentiva meno il rumore del selciato), e ripensava in tutti i particolari le espressioni del litigio del giorno prima. Riandando sempre più indietro, dalle parole offensive della discussione che le tornavano in mente a quello che ne era stato il motivo, ella giunse finalmente all'inizio della conversazione. Per lungo tempo non poté credere che il dissidio fosse cominciato da una conversazione così innocua, così poco vicina al cuore di chiunque. E realmente era stato così. Tutto era cominciato dal fatto che egli aveva preso in giro i ginnasi femminili, ritenendoli inutili, e lei ne aveva preso le difese. Egli aveva trattato irriverentemente l'istruzione femminile, in generale, e aveva detto che Hanna, l'inglese protetta da Anna, non aveva nessun bisogno di conoscere la fisica. Questo aveva irritato Anna. 

- Io non m'aspetto che vi ricordiate di me, dei miei sentimenti, come se ne può ricordare una persona che ama, ma mi aspetto soltanto un po' di delicatezza - ella aveva detto. 

E invero, egli era diventato rosso di collera e aveva detto qualcosa di spiacevole. Ella non ricordava cosa gli avesse risposto, ma soltanto che a questo punto, a proposito di qualche cosa, egli, evidentemente col desiderio di farle male, aveva detto: 

- Mi spiace il vostro entusiasmo per questa bambina, è vero, perché vedo che esso è innaturale. 

La crudezza con la quale egli distruggeva il mondo da lei creato con tanta fatica per sopportare la propria vita penosa, l'ingiustizia con la quale l'accusava di finzione, di mancanza di naturalezza, l'avevano indignata. 

- Sono molto spiacente che solo ciò che è volgare e materiale sia comprensibile e naturale per voi - ella aveva detto, ed era uscita dalla stanza. 

Quando la sera prima egli era venuto da lei, essi non avevano ricordato il diverbio che c'era stato, ma tutti e due avevano sentito che esso era appianato, ma non scomparso. 

Tutto quel giorno egli non era stato in casa e lei provava una sensazione di solitudine e di pena nel sentirsi in urto con lui; così che voleva dimenticare e perdonare tutto e far la pace, voleva accusare se stessa e assolvere lui. 

``Io stessa sono colpevole. Sono irritabile, sono insensatamente gelosa. Farò la pace con lui, e partiremo per la campagna, là sarò più calma'' ella si diceva. 

``Innaturale - ella ricordò a un tratto non tanto la parola quanto l'intenzione di farle del male, che più di tutto l'aveva offesa. - Lo so quello che voleva dire; voleva dire: è innaturale amare una creatura estranea quando non si ama la propria figlia. Cosa capisce lui dell'amore per i bambini, del mio amore per Serëza, che ho sacrificato per lui? Ma questo desiderio di farmi del male! No, ama un'altra donna, non può essere altrimenti''. 

E visto che, desiderando calmarsi, aveva compiuto di nuovo il giro dei pensieri da lei fatto tante volte ed era tornata all'irritazione di prima, inorridì di se stessa. ``Davvero non è possibile? Davvero non posso prender la cosa su di me? - si disse e cominciò di nuovo, daccapo. - È sincero, è onesto, mi ama. Io lo amo, a giorni uscirà il divorzio. E di che cosa c'è bisogno ancora? C'è bisogno di calma, di fiducia, e io prenderò la cosa su di me. Sì, adesso, quando verrà, dirò che sono io colpevole, quantunque non lo sia, e partiremo''. 

E, per non pensare più e per non cedere all'irritazione, sonò e fece portar dentro i bauli per mettervi la roba da mandare in campagna. 

Alle dieci venne Vronskij. 

\capitolo{XXIV}\label{xxiv-5} 

- Be', c'è stata allegria? - chiese lei, uscendogli incontro con un'espressione colpevole e mansueta nel viso. 

- Come al solito - egli rispose, comprendendo immediatamente da un solo sguardo, che ella era in buona disposizione d'animo. S'era già abituato a questi passaggi, e quel giorno ne era particolarmente contento, perché anche lui era nella migliore disposizione d'animo. 

- Che vedo! Questo sì, che va bene! - disse, indicando i bauli in anticamera. 

- Sì, bisogna partire. Sono andata a passeggio in vettura e si stava tanto bene, che m'è venuta voglia di andare in campagna. Perché non ti trattiene nulla, vero? 

- Non desidero che questo. Vengo subito e parleremo, mi cambio soltanto. Fa' portare il tè. 

Ed egli andò nello studio. 

C'era qualcosa di offensivo nell'aver egli detto: ``Questo sì, che va bene'' così come si dice a un bambino che abbia cessato di far capricci. E ancora più offensivo era quel contrasto fra il tono colpevole di lei e quello di lui sicuro di sé; e per un attimo ella sentì il desiderio di lotta che insorgeva in lei; ma, fatto uno sforzo su di sé, lo soffocò e accolse Vronskij ancora allegra. 

Quand'egli venne da lei, ella gli raccontò, ripetendo in parte delle parole preparate, la propria giornata e i suoi progetti per la partenza. 

- Sai, m'è venuta quasi un'ispirazione - ella diceva. - Perché aspettare il divorzio qui? Io non posso aspettare più. Non voglio sperare, non voglio sentir dire nulla del divorzio. Ho stabilito che questo non avrà più influenza sulla mia vita. Anche tu sei d'accordo? 

- Oh sì - egli disse, dopo aver guardato con inquietudine il viso di lei agitato. 

- E che avete fatto là? chi c'era? - diss'ella, dopo essere stata un poco zitta. 

Vronskij nominò gli ospiti. 

- Il pranzo è stato ottimo, e la regata delle imbarcazioni e tutto è stato abbastanza simpatico, ma a Mosca non possono vivere senza il ridicule. È apparsa una certa signora, la maestra di nuoto della regina di Svezia, e ha fatto sfoggio della propria abilità. 

- Come? ha nuotato? - domandò Anna, accigliandosi. 

- In un certo costume de natation rosso, lei vecchia e deforme. E allora quando andiamo via? 

- Che fantasia sciocca! Be', nuota in modo particolare? - disse Anna senza rispondere. 

- Niente di speciale. Lo sto dicendo, una cosa tremendamente sciocca. E allora quando pensi di andar via? 

Anna scosse il capo, come desiderando di scacciare un pensiero spiacevole. 

- Quando andar via? Ma quanto prima, tanto meglio. Domani non faremo in tempo. Domani l'altro. 

- Sì\ldots{} no, aspetta. Domani l'altro è domenica, devo andare da maman - disse Vronskij, confuso, perché appena pronunciato il nome della madre, aveva sentito su di sé uno sguardo sospettoso e fisso. La confusione di lui confermò i sospetti di lei. Si accese in viso e si allontanò. Adesso non era più la maestra di nuoto della regina di Svezia che appariva ad Anna, ma la principessina Sorokina, che abitava in campagna nei dintorni di Mosca, insieme alla contessa Vronskaja. 

- Puoi andare domani - disse. 

- Ma no. Per l'affare per cui vado, le procure e i denari non si riscuotono domani - egli rispose. 

- Se è così, non partiremo affatto. 

- Ma perché? 

- Io non vado via più tardi. O lunedì o mai più. 

- E perché mai? - disse Vronskij, quasi con stupore. - Non ha mica senso questo! 

- Per te questo non ha senso, perché non ti importa nulla di me. Tu non vuoi capire la mia vita. L'unica cosa che mi occupava qui era Hanna. Tu dici che è una finzione. Hai pure detto ieri che non amavo mia figlia, ma che fingevo di amare questa inglese, il che era innaturale; io vorrei sapere quale vita qui può essere naturale per me. 

Per un attimo ella ritornò in sé e inorridì d'esser venuta meno alla propria intenzione. Ma, anche sapendo di rovinarsi, non riusciva a trattenersi, non poteva non fargli vedere come egli avesse torto, non poteva sottomettersi a lui. 

- Io non ho mai detto questo, ho detto che non avevo simpatia per quest'amore improvviso. 

- Perché tu, che ti vanti della tua dirittura, non dici la verità? 

- Io non mi vanto mai e non dico mai quello che non è vero - egli disse piano, trattenendo l'ira che si sollevava in lui. - È un gran peccato, se tu non rispetti\ldots{} 

- Il rispetto l'hanno inventato per nascondere il vuoto là dove dovrebbe essere l'amore\ldots{} E se tu non mi ami più è meglio ed è più onesto dirlo. 

- No, questo diventa insopportabile! - gridò Vronskij, alzandosi dalla sedia. E, fermatosi dinanzi a lei, pronunciò adagio: - Perché metti a prova la mia pazienza? - disse con un tono tale come se avesse voluto dire molte cose, e si contenesse. - Essa ha dei limiti. 

- Che volete dire con questo? - ella gridò, esaminando con orrore la esplicita espressione di odio che era su tutto il viso di lui e in particolare negli occhi crudeli, minacciosi. 

- Voglio dire\ldots{} - egli voleva cominciare, ma si fermò. - Posso sapere, che cosa volete da me? 

- Che cosa posso volere? Posso volere soltanto che non mi abbandoniate, come avete in mente - diss'ella, comprendendo tutto quello ch'egli non aveva detto fino in fondo. - Ma questo non lo voglio, è secondario. Io voglio l'amore e l'amore non c'è. Perciò tutto è finito. 

Ella si diresse verso la porta. 

- Aspetta! Aspetta! - disse Vronskij, senza distendere la piega cupa delle sopracciglia e fermandola per un braccio. - Di che si tratta? Io ho detto che bisogna rimandare la partenza di tre giorni, tu in risposta a questo hai detto che mento, che sono un uomo disonesto. 

- Sì, e ripeto che l'uomo che mi rinfaccia di aver sacrificato tutto per me - disse ella ricordando le parole amare della lite di prima - è peggiore di un uomo disonesto, è un uomo senza cuore. 

- No, ci sono dei limiti alla pazienza - egli gridò e lasciò andare rapidamente il braccio di lei. 

``Egli mi odia, è chiaro'' ella pensò, e in silenzio, senza voltarsi, a passi incerti uscì dalla stanza. 

``Ama un'altra donna, è ancor più chiaro - ella si diceva, entrando in camera sua. - Io voglio amore e amore non c'è. Perciò tutto è finito - ella ripeteva le parole già dette - e bisogna definire''. 

``Ma come?'' si domandò e sedette su di una poltrona dinanzi allo specchio. 

I pensieri su dove sarebbe andata adesso, se dalla zia presso la quale era stata allevata, da Dolly, o semplicemente sola all'estero, e su quello che faceva adesso lui nello studio, solo, se questo era un litigio definitivo, o se era possibile ancora far la pace, e su quello che adesso avrebbero detto di lei tutte le antiche conoscenti di Pietroburgo, come avrebbe visto la cosa Aleksej Aleksandrovic, e molti altri pensieri su quello che sarebbe accaduto, dopo la rottura, le venivano in mente, ma ella non si abbandonava con tutta l'anima a questi pensieri. Nella sua anima c'era un certo pensiero confuso, che la interessava unicamente, ma di cui non riusciva a rendersi conto. Ricordando ancora una volta Aleksej Aleksandrovic, ricordò anche il tempo della propria malattia, dopo il parto, e quel sentimento che allora non la lasciava. ``Perché non sono morta?''. Le tornavano in mente le sue parole di allora e il sentimento di allora. E a un tratto capì quello che c'era nell'anima sua. Sì, era quel pensiero solo che risolveva tutto. ``Sì, morire! E la vergogna e l'infamia di Aleksej Aleksandrovic e di Serëza, e la mia orribile vergogna, tutto si salva con la morte. Morire, e lui si pentirà, avrà pietà, amerà, soffrirà per me''. Con un sorriso di compassione verso se stessa fisso sul viso, ella sedeva nella poltrona, togliendo e infilando gli anelli dalla mano destra, figurandosi con chiarezza, sotto vari aspetti, i sentimenti di lui dopo la sua morte. 

Dei passi che si avvicinavano, i passi di lui, la distrassero. Come se fosse occupata nel mettere a posto gli anelli, ella non si voltò neppure verso di lui. 

Egli le si accostò e, presala per una mano, disse piano: 

- Anna, andiamo via domani l'altro, se vuoi. Acconsento a tutto. 

Ella taceva. 

- Ebbene? - egli domandò. 

- Lo sai tu stesso - diss'ella e nello stesso momento, non avendo più la forza di contenersi, si mise a singhiozzare. 

- Lasciami, lasciami! - ella diceva fra i singhiozzi. - Domani parto\ldots{} Farò di più. Chi sono? sono una donna perduta. Una pietra al tuo collo. Non voglio tormentarti, non voglio! Ti libererò. Tu non mi ami, tu ami un'altra! 

Vronskij la supplicava di calmarsi e la rassicurava che la sua gelosia non aveva un'ombra di fondamento, che non aveva mai cessato di amarla e che l'amava più di prima. 

- Anna, perché tormentare te e me? - egli diceva, baciandole le mani. Sul viso di lui, adesso, si esprimeva la tenerezza, e a lei sembrava di sentire con l'orecchio il suono delle lacrime nella voce di lui e sulla propria mano ne sentiva l'umidore. E in un attimo, la disperata gelosia di Anna si cambiò in una disperata, appassionata tenerezza: lo abbracciò, gli coprì di baci la testa, il collo, le mani. 

\capitolo{XXV}\label{xxv-5} 

Sentendo che la riconciliazione era avvenuta in pieno, Anna fin dalla mattina si mise con lena a fare i preparativi per la partenza. Sebbene non si fosse deciso se partivano il lunedì o il martedì (il giorno prima avevano ceduto l'una all'altro), Anna si preparava con cura alla partenza, sentendosi ormai del tutto indifferente al fatto che andassero via un giorno prima o dopo. Era in piedi nella stanza, curva su di un baule aperto, scegliendo le sue cose, quando egli entrò da lei, già vestito, prima del solito. 

- Vado subito da maman, il denaro me lo può mandare per mezzo di Egor. E domani sono pronto ad andare via - egli disse. 

Per quanto ella fosse di buonumore, il ricordo della gita in campagna la punse. 

- No, tanto neanch'io farò in tempo - ella disse subito, e pensò: ``allora si potevano disporre le cose in modo da fare come volevo io''. - No, fa' come volevi. Va' in sala da pranzo, vengo subito, devo scegliere fra questa roba inutile - diss'ella passando ancora qualcosa sul braccio di Annuška, sul quale c'era già una montagna di roba. 

Vronskij mangiava la sua bistecca, quand'ella entrò in sala da pranzo. 

- Non puoi immaginare come siano diventate prive di attrattiva per me queste stanze - ella disse, sedendosi accanto a lui davanti al proprio caffè. - Non c'è niente di più detestabile di queste chambres garnies. Non c'è espressione, non c'è anima. Quest'orologio, le tende e soprattutto le tappezzerie sono un incubo. Penso a Vozdvizenskoe come alla terra promessa. Non mandi via anche i cavalli? 

- No, andranno via dopo di noi. E tu vai in qualche posto? 

- Volevo andare dalla Wilson. Ho portato dei vestiti da lei. Allora proprio domani - disse con voce gaia; ma a un tratto il suo viso cambiò. 

Il cameriere di Vronskij venne a chiedere la ricevuta di un telegramma da Pietroburgo. Non c'era nulla di speciale che Vronskij ricevesse un telegramma, ma egli, come desiderando di nasconderle qualcosa, disse che la ricevuta era nello studio e si voltò con premura verso di lei. 

- Domani finirò tutto assolutamente. 

- Di chi è il telegramma? - ella domandò, senz'ascoltarlo. 

- Di Stiva - egli rispose controvoglia. 

- E perché non me l'hai fatto vedere? Che mistero può esistere tra Stiva e me? 

Vronskij fece tornare il cameriere e ordinò di portare il telegramma. 

- Non te lo volevo far vedere perché Stiva ha la mania dei telegrammi; perché telegrafare, quando nulla è deciso? 

- Per il divorzio? 

- Sì, ma lui scrive: ``Non ho potuto ancora ottenere nulla. A giorni ha promesso una risposta definitiva''. Ma ecco, leggi. 

Con le mani tremanti Anna prese il telegramma e lesse quelle stesse cose che Vronskij aveva detto. Alla fine, era ancora aggiunto: ``C'è poca speranza, ma farò il possibile e l'impossibile''. 

- Ieri ho detto che per me era proprio lo stesso ottenere e anche non ottenere il divorzio - diss'ella, arrossendo. - Non c'era nessun bisogno di nascondermelo. - ``Così egli può nascondere e nasconde la sua corrispondenza con le donne'' ella pensò. 

- E Jašvin voleva venire stamattina con Vojtov - disse Vronskij; - pare che abbia vinto a Pevcov tutto, e anche più di quello che lui può pagare, intorno ai sessantamila rubli. 

- No - ella disse, irritata perché lui, con questo mutar di discorso, le mostrava chiaramente ch'ella era irritata - perché mai pensi che questa notizia mi interessi tanto da dovermela perfino nascondere? Io ho detto che non voglio pensarci, e desidero che tu te ne interessi tanto poco quanto me. 

- Io me ne interesso perché mi piace la chiarezza - egli disse. 

- La chiarezza non è nella forma, ma nell'amore - ella disse, irritandosi sempre più non per le parole ma per il tono di fredda calma con cui egli parlava. - Perché lo desideri? 

``Dio mio, di nuovo a parlare dell'amore'' egli pensò, facendo una smorfia. 

- Ma lo sai perché: per te e per i figli che ci saranno - egli disse. 

- Figli non ce ne saranno. 

- È un gran peccato - egli disse. 

- Tu hai bisogno dei figli: e a me non pensi? - ella disse, avendo completamente dimenticato o non avendo sentito ch'egli aveva detto: per te e per i figli. 

La questione della possibilità di avere figli da lungo tempo era in discussione e la irritava. Il desiderio di lui di avere figli ella lo attribuiva al fatto ch'egli non apprezzasse la sua bellezza. 

- Ah, io ho detto: per te. Soprattutto per te - egli replicò, facendo una smorfia quasi di dolore - perché sono sicuro che la maggior parte della tua irritazione proviene dalla indeterminatezza della situazione. 

``Sì, ecco, adesso ha smesso di fingere, e si vede tutto il suo freddo odio verso di me'' ella pensò, senza ascoltare le parole, ma esaminando con orrore quel freddo e crudele giudice che, stuzzicandola, guardava dagli occhi di lui. 

- La ragione non è questa - ella disse - e io non capisco neppure come la causa di quella che tu chiami mia irritazione possa essere il fatto ch'io sia completamente in tuo potere. Che indeterminatezza di situazione c'è mai qui? al contrario. 

- Mi spiace molto che tu non voglia capire - la interruppe lui, desiderando d'esprimere il proprio pensiero: - l'indeterminatezza consiste nel fatto che a te pare ch'io sia libero. 

- Riguardo a questo puoi essere completamente tranquillo - ella disse e, voltategli le spalle, si mise a bere il caffè. 

Sollevò la tazza, staccando il mignolo, e l'accostò alla bocca. Dopo averne bevuti alcuni sorsi ella lo guardò e, dall'espressione del viso di lui, capì chiaramente che gli erano disgustosi la mano e il gesto e il suono ch'ella produceva con le labbra. 

- Per me è proprio indifferente quello che pensa tua madre e in quale maniera voglia darti moglie - ella disse, deponendo la tazza con la mano tremante. 

- Ma noi non parliamo di questo. 

- No, proprio di questo. E credi pure che per me una donna senza cuore, sia vecchia o no, tua madre o un'estranea, non m'interessa, e io non ne voglio sapere. 

- Anna, ti prego, non parlare senza rispetto di mia madre. 

- Una donna che non ha indovinato col cuore in che cosa consista la felicità e l'onore di suo figlio, quella donna non ha cuore. 

- Ti ripeto la mia preghiera: non parlare senza rispetto d'una madre che io rispetto - diss'egli, alzando la voce e guardandola severo. 

Ella non rispondeva. Guardando fisso lui, il suo viso, le sue mani, ricordò tutti i particolari della riconciliazione del giorno prima, e le carezze appassionate di lui. ``Queste carezze, proprio le stesse, le ha prodigate, le prodigherà e le vuol prodigare ad altre donne!'' ella pensava. 

- Tu non ami tua madre. Sono tutte frasi, frasi e frasi! - disse lei, guardandolo con odio 

- E se è così, allora bisogna\ldots{} 

- Bisogna decidersi e io mi son decisa - ella disse, e voleva andarsene, ma intanto entrò nella stanza Jašvin. Anna lo salutò e si fermò. 

Perché in quel momento, in cui nell'anima sua c'era tempesta ed ella sentiva d'essere a una svolta della vita che poteva avere orribili conseguenze, perché proprio in quel momento ella avesse bisogno di fingere davanti a un essere estraneo, che presto o tardi avrebbe pur saputo tutto, non lo sapeva; ma, calmata immediatamente in sé la tempesta interiore, si sedette e prese a parlare con l'ospite. 

- Be', come va il vostro affare? avete avuto il vostro credito? - disse a Jašvin. 

- Ma nulla; pare che non riceverò tutto, e mercoledì bisogna andar via. E voi quando? - disse Jašvin, guardando accigliato Vronskij e indovinando, evidentemente, la lite avvenuta. 

- Sembra domani l'altro - disse Vronskij. 

- Voi, del resto, vi preparate da un pezzo. 

- Ma ormai decisamente - disse Anna, guardando diritto negli occhi Vronskij con uno sguardo tale che gli diceva di non pensare neppure alla possibilità di una riconciliazione. - Possibile che non vi faccia pena quel disgraziato di Pevcov? - continuò la conversazione con Jašvin. 

- Non mi sono mai domandato, Anna Arkad'evna, se mi faceva pena o non mi faceva pena. Perché tutto il mio patrimonio è qui - egli mostrò la tasca laterale - e adesso sono un uomo ricco; ma oggi andrò al club e forse ne uscirò pezzente. Perché quegli che siede al tavolo con me vuol lasciarmi senza la camicia, e io lui. Lottiamo, in questo sta il gusto. 

- Via, e se foste ammogliato? - disse Anna - come farebbe vostra moglie? 

Jašvin si mise a ridere. 

- Proprio per questo, si vede, non mi sono ammogliato, e non ne ho mai avuta l'intenzione. 

- E Helsingfors? - disse Vronskij, entrando nella conversazione e guardando Anna che aveva sorriso. Nell'incontrare lo sguardo di lui, il viso di Anna, d'un tratto, prese un'espressione dura, come a dirgli: ``Non è dimenticato. È sempre lo stesso''. 

- Possibile che non siate stato mai innamorato? - ella disse a Jašvin. 

- Oh Signore! quante volte! Ma, capirete, uno può sedersi a giocare a carte, pronto ad alzarsi quando è l'ora d'un rendez-vous. Io, invece, posso occuparmi d'amore solo per quel tempo che mi consenta di non arrivare in ritardo la sera alla partita. Sistemo sempre così le cose. 

- No, non domando di questo, ma di quello che è stato. - Ella avrebbe voluto dire ``Helsingfors'', ma non voleva ripetere una parola detta da Vronskij. 

Venne Vojtov, che aveva comprato uno stallone; Anna si alzò e uscì dalla stanza. 

Prima di andar via, Vronskij passò da lei. Ella voleva fingere di cercar qualcosa sulla tavola, ma, vergognandosi di fingere, lo guardò diritto in faccia con uno sguardo freddo. 

- Di che cosa avete bisogno? - gli domandò in francese. 

- Di prendere il certificato per Gambetta, l'ho venduto - egli disse con un tono tale che esprimeva più chiaramente delle parole: ``per spiegarmi non ho tempo e non porterebbe a nulla''. 

``Io non sono colpevole in nulla verso di lei - egli pensava. - Se vuole punirsi, tant pis pour elle''. Ma uscendo, gli sembrò ch'ella avesse detto qualcosa, e il suo cuore tremò di pena per lei. 

- Cosa, Anna? - egli domandò. 

- Io, nulla - ella rispose con altrettanta freddezza e calma. 

``Ebbene, se è nulla, allora tant pis'' egli pensò, divenuto di nuovo freddo, si voltò e uscì. Uscendo vide nello specchio il viso di lei, pallido, con le labbra tremanti. Voleva fermarsi e dirle una parola per consolarla, ma le gambe lo portarono via dalla stanza, prima che avesse pensato cosa dire. Tutta quella giornata la passò fuori di casa e, quando venne la sera tardi, la donna gli disse che Anna Arkad'evna aveva mal di capo e lo pregava di non entrare da lei. 

\capitolo{XXVI}\label{xxvi-5} 

Non era ancora mai passato un intero giorno in lite. Quel giorno era la prima volta. E non era una lite. Era l'evidente ammissione d'un definitivo raffreddamento. Le si poteva forse lanciare uno sguardo quale egli le aveva lanciato quando era entrato nella stanza a prendere il certificato? Guardarla, vedere che il suo cuore si spezzava di disperazione e passarle accanto con quel viso impassibile e calmo? Non solo egli si era raffreddato verso di lei, ma la odiava, perché amava un'altra donna, era chiaro. 

E, ricordando tutte le parole crudeli ch'egli le aveva detto, Anna inventava ancora le parole che, evidentemente, egli avrebbe desiderato e potuto dirle, e s'irritava ancora di più. 

``Io non vi trattengo - egli poteva dirle. - Potete andare dove volete. Non avete voluto divorziare da vostro marito per tornare a lui, probabilmente. Tornate. Se avete bisogno di denaro, ve ne darò. Di quanti rubli avete bisogno?''. 

Tutte le parole più crudeli che può dire un uomo volgare, egli le diceva a lei nell'immaginazione sua, e lei non gliele perdonava, come se realmente egli gliele avesse dette. 

``E non è appena ieri che m'ha giurato amore, lui, uomo sincero e onesto? Non mi son forse disperata senza ragione già altre volte?'' si diceva dopo. 

Tutta quella giornata, tranne le due ore che passò dalla Wilson, Anna visse nel dubbio se tutto era finito o se c'era speranza di rappacificarsi, se doveva partire subito o vederlo ancora una volta. L'aveva aspettato tutto il giorno, e la sera, ritirandosi in camera sua, dopo aver ordinato di dire che aveva mal di capo, aveva pensato: ``Se egli verrà, malgrado le parole della cameriera, allora vuol dire che mi ama ancora. Altrimenti vuol dire che tutto è finito e allora deciderò quello che devo fare!''. 

La sera sentì il rumore del carrozzino di lui che si fermava, sentì la sua scampanellata, i suoi passi e la conversazione con la donna: egli aveva creduto quanto gli dicevano, non aveva voluto indagare ed era andato in camera sua. Tutto era finito, dunque. 

E la morte, come l'unico mezzo per far tornare nel cuore di lui l'amore, per punirlo e riportare vittoria in quella lotta che lo spirito del male, stabilitosi nel cuore di lei, conduceva con lui, le apparve chiaramente e con vivezza. 

Adesso era indifferente: andare o non andare a Vozdvizenskoe, ricevere o non ricevere il divorzio dal marito, tutto era inutile. Una cosa sola era necessaria: punirlo. 

Quando ebbe versato la solita dose d'oppio ed ebbe pensato che bastava soltanto bere tutta la fiala per morire, questo le parve così facile e semplice, che si mise a pensare di nuovo con piacere come egli si sarebbe tormentato, pentito, come avrebbe amato la sua memoria, quando sarebbe stato ormai troppo tardi. Ella giaceva nel letto con gli occhi aperti, guardando, alla luce di una candela che stava per spegnersi, la cornice modellata del soffitto e l'ombra di un paravento che ne invadeva una parte, e immaginava con chiarezza quello ch'egli avrebbe provato quando lei non ci sarebbe stata più e sarebbe stata soltanto un ricordo per lui. ``Come ho potuto dirle quelle parole crudeli? - avrebbe detto. - Come ho potuto uscir dalla stanza senza dirle nulla? Ma adesso lei non c'è più. Se n'è andata per sempre da noi. È là\ldots{}''. A un tratto l'ombra del paravento tentennò, invase tutta la cornice, tutto il soffitto, altre ombre dall'altra parte le si precipitarono incontro, per un attimo le ombre corsero via, ma poi avanzarono con rinnovata rapidità, tentennarono un po', si confusero, e tutto si fece buio. ``La morte!'' pensò. E un tale terrore la prese, che a lungo non poté capire dov'era e a lungo non poté trovare con le mani tremanti i fiammiferi e accendere un'altra candela al posto di quella che s'era consumata e spenta. ``No, tutto pur di vivere! Perché io l'amo. Perché lui mi ama! Questo è stato e passerà'' ella diceva, sentendo che le lacrime della gioia del ritorno alla vita le scorrevano per le guance. E, per liberarsi dal terrore, andò in fretta da lui nello studio. 

Nello studio egli dormiva di un sonno profondo. Gli si avvicinò e, illuminandogli il viso dall'alto, lo guardò a lungo. Adesso, quando dormiva, lo amava tanto che nel vederlo non poteva trattenere le lacrime di tenerezza; ma sapeva che, svegliandosi, l'avrebbe guardata con uno sguardo freddo, cosciente di aver ragione, e che, prima di parlargli del proprio amore, ella non avrebbe potuto non dimostrargli come egli fosse colpevole dinanzi a lei. Tornò in camera sua senza svegliarlo, e dopo una seconda dose di oppio, verso l'alba, si addormentò di un sonno pesante, non pieno, durante il quale non cessò di sentire se stessa. 

La mattina un incubo pauroso, che le era apparso varie volte nei sogni, ancora prima del suo legame con Vronskij, le apparve di nuovo e la fece svegliare. Un vecchietto con la barba arruffata faceva qualcosa, chino su di un ferro, mentre diceva delle parole francesi senza senso, e lei come sempre in quell'incubo (ciò che ne formava proprio l'orrore), sentiva che quel vecchio non faceva nessun caso a lei. E si svegliò coperta di un sudore freddo. 

Quando si fu alzata, le venne in mente, come in una nebbia, la giornata precedente. 

``C'è stata una lite. C'è stato quello che è già accaduto altre volte. Io ho detto che avevo mal di capo, e lui non è entrato. Domani andiamo via, bisogna vederlo e prepararsi per la partenza'' ella si disse. E avendo saputo ch'egli era già nello studio, andò da lui. Passando per il salotto sentì fermarsi all'ingresso una vettura, e, guardando dalla finestra, vide una vettura dalla quale si affacciava una fanciulla con un cappellino lilla, che ordinava qualcosa al cameriere che bussava. Dopo un parlottio in anticamera, qualcuno andò su, e, accanto al salotto, si sentirono i passi di Vronskij. Egli scendeva le scale a passo svelto. Ecco, era uscito senza cappello sulla scalinata e s'era avvicinato alla vettura. La fanciulla col cappellino lilla gli consegnò un pacchetto. Vronskij le disse qualcosa sorridendo. La vettura si allontanò; lui tornò, correndo rapido su per la scala. 

La nebbia, che avvolgeva tutto nell'animo di lei, si dissipò a un tratto. I sentimenti del giorno prima strinsero con rinnovato dolore il cuore malato. Adesso non poteva capire come avesse potuto umiliarsi tanto da passare tutta una giornata con lui, in casa sua. Ella entrò nello studio per annunciargli la propria decisione. 

- È la Sorokina con la figlia che è passata e m'ha portato i denari e le carte da parte di Ma. Ieri non ho potuto riceverli. Come va il tuo mal di capo, meglio? - egli disse tranquillo, senza desiderar di scorgere e intendere l'espressione cupa e grave del viso di lei. 

Ella lo guardava in silenzio, fissa, rimanendo in piedi al centro della stanza. Egli la guardò, si accigliò per un attimo e seguitò a leggere una lettera. Lei si voltò e andò via lentamente dalla stanza. Egli poteva ancora farla tornare, ma ella giunse fino alla porta, e lui taceva sempre, e si sentiva soltanto il fruscio del foglio di carta girato. 

- Sì, a proposito - disse egli, mentre lei era già sulla porta - domani andiamo via decisamente, vero? 

- Voi, ma non io - ella disse, voltandosi verso di lui. 

- Anna, così è impossibile vivere\ldots{} 

- Voi, ma non io - ella ripeté. 

- Diventa insopportabile! 

- Voi\ldots{} voi ve ne pentirete - ella disse e uscì. 

Spaventato dall'espressione disperata con cui erano state dette queste parole, egli saltò su e voleva correrle dietro, ma, ritornato in sé, sedette di nuovo e, stretti fortemente i denti, aggrottò le sopracciglia. Quella minaccia, informe, com'egli la riteneva, d'un qualche cosa, lo irritò. ``Ho provato tutto - pensò - rimane una cosa sola: non farci caso'' e cominciò a prepararsi ad andare in città e di nuovo dalla madre, dalla quale bisognava ricevere la firma per le procure. 

Ella sentì il suono dei passi di lui nello studio e nella sala da pranzo. Vicino al salotto egli si fermò. Ma non voltò per andare da lei, diede soltanto l'ordine che consegnassero lo stallone a Vojtov in sua assenza. Poi ella sentì come facevano venire avanti il carrozzino, come si apriva la porta ed egli ne usciva di nuovo. Ma ecco, egli rientrava nel vestibolo, e qualcuno veniva su di corsa. Era il cameriere che veniva a prendere i guanti dimenticati. Ella si avvicinò alla finestra e vide che, senza guardare, egli prendeva i guanti e, toccata con la mano la schiena del cocchiere, diceva qualcosa. Poi, senza guardare le finestre, sedette nella sua solita posa nel carrozzino, poggiando una gamba sull'altra, e, infilando un guanto, scomparve dietro l'angolo. 

\capitolo{XXVII}\label{xxvii-5} 

``È andato via! È finita!'' si disse Anna, in piedi accanto alla finestra, e, in risposta a questo problema, le impressioni del buio, per la candela che s'era spenta, e del sogno terribile si fusero in una, riempiendole il cuore di fredda paura. 

``No, questo è impossibile!'' gridò e, attraversata la stanza, sonò. Le sembrava così pauroso restar sola ora, che, senza aspettare che giungesse il cameriere, gli andò incontro. 

- Informatevi dove è andato il conte - ella disse. 

L'uomo rispose che il conte era andato alle scuderie. 

- Ha ordinato di dirvi che se desiderate uscire, il carrozzino ritornerà subito. 

- Va bene. Aspettate. Scrivo subito un biglietto. Mandate Michajla col biglietto alle scuderie. Presto. 

Sedette e scrisse: 

``Sono colpevole. Torna a casa, dobbiamo spiegarci. In nome di Dio vieni, sono spaventata''. 

Suggellò e consegnò all'uomo. 

Aveva paura di rimaner sola, adesso, e, dietro all'uomo, uscì dalla stanza e andò in quella dei bambini. 

``Ma non è lui, non è lui! Dove sono i suoi occhi azzurri, il caro e timido sorriso?'' questo fu il primo suo pensiero, quando vide la bambina, rossa e paffuta, con i capelli neri ondulati, invece di Serëza che, nella confusione delle idee, ella s'aspettava di vedere nella camera dei bambini. La bimba, sedendo alla tavola, la batteva con forza e ostinazione con un turacciolo, e guardava senza espressione la madre con due occhi neri simili a more. Dopo aver risposto all'inglese che si sentiva bene e che l'indomani partiva per la campagna, Anna sedette accanto alla piccina e cominciò a far girare davanti a lei il turacciolo della caraffa. Ma il riso forte e sonoro della bambina e un movimento ch'ella fece con un sopracciglio le ricordarono con tanta vivezza Vronskij che, trattenendo i singhiozzi, si alzò in fretta e uscì. ``Possibile che tutto sia finito? No, non può essere - ella pensava - egli tornerà. Ma come mi spiegherà quel sorriso, quell'animazione dopo aver parlato con lei? Ma anche se non lo spiegherà, tuttavia ci crederò. Se non ci crederò, allora mi rimane una cosa sola\ldots{} e non voglio''. 

Guardò l'orologio. Erano passati dodici minuti. ``Adesso ha già ricevuto il biglietto e torna indietro. Ma è poco, ancora dieci minuti\ldots{} Ma cosa sarà se non viene? No, questo non può essere. Bisogna che non mi veda con gli occhi rossi di pianto. Andrò a lavarmi. Sì, sì, mi sono pettinata o no? - si domandò. Tastò la testa con la mano. - Sì, mi sono pettinata, ma quando, non lo ricordo assolutamente''. Non credeva neppure alla propria mano e si avvicinò alla specchiera per veder se era veramente pettinata o no. Era pettinata e non poteva ricordare quando l'avesse fatto. 

``Chi è?'' pensava, guardando nello specchio il proprio viso infiammato, con gli occhi stranamente scintillanti, che la fissavano con spavento. ``Ma sono io'' ella capì a un tratto e, osservandosi tutta, sentì su di sé i baci di lui e, rabbrividendo, scosse le spalle. Poi sollevò una mano alle labbra e la baciò. 

``Cos'è impazzisco?'' e andò nella stanza da letto dove c'era Annuška che rassettava la camera. 

- Annuška - disse, fermandosi davanti a lei, guardando la cameriera, senza sapere lei stessa quel che le avrebbe detto. 

- Volevate andare da Dar'ja Aleksandrovna - disse la cameriera come se avesse capito. 

- Da Dar'ja Aleksandrovna? Sì, andrò. 

``Quindici minuti per andare, quindici per tornare indietro. Egli sta per venire, arriverà subito - e tirò fuori l'orologio e lo guardò. - Ma come ha potuto andar via, lasciandomi in uno stato simile? Come può vivere senza far pace con me?''. Si avvicinò alla finestra e si mise a guardare per la strada. Come tempo, egli avrebbe potuto già tornare. Ma il calcolo poteva non essere giusto, ed ella si diede a ricordare nuovamente quando era andato via, e a calcolare i minuti. 

Mentre si allontanava verso la pendola grande per controllare l'ora, qualcuno giunse in vettura. Guardando dalla finestra vide il carrozzino di lui. Ma nessuno saliva la scala e giù si sentivano delle voci. Era l'inserviente che tornava col carrozzino. Ella gli scese incontro. 

- Il conte non s'è trovato. Era partito per la linea di Niznij-Novgorod. 

- Che c'è? cosa?\ldots{} - ella disse rivolta al rubizzo e allegro Michajla che le rendeva il biglietto. ``Ma lui dunque non l'ha ricevuto'' ella ricordò. 

- Va' con questo stesso biglietto, in campagna, dalla contessa Vronskaja, sai? E porta immediatamente la risposta - disse all'inserviente. 

``E io che farò mai? - pensò. - Sì, andrò da Dolly, è vero, se no impazzisco. Sì, posso ancora telegrafare''. E scrisse un telegramma: 

``Mi è indispensabile parlarvi, venite subito''. 

Spedito il telegramma andò a vestirsi. Già pronta e col cappello, guardò negli occhi Annuška, grassa e calma. Si vedeva una compassione manifesta in quei piccoli, buoni occhi grigi. 

- Annuška, cara, che devo fare? - disse Anna, singhiozzando, lasciandosi andare sopra una poltrona. 

- Ma perché vi inquietate tanto, Anna Arkad'evna! Questo succede. Andate, vi distrarrete - disse la cameriera. 

- Sì, andrò - disse Anna tornando in sé e alzandosi. - E se verrà un telegramma quando non ci sarò, mandatemelo da Dar'ja Aleksandrovna. No, tornerò io stessa. 

``Sì, non bisogna stare a pensare, bisogna fare qualcosa, andare, soprattutto, andar via da questa casa - ella disse, prestando ascolto con orrore al terribile ribollimento che avveniva nel suo cuore, e in fretta uscì e salì nel carrozzino. 

- Dove comandate? - Domandò Pëtr, prima di sedersi a cassetta. 

- Alla Znamenka, dagli Oblonskij. 

\capitolo{XXVIII}\label{xxviii-5} 

Il tempo era chiaro. Tutta la mattina era caduta giù una pioggerella fitta, minuta, e adesso s'era schiarito da poco. I tetti di ferro, le lastre dei marciapiedi, i ciottoli del selciato, le ruote e il cuoio, il rame e lo stagno delle carrozze, tutto luccicava vivido al sole di maggio. Erano le tre ed era l'ora più animata per le strade. 

Sedendo in un angolo del comodo carrozzino, che si dondolava appena sulle molle elastiche all'andatura veloce dei cavalli grigi, Anna, in mezzo al frastuono incessante delle ruote e alle impressioni che si succedevano rapide all'aria aperta, esaminando di nuovo uno dopo l'altro gli avvenimenti degli ultimi giorni, vide la propria situazione completamente diversa da come le era sembrata a casa. Adesso anche il pensiero della morte non le sembrava più così terribile e chiaro, e la morte stessa non le appariva più inevitabile. Adesso si rimproverava l'umiliazione alla quale s'era lasciata andare. ``Lo supplico di perdonarmi. Mi sono sottomessa a lui. Mi sono riconosciuta colpevole. Perché? Non posso forse vivere senza di lui?''. E senza rispondere alla domanda come avrebbe vissuto senza di lui, si mise a leggere le insegne. ``Ufficio e deposito. Dentista\ldots{} Sì, dirò tutto a Dolly. Vronskij non le piace. Proverò vergogna, dolore, ma le dirò tutto. Lei mi vuole bene, e io seguirò il suo consiglio. Non mi assoggetterò a lui; non gli permetterò di plasmarmi. Filippov, ciambelle\ldots{} Dicono che portino la pasta a Pietroburgo. L'acqua di Mosca è così buona. E i pozzi e i biscotti di Mytišci''. 

E ricordò come molto tempo addietro, quando aveva appena diciassette anni, c'era andata con la zia per la Pentecoste. ``Ancora coi cavalli. Possibile che fossi io, con le mani rosse? Tante cose di quelle che allora mi sembravano così splendide e irraggiungibili sono diventate insignificanti, e quello che c'era allora, adesso è irraggiungibile per sempre. Avrei creduto, allora, di poter arrivare a tanta umiliazione? Come sarà orgoglioso e soddisfatto, per aver ricevuto il mio biglietto! Ma io gli dimostrerò\ldots{} Che cattivo odore ha questa vernice! Perché non fanno che verniciare e costruire? Mode e confezioni'' ella leggeva. Un uomo la salutò. Era il marito di Annuška. ``I nostri parassiti - ella ricordò come diceva Vronskij. - I nostri? Perché i nostri? È orribile che non si possa estirpare dalla radice il passato. Non si può estirpare, ma se ne può sperdere la memoria. E io la sperderò''. E a questo punto ricordò il suo passato con Aleksej Aleksandrovic, come l'avesse cancellato dalla propria memoria. ``Dolly penserà che io abbandono il secondo marito e che perciò certamente ho torto. Voglio forse aver ragione, io? Non posso!'' si disse, e le venne voglia di piangere. Ma si mise immediatamente a pensare di che cosa potessero sorridere tanto quelle due ragazze. ``Forse a proposito dell'amore? Non sanno come sia poco allegro, come sia vile\ldots{} Il viale e i bambini. Tre ragazzi corrono, giuocano ai cavalli. Serëza! E io perderò tutto e non farò tornare lui. Sì, tutto è perduto, s'egli non torna. Forse è arrivato in ritardo per il treno, e adesso è già tornato. Ecco, vuoi un'altra umiliazione! - disse a se stessa. - No, io andrò da Dolly e le dirò apertamente: sono infelice, me lo merito, sono colpevole, ma sono così infelice, aiutami! Questi cavalli, questo carrozzino, come mi vedo ripugnante in questo carrozzino. Tutto è suo, ma non vedrò più nulla''. 

Immaginando le parole con le quali avrebbe detto tutto a Dolly e avvelenandosi deliberatamente il cuore, Anna cominciò a salire la scala. 

- C'è qualcuno? - chiese in anticamera. 

- Katerina Aleksandrovna Levina - rispose il cameriere. 

``Kitty, quella stessa Kitty di cui è stato innamorato Vronskij! - pensò Anna. - Quella stessa che egli ricordava con amore. Si rammarica di non averla sposata. E di me si ricorda con odio, e si rammarica d'essersi unito a me''. 

Fra le sorelle, quando Anna arrivò, si parlava dell'allattamento. Dolly uscì sola incontro all'ospite, che in quel momento disturbava la loro conversazione. 

- Ah, non sei ancora partita? Volevo venire da te - disse - oggi ho ricevuto una lettera da Stiva. 

- Anche noi abbiamo ricevuto un telegramma - rispose Anna, voltandosi per vedere Kitty. 

- Scrive che non riesce a capire che cosa precisamente voglia Aleksej Aleksandrovic, ma che non partirà senza una risposta. 

- Pensavo che da te vi fosse qualcuno. Si può leggere la lettera? 

- Sì, Kitty - disse Dolly confusa - è rimasta nella camera dei bambini. È stata molto malata. 

- L'ho sentito. Si può leggere la lettera? 

- La porto subito. Ma egli non rifiuta; al contrario Stiva spera - disse Dolly, fermandosi sulla porta. 

- Io non spero, e non lo desidero neanche - disse Anna. 

``Cos'è mai questo? Kitty considera umiliante per lei incontrarmi? - pensava Anna rimasta sola. - E forse ha ragione. Ma non è lei, che è stata innamorata di Vronskij, non è lei che deve dimostrarmelo, anche se è vero. Lo so che, nella mia situazione, non mi può ricevere nessuna donna per bene. Lo so che da quel primo momento gli ho sacrificato tutto. Ed ecco la ricompensa! Oh, come lo odio! E perché son venuta qua? Sto ancora peggio, mi è ancora più penoso''. Ella sentiva nell'altra stanza le voci delle sorelle che parlavano fra di loro. ``E che cosa dirò a Dolly adesso? Devo consolare Kitty con la mia infelicità, sottomettendomi alla sua protezione? No, ma anche Dolly non capirà nulla. Ed è inutile che le parli. Sarebbe interessante soltanto veder Kitty e farle vedere come disprezzo tutti e tutto, come per me, adesso, tutto sia indifferente''. 

Dolly entrò con la lettera. Anna la lesse e la consegnò in silenzio. 

- Tutto questo lo sapevo - disse. - E non mi interessa affatto. 

- Ma perché poi? Io, al contrario, spero - disse Dolly, guardando Anna con curiosità. Non l'aveva mai vista in uno stato così strano, irritato. - Tu quando vai via? - ella domandò. 

Anna, socchiusi gli occhi, guardava davanti a sé e non le rispondeva. 

- Ebbene, Kitty si nasconde per non vedermi? - disse, guardando la porta e arrossendo. 

- Oh, che sciocchezze! Dà il latte e la cosa non procede bene, le stavo consigliando\ldots{} È molto contenta. Verrà subito - diceva Dolly con imbarazzo, non sapendo dire quello che non era vero. - Ma eccola. 

Avendo saputo che era venuta Anna, Kitty non voleva venir fuori; ma Dolly l'aveva persuasa. Raccolte le proprie forze, Kitty comparve, e, arrossendo, si avvicinò e le diede la mano. 

- Sono molto contenta - disse con voce tremante. 

Kitty era sconcertata dalla lotta che avveniva in lei fra l'inimicizia verso quella donna perversa e il desiderio di esserle indulgente; ma non appena vide il viso bello, simpatico di Anna, tutta l'inimicizia scomparve immediatamente. 

- Non mi sarei sorpresa se non aveste neppure voluto incontrarvi con me. Sono abituata a tutto. Siete stata malata? Sì, siete cambiata - disse Anna. 

Kitty sentiva che Anna la guardava con ostilità. Ella spiegò questa ostilità con la situazione di disagio in cui si sentiva adesso, di fronte a lei, Anna che prima la proteggeva, e ne provò pena. 

Parlarono della malattia, del bambino, di Stiva, ma evidentemente nulla interessava Anna. 

- Sono passata a salutarti - diss'ella, alzandosi. 

- E quando andate via? 

Ma Anna si voltò di nuovo verso Kitty, senza rispondere. 

- Sì, sono molto contenta d'avervi vista - ella disse con un sorriso. - Ho tanto sentito parlare di voi da tutte le parti, perfino da vostro marito. È stato da me, e m'è piaciuto molto - soggiunse con un'evidente intenzione perversa. - Dov'è? 

- È andato in campagna - disse Kitty, arrossendo. 

- Salutatelo da parte mia, salutatelo senza meno. 

- Senza meno! - ripeté ingenuamente Kitty, guardandola negli occhi con pena. 

- Allora addio, Dolly - e, baciata Dolly e stretta la mano a Kitty, Anna uscì frettolosa. 

- Sempre la stessa e sempre così affascinante. È molto bella! - disse Kitty , rimasta sola con la sorella. - Ma c'è qualcosa in lei che fa pena. Tanta pena! 

- No, oggi in lei c'è qualcosa di strano - disse Dolly. - Quando l'ho accompagnata in anticamera, m'è parso che avesse voglia di piangere. 

\capitolo{XXIX}\label{xxix-5} 

Anna sedette nel carrozzino in uno stato peggiore di quello in cui era uscita di casa. Ai tormenti di prima s'era unito adesso un senso di offesa e di ripulsione che aveva chiaramente avvertito nell'incontro con Kitty. 

- Dove andate? a casa? - chiese Pëtr. 

- Sì a casa - disse lei, senza neppur pensare dove andasse ora. 

``Come mi guardavano, come qualcosa di terribile, di incomprensibile e di curioso! Che cosa può raccontare quello lì all'altro con tanto calore? - ella pensava, guardando due passanti. - Si può forse raccontare ciò che si sente a un altro? Io volevo raccontarlo a Dolly ed è stato bene che non l'abbia fatto. Come sarebbe stata contenta della mia sventura! L'avrebbe nascosto; ma il sentimento principale sarebbe stata la gioia che io fossi punita per quel piacere che lei mi invidiava. Kitty poi sarebbe stata ancor più contenta. Come la vedo tutta da parte a parte! Sa che io sono stata più gentile del solito verso suo marito. Ed è gelosa di me e mi odia. E mi disprezza, per giunta. Ai suoi occhi io sono una donna immorale. Se fossi una donna immorale avrei potuto fare innamorare di me suo marito\ldots{} se avessi voluto. Ma io non lo volevo neanche. Quello lì è contento di sé - pensò di un signore grasso, rosso in viso, che veniva verso di lei in carrozza, il quale, scambiandola per una conoscente, aveva sollevato il cappello lucido sopra la lucida testa calva e poi s'era convinto d'essersi sbagliato. - Pensava di conoscermi. E mi conosce così poco, come poco mi conosce chiunque altro al mondo. Io stessa non mi conosco. Conosco i miei appetiti, come dicono i francesi. Ecco, loro desiderano questo gelato sporco. Questo, loro lo sanno con sicurezza - pensava, guardando due ragazzini che avevano fermato un gelataio, che si toglieva di capo il recipiente e s'asciugava con l'orlo dell'asciugamano il viso sudato. - Tutti noi desideriamo roba dolce, buona. Se non ci sono confetti, allora gelato sporco. E Kitty lo stesso: se non Vronskij, allora Levin. E mi invidia. E mi odia. E tutti noi ci odiamo a vicenda. Io Kitty, Kitty me. Ecco, questa è la verità. Tjut'kin, coiffeur\ldots{} Je me fais coiffer par Tjut'kin\ldots{} Glielo dirò, quando arriverà - pensò e sorrise. Ma nello stesso momento si ricordò che non aveva nessuno cui dire qualcosa di divertente. - E poi non c'è nulla di ameno, di allegro. Tutto è disgustoso. Suonano a vespro, e questo mercante si fa il segno della croce con tanta cura come se temesse di lasciarsi sfuggire qualcosa. Perché queste chiese, questo suono, questa menzogna? Soltanto per nascondere che ci odiamo tutti a vicenda, come questi vetturali che si ingiuriano con tanta cattiveria. Jašvin dice: `lui vuol lasciare me senza camicia, e io lui'. Ecco, questa è la verità!''. 

In questi pensieri, che l'avevano tanto presa da non farla pensare più alla propria situazione, la sorprese l'arrestarsi della carrozza vicino ai gradini di casa sua. Visto il cocchiere che le veniva incontro, ricordò soltanto allora d'aver spedito il telegramma e il biglietto. 

- C'è risposta? - domandò. 

- Guardo subito - rispose il portiere e, data un'occhiata al banco, tirò fuori e le porse la busta sottile, quadrata di un telegramma ``Non posso arrivare prima delle dieci. Vronskij'' ella lesse. 

- E l'inserviente non è tornato? 

- Nossignora - rispose il portiere. 

``E se è così, so quello che devo fare - ella disse e, sentendo insorgere in sé un'ira indefinita e un bisogno di vendetta, andò sopra di corsa. - Andrò io stessa da lui. Prima di partire per sempre gli dirò tutto. Non ho mai odiato nessuno come quest'uomo!'' ella pensava. Visto il cappello di lui all'attaccapanni, rabbrividì di repulsione. Non considerava che il telegramma di lui era la risposta al suo telegramma, e ch'egli non aveva ancora ricevuto il biglietto. Lo immaginava mentre con la madre e con la Sorokina discorreva tranquillo e gioiva delle sofferenze di lei. ``Sì, bisogna andare presto'' si disse senza pensare dove andare. Desiderava staccarsi al più presto dalle sensazioni che provava in quell'orribile casa. La servitù, i muri, gli oggetti, qui tutto suscitava in lei repulsione e rancore e l'opprimeva come un peso. 

``Sì, bisogna andare alla stazione ferroviaria, e se no, allora, andare là e coglierlo sul fatto''. Anna guardò nei giornali l'orario dei treni. La sera, il treno partiva alle otto e due minuti. ``Sì, farò in tempo''. Ordinò di attaccare altri cavalli e si occupò di mettere in una sacca da viaggio le cose indispensabili per qualche giorno. Sapeva che non sarebbe più tornata lì. Aveva deciso confusamente, fra i progetti che le erano venuti in mente, anche questo, che, dopo quanto sarebbe accaduto, alla stazione o nella tenuta della contessa Vronskaja, sarebbe andata per la linea di Niznij-Novgorod fino alla prima stazione e sarebbe rimasta là. 

Il pranzo era in tavola; ella si avvicinò, annusò il pane e il formaggio, e, convintasi che l'odore di tutti i cibi le riusciva nauseante, ordinò di far venire la vettura e uscì. La casa gettava un'ombra che attraversava ormai tutta la strada, ed era una serata chiara, ancora tiepida al sole. E Annuška che l'accompagnava con la roba, e Pëtr che riponeva la roba nel carrozzino e il cocchiere, evidentemente scontento, tutti la nauseavano e la irritavano con le loro parole e i loro gesti. 

- Non ho bisogno di te, Pëtr. 

- E come si fa per il biglietto? 

- Be', come vuoi, per me è lo stesso - disse lei con stizza. 

Pëtr saltò a cassetta e, messosi le mani sui fianchi, ordinò di andare alla stazione. 

\capitolo{XXX}\label{xxx-5} 

``Ecco, di nuovo! Di nuovo capisco tutto'' si disse Anna, non appena il carrozzino si mosse e sobbalzando rintronò sul lastrico, e di nuovo, una dopo l'altra, cominciarono a succedersi le impressioni. 

``Sì, qual'è l'ultima cosa a cui pensavo così chiaramente? - cercava di ricordare. - Tjut'kin coiffeur? No, non è quello. Sì, quel che dice Jašvin: la lotta per l'esistenza e l'odio sono le uniche cose che leghino gli uomini. No, andate inutilmente - disse rivolta col pensiero a una compagnia, in un calessino dal tiro a quattro, che, evidentemente, andava a divertirsi fuori città. - E il cane che portate con voi non vi aiuterà. Non sfuggirete a voi stessi''. Gettato uno sguardo dalla parte verso la quale Pëtr si voltava, vide un operaio ubriaco fradicio, con la testa ciondoloni, che una guardia portava chi sa dove. ``Ecco, questo piuttosto - ella pensò. - Io e il conte Vronskij però non l'abbiamo provato questo piacere, sebbene ci aspettassimo molto da esso''. E per la prima volta rivolse quella luce chiara, in cui vedeva tutto, verso i propri rapporti con lui, ai quali prima aveva evitato di pensare. ``Che cercava egli in me? Non tanto l'amore quanto la soddisfazione della vanità''. Ricordò le parole di lui, l'espressione del viso, che somigliava a un docile cane da caccia, nei primi tempi del loro legame. E tutto adesso lo confermava. ``Sì, in lui c'era il trionfo del successo, della vanità. S'intende, c'era anche l'amore, ma la parte maggiore era l'orgoglio del successo. Egli si vantava di me. Adesso è passato. Non c'è di che essere orgoglioso, ma di che vergognarsi. M'ha preso tutto quello che poteva, e adesso non gli sono più necessaria. Sente il peso di me e cerca di non essere disonesto nei miei riguardi. Ieri se l'è lasciato sfuggire: vuole il divorzio e il matrimonio per bruciare le sue navi. Mi ama, ma come? The zest is gone. Questo qui vuole far colpo su tutti ed è molto soddisfatto di sé - pensava, guardando un commesso rosso in faccia che andava su di un cavallo da corsa. - Sì, quel gusto in me non lo trova più. Se andrò via da lui, in fondo all'anima sarà contento''. 

Non era una sua supposizione; ella vedeva ciò chiaramente in quella luce penetrante che le scopriva adesso il senso della vita e dei rapporti umani. 

``Il mio amore si fa sempre più appassionato ed egoistico, e il suo non fa che spegnersi, ecco perché ci dividiamo - ella seguitò a pensare. - E non vi si può rimediare. Io ho tutto in lui solo, e pretendo che egli mi si dia sempre di più. E lui sempre di più vuole allontanarsi da me. Noi, prima di giungere al nostro legame, ci siamo proprio andati incontro, così ora ci dividiamo andando irresistibilmente verso parti opposte. E cambiare questo non si può. Lui mi ha detto che sono insensatamente gelosa e io stessa mi sono detta che sono insensatamente gelosa; ma non è vero. Non sono gelosa, sono scontenta, invece. Ma\ldots{} - aprì la bocca e cambiò posto nel carrozzino per l'agitazione suscitata in lei dal pensiero che le era venuto a un tratto. - S'io potessi essere qualcos'altro, invece dell'amante che ama appassionatamente le sole sue carezze; ma io non posso e non voglio essere null'altro. E con questo desiderio io suscito in lui la repulsione, e lui in me il rancore, e non può essere altrimenti. Non so io, forse, che egli non si metterebbe a ingannarmi, che non ha intenzioni circa la Sorokina, che non è innamorato di Kitty, che non mi tradirà? Tutto questo lo so, ma per questo non sto meglio. Se lui, senza amarmi, sarà buono, tenero con me per dovere, e non ci sarà quello che io voglio, questo è mille volte peggiore anche dell'odio! Questo è l'inferno! Ed è proprio così. Lui non mi ama già più da tempo. E dove finisce l'amore, comincia l'odio\ldots{} Queste strade non le conosco per nulla. Vi sono delle montagnole, e poi sempre case, case\ldots{} E in queste case sempre uomini, uomini\ldots{} Quanti ce ne sono, e sono senza fine e tutti si odiano a vicenda. Ebbene, ammettiamo che io trovi quello che voglio per essere felice. Ecco. Ottengo il divorzio, Aleksej Aleksandrovic mi dà Serëza, e io sposo Vronskij''. Ricordatasi di Aleksej Aleksandrovic, immediatamente, con una straordinaria chiarezza, se lo raffigurò davanti a sé come vivo, con i suoi occhi mansueti, senza vita, spenti, le vene turchine sulle mani bianche, le intonazioni di voce e lo scricchiolio delle dita e, ricordatasi di quel sentimento che c'era stato fra di loro e che pure s'era chiamato amore, rabbrividì di repulsione. ``Allora dunque, otterrò il divorzio e sarò moglie di Vronskij. Ebbene, Kitty smetterà di guardarmi come mi guardava oggi? No. E Serëza smetterà di chiedere e di pensare ai miei due mariti? E fra me e Vronskij che sentimento nuovo inventerò mai? È possibile, non pure una qualche felicità, ma la fine del tormento? No e no! - ella si rispose adesso, senza la più piccola esitazione. - È impossibile! Noi siamo separati dalla vita, e io faccio la sua infelicità, lui la mia, e non si può rifare né lui, né me. Tutti i tentativi sono stati fatti, la vite s'è spanata\ldots{} Già, una mendicante con un bambino. Pensa che si provi pena di lei. Non siamo forse tutti gettati nel mondo per odiarci a vicenda, e poi tormentare noi stessi e gli altri? Passano degli studenti di ginnasio, ridono. Serëza? - si ricordò. - Anch'io pensavo di volergli bene, e mi commovevo dinanzi alla mia tenerezza. E ho vissuto senza di lui, e l'ho scambiato con un altro amore, e non mi sono lamentata di questo baratto finché mi sono contentata di quest'altro amore''. E ricordò con repulsione quello che chiamava ``quest'amore''. E la lucidità con cui ora vedeva la propria vita e quella di tutte le persone la rallegrava. ``Così siamo e io, e Pëtr, e il cocchiere Fëdor, e quel mercante, e tutte quelle persone che vivono là lungo la Volga, dove quegli avvisi invitano ad andare, e dappertutto e sempre'' ella pensava, mentre si era avvicinata alla costruzione bassa della ferrovia di Niznij-Novgorod e le erano corsi incontro i facchini. 

- Comandate il biglietto fino a Obiralovka? - disse Pëtr. 

Lei aveva completamente dimenticato dove e perché andava, e soltanto con un grande sforzo poté capire la domanda. 

- Sì - disse, tendendo il portamonete col denaro e, infilato al braccio il sacchetto rosso, uscì dal carrozzino. 

Dirigendosi fra la folla verso la sala d'aspetto di prima classe, ella riandava con la mente a tutti i particolari della sua situazione, a tutte le decisioni fra cui ondeggiava. E di nuovo ora la speranza, ora la disperazione cominciarono, nei soliti punti dolenti, ad avvelenare le ferite del suo cuore tormentato, che batteva paurosamente. Seduta su di un divano a forma di stella, in attesa del treno, guardando con ripugnanza quelli che entravano e uscivano (tutti erano disgustosi per lei), pensava ora come sarebbe arrivata alla stazione, o come gli avrebbe scritto un biglietto e cosa gli avrebbe scritto, ora come adesso egli si lamentasse con la madre della propria situazione (senza rendersi conto delle sofferenze di lei) e come lei sarebbe entrata nella stanza e cosa gli avrebbe detto. Ora pensava come avrebbe potuto essere ancora felice la vita e come lo amasse e lo odiasse tormentosamente, e come paurosamente le battesse il cuore. 

\capitolo{XXXI}\label{xxxi-5} 

Squillò un campanello, e passarono alcuni giovani, orribili, insolenti e frettolosi, nello stesso tempo intenti a cogliere l'impressione che producevano. Passò anche Pëtr attraverso la sala, con le ghette e la livrea, il viso ottuso e animalesco, e si avvicinò a lei per accompagnarla al treno. Gli uomini rumorosi fecero silenzio, mentre ella passava accanto a loro sulla banchina, e uno di loro mormorò qualcosa dietro di lei a un altro, qualcosa, si intende, di volgare. Ella salì sull'alto predellino e sedette sola in uno scompartimento su di un sudicio divano a molle che una volta era stato bianco. Il sacchetto rimbalzò sulle molle, e poi si fermò. Pëtr, in segno di addio, si tolse, presso il finestrino e con un sorriso ebete, il suo cappello gallonato; un capotreno insolente sbatté la porta e abbassò la maniglia. Una signora sgraziata, con un vestito ridicolmente ampio dietro (Anna col pensiero denudò quella donna e inorridì della sua deformità), con una bambina che rideva forzatamente, passarono di corsa lì sotto. 

- Da Katerina Andreevna, sempre da lei, ma tante! - gridò la bambina. 

``La bambina, anche quella è sfigurata e smorfiosa'' pensò Anna. Per non vedere nessuno si alzò svelta e sedette accanto al finestrino opposto nello scompartimento vuoto. Un informe contadino sudicio, con un berretto di sotto al quale spuntavano dei capelli arruffati, passò vicino a quel finestrino, chino verso le ruote della vettura. ``C'è qualcosa di noto in questo informe contadino'' pensò Anna. E, ricordatasi del sogno, si allontanò, tremando di paura, verso la parte opposta. Un capotreno apriva la porta, per fare entrare una coppia. 

- Desiderate uscire? 

Anna non rispose. Il capotreno e quelli che erano entrati non notarono, sotto il velo, il terrore sul viso di lei. Ella tornò nel suo angolo e sedette. La coppia sedette dalla parte opposta, esaminando con attenzione, sott'occhio, il vestito di lei. Sia il marito che la moglie sembravano ripugnanti ad Anna. Il marito domandò il permesso di fumare, evidentemente, non per fumare, ma per intavolare discorso con lei. Ottenutone il consenso, si mise a dire in francese alla moglie che ancor più che di fumare, aveva bisogno di parlare. Parlavano, fingendo, di sciocchezze, sol perché lei sentisse. Anna vedeva chiaramente che erano annoiati l'uno dell'altra e che si odiavano a vicenda. E non potevano non odiarsi simili pietosi esseri deformi. 

Si sentì un secondo campanello seguìto da un trasportar di bagagli, da grida e da risate. Per Anna era così chiaro che nessuno aveva di che rallegrarsi, che quelle risate la irritarono fino a farla soffrire e le venne la voglia di tapparsi le orecchie per non sentirle. Finalmente squillò un terzo campanello, echeggiò un fischio, si sentì uno stridio di catene, una forte scossa e il marito si fece il segno della croce. ``Sarebbe interessante chiedergli cosa intende con questo'' pensò Anna con cattiveria. Guardando di fianco alla moglie, ella osservava attraverso il finestrino le persone che avevano accompagnato i congiunti al treno e che stavano in piedi sulla banchina, e pareva proprio che andassero all'indietro. Scotendosi regolarmente sui binari, la vettura in cui era seduta Anna scivolò lungo la banchina, accanto a un muro di pietra, a un disco e ad altre vetture; con un suono sottile le ruote risonarono più scorrevoli e più oleate sulle rotaie; il finestrino s'illuminò del sole vivido della sera e un vento leggero si mise a giocare con la tendina. Anna dimenticò i suoi vicini di vagone e, al leggero dondolio della corsa, aspirando l'aria fresca, si mise di nuovo a pensare. 

``Sì, a che punto mi son fermata? Al fatto che non riesco a inventare una situazione in cui la vita non sia un tormento, che noi tutti siamo creati per tormentarci, e che noi tutti lo sappiamo e tutti inventiamo dei mezzi per ingannarci. E quando si vede la verità, che mai si può fare?''. 

- La ragione è data all'uomo per liberarsi di quello che lo inquieta - disse in francese la signora, evidentemente soddisfatta della propria frase e facendo smorfie con la lingua. 

Queste parole parvero rispondere al pensiero di Anna. 

``Liberarsi di quello che lo inquieta'' ripeté Anna. E, guardando il marito dalle guance rosse e la moglie magra, ella capì che la moglie malaticcia si considerava una donna incompresa e che il marito la ingannava, mantenendo in lei questa opinione su se stessa. Ad Anna pareva di vedere la loro storia e tutti gli angoli remoti dell'anima loro, mentre spostava su di essi la sua luce. Ma lì non c'era nulla di interessante, e continuò il suo pensiero. 

``Sì, mi agita molto, e la ragione è data per liberarsene; perciò bisogna liberarsene. E perché non spegnere la candela, quando non c'è più nulla da guardare, quando fa ribrezzo guardare tutto? Ma come? Perché questo capotreno è passato di corsa sulla traversa? perché gridano quei giovani, in quello scompartimento? Perché parlano, perché ridono? Tutto è menzogna, tutto inganno, tutto malvagità\ldots{}''. 

Quando il treno entrò in stazione, Anna uscì tra la folla degli altri passeggeri e, allontanandosi da loro come da lebbrosi, si fermò sulla banchina, cercando di ricordare perché era arrivata là e cosa avesse intenzione di fare. Tutto quello che prima le sembrava possibile, adesso era così difficile a considerarsi, specialmente tra la folla rumoreggiante di tutte quelle persone deformi, che non la lasciavano in pace. Ora i facchini accorrevano da lei, offrendole i loro servigi, ora dei giovani, battendo coi tacchi le assi della banchina e discorrendo forte, la esaminavano, ora quelli che venivano incontro si facevano di lato non dalla parte giusta. Ricordatasi che voleva proseguire, se non ci fosse stata risposta, fermò un facchino e domandò se era venuto un cocchiere con un biglietto per il conte Vronskij. 

- Il conte Vronskij? Per incarico suo sono stati qui proprio ora. Venivano incontro alla principessa Sorokina con la figlia. E il cocchiere com'è? 

Mentre ella parlava col facchino, Michajla, rosso e allegro, con un elegante pastrano turchino e la catena, evidentemente orgoglioso d'avere eseguito così bene la commissione, le si avvicinò e le porse un biglietto. Ella aprì e il cuore le si strinse ancor prima di leggere. 

``Mi dispiace molto che il biglietto non m'abbia trovato. Verrò alle dieci'' scriveva Vronskij con una scrittura trascurata. 

``Ecco! Me l'aspettavo!'' si disse con un sorriso cattivo. 

- Va bene, allora va' a casa - disse piano, rivolta a Michajla. Ella parlava piano perché la rapidità dei battiti del cuore le impediva di respirare. ``No, non ti permetterò di tormentarmi'' ella pensò, rivolta con minaccia, non a lui, né a se stessa, ma a chi le imponeva di tormentarsi, e si incamminò per la banchina lungo la stazione. 

Due cameriere che camminavano sulla banchina si voltarono a guardarla, facendo ad alta voce qualche apprezzamento sul suo vestito: ``sono veri'' dissero dei pezzi ch'ella aveva addosso. I giovani non la lasciavano in pace. Di nuovo le passarono accanto, guardandola in viso e gridando fra le risa qualcosa con voce contraffatta. Il capostazione, passando, le domandò se partiva. Un ragazzo, venditore di kvas, non le toglieva gli occhi di dosso. ``Dio mio, dove andare?'' ella pensava, allontanandosi sempre più sulla banchina. Alla fine si fermò. Le signore e i bambini, che erano venuti a incontrare un signore con gli occhiali e che ridevano e parlavano forte, tacquero, esaminandola, quand'ella giunse alla loro altezza. Ella affrettò il passo e si allontanò da loro verso l'orlo della banchina. Si avvicinava un treno merci. La banchina si mise a tremare e a lei parve d'essere di nuovo in viaggio. 

E a un tratto si ricordò dell'uomo schiacciato al suo primo incontro con Vronskij e capì quello che doveva fare. Dopo essere scesa con passo veloce, leggero, per i gradini che andavano verso le rotaie, si fermò accanto al treno che le passava vicinissimo. Guardava la parte sottostante dei carri, le viti e le catene e le ruote alte di ghisa del primo carro che scivolava lento, e cercava di stabilire con l'occhio il punto mediano fra le ruote anteriori e le posteriori e il momento in cui questo punto mediano sarebbe stato di fronte a lei. 

``Là - si diceva, guardando nell'ombra del carro la sabbia mista a carbone di cui erano sparse le traverse - là, proprio nel mezzo, e lo punirò, e mi libererò da tutti e da me stessa''. 

Voleva cadere sotto il primo vagone che giungesse alla sua altezza nel punto mediano; ma il sacchetto rosso che aveva preso a togliere dal braccio, la trattenne, ed era già tardi; il punto mediano le era passato accanto. Bisognava aspettare il vagone seguente. Un sentimento simile a quello che provava quando, facendo il bagno, si preparava a entrar nell'acqua, la prese, ed ella si fece il segno della croce. Il gesto abituale della croce suscitò nell'anima sua tutta una serie di ricordi verginali e infantili, e a un tratto l'oscurità che per lei copriva tutto si lacerò, e la vita le apparve per un attimo con tutte le sue luminose gioie passate. Ma ella non staccava gli occhi dalle ruote del secondo vagone che si avvicinava. E proprio nel momento in cui il punto mediano fra le ruote giunse alla sua altezza, ella gettò indietro il sacchetto rosso, ritirò la testa fra le spalle, cadde sulle mani sotto il vagone e con movimento leggero, quasi preparandosi a rialzarsi subito, si lasciò andare in ginocchio. E in quell'attimo stesso inorridì di quello che faceva. ``Dove sono? che faccio? perché?''. Voleva sollevarsi, ripiegarsi all'indietro, ma qualcosa di enorme, di inesorabile le dette un urto nel capo e la trascinò per la schiena. ``Signore, perdonami tutto!'' ella disse, sentendo l'impossibilità della lotta. Un contadino, dicendo qualcosa, lavorava su del ferro. E la candela, alla cui luce aveva letto il libro pieno di ansie e di inganni, di dolore e di male, avvampò di una luce più viva che mai, le schiarì tutto quello che prima era nelle tenebre, crepitò, prese ad oscurarsi e si spense per sempre. 

\parte{PARTE OTTAVA}\label{parte-ottava} 

\capitolo{I}\label{i-7} 

Erano passati quasi due mesi. Si era già alla metà di un'estate calda, e Sergej Ivanovic, soltanto adesso, era pronto a lasciare Mosca. 

La vita di Sergej Ivanovic aveva avuto, nel frattempo, i suoi avvenimenti. Già da un anno circa egli aveva finito il suo libro, frutto di un lavoro di sei anni, intitolato Saggio di una rassegna delle basi e delle forme di stato in Europa e in Russia. Alcune parti di questo libro e l'introduzione erano state pubblicate in periodici, e altre erano state lette da Sergej Ivanovic a persone del suo ambiente, così che le idee di questo lavoro non potevano mai essere una novità assoluta per il pubblico; tuttavia Sergej Ivanovic si aspettava che il suo libro, uscendo, dovesse produrre una seria impressione sulla società e, se non proprio una rivoluzione nella scienza, in ogni caso un grande fermento nel mondo scientifico. 

Questo libro, dopo un accurato lavoro di lima, era stato pubblicato l'anno prima e spedito ai librai. 

Senza domandarne a nessuno, rispondendo svogliatamente e con finta indifferenza alle domande degli amici su come andava il libro, senza chiederne neppure ai librai se veniva comprato, Sergej Ivanovic aveva spiato con vigilanza, con ansia, la prima impressione che il suo libro produceva in società e fra i letterati. 

Ma passò una settimana, ne passarono due, tre e nella società non si notava alcuna impressione; gli amici specialisti e studiosi, a volte, evidentemente per cortesia, ne cominciavano a parlare. Ma gli altri suoi conoscenti, non interessati a un libro di contenuto scientifico, non ne parlavano affatto. E nella società, attirata in questo momento da altri interessi, vi era una completa indifferenza. Anche nelle critiche letterarie, per tutto un mese, non ci fu neppure una parola sul libro. 

Sergej Ivanovic calcolava fin nei particolari il tempo necessario per scrivere una recensione; ma passò un mese, ne passò un altro, sempre lo stesso silenzio. 

Soltanto nel ``Severnyj zuk'', in uno scherzoso articolo sul cantante Drabanti, che aveva perso la voce, erano dette, a questo proposito, alcune parole sprezzanti sul libro di Koznyšev, che mostravano che il libro, già da tempo, era condannato da tutti e abbandonato all'irrisione generale. 

Finalmente il terzo mese, in una rivista seria, apparve un articolo critico. Sergej Ivanovic conosceva l'autore dell'articolo. L'aveva incontrato una volta da Golubcov. 

L'autore dell'articolo era un giornalista molto giovane e malato, molto vivace come scrittore, ma straordinariamente incolto e timido nei rapporti personali. 

Malgrado il suo assoluto disprezzo per l'autore, Sergej Ivanovic si accinse alla lettura dell'articolo con piena considerazione. L'articolo era orribile. 

Evidentemente, l'articolista aveva inteso il libro in modo da renderne impossibile l'interpretazione. Ma aveva disposto così bene le citazioni che, per coloro i quali non avevano letto il libro (ed evidentemente quasi nessuno lo aveva letto), era del tutto chiaro che il libro non era altro che un cumulo di parole altisonanti, e per di più adoperate a sproposito (cosa che i punti interrogativi mettevano in rilievo), e che l'autore era una persona completamente ignorante. E tutto ciò era così spiritoso che neppure Sergej Ivanovic avrebbe respinto uno spirito simile; e appunto questo era orribile. 

Malgrado l'assoluta coscienziosità con cui Sergej Ivanovic controllava la giustezza degli argomenti del recensore, non si fermò neppure un attimo sui difetti e sugli errori che gli erano stati derisi, perché era troppo evidente che tutto questo era stato fatto con intenzione; tuttavia subito, involontariamente, riandò col pensiero, fin nei più piccoli particolari, al suo incontro e alla sua conversazione con l'autore dell'articolo. 

- Che l'abbia offeso in qualche modo? - si chiedeva Sergej Ivanovic. 

E ricordatosi come, nell'incontro, avesse corretto quel giovane in una parola che rivelava la sua ignoranza, Sergej Ivanovic trovò la spiegazione del senso dell'articolo. 

Dopo questo articolo seguì un silenzio di morte sul libro, sia da parte della stampa che della pubblica opinione, e Sergej Ivanovic vedeva che la sua opera, frutto di sei anni di lavoro, elaborata con tanto amore e tanta fatica, era passata senza lasciar traccia. 

La situazione di Sergej Ivanovic era ancor più penosa per il fatto che finito il libro, egli non aveva più l'occupazione dello scrivere, che prima prendeva tanta parte del suo tempo. 

Sergej Ivanovic era un uomo intelligente, colto, sano, attivo e non sapeva come adoperare la propria attività. I discorsi nei salotti, nei congressi, nelle riunioni, nei comitati, dovunque si parlasse, occupavano una parte del suo tempo; ma, vecchio abitante di città, non si concedeva di perdersi tutto in discorsi, come il suo inesperto fratello quand'era a Mosca; gli rimanevano così ancora molto tempo libero e molte energie intellettuali. 

Per sua fortuna, in quel periodo per lui penoso a causa dell'insuccesso del libro, in luogo delle questioni dei credenti di altre fedi, degli amici americani, della carestia di Samara, dell'esposizione, dello spiritismo, era sorta la questione slava, che fino allora languiva in seno alla società, e Sergej Ivanovic, che anche prima ne era stato uno dei promotori, vi si dedicò completamente. 

Nella cerchia delle persone a cui apparteneva Sergej Ivanovic, in quel momento non si scriveva altro che della guerra serba. Tutto quello che fa di solito una folla oziosa, per ammazzare il tempo, adesso si faceva a beneficio degli slavi. I balli, i concerti, i pranzi, i discorsi, le acconciature femminili, la birra, le trattorie, tutto testimoniava la simpatia per gli slavi. 

Con gran parte di quello che si diceva e si scriveva in quell'occasione, Sergej Ivanovic non era d'accordo nei particolari. Egli vedeva che la questione slava era diventata una di quelle questioni di moda che sempre, sostituendosi le une alle altre, servono alla società come materia d'interesse; vedeva che c'erano molte persone, che avevano scopi interessati, ambiziosi, che si occupavano di quella impresa. Riconosceva che i giornali stampavano molte cose inutili ed esagerate col solo scopo di richiamare l'attenzione e di gridare più degli altri. Vedeva che, in quella generale infatuazione della società, erano usciti fuori e gridavano più forte degli altri tutti i falliti e gli offesi; comandanti in capo senza eserciti, ministri senza ministero, giornalisti senza giornali, capipartito senza partito. Vedeva che in questo c'era molto di vacuo e ridicolo; ma vedeva e riconosceva un indubitabile e sempre crescente entusiasmo che aveva riunito in un tutto unico le classi della società, e per il quale non si poteva non aver simpatia. Il massacro dei correligionari e dei fratelli slavi aveva suscitato la simpatia verso coloro che soffrivano e l'indignazione contro gli oppressori. E l'eroismo dei serbi e dei montenegrini, che lottavano per una grande causa, aveva generato in tutto il popolo il desiderio di aiutare i fratelli non più con la parola, ma con l'azione. 

Inoltre c'era un altro fenomeno, soddisfacente per Sergej Ivanovic: era questo il manifestarsi di una opinione pubblica. La società aveva espresso in modo preciso il proprio desiderio. L'anima popolare aveva ricevuto un'espressione, come diceva Sergej Ivanovic. E quanto più egli si occupava di tale impresa, tanto più evidente gli appariva come questa impresa dovesse assumere proporzioni enormi, dovesse, cioè, fare epoca. 

Egli consacrò tutto se stesso al servizio di questa grande impresa, e dimenticò di pensare al suo libro. 

Adesso il suo tempo era occupato, così che non riusciva a rispondere a tutte le lettere e alle richieste che gli venivano rivolte. 

Dopo aver lavorato tutta la primavera e parte dell'estate, soltanto nel mese di luglio era pronto per andare in campagna dal fratello. 

Andava a riposarsi per un paio di settimane, e proprio nel sacrario del popolo, nella solitudine della campagna, andava a godere la visione di quel risveglio dello spirito nazionale, del quale lui e tutti gli abitanti di città erano pienamente convinti. Katavasov, che da lungo tempo voleva mantenere la promessa fatta a Levin di essere per un po' suo ospite, era partito insieme con lui. 

\capitolo{II}\label{ii-7} 

Sergej Ivanovic e Katavasov avevano appena fatto in tempo ad avvicinarsi alla stazione, quel giorno particolarmente animata di gente, della ferrovia di Kursk, scendere dalla carrozza e guardare il cameriere che li seguiva con la roba, che sopraggiunsero anche dei volontari su quattro vetture da nolo. Alcune signore con dei fasci di fiori andarono loro incontro, e i volontari, accompagnati dalla folla che s'era precipitata dietro di loro, entrarono nella stazione. 

Una delle signore che erano andate incontro ai volontari, uscendo dalla sala, si rivolse a Sergej Ivanovic. 

- Voi pure siete venuto ad accompagnarli? - domandò in francese. 

- No, parto, principessa. Vado a riposarmi da mio fratello. E voi accompagnate sempre? - disse Sergej Ivanovic con un sorriso appena percettibile. 

- Sì, non si può far diversamente! - rispose la principessa. - È vero che da noi ne sono partiti già ottocento? Malvinskij non mi credeva. 

- Più di ottocento. Se si contano quelli che sono stati inviati non direttamente da Mosca, già più di mille - disse Sergej Ivanovic. 

- Ecco. Lo dicevo, appunto! - soggiunse gioiosa la signora. - Ed è vero che adesso è stato offerto quasi un milione? 

- Di più, principessa. 

- E qual'è il comunicato di oggi? Hanno battuto di nuovo i turchi. 

- Sì, ho letto - rispose Sergej Ivanovic. Parlavano dell'ultimo bollettino, il quale confermava che per tre giorni di seguito i turchi erano stati battuti su tutti i punti e fuggivano, e che per l'indomani si prevedeva un combattimento decisivo. 

- Ah, sì, sapete, un ottimo giovane ha chiesto di andare. Non so perché abbiano fatto delle difficoltà. Vi volevo pregare, io lo conosco, scrivete un biglietto, per favore. È mandato dalla contessa Lidija Ivanovna. 

Dopo aver domandato i particolari che la principessa conosceva sul giovane che chiedeva di partire, Sergej Ivanovic, passato in prima classe, scrisse un biglietto a colui dal quale dipendeva la cosa e lo consegnò alla principessa. 

- Sapete, il conte Vronskij, il famoso\ldots{} parte con questo treno - disse la principessa con un sorriso trionfante e significativo, mentre egli, ritrovatala, le consegnava il biglietto. 

- Ho sentito che partiva, ma non sapevo quando. Con questo treno? 

- L'ho visto. È qui: la madre sola lo accompagna. Tuttavia è questa la cosa migliore che potesse fare. 

- Oh sì, s'intende. 

Mentre parlavano, la folla passò con furia accanto a loro verso la tavola da pranzo. Anche loro si avvicinarono e sentirono la voce forte di un signore che, con una coppa in mano, faceva un discorso ai volontari. ``Servire per la fede, per l'umanità, per i nostri fratelli - diceva il signore, alzando sempre più la voce. - La madre Mosca vi benedice per la grande impresa. zivio!'' - egli concluse forte e con le lacrime agli occhi. 

Tutti gridarono ``zivio'' e ancora una nuova folla irruppe nella sala e fece quasi cadere la principessa. 

- Eh, principessa, che discorso! - disse, esplodendo di un sorriso gioioso, Stepan Arkad'ic, che era comparso a un tratto in mezzo alla folla. - Non è vero che ha parlato bene? con calore? Bravo! Anche Sergej Ivanic! Ecco, sarebbe bene che anche voi, da parte vostra, parlaste così! qualche parola, sapete, un incoraggiamento, voi lo fate così bene - soggiunse, con un sorriso affabile, rispettoso e prudente, spingendo leggermente per un braccio Sergej Ivanovic. 

- No, parto subito. 

- Dove? 

- In campagna, da mio fratello - rispose Sergej Ivanovic. 

- Allora vedrete mia moglie. Le ho scritto, ma voi la vedrete prima; per favore, ditele che mi avete visto e che all right. Lei capirà. Ma del resto, ditele, siate buono, che sono stato nominato membro dell'agenzia\ldots{} Su, ma lei capirà! Sapete, les petites misères de la vie humaine - disse, rivolto alla principessa, come a scusarsi. - E la Mjagkaja però, non Liza ma Bibiche, manda mille fucili e dodici suore. Ve l'ho detto? 

- Sì, ho sentito - rispose di malavoglia Koznyšev. 

- Ma è un peccato che partiate - disse Stepan Arkad'ic. - Domani offriamo un pranzo a due parenti: Dimer-Bartnjanskij, quello di Pietroburgo, e il nostro Veselovskij, Griša. Vanno tutti e due. Veselovskij ha preso moglie da poco. Ecco un uomo coraggioso! Non è vero, principessa? - si rivolse alla signora. 

La principessa, senza rispondere, guardava Koznyšev. Ma il fatto che Sergej Ivanovic e la principessa desiderassero liberarsi di lui, non turbava per nulla Stepan Arkad'ic. Egli guardava sorridendo ora la piuma del cappello della principessa, ora di lato, come a ricordarsi di qualche cosa. Avendo scorto una signora che passava con una cassetta, la chiamò presso di sé e mise dentro un biglietto da cinque rubli. 

- Non posso veder passare tranquillamente queste cassette finché ho denaro in tasca - disse. - E com'è il bollettino di oggi? Bravi i montenegrini! 

- Cosa dite! - egli gridò quando la principessa gli disse che Vronskij partiva con quel treno. 

Per un attimo il viso di Stepan Arkad'ic espresse tristezza, ma dopo un momento, quando, molleggiando su ciascuna gamba e accomodandosi le fedine, entrò nella sala dove era Vronskij, Stepan Arkad'ic aveva già del tutto dimenticato quei propri disperati singhiozzi sul corpo della sorella, e vedeva in Vronskij solo l'eroe e il vecchio amico. 

- Con tutti i suoi difetti non gli si può non render giustizia - disse la principessa a Sergej Ivanovic, non appena Oblonskij si fu allontanato da loro. - Ecco proprio una natura veramente russa, slava! Temo soltanto che a Vronskij dispiacerà vederlo. Qualunque cosa diciate, commuove la sorte di quest'uomo. Parlate un po' con lui, in viaggio - aggiunse. 

- Sì, forse, se capiterà. 

- A me non è mai piaciuto. Ma questo riscatta molte cose. Non solo va lui stesso, ma conduce uno squadrone a proprie spese. 

- Sì, ho sentito. 

Si udì un campanello. Tutti si affollarono alla porta. 

- Eccolo! - esclamò la principessa, indicando Vronskij con un cappotto lungo e un cappello nero a larghe falde, che camminava al braccio della madre. Oblonskij camminava accanto a lui, dicendo animatamente qualcosa. 

Vronskij guardava accigliato davanti a sé, quasi senza ascoltare quello che diceva Stepan Arkad'ic. 

Probabilmente, per indicazione di Oblonskij, egli si voltò a guardare dalla parte dove stavano la principessa e Sergej Ivanovic, e sollevò il cappello in silenzio. Il suo viso invecchiato, che esprimeva la sofferenza, pareva impietrito. 

Uscito dalla banchina, Vronskij, lasciata la madre, scomparve in silenzio nello scompartimento di una vettura. 

Sulla banchina echeggiava ``Dio proteggi lo zar''; poi si udirono grida di ``urrà'' e ``zivio!''. Uno dei volontari, un uomo alto, molto giovane, dal petto incavato, salutava in modo molto clamoroso, agitando sopra il capo un cappello di feltro e un fascio di fiori. Dietro di lui, sporgevano la testa, pure salutando, due ufficiali e un uomo anziano, dalla gran barba, con un berretto sporco di grasso. 

\capitolo{III}\label{iii-7} 

Salutata la principessa, Sergej Ivanovic, insieme con Katavasov che s'era avvicinato, entrò in una vettura piena zeppa, e il treno si mosse. 

Alla stazione di Carycin il treno fu accolto da un coro di giovani, armonioso, che cantava ``Gloria a te!''. Di nuovo i volontari salutarono e sporsero le teste, ma Sergej Ivanovic non prestò loro attenzione: aveva avuto già tanto a che fare con i volontari che ne conosceva ormai le caratteristiche, e tutto questo non lo interessava più. Katavasov invece, che fra le sue occupazioni scientifiche non aveva avuto occasione d'osservare i volontari, se ne interessava molto e interrogava Sergej Ivanovic. 

Sergej Ivanovic gli consigliò di passare in seconda classe per parlare lui stesso con loro. Alla stazione seguente Katavasov seguì questo consiglio. Passò in seconda e fece conoscenza con i volontari. Stavano seduti in un angolo dello scompartimento, discorrendo forte e sapendo, evidentemente, che l'attenzione dei passeggeri e di Katavasov, che era entrato, era rivolta verso di loro. Più forte di tutti parlava l'adolescente alto dal petto incavato. Era ubriaco, si vedeva, e raccontava d'una certa storia capitata nel loro istituto. Di fronte a lui era seduto un ufficiale non più giovane con una maglia militare austriaca della divisa della Guardia. Egli ascoltava, sorridendo, il parlatore e lo interrompeva. Un terzo, in divisa d'artigliere, sedeva su di una valigia accanto a loro. Un quarto dormiva. 

Entrato in discorso con l'adolescente, Katavasov venne a sapere che era un ricco mercante moscovita, che aveva scialacquato un gran patrimonio prima di aver compiuto ventidue anni. Non piacque a Katavasov perché era effeminato, viziato e debole di salute; si vedeva che era sicuro, specialmente adesso dopo aver bevuto, di compiere un atto eroico, e si vantava nella maniera più antipatica. 

L'altro, l'ufficiale a riposo, produsse pure un'impressione sgradita su Katavasov. Era, si vedeva, un uomo che aveva provato tutto. Era stato nelle ferrovie, intendente, e lui stesso aveva fondato delle fabbriche e parlava di tutto, adoperando senza nessuna necessità, a sproposito, parole scientifiche. 

Il terzo, l'artigliere, al contrario, piacque molto a Katavasov. Era un uomo modesto, silenzioso che, evidentemente, si inchinava dinanzi alla posizione dell'ufficiale della Guardia e dinanzi all'eroica abnegazione del mercante e, per conto suo, nulla diceva di sé. Quando Katavasov gli domandò cosa lo avesse indotto ad andare in Serbia, rispose modestamente: 

- Ma cosa vuoi mai, vanno tutti, bisogna pure aiutare i serbi. Fanno pena. 

- Eh, sì, specialmente di artiglieri come voi, là ce n'è pochi - disse Katavasov. 

- Io non ho mica servito molto nell'artiglieria; può darsi anche che mi mettano in fanteria o in cavalleria. 

- Ma come in fanteria, se hanno più bisogno di tutto di artiglieri? - disse Katavasov, deducendo, dall'età dell'artigliere, ch'egli dovesse avere già un grado importante. 

- Non ho servito molto in artiglieria, sono a riposo come junker - disse e cominciò a spiegare perché non aveva superato l'esame. 

Tutto questo insieme produsse un'impressione spiacevole su Katavasov, e quando i volontari uscirono nella stazione a bere, Katavasov voleva confessare a qualcuno la propria impressione sfavorevole. Un vecchietto di passaggio, in cappotto militare, aveva prestato ascolto tutto il tempo alla conversazione di Katavasov coi volontari. Rimasto da solo con lui, Katavasov gli rivolse la parola. 

- Ma che diversità di condizione fra tutte queste persone che vanno là! - disse vagamente Katavasov, desiderando di esprimere la propria opinione e nello stesso tempo di sapere l'opinione del vecchietto. 

Il vecchietto era un militare che aveva fatto due campagne. Sapeva cos'era un militare, e dall'aspetto e dal parlare di quei signori, dal piglio con cui, in viaggio, si attaccavano alla borraccia, li giudicava cattivi militari. Inoltre, abitava in un capoluogo di distretto e aveva voglia di raccontare come, dalla sua cittadina, fosse andato un soldato in congedo illimitato, ubriacone e ladro, che più nessuno assumeva come lavoratore. Ma, sapendo per esperienza che, con l'odierno stato d'animo della società, era pericoloso esprimere un'opinione contraria a quella generale, e in particolare biasimare i volontari, anche lui osservava Katavasov. 

- Eh, là hanno bisogno di gente - egli disse, ridendo con gli occhi. E si misero a parlare delle ultime notizie militari, e tutti e due nascosero, l'uno all'altro, la propria perplessità sul fatto che s'aspettasse per l'indomani un combattimento quando i turchi, secondo l'ultima informazione, erano stati battuti su tutta la linea. E così, senza aver detto nessuno dei due la propria opinione, si separarono. 

Katavasov, entrato nel suo scompartimento, andando involontariamente contro coscienza, raccontò a Sergej Ivanovic le osservazioni sui volontari, dalle quali risultava che erano ottimi figlioli. 

A una grande stazione di una città, di nuovo canti e grida accolsero i volontari, apparvero di nuovo raccoglitrici e raccoglitori di offerte con le cassette, e le signore del capoluogo del governatorato offrirono fasci di fiori ai volontari e li seguirono al ristorante; ma tutto questo era in tono molto più debole e in proporzioni minori che non a Mosca. 

\capitolo{IV}\label{iv-7} 

Durante il tempo della fermata nella stazione del capoluogo del governatorato, Sergej Ivanovic non andò al ristorante, ma si mise a camminare avanti e indietro sulla banchina. 

Passando per la prima volta accanto allo scompartimento di Vronskij, notò che il finestrino era chiuso. Ma, passando una seconda volta, vide al finestrino la vecchia contessa. Ella chiamò a sé Koznyšev. 

- Ecco, vado, lo accompagno fino a Kursk - ella disse. 

- Sì, ho sentito - disse Sergej Ivanovic, fermandosi vicino al finestrino e dandovi un'occhiata dentro - che bel gesto da parte sua! - soggiunse dopo aver notato che Vronskij non era nello scompartimento. 

- Sì, dopo la sventura, che cosa mai doveva fare? 

- Che cosa terribile! - disse Sergej Ivanovic. 

- Ah, cosa ho passato! Ma entrate\ldots{} Ah, cosa ho passato! - ella ripeté, quando Sergej Ivanovic entrò e sedette accanto a lei sul divano. - Non si può immaginare! Per sei settimane, non ha parlato con nessuno e ha mangiato solo quando lo supplicavo io. E neppure per un momento lo si poteva lasciare solo. Avevamo portato via tutto quello con cui poteva uccidersi; stavamo a pianterreno, ma non si poteva prevedere nulla. Perché lo sapete, s'era già sparato una volta, pure per lei - disse, e le sopracciglia della vecchietta si contrassero a questo ricordo. - Sì, è finita proprio come doveva finire una donna simile. Perfino la morte ha scelto vile, bassa. 

- Non sta a noi giudicare, contessa - disse Sergej Ivanovic con un sospiro - ma capisco come questo sia stato penoso per voi. 

- Ah, non lo dite! Io stavo nella mia villa, e lui era da me. Portano un biglietto. Lui scrive la risposta e la manda via. Noi non sapevamo nulla, che lei fosse proprio lì, alla stazione. La sera me n'ero appena andata in camera mia, quando la mia Mary mi dice che alla stazione una signora s'è gettata sotto il treno. Fu come se qualcosa m'avesse colpito! Capii che era lei. La prima cosa che dissi fu: ``non lo dite a lui''. Ma gliel'avevano già detto. Il cocchiere s'era trovato là e aveva visto tutto. Quando io accorsi in camera sua, egli era già fuori di sé; era tremendo a guardarlo. Non disse nemmeno una parola e corse là. Non so più cosa ci fu, ma lo portarono come un morto. Io non l'avrei riconosciuto. Prostration complète, diceva il dottore. Poi cominciò come una frenesia. Ah, che dire! - disse la contessa, facendo un gesto sconsolato con la mano. - Un momento tremendo. No, qualunque cosa diciate, è stata una donna perversa. Ma che passioni disperate sono queste! È sempre per mostrar qualcosa di particolare. Ecco che lei proprio l'ha dimostrato. Ha rovinato se stessa e due ottime persone: suo marito e il mio povero figliolo. 

- E suo marito che fa? - chiese Sergej Ivanovic. 

- Ha preso la figlia di lei. Alësa nei primi tempi consentiva a tutto. Ma adesso lo tormenta orribilmente il fatto d'aver dato la propria figlia a una persona estranea, ma non vuol rimangiarsi la parola. Karenin è venuto al funerale. Ma noi abbiamo cercato di non farlo incontrare con Alësa. Per lui, per il marito tuttavia, la vita è più facile. Lei l'ha liberato. Ma il mio povero figliolo s'era dato tutto a lei. Aveva abbandonato tutto: la carriera, me, e lei non solo non ha avuto pietà, ma l'ha proprio distrutto, deliberatamente. No, qualunque cosa diciate, la stessa sua morte è la morte di una donna bassa, senza religione. Iddio mi perdoni, ma non posso non odiare la sua memoria, vedendo la rovina di mio figlio. 

- Ma adesso come sta? 

- È Dio che ci ha aiutato, con questa guerra serba. Io sono vecchia, non ci capisco nulla, ma questa gliel'ha mandata Iddio. S'intende che io, come madre, provo spavento; e soprattutto, dicono, ce n'est pas très bien vu à Petersbourg. Ma che fare? Questo solo poteva sollevarlo. Jašvin, un suo amico, ha perduto tutto al giuoco e s'è preparato ad andare in Serbia. È passato da lui e l'ha convinto. Adesso questo lo occupa. Voi, per favore, parlate con lui, desidero farlo distrarre. È così triste! E per disgrazia gli è sopraggiunto un gran mal di denti. Ma sarà molto contento di vedervi. Per favore, parlate un po' con lui: sta camminando da questa parte. 

Sergej Ivanovic disse ch'era molto contento, e passò dall'altra parte del treno. 

\capitolo{V}\label{v-7} 

Nell'ombra serale, obliqua, dei sacchi ammassati sulla banchina, Vronskij, nel suo cappotto lungo, col cappello abbassato e le mani in tasca, camminava, voltandosi rapidamente a ogni venti passi, come una belva in gabbia. A Sergej Ivanovic, mentre si avvicinava, parve che Vronskij lo avesse scorto, ma che fingesse di non vederlo. Per Sergej Ivanovic era indifferente. Egli era al di sopra di ogni considerazione personale nei riguardi di Vronskij. 

In quel momento Vronskij, agli occhi di Sergej Ivanovic, era un collaboratore importante di una grande impresa, e Koznyšev stimava suo dovere incoraggiarlo e approvarlo. Gli si avvicinò. 

Vronskij si fermò, lo guardò fisso, lo riconobbe e, fatti alcuni passi incontro a Sergej Ivanovic, gli strinse forte la mano. 

- Forse voi non desiderate neppure di vedermi - disse Sergej Ivanovic; - ma non posso esservi utile? 

- Non c'è nessuno che io possa vedere meno spiacevolmente di voi - disse Vronskij. - Perdonatemi. Cose piacevoli per me nella vita non esistono. 

- Capisco, volevo offrirvi i miei servigi - disse Sergej Ivanovic, esaminando il viso, evidentemente sofferente, di Vronskij. - Non avete bisogno d'una lettera per Ristic, per Milan? 

- Oh, no! - disse Vronskij, quasi stentando a capire. - Se per voi è indifferente, camminiamo. Nelle vetture c'è una afa tale. Una lettera? No, vi ringrazio; per morire non c'è bisogno di raccomandazioni. Se non ai turchi\ldots{} - diss'egli, sorridendo solo con la bocca. Gli occhi continuavano ad avere una espressione di sofferenza esasperata. 

- Sì, ma forse vi sarebbe più facile entrare in relazioni, che tuttavia sono indispensabili, con una persona preparata. Ma, come volete! Sono stato molto contento sentendo della vostra decisione. Anche così, ci sono già tanti attacchi contro i volontari, che un uomo come voi li solleva nell'opinione pubblica. 

- Io, come uomo - disse Vronskij - sono buono perché la vita per me non vale nulla. E che in me ci sia abbastanza forza fisica per sfondare un quadrato e romperlo o rimanerci, questo lo so. Sono contento che ci sia qualcosa per cui dare la mia vita, la quale non è che non mi sia necessaria, ma m'è venuta in odio. A qualcuno servirà - ed egli fece un movimento impaziente con lo zigomo per un doloroso, incessante mal di denti, che gli impediva perfino di parlare con l'espressione che voleva. 

- Tornerete a nuova vita, ve lo predìco - disse Sergej Ivanovic, sentendosi commosso. - La liberazione dei propri fratelli dal giogo è un fine degno e della morte e della vita. Che Iddio vi conceda buon successo esterno, e la pace interiore - soggiunse e tese la mano. 

Vronskij strinse forte la mano tesa di Sergej Ivanovic. 

- Sì, come strumento posso servire a qualcosa. Ma come uomo, sono un rudere - egli disse dopo una pausa. 

L'attanagliante dolore al dente robusto, che gli riempiva di saliva la bocca, gli impediva di parlare. Tacque, esaminando le ruote di un tender che scivolava lento e scorrevole sulle rotaie. 

E a un tratto non un dolore, ma un disagio interiore, tormentoso e complesso, che pure non era dolore, lo obbligò a dimenticare per un momento il mal di denti. Guardando il tender e le rotaie, sotto l'influsso della conversazione con un amico che non aveva rivisto dopo la propria sventura, gli tornò a un tratto in mente lei, cioè quello che rimaneva ancora di lei, quand'egli era entrato, correndo come un pazzo, nella caserma della stazione ferroviaria: sul tavolo della caserma il corpo insanguinato, disteso senza ritegno in mezzo agli estranei, ancora pieno di vita recente; la testa intatta reclinata indietro con le trecce pesanti e i capelli inanellati sulle tempie, e sul viso delizioso, dalla bocca rossa socchiusa, una strana espressione di pena rappresa sulle labbra e spaventosa negli occhi non chiusi e fissi, quasi stesse per pronunciare quella frase terribile che gli aveva detto durante il litigio, ch'egli se ne sarebbe pentito. 

Ed egli cercava di ricordarla come era quando l'aveva incontrata la prima volta, pure alla stazione, misteriosa e incantevole, piena d'amore, che cercava e dava la felicità, e non così crudelmente vendicativa come gli tornava alla memoria nell'ultimo istante. Egli cercava di ricordare i momenti migliori passati con lei; ma questi momenti erano avvelenati per sempre. Egli ricordava di lei solo quella minaccia trionfante, che aveva compiuto per ottenere un rimorso non necessario a nessuno, ma indistruttibile. Cessò di sentire il dolore al dente, e i singhiozzi gli contrassero il viso. 

Passando due volte davanti ai sacchi, in silenzio, e tornato padrone di sé, si voltò con calma verso Sergej Ivanovic: 

- Non si è avuto un bollettino dopo quello di ieri? Sì, li hanno battuti per la terza volta, ma per domani intanto si aspetta un combattimento decisivo. 

E, dopo aver parlato ancora della proclamazione di Milan a sovrano e delle conseguenze enormi che questo poteva avere, si separarono andando nelle rispettive vetture, al secondo squillo di campanello. 

\capitolo{VI}\label{vi-7} 

Non sapendo quando sarebbe potuto partire da Mosca, Sergej Ivanovic non aveva telegrafato al fratello di mandarlo a rilevare. Levin non era in casa quando Katavasov e Sergej Ivanovic, su di un piccolo tarantas noleggiato alla stazione, neri di polvere, alle undici passate, si avvicinarono alla scalinata della casa di Pokrovskoe. Kitty, che era seduta al balcone col padre e la sorella, riconobbe il cognato e corse giù ad accoglierlo. 

- Come, non vi vergognate di non farcelo sapere? - disse, tendendo la mano a Sergej Ivanovic e porgendogli la fronte. 

- Siamo arrivati benissimo e non v'abbiamo incomodato - rispose Sergej Ivanovic. - Son così impolverato che temo di toccarvi. Ero tanto occupato che non sapevo quando sarei fuggito. E voi come prima - diss'egli, sorridendo - vi godete una tranquilla felicità al di fuori delle correnti, nel vostro tranquillo porto. Ecco che anche il nostro Fëdor Vasil'ic finalmente si è deciso. 

- Non sono un negro, mi laverò, avrò un aspetto umano - disse Katavasov col suo abituale brio, dando la mano e sorridendo, in modo particolare, con i denti che spiccavano nel viso annerito. 

- Kostja sarà molto contento. È andato alla fattoria. Sarebbe tempo che venisse. 

- S'occupa sempre dell'azienda. Ecco, proprio come in un porto - disse Katavasov. - E noi in città, eccettuata la guerra serba, non vediamo nulla. Be', come vede la cosa il mio amico? Forse un po' diversamente dagli altri. 

- Ma lui la vede così, come tutti - rispose Kitty un po' confusa, voltandosi a guardare Sergej Ivanovic. - Allora manderò a chiamarlo. E da noi c'è ospite papà. È arrivato dall'estero da poco. 

E, dato l'ordine di mandare a chiamare Levin e di condurre gli ospiti impolverati a lavarsi, l'uno nello studio, l'altro nella camera che era stata di Dolly, e di preparare la colazione agli ospiti, compiacendosi della facoltà di muoversi agilmente di cui era stata privata durante la gravidanza, andò di corsa al balcone. 

- Ci sono Sergej Ivanovic e Katavasov, il professore - disse. 

- Oh, con questo caldo, la cosa è pesante! - disse il principe. 

- No, papà, è molto simpatico, e Kostja gli vuole molto bene - disse sorridendo Kitty, come a supplicarlo di qualcosa, dopo aver notato l'espressione di canzonatura sul viso del padre. 

- Ma io non dico nulla. 

- Va' tu, Dolly, tesoro mio - disse alla sorella - e intrattienili. Hanno visto Stiva alla stazione, sta bene. E io corro da Mitja. Neanche a farlo apposta, non gli ho dato il latte dall'ora del tè. Adesso s'è svegliato e probabilmente grida. - E, sentendo un afflusso di latte, andò nella camera del bambino a passo svelto. 

Realmente, non era che avesse indovinato (il suo legame col bambino non era ancora spezzato), ma aveva capito con certezza, all'afflusso di latte che sentiva in sé, ch'egli mancava di nutrimento. 

Sapeva che gridava, ancor prima di avvicinarsi alla camera del bambino. E realmente egli gridava. Ne sentì la voce e accelerò il passo. Ma quanto più presto ella camminava, tanto più forte egli gridava. La sua voce era buona, sana, ma affamata e impaziente. 

- È un pezzo, njanja, è un pezzo? - diceva in fretta Kitty, sedendosi su di una seggiola e preparandosi a dare il latte. - Ma datemelo presto, dunque! Ah, njanja, come siete noiosa, via, la cuffietta gliela legherete dopo! 

Il bambino si strozzava a furia di gridare per la fame. 

- Ma non si può mica, matuška - disse Agaf'ja Michajlovna, che era quasi sempre presente nella camera del bambino. - Bisogna metterlo in ordine. Ahu, ahu! - ella cantava sopra di lui, senza prestare attenzione alla madre. 

La njanja portò il bambino alla madre. Agaf'ja Michajlovna gli camminava dietro col viso rasserenato dalla tenerezza. 

- Conosce, conosce. Come è vero Iddio, matuška, Katerina Aleksandrovna, m'ha conosciuta! - gridava Agaf'ja Michajlovna più forte del bambino. 

Ma Kitty non ascoltava le parole di lei. La sua impazienza cresceva come l'impazienza del bambino. 

Per l'impazienza la faccenda, per un pezzo, non si avviò. Il bambino non afferrava quello che doveva e si arrabbiava. 

Finalmente, dopo un disperato grido soffocato, dopo un inghiottire a vuoto, la cosa si sistemò, e la madre e il bambino si sentirono contemporaneamente calmi e placati. 

- Però anche lui, poverino, è tutto sudato - disse sottovoce Kitty, palpando il bambino. - Perché pensate che conosca? - aggiunse, osservando di traverso gli occhi del bambino che le sembrava guardassero furbi di sotto alla cuffietta abbassatasi, le piccole guance che riprendevano fiato uniformemente, e la manina dalla palma rossa, con la quale faceva dei movimenti circolari. 

- Non può essere! Se conoscesse, allora conoscerebbe me - disse Kitty, all'assicurazione di Agaf'ja Michajlovna, e sorrise. 

Sorrise perché, quantunque dicesse ch'egli non poteva riconoscere, col cuore sapeva che non solo riconosceva Agaf'ja Michajlovna, ma sapeva e capiva tutto; e sapeva e capiva ancora molte cose che nessuno sapeva e che lei, madre, aveva imparato a conoscere e aveva cominciato a capire grazie a lui. Per Agaf'ja Michajlovna, per la njanja, per il nonno, perfino per il padre, Mitja era un essere vivo che esigeva per sé soltanto cure materiali; ma per la madre egli era da tempo un essere morale, con cui c'era già tutta una storia di rapporti spirituali. 

- Ma ecco che si sveglierà, se Dio vuole, lo vedrete da voi. Appena faccio così, lui si rischiara, golubcik. Si fa subito raggiante come una giornata serena - diceva Agaf'ja Michajlovna. 

- E va bene, va bene, lo vedremo, allora - mormorò Kitty. - Adesso andate, si addormenta 

\capitolo{VII}\label{vii-7} 

Agaf'ja Michajlovna uscì in punta di piedi; la njanja abbassò la tendina, scacciò le mosche di sotto alla cortina di mussola del lettuccio e un calabrone che batteva contro i vetri della finestra, e si sedette, agitando un ramo di betulla, quasi vizzo, sopra la madre e il bambino. 

- Che caldo, che caldo! Se almeno Iddio ci mandasse una pioggerella - disse. 

- Sì, sì, sst\ldots{} - rispose soltanto Kitty, dondolandosi lievemente e premendo con tenerezza il braccio paffuto, come stretto al polso da un filo, che Mitja agitava sempre più debolmente, ora chiudendo, ora aprendo gli occhietti. Questo braccino confondeva Kitty; aveva voglia di baciare quel braccino, ma aveva paura di farlo, per non svegliare il bimbo. Il braccino finalmente cessò di muoversi, e gli occhi si chiusero. Solo di tanto in tanto, il bambino sollevava le lunghe ciglia ricurve, fissava la madre con gli occhi umidi, che nella penombra sembravano neri. La njanja cessò di agitare il ramo e si assopì. Di sopra si udì uno scoppio di voce del vecchio principe e uno scroscio di risa di Katavasov. 

``Di sicuro, si son messi a parlare senza di me - pensava Kitty - tuttavia mi spiace che Kostja non ci sia. È andato di sicuro nell'arniaio. Per quanto sia triste, il fatto che vada spesso là, mi fa piacere. Ciò lo distrae. Adesso è diventato più allegro, più buono che non in primavera. S'era fatto così cupo e tormentato che cominciavo ad avere paura di lui. E come è buffo!'' sussurrò, sorridendo. 

Ella sapeva quello che tormentava suo marito. Era la propria mancanza di fede. Se avessero domandato a lei se riteneva che, nella vita futura, non credendo, egli si sarebbe perduto, avrebbe dovuto convenire che si sarebbe perduto; eppure la mancanza di fede in lui, non la tormentava; ed ella, che doveva riconoscere non esserci salvezza per un miscredente, pur amando più di tutto al mondo l'anima di suo marito, pensava con un sorriso alla miscredenza di lui e fra di sé diceva ch'egli era buffo. 

``Perché tutto l'anno non fa che leggere certe filosofie? - pensava. - Se tutto questo è scritto in quei libri, lui li può capire. Ma se c'è falsità, perché leggerli allora? Lui stesso dice che vorrebbe credere. Allora perché non crede? Probabilmente perché pensa molto. E pensa molto per la solitudine. È sempre solo, solo. Con noi non può dire tutto. Penso che questi ospiti gli facciano piacere, particolarmente Katavasov. Gli piace ragionare con lui'' ella pensò, ma subito cominciò a considerare dove sarebbe stato più comodo mettere a dormire Katavasov, separatamente o insieme a Sergej Ivanyc. E qui le venne a un tratto un pensiero che la fece trasalire di agitazione e riuscì perfino a inquietare Mitja che, per questo, la guardò severo. ``La lavandaia, mi pare, non ha ancora portato la biancheria, e per gli ospiti la biancheria da letto è tutta fuori. Se non si dànno ordini, Agaf'ja Michajlovna darà a Sergej Ivanyc della biancheria usata'' e, a questo solo pensiero, il sangue affluì al viso di Kitty. 

``Sì, darò ordini'' ella stabilì e, ritornando alle idee di prima, ricordò che qualcosa d'importante riguardo all'anima non era stato ancora finito di pensare, e cominciò a ricordare cosa. ``Sì, Kostja è miscredente'' ricordò di nuovo con un sorriso. 

``Via, miscredente! Meglio che sia sempre così, invece d'essere come la signora Stahl o come volevo essere io allora, all'estero. No, lui, poi, non si metterà a fingere''. 

E un tratto recente della sua bontà le sorse dinanzi con chiarezza. Due settimane prima era arrivata una lettera contrita di Stepan Arkad'ic per Dolly. Egli la supplicava di salvare il suo onore, di vendere i propri possessi per pagare i debiti. Dolly s'era disperata, aveva odiato il marito, l'aveva disprezzato, compianto, si era decisa a divorziare, a rifiutare, ma aveva finito con l'acconsentire a vendere una parte dei propri possessi. Dopo questo, Kitty ricordò, con un involontario sorriso di commozione, il turbamento di suo marito, il suo ripetuto imbarazzo nell'avvicinarsi alla questione che lo interessava e come finalmente, escogitato l'unico mezzo per aiutare Dolly senza offenderla, avesse proposto a Kitty di cederle la sua parte dei possedimenti, cosa che a lei non era venuta in mente prima. 

``E che miscredente? Col suo cuore, con quel terrore di addolorare chiunque, perfino un bambino! Tutto per gli altri, nulla per sé. Sergej Ivanovic pensa proprio che sia un dovere di Kostja essere il suo amministratore. Così anche la sorella. Adesso Dolly con i suoi bambini è sotto la sua tutela. E tutti questi contadini che vengono ogni giorno da lui, come se egli fosse obbligato a servirli''. 

``Sì, sii soltanto come tuo padre, soltanto così'' ella diceva, dando Mitja alla njanja e toccandone col labbro la guancia. 

\capitolo{VIII}\label{viii-7} 

Dal momento in cui, nel vedere morire il fratello amato, Levin aveva considerato per la prima volta la questione della vita e della morte attraverso le nuove convinzioni, come egli le chiamava, che insensibilmente avevano sostituito le credenze infantili e giovanili, nel periodo che per lui era andato dai venti ai trentaquattro anni, aveva provato orrore non tanto della morte, quanto di una vita senza la minima conoscenza di ciò che essa è, donde viene, a che scopo e perché. L'organismo, la sua distruzione, l'indistruttibilità della materia, la legge di conservazione della forza, l'evoluzione, erano tutte parole che, in lui, avevano preso il posto della fede d'un tempo. Queste parole, e le concezioni ad esse legate, andavano molto bene per gli scopi intellettuali; ma per la vita non davano nulla, e Levin si sentì, a un tratto, nella situazione d'un uomo che abbia scambiato una pelliccia calda per un vestito di mussola e che, per la prima volta, al gelo, si persuada in modo indubitabile, non con ragionamenti ma con tutto il suo essere, che per lui è come se fosse nudo e che deve inevitabilmente perire in modo tormentoso. 

Da quel momento, pur senza rendersene conto e continuando a vivere come prima, Levin non aveva cessato di provare il terrore della propria ignoranza. 

Inoltre, sentiva confusamente che ciò che chiamava le sue convinzioni, era non solo ignoranza, ma un modo di pensare col quale era impossibile raggiungere la conoscenza di quello che gli occorreva. 

Nel primo tempo del matrimonio, le nuove gioie, i nuovi doveri da lui conosciuti avevano completamente soffocato questi pensieri: ma negli ultimi tempi, dopo il parto della moglie, quando aveva vissuto a Mosca inoperoso, a Levin si era presentato sempre più frequente e insistente un problema che chiedeva soluzione. 

Il problema per lui consisteva in questo: ``Se io non riconosco quelle risposte che dà il cristianesimo alle domande della vita, allora quali risposte riconosco?''. E non riusciva in nessun modo a trovare in tutto l'arsenale delle proprie convinzioni non solo una qualche risposta, ma nulla che fosse simile a una risposta. 

Era nella situazione di un uomo che cerca il cibo in una bottega di giocattoli o di armi. Involontariamente, senza averne egli stesso coscienza, adesso, in ogni libro, in ogni conversazione, in ogni persona cercava i rapporti con tali questioni e la loro soluzione. 

Più di tutto in quel tempo lo stupiva e lo sconvolgeva il fatto che la maggioranza delle persone del suo ambiente e della sua età, avendo scambiato, come lui, le credenze di prima con le stesse convinzioni nuove che aveva lui, non vedevano in questo alcun danno ed erano del tutto contente e tranquille. Così che, oltre la questione principale, ancora altre questioni tormentavano Levin: erano sincere quelle persone? non fingevano? Oppure, non capivano in un qualche modo diverso, più chiaramente di lui, le risposte che dava la scienza alle questioni che lo interessavano? Ed egli studiava accuratamente e le opinioni di quelle persone e i libri che enunciavano quelle risposte. 

Una cosa sola aveva trovato, dal momento in cui tali questioni avevano cominciato a interessarlo, ed era che egli sbagliava nel supporre, dai ricordi del suo ambiente giovanile universitario, che la religione avesse già fatto il suo tempo e che non esistesse più. Tutte le persone buone, a lui vicine per rapporti di vita, credevano. E il vecchio principe, e L'vov, che gli piaceva tanto, Sergej Ivanyc, tutte le donne credevano, e sua moglie credeva, così come egli aveva creduto nella prima infanzia, e così credeva il novantanove per cento del popolo russo, tutto quel popolo la cui vita gli ispirava il più grande rispetto. 

Un'altra cosa era che, avendo letto molti libri, s'era convinto che le persone le quali condividevano le sue opinioni non intendevano null'altro e, senza spiegar nulla, negavano non soltanto le questioni, senza la soluzione delle quali egli sentiva di non poter vivere, ma cercavano di risolvere questioni del tutto diverse, che non potevano interessarlo, come, per esempio, quella dell'evoluzione degli organismi, quella della spiegazione meccanica dell'anima e simili. 

Inoltre, durante il parto della moglie, gli era accaduto un avvenimento straordinario. Lui, che non credeva, s'era messo a pregare, e, nel momento in cui aveva pregato, aveva creduto. Ma quel momento era passato, e a quello stato d'animo d'allora egli non poteva dare alcun posto nella propria vita. 

Non poteva ammettere che in quel momento aveva conosciuto la verità e che ora si sbagliava; perché, non appena cominciava a pensare con calma, tutto si frantumava in mille pezzi; non poteva riconoscere nemmeno che in quel momento si era sbagliato perché gli era caro lo stato d'animo di allora e, riconoscendolo come un risultato della propria debolezza, avrebbe contaminato quegli attimi. Era in una penosa disarmonia con se stesso e tendeva tutte le forze dell'animo per uscirne. 

\capitolo{IX}\label{ix-7} 

Questi pensieri lo facevano soffrire e lo tormentavano ora in maniera più debole, ora più forte, ma non lo abbandonavano mai. Leggeva e pensava, e quanto più leggeva e pensava, tanto più lontano si sentiva dallo scopo che perseguiva. 

Negli ultimi tempi a Mosca e in campagna, convintosi che nei materialisti non avrebbe trovato una risposta, aveva letto e riletto Platone e Spinoza, Kant e Schelling, Hegel e Schopenhauer, filosofi questi che spiegavano la vita da un punto di vista non materialistico. 

I pensieri gli sembravano fecondi quando leggeva o quando si figurava le confutazioni di altre dottrine, in particolare materialistiche; ma non appena leggeva o immaginava da sé la soluzione delle questioni, allora si ripeteva sempre la stessa cosa. Seguendo una definizione già data di parole oscure, come spirito, volontà, libertà, sostanza, entrando apposta in quella rete di parole che gli ponevano i filosofi o che egli stesso si poneva, cominciava quasi a capire qualcosa. Ma bastava dimenticare l'artificioso corso del pensiero e tornare, mentre pensava secondo il filo dato, a quello che nella vita lo soddisfaceva, che improvvisamente tutta quell'artificiosa impalcatura crollava come un castello di carte, e appariva chiaro che l'edificio era fatto con quelle stesse parole trasposte, indipendentemente da qualcosa che nella vita era più importante della ragione. 

Un certo tempo, leggendo Schopenhauer, sostituì al posto della volontà l'amore; e questa nuova filosofia per un paio di giorni, finché vi rimase dentro, lo consolò; ma crollò proprio alla stessa maniera quando la osservò dalla vita, e si rivelò un vestito di mussola che non teneva caldo. 

Suo fratello Sergej Ivanovic gli consigliò di leggere le opere teologiche di Chomjakov. Levin lesse il secondo volume delle opere di Chomjakov e, malgrado il tono polemico, elegante e spiritoso che dapprima l'aveva allontanato, fu colpito dalla dottrina sulla Chiesa da esso esposta. Lo colpì il pensiero che la comprensione delle verità divine non era data all'uomo, ma era data all'insieme degli uomini uniti dall'amore, alla Chiesa. Lo rallegrò il pensiero di come fosse più facile credere alla Chiesa esistente, presentemente viva, che costituiva tutte le credenze degli uomini e che aveva a capo Iddio e perciò era santa e infallibile, e da essa poi accogliere le credenze in Dio, nella creazione, nel peccato, nella redenzione, anziché cominciare da Dio, da un Dio lontano, misterioso, dalla creazione e via di seguito. Ma avendo poi letto una storia della Chiesa di uno scrittore cattolico e una storia della Chiesa di uno scrittore ortodosso e visto che tutte e due le Chiese, infallibili per loro natura, si negavano reciprocamente, egli si disincantò anche della dottrina di Chomjakov, e quest'edificio si dissolse in polvere come le costruzioni filosofiche. 

Tutta quella primavera visse fuori di sé ed ebbe momenti paurosi. 

``Senza la conoscenza di quel che sono qui, non si può vivere. Ma sapere questo non posso, di conseguenza non si può vivere'' si diceva Levin. 

``Nel tempo infinito, nell'infinità della materia, nello spazio infinito nasce un piccolo organismo; questa bollicina si tiene un po' in alto e poi scoppia, e questa bollicina sono io''. 

Era una tormentosa menzogna, ma era l'unico, l'ultimo risultato del secolare lavoro del pensiero umano in quella direzione. 

Era quella l'ultima credenza nella quale si sistemavano tutte le ricerche del pensiero umano in quasi tutti i campi. Era la convinzione che dominava, e Levin, fra tutte le altre spiegazioni, assimilò proprio questa, come la più chiara, tuttavia senza sapere lui stesso quando e come. 

Ma questa non solo era una menzogna, ma era la crudele irrisione di una certa forza perversa, infame, contraria e tale che non ci si poteva sottomettere. 

Bisognava liberarsi da questa forza. E la liberazione era nelle mani di ognuno. Bisognava far cessare la dipendenza dal male. E non v'era che un mezzo: la morte. 

E Levin, padre di famiglia felice, uomo sano, fu varie volte così vicino al suicidio, che nascose una corda per non impiccarsi, ed ebbe paura di andar col fucile per non spararsi. 

Ma Levin non si sparò e non si impiccò e continuò a vivere. 

\capitolo{X}\label{x-7} 

Quando Levin pensava che cosa mai egli fosse e per quale mai cosa vivesse, non trovava una risposta e si dava alla disperazione; ma quando cessava di chiederselo, pareva sapere cos'era e per che cosa vivesse, perché agiva e viveva in modo fermo e deciso; anzi in quegli ultimi tempi viveva con molta più fermezza e decisione di prima. 

Tornato in campagna al principio di giugno, era tornato anche alle sue solite occupazioni. L'azienda rurale, i rapporti con i contadini e con i vicini, l'azienda domestica, gli affari della sorella e del fratello che gli erano sulle spalle, i rapporti con la moglie, i parenti, le preoccupazioni per il bambino, la nuova caccia alle api, alla quale si era appassionato fin dalla primavera, occupavano tutto il suo tempo. 

Questi affari lo occupavano non perché egli li giustificasse da un qualche punto di vista generale, come soleva far prima; al contrario, adesso, da una parte deluso dall'insuccesso delle precedenti opere per la comune utilità, dall'altra troppo preso dai suoi pensieri e dal gran numero di faccende che gli piombavano da ogni lato, aveva completamente abbandonato ogni considerazione sull'utilità generale, e questi affari lo occupavano soltanto perché gli pareva di dover fare quello che faceva, di non poter fare altrimenti. 

Un tempo (la cosa era cominciata quasi dall'infanzia e si era accentuata sempre più fino alla piena virilità), quando egli cercava di far qualcosa che procurava il bene di tutti, dell'umanità, della Russia, di tutto il villaggio, aveva notato che i pensieri al riguardo erano piacevoli, ma l'attività stessa era sempre slegata, non c'era la piena sicurezza che la cosa fosse assolutamente necessaria; e quella stessa attività, che in principio sembrava così vasta, si restringeva sempre più, si riduceva a nulla. Adesso invece, quando, dopo il matrimonio, egli aveva cominciato a restringere sempre più la vita per se stesso, pur non provando più nessuna gioia al pensiero della propria attività, aveva la certezza che la sua opera fosse necessaria e vedeva che essa riusciva molto meglio di prima e diveniva sempre più vasta. 

Adesso, quasi contro la propria volontà, egli si conficcava sempre più nella terra come un aratro, così che ormai non poteva più uscirne senza rivoltare il solco. 

Che la famiglia vivesse come erano abituati a vivere i padri e i nonni, cioè nelle stesse condizioni di cultura, e che nelle stesse condizioni venissero educati i figli, era indubbiamente necessario così come si pranza quando si ha voglia di mangiare; e così come preparare il pranzo, era altrettanto necessario condurre la macchina economica a Pokrovskoe in modo che rendesse. Così come indubbiamente bisognava pagare un debito, bisognava pure tenere la terra patrimoniale in una situazione tale che il figlio, ricevutala in eredità, fosse grato al padre, così come Levin lo era stato al nonno per tutto quello che aveva costruito e piantato. E per questo non bisognava dare in fitto la terra, ma coltivarla da sé, tenere il bestiame, concimare i campi, piantare i boschi. 

Non si potevano non curare gli affari di Sergej Ivanovic, della sorella, di tutti i contadini che venivano a chieder consiglio e che vi si erano abituati, così come non si può abbandonare un bambino che si tenga per mano. Bisognava prendersi cura delle comodità della cognata invitata coi figliuoli, e della moglie e del bambino e non si poteva non dedicare loro una sia pur piccola parte del giorno. 

E tutto questo insieme con la caccia alla selvaggina e la nuova caccia alle api, riempiva tutta quella vita di Levin che non aveva nessun senso per lui quando pensava. 

Ma, oltre al fatto che Levin sapeva bene cosa dovesse fare, così egli sapeva pure come dovesse far tutto questo e quale faccenda fosse più importante dell'altra. 

Sapeva che si dovevano assumere dei lavoratori al minor prezzo possibile; ma che assumerli come servi, dando il denaro in anticipo, a un prezzo minore di quanto costavano, non si doveva, anche se questo era molto vantaggioso. Vender la paglia ai contadini, durante la carestia, si poteva, anche se ne veniva compassione; ma la locanda e la taverna, anche se rendevano, bisognava distruggerle. Per il taglio dei boschi bisognava punire il più severamente possibile, ma per il bestiame fatto pascolare abusivamente non si potevano prendere multe; e benché questo addolorasse i guardiani e distruggesse il timore, non si poteva non lasciare pascolare il bestiame abusivamente. 

A Pëtr, che pagava il dieci per cento al mese a uno strozzino, bisognava fare un prestito per riscattarlo; ma non si poteva condonare né differire il tributo ai contadini insolventi. Non si poteva lasciar passare all'amministratore che un praticello non fosse stato falciato e l'erba si fosse perduta per niente; ma si potevano anche non falciare le ottanta desjatiny dove era stato piantato un bosco giovane. Non si poteva perdonare un lavoratore che al tempo del lavoro se ne andasse a casa perché gli era morto il padre e, per quanta pena suscitasse, bisognava pagarlo di meno per i mesi in cui era stato assente; ma non si poteva non dare la mensilità anche ai vecchi servi che non facevano nulla. 

Levin sapeva pure che, tornando a casa, bisognava prima di tutto andar dalla moglie che stava poco bene, mentre i contadini, che lo aspettavano già da tre ore, potevano aspettare ancora; ma sapeva che, malgrado tutto il piacere da lui provato a metter dentro uno sciame, bisognava privarsi di quel piacere e, lasciato il vecchio a metter dentro lo sciame senza di lui, bisognava andare a ragionare con i contadini che l'avevano trovato nell'arniaio. 

Se agiva bene o male non lo sapeva, e non soltanto non si sarebbe messo adesso a dimostrarlo, ma evitava discorsi e pensieri in proposito. 

I ragionamenti lo portavano al dubbio e gli impedivano di vedere quel che si doveva e quel che non si doveva fare. Quando invece non pensava, ma viveva, sentiva continuamente nell'animo suo la presenza di un giudice infallibile che decideva quale delle due azioni fosse la migliore e quale peggiore, e, non appena agiva in modo diverso da come si doveva, lo sentiva immediatamente. 

Così egli viveva, non sapendo e non vedendo la possibilità di sapere che cosa mai egli fosse e perché mai fosse al mondo, tormentandosi per questa sua ignoranza fino al punto da temere il suicidio, e nello stesso tempo aprendosi nella vita con fermezza la propria strada, ben tracciata e tutta sua. 

\capitolo{XI}\label{xi-7} 

Il giorno in cui Sergej Ivanovic giunse a Pokrovskoe, Levin era in una delle sue giornate più tormentose. 

Era il periodo più fervido di lavoro, quando in tutto il popolo si manifesta una così straordinaria tensione dello spirito di sacrificio nel lavoro, come non si manifesta mai in altre occasioni di vita, e che sarebbe altamente apprezzabile se le persone che dimostrano queste qualità le apprezzassero loro stesse, se questa tensione non si ripetesse ogni anno e se le conseguenze di essa non fossero così semplici. 

Falciare e mietere la segala e l'avena e trasportarle, finire di falciare i prati, dividere a mezzo il maggese, sgranare le sementi e seminare il grano autunnale, tutto questo sembra semplice e ordinario; ma per giungere a questo bisogna che, dal grande al piccolo, tutta la gente di campagna lavori incessantemente, in quelle tre o quattro settimane, tre volte più del normale, nutrendosi di kvas, di cipolla e di pane nero, battendo e portando di notte i covoni e dedicando al sonno non più di due o tre ore sulle ventiquattro. E questo si fa ogni anno in tutta la Russia. 

Avendo vissuto la maggior parte della sua vita in campagna e in rapporti intimi con la gente di campagna, Levin, nel periodo di lavoro, sentiva sempre che quella generale eccitazione del popolo si comunicava anche a lui. 

Fin dalla mattina era andato alla prima seminagione della segala, a vedere l'avena che portavano alle biche e, tornato a casa per l'ora in cui si alzavano la moglie e la cognata, aveva bevuto il caffè con loro e se n'era andato a piedi a una fattoria dove si doveva mettere in funzione una battitrice, impiantata di recente, per la preparazione delle sementi. 

Tutto quel giorno Levin, discorrendo con l'amministratore e i contadini, e a casa con la moglie, con Dolly e i bambini e con il suocero, aveva pensato sempre all'unica cosa che l'occupava in quel tempo, oltre le cure dell'azienda; e in tutto aveva cercato un riferimento alla propria domanda: ``Che cosa sono mai io? e dove sono? e perché sono qui?''. 

Stando in piedi, al fresco di un granaio ricoperto di recente da una grata di foglie odorose di nocciuolo non ancora fissata ai freschi travicelli scortecciati di tremula dal tetto di paglia, Levin guardava attraverso il portone aperto, ora la polvere secca e amara della battitura, che si ammucchiava e saltellava, ora l'erba dell'aia illuminata dal sole caldo e la paglia fresca allora allora portata fuori da una tettoia, ora le rondini dal capo variegato e il petto bianco, che con un sibilo entravano volando sotto il tetto, e battendo le ali si fermavano nel vano del portone, ora la gente che formicolava nel granaio scuro e polveroso. Pensieri strani gli venivano in mente. 

``Perché si fa tutto questo? - pensava. - Perché io sto qui, perché li costringo a lavorare? Come mai sono tutti affaccendati e cercano di mostrami il loro zelo? Perché mai questa vecchia Matrëna si affanna, amica mia? (L'ho curata quando nell'incendio le cadde una trave addosso) - pensava, guardando la vecchia allampanata che, movendo il grano col rastrello, camminava con sforzo per l'aia diseguale e scabra coi piedi nudi abbronzati. - Allora è guarita, ma oggi o domani, fra dieci anni la metteranno sotto terra e non rimarrà nulla né di lei, né di questa elegantona con la giacchetta rossa che toglie la spiga dalla pula con un movimento così agile, delicato. Anche lei sotterreranno, e questo cavallo pezzato molto presto - pensava guardando un cavallo che trascinava il ventre a stento e respirava frequentemente con le froge rigonfie, oltrepassando una ruota ricurva che gli si moveva sotto - lo sotterreranno, e anche Fëdor il porgitore, con quella barba ricciuta piena di pula e la camicia strappata sulla spalla bianca, sotterreranno. E lui rompe i covoni, e comanda qualcosa, e sgrida le donne, e con un movimento rapido accomoda la cinghia del volante. E soprattutto, non soltanto loro, ma me sotterreranno, e non ne rimarrà più nulla. A che scopo?''. 

Pensava, e nello stesso tempo guardava l'orologio per calcolare quanto si batteva in un'ora. Aveva bisogno di saperlo per assegnare il da fare per la giornata, calcolando da questo. 

``Presto sarà un'ora, e hanno cominciato soltanto il terzo mucchio'' pensò Levin, si avvicinò al porgitore e, sovrastando con la voce il rumore della macchina, gli disse di porgere meno fitto. 

- Ne dài troppo alla volta, Fëdor! Vedi, si ostruisce, perciò non rende. Fallo pari! 

Fëdor, annerito dalla polvere appiccicatasi al viso sudato, gridò qualcosa in risposta, ma continuava a non fare come voleva Levin. 

Levin, avvicinatosi al cassone, allontanò Fëdor e si mise lui stesso a porgere. 

Dopo aver lavorato fino al pranzo dei contadini, prima del quale non rimaneva molto tempo, uscì dal granaio insieme col porgitore e si mise a parlare con lui, fermandosi accanto a una bica gialla di segala mietuta e disposta con precisione sull'aia per la sementa. Il porgitore era d'un villaggio lontano, di quello stesso in cui Levin aveva assegnato la terra al capo dell'artel'. Adesso era stata data in affitto al portiere. 

Levin si mise a parlare di questa terra con Fëdor il porgitore e domandò se per l'anno prossimo non avrebbe preso la terra Platon, un ricco e buon contadino dello stesso villaggio. 

- Il prezzo è caro, Platon non può guadagnarci, Konstantin Dmitric - rispose il contadino, tirando fuori delle spighe dal petto sudato. 

- Ma come mai Kirillov guadagna? 

- Mitjucha - così il contadino chiamò con disprezzo il portiere - come non dovrebbe guadagnare, Konstantin Dmitric! Lui pigia e tira fuori il suo. Non ha compassione d'un cristiano. Ma zio Fokanyc - così egli chiamava il vecchio Platon - si metterà forse a strappare la pelle all'uomo? Dove darà a credito, dove anche calerà sul prezzo. E non arriverà a guadagnare. Lui agisce da uomo. 

- Ma perché mai calerà sul prezzo? 

- Ma così, perché le persone sono diverse; uno vive solo per il proprio bisogno, come, per esempio, Mitjucha, pensa solo a riempirsi la pancia; ma Fokanyc è un vecchio veritiero. Vive per l'anima. Si ricorda di Dio. 

- Come si ricorda di Dio? Come vive per l'anima? - gridò, quasi, Levin. 

- Ma si sa come: secondo la verità, secondo il volere di Dio. Perché le persone sono diverse. Ecco, prendiamo magari voi, anche voi non offendereste un uomo\ldots{} 

- Sì, sì, addio! - esclamò Levin, ansimando per l'agitazione e, voltatosi, prese il bastone e andò via rapidamente verso casa. Alle parole del contadino su Fokanyc che viveva per l'anima, secondo verità, secondo il volere di Dio, pensieri confusi, densi di significato, pareva avessero fatto irruzione in folla, come se venissero da chi sa quale luogo remoto e che, tendendo tutti a una sola mèta, si fossero messi a turbinare nel suo capo, accecandolo con la loro luce. 

\capitolo{XII}\label{xii-7} 

Levin camminava a grandi passi per la strada maestra, prestando attenzione non tanto ai propri pensieri (non riusciva ancora a distinguerli), quanto allo stato d'animo, non provato mai prima. 

Le parole dette dal contadino avevano prodotto nell'animo suo l'azione di una scintilla elettrica che avesse trasformato e unito in una cosa sola l'intero sciame di pensieri disordinati, impotenti, staccati, che non avevano mai cessato di accompagnarlo. Questi pensieri lo avevano occupato senza che egli stesso se ne accorgesse anche nel momento in cui aveva parlato dell'assegnazione della terra. 

Sentiva nell'animo suo qualcosa di nuovo e palpava con godimento questa cosa nuova, non sapendo cosa fosse. 

``Vivere non per i propri bisogni, ma per Dio. Per quale Dio? E cosa si può dire di più insensato di quello ch'egli ha detto? Ha detto che non bisogna vivere per i propri bisogni, cioè che non bisogna vivere per quello che comprendiamo, verso cui siamo attratti, di cui sentiamo desiderio, ma che bisogna vivere per qualcosa di incomprensibile, per un Dio che nessuno può capire, né definire. E allora? Non ho forse inteso queste parole insensate di Fëdor? E, dopo averle intese, ho forse dubitato della loro giustezza? le ho giudicate sciocche, poco chiare, inesatte? 

No, le ho intese e proprio così come intende lui, le ho intese pienamente e con maggiore chiarezza ch'io non intenda qualunque altra cosa nella vita; e mai nella mia vita ho dubitato né posso dubitare di questo. E non io solo, ma tutto, tutto il mondo intende pienamente questa sola cosa e di questa sola cosa non dubita e vi consente sempre. 

Fëdor dice che il portiere Kirillov vive per la pancia. È comprensibile e ragionevole. Noi tutti, esseri ragionevoli, non possiamo vivere altrimenti che per la pancia. E a un tratto lo stesso Fëdor dice che vivere per la pancia è male e che bisogna vivere per la verità, per Dio, e io lo intendo da un accenno! E io, e milioni di uomini che hanno vissuto secoli fa e che vivono adesso, i contadini, i poveri di spirito e i saggi, che hanno pensato e scritto su questo e dicono la stessa cosa con il loro linguaggio confuso, noi tutti siamo d'accordo su questa sola cosa: per che cosa si debba vivere e cosa sia il bene. Io con tutte le persone non ho che una sola conoscenza ferma, indubitabile e chiara; e questa conoscenza non può essere spiegata con la ragione; è al di fuori di essa e non ha nessuna causa e non può avere nessun effetto. 

Se il bene ha una causa, non è più bene; se ha un effetto, la ricompensa, pure non è bene. Perciò, il bene è al di fuori della catena delle cause e degli effetti. 

E questo appunto io lo so e tutti lo sappiamo. 

Ma io cercavo dei miracoli, mi rammarico di non aver visto un miracolo che mi avesse persuaso. Ed ecco il miracolo, l'unico possibile, continuamente attuale, che da ogni parte mi circonda, e io non me ne accorgevo! 

Quale miracolo può essere mai più grande di questo? 

``Possibile che io abbia trovato la soluzione di tutto, che le mie pene adesso siano finite?'' pensava Levin, camminando per la strada polverosa, senza sentire né caldo, né stanchezza e provando una sensazione di sollievo dopo una lunga sofferenza. Questa sensazione era così gioiosa che gli sembrava inverosimile. Ansava per l'agitazione e, non avendo la forza di andare avanti, scese dalla strada nel bosco e sedette all'ombra delle tremule sull'erba non falciata. Tolse il cappello dalla testa sudata e si coricò, appoggiandosi a un braccio, sull'erba del bosco, densa di umori e simile alla bardana. 

``Sì, bisogna riaversi e capire'' pensava, guardando l'erba non calpestata che era dinanzi a lui, e seguendo i movimenti di un piccolo scarabeo verde che saliva su per lo stelo d'una gramigna ed era trattenuto, nella sua ascesa, da un filo d'erba egizia. ``Che cosa ho scoperto? - si domandò, voltando dall'altra parte lo stelo perché non disturbasse lo scarabeo, e piegando un altro filo d'erba perché lo scarabeo passasse su di esso. - Che cosa mi rallegra? che cosa ho mai scoperto? 

Prima io dicevo che nel mio corpo, come in questo filo d'erba e in questo piccolo scarabeo (ecco che non ha voluto andare su questo filo d'erba, ha raddrizzato le ali, ha preso il volo), si compiva, secondo le leggi fisiche, chimiche, fisiologiche, uno scambio di materia. E in tutti noi, insieme con le tremule, con le nubi e le nebulose si compiva una evoluzione. Evoluzione da che cosa? in che cosa? Una sconfinata evoluzione e lotta\ldots{} Come se ci potesse essere una qualche direzione e una lotta nell'infinito! E mi stupivo che, malgrado la più grande tensione di pensiero su questa via, non mi si scoprisse, tuttavia, il senso della vita, il senso dei miei impulsi e delle mie aspirazioni. Ma il senso dei miei impulsi è così chiaro in me, che io costantemente vivo secondo questo e mi sono sorpreso e mi sono rallegrato quando il contadino me lo ha enunciato: vivere per Dio, per l'anima. 

Io non ho scoperto nulla. Ho soltanto imparato a conoscere quello che sapevo. Ho capito quella forza che, non solo nel passato, la vita m'ha data, ma che mi dà anche adesso. Mi sono liberato da un inganno, ho imparato a conoscere il padrone''. 

E in breve ripeté a se stesso tutto il corso dei propri pensieri in quegli ultimi due anni, sorti con la chiara, evidente idea della morte alla vista del fratello caro, malato senza speranza. 

Avendo allora, per la prima volta, capito chiaramente che per ogni uomo come lui non c'era niente, all'infuori della sofferenza, della morte, dell'oblio completo, aveva deciso che così non si poteva vivere, che bisognava o spiegare la propria vita in modo che non apparisse una malvagia irrisione d'un qualche demone, o spararsi. 

Non aveva fatto né l'una né l'altra cosa, e aveva continuato a vivere, invece, a pensare e a sentire; in quel tempo aveva perfino preso moglie e aveva provato molte gioie ed era stato felice, sempre che non avesse pensato al senso della propria vita. 

Che cosa mai significava questo? Significava ch'egli aveva vissuto bene, ma aveva pensato male. 

Aveva vissuto (senza averne coscienza) di quelle verità spirituali ch'egli aveva succhiato col latte, e aveva pensato non solo senza riconoscere queste verità, ma eludendole con cura. 

Adesso gli era chiaro che egli aveva potuto vivere soltanto grazie a quella fede in cui era stato educato. 

``Che cosa sarei e come avrei vissuto la mia vita, se non avessi avuto questa fede, se non avessi saputo che bisognava vivere per Dio e non per i propri bisogni? Avrei rubato, mentito, ucciso. Nulla di quello che costituisce la gioia principale della mia vita esisterebbe per me''. E, facendo i più grandi sforzi di immaginazione, non riusciva tuttavia a figurarsi l'essere feroce che lui stesso sarebbe stato se non avesse saputo perché viveva. 

``Io cercavo una risposta alla mia domanda. E la risposta alla mia domanda non poteva darmela il pensiero: esso è incommensurabile con la domanda. La risposta me l'ha data la stessa vita nella mia conoscenza di quello che è bene e di quello che è male. E questa conoscenza non l'ho acquistata con nulla, ma mi è stata data insieme agli altri, data perché non la potevo prendere da nessuna parte. 

Di dove ho preso ciò? Sono forse giunto con la ragione a convincermi che bisogna amare il prossimo e non soffocarlo? Me l'hanno detto nell'infanzia, e io ci ho creduto con gioia, perché mi dicevano quello che avevo nell'animo. E chi l'ha scoperto? Non la ragione. La ragione ha scoperto la lotta per l'esistenza e la legge che esige che siano soffocati tutti quelli che ostacolano il soddisfacimento dei miei desideri. Questa è la deduzione della ragione. E che si debba amare un altro non poteva scoprirlo la ragione, perché è una cosa irragionevole''. 

``Sì, orgoglio'' si disse, buttandosi sul ventre e cominciando a legare in nodo gli steli delle erbe, cercando di non spezzarli. 

``E non soltanto orgoglio dell'intelletto, ma insipienza dell'intelletto. E soprattutto un inganno, proprio un inganno dell'intelletto. Vera e propria frode dell'intelletto'' ripeté. 

\capitolo{XIII}\label{xiii-7} 

E a Levin venne in mente una scena svoltasi di recente fra Dolly e i bambini. I bambini, rimasti soli, avevano cominciato a bruciare i lamponi sulle candele e a bere il latte a garganella. La madre, coltili sul fatto, in presenza di Levin aveva cominciato a predicare loro quanta fatica costasse ai grandi quello che loro distruggevano, e che questa fatica si faceva per loro, che se avessero rotto le tazze, non avrebbero avuto dove bere il tè, e se avessero versato il latte, non avrebbero avuto nulla da mangiare e sarebbero morti di fame. 

E Levin era rimasto colpito dalla calma, sommessa incredulità con la quale i bambini avevano ascoltato le parole della mamma. Erano dispiaciuti solo perché era cessato per loro un giuoco attraente, ma non credevano neppure una parola di quel che diceva la madre. E non potevano credere perché non potevano immaginare, in tutto il suo complesso, quello di cui godevano, e non potevano quindi immaginare che quello che distruggevano fosse proprio quello di cui si nutrivano. 

``Tutto questo va da sé - pensavano - e d'interessante e d'importante in questo non c'è nulla, perché questo è sempre stato e così sarà. È sempre la stessa cosa. A questo non dobbiamo pensare, è già bell'e pronto; ma noi abbiamo voglia di inventare qualcosa di nostro e di nuovo. E noi abbiamo inventato che nella tazza ci mettiamo i lamponi e poi li arrostiamo su una candela e il latte ce lo versiamo direttamente in bocca l'un l'altro. Questo è allegro e nuovo e non è peggiore del sistema di bere nelle tazze''. 

``Non facciamo forse lo stesso noi, non lo facevo io, cercando con la ragione il senso delle forze della natura e il senso della vita dell'uomo?'' continuò a pensare Levin. 

``E non fanno forse la stessa cosa tutte le teorie filosofiche, per la via del pensiero, strana e impropria dell'uomo, conducendolo alla conoscenza di quello ch'egli da lungo tempo conosce e sa con tanta giustezza, che senza di quello non potrebbe neppur vivere? Non si vede forse chiaramente, nell'evoluzione della teoria di ogni filosofo, ch'egli sa in precedenza, in modo altrettanto indubitabile quanto il contadino Fëdor, ma per nulla affatto in modo più chiaro, il senso principale della vita e soltanto per la dubbia via dell'intelletto vuole tornare a quello che tutti sanno? 

Ma, lasciamo soli i bambini, che si procurino, che si facciano da loro stessi le stoviglie, mungano il latte, ecc. Si metterebbero forse a far birichinate? Morirebbero di fame. E allora, lasciateci andare con le nostre passioni, coi nostri pensieri, senza l'idea dell'unico Dio e Creatore! O almeno senza l'idea di quel che sia bene, senza la spiegazione del male morale. 

E allora, provate a costruire qualcosa senza queste idee! 

Noi distruggiamo solo perché siamo spiritualmente sazi. Proprio come i bambini! 

Donde ricevo la conoscenza gioiosa, comune al contadino, che solo mi dà la tranquillità dell'anima? Da dove ho preso ciò? 

Io, educato nel concetto di Dio, da cristiano, dopo aver riempito tutta la mia vita di quei beni spirituali che mi ha dato il cristianesimo, ricolmo di questi beni e di essi vivente, io, come un bambino, senza capirli, distruggo, cioè voglio distruggere quello di cui vivo. E non appena incombe un momento grave della vita, come i bambini quando hanno freddo e fame, vado verso di Lui, e ancora meno dei bambini, che la madre sgrida per le loro birichinate infantili, sento che i miei infantili tentativi di agitarmi per troppo benessere non mi sono valsi. 

Sì, quello che so, non lo so con la ragione, ma mi è dato, mi è rivelato e io lo intendo attraverso il cuore, credo nella cosa principale che professa la Chiesa''. 

``La Chiesa, la Chiesa!'' ripeté Levin, mettendosi a giacere dall'altro lato e, appoggiandosi su di un braccio, cominciò a guardare lontano un gregge che dall'altra parte scendeva verso il fiume. 

``Ma posso io credere a tutto quello che professa la Chiesa? - pensava, mettendosi alla prova e immaginando tutto quello che avrebbe potuto distruggere la sua calma di adesso. Cominciò a ricordare proprio quelle dottrine della Chiesa che più gli erano parse strane e che lo inducevano in tentazione. - La creazione? E con che cosa mai spiegavo l'esistenza? Con l'essere? Col nulla? Il demone e il peccato? E con che cosa spiegavo il male?\ldots{} Il Redentore?\ldots{} 

Ma io nulla, nulla so, né posso sapere se non quello che mi è stato detto insieme agli altri''. 

E adesso gli sembrava che non ci fosse neppure una delle credenze della Chiesa che infrangesse la più grande: la fede in Dio, nel bene, come unica missione dell'uomo. A ogni credenza della Chiesa poteva essere sostituita la credenza nel soddisfare la verità in luogo delle proprie esigenze. E ognuna non solo non distruggeva queste, ma era indispensabile, perché si compisse quel miracolo essenziale, che continuamente appariva sulla terra, e che consisteva nel fatto che fosse possibile a ognuno, d'accordo con milioni di persone fra le più svariate, sagge o folli, giovani o vecchie, d'accordo con tutti, col contadino, con L'vov, con Kitty, con i mendicanti e con i sovrani, intendere indubitatamente la stessa cosa e comporre quella vita dell'anima, per la quale soltanto vale la pena di vivere e che sola apprezziamo. 

Sdraiato sul dorso, egli ora guardava il cielo alto, senza nuvole. ``Non so io forse che quello è lo spazio infinito e non già una volta rotonda? Ma per quanto socchiuda gli occhi e sforzi la vista, non posso non vederlo rotondo e limitato, e, malgrado la mia conoscenza dello spazio infinito, ho senza dubbio alcuno ragione quando vedo una solida volta azzurra, e ho più ragione che non quando mi sforzo di vedere al di là di essa''. 

Levin cessò di pensare e pareva soltanto prestare ascolto alle voci misteriose che, con gioia e con affanno, parlavano fra di loro di una certa cosa. 

``È forse questa la fede? - pensò, temendo di credere alla propria felicità. - Dio mio, Ti ringrazio!'' egli pronunciò, soffocando i singhiozzi che salivano e asciugando con tutte e due le mani le lacrime di cui gli s'erano riempiti gli occhi. 

\capitolo{XIV}\label{xiv-7} 

Levin guardava dinanzi a sé e vedeva il gregge, poi il suo calesse al quale era attaccato Voronoj, e il cocchiere che, avvicinatosi al gregge, diceva qualcosa al pastore; sentì poi vicino a sé il suono delle ruote e lo sbuffare del cavallo ben pasciuto; ma era così assorto nei suoi pensieri che non suppose neppure perché il cocchiere venisse verso di lui. 

Se ne ricordò solo quando il cocchiere gli fu proprio accosto e lo chiamò. 

- Ha mandato la signora. Sono arrivati il fratello e anche un certo signore. 

Levin salì sul calesse e prese le redini. 

Come se si fosse svegliato da un sonno, Levin stentò a tornare in sé. Esaminava il cavallo ben pasciuto che s'era ricoperto di schiuma fra le cosce e sul collo, dove fregavano le redini, esaminava il cocchiere Ivan, ch'era seduto accanto a lui, e ricordava che attendeva il fratello, che sua moglie probabilmente era inquieta per la sua lunga assenza, e cercava di indovinare chi fosse l'ospite arrivato col fratello. E il fratello, e la moglie e l'ospite ignoto adesso gli apparivano diversamente da prima. Gli sembrava che ormai i rapporti con tutte le persone sarebbero stati diversi. 

``Con mio fratello adesso non ci sarà più quel distacco che c'è sempre stato fra noi! discussioni non ce ne saranno; con Kitty non ci saranno più litigi; con l'ospite, chiunque esso sia, sarò affabile e buono; con la servitù, con Ivan, tutto sarà diverso''. 

Trattenendo con le redini tese il buon cavallo che sbuffava d'impazienza e chiedeva di camminare, Levin si voltava a guardare Ivan seduto accanto a lui, che non sapeva che cosa fare con le sue mani rimaste inoperose e premeva continuamente la camicia che si gonfiava. Cercava un pretesto per cominciare un discorso con lui. Egli voleva dire che Ivan aveva fatto male a tirare in alto la cinghia della stanga, ma questo poteva sembrare un rimprovero, e lui avrebbe voluto un discorso cordiale. Ma non gli veniva altro in mente. 

- Vi prego di prendere a destra, altrimenti troverete un ceppo - disse il cocchiere, correggendo Levin nella guida. 

- Per favore, non toccare e non farmi la lezione! - disse Levin, stizzito da questa intromissione del cocchiere. Come sempre, l'intromissione aveva causato la stizza, ed egli sentì subito, con tristezza, quanto errata fosse la supposizione che il suo stato d'animo potesse mutarlo immediatamente a contatto con la realtà. 

Circa un quarto di versta prima di giungere a casa, Levin scorse Tanja e Griša che gli correvano incontro. 

- Zio Kostja! Stanno venendo la mamma e il nonno, e Sergej Ivanyc, e ancora qualcuno - dicevano, arrampicandosi sul calesse. 

- Ma chi? 

- Uno terribilissimo! E fa così con le braccia - disse Tanja, sollevandosi sul calesse e facendo il verso a Katavasov. 

- Ma vecchio o giovane? - domandò ridendo Levin, al quale il gesticolare di Tanja aveva ricordato qualcuno. 

``Ah, purché non sia una persona spiacevole!'' pensò Levin. 

Soltanto dopo aver girato oltre la svolta della strada e dopo aver visto coloro che gli venivano incontro, Levin riconobbe Katavasov in cappello di paglia, che camminava agitando le braccia, proprio come aveva fatto Tanja. 

A Katavasov piaceva molto parlare di filosofia, avendone egli stesso ricevuta un'idea dai naturalisti che non si erano mai occupati di filosofia; e a Mosca, negli ultimi tempi, Levin aveva discusso molto con lui. 

E una di quelle conversazioni, in cui Katavasov, evidentemente, pensava d'aver avuto il sopravvento, fu la prima cosa che ricordò Levin, dopo averlo riconosciuto. 

``No, ormai discutere ed esprimere alla leggera le mie idee, non lo farò a nessun costo'' pensò. 

Sceso dal calesse e salutati il fratello e Katavasov, Levin domandò della moglie. 

- Ha portato Mitja al Kolok - era un bosco vicino casa. - Voleva sistemarlo là, in casa c'è caldo - disse Dolly. Levin aveva sempre sconsigliato la moglie di portare il bambino nel bosco, ritenendolo pericoloso, e questa notizia gli spiacque. 

- Corre con lui da una parte all'altra - disse sorridendo il principe. - Io le ho consigliato di portarlo sulla ghiacciaia. 

- Voleva venire dalle api. Pensava che tu fossi là - disse Dolly. 

- Be', che fai? - disse Sergej Ivanovic, rimanendo indietro agli altri e mettendosi all'altezza del fratello. 

- Ma, nulla di speciale. Come sempre mi occupo dell'azienda - rispose Levin. - Ebbene, tu rimani qui per molto? Ti aspettavamo da tempo. 

- Per un paio di settimane. C'è molto da fare a Mosca. 

A queste parole gli occhi dei fratelli si incontrarono e Levin, malgrado il desiderio costante e ora particolarmente forte in lui di essere in rapporti cordiali, e soprattutto semplici, con il fratello, sentì che provava un senso di disagio a guardarlo. Abbassò gli occhi, e non seppe che dire. 

Cercando argomenti di conversazione che potessero piacere a Sergej Ivanovic e distrarlo dal discorso sulla guerra serba e sulla questione slava, a cui aveva alluso accennando alle occupazioni di Mosca, Levin cominciò a parlare del libro di Sergej Ivanovic. 

- Ci sono state recensioni, dunque, del tuo libro? - domandò. 

Sergej Ivanovic sorrise della domanda premeditata. 

- Nessuno se ne occupa e io meno degli altri - disse. - Guardate, Dar'ja Aleksandrovna, ci sarà una pioggerella - soggiunse, indicando con l'ombrello alcune nuvolette bianche ch'erano apparse sopra le cime delle tremule. 

E bastarono queste parole perché quei rapporti reciproci non ostili, ma freddi, che Levin voleva tanto evitare, si stabilissero di nuovo tra i fratelli. Levin si avvicinò a Katavasov. 

- Come avete fatto bene a pensar di venire! - gli disse. 

- Mi preparavo da molto tempo. Adesso discorreremo, vedremo. Spencer l'avete letto? 

- No, non l'ho finito di leggere - disse Levin. - Del resto, non ne ho bisogno, ora. 

- Come mai? è interessante. Perché? 

- Cioè, mi sono definitivamente convinto che le soluzioni dei problemi che mi interessano non le troverò né in lui né in quelli simili a lui. Adesso\ldots{} 

Ma l'espressione calma e allegra del viso di Katavasov lo colpì, a un tratto, e gli venne pietà dello stato d'animo proprio, che, evidentemente, egli turbava con quella conversazione; si ricordò del suo proposito e si fermò. 

- Del resto, parleremo dopo - soggiunse. - Se vogliamo andare all'arniaio, allora di qua, per questo sentiero - disse rivolto a tutti. 

Giunti, per un sentiero stretto, a una prateria non falciata, coperta da una parte da violacciocche vivaci, in mezzo alle quali s'infoltivano i cespugli alti color verde scuro dell'elleboro, Levin dispose i suoi ospiti all'ombra spessa e fresca delle giovani tremule, su di una panchina e su di alcuni tronchi, preparati apposta per i visitatori dell'arniaio che temevano le api; lui stesso andò sul limitare, per portare ai bambini e ai grandi pane, cetrioli e miele fresco. 

Cercando di fare pochi movimenti rapidi e prestando ascolto alle api che gli volavano intorno sempre più frequenti, giunse per il sentiero fino all'izba. Proprio all'ingresso un'ape, impigliataglisi nella barba, cominciò a ronzare, ma egli la liberò con cautela. Entrato nell'ingresso ombroso, tolse dal muro la maschera appesa a un gancio e, messala e ficcate le mani in tasca, entrò nell'arniaio recinto. Qua, in file regolari, legate con rami di tiglio ai pali, stavano, in mezzo al luogo falciato, le vecchie arnie, ognuna con la propria storia, e lungo le pareti della siepe le nuove, messe lì quell'anno. Dinanzi alle aperture delle arnie le api e i fuchi, che giravano e si urtavano nello stesso posto, abbagliavano la vista, e in mezzo a loro, sempre nella stessa direzione, verso il bosco, su di un tiglio fiorito e indietro verso le arnie, passavano volando le api operaie cariche di polline e in cerca di bottino. 

Negli orecchi non cessavano di echeggiare suoni vari, ora d'un'ape operaia intenta al lavoro, che passava volando rapida, ora d'un fuco strombettante, ozioso, ora di api sentinelle agitate, che difendevano dal nemico il proprio bene, pronte a pungere. Dalla parte della cinta il vecchio piallava un cerchio e non aveva visto Levin che, in silenzio, si fermò in mezzo all'arniaio. 

Era contento dell'occasione di rimanere solo, per tornare in sé dalla realtà, che già aveva fatto in tempo a umiliare il suo stato d'animo. 

Ricordò che aveva già avuto il tempo di arrabbiarsi con Ivan, di mostrar freddezza verso il fratello e di parlare con leggerezza con Katavasov. 

``Possibile che sia stato solo lo stato d'animo di un istante e che tutto passi senza lasciare traccia?'' pensò. 

Ma nello stesso momento, tornato al proprio stato d'animo sentì con gioia che qualcosa di nuovo e d'importante era accaduto in lui. La realtà aveva velato solo temporaneamente quella calma dell'anima che egli aveva trovato, e che tuttavia era intatta in lui. 

Proprio come le api che adesso gli volteggiavano intorno, lo minacciavano e lo distraevano, gli toglievano la piena calma fisica, lo obbligavano a contrarsi per sfuggirle, proprio così le preoccupazioni, avvolgendolo, dal momento in cui era salito sul calesse, lo avevano privato della libertà dell'anima; ma ciò era durato soltanto finché vi era rimasto in mezzo. Come, malgrado le api, la sua forza fisica era intatta, così era anche intatta la sua forza spirituale, recentemente scoperta da lui. 

\capitolo{XV}\label{xv-7} 

- E sai, Kostja, con chi è venuto in qua Sergej Ivanovic? - disse Dolly, dopo aver distribuito cetrioli e miele ai bambini. - Con Vronskij! Va in Serbia. 

- E neppure solo, ma conduce uno squadrone a sue spese! - disse Katavasov. 

- Questo gli si addice - disse Levin. - Ma vanno forse ancora volontari? - soggiunse, dopo aver guardato Sergej Ivanovic. 

Sergej Ivanovic, senza rispondere, tirava fuori cautamente con un coltello smussato, da una tazza, nella quale c'era in un angolo un favo bianco di miele, un'ape ancora viva, appiccicatasi al miele ch'era colato di sotto. 

- E come ancora! Se aveste veduto cosa c'era ieri alla stazione! - disse Katavasov, addentando rumorosamente un cetriolo. 

- Su, e come capire ciò? In nome di Cristo, Sergej Ivanovic, spiegatemi: dove vanno tutti questi volontari, con chi sono in guerra? - domandò il vecchio principe, evidentemente continuando una conversazione avviata quando Levin ancora non c'era. 

- Coi turchi - rispose Sergej Ivanovic, sorridendo tranquillamente, dopo aver liberato l'ape che muoveva le zampine senza speranza di soccorso, annerita dal miele, e facendola scendere dal coltello su di una foglia spessa di tremula. 

- Ma chi ha mai dichiarato la guerra ai turchi? Ivan Ivanovic Ragozov e la contessa Lidija Ivanovna con la signora Stahl? 

- Nessuno ha dichiarato la guerra, ma la gente compatisce le sofferenze del prossimo e desidera aiutarlo - disse Sergej Ivanovic. 

- Ma il principe non parla d'aiuto - disse Levin, prendendo le parti del suocero - ma di guerra. Il principe dice che i privati non possono prendere parte a una guerra senza il permesso del governo. 

- Kostja, guarda, è un'ape! Davvero, ci pungerà tutti! - disse Dolly, difendendosi da una vespa. 

- Ma questa non è neanche un'ape, è una vespa - disse Levin. 

- Ebbene, qual'è la vostra teoria? - disse con un sorriso Katavasov a Levin, evidentemente per sfidarlo a una discussione. - Perché i privati non ne hanno il diritto? 

- Be', la mia teoria è questa: la guerra, da una parte è una cosa tanto bestiale, crudele e tremenda, che nessun uomo, non dico poi un cristiano, può prendersi personalmente la responsabilità di cominciare una guerra, ma lo può soltanto un governo che vi sia chiamato e vi sia condotto ineluttabilmente. Dall'altra parte, e secondo coscienza, e secondo il buon senso, negli affari di stato, in particolare nella questione di guerra, i cittadini rinunciano alla propria volontà personale. 

Sergej Ivanovic e Katavasov cominciarono a parlare nello stesso tempo con obiezioni già pronte. 

- Sta proprio lì il fatto, amico mio, che ci possono essere dei casi, in cui il governo non adempie la volontà dei cittadini, e allora la società dichiara la propria volontà - disse Katavasov. 

Ma Sergej Ivanovic, evidentemente, non approvava questa opinione. Alle parole di Katavasov aggrottò le sopracciglia e disse un'altra cosa. 

- Fai male a porre la questione così. Qui non c'è dichiarazione di guerra, ma semplicemente l'espressione di un sentimento umano, cristiano. Uccidono i fratelli, dello stesso sangue e della stessa fede. Ammettiamo che non siano né fratelli, né della stessa fede, ma semplicemente dei bambini, delle donne e dei vecchi; il sentimento s'indigna, e la gente russa corre per aiutare e far cessare questi orrori. Immagina di camminare per una strada e di vedere che degli ubriachi percuotono una donna o un bambino; io penso che tu non staresti là a domandare se sia dichiarata o no la guerra a quest'uomo, ma ti scaglieresti su di lui e difenderesti l'offeso. 

- Ma non l'ucciderei - disse Levin. 

- Sì, lo uccideresti. 

- Non so, se, vedendo questo, mi abbandonerei al mio sentimento immediato; ma in precedenza non posso dirlo. E un sentimento simile, immediato, per l'oppressione degli slavi non c'è e non può esserci. 

- Forse per te no. Ma per gli altri c'è - disse Sergej Ivanovic, aggrottando le sopracciglia, con aria scontenta. - Nel popolo è viva la tradizione sulla gente ortodossa che soffre sotto il giogo degli ``empi agareni''. Il popolo ha sentito le sofferenze dei suoi simili e ha parlato. 

- Forse - disse evasivo Levin - ma io non lo vedo; io stesso sono popolo e non lo sento. 

- Ecco, anch'io - disse il principe. - Stavo all'estero, leggevo i giornali e, confesso, ancora prima degli orrori bulgari, non capivo in nessun modo perché tutti i russi, a un tratto, avessero preso tanto ad amare i fratelli slavi, mentre io non sentivo nessun amore per loro. Mi addoloravo molto, pensavo d'essere un mostro o che fosse Karlsbad ad agire così su di me. Ma arrivato qua, mi sono tranquillizzato; vedo che oltre me ci sono anche altre persone che si interessano soltanto della Russia, e non dei fratelli slavi. Ecco, anche Konstantin. 

- Le opinioni individuali qui non significano niente - disse Sergej Ivanovic - non c'entrano le opinioni personali quando tutta la Russia, il popolo, ha espresso la sua volontà. 

- Ma scusatemi, io non lo vedo. Il popolo, quanto a saperlo, non lo sa - disse il principe. 

- No, papà\ldots{} e come no? E domenica in chiesa? - disse Dolly prestando ascolto alla conversazione.- Dammi un asciugamano, per favore - disse al vecchio che guardava i bambini con un sorriso. - Non può essere che tutti\ldots{} 

- Ma cosa domenica in chiesa? Al prete hanno dato l'ordine di leggere. Lui ha letto. Loro non hanno capito nulla, sospiravano come ad ogni predica - seguitò il principe. - Poi hanno detto loro che, ecco, si faceva una colletta in chiesa per un'impresa salutare per l'anima; ebbene, loro hanno tirato fuori una copeca per ciascuno e l'hanno data, ma per cosa, non lo sanno loro stessi. 

- Il popolo non può non sapere; la coscienza dei propri destini c'è sempre nel popolo, e in momenti come questi essa si chiarisce in lui - disse recisamente Sergej Ivanovic, gettando uno sguardo al vecchio apicultore. 

Il bel vecchio dalla barba nera con qualche pelo grigio e dai capelli folti d'argento, stava in piedi, immobile, tenendo la tazza col miele, guardando i signori con cordialità e con calma dall'alto della sua statura, evidentemente senza capire e non desiderando di capire nulla. 

- È proprio così - diss'egli alle parole di Sergej Ivanovic, scotendo significativamente il capo. 

- Ma ecco, domandate a lui. Lui non sa e non pensa nulla - disse Levin. - Hai sentito della guerra, Michajlyc? - si rivolse a lui. - Ecco, quello che hanno letto in chiesa. Tu che ne pensi mai? Dobbiamo far la guerra per i cristiani? 

- Che dobbiamo mai pensare? Aleksandr Nikolaevic, lo zar, ha pensato per noi, penserà per noi anche in tutti gli affari. Lui vede meglio\ldots{} Devo portare dell'altro pane? Posso darne ancora al bambino? - disse rivolto a Dar'ja Aleksandrovna, indicando Griša, che finiva di mangiare una crosta. 

- Io non ho bisogno di domandare - disse Sergej Ivanovic; - abbiamo visto e vediamo centinaia e centinaia di persone che abbandonano tutto per servire una causa giusta, vengono da tutte le estremità della Russia ed esprimono sinceramente e chiaramente il proprio pensiero e scopo. Portano i loro soldi o vanno loro stessi e dicono sinceramente perché. E cosa vuol dire questo? 

- Vuol dire, secondo me - disse Levin che cominciava a scaldarsi - che in un popolo di ottanta milioni di uomini si troveranno sempre non centinaia, come adesso, ma decine di migliaia di persone che hanno perduto una posizione sociale, persone turbolente, che sono sempre pronte a entrare nella banda di Pugacëv, a Chiva, in Serbia\ldots{} 

- Io ti dico che non sono centinaia e non sono persone turbolente, ma i migliori rappresentanti del popolo! - disse Sergej Ivanyc con un'irritazione tale, come se avesse difeso il suo ultimo bene. - E le offerte? In questo tutto il popolo esprime la propria volontà. 

- Questa parola ``popolo'' è indeterminata - disse Levin. - Gli scrivani comunali, i maestri e i contadini uno su mille, forse, sanno di che si tratta. Ma gli altri ottanta milioni, come Michajlyc, non solo non esprimono la propria volontà, ma non hanno neppure la minima idea di quello su cui dovrebbero esprimere la propria volontà. Che diritto abbiamo mai di dire che è la volontà del popolo? 

\capitolo{XVI}\label{xvi-7} 

Sergej Ivanovic, esperto in dialettica, senza obiettare, portò immediatamente la conversazione su di un altro campo. 

- Ma se tu vuoi conoscere lo spirito del popolo in maniera matematica, allora, s'intende, ottenere ciò è molto difficile. E il suffragio non è introdotto da noi e non può essere introdotto, perché non esprime la volontà del popolo; ma per questo ci sono altre vie. Si sente nell'aria, si sente nel cuore. Non parlo poi di quelle correnti sottomarine, che si sono mosse nel mare stagnante del popolo e che sono chiare per qualsiasi persona che non abbia prevenzioni; guarda la società in senso stretto. Tutti i più svariati partiti del mondo dell'intelligencija, tanto ostili prima, si sono fusi in una sola cosa. Ogni dissenso è finito, tutti gli organi pubblici dicono sempre la stessa cosa, tutti hanno sentito la forza naturale che li ha afferrati e li porta in una stessa direzione. 

- Ma sono i giornali che dicono tutti la stessa cosa - disse il principe. - È vero. Tutti la stessa cosa, proprio come ranocchie prima del temporale. E proprio per causa loro non si può sentir nulla. 

- Ranocchie o non ranocchie, io giornali non ne pubblico e non li voglio difendere; ma parlo dell'unità di pensiero nel mondo dell'intelligencija - disse Sergej Ivanovic, rivolto al fratello. Levin voleva rispondere, ma il vecchio principe lo interruppe. 

- Eh, via, su questa unità di pensiero si può dire ancora un'altra cosa - disse il principe. - Ecco, io ho un genero, Stepan Arkad'ic, lo conoscete. Adesso ha avuto il posto di membro del comitato di una commissione e qualcosa ancora, non ricordo. Certo là non c'è nulla da fare; ebbene, Dolly, non è mica un segreto, ma ci sono ottomila rubli di stipendio. Provate, chiedetegli se il suo impiego è utile, vi dimostrerà ch'è utilissimo. Ed è uomo sincero, ma non si può non credere all'utilità degli ottomila rubli. 

- Sì, mi ha pregato di riferire a Dar'ja Aleksandrovna che ha avuto il posto - disse scontento Sergej Ivanovic, ritenendo che il principe parlasse a sproposito. 

- Appunto così è l'unità di pensiero dei giornali. Me l'hanno spiegato: non appena c'è una guerra, hanno un reddito due volte maggiore. E come non devono ritenere che le sorti del popolo e degli slavi\ldots{} e tutto ciò? 

- A me molti giornali non piacciono, ma questo è ingiusto - disse Sergej Ivanovic. 

- Io porrei soltanto una condizione - seguitò il principe. - Alphonse Karr lo scrisse benissimo prima della guerra di Prussia: ``Voi stimate che la guerra sia indispensabile? Benissimo. Chi predica la guerra, vada in una legione speciale, d'avanguardia, e all'assalto, all'attacco, innanzi a tutti''. 

- Saranno belli i direttori! - disse Katavasov, mettendosi a ridere forte, immaginando i direttori che conosceva in quella legione scelta. 

- Macché, scapperanno via - disse Dolly - daranno soltanto noia. 

- E se scapperanno, di dietro bisognerà tirare a mitraglia o metter dei cosacchi con le fruste - disse il principe. 

- Ma questo è uno scherzo, e uno scherzo poco buono, scusatemi principe - disse Sergej Ivanovic. 

- Io non vedo come questo sia uno scherzo, che\ldots{} - voleva cominciare Levin, ma Sergej Ivanovic lo interruppe. 

- Ogni membro della società è chiamato a fare il lavoro che gli è proprio - disse egli. - Anche gli uomini di pensiero compiono il loro lavoro esprimendo l'opinione pubblica. E l'unanime e piena espressione dell'opinione pubblica è un merito della stampa e nello stesso tempo un fenomeno che rallegra. Vent'anni fa avremmo taciuto, e ora si sente la voce del popolo russo, che è pronto a levarsi come un solo uomo, ed è pronto a sacrificarsi per i fratelli oppressi; è un grande passo e un pegno di forza. 

- Ma non si tratta mica soltanto di far dei sacrifici, ma anche di uccidere i turchi - disse timido Levin. - Il popolo fa sacrifici ed è pronto a far sacrifici per la propria anima, e non per uccidere - egli soggiunse, collegando involontariamente la conversazione con le idee che lo occupavano tanto. 

- Come per l'anima? Per un naturalista, capirete, è una espressione imbarazzante. Che cos'è l'anima? - disse sorridendo Katavasov. 

- Ah, lo sapete! 

- Ecco, com'è vero Dio, non ne ho la più pallida idea! - disse con una forte risata Katavasov. 

- ``Io non sono venuto a metter pace, ma guerra'' dice Cristo - obiettò da parte sua Sergej Ivanyc, riportando semplicemente, come fosse la cosa più comprensibile, quello stesso passo del Vangelo che più di tutti tormentava Levin. 

- È proprio così - ripeté di nuovo il vecchio, che stava proprio accanto a loro, rispondendo a un'occhiata gettata per caso su di lui. 

- No, amico mio, siete battuto, battuto, completamente battuto - gridò allegramente Katavasov. 

Levin arrossì di stizza, non perché era stato battuto, ma perché s'era intrattenuto e s'era messo a discutere. 

``No, io non posso discutere con loro - pensava - loro hanno addosso una corazza impenetrabile, e io sono nudo''. 

Vedeva che convincere il fratello e Katavasov non si poteva, e ancor meno vedeva la possibilità di consentire con loro. Ciò che essi predicavano era quella stessa superbia d'intelletto che lo aveva quasi rovinato. Non poteva consentire che decine di persone, nel cui numero era anche suo fratello, avessero il diritto, in base a quello che dicevano loro le centinaia di ciarlatani volontari che giungevano nelle capitali, di affermare che essi, con i giornali, esprimevano la volontà e il pensiero del popolo, e un pensiero che s'esprimeva nella vendetta e nell'uccisione. Non poteva consentire con questo, perché non vedeva l'espressione di questi pensieri nel popolo in mezzo al quale viveva, e non trovava questi pensieri in se stesso (ed egli non poteva considerarsi nient'altro che una delle persone componenti il popolo russo), e soprattutto perché insieme col popolo non sapeva, non poteva sapere in che cosa consistesse il bene comune, ma sapeva con certezza che il raggiungimento di questo bene comune era possibile soltanto con un severo adempimento di quella legge di bontà che si rivela a ogni uomo, e perciò non poteva desiderare la guerra e predicare in favore di scopi comuni, quali che essi fossero. Egli diceva, insieme con Michajlyc e col popolo, che aveva espresso il proprio pensiero nella leggenda della chiamata dei Variaghi: ``Regnate e siate i nostri signori. Noi promettiamo con gioia una completa sottomissione. Tutto il lavoro, tutte le umiliazioni, tutti i sacrifici ce li assumiamo noi; ma che non siamo noi a giudicare e decidere''. E ora il popolo, secondo le parole di Sergej Ivanyc, rinunciava a tale diritto, comprato a così caro prezzo. 

Desiderava ancora dire che, se l'opinione pubblica era giudice infallibile, allora perché la rivoluzione, la Comune non erano altrettanto legittime come il movimento a favore degli slavi? Ma tutti questi erano pensieri che non potevano decidere nulla. Una sola cosa si poteva indubbiamente vedere: era che nel momento presente la discussione irritava Sergej Ivanovic e perciò discutere era male; e Levin tacque e rivolse l'attenzione degli ospiti sul fatto che le nuvolette s'erano addensate e che, per sfuggire alla pioggia, era meglio tornarsene a casa. 

\capitolo{XVII}\label{xvii-7} 

Il principe e Sergej Ivanyc salirono sul calesse e andarono via; gli altri affrettarono il passo, ritornarono a casa a piedi. 

Ma la nuvola ora sbiancandosi, ora oscurandosi, avanzava così rapida che bisognava accelerare ancora di velocità per giungere in tempo a casa prima della pioggia. Le nuvole che avanzavano, basse e nere, come fumo misto a fuliggine, correvano per il cielo con straordinaria rapidità. Fino a casa c'erano ancora duecento passi, e s'era già alzato il vento e, da un secondo all'altro, ci si poteva aspettare un acquazzone. 

I bambini correvano avanti con uno stridio spaventato e gioioso. Dar'ja Aleksandrovna, lottando con fatica con le sottane che le si erano come incollate alle gambe, non camminava più, ma correva, senza togliere gli occhi di dosso ai fanciulli. Gli uomini, trattenendo il cappello, camminavano a passi lunghi. Erano già proprio alla scalinata, quando una grossa goccia batté e si infranse contro l'estremità della grondaia di ferro. I bambini, e dietro loro i grandi, corsero con allegro vocìo sotto la protezione del tetto. 

- Katerina Aleksandrovna? - domandò Levin ad Agaf'ja Michajlovna che li aveva accolti nell'anticamera con fazzoletti e scialli. 

- Pensavamo fosse con voi - ella disse. 

- E Mitja? 

- Al Kolok, forse, e la njanja è con lui. 

Levin afferrò gli scialli e corse via al Kolok. 

In quel breve spazio di tempo la nuvola s'era già tanto avanzata col suo centro sul sole, che s'era fatto buio come in una eclissi. Il vento pareva insistere ostinatamente, fermava Levin e, strappando le foglie e i fiori dai tigli e rendendo spogli, informi e strani i rami bianchi delle betulle, piegava tutto da un sol lato: le acacie, i fiori, la bardana, l'erba e le cime degli alberi. Le ragazze, che lavoravano in giardino, corsero strillando sotto il tetto della camera della servitù. La cortina bianca della pioggia dirotta aveva già invaso tutta la selva lontana e metà dei campi vicini, e si avanzava rapida verso il Kolok. L'umidità della pioggia, frantumata in gocce minute, si sentiva nell'aria. 

Piegando la testa in avanti e lottando col vento che gli strappava gli scialli, Levin giungeva di corsa al Kolok e vedeva già qualcosa di bianco dietro una quercia, quando, a un tratto, tutto si infiammò, tutta la terra prese fuoco e fu come se, sopra il capo, gli si fosse spaccata la volta celeste. Aperti gli occhi abbacinati, Levin, attraverso lo spesso velo della pioggia, che adesso lo separava dal Kolok, vide con orrore prima di tutto la cima verde della quercia, a lui nota, in mezzo al bosco, che aveva mutato stranamente la sua posizione. ``Possibile che l'abbia spezzata?'' fece appena in tempo a pensare Levin, accelerando sempre più il passo, quando la cima della quercia si nascose dietro alle altre piante, ed egli sentì lo schianto del grande albero caduto sugli altri. 

La luce del fulmine, il fragore del tuono e la sensazione del corpo istantaneamente percosso dal freddo si fusero per Levin in una sola impressione d'orrore. 

- Dio mio! Dio mio! che non sia su di loro! - esclamò. 

E sebbene pensasse subito come fosse insensata la propria implorazione perché i suoi non fossero uccisi dalla quercia caduta in quel momento, la ripeté, sapendo che non poteva far nulla di meglio di quella preghiera insensata. 

Giunto di corsa al posto dove erano di solito, non li trovò. Erano all'altra estremità del bosco, sotto un vecchio tiglio, e lo chiamavano. Due figure, vestite di scuro (prima erano in chiaro), stavano curve sopra qualcosa. Erano Kitty e la njanja. La pioggia diminuiva e cominciava già a schiarire, quando Levin giunse correndo presso di loro. La njanja aveva la parte inferiore del vestito asciutto, ma addosso a Kitty il vestito s'era bagnato da parte a parte, e si era incollato tutto al corpo. Benché non piovesse più, esse rimanevano sempre nella stessa posizione in cui s'erano messe quando s'era scatenato il temporale: tutte e due stavano in piedi, chine sopra la carrozzina con un ombrellino verde. 

- Vivi? intatti? Sia lodato Iddio! - esclamò Levin, sguazzando nell'acqua rimastagli in una scarpa che gli sfuggiva, piena d'acqua, e accorrendo presso di loro. 

Il viso arrossato e bagnato di Kitty era rivolto a lui e sorrideva timido sotto il cappello che aveva cambiato foggia. 

- Ma come non ti vergogni! Non capisco come si possa essere così imprudenti! - egli assalì con stizza la moglie. 

- Com'è vero Dio, non ne ho colpa. Volevo proprio allora andarmene, quando lui ci ha fatto perder tempo. Bisognava cambiarlo. Aveva appena\ldots{} - cominciò a scusarsi Kitty. 

Mitja era incolume, asciutto e non aveva cessato di dormire. 

- Via, sia lodato Iddio! non so quello che dico. 

Raccolsero le fasce bagnate, la njanja tirò fuori il bambino e lo portò in braccio. Levin camminava accanto alla moglie, stringendole la mano, di nascosto alla njanja, con aria colpevole per la propria stizza. 

\capitolo{XVIII}\label{xviii-7} 

Durante tutto il giorno, nelle conversazioni più svariate alle quali sembrava partecipare soltanto con la parte esteriore del proprio intelletto, Levin, malgrado la delusione per il mutamento che doveva avvenire in lui, non cessava di sentire con gioia la pienezza del proprio cuore. 

Dopo la pioggia era troppo umido per andare a passeggio; inoltre anche le nuvole temporalesche non sparivano dall'orizzonte e ora là, ora qua passavano, tonando e facendosi scure all'estremità del cielo. Tutta la compagnia trascorse il resto della giornata in casa. 

Discussioni non se ne intavolarono più, e, al contrario, dopo pranzo tutti furono nella migliore disposizione d'animo. 

Katavasov, da principio, fece ridere le signore coi suoi scherzi originali, che piacevano sempre tanto appena lo si conosceva; ma poi, invitato da Sergej Ivanovic, raccontò le sue osservazioni molto interessanti sulla differenza di carattere e perfino di fisionomia tra femmine e maschi delle mosche domestiche, e sulla loro vita. Anche Sergej Ivanovic fu allegro, e al tè, invitato dal fratello, espose il suo punto di vista sull'avvenire della questione orientale, e così semplicemente e bene che tutti l'ascoltarono con piacere. 

Solamente Kitty non poté ascoltarlo fino in fondo: la chiamarono a fare il bagno a Mitja. 

Di lì a poco, dopo Kitty, chiamarono anche Levin nella camera del bambino. 

Lasciato il suo tè e rammaricandosi anche lui di interrompere una conversazione interessante, e nello stesso tempo inquieto perché lo chiamavano (giacché questo accadeva in casi rari), Levin andò nella camera del bambino. 

Sebbene le idee non finite di ascoltare di Sergej Ivanovic, su come il mondo liberato dagli slavi e forte di quaranta milioni di uomini dovesse, con la Russia, incominciare una nuova epoca nella storia, lo interessassero molto, come qualcosa di completamente nuovo per lui, e sebbene l'inquietassero la curiosità e l'ansia per la chiamata insolita, pure, quando si trovò solo, fuori dal salotto, egli ricordò immediatamente i suoi pensieri del mattino. E tutte quelle considerazioni sull'importanza dell'elemento slavo nella storia mondiale gli apparvero così insignificanti in confronto a ciò che accadeva nell'anima sua, che dimenticò subito tutto quanto, e si riportò al medesimo stato d'animo che aveva avuto la mattina. 

Adesso non ricordava, come gli accadeva prima, tutta la successione del pensiero (non ne aveva bisogno). Si riportò immediatamente al sentimento che l'aveva guidato, che era collegato a quei pensieri, e trovò nell'anima sua questo sentimento ancora più forte e definito di prima. Adesso non gli accadeva quello che accadeva nei tentativi di calmarsi che si inventava prima, quando bisognava ricostruire tutta la successione del pensiero per trovare il sentimento. Adesso, al contrario, il senso di gioia e di serenità era più vivo di prima e il pensiero non poteva tener dietro al sentimento. 

Camminava su e giù per la terrazza e guardava due stelle che apparivano nel cielo già fattosi scuro e, a un tratto, ricordò: ``Sì, guardando il cielo, pensavo che la volta che vedevo non era un'illusione\ldots{} però, non ho finito di pensare a qualcosa che devo aver nascosto a me stesso. Ma, qualunque cosa sia, non potranno esservi obiezioni. Basterà pensare un po' e tutto si chiarirà!''. 

Mentre stava per entrare nella camera del bambino, ricordò quello ch'egli aveva nascosto a se stesso. Se la dimostrazione principale della Divinità era la Sua rivelazione di quello che è bene, perché allora questa rivelazione si limitava alla sola Chiesa cristiana? Che rapporti avevano con questa rivelazione le credenze dei buddisti, dei maomettani, che anch'essi professavano e operavano il bene? 

Gli sembrava d'aver una risposta a questa domanda; ma non fece in tempo a esprimerla a se stesso, che era già entrato nella camera del bambino. 

Kitty era in piedi, con le maniche rimboccate, accanto alla vasca da bagno, curva sul bambino che veniva lavato e, sentiti i passi del marito, voltò il viso a lui, e col sorriso lo chiamò a sé. Con una mano sosteneva sotto il capo il bambino paffuto che annaspava sulla schiena e divaricava le gambette; con l'altra, tendendo il muscolo, premeva su di lui la spugna. 

- Su, ecco, guarda, guarda! - ella disse quando il marito le fu accanto. - Agaf'ja Michajlovna ha ragione: riconosce. 

Si trattava del fatto che quel giorno Mitja evidentemente, senza più alcun dubbio, riconosceva tutti i suoi. 

Non appena Levin si fu avvicinato alla vasca da bagno, gli fu offerto subito un esperimento, e l'esperimento riuscì in pieno. La cuoca, chiamata apposta per questo, si chinò sopra il bambino. Egli aggrottò le sopracciglia e scosse il capo negativamente. Si chinò Kitty sopra di lui, egli s'illuminò d'un sorriso, si appoggiò con le mani alla spugna e fece un suono con le labbra così soddisfatto e strano, che non solo Kitty e la njanja, ma anche Levin venne preso da un improvviso entusiasmo. 

Tirarono fuori il bambino dal bagno su una mano sola, gli versarono dell'acqua addosso, lo involtolarono in un lenzuolo, lo asciugarono e, dopo un acuto gridio, lo tesero alla madre. 

- Be', sono contenta che cominci a volergli bene - disse Kitty al marito, dopo che, col bambino al petto, si fu messa tranquillamente a sedere, al posto abituale. - Sono molto contenta. Se no questo cominciava già ad addolorarmi. Tu dicevi che non sentivi niente per lui. 

- No, dicevo forse di non sentire? Dicevo soltanto che ero deluso. 

- Come, di che deluso? 

- Non che fossi deluso di lui, ma del mio sentimento; m'aspettavo di più. Mi aspettavo che, come una sorpresa, sarebbe sbocciato in me un nuovo, piacevole sentimento. E a un tratto, invece di questo, ripugnanza, compassione\ldots{} 

Ella lo ascoltava attenta, curva sul bimbo, infilando nelle dita sottili gli anelli che aveva tolto per fare il bagno a Mitja. 

- E soprattutto, c'è tanta più apprensione e tanta più pena che non piacere. Oggi, dopo quello spavento, durante il temporale, ho capito come gli voglio bene. 

Kitty si illuminò d'un sorriso. 

- Ti sei spaventato molto? - ella disse. - Anch'io, ma ne sento più spavento ora che è passato. Andrò a vedere la quercia. E come è simpatico Katavasov! Ma, in generale, tutto il giorno è stato così piacevole. E tu con Sergej Ivanyc sei così caro, quando vuoi\ldots{} Su, va' da loro, dopo il bagno, qui, c'è sempre caldo e vapore. 

\capitolo{XIX}\label{xix-7} 

Uscito dalla camera del bambino e rimasto solo, Levin ricordò subito il pensiero in cui c'era qualcosa di poco chiaro. 

Invece di andare nel salotto, dal quale giungevano le voci, si fermò sulla terrazza e, appoggiatosi coi gomiti alla balaustrata, si mise a guardare il cielo. 

S'era già fatto completamente buio, e a sud, dove egli guardava, non c'erano nuvole. Le nuvole erano dalla parte opposta. Di là s'accendeva un lampeggio e si sentiva un rimbombo lontano. Levin prestava ascolto alle gocce che cadevano eguali dai tigli in giardino e guardava il triangolo di stelle a lui noto e la Via Lattea con la ramificazione che l'attraversava. A ogni guizzo di lampo non solo la Via Lattea, ma anche le stelle lucenti sparivano; non appena il lampo si spegneva, di nuovo, come gettate da una mano precisa, apparivano tutte negli stessi punti. 

``Ebbene, cos'è che mi turba?'' si disse Levin, sentendo in precedenza che la soluzione del dubbio, pur senza saperlo, era già pronta nell'animo suo. 

``Sì, l'unica evidente, indubitabile manifestazione della Divinità è la legge del bene, che è manifestata al mondo dalla rivelazione e che io sento in me, e nel riconoscimento di questa non è che mi unisca, ma, volere o no, sono unito con gli altri uomini in una sola società di credenti che si chiama la Chiesa. Già, e gli ebrei, i maomettani, i confucianisti, i buddisti, che cosa sono mai? - e si pose quella domanda che gli sembrava così pericolosa. - Possibile che queste centinaia di milioni di uomini siano privati di quel bene migliore, senza il quale la vita non ha senso?''. Si fece pensieroso, ma immediatamente si corresse. ``Ma cosa mai mi chiedo? - si disse. - Chiedo il rapporto che hanno con la Divinità le più svariate credenze dell'umanità intera. Domando della comune manifestazione di Dio per tutto l'universo con tutte queste nebulose. E che faccio? A me personalmente, al mio cuore è aperta una conoscenza indubitabile, irraggiungibile con la ragione, e io ostinatamente voglio esprimere con la ragione e a parole questa conoscenza''. 

``Non so forse che le stelle non camminano?'' egli domandò interrogando un vivido pianeta che aveva già cambiata la propria posizione passando al ramo superiore d'una betulla. 

``Ma io, guardando il moto delle stelle, non posso immaginarmi la rotazione della terra, e ho ragione di dire che le stelle camminano. 

E gli astronomi potrebbero forse capire e calcolare qualcosa, se prendessero in considerazione tutti i complessi e svariati movimenti della terra? Tutte le loro meravigliose conclusioni sulle distanze, sul peso, sui moti e le rivoluzioni dei corpi celesti sono basate soltanto sul movimento apparente degli astri intorno alla terra immobile, su quello stesso moto che adesso è dinanzi a me e che è stato così per milioni di persone durante secoli ed è stato e sarà sempre eguale e potrà essere sempre controllato. E proprio così come sarebbero oziose e incerte le conclusioni degli astronomi non basate sulle osservazioni del cielo visibile, in rapporto con un meridiano e un orizzonte, così sarebbero oziose e incerte anche le mie conclusioni non basate su quella comprensione del bene, che è stata e sarà sempre eguale per tutti e che mi viene dischiusa dal cristianesimo e può essere sempre controllata nell'anima mia. La questione poi delle altre credenze e dei loro rapporti con la Divinità, non ho io il diritto e la possibilità di risolverla''. 

- Ah, non te ne sei andato? - disse a un tratto la voce di Kitty, che tornava in salotto. - Che hai, sei agitato da qualcosa? - ella disse, osservandogli attentamente il viso al chiarore delle stelle. 

Tuttavia ella non avrebbe scorto bene il viso di lui se di nuovo un lampo, che nascose le stelle, non lo avesse illuminato. Alla luce del lampo ella guardò bene tutto il suo viso e, avendo visto ch'egli era calmo e gioioso, gli sorrise. 

``Lei capisce - egli pensava - sa a che cosa penso. Devo dirglielo o no? Sì, glielo dirò''. Ma nel momento in cui egli voleva cominciare a parlare prese a parlare anche lei. 

- Ecco, Kostja, fammi un piacere - ella disse - va' nella stanza d'angolo e guarda come hanno accomodato tutto per Sergej Ivanovic. Che ci vada io non sta bene. L'hanno messo il lavabo nuovo? 

- Va bene, ci andrò senz'altro - disse Levin, alzandosi e baciandola. 

``No, non bisogna parlare - egli pensò, quand'ella gli passò avanti. - È un segreto necessario, importante per me solo e inesprimibile a parole. 

Questo nuovo sentimento non mi ha cambiato, non mi ha reso felice, non mi ha rischiarato a un tratto, come sognavo, proprio come il sentimento per mio figlio. Anche qui non c'è stata nessuna sorpresa. E fede o non fede, non so cosa sia, ma questo sentimento è entrato in me egualmente inavvertito, attraverso la sofferenza, e si è fermato saldamente nell'anima. 

Mi arrabbierò sempre alla stessa maniera contro Ivan il cocchiere, sempre alla stessa maniera discuterò, esprimerò a sproposito le mie idee, ci sarà lo stesso muro fra il tempio dell'anima mia e quello degli altri, e perfino mia moglie accuserò sempre alla stessa maniera del mio spavento e ne proverò rimorso; sempre alla stessa maniera, non capirò con la ragione perché prego e intanto pregherò, ma la mia vita adesso, tutta la mia vita, indipendentemente da tutto quello che mi può accadere, ogni suo attimo, non solo non è più senza senso, come prima, ma ha un indubitabile senso di bene, che io ho il potere di trasfondere in essa!''. 

\vspace*{\fill}
\begin{center}\sffamily\textbf{FINE}\end{center}
\vspace*{\fill}

\cleardoublepage

\vspace*{\fill}
\begin{center}\sffamily Questo documento è stato composto con \Hologo{LuaLaTeX}.\end{center}

\cleardoublepage

\pagecolor{rosso}\null\thispagestyle{empty}
\end{document}