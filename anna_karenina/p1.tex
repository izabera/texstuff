\parte{PARTE PRIMA}
\pagestyle{pagina}

\capitolo{I}Tutte le famiglie felici sono simili le une alle altre; ogni famiglia infelice è infelice a modo suo.

Tutto era sottosopra in casa Oblonskij. La moglie era venuta a sapere che il marito aveva una relazione con la governante francese che era stata presso di loro, e aveva dichiarato al marito di non poter più vivere con lui nella stessa casa. Questa situazione durava già da tre giorni ed era sentita tormentosamente dagli stessi coniugi e da tutti i membri della famiglia e dai domestici. Tutti i membri della famiglia e i domestici sentivano che non c'era senso nella loro convivenza, e che della gente incontratasi per caso in una qualsiasi locanda sarebbe stata più legata fra di sé che non loro, membri della famiglia e domestici degli Oblonskij. La moglie non usciva dalle sue stanze; il marito era già il terzo giorno che non rincasava. I bambini correvano per la casa abbandonati a loro stessi; la governante inglese si era bisticciata con la dispensiera e aveva scritto un biglietto ad un'amica chiedendo che le cercasse un posto; il cuoco se n'era già andato via il giorno prima durante il pranzo; sguattera e cocchiere avevano chiesto di essere liquidati. 

Tre giorni dopo il litigio, il principe Stepan Arkad'ic Oblonskij - Stiva, com'era chiamato in società - all'ora solita, cioè alle otto del mattino, si svegliò non nella camera della moglie, ma nello studio, sul divano marocchino. Rigirò il corpo pienotto e ben curato sulle molle del divano, come se volesse riaddormentarsi di nuovo a lungo, rivoltò il cuscino, lo abbracciò forte e vi appoggiò la guancia; ma a un tratto fece un balzo, sedette sul divano e aprì gli occhi. 

``Già già, com'è andata? - pensava riandando al sogno. - Già, com'è andata? Ecco\ldots{} Alabin aveva dato un pranzo a Darmstadt; no, non Darmstadt, ma qualcosa d'America. Già, ma là, Darmstadt era in America. Sì, sì, Alabin aveva dato un pranzo su tavoli di vetro, già, e i tavoli cantavano `Il mio tesoro', eh no, non `Il mio tesoro', ma qualcosa di meglio; e c'erano poi certe piccole caraffe, ed anche queste erano donne'' ricordava. 

Gli occhi di Stepan Arkad'ic presero a brillare allegramente ed egli ricominciò a pensare sorridendo: ``Eh già, si stava bene, tanto bene. Ottime cose là; ma prova un po' a parlarne e a pensarne; da sveglio neanche arrivi a dirle''. E, notata una striscia di luce che filtrava da un lato della cortina di panno, sporse allegramente i piedi fuori dal divano, cercò con essi le pantofole di marocchino dorato ricamategli dalla moglie (dono per l'ultimo suo compleanno), e per vecchia abitudine, ormai di nove anni, senza alzarsi, allungò il braccio verso il posto dove, nella camera matrimoniale, era appesa la vestaglia. E in quel momento, a un tratto, ricordò come e perché non dormiva nella camera della moglie, ma nello studio, il sorriso gli sparve dal volto; corrugò la fronte. 

- Ahi, ahi, ahi! - mugolò, ricordando quanto era accaduto, e gli si presentarono di nuovo alla mente tutti i particolari del litigio, la situazione senza via di uscita e, più tormentosa di tutto, la propria colpa. 

``Già, lei non perdonerà, non può perdonare. E quel ch'è peggio è che la colpa di tutto è mia\ldots{} la colpa è mia, eppure non sono colpevole! Proprio in questo sta il dramma'' pensava. ``Ahi, ahi, ahi!'' ripeteva con disperazione, ricordando le impressioni più penose per lui di quella rottura. 

Più spiacevole di tutto il primo momento, quando, tornato da teatro, allegro e soddisfatto, con un'enorme pera in mano per la moglie, non l'aveva trovata nel salotto; con sorpresa non l'aveva trovata neanche nello studio, e infine l'aveva scorta in camera con in mano il malaugurato biglietto che aveva rivelato ogni cosa. 

Lei, quella Dolly eternamente preoccupata e inquieta, e non profonda, come egli la giudicava, sedeva immobile, con il biglietto in mano, e lo guardava con un'espressione di orrore, d'esasperazione e di rabbia. 

- Cos'è questo biglietto, cos'è? - chiedeva mostrando il biglietto. 

E a quel ricordo, come talvolta accade, ciò che tormentava Stepan Arkad'ic non era tanto il fatto in se stesso, quanto il modo col quale egli aveva risposto alle parole della moglie. 

Gli era accaduto in quel momento quello che accade alle persone che vengono inaspettatamente accusate di qualcosa di troppo vergognoso. Non aveva saputo adattare il viso alla situazione in cui era venuto a trovarsi di fronte alla moglie dopo la scoperta della propria colpa. Invece di offendersi, negare, giustificarsi, chiedere perdono, rimanere magari indifferente - tutto sarebbe stato meglio di quel che aveva fatto - il suo viso, in modo del tutto involontario (azione riflessa del cervello, pensò Stepan Arkad'ic, che amava la fisiologia), in modo del tutto involontario, aveva improvvisamente sorriso del suo usuale, buono e perciò stupido sorriso. 

Questo stupido sorriso non riusciva a perdonarselo. Visto quel sorriso, Dolly aveva rabbrividito come per un dolore fisico; era scoppiata, con l'impeto che le era proprio, in un diluvio di parole dure, ed era corsa via di camera. Da quel momento non aveva più voluto vedere il marito. 

``Tutta colpa di quello stupido sorriso - pensava Stepan Arkad'ic. - Ma che fare, che fare?'' si chiedeva con disperazione, e non trovava risposta 

\capitolo{II}Stepan Arkad'ic era un uomo leale con se stesso. Non poteva ingannare se stesso e convincersi d'essere pentito del suo modo di agire. Non poteva, in questo momento, pentirsi di non essere più innamorato - lui, bell'uomo trentaquattrenne, facile all'amore - di sua moglie, di un anno solo più giovane, madre di cinque bambini vivi e di tre morti. Era pentito solo di non averlo saputo nascondere più abilmente alla moglie. Ma sentiva tutto il peso di questa situazione e commiserava la moglie, i figli e se stesso. Forse avrebbe cercato di nascondere più accortamente le proprie colpe alla moglie, se avesse previsto che questa scoperta avrebbe agito tanto su di lei. A questo non aveva riflettuto mai con chiarezza; tuttavia, vagamente, si figurava che sua moglie, da tempo, indovinasse che egli non le era fedele e chiudesse un occhio. Gli sembrava inoltre che lei, donna esaurita, invecchiata, non più bella e per nulla affatto interessante, semplice, buona madre di famiglia soltanto, dovesse, per un senso di giustizia, essere indulgente. Era avvenuto il contrario. 

``Ah, è terribile! Ahi, ahi, ahi, ahi! Terribile! - si ripeteva Stepan Arkad'ic e non riusciva a trovare una via d'uscita. - E come andava tutto bene prima d'ora! Come vivevamo bene! Lei era contenta, felice dei bambini; io non l'ostacolavo in nulla, la lasciavo libera di regolarsi come voleva, coi bambini, con la casa. È vero, non è bello che quella sia stata governante in casa nostra! Non è bello! C'è qualcosa di triviale, di volgare nel far la corte alla propria governante. Ma che governante! - e ricordò con vivezza il riso e gli occhi neri assassini di m.lle Rolland. - Del resto finché è stata in casa nostra, io non mi sono permesso nulla. E il peggio di tutto è che già\ldots{} Ci voleva proprio tutto questo, neanche a farlo apposta! Ah, ahi, ahi! Ma che fare, che fare?'' 

Una risposta che non c'era all'infuori della risposta comune che dà la vita a tutte le più complicate e insolubili questioni, e la risposta è questa: bisogna vivere delle piccole necessità del giorno, smemorarsi. Nel sogno non è più possibile; almeno fino a stanotte, non si può tornare alla musica che cantavano le donne-caraffe; ci si deve dunque smemorare con il sonno della vita. 

``Staremo a vedere'' si disse Stepan Arkad'ic e, alzatosi, indossò la veste da camera grigia dalla fodera di seta azzurra, fermò i due lacci con un nodo, e introdotta aria a sazietà nella vasta cavità toracica, coll'usuale passo deciso dei suoi piedi all'infuori che così leggermente sostenevano il corpo pienotto, si avviò alla finestra, sollevò la tenda e sonò forte. Entrò subito il suo vecchio amico, Matvej il maggiordomo, che portava il vestito, le scarpe e un telegramma. Dietro a Matvej entrò anche il barbiere con l'occorrente per la barba. 

- Ci sono carte d'ufficio? - chiese Stepan Arkad'ic dopo aver preso il telegramma, sedendosi di fronte allo specchio. 

- Sulla tavola - rispose Matvej. Guardò interrogativamente, con interesse, il padrone, e, dopo aver atteso un poco, aggiunse con un sorriso ammaliziato: - Sono venuti da parte del signor cocchiere. 

Stepan Arkad'ic non rispose nulla e guardò soltanto Matvej nello specchio: nello sguardo che incrociarono era evidente come si intendessero l'un l'altro. Lo sguardo di Stepan Arkad'ic sembrava chiedere: ``Perché dici questo? che forse non sai?''. Matvej ficcò le mani nelle tasche del giubbetto, tirò indietro una gamba in silenzio, bonariamente, sorridendo appena, guardò il padrone. 

- Ho detto loro di venire la prossima domenica, e che fino allora non si disturbino e non disturbino voi inutilmente - disse con una frase evidentemente già preparata. 

Stepan Arkad'ic capì che Matvej voleva scherzare e attirare su di sé l'attenzione. Aperto il telegramma, lo lesse, correggendo per intuito le parole, come sempre alterate, e il viso gli si illuminò. 

- Matvej, mia sorella Anna Arkad'evna viene domani - disse, arrestando per un attimo la mano lustra e grassoccia del barbiere che andava tracciando una via rosea tra le lunghe fedine ricciute. 

- Sia lodato Iddio - disse Matvej, mostrando con la risposta di capire, allo stesso modo del padrone, il significato di questo arrivo, e che cioè Anna Arkad'evna, sorella carissima di Stepan Arkad'ic, poteva contribuire alla riconciliazione tra marito e moglie. 

- Sola o col consorte? - chiese Matvej. 

Stepan Arkad'ic, che non poteva parlare perché il barbiere era alle prese col labbro superiore, alzò un dito solo. Matvej fece cenno col capo nello specchio. 

- Sola. C'é da preparare di sopra? 

- Chiedilo a Dar'ja Aleksandrovna; dove dirà lei. 

- A Dar'ja Aleksandrovna? - ripeté con aria dubbiosa Matvej. 

- Sì, diglielo. Ecco, prendi il telegramma, riferiscimi poi. 

``Volete provare'' pensò Matvej, ma disse solo: 

- Sissignore. 

Stepan Arkad'ic era già lavato e pettinato e si preparava a vestirsi quando Matvej, camminando lentamente con le scarpe che scricchiolavano, rientrò nella stanza col telegramma in mano. Il barbiere era già andato via. 

- Dar'ja Aleksandrovna ha ordinato di dirvi che parte. Che faccia pure come piace a lui, cioè a voi - disse, ridendo solo con gli occhi e, cacciate le mani in tasca e chinato il capo da un lato, fissò il padrone. 

Stepan Arkad'ic tacque. Poi un sorriso buono e un po' pietoso apparve sul suo bel viso. 

- Eh, Matvej - disse, scotendo il capo. 

- Non è nulla, signore; tutto si appianerà - disse Matvej. 

- Si appianerà? 

- Proprio così. 

- Credi? Chi c'è di là? - chiese Stepan Arkad'ic sentendo dietro la porta un fruscio di abito femminile. 

- Sono io, signore - disse una voce di donna, e di dietro la porta si sporse il viso severo e butterato di Matrëna Filimonovna, la njanja. 

- E allora, Matrëna? - domandò Stepan Arkad'ic andandole incontro sulla porta. Sebbene Stepan Arkad'ic fosse per ogni verso colpevole di fronte alla moglie, ed egli stesso lo sentisse, quasi tutti in casa, persino la njanja, la più grande amica di Dar'ja Aleksandrovna, erano dalla parte sua. 

- E allora? - disse con aria afflitta. 

- Andate da lei, signore, dichiaratevi ancora colpevole. Forse Iddio lo concederà. Si tormenta molto ed è una pena guardarla, e poi tutto in casa va alla malora. Ci si deve preoccupare dei bambini, signore. Accusatevi, signore. Che fare? Fatto il male\ldots{} 

- Eh già, non mi riceverà\ldots{} 

- E voi fate il dover vostro. Dio è misericordioso, pregate Iddio, signore, pregate Iddio. 

- E va bene; va'\ldots{} - disse Stepan Arkad'ic, arrossendo improvvisamente. - Su vestiamoci - disse rivolto a Matvej, e con fare deciso si tolse la veste da camera. 

Matvej teneva in mano, soffiandovi sopra come a togliere qualcosa di invisibile, la camicia disposta a collare, e con evidente soddisfazione ne circondò il corpo ben curato del padrone. 

\capitolo{III}Vestitosi, Stepan Arkad'ic si spruzzò di profumo, assestò le maniche della camicia, distribuì per le tasche con gesti abituali le sigarette, il portafoglio, i fiammiferi, l'orologio con la catena doppia e i ciondoli e, spiegazzato il fazzoletto, sentendosi pulito, profumato, sano e, malgrado il suo guaio, fisicamente allegro, si avviò, tentennando leggermente su ciascuna gamba, verso la sala da pranzo dove già l'aspettavano il caffè e, accanto al caffè, le lettere e le carte del tribunale. 

Lesse le lettere. Una era molto spiacevole: era del compratore del bosco di sua moglie. Il bosco doveva essere necessariamente venduto; ma ora, fino alla riconciliazione, non se ne poteva parlare. Più increscioso di tutto era il fatto che si veniva in tal modo a frammischiare una questione di denaro al prossimo avvenimento della riconciliazione. E il pensiero ch'egli potesse lasciarsi guidare da una questione di denaro, che per la vendita del bosco cercasse di far pace con la moglie, questo pensiero l'offendeva. 

Letta la posta, Stepan Arkad'ic tirò a sé le carte d'ufficio: sfogliò in fretta due pratiche, segnò con un grosso lapis qualche annotazione e, allontanate le carte, cominciò a sorbire il caffè e nello stesso tempo, aperto il giornale della mattina, ancora umido, prese a leggerlo. 

Stepan Arkad'ic riceveva e leggeva un giornale liberale, non estremista, ma della tendenza che la maggioranza sosteneva. Benché non lo interessassero in modo particolare né scienza, né arte, né politica, egli si atteneva strettamente alle opinioni alle quali, in tutte queste materie, si attenevano la maggioranza e il suo giornale, e le cambiava soltanto quando le cambiava la maggioranza, o per meglio dire non lui le cambiava, ma esse stesse, inavvertitamente, si cambiavano in lui. 

Stepan Arkad'ic non sceglieva né le tendenze né le opinioni, ma queste stesse tendenze e opinioni giungevano a lui da sole, proprio allo stesso modo come non lui sceglieva la foggia del cappello o del soprabito, ma adottava quella che era di moda. E per lui, che viveva nella società più in vista, avere delle opinioni, oltre al bisogno di una certa attività di pensiero che normalmente si sviluppa negli anni della maturità, era così indispensabile come avere un cappello. E anche se c'era una ragione per preferire la tendenza liberale a quella conservatrice, cui si atteneva la maggioranza del suo ambiente, questa consisteva non solo nel fatto che egli trovava la tendenza liberale più ragionevole, ma anche perché questa era in realtà più conforme al suo modo di vivere. Il partito liberale diceva che in Russia tutto andava male, ed in effetti Stepan Arkad'ic aveva molti debiti e il denaro non gli bastava proprio. Il partito liberale diceva che il matrimonio era un'istituzione superata ed era necessario riformarlo, e in realtà la vita familiare dava scarse soddisfazioni a Stepan Arkad'ic e lo costringeva a mentire e a fingere, il che era affatto avverso alla sua natura. Il partito liberale diceva, o meglio faceva intendere, che la religione era soltanto un freno per la parte incolta della popolazione, e in realtà Stepan Arkad'ic non poteva sopportare, senza che gli dolessero le gambe, neppure il più piccolo Te Deum, e non poteva capire che senso avessero tutte quelle tremende altisonanti parole sull'altro mondo, quando anche in questo era così piacevole vivere. Inoltre a Stepan Arkad'ic, che amava gli scherzi ameni, faceva piacere turbare talvolta qualche pacifico essere col dire, che se ci si vuole inorgoglire della razza, non conviene fermarsi a Rjurik e rinnegare il progenitore, la scimmia. Dunque le opinioni liberali erano divenute un'abitudine per Stepan Arkad'ic e gli piaceva il suo giornale, così come il sigaro dopo il pranzo, per quella leggera nebbia che gli generava in testa. Lesse l'articolo di fondo, nel quale si spiegava che ``al tempo nostro del tutto invano si levan querele contro il radicalismo, il quale minaccia di inghiottire tutti gli elementi conservatori, e che il governo non si decide a prendere delle misure per soffocare l'idra rivoluzionaria; che al contrario, secondo la nostra opinione, il pericolo risiede non già nella presunta idra rivoluzionaria, ma nel tradizionalismo ostinato che rallenta il progresso'' e così di seguito. Lesse anche un altro articolo, finanziario, nel quale si parlava del Bentham e dello Stuart Mill e si lanciavano frecciate al ministero. Con la prontezza di spirito che gli era propria egli afferrava il senso di ogni frecciata: da chi veniva e contro chi era diretta e in quale occasione, e questo, come sempre, gli procurava un certo piacere. Ma oggi questo piacere era avvelenato dal ricordo dei consigli di Matrëna Filimonovna e dal fatto che in casa tutto andava tanto male. Lesse pure che il conte Beist, come correva voce, era partito per Wiesbaden, e che si vendeva una carrozza leggera, e che una persona giovane faceva una proposta; ma queste notizie non gli davano più il solito tranquillo, ironico compiacimento di una volta. 

Finito il giornale, la seconda tazza di caffè e la ciambellina al burro, s'alzò scrollando le briciole dal panciotto e, allargando il petto ampio, sorrise di piacere: non perché avesse in animo qualcosa di particolarmente lieto, ma solo perché la buona digestione gli procurava quel sorriso di gioia. 

Ma quel sorriso di gioia gli fece tornare subito tutto in mente ed egli si fece pensieroso. 

Due voci infantili (Stepan Arkad'ic riconobbe le voci di Griša, il più piccolo, e di Tanja, la maggiore) si udirono dietro la porta. Avevano trascinato e lasciato cadere qualcosa. 

- Lo dicevo io che non si possono lasciar sedere i passeggeri sull'imperiale - gridava in inglese la bimba - ora, su, raccatta. 

``È tutto sottosopra - pensò Stepan Arkad'ic - ecco, i bambini scorrazzano da soli''. E fattosi sulla porta, li chiamò. Essi lasciarono la scatola che rappresentava il treno ed entrarono dal padre. 

La bimba, beniamina del padre, corse franca ad abbracciarlo e ridendo gli si appese al collo, rallegrandosi come sempre del noto profumo che si spandeva dalle sue fedine. Baciatolo infine sul volto arrossato per la posizione inclinata e raggiante di tenerezza, la bimba sciolse le braccia per scappar via, ma il padre la trattenne. 

- E la mamma? - chiese passando la mano sul collo liscio e morbido della figlia. - Buongiorno - disse poi sorridendo al piccolo che salutava. 

Aveva coscienza di amare meno il bambino e si sforzava di essere imparziale, ma il bambino lo sentiva e non sorrise al sorriso freddo del padre. 

- La mamma? S'è alzata - rispose la bimba. 

Stepan Arkad'ic sospirò. ``Già; non avrà dormito tutta la notte'' pensò. 

- Ma è di buon umore? 

La bambina sapeva che fra padre e madre c'era stata una certa questione e che la madre non poteva essere di buon umore; e il padre doveva saperlo, mentre ora fingeva, chiedendone con tanta disinvoltura. Arrossì per il padre. Egli capì subito e arrossì anche lui. 

- Non so - disse. - Non ha detto di studiare, ha detto di andare a spasso con miss Hull dalla nonna. 

- Su, va', Tancurocka mia. Ah, già, aspetta - disse trattenendola ancora e guardandole la manina morbida. 

Prese dal camino, là dove l'aveva messa il giorno prima, una scatola di dolci e gliene diede due, scegliendole i preferiti, uno di cioccolato e uno fondente. 

- A Griša? - disse la bambina indicando quello di cioccolato. 

- Sì, sì. - E accarezzando ancora una volta le piccole spalle, la baciò alla radice dei capelli e sul collo e la lasciò andare. 

- La carrozza è pronta - disse Matvej. - C'è poi una persona che chiede di voi - aggiunse. 

- È molto che è qui? - chiese Stepan Arkad'ic. 

- Da una mezz'ora. 

- Ma quante volte ti ho detto di annunziare subito! 

- Bisogna pur darvi il tempo di prendere almeno il caffè - disse Matvej con quel tono fra il confidenziale e lo screanzato che non dava la possibilità di arrabbiarsi. 

- Su, fa' passare subito - disse Oblonskij aggrottando le sopracciglia dalla stizza. 

La signora, moglie del capitano in seconda Kalinin, chiedeva una cosa assurda e sciocca; ma Stepan Arkad'ic, secondo la sua abitudine, la fece sedere, l'ascoltò con attenzione, senza interromperla, le consigliò dettagliatamente a chi e come dovesse rivolgersi, e le scrisse perfino alla svelta e bene, con la sua grossa, larga e bella scrittura chiara, un biglietto per la persona che avrebbe potuto aiutarla. Congedata la moglie del capitano in seconda, Stepan Arkad'ic prese il cappello e si fermò, cercando di ricordare se non avesse dimenticato qualcosa. Gli parve di non aver dimenticato nulla, fuorché quello che voleva dimenticare, la moglie. 

``Ah, sì''. Abbassò il capo e il suo bel viso prese un'aria afflitta. ``Andare o non andare?'' si diceva. E una voce interna gli diceva di non andare, che oltre a falsità non poteva esserci altro, che riparare, accomodare le loro relazioni non era più possibile, perché non era possibile rendere lei di nuovo attraente e capace di suscitare l'amore, e lui vecchio e incapace di amare. Dunque, oltre a falsità e menzogna, non ne poteva uscir fuori nulla, e la falsità e la menzogna erano avverse alla sua natura. 

``Eppure prima o poi bisogna farlo; non si può restar così'' disse, cercando di farsi coraggio. Raddrizzò il petto, tirò fuori una sigaretta, l'accese, ne aspirò due boccate, la gettò in un portacenere di madreperla a conchiglia, attraversò il salotto oscuro a passi svelti, e aprì l'altra porta che dava nella camera della moglie. 

\capitolo{IV}Dar'ja Aleksandrovna, in veste da notte, con le trecce ormai rade, un tempo folte e belle, appuntate alla nuca, col viso asciutto, affilato, e i grandi occhi spauriti che risaltavano nella magrezza del viso, stava in piedi in mezzo alle cose gettate alla rinfusa per la stanza, dinanzi a un armadio aperto dal quale sceglieva qualcosa. Udito il passo del marito, si fermò guardando la porta e cercando inutilmente di dare al viso un'espressione severa e sprezzante. Sentiva di aver paura di lui, paura dell'incontro imminente. Aveva tentato proprio allora di fare quello che aveva tentato già dieci volte in quei tre giorni: preparare la roba sua e dei bambini per trasportarla dalla madre, ma poi, di nuovo, non aveva saputo decidersi: eppure anche ora, come le altre volte, diceva a se stessa che così non poteva durare, che doveva fare qualcosa, punirlo, svergognarlo, vendicarsi almeno in minima parte del male che le aveva fatto. Si diceva ogni volta che lo avrebbe lasciato, ma sentiva che questo era impossibile; era impossibile perché non poteva disabituarsi a considerarlo suo marito e ad amarlo. Sentiva, inoltre, che se qui, in casa sua, riusciva appena ad aver cura dei suoi cinque bambini, la cosa sarebbe stata ancora più difficile là, dove sarebbe andata a stare con tutti loro. E proprio in quei tre giorni, il più piccolo si era ammalato perché gli avevano dato del brodo guasto, mentre il giorno innanzi gli altri erano quasi rimasti senza mangiare. Sentiva che non era possibile andar via; ma, ingannando se stessa, preparava la roba e si fingeva di partire. 

Visto il marito, tuffò la mano in un cassetto dell'armadio, come se cercasse qualcosa, e girò lo sguardo su di lui solo quando le fu proprio accanto. Ma il viso al quale aveva voluto dare un'espressione severa e decisa, esprimeva smarrimento e pena. 

- Dolly! - disse lui con voce timida e sommessa. Aveva ritirato la testa nelle spalle e voleva avere un'aria afflitta e contrita, ma suo malgrado, raggiava freschezza e salute. 

Con un'occhiata rapida dalla testa ai piedi ella notò la figura di lui raggiante freschezza e salute. ``Già, lui è felice e soddisfatto - pensò - e io? E anche questa bontà disgustosa, che lo fa amare e lodare da tutti, io la detesto questa sua bontà'' pensò. La bocca le si contrasse, il muscolo della guancia prese a tremare dalla parte destra del viso pallido e nervoso. 

- Che vi occorre? - disse con voce affrettata, sorda, non sua. 

- Dolly! - ripeté lui con un fremito nella voce. - Anna arriva oggi. 

- Ebbene, a me che importa? Io non posso riceverla! - gridò lei. 

- Eppure, Dolly\ldots{} 

- Andate via, andate via - gridò senza guardarlo, come se questo grido fosse provocato da un male fisico. 

Stepan Arkad'ic aveva potuto rimaner tranquillo quando aveva pensato a sua moglie, aveva potuto sperare che tutto si sarebbe ``appianato'', così come diceva Matvej, aveva potuto leggere tranquillamente il giornale e bere il caffè; ma quando vide il viso tormentato e dolente di lei, quando udì quel tono di voce rassegnato e affranto, il respiro gli si mozzò, qualcosa gli venne alla gola e gli occhi gli brillarono di lacrime. 

- Dio mio, che ho fatto! Dolly! Per amor di Dio\ldots{} Del resto\ldots{} - ma non poté continuare: un singhiozzo gli si era fermato in gola. Ella sbatté l'armadio e si voltò a guardarlo. - Dolly, cosa posso dire? Solo una cosa: perdona, perdona\ldots{} Ricorda\ldots{} nove anni di vita non possono forse far perdonare un minuto, un minuto\ldots{} 

Ella aveva abbassato gli occhi e ascoltava quello ch'egli stava per pronunciare, quasi supplicandolo di dire qualcosa che potesse dissuaderla. 

- Un minuto di esaltazione - riprese a dire lui, e voleva continuare, ma a questa parola, come per un male fisico, a lei si strinsero i denti e di nuovo il muscolo della guancia prese a tremare dalla parte destra del viso. 

- Andate via, andate via! - gridò con voce ancora più tagliente - e non mi venite a parlare delle vostre esaltazioni e delle vostre sconcezze! 

Voleva andar via, ma vacillò e si aggrappò alla spalliera della sedia per sorreggersi. Il viso di lui si dilatò, le labbra si gonfiarono, gli occhi si riempirono di lacrime. 

- Dolly! - pronunziò ormai singhiozzando. - In nome di Dio, pensa ai bambini, loro non sono colpevoli. Sono io il colpevole, e tu puniscimi, ordinami di scontare la mia pena. In quello che posso, sono pronto a tutto! Sono colpevole, non ci sono parole, come sono colpevole! Ma, Dolly, perdona! 

Ella si mise a sedere. Egli sentiva il respiro grave di lei e gliene veniva una pena indicibile. Più volte ella si provò a parlare, ma non poté. Egli aspettava. 

- Tu ti ricordi dei bambini per giocare con loro, mentre io sì che me ne ricordo, e lo so oramai che sono rovinati - disse lei, usando evidentemente una delle frasi che in quei tre giorni s'era ripetuta più d'una volta. 

Gli aveva parlato col ``tu'', ed egli la guardò riconoscente, e si mosse per prenderle una mano, ma lei si scostò con avversione. 

- Io mi ricordo dei bambini e farei di tutto al mondo per salvarli, ma non so io stessa come salvarli: se sottrarli al padre o abbandonarli a un padre depravato. Sì, depravato\ldots{} Eh sì, ditemi voi, dopo quello\ldots{} che c'è stato, è forse possibile vivere insieme? È possibile forse? Dite voi, è possibile? - ripeté alzando la voce. 

- Dopo che mio marito, il padre dei miei figli ha una relazione con la governante dei suoi bambini\ldots{} 

- Ma che fare, che fare? - diceva lui con voce pietosa, non sapendo egli stesso che dire e abbassando sempre più il capo. 

- Mi fate ribrezzo, disgusto! - gridò lei, riscaldandosi ancora di più. - Le vostre lacrime cosa sono? acqua! Non mi avete mai amata, non avete cuore, non siete generoso! Siete vile, abietto, mi siete estraneo, sì, del tutto estraneo - e pronunziò la parola ``estraneo'', per lei terribile, con pena e rancore. 

Egli la guardò e l'odio che appariva sul viso di lei lo sgomentò e sorprese. Non capiva che quella sua pietà verso di lei la irritava, perché vedeva in lui la compassione, ma non l'amore. ``Mi odia - pensò. - Non perdonerà''. 

- È terribile, è terribile - disse. 

Nel frattempo, nella stanza accanto, probabilmente perché caduto, un bimbo si mise a gridare: Dar'ja Aleksandrovna tese l'orecchio, e il viso d'un tratto le si raddolcì. 

Parve rientrare in sé per qualche istante e, come se non sapesse dov'era e cosa stesse facendo, si alzò in fretta e si avviò alla porta. 

``Ma allora vuol sempre bene al mio bambino - pensò lui, avendo notato il mutar del viso al grido del piccolo - al mio bambino; e come può odiare tanto me?''. 

- Dolly, ancora una parola - disse seguendola. 

- Se mi seguite, chiamerò gente, i bambini! Che tutti sappiano che siete un mascalzone! Me ne vado oggi stesso e voi restate pure qua a vivere con la vostra amante! 

E uscì, sbattendo la porta. 

Stepan Arkad'ic sospirò, si asciugò il viso e a passi lenti si avviò per uscire. ``Matvej dice che si appianerà; ma come? Io non ne vedo neppure la possibilità. Ahi, ahi, che orrore! E come gridava, e in che modo triviale! - diceva a se stesso ricordando le grida e le parole `mascalzone' e `amante'. - E forse le ragazze hanno sentito! Terribilmente triviale, terribilmente''. Stepan Arkad'ic si fermò per qualche istante, si asciugò gli occhi, sospirò e, raddrizzato il busto, uscì dalla camera. 

Era venerdì, e nella sala da pranzo l'orologiaio tedesco dava corda all'orologio. Stepan Arkad'ic si ricordò della sua battuta di spirito su quell'orologiaio calvo e preciso: ``Il tedesco è stato caricato per tutta la vita per caricare orologi'' e sorrise. A Stepan Arkad'ic piaceva una bella battuta. ``Ma forse davvero tutto `si appianerà'! Bella frase: `si appianerà' - pensò. - Bisogna farla circolare''. 

- Matvej! - chiamò. - Prepara tutto con Mar'ja per Anna Arkad'evna, di là nel salotto - disse a Matvej che era apparso. 

- Sissignore. 

Stepan Arkad'ic infilò la pelliccia e uscì fuori. 

- Non pranzerete a casa? - chiese Matvej, accompagnandolo. 

- Non so, come capiterà. Ecco, prendi per la spesa - disse dandogli dieci rubli dal portafoglio. - Basta? 

- Basti o non basti, ci si deve rigirare - rispose Matvej, sbattendo lo sportello e indietreggiando verso l'ingresso. 

Dar'ja Aleksandrovna intanto, acquietato il bambino e capito, dal rumore della carrozza, ch'egli se n'era andato, tornò di nuovo in camera. Era l'unico suo rifugio dalle cure familiari che la opprimevano non appena ne usciva fuori. E anche ora, in quei pochi momenti che aveva passato nella camera dei bambini, la governante inglese e Matrëna Filimonovna si erano affrettate a farle alcune domande che non ammettevano indugio e alle quali solo lei poteva rispondere: cosa mettere indosso ai bambini per andare a spasso, dare o no il latte, mandare a chiamare oppure no un altro cuoco. 

- Ah, lasciatemi, lasciatemi! - aveva detto e, tornata in camera, si era seduta di nuovo là dove aveva parlato col marito, stringendo le mani smagrite con gli anelli che scivolavano dalle dita ossute, e aveva cominciato a ripensare a tutto il colloquio avvenuto. ``È andato via. Ma l'ha finita poi con quella? Possibile che la veda ancora? Perché non gliel'ho chiesto? No, no, non ci si può riunire. E anche se dovessimo restare nella stessa casa, saremmo estranei. Per sempre estranei! - ripeté di nuovo, e con particolare significato, questa parola per lei terribile. - E come l'ho amato, Dio mio, come l'ho amato! E ora, non l'amo forse? Non l'amo forse più di prima? È terribile, soprattutto il fatto che\ldots{}'' cominciò, ma non finì il pensiero, che già Matrëna Filimonovna si era affacciata alla porta. 

- Su via, mandate a chiamare mio fratello - disse - almeno preparerà il pranzo; se no, come ieri, fino alle sei i bambini non avran mangiato. 

- Va bene, vengo, vengo a dare gli ordini. Non hanno mandato a prendere il latte fresco? 

E Dar'ja Aleksandrovna s'ingolfò nelle cure del giorno, e per un po' sommerse in esse la sua pena. 

\capitolo{V}Stepan Arkad'ic a scuola aveva studiato bene, grazie alle sue buone capacità, ma, pigro e svagato, aveva finito gli studi tra gli ultimi. Tuttavia, pur conducendo una vita sempre scapestrata, in età ancor giovane, con un titolo modesto, aveva ottenuto il posto ragguardevole e ben retribuito di capo di uno degli uffici amministrativi di Mosca. Aveva avuto questo posto per mezzo del marito di Anna, Aleksej Aleksandrovic Karenin, il quale occupava uno dei più alti gradi nel ministero a cui apparteneva l'ufficio; ma se Karenin non avesse designato suo cognato a quel posto, Stiva Oblonskij, per mezzo di un centinaio di alti personaggi, fratelli, sorelle, prozii, zii, zie, avrebbe avuto quel posto o altro equivalente con quei seimila rubli di stipendio che gli erano necessari, perché i suoi affari, malgrado la considerevole proprietà della moglie, andavano male. 

Una buona metà della società di Mosca e Pietroburgo era in relazioni di parentela o di amicizia con Stepan Arkad'ic. Egli era nato nella cerchia di coloro che erano o erano in seguito diventati i potenti di quel mondo. Un terzo dei funzionari di stato, i vecchi, erano amici di suo padre e lo avevano visto nascere; un altro terzo gli davano del ``tu'' e un terzo ancora erano suoi buoni conoscenti. Pertanto, i dispensatori di beni terreni sotto forma di posti, appalti, concessioni e cose simili, erano tutti amici suoi e non avrebbero mai lasciato fuori uno dei loro. Così Oblonskij non aveva dovuto brigare per ottenere un posto vantaggioso; gli era bastato non rifiutare, non avere invidie, non leticare, non offendersi, cose tutte ch'egli neppure faceva per quella bonarietà che gli era propria. Gli sarebbe parso ridicolo se gli avessero detto che non avrebbe ottenuto un posto retribuito con lo stipendio che gli era necessario, dal momento che non pretendeva niente di eccezionale, ma voleva solo quello che avevano gli altri suoi coetanei quando, non peggio di chiunque altro, egli era in grado di adempiere una funzione di tal genere. 

A Stepan Arkad'ic volevano bene tutti quelli che lo conoscevano non solo per quel suo carattere buono e gioviale e per la sua indubbia onestà, ma perché in quel suo bell'aspetto luminoso, negli occhi splendenti, nelle sopracciglia e nei capelli neri, nel colorito bianco e rosso del viso vi era qualcosa che agiva in modo cordiale e festoso sul fisico delle persone che lo incontravano. ``Oh, Stiva! Oblonskij! Eccolo!'' dicevano quasi sempre con un sorriso di gioia, incontrandolo. E anche se talvolta ci si rendeva conto che, dopo una conversazione con lui, non succedeva nulla di particolarmente gioioso, l'indomani, due giorni dopo, tutti di nuovo si rallegravano nell'incontrarlo, proprio allo stesso modo. 

Occupando già da tre anni il posto di capo di uno degli uffici amministrativi di Mosca, Stepan Arkad'ic aveva conquistato, oltre la simpatia, la stima dei colleghi, dei dipendenti, dei superiori, e di tutti coloro che avevano a che fare con lui. Le principali qualità che gli procuravano la stima generale in ufficio consistevano, in primo luogo, in una straordinaria indulgenza verso gli altri, basata sulla coscienza dei propri difetti; in secondo luogo, in un grande liberalismo, non quello di cui leggeva nei giornali, ma quello ch'egli aveva nel sangue e che gli faceva trattare perfettamente allo stesso modo tutte le persone, di qualunque classe o condizione fossero; e in terzo luogo, e questa era la cosa più importante, in un'assoluta indifferenza verso gli affari che trattava, per cui non se ne appassionava mai e non commetteva errori. 

Arrivato in ufficio, Stepan Arkad'ic, accompagnato da un usciere ossequioso che gli portava la cartella, passò nel suo gabinetto particolare, indossò la divisa ed entrò in aula. Gli scrivani e gli impiegati si alzarono tutti, salutandolo con rispetto e giovialità. Stepan Arkad'ic, in fretta come sempre, andò al proprio posto, strinse la mano ai colleghi e sedette. Scherzò e discorse proprio quel tanto che era conveniente, e cominciò il lavoro. Nessuno più di Stepan Arkad'ic sapeva con maggiore precisione il limite tra la cordialità confidenziale e il tono ufficiale, così necessario al piacevole disbrigo degli affari. Il segretario, con giovialità e rispetto, come del resto tutti nell'ufficio di Stepan Arkad'ic, gli si accostò con alcune carte e riferì con quel tono familiarmente libero che era stato introdotto da Stepan Arkad'ic. 

- Siamo riusciti così ad avere notizie dell'amministrazione provinciale di Penza. Ecco, non vi piacerebbe\ldots{} 

- Le avete avute finalmente - prese a dire Stepan Arkad'ic, fermando col dito la carta. - Allora, signori\ldots{} - E la seduta cominciò. 

``Se sapessero - pensava chinando la testa con aria d'importanza nell'ascoltare il rapporto - che ragazzo colpevole era mezz'ora fa il loro capo!''. E gli occhi gli ridevano alla lettura del rapporto. La seduta doveva durare fino alle due, senza interruzione; alle due, intervallo e colazione. 

Non erano ancora le due quando la grande porta a vetri dell'aula si aprì improvvisamente e qualcuno entrò. Tutti i membri ritratti sotto il ritratto dell'imperatore e al di là dello specchio a tre facce, lieti della distrazione, si voltarono a guardare verso la porta; ma l'usciere che stava all'ingresso respinse subito colui che s'era infilato e richiuse la porta a vetri. 

Quando tutto il rapporto fu letto, Stepan Arkad'ic si alzò stiracchiandosi e, pagando il proprio tributo al liberalismo dell'epoca, tirò fuori, ancora nell'aula, una sigaretta, e si avviò nel suo ufficio. Due colleghi, il vecchio funzionario Nikitin e il gentiluomo di camera Grinevic, uscirono con lui. 

- Dopo colazione arriveremo a finire - disse Stepan Arkad'ic. 

- Altro che arriveremo! - disse Nikitin. 

- Ma deve essere un furbo matricolato quel Fomin - disse Grinevic accennando a un personaggio implicato nell'affare di cui si discuteva. 

Alle parole di Grinevic Stepan Arkad'ic si accigliò, facendo capire con questo che non era corretto dare un giudizio prima del tempo, e non rispose nulla. 

- Chi è entrato? - chiese all'usciere. 

- Un tale, eccellenza, senza chiedere permesso, s'è fissato dentro appena mi sono girato. Domandava di voi. Io dico: quando usciranno i membri, allora\ldots{} 

- Dov'è? 

- È forse uscito nell'ingresso, non faceva che camminare. Eccolo - disse l'usciere, indicando un uomo di costituzione forte, largo di spalle, con la barba ricciuta, il quale, senza togliersi il berretto di montone, saliva lesto e leggero i gradini consumati della scala di pietra. Uno di quelli che scendevano, un impiegato magrolino con una cartella sotto il braccio, fermatosi, guardò con riprovazione le gambe di colui che correva e fissò interrogativamente Oblonskij. 

Stepan Arkad'ic era dritto in cima alla scala. Il suo viso bonario, che splendeva emergendo dal bavero ricamato dell'uniforme, s'illuminò ancor più quando riconobbe chi correva. 

- Ma è proprio lui! Levin, finalmente! - esclamò con un sorriso cordialmente canzonatorio, guardando Levin che gli si avvicinava. - Com'è che non hai disdegnato di venirmi a pescare in quest'antro? - disse Stepan Arkad'ic baciando l'amico, non contento di una stretta di mano. - Sei qui da un pezzo? 

- Sono arrivato or ora, e avevo una gran voglia di vederti - rispose Levin, guardandosi attorno timido e, nello stesso tempo, inquieto e contrariato. 

- Su, andiamo nel mio gabinetto - disse Stepan Arkad'ic, conoscendo la timidezza ombrosa e scontrosa dell'amico; e, presolo per un braccio, lo trascinò dietro di sé come per guidarlo in mezzo ai pericoli. 

Stepan Arkad'ic si dava del ``tu'' con quasi tutti i suoi conoscenti: coi vecchi di sessant'anni, coi ragazzi di venti; con gli attori, coi ministri, coi negozianti e con gli aiutanti generali; così che molti di quelli che gli davano del ``tu'' si trovavano ai due punti estremi della scala sociale, e molti si sarebbero stupiti nel constatare di avere qualcosa di comune per mezzo di Oblonskij. Egli dava del ``tu'' a tutti quelli con i quali beveva lo champagne, e di champagne ne beveva con tutti; perciò, incontrandosi in presenza dei suoi dipendenti con i suoi ``tu'' vergognosi, come chiamava scherzando molti amici, sapeva diminuire, con quel tatto che gli era proprio, la spiacevolezza dell'impressione che potevano riportarne i dipendenti. Levin non era un ``tu'' vergognoso, ma Oblonskij intuì che Levin pensava ch'egli potesse non desiderare di mostrare la propria intimità con lui dinanzi ai propri dipendenti, e perciò si affrettò a condurlo nel proprio gabinetto. 

Levin era quasi della stessa età di Oblonskij e si davano del ``tu'' non solo per lo champagne. Levin gli era compagno e amico di prima giovinezza. Si volevano bene, malgrado la diversità dei caratteri e dei gusti, così come si vogliono bene gli amici incontratisi nella prima giovinezza. Malgrado ciò, come capita spesso fra persone che hanno scelto generi diversi di attività, ciascuno di loro, pur giustificando col ragionamento l'attività dell'altro, finiva col disprezzarla dentro di sé. A ciascuno sembrava che la vita che egli stesso conduceva fosse la vera vita, mentre l'altra, quella che conduceva l'amico, non ne fosse che la parvenza. Oblonskij non poteva trattenere un lieve riso canzonatorio alla vista di Levin. L'aveva visto già varie volte arrivare a Mosca dalla campagna dove faceva qualcosa; che cosa facesse precisamente, Stepan Arkad'ic non aveva mai potuto capir bene e non se ne curava. Levin veniva a Mosca sempre agitato, frettoloso, un po' timido e urtato da questa timidezza, e quasi sempre con delle vedute nuove e inaspettate su tutte le cose. Stepan Arkad'ic ne rideva e se ne compiaceva. Nello stesso preciso modo Levin disprezzava dentro di sé il modo di vivere cittadino dell'amico e quel suo impiego che considerava sciocco e vuoto, e ci rideva su. Ma la differenza consisteva in questo: Oblonskij, facendo quello che fanno tutti, rideva con sicurezza e bonarietà, Levin, invece, senza sicurezza e, a volte, con dispetto. 

- Ti aspettavamo da tempo - disse Stepan Arkad'ic entrando nello studio e lasciando il braccio di Levin come a dire che là non c'erano più pericoli. - Sono molto contento di rivederti. Ebbene, come va? Quando sei arrivato? 

Levin taceva, sbirciando le due facce dei colleghi di Oblonskij che non conosceva, e in particolar modo dell'elegante Grinevic dalle dita affilate e bianche, e dalle unghie così lunghe, gialle e ricurve in punta, e dai gemelli della camicia così grossi e luccicanti che queste mani, evidentemente, avevano assorbito tutta la sua attenzione e non gli davano libertà di pensiero. Oblonskij lo notò subito, e sorrise. 

- Ah, già, permettete che vi presenti - disse. - I miei colleghi Filipp Ivanovic Nikitin e Michail Stanislavic Grinevic - e, rivolto verso Levin: - Il fautore del consiglio distrettuale, l'uomo nuovo del consiglio, il ginnasta che solleva con una mano sola cinque pudy, l'allevatore di bestiame, il cacciatore, nonché amico mio, Konstantin Levin, fratello di Sergej Ivanyc Koznyšev. 

- Molto piacere - disse il vecchietto. 

- Ho l'onore di conoscere vostro fratello Sergej Ivanyc - disse Grinevic porgendo la mano affilata dalle unghie lunghe. 

Levin si accigliò, strinse la mano e si rivolse subito a Oblonskij. Pur avendo una grande considerazione per il fratellastro, scrittore noto in tutta la Russia, non poteva sopportare che ci si rivolgesse a lui, non come Konstantin Levin ma come al fratello del famoso Koznyšev. 

- No, non sono più consigliere distrettuale. Ho litigato con tutti, e non vado più alle riunioni - disse a Oblonskij. 

- Hai fatto presto, però!- disse Oblonskij con un sorriso. - Ma come, perché? 

- È una storia lunga. Una volta o l'altra te la racconterò - disse Levin prendendo però subito a raccontarla. - Ecco, per dirla in breve, mi sono convinto che non c'è e non può esserci alcuna attività distrettuale; - cominciò come se qualcuno l'avesse offeso allora allora: - da una parte è un giuoco; si giuoca al parlamento, ed io non sono abbastanza giovane, né abbastanza vecchio per divertirmi coi balocchi; dall'altra - e qui balbettò - è un mezzo per guadagnare denaro per la coterie del distretto. Prima c'erano le tutele, i tribunali, ora invece c'è il consiglio distrettuale; non è una forma di subordinazione, ma una forma di stipendio non meritato - disse con tanto calore come se qualcuno dei presenti avversasse la sua opinione. 

- Eh! Ma tu, a quanto vedo, sei ancora in una nuova fase, in quella conservatrice - disse Stepan Arkad'ic. - Ma, del resto, di questo parleremo dopo. 

- Sì, dopo. Ma io avevo bisogno di vederti - disse Levin, fissando con antipatia la mano di Grinevic. 

Stepan Arkad'ic sorrise appena percettibilmente. 

- Be', dicevi che mai più ti saresti messo un vestito all'europea? - disse guardandogli il vestito nuovo, fatto evidentemente da un sarto francese. - Eh, già, vedo, siamo in una fase nuova. 

Levin arrossì improvvisamente, ma non come arrossiscono le persone adulte, leggermente, senza avvertirlo, ma come arrossiscono i ragazzi quando sentono d'essere ridicoli con la loro timidezza e, vergognandosene, arrossiscono ancora di più fin quasi alle lacrime. Ed era così strano vedere quel viso intelligente, maschio diventare così infantile, che Oblonskij smise di guardarlo. 

- E allora, dove ci vediamo? Ho molto bisogno di parlarti - disse Levin. 

Oblonskij si mise a riflettere. 

- Ecco, andiamo a far colazione da Gurin e parleremo là. Fino alle tre son libero. 

- No - rispose Levin dopo aver pensato un po'; - devo ancora andare in giro. 

- Su via, andiamo a pranzare insieme. 

- Pranzare? Ma io non ho bisogno di niente di straordinario, solo due parole; devo farti una domanda, e a chiacchierare ci penseremo poi. 

- E allora, dille subito queste due parole, così a pranzo chiacchiereremo. 

- Eccole, le due parole; - disse Levin - del resto, niente di straordinario. 

E la sua faccia prese a un tratto un'espressione cattiva, dovuta allo sforzo fatto per vincere la propria timidezza. 

- Che fanno gli Šcerbackij? Tutto come prima? - disse. 

- Tu hai detto due parole, ma io non posso rispondere con due parole, perché\ldots{} Scusami un momento\ldots{} 

Era entrato il segretario che, con la deferenza familiare e la modesta consapevolezza, comune a tutti i segretari, della propria superiorità, rispetto al capo, nella conoscenza degli affari, si era avvicinato con le carte a Oblonskij e, con aria interrogativa, aveva cominciato a esporre una certa difficoltà. Stepan Arkad'ic, senza finir di ascoltare, pose affabilmente una mano sulla manica del segretario. 

- No, fate come già vi ho detto - disse addolcendo con un sorriso l'osservazione e, spiegato in breve come intendeva l'affare, allontanò le carte e disse: - Fate così, vi prego, così, Zachar Nikitic. 

Il segretario, confuso, si allontanò. Levin, che durante il colloquio con il segretario aveva avuto modo di rimettersi completamente, stava in piedi, poggiando tutte e due le mani ad una sedia, e sul suo viso vi era un'attenzione ilare. 

- Non capisco, non capisco - diceva. 

- Cosa non capisci? - chiese Oblonskij sorridendo anche lui allegramente e tirando fuori una sigaretta. Si aspettava da Levin qualche uscita strana. 

- Non capisco quello che fate - disse Levin alzando le spalle. - Come puoi prendere tutto questo sul serio? 

- Perché? 

- Ma perché qui non avete nulla da fare. 

- Tu credi così, ma noi siamo sovraccarichi di lavoro. 

- Scartoffie! Già, ma tu ci sei tagliato per questo - aggiunse Levin. 

- Allora tu credi che io manchi di qualcosa? 

- Forse sì - disse Levin. - Tuttavia ammiro la tua importanza e sono orgoglioso di avere un così grand'uomo per amico. Ma tu non hai risposto alla mia domanda - aggiunse guardando Oblonskij con uno sforzo disperato, diritto negli occhi. 

- E va bene, e va bene. Aspetta un po' e arriverai a questo anche tu. Finché hai tremila desjatiny nel distretto di Karazin e questi muscoli e la freschezza di una ragazzina di dodici anni, va tutto bene, ma poi ci arriverai anche tu. Già, ecco, a proposito di quello che mi chiedevi; nessun cambiamento, ma peccato che tu sia stato lontano tanto tempo. 

- Perché? che c'è? - chiese Levin spaventato. 

- No, nulla - rispose Oblonskij. - Ne riparleremo. Ma tu per quale particolare motivo sei venuto? 

- Ah, di questo parleremo poi - disse Levin, arrossendo di nuovo fino alle orecchie. 

- Su, va bene, ho capito - disse Stepan Arkad'ic. - Vedi: ti avrei invitato a casa, ma mia moglie non sta bene. Ecco, però; se le vuoi vedere, oggi sono certamente al giardino zoologico, dalle quattro alle cinque. Kitty va a pattinare. Tu va' là; io passerò, e andremo a pranzare insieme in qualche posto. 

- Benissimo, arrivederci, allora. 

- Guarda, io ti conosco; tu sei capace di scordartene o di partirtene subito per la campagna! - rise Stepan Arkad'ic. 

- No, certamente. 

E dopo essersi ricordato di non aver salutato i colleghi di Oblonskij soltanto quand'era già sulla porta, Levin uscì dall'ufficio. 

- Deve essere un proprietario pieno di energia - disse Grinevic, quando Levin fu uscito. 

- Sì, amico mio - disse Stepan Arkad'ic annuendo col capo - ecco un uomo felice! Tremila desjatiny nel distretto di Karazin, tutto davanti a sé e quanta vitalità! Non così noi! 

- Perché vi lamentate, Stepan Arkad'ic? 

- Va male, proprio male - disse Stepan Arkad'ic sospirando pesantemente. 

\capitolo{VI}Quando Oblonskij aveva chiesto a Levin per quale motivo particolare fosse venuto, Levin s'era fatto rosso e s'era irritato con se stesso d'essersi fatto rosso, perché non gli aveva saputo rispondere: ``Son venuto a chiedere la mano di tua cognata'' pur essendo venuto proprio per questo. 

Le famiglie dei Levin e degli Šcerbackij erano vecchie casate di nobili moscoviti ed erano sempre state fra loro in rapporti di intima amicizia. Questi rapporti si erano fatti più stretti durante lo studentato di Levin. Levin si era presentato ed era entrato all'università insieme al giovane principe Šcerbackij, fratello di Dolly e di Kitty. In quel tempo Levin andava spesso in casa Šcerbackij ed era innamorato di casa Scerbackij. Per quanto ciò possa sembrare strano, Konstantin Levin era proprio innamorato della casa, della famiglia, in particolar modo della parte femminile degli Šcerbackij. Levin non ricordava sua madre, e l'unica sua sorella era più grande di lui, di modo che per la prima volta in casa Šcerbackij aveva conosciuto quell'ambiente di vecchia famiglia nobile, colta e onesta, del quale era stato privato per la morte della madre e del padre. Tutti i membri di questa famiglia, ed in particolare la parte femminile, gli apparivano avvolti in un certo misterioso velo di poesia; ed egli non solo non vedeva in loro alcun difetto, ma sotto questo poetico velo che li avvolgeva, immaginava i sentimenti più elevati e ogni possibile perfezione. Per qual motivo le tre signorine dovessero parlare un giorno in francese e un giorno in inglese; per qual motivo, in determinate ore, sonassero alternativamente il pianoforte i cui suoni giungevano su in camera del fratello dove gli amici studiavano; perché venissero insegnanti di letteratura francese, di musica, di disegno e di ballo; per qual motivo, a una data ora, tutte e tre le signorine con m.lle Linon giungessero in carrozza al boulevard Tverskoj avvolte nelle pelliccette rasate: Dolly in una lunga, Natalie in una meno lunga e Kitty in una del tutto corta, così che apparissero le sue gambette ben fatte nelle calze rosse attillate; per qual motivo dovessero passeggiare sul boulevard Tverskoj, accompagnate da un servitore con la coccarda dorata sul cappello; tutto questo e molto altro ancora di quel che si faceva nel loro mondo misterioso, egli non riusciva a capire; sapeva però che tutto quello che si faceva là era bello, ed era innamorato della misteriosità di quello che vi si compiva. 

Durante il suo studentato, era stato lì lì per innamorarsi della maggiore, Dolly; ma ben presto l'avevano data in sposa a Oblonskij. Aveva preso ad innamorarsi della seconda. Sentiva che avrebbe dovuto innamorarsi di una delle sorelle, ma non sapeva di quale precisamente. Ma anche Natalie, appena apparsa in società, andò sposa al diplomatico L'vov. Kitty era ancora ragazzina quando Levin finì l'università. Il giovane Šcerbackij, entrato in marina, morì nel mar Baltico e i rapporti di Levin con gli Šcerbackij, malgrado la sua amicizia con Oblonskij, divennero più radi. Ma quando, al principio dell'inverno, Levin giunse a Mosca dopo un anno di campagna e rivide gli Šcerbackij, capì di quale delle tre sorelle la sorte aveva destinato che egli si innamorasse. Nulla di più semplice doveva sembrare che lui, giovane di buona famiglia, benestante, trentaduenne, chiedesse la mano della principessina Šcerbackaja; con tutta probabilità sarebbe stato subito giudicato un buon partito. Ma Levin era innamorato, e gli sembrava che Kitty fosse, sotto ogni aspetto, una tale perfezione, un essere così superiore ad ogni altro sulla terra, e lui invece così umile e basso, da non poter neppure formulare il pensiero che gli altri ed ella stessa lo giudicassero degno di lei. 

Dopo aver passato due mesi a Mosca, come avvolto in una nebbia, vedendo Kitty ogni giorno in società dove aveva preso ad andare per incontrarla, Levin aveva improvvisamente deciso che la cosa non era possibile, ed era ripartito per la campagna. 

La convinzione di Levin che la cosa non andasse si basava sull'idea che agli occhi dei familiari egli dovesse sembrare un partito poco convincente, non degno della deliziosa Kitty, e che la stessa Kitty non potesse amarlo. Agli occhi dei parenti egli non aveva nessuna attività stabile e definita e nessuna posizione in società; a trentadue anni, alla sua stessa età, i suoi coetanei erano già chi colonnello e aiutante di campo, chi professore di università, chi direttore di banca o delle ferrovie, chi capufficio come Oblonskij; e lui invece (lo sapeva bene come appariva agli altri) era un proprietario di terre, che si occupava dell'allevamento delle vacche, del tiro alle beccacce e di costruzioni; era cioè un giovane senza talento, dal quale non era uscito fuori nulla, e che faceva, secondo il giudizio della gente di mondo, proprio quello che fanno gli uomini che non sono buoni a nulla. 

La stessa misteriosa e deliziosa Kitty non poteva amare un uomo così brutto, come egli stesso si considerava, e, quel ch'era peggio, così semplice, che non brillava in nulla. Oltre a ciò i suoi primi rapporti con Kitty, rapporti di un giovane verso una bambina sorti per l'amicizia col fratello, gli sembravano un altro ostacolo all'amore. A un brav'uomo brutto, come si considerava lui, si poteva voler bene come a un amico, ma per innamorarsene, com'era innamorato lui di Kitty, avrebbe dovuto essere un bell'uomo, e soprattutto un uomo interessante. 

Aveva sentito dire che spesso le donne amano uomini brutti e rudi; ma non ci credeva, perché giudicava da se stesso, che non avrebbe potuto amare se non donne belle, affascinanti, eccezionali. 

Ma, trascorsi due mesi in campagna, in solitudine, si era convinto che questo non era uno di quegli innamoramenti che aveva provato nella prima giovinezza; che questo sentimento non gli dava un attimo di tregua, che non poteva vivere senza risolvere la questione se ella sarebbe stata o no sua moglie; che la sua disperazione derivava solo dalla sua fantasia e che non aveva prova alcuna per credere di dover essere respinto. E adesso era arrivato a Mosca con la ferma decisione di chiedere la mano di Kitty e di sposarsi, se fosse stato accolto. Se no\ldots{} se l'avessero respinto, non sapeva neppure immaginare cosa sarebbe successo di lui. 

\capitolo{VII}Giunto a Mosca col treno della mattina, Levin si era fermato dal fratellastro maggiore Koznyšev, e, mutato d'abito, era entrato nello studio col proposito di dirgli subito per quale motivo era venuto a chiedergli consiglio; ma il fratello non era solo. C'era da lui un noto professore di filosofia che era venuto da Char'kov proprio per chiarire una divergenza sorta fra di loro a proposito di una questione importante. Il professore conduceva un'accesa polemica contro i materialisti e Sergej Koznyšev seguiva con interesse tale polemica e, dopo aver letto l'ultimo articolo del professore, gli aveva scritto in una lettera le proprie obiezioni, rimproverandogli le troppo larghe concessioni fatte ai materialisti. E il professore era venuto subito per discutere la cosa. Il discorso era avviato sulla questione di moda; esiste o no un limite fra i fenomeni psichici e quelli fisiologici, e dove esso si trova? 

Sergej Ivanovic andò incontro al fratello con l'usuale sorriso cortesemente freddo che aveva per tutti e, presentandolo al professore, continuò il discorso. 

L'ometto giallognolo, occhialuto, dalla fronte bassa, si distolse un attimo dalla conversazione per salutare, e riprese il discorso senza fare attenzione a Levin. Levin sedette in attesa che il professore se ne andasse, quando improvvisamente prese interesse all'argomento. 

Levin si era spesso imbattuto negli articoli di cui si parlava e li aveva letti in riviste, per completare le sue cognizioni di laureato in scienze naturali; ma non aveva mai collegato quelle deduzioni scientifiche sull'origine zoologica dell'uomo, sui riflessi, sulla biologia o sulla sociologia, ai problemi sul significato della vita e della morte che negli ultimi tempi pur gli venivano in mente sempre e sempre più spesso. 

Nell'ascoltare la conversazione del fratello col professore, notava che essi collegavano le questioni scientifiche a quelle dello spirito; alcune volte si avvicinavano a quest'ultime, ma ogni volta che si avvicinavano al punto che a lui sembrava essenziale, se ne ritraevano immediatamente e si ingolfavano nel campo delle disquisizioni sottili, delle riserve, delle citazioni, delle allusioni, dei rinvii a nomi autorevoli, ed egli stentava a capire di che cosa parlassero. 

- Io non posso ammettere - diceva Sergej Ivanovic con la sua abituale chiarezza e precisione di pensiero ed eleganza di eloquio, - io non posso in nessun modo essere d'accordo col Keiss nell'ammettere che tutta la mia visione del mondo esteriore derivi dalle sensazioni. Il concetto fondamentale dell'essere non ci viene dalla sensazione, giacché non abbiamo neanche un organo speciale che ci trasmetta questo concetto. 

- Sì, ma loro, Wurst e Knaust e Pripasov, vi risponderanno che il vostro concetto dell'essere deriva dall'insieme di tutte le sensazioni, che questo concetto dell'essere è il risultato delle sensazioni. Wurst dice addirittura che non appena viene a mancare la sensazione cessa anche la nozione dell'essere. 

- Io dico al contrario\ldots{} - cominciò Sergej Ivanovic. 

Ma a questo punto parve di nuovo a Levin che essi, avvicinatisi al punto essenziale, se ne ritraessero e decise di rivolgere una domanda al professore. 

- Allora, dunque, se i miei sensi sono annientati, se il mio corpo muore, non vi è più esistenza alcuna? - chiese. 

Il professore, contrariato, e come colto, per l'interruzione, da un dolore intellettuale, si voltò verso lo strano interlocutore che aveva più l'aria di un facchino che di un filosofo, e portò gli occhi su Sergej Ivanovic come a dirgli: ``Che cosa rispondere qui?''. Ma Sergej Ivanovic, che era lontano dal parlare con lo sforzo e la unilateralità con cui parlava il professore, e che aveva nella mente abbastanza spazio per rispondere al professore e per cogliere nello stesso tempo il semplice spontaneo punto di vista con cui era stata formulata la domanda, sorrise e disse: 

- Non abbiamo ancora il diritto di risolvere una questione simile. 

- Non abbiamo dati - asserì il professore e continuò le sue argomentazioni. 

- No - diceva - io fo notare che se, come dice precisamente il Pripasov, la percezione ha come base la sensazione, noi dobbiamo allora distinguere rigorosamente questi due concetti. 

Levin non ascoltava già più e aspettava solo che il professore se ne andasse. 

\capitolo{VIII}Quando il professore se ne fu andato, Sergej Ivanovic si rivolse al fratello: \\
- Sono molto contento che tu sia venuto. Per molto? Come vanno le nostre cose? 

Levin sapeva che le cose di casa interessavano molto poco il fratello maggiore e che solo per compiacenza gliene chiedeva; rispose perciò soltanto circa la vendita del frumento e il ricavato. 

Avrebbe voluto dire al fratello della sua intenzione di sposarsi e chiedergli consiglio, ed era fermamente deciso a questo; ma dopo aver visto il fratello, dopo aver ascoltato la conversazione con il professore, e udito quel tono involontario di protezione col quale il fratello gli chiedeva delle faccende amministrative (il fondo materno era indiviso e Levin si occupava di entrambe le parti), Levin sentì che c'era qualcosa che gli impediva di parlare al fratello della sua decisione di sposarsi. Sentiva che il fratello non avrebbe visto la cosa così come egli avrebbe voluto. 

- Ebbene, come va da voi il consiglio distrettuale? - domandò Sergej Ivanovic che si interessava molto dell'istituzione del consiglio cui attribuiva grande importanza. 

- Ma, davvero, non so\ldots{} 

- Ma come? Ma tu non sei membro del consiglio distrettuale? 

- No, non lo sono più; me ne sono uscito e non vado più alle riunioni. 

- Peccato! - esclamò Sergej Ivanovic, accigliandosi. 

Levin prese a dire a sua discolpa quello che si faceva nelle riunioni del distretto. 

- Ecco, sempre così - lo interruppe Sergej Ivanovic. - Noi russi siamo fatti così. Forse è anche una nostra buona qualità\ldots{} la facoltà di vedere sempre i nostri difetti; ma noi esageriamo, e ci consoliamo con l'ironia che abbiamo sempre pronta sulle labbra. Io ti dico solo questo: metti in mano a un altro popolo d'Europa un'istituzione come il nostro consiglio\ldots{} i tedeschi e gli inglesi ne caverebbero la libertà; noi invece, ci ridiamo su. 

- Ma che fare? - disse Levin mortificato. - Era il mio ultimo esperimento e l'ho fatto con tutta l'anima. Non posso. Non sono adatto. 

- Non è che non sei adatto - disse Sergej Ivanovic - tu non guardi la cosa così come va guardata. 

- Forse - disse Levin scoraggiato. 

- Sai, Nikolaj è di nuovo qui. 

Il fratello Nikolaj, germano e maggiore di Konstantin Levin e fratello per parte di madre di Sergej Ivanovic, era un uomo rovinato che aveva sperperato la maggior parte del suo patrimonio, frequentava l'ambiente più strano e più guasto, ed era in lite coi fratelli. 

- Cosa dici? - gridò Levin. - Come lo sai? 

- Prokofij l'ha visto per istrada. 

- Qui, a Mosca? e dov'è? lo sai? - Levin s'alzò dalla sedia, come se volesse andar via subito. 

- Mi dispiace d'averti detto questo - disse Sergej Ivanovic, scrollando il capo all'agitazione del fratello minore. - Ho cercato di sapere dove vive e gli ho mandato la sua cambiale intestata a Trubin che ho pagato io. Ecco quello che mi ha risposto. 

E Sergej Ivanovic prese un biglietto di sotto a un fermacarte e lo porse al fratello. 

Levin lesse quello che vi era stato tracciato con una scrittura strana, a lui familiare: ``Chiedo umilmente di essere lasciato in pace. Questa è l'unica cosa che pretendo dai miei cari fratelli. Nikolaj Levin''. 

Levin lesse e, senza alzar la testa, rimase in piedi davanti a Sergej Ivanovic col biglietto in mano. 

Nell'animo suo lottavano in quel momento il desiderio di ignorare il fratello disgraziato e la coscienza che ciò sarebbe stato male. 

- È evidente che vuole offendermi - continuò Sergej Ivanovic. - Ma non può offendermi; e io vorrei aiutarlo con tutta l'anima, ma so che non è possibile. 

- Eh, sì - ripeté Levin. - Capisco e apprezzo il tuo atteggiamento verso di lui; ma io andrò da lui. 

- Se ne hai voglia, vacci, ma non te lo consiglio - disse Sergej Ivanovic. - Non già per me, non temo certo che egli ti metta in urto con me; ma è per te, che ti consiglio di non andare. Aiutarlo non si può. Comunque fa' come vuoi. 

- È probabile che non si possa neanche aiutarlo, ma io sento\ldots{} proprio in questo particolare momento\ldots{} già, ma questa è un'altra cosa\ldots{} Insomma, sento che non posso restarmene tranquillo. 

- Io questo non lo capisco - disse Sergej Ivanovic. - Una cosa capisco invece - aggiunse - che questa è una lezione di umiltà. Da che nostro fratello Nikolaj è diventato quello che è, io ho preso a considerare in modo diverso e con maggiore indulgenza ciò che si chiama abiezione\ldots{} Lo sai cosa ha fatto\ldots{} 

- Ah, tremendo, tremendo! - ripeté Levin. 

Dopo aver avuto dal domestico di Sergej Ivanovic l'indirizzo del fratello, Levin avrebbe voluto andare immediatamente da lui; ma, riflettendo, aveva deciso di rinviare la visita alla sera. Prima di tutto, per avere serenità di spirito, doveva decidere la faccenda per la quale era venuto a Mosca. Così dalla casa del fratellastro, Levin era passato all'ufficio di Oblonskij e, informatosi degli Šcerbackij, era andato dove gli era stato detto che avrebbe potuto trovare Kitty. 

\capitolo{IX}Alle quattro precise, col cuore che gli batteva, Levin scese dalla vettura al giardino zoologico e si incamminò per un viottolo verso le montagne di ghiaccio e verso il campo di pattinaggio dove era sicuro di trovare lei, perché all'ingresso aveva visto la carrozza degli Šcerbackij. 

La giornata era chiara, gelida. All'ingresso c'erano file e file di carrozze, slitte, vetturini e gendarmi. Una folla ben vestita, coi cappelli che luccicavano al sole forte, brulicava all'ingresso e per i viali ben spazzati, fra le casette russe dalle travi scolpite, mentre le vecchie betulle frondose del giardino, con tutti i rami curvi per la neve, sembravano adorne di nuove pianete di gala. 

Levin andava per un viottolo verso il campo di pattinaggio, e diceva a se stesso: ``Non bisogna agitarsi\ldots{} bisogna star tranquilli. Perché? Come mai? Taci, sciocco'' diceva rivolto al cuore. Ma quanto più cercava di calmarsi, tanto più gli si mozzava il respiro. Un amico lo incontrò e lo chiamò, ma Levin non riconobbe chi era. Si accostò alle montagne di ghiaccio sulle quali stridevano le catene delle piccole slitte rotolanti e risonavano voci allegre. Fece ancora alcuni passi, e davanti a lui si aprì il campo di pattinaggio e, subito, in mezzo a tutti quelli che pattinavano riconobbe lei. 

Riconobbe che era là per la gioia e l'ansia che gli afferrarono il cuore. Lei stava in piedi, parlando con una signora, all'estremo opposto del campo. Non c'era nulla di particolare, almeno così sembrava, nell'abito suo e nella sua posa; ma per Levin era così facile riconoscere lei tra tanta gente, così come una pianta di rose fra le ortiche. Tutto prendeva luce da lei: era lei il sorriso che illuminava tutto, d'ogni intorno. ``Ma potrò davvero scendere là sul ghiaccio, accostarmi?'' pensò. Il luogo dove lei era gli sembrò un impenetrabile luogo sacro, e per un attimo fu sul punto di andarsene, tanta agitazione lo aveva preso. Dovette fare uno sforzo su se stesso e considerare che accanto a lei camminava gente di ogni specie e che anche lui poteva andare là a pattinare. Scese, evitando di guardarla a lungo, come si fa col sole, ma vedeva lei, come si vede il sole, anche senza guardare. 

In quel giorno della settimana e a quell'ora si riunivano sul ghiaccio persone di uno stesso gruppo che si conoscevano fra di loro. C'erano i campioni del pattinaggio, che si esibivano con arte, e c'erano quelli che imparavano reggendosi alle sedie, con movimenti timidi e impacciati, e c'erano ragazzi e persone anziane che pattinavano per ragioni igieniche: tutti parvero a Levin persone elette e felici perché erano là, vicino a lei. I pattinatori, invece, sembravano sorpassarla con assoluta indifferenza, raggiungerla, parlare persino e divertirsi in modo del tutto indipendente da lei, profittando del ghiaccio ottimo e del buon tempo. 

Nikolaj Šcerbackij, cugino di Kitty, in giacchetta corta e pantaloni stretti, sedeva su di una panchina, coi pattini ai piedi e, visto Levin, gli gridò: 

- Olà, il primo pattinatore di Russia! Da quanto tempo siete qui? Ottimo ghiaccio, mettetevi i pattini. 

- Non li ho neppure - rispose Levin, sorpreso di quell'ardire e di quella disinvoltura alla presenza di lei, che egli, anche senza guardare, non perdeva mai di vista. Sentiva che il sole si avvicinava. Ella era in un canto e, strette ad angolo ottuso le gambe sottili negli stivaletti, visibilmente incerta, gli pattinò incontro. Un ragazzo in costume russo, che gesticolava in maniera disperata e si piegava verso terra, la sorpassò. Ella non pattinava ancora del tutto sicura; e, cacciate le mani fuori del piccolo manicotto, sospeso a un cordone, le teneva pronte; guardando Levin che aveva riconosciuto, sorrideva a lui e alla propria timidezza. Superata la curva, si dette una leggera spinta con la gamba agile e pattinò diritta verso Šcerbackij; afferratasi a lui con la mano, fece, sorridendo, un cenno col capo a Levin. Era più bella di quanto non immaginasse. 

Quando Levin la pensava poteva rappresentarsi al vivo tutta lei, e in particolare l'incanto di quella testina bionda dall'espressione infantile, limpida e buona, così graziosamente posata sulle spalle ben fatte di fanciulla. L'infantile espressione del viso congiunta alla bellezza sottile della figura formavano il suo incanto particolare, che egli aveva ben presente; ma quello che in lei lo colpiva sempre come cosa inattesa, era l'espressione degli occhi miti, tranquilli e schietti, e quel sorriso che trasportava Levin in un mondo magico nel quale si sentiva intenerito e placato, così come ricordava di essere stato nei pochi giorni felici della sua prima infanzia. 

- Da molto qui? - disse lei, dandogli la mano. - Grazie - aggiunse quando egli raccattò il fazzoletto cadutole dal manicotto. 

- Io? io, da poco, ieri\ldots{} quest'oggi, cioè\ldots{} sono arrivato - rispose Levin non avendo capito subito la domanda per l'agitazione. - Volevo venire da voi - aggiunse, ma, ricordatosi subito con quale intenzione la cercava, si turbò e arrossì. - Non sapevo che pattinaste, ed anche bene. 

Lei lo guardò con attenzione come se desiderasse capire la causa di quel turbamento. 

- Bisogna tenerla in conto la vostra lode. Qui corre voce che siate il miglior pattinatore - disse lei, scotendo con la piccola mano inguantata gli aghi di brina che si erano posati sul manicotto. 

- Già, una volta pattinavo con passione; volevo raggiungere la perfezione. 

- Voi fate tutto con passione, a quanto pare - disse lei sorridendo. - Ho tanta voglia di vedere come pattinate. Mettetevi i pattini e andiamo a pattinare insieme. 

``Pattinare insieme? È mai possibile?'' pensò Levin guardandola. 

- Me li infilo subito - disse. 

E andò a mettersi i pattini. 

- Da un pezzo non vi si vedeva, signore - disse l'uomo dei pattini alzandogli un piede e avvitando il tacco.- Dopo di voi, di signori, non ce n'è stato più nessuno in gamba. Va bene così? - diceva, stringendo le cinghie. 

- Bene, bene, presto per favore - rispondeva Levin, trattenendo a stento il sorriso di felicità che gli appariva involontariamente sul viso. ``Sì: ecco la vita - pensò - ecco la felicità. Insieme, ha detto lei, andiamo a pattinare insieme. Dirglielo ora? Ma, ecco perché ho paura di parlare, perché sono felice, felice sia pure di speranza\ldots{} E allora? Ma si deve, si deve! Bando alla paura!''. 

Levin si alzò, si tolse il soprabito e, correndo sul ghiaccio non levigato intorno al casotto, si lanciò di corsa sulla superficie liscia e pattinò senza sforzo, rallentando e dirigendo la corsa, come spinto solo dalla propria volontà. Le si accostò timido, ma di nuovo il sorriso di lei lo placò e rasserenò. 

Gli dette la mano e si avviarono insieme aumentando l'andatura, e quanto più questa diveniva veloce tanto più forte ella stringeva la mano di lui. 

- Con voi avrei imparato più presto; non so perché, mi sento sicura con voi - gli disse. 

- Ed anch'io mi sento sicuro quando voi vi appoggiate a me - disse lui; ma spaventato di quello che aveva detto, arrossì. E infatti, appena pronunziate quelle parole, fu come se il sole si fosse nascosto dietro le nuvole: il viso di lei perse la sua tenerezza, e Levin riconobbe il giuoco a lui noto del viso che rivelava lo sforzo del pensiero: sulla fronte spianata era apparsa una piccola ruga. 

- C'è qualcosa che vi spiace? Ma già io non ho il diritto di chiedere - aggiunse in fretta. 

- E perché no? No, non c'è nulla che mi spiaccia - rispose, fredda, lei, e aggiunse subito: - Non avete visto m.lle Linon? 

- Non ancora. 

- Andate da lei, vi vuole tanto bene. 

``Cos'è questo? L'ho contrariata. Signore, aiutami!'' pensò Levin e corse verso la vecchia francese dai riccioli grigi, seduta sulla panchina. Costei l'accolse come un vecchio amico, mostrando nel sorriso i suoi denti finti. 

- Ecco, si cresce, non è vero? - gli disse indicando con gli occhi Kitty - e noi si invecchia. Tiny bear è già grande! - continuò la francese ridendo e ricordando lo scherzo sulle tre signorine ch'egli chiamava col nome dei tre orsacchiotti della fiaba inglese. - Ricordate, voi la chiamavate così? 

Egli non ricordava proprio nulla, ma lei rideva e si compiaceva di questo scherzo da più di dieci anni. 

- Su, su andate a pattinare. La nostra Kitty ha cominciato a pattinare bene, non è vero? 

Quando Levin corse di nuovo verso Kitty, il viso di lei non era più severo, gli occhi guardavano sinceri e dolci, eppure a Levin parve che nella sua dolcezza ci fosse una particolare intonazione di calma voluta. E gliene venne tristezza. Dopo aver parlato un po' della vecchia governante, delle sue originalità, ella gli chiese della sua vita. 

- Non vi vien noia d'inverno, in campagna? - disse. 

- No, niente noia, sono tanto occupato - disse lui sentendo d'essere soggiogato da quel tono calmo al quale non avrebbe avuto la forza di sottrarsi, proprio così com'era successo al principio dell'inverno. 

- Siete venuto per molto tempo? - gli chiese Kitty. 

- Non lo so - rispose lui, senza pensare a quel che diceva. Gli era venuto in mente il pensiero che se si fosse avvezzato a quel tranquillo tono di amicizia, sarebbe di nuovo partito senza aver risolto nulla, e decise di opporvisi. 

- Come, non lo sapete? 

- Non so, dipende da voi - disse, ed ebbe subito paura delle proprie parole. 

O ch'ella non avesse sentito quelle parole, o che non avesse voluto sentirle, certo sembrò inciampicare, batté due volte col piedino a terra e pattinò in fretta via da lui. Si avvicinò pattinando a m.lle Linon, le disse qualcosa e si diresse verso il casotto dove le signore toglievano i pattini. 

``Dio mio, che ho fatto! Signore Iddio! Aiutami, guidami tu!'' diceva Levin pregando e, sentendo nello stesso tempo un bisogno di moto violento, prendeva la rincorsa e disegnava giri esterni e interni. 

In quel momento uno dei giovani, il più abile dei nuovi pattinatori, con la sigaretta in bocca, uscì dal caffè sui pattini e, presa la rincorsa, si slanciò giù per gli scalini, strepitando e saltellando. Era volato giù, e, senza cambiar neppure la libera posizione delle braccia, aveva ripreso a pattinar sul ghiaccio. 

- Ah, ecco un esercizio nuovo - disse Levin, e corse subito a tentarlo. 

- Volete ammazzarvi! Ci vuol allenamento! - gli gridò Nikolaj Šcerbackij. 

Levin salì i gradini, prese la rincorsa quanto più poté dall'alto e si lanciò giù, mantenendosi in equilibrio con le braccia nel movimento insolito. Sull'ultimo scalino inciampò, ma, sfiorato appena il ghiaccio con la mano, fece un movimento brusco, si raddrizzò e, ridendo e pattinando, volò via. 

``Bravo, simpatico - pensò Kitty, uscendo in quel momento dal casotto con m.lle Linon e guardandolo con un sereno sorriso carezzevole, come un fratello al quale si vuol bene. - Possibile che io sia colpevole, possibile che abbia fatto qualcosa di male? Dicono: civetteria. Lo so che non amo lui, ma intanto con lui ci sto volentieri e lui è così bravo. Ma perché ha detto quella cosa?\ldots{}'' pensava. 

Nel veder Kitty che andava via e la madre che la raggiungeva sulle scale, Levin, tutto rosso ancora per il movimento brusco che aveva fatto, si fermò a riflettere. Si tolse i pattini e raggiunse all'uscita madre e figlia. 

- Molto lieta di vedervi - disse la principessa. - Il giovedì, come sempre, riceviamo. 

- Allora, oggi? 

- Saremo molto lieti di vedervi - rispose asciutta la principessa. 

Questo tono secco amareggiò Kitty, ed ella non poté trattenersi dall'attenuare la freddezza della madre. Girò la testa e con un sorriso disse: 

- A rivederci. 

In quel momento Stepan Arkad'ic, col cappello di traverso, il viso e gli occhi luccicanti, entrava nel giardino come un trionfatore. Ma, avvicinatosi alla suocera, rispose con viso contrito e confuso alle domande di lei sulla salute di Dolly. Dopo aver parlato a voce bassa e sommessa con la suocera, raddrizzò il torace e prese Levin sottobraccio. 

- E allora, andiamo? - chiese. - Non ho fatto che pensare a te e sono molto contento che tu sia venuto - disse, guardandolo negli occhi con aria significativa. 

- Andiamo, andiamo - rispose felice Levin che non cessava di ascoltare il tono della voce che aveva detto ``a rivederci'' e di vedere il sorriso col quale le parole erano state dette. 

- All'``Inghilterra'' o all'``Ermitage''? 

- Per me è lo stesso. 

- Su, all'``Inghilterra'' - disse Stepan Arkad'ic e scelse l'``Inghilterra'' perché all'``Inghilterra'' aveva un debito più grosso che non all'``Ermitage'', e riteneva suo dovere farsi vedere in quel ristorante. - Hai una vettura? Benissimo, perché ho rimandato indietro la mia. 

Per tutta la strada gli amici tacquero. Levin pensava cosa potesse significare quel mutamento di espressione nel viso di Kitty, e ora rassicurava se stesso col dirsi che una speranza c'era, ora si abbandonava alla disperazione sembrandogli chiaro che la sua speranza fosse completamente insensata; intanto si sentiva tutto un altro uomo, non più simile a quello che era stato fino al sorriso di lei e fino alle parole ``a rivederci''. 

Stepan Arkad'ic durante il percorso componeva la lista del pranzo. 

- Ti piace il rombo? - disse a Levin quando furono giunti. 

- Che cosa? - domandò Levin. - Il rombo? Ah, sì, mi piace straordinariamente il rombo. 
\enlargethispage{1\baselineskip}

\capitolo{X}Quando Levin entrò nel locale con Oblonskij, non poté fare a meno di notare una certa particolare espressione, come una vivacità contenuta nel viso e nella figura tutta di Stepan Arkad'ic. Oblonskij si tolse il cappotto e, col cappello calato da un lato, passò nella sala da pranzo, dando gli ordini ai tartari che gli si erano messi dietro, in frac e col tovagliolo sul braccio. Salutando a destra e a sinistra gli amici che si trovavano là e che lo salutavano dovunque con gioia, si avvicinò al banco, prese come antipasto vodka e pesce salato e disse qualcosa alla francese che sedeva alla cassa tutta pitturata e ricoperta di nastri, pizzi e ghirigori, in modo che anche questa si mise a ridere schiettamente. Levin non bevve la vodka solo perché gli dava fastidio quella francese che sembrava fatta di capelli finti, poudre de riz e vinaigre de toilette. Si allontanò in fretta da lei come da un luogo sudicio. L'animo suo era tutto pieno del ricordo di Kitty e nei suoi occhi splendeva un sorriso di trionfo e di felicità. 

- Di qua, eccellenza, prego, qua nessuno disturberà vostra eccellenza - diceva un vecchio tartaro biancastro, che più degli altri gli si era appiccicato, con un vasto ventre che sporgeva tra le falde del frac aperte. - Prego, eccellenza, - diceva a Levin, mostrando di occuparsi, in segno di deferenza verso Stepan Arkad'ic, anche dell'ospite. 

Dopo aver steso, in un batter d'occhio, una tovaglia di bucato su di un tavolo tondo già ricoperto di un'altra tovaglia, proprio sotto a un doppiere di bronzo, accostò le sedie di velluto e si piantò davanti a Stepan Arkad'ic col tovagliolo e la lista in mano, aspettando ordini. 

- Se vostra eccellenza ordina un salottino separato, subito se ne farà uno libero: il principe Golycin con una signora. Sono arrivate le ostriche fresche. 

- Ah, le ostriche! 

Stepan Arkad'ic si mise a pensare. 

- Dobbiamo cambiare piano, Levin? - disse fermando un dito sulla carta. Il suo viso esprimeva seria perplessità. - Son buone le ostriche? Bada, ve'! 

- Di Flensburg, eccellenza, non di Ostenda. 

- Flensburg o non Flensburg, sono poi fresche? 

- Le abbiamo avute ieri, eccellenza. 

- E va bene, non potremmo forse incominciare dalle ostriche e poi cambiare tutto il piano? Eh? 

- Per me è lo stesso. Per me meglio di tutto\ldots{} zuppa di cavoli e polenta. Ma qui non c'è di questa roba. 

- Kaša a la rjùss? desidera il signore? - disse il tartaro chinandosi su Levin come una balia sul bambino. 

- No, scherzi a parte, per me va bene quello che sceglierai tu. Ho pattinato un po' e ora ho voglia di mangiare. E non credere - aggiunse notando sul viso di Oblonskij un'aria di disappunto - che non apprezzi la tua scelta. Mangerò e con gusto. 

- Altro che! Di' quello che vuoi, ma questo è uno dei piaceri della vita - disse Stepan Arkad'ic. - Su, allora, amico mio, dacci due dozzine, ma forse è poco, tre dozzine di ostriche, una minestra di radiche\ldots{} 

- Prentanjèr - riprese il tartaro. Ma Stepan Arkad'ic evidentemente non voleva concedergli la soddisfazione di chiamare le pietanze in francese. 

- Di radiche, sai. Poi del rombo con una salsa densa, poi del rosbif: ma guarda che sia buono. Un cappone, e che so, via, della macedonia di frutta. 

Il tartaro, ricordatosi che Stepan Arkad'ic aveva l'abitudine di non nominare mai le portate in francese, non gli tenne dietro a ripetere, ma si concesse infine la soddisfazione di elencare tutta l'ordinazione secondo la carte: ``Sup prentanjèr, tjurbò sos Bomaršé, pulàrd alestragón, maseduàn de frjuì''; - e subito, come una molla, riposta la lista rilegata e presane un'altra, quella dei vini, la sottopose a Stepan Arkad'ic. 

- E cosa berremo? 

- Per me quello che vuoi tu; pur che non sia molto\ldots{} Dello champagne. 

- Come? In principio? Ma sì, hai ragione. Ti piace quello di marca bianca? 

- Kašé blan - riprese il tartaro. 

- Su, via, dacci marca bianca sulle ostriche, e poi vedremo. 

- Sissignore. E quale vino da pasto? 

- Del nuits; ma no, allora è meglio il classico chablis. 

- Sissignore, il solito formaggio? 

- Ma sì; del parmigiano. O te ne piace un altro? 

- No, per me è lo stesso - disse Levin trattenendo a stento un sorriso. 

E il tartaro con le falde svolazzanti, corse via e dopo cinque minuti entrò volando con un vassoio di ostriche aperte sui gusci di madreperla e una bottiglia fra le dita. 

Stepan Arkad'ic spiegazzò il tovagliolo inamidato, se lo ficcò nel panciotto e, posate tranquillamente le braccia sulla tavola, prese a occuparsi delle ostriche. 

- Non sono cattive - diceva, strappando con la forchetta d'argento le ostriche in guazzo dal guscio di madreperla e inghiottendone una dietro l'altra. 

- Non sono cattive - ripeteva, alzando gli occhi umidi e lustri ora su Levi, ora sul tartaro. 

Levin mangiava anche lui le ostriche, sebbene il pane bianco col formaggio gli piacesse di più. Ma si beava a guardare Oblonskij. Perfino il tartaro che aveva stappato lo champagne e lo versava nelle larghe coppe sottili guardava Stepan Arkad'ic con un evidente sorriso di compiacimento, aggiustandosi la cravatta bianca. 

- Ma non ti piacciono le ostriche? - disse Stepan Arkad'ic vuotando la coppa - o forse sei preoccupato? Eh? 

Voleva che Levin stesse di buon umore. Non che Levin non fosse di buon umore, ma era piuttosto impacciato. Con quello che aveva nell'animo provava sgomento e disagio in quel ristorante, in mezzo a salottini riservati dove si pranzava con donne, fra un andirivieni di gente e in mezzo a tutta quella mostra, a quello sfoggio di bronzi, specchi, becchi a gas e tartari. Tutto questo lo offendeva. Aveva paura di contaminare quel che gli riempiva l'anima. 

- Io? Sì, sono preoccupato; ma poi tutto questo mi dà soggezione - disse. - Tu non puoi immaginare come per me, abitante della campagna, tutto questo sia strano, così come le unghie di quel signore che ho visto da te\ldots{} 

- Già, ho visto che le unghie del povero Grinevic ti interessavano molto - disse, ridendo, Stepan Arkad'ic. 

- Non riesco a capire - rispose Levin. - Ma tu cerca di metterti nei panni miei, mettiti dal punto di vista dell'abitante di campagna. Noi in campagna cerchiamo di avere le mani fatte in modo che sia comodo lavorarci, perciò le unghie le tagliamo, e qualche volta ci rimbocchiamo le maniche. E qui invece c'è chi lascia crescere le unghie finché reggono e si aggancia ai polsi bottoni che paion piattini, in modo da non poter far più nulla con le mani. 

Stepan Arkad'ic sorrideva allegro. 

- Eh, già. Questo vuol dire che per lui il lavoro manuale non è più necessario. È il cervello che lavora\ldots{} 

- Sarà. Ma per me ciò è strano; così come, per me, è strano che, mentre noi abitanti di campagna cerchiamo di saziarci al più presto per metterci in condizione di compiere il nostro lavoro, noi due, in questo momento, stiamo facendo di tutto per non saziarci; e per questo mangiamo le ostriche\ldots{} 

- Su via, ma s'intende - riprese Stepan Arkad'ic. - Ma è proprio in questo lo scopo dell'evoluzione: nel fare di tutto un godimento. 

- Se questo è lo scopo, aspirerei a essere un selvaggio. 

- Sei un selvaggio anche così. Voi Levin siete tutti selvaggi. 

Levin sospirò. Si ricordò del fratello Nikolaj, provò vergogna e pena e si accigliò, ma Oblonskij prese a parlare di un argomento che lo distrasse subito. 

- E allora, ci vai stasera dai nostri, dagli Šcerbackij? - disse, allontanando i gusci vuoti e scabri, avvicinando a sé il formaggio e ammiccando significativamente con gli occhi. 

- Sì, ci vado senz'altro - rispose Levin. - Benché sia convinto che la principessa mi abbia invitato controvoglia. 

- Ma che dici! Sciocchezze! È il suo modo di fare\ldots{} Su, via, amico, dacci la minestra!\ldots{} È il suo modo di fare, grande dame - disse Stepan Arkad'ic. - Anch'io verrò ma prima devo andare alla prova di canto della contessa Bonina. Eh già, come si fa a dire che non sei un selvaggio? Come spiegare che sul più bello sei scomparso da Mosca? Gli Šcerbackij mi chiedevano di te continuamente, come se io dovessi sapere. E io so una sola cosa: che fai sempre quello che nessuno fa. 

- Già - disse Levin lentamente e con emozione. - Tu hai ragione, sono un selvaggio. Ma questa mia selvatichezza non consiste nel fatto che me ne sono andato, ma che son venuto. Ora io son venuto\ldots{} 

- Oh che uomo felice! - esclamò Stepan Arkad'ic guardando Levin negli occhi. 

- E perché? 

- ``Conosco i cavalli ardenti da certi loro segni; conosco i giovani innamorati dagli occhi'' - declamò Stepan Arkad'ic. - Tu hai tutto l'avvenire davanti a te. 

- E che forse tu hai già tutto nel passato? 

- No, non avrò solo il passato, ma tu hai l'avvenire, mentre io ho il presente, e anche quello a sbalzi. 

- Ma che c'è? 

- Non va bene, non va bene. Ma io di me non voglio parlare, e poi, dopo tutto, non si può neanche spiegare - disse Stepan Arkad'ic. - Ma tu perché mai sei venuto a Mosca?\ldots{} Ehi, piglia su! - gridò al tartaro. 

- Non l'indovini? - rispose Levin senza staccare da Stepan Arkad'ic i suoi occhi luminosi. 

- L'indovino, ma non posso cominciare io a parlarne. Già da questo puoi vedere se colgo o no nel segno - disse Stepan Arkad'ic, guardando Levin con un sorriso sottile. 

- E allora che ne dici? - disse Levin con voce tremante e sentendo vibrare tutti i muscoli del viso. - Come la vedi tu la cosa? 

Stepan Arkad'ic bevve lentamente il suo bicchiere di chablis, senza staccare gli occhi da Levin. 

- Io? - disse Stepan Arkad'ic - io non desidero niente più di questo. È la cosa migliore che possa accadere. 

- Ma tu non ti sbagli? Sai bene di che parliamo? - ripeté Levin, ficcando gli occhi nel suo interlocutore. - Credi che sia possibile? 

- Credo che sia possibile. E perché mai impossibile? 

- Ma pensi proprio che sia possibile? No, dimmi tutto quello che pensi! E se mi aspetta un rifiuto? E io anzi ne sono certo\ldots{} 

- Perché pensi questo? - disse Stepan Arkad'ic sorridendo a quell'agitazione. 

- A volte così mi sembra. Certo questo sarebbe terribile per me e per lei. 

- Be', veramente, in ogni caso, per una ragazza non c'è nulla di terribile. Ogni ragazza è lusingata di essere chiesta in matrimonio. 

- Già, ogni ragazza, ma non lei. 

Stepan Arkad'ic sorrise. Conosceva bene il sentimento di Levin; sapeva che per lui tutte le ragazze del mondo si dividevano in due categorie: nella prima c'erano tutte le ragazze di questo mondo tranne lei, e queste ragazze avevano tutte le debolezze umane ed erano esseri molto comuni; nella seconda, c'era lei sola e non aveva nessuna debolezza, ed era superiore ad ogni cosa umana. 

- Aspetta, prendi la salsa - disse trattenendo il braccio di Levin che allontanava da sé la salsa. 

Levin si servì docilmente, ma non permise a Stepan Arkad'ic di mangiare. 

- No, aspetta, aspetta - diceva. - Tu devi capire che questo per me è questione di vita o di morte. Io non ne ho mai parlato con nessuno. E con nessun altro posso parlare di questo se non con te. Perché, ecco, io e te siamo estranei l'uno all'altro: gusti diversi, opinioni, tutto. Ma io so che tu mi vuoi bene e mi capisci e per questo ti voglio un gran bene. Ma in nome di Dio sii sincero con me. 

- Io ti dico quello che penso - disse Stepan Arkad'ic, sorridendo. - Ma io ti dirò di più: mia moglie, una donna straordinaria\ldots{} - Stepan Arkad'ic sospirò, ricordando i suoi rapporti con la moglie, e, sostando un attimo, continuò: - ha il dono dell'introspezione. Vede da una parte all'altra; ma questo è poco, sa quello che accadrà, specie in materia di matrimoni. Per esempio, ha predetto che la Šachovskaja avrebbe sposato Brentel'n. Nessuno ci voleva credere, ed è stato così. Ebbene, lei è dalla parte tua. 

- Come? 

- Così: non solo ti vuol bene, ma dice che Kitty sarà certamente tua moglie. 

A queste parole il viso di Levin s'illuminò d'un tratto di quel sorriso ch'è vicino alle lacrime della commozione. 

- Lei dice questo! - gridò Levin. - Ho sempre detto che tua moglie è un tesoro! E ora basta, basta, non ne parliamo più! - disse, alzandosi. 

- Sì, va bene, mettiti a sedere. 

Levin non poteva stare seduto. Andò su e giù due volte con passo deciso per la stanza che sembrava una piccola gabbia. Sbatté le palpebre per non mostrare le lacrime e solo allora sedette di nuovo a tavola. 

- Tu comprendi - disse - che questo non è un innamoramento. Sono stato innamorato ma non è questo. Questo non è un sentimento mio, ma è una forza esterna che si è impossessata di me. Ero andato via perché avevo concluso che ciò non poteva essere, cioè, intendimi, come una felicità che non poteva esistere sulla terra; ma ho lottato con me stesso e ora vedo che senza di questo non c'è vita. E bisogna dunque decidere\ldots{} 

- E per questo sei andato via? 

- Ah, lascia stare! Quanti pensieri! Quante cose ti devo chiedere! Ascolta. Tu già non puoi immaginare che cosa hai fatto ora per me nel dirmi ciò. Sono così felice da diventare quasi disgustoso; ho dimenticato tutto. Ho saputo oggi che mio fratello Nikolaj\ldots{} Anche di lui mi sono scordato. Mi sembra che anche lui debba essere felice. Questa è una specie di pazzia. Ma c'è una cosa che è terribile\ldots{} Ecco, tu ti sei sposato, tu certamente lo conosci questo sentimento\ldots{} Ed è terribile questo, che noi\ldots{} non più giovani, già con un passato\ldots{} non di amore, ma di peccato\ldots{} ci avviciniamo a un tratto a un essere puro, ignaro. È ripugnante, e non si può non sentirsene indegni. 

- Su, via, tu di peccati ne hai pochi. 

- Eppure, eppure - disse Levin - ``considerando con disgusto la mia vita, fremo e maledico e amaramente mi dolgo''. Proprio così. 

- Che fare? Così è fatto il mondo - disse Stepan Arkad'ic. 

- L'unica mia consolazione è in quella preghiera che ho sempre amata: ``Non secondo i miei meriti, ma secondo la tua misericordia, perdonami''. Soltanto così anche lei può perdonare. 

\capitolo{XI}Levin bevve la sua coppa e i due rimasero in silenzio. \\
- Una cosa nuova devo dirti. Conosci Vronskij? - chiese Stepan Arkad'ic a Levin. 

- No, non lo conosco. Perché me lo chiedi? 

- Versane un'altra - disse Stepan Arkad'ic al tartaro che aveva cessato di riempire le coppe e che gironzolava intorno a loro proprio quando non era necessario. 

- Perché dovrei conoscere Vronskij? 

- Dovresti conoscere Vronskij perché è uno dei tuoi rivali. 

- E che tipo è questo Vronskij? - chiese Levin e il viso suo tramutò l'espressione d'infantile entusiasmo che proprio allora aveva incantato Oblonskij in un'espressione torva e spiacevole. 

- Vronskij è uno dei figli del conte Kirill Ivanovic Vronskij ed è uno dei più bei campioni della gioventù dorata di Pietroburgo. L'ho conosciuto a Tver' quando prestavo servizio là e lui ci veniva per l'arruolamento delle reclute. Ricco sfondato, bello, grandi relazioni, aiutante di campo dello zar e, nello stesso tempo, molto simpatico, un buon ragazzo. Ma oltre che un buon ragazzo, come ho potuto poi conoscerlo qui, è anche colto e intelligente; un giovane che si farà strada. 

Levin si faceva scuro in viso e taceva. 

- Dunque costui è comparso qua dopo di te e, a quanto mi pare di aver capito, è innamorato pazzo di Kitty, e tu capirai che la madre\ldots{} 

- Scusami, ma non capisco nulla - disse Levin cupo e accigliato. E subito si ricordò di suo fratello Nikolaj e come fosse stato perfido l'averlo dimenticato. 

- Aspetta, aspetta - disse Stepan Arkad'ic, sorridendogli e toccandogli il braccio. - Io ti ho detto quello che so, e ti ripeto che per quanto si possa indovinare in cose tanto sottili e delicate, mi sembra che le probabilità siano dalla parte tua. 

Levin si abbandonò all'indietro sulla sedia; il suo viso era pallido. 

- Ma io ti consiglio di decidere la questione al più presto - continuò Oblonskij, riempiendogli la coppa. 

- No, grazie, non posso bere più - disse Levin, allontanando la coppa. - Mi ubriacherei\ldots{} E tu, come te la passi? - continuò, volendo cambiare discorso. 

- Ancora una parola: in ogni caso ti consiglio di risolvere la cosa al più presto. Non ti consiglio di parlare oggi - disse Stepan Arkad'ic. - Va' domattina a far la tua domanda secondo l'uso classico, e che Dio ti benedica\ldots{} 

- Be', non dicevi sempre di voler venire a caccia da me? Ecco, vieni a primavera - disse Levin. 

Ora egli si pentiva con tutta l'anima di aver cominciato quel discorso con Stepan Arkad'ic. Il sentimento tutto suo era contaminato dal discorso su quel tale ufficiale di Pietroburgo suo rivale e dalle supposizioni e dai consigli di Stepan Arkad'ic. 

Stepan Arkad'ic sorrideva. Capiva quel che avveniva nell'animo di Levin. 

- Verrò un giorno o l'altro - disse. 

- Eh, già, amico mio, le donne\ldots{} ecco l'elica intorno alla quale tutto gira. Ecco, anche le mie cose vanno male. E tutto per colpa delle donne. Dimmi tu sinceramente - continuò - dopo aver tirato fuori un sigaro e tenendo la coppa con una mano - dammi un consiglio. 

- A che proposito? 

- Ecco qua. Mettiamo che tu sia ammogliato, che ami tua moglie, ma che tu abbia perso la testa per un'altra donna. 

- Scusa, ma io questo non lo capisco; come se, ecco, proprio così, io ora, dopo essermi saziato, passando accanto a quel negozio di ciambelle ne rubassi una. 

Gli occhi di Stepan Arkad'ic brillavano più del solito. 

- E perché? La ciambella a volte è così profumata che non puoi resistere. 

\begin{quote}
Himmlisch ist's wenn ich bezwungen

Meine irdische Begier; 

Aber noch wenn's nicht gelungen, 

Hatt'ich auch recht hubsch Plaisir!
\end{quote} 

Dicendo questo Stepan Arkad'ic sorrideva finemente. Anche Levin non poté non sorridere. 

- Sì, ma scherzi a parte - continuò Oblonskij - immagina una donna graziosa, un essere mite, affettuoso, povero, solo che abbia sacrificato ogni cosa. Ora, quando tutto è già avvenuto\ldots{} tu m'intendi, si può forse buttarla via? Ammettiamo pure: troncare per non distruggere la propria vita familiare; ma non si può forse avere pena di lei, provvedere, mitigare? 

- Eh, già, scusami. Tu sai, per me le donne si dividono in due categorie\ldots{} cioè, no, più esattamente: vi sono le donne e vi sono\ldots{} Io di magnifiche creature cadute non ne ho viste e non ne vedrò mai, e le donne come quella francese lì al banco, coi ricci, quelle per me sono vermi, e tutte quelle cadute sono tali. 

- E quella del Vangelo? 

- Ah, lascia stare! Cristo non avrebbe mai detto quelle parole, se avesse preveduto quanto se ne sarebbe abusato. Di tutto il Vangelo non si ricordano che quelle parole. Del resto io non dico ciò che penso, ma ciò che sento. Ho avversione per le donne cadute. Tu hai paura dei ragni e io di quei vermi. E tu certamente non hai studiato i ragni e non conosci le loro abitudini: e neanche io. 

- Va bene a parlare così, per te: mi sembri quel tal signore del Dickens che gettava con la mano sinistra dietro la spalla destra tutte le questioni spinose. Ma la negazione di un fatto non ne è la soluzione. Che fare mai, dimmi, che fare? Tua moglie invecchia e tu sei pieno di vita. Non fai in tempo a girarti che già senti di non potere più amare di amore tua moglie, per quanto la stimi. E qui a un tratto ti capita l'amore e sei perduto, sei perduto! - esclamò con sommessa disperazione Stepan Arkad'ic. 

Levin sorrise. 

- Già, e sei perduto - continuò Oblonskij. - Ma che fare? 

- Non rubare le ciambelle. 

Stepan Arkad'ic scoppiò a ridere. 

- Oh, il moralista! Ma tu devi capire che qui ci sono due donne: una insiste solo sui suoi diritti, e questi diritti sono l'amore che tu non puoi più darle; l'altra invece ti sacrifica tutto e non ti chiede nulla. Che devi fare? Come regolarti? Qui sta il dramma pauroso. 

- Se vuoi sapere il mio punto di vista, ti dirò che in questo non ci scorgo dramma. Ed ecco perché? Per me l'amore\ldots{} tutti e due gli amori che, ricordi, Platone definisce nel suo Convito, tutti e due questi amori servono di pietra di paragone degli uomini. Alcuni comprendono l'uno, altri l'altro. E quelli che comprendono solo l'amore non platonico parlano a vuoto di dramma. In un amore simile non può esservi dramma. ``Vi ringrazio umilmente per il piacere, i miei rispetti'' ed ecco tutto il dramma. E per l'amore platonico neppure può esservi dramma perché in un amore simile tutto è chiaro, puro, perché\ldots{} 

In quel momento Levin si ricordò delle sue colpe e della lotta interiore che aveva vissuto e inaspettatamente aggiunse: 

- Ma forse hai ragione, in fin dei conti, anche tu. Anzi, molto probabilmente\ldots{} Ma io non so, non so proprio. 

- Ecco, vedi - disse Stepan Arkad'ic - tu sei un uomo tutto d'un pezzo. Questo è il tuo pregio e il tuo difetto. Tu sei tutto d'un pezzo e vorresti che la vita fosse fatta di avvenimenti integrali, e questo non succede. Ecco, tu disprezzi l'attività del pubblico impiego, perché vorresti che essa corrispondesse sempre allo scopo, e questo non succede. Vorresti che l'attività di un uomo avesse sempre uno scopo, che l'amore e la vita familiare fossero tutt'uno. E questo non succede. Tutta la varietà, la delizia, la bellezza della vita son fatte d'ombre e di luci. 

Levin sospirò e non rispose nulla. Pensava alle cose sue e non ascoltava già più Oblonskij. 

E a un tratto tutti e due sentirono che, pur essendo amici, pur avendo pranzato insieme e bevuto il vino, cosa che ancor più avrebbe dovuto avvicinarli, tuttavia ognuno di loro pensava solo alle proprie cose, e a ciascuno non importava nulla dell'altro. Oblonskij conosceva già questo estremo distacco che avviene, in luogo della fusione, dopo un pranzo, e sapeva bene cosa si dovesse fare in casi simili. 

- Il conto! - gridò, e uscì nella sala accanto dove subito incontrò un aiutante di campo e si mise a parlare con lui di un'attrice e di chi la manteneva. E subito, parlando con l'aiutante di campo, Oblonskij provò sollievo e respirò dopo il colloquio con Levin che lo aveva sempre più sottoposto a uno sforzo intellettuale e spirituale troppo intenso. 

Quando il tartaro comparve col conto di 26 rubli e alcune copeche con l'aggiunta per la vodka, Levin che in altro momento, da buon provinciale, sarebbe inorridito per la propria quota di 14 rubli, non ci fece caso; pagò e si diresse verso casa per cambiar d'abito e andar dagli Šcerbackij dove si sarebbe decisa la sua sorte. 

\capitolo{XII}La principessina Šcerbackaja aveva diciotto anni. Era il primo inverno che faceva il suo ingresso nel gran mondo. I suoi successi erano superiori a quelli delle sorelle e superiori anche a quelli che la principessa si aspettava. Non solo i giovani che frequentavano i balli moscoviti erano tutti più o meno innamorati di Kitty, ma fin dal principio dell'inverno si erano presentati due partiti seri: Levin e, subito dopo la partenza di lui, il conte Vronskij. 

L'apparizione di Levin al principio dell'inverno, le visite frequenti e il suo evidente amore per Kitty erano stato l'oggetto dei primi discorsi seri fra i genitori di Kitty sul suo avvenire, e di litigi fra il principe e la principessa. Il principe era dalla parte di Levin; diceva che non desiderava nulla di meglio per Kitty. La principessa invece, con l'abitudine propria delle donne di girar la questione, diceva che Kitty era troppo giovane, che Levin non mostrava in nessun modo di aver intenzioni serie, che Kitty non mostrava affetto per lui e altre cose: ma non diceva la ragione principale, che s'aspettava, cioè, un partito migliore per sua figlia, e che Levin non le era simpatico, che non lo capiva. Quando Levin partì all'improvviso, la principessa ne fu contenta e diceva trionfante al marito: ``Vedi, avevo ragione io''. 

Quando poi apparve Vronskij, ella fu ancora più contenta, riconfermandosi nella propria idea, che cioè Kitty non doveva trovare un partito semplicemente buono, ma brillante. 

Per la madre non c'era paragone tra Levin e Vronskij. In Levin non le piacevano quegli strani e taglienti giudizi e quella sua mancanza di disinvoltura dovuta a orgoglio, come ella supponeva, e a quella sua vita di campagna, selvaggia a suo parere, fra bestie e contadini. Non le era piaciuto neanche troppo il fatto che, innamoratosi della figlia, avesse frequentato la casa per un mese e mezzo, quasi aspettando qualcosa e osservando, come se temesse di offenderla nel chiederla in isposa, e senza capire che, frequentando una casa dove c'era una ragazza da marito, fosse necessario dichiararsi. E poi a un tratto, senza aver parlato, era andato via. ``Meno male che è così poco attraente che Kitty non si è innamorata di lui'' pensava la madre. 

Vronskij invece soddisfaceva in pieno tutte le aspirazioni della madre. Molto ricco, intelligente, di famiglia nota, sulla via di una brillante carriera militare a corte, era un uomo affascinante. Non si poteva desiderare nulla di meglio. 

Ai balli Vronskij faceva apertamente la corte a Kitty, ballava con lei e ne frequentava la casa; non si poteva dubitare, dunque, della serietà delle sue intenzioni. Nonostante ciò la madre aveva passato tutto l'inverno in grande inquietudine e turbamento. 

La principessa si era sposata trent'anni prima, pronuba una zia. Il fidanzato, del quale si erano già prese informazioni, era venuto, aveva visto la sposa, si era fatto vedere lui stesso; la zia aveva saputo e riferito l'effetto prodotto. L'impressione era stata favorevole; così nel giorno stabilito era stata fatta ai genitori ed era stata da essi accolta la domanda di matrimonio. Tutto era andato in modo facile e semplice, almeno così sembrava alla principessa. Per le figliuole, invece, aveva provato come non fosse facile né semplice la faccenda, pur così comune, di dar marito alle figliuole. Quante ansie, quanti mutamenti di pensiero, quanto denaro speso, quanti urti col marito per i matrimoni delle prime due figlie, Dar'ja e Natal'ja! Adesso, nel presentare in società la più piccola, provava le stesse ansie, gli stessi dubbi; ed erano ancora più gravi che per le figlie maggiori le discussioni col marito. Il vecchio principe, come del resto tutti i padri, era particolarmente severo per l'onore e la virtù delle figliuole e ne era irragionevolmente geloso, specie di Kitty che era la beniamina; ogni momento faceva una scenata alla principessa perché comprometteva la figlia. La principessa si era abituata a questo già quando si era trattato delle altre due, ma ora sentiva che la suscettibilità del principe aveva maggior fondamento. Vedeva che negli ultimi tempi molte cose erano cambiate nelle usanze mondane; che i doveri di una madre erano diventati ancora più difficili. Vedeva che le coetanee di Kitty formavano certi gruppi, frequentavano certi corsi, trattavano con disinvoltura gli uomini, andavano sole in carrozza per le strade, molte di esse non facevano già più l'inchino, e, quel ch'era peggio, erano tutte fermamente convinte che la scelta del marito fosse affar loro e non dei genitori. ``Oggigiorno non ci si marita più come prima'' pensavano e dicevano tutte queste ragazze e anche tutte le persone anziane. Ma come si facesse ora a maritar le figlie, la principessa non riusciva a saperlo da nessuno. L'uso francese - ai genitori la decisione della sorte dei figli - non era accolto, era criticato. L'uso inglese - piena libertà alla ragazza - non era accolto ugualmente ed era impossibile nella società russa. L'uso russo della mediazione era considerato come cosa sconveniente sulla quale tutti ridevano, compresa la principessa. Ma come ci si dovesse maritare e come si dovesse dar marito, nessuno lo sapeva. Tutte le persone con le quali capitava alla principessa di parlarne, le dicevano una cosa sola: ``Su via, di grazia, oggigiorno è tempo di abbandonare tutto questo vecchiume. Sono i giovani che debbono sposarsi e non i genitori; bisogna lasciare ai giovani la facoltà di decidere come vogliono loro''. Ma era un bel dire per chi non aveva figliuole; e la principessa temeva che, facendo conoscenze, la figlia avrebbe potuto innamorarsi, e innamorarsi di chi non aveva nessuna intenzione matrimoniale o di chi non era adatto come marito. E per quanto tutti dicessero alla principessa che al giorno d'oggi i giovani devono da soli costruire il proprio avvenire, non riusciva ad ammetterlo, così come non avrebbe potuto ammettere che, in una qualsiasi epoca, i giocattoli migliori per i bambini di cinque anni potessero essere le pistole cariche. E perciò la principessa era ancora più inquieta per Kitty di quanto non lo fosse stata per le figliuole più grandi. 

Attualmente temeva che Vronskij si limitasse solo a far la corte alla figlia. Si accorgeva che la ragazza era già innamorata di lui, ma si rassicurava pensando che egli era un uomo d'onore e che perciò non avrebbe fatto questo. Ma sapeva pure come, con l'attuale libertà di costumi, fosse facile far perdere la testa ad una ragazza, e come, in genere, gli uomini guardassero con leggerezza a una colpa di questo genere. La settimana prima Kitty aveva raccontato alla madre la sua conversazione con Vronskij durante una mazurca. Questa conversazione aveva tranquillizzato in parte la principessa, ma del tutto serena ella non poteva sentirsi. Vronskij aveva detto a Kitty che, tanto lui che suo fratello, erano così abituati a sottostare in tutto alla madre, che non decidevano mai nulla di importante senza essersi prima consigliati con lei. ``E ora aspetto come una fortuna particolare l'arrivo della mamma da Pietroburgo'' aveva detto lui. 

Kitty aveva raccontato la cosa senza dare alcun peso a queste parole. La madre invece le aveva interpretate diversamente. Sapeva che si aspettava la vecchia signora da un giorno all'altro; sapeva che la vecchia signora sarebbe stata contenta della scelta del figlio, e le pareva strano ch'egli, solo per timore di offendere la madre, non facesse ancora la sua proposta di matrimonio; tuttavia desiderava tanto il matrimonio, e soprattutto la quiete ai propri affanni, che credeva a questo. Per quanto fosse amaro constatare la sfortuna della prima figlia, Dolly, che stava per separarsi dal marito, l'agitazione per la sorte della minore soffocava in lei ogni altro sentimento. Quel giorno con l'apparire di Levin le si era aggiunta una nuova inquietudine. Temeva che la figlia, che pur un tempo - così le era parso - aveva avuto della simpatia per Levin, rifiutasse per troppa onestà Vronskij, e temeva che, per un insieme di cose, l'arrivo di Levin non avesse a confondere e a ostacolare un affare già così prossimo alla conclusione. 

- Ma è molto che è arrivato? - disse, accennando a Levin, la principessa nel tornare a casa. 

- Oggi, maman. 

- Io voglio dire una cosa sola\ldots{} - cominciò la principessa e dal suo viso serio e animato, Kitty indovinò su quale argomento sarebbe scivolato il discorso. 

- Mamma - disse, avvampando in viso e volgendosi con vivacità. - Vi prego, vi prego, non mi parlate di questo. Io so, so tutto. 

Desiderava la stessa cosa che desiderava la madre; ma i motivi del desiderio materno la offendevano. 

- Io voglio dire solo che, dopo aver incoraggiato uno\ldots{} 

- Mamma, amore mio, in nome di Dio, non parlate. Fa così paura parlare di questo. 

- Non ne parlerò - disse la madre vedendo le lacrime negli occhi della figlia. - Ma una cosa sola, figliuola mia: tu mi hai promesso che non avrai segreti per me. Vero? 

- Mai, mamma, nessuno, - rispose Kitty, arrossendo e guardando dritto in faccia alla madre. - Ma ora non ho nulla da dire. Io\ldots{}io se volessi, non so, cosa dire e come\ldots{}non so\ldots{} 

``No, non può mentire con questi occhi'' pensò la madre, sorridendo di quell'agitazione e di quella felicità. La principessa sorrideva perché capiva come appariva grande e importante a lei, poverina, quello che accadeva nell'animo suo. 

\capitolo{XIII}Kitty, dopo pranzo e fino al principio della serata, provò una sensazione simile a quella che prova un giovane prima del combattimento. Il cuore le batteva forte e il pensiero non riusciva a fermarsi su nulla. 

Sentiva che quella sera, quando i due uomini si sarebbero incontrati per la prima volta, si sarebbe decisa la sua sorte. E se li raffigurava continuamente, ora distinti, ora tutti e due insieme. Quando pensava al passato, con gioia e tenerezza si fermava sui ricordi dei suoi rapporti con Levin. I ricordi d'infanzia e l'amicizia di Levin col suo fratello morto davano un particolare poetico incanto ai suoi rapporti con lui. Il suo amore per lei, di cui era sicura, la lusingava e rallegrava. E le era naturale pensare a Levin. Al pensiero di Vronskij invece si frammischiava un certo impaccio, pur essendo egli un perfetto e sereno uomo di mondo; sembrava esserci una certa falsità, non in lui - era molto semplice e cortese - ma piuttosto in lei; mentre con Levin si sentiva completamente spontanea e serena. Ma intanto, quando pensava all'avvenire con Vronskij, le si presentava un luminoso sfondo di felicità; mentre con Levin l'avvenire si presentava nebbioso. 

Salita in camera per indossare l'abito da sera, gettò un'occhiata allo specchio, e si accorse con gioia che era in una delle sue giornate migliori, nel pieno possesso di tutte le sue attrattive, e questo le era tanto necessario per quello che stava per avvenire. Sentiva in sé la calma esteriore e la libera grazia dei movimenti. 

Alle sette e mezzo, appena discesa in salotto, il cameriere annunciò: ``Konstantin Dmitric Levin''. La principessa era ancora in camera sua e il principe non era uscito fuori. ``Ci siamo'' pensò Kitty, e tutto il sangue le affluì al cuore. Nel guardarsi allo specchio ebbe paura del proprio pallore. 

Ormai sapeva con certezza che egli era venuto prima proprio per trovarla sola e farle la sua proposta di matrimonio. E allora soltanto, per la prima volta, la cosa le apparve sotto un aspetto completamente nuovo, diverso. Ora soltanto lei capiva che la questione non riguardava lei sola: con chi sarebbe stata felice e chi amava, ma che in quel momento lei avrebbe dovuto offendere un uomo a cui voleva bene. E offenderlo crudamente\ldots{} Perché? Perché lui, povero caro, l'amava, era innamorato di lei. Ma non c'era nulla da fare; così doveva andare. 

``Dio mio, e dovrò dirglielo proprio io? - pensò. - E che cosa gli dirò? Gli dirò forse che non gli voglio bene? Ma questo non è vero! Che gli dirò allora? Dirò che amo un altro. No, non è possibile. Allora me ne vado via\ldots{}''. 

Si era già accostata alla porta, quando udì il passo di lui. ``No, non è onesto. Ma perché ho paura? Non ho fatto nulla di male. Sarà quel che sarà. Dirò la verità. E poi con lui non ci si può sentire impacciati. Eccolo'' si disse vedendo tutta la sua forte e timida figura con gli occhi scintillanti, rivolti verso di lei. Ella lo guardò diritto nel viso, quasi supplicandolo di farle grazia, e gli porse la mano. 

- Son venuto prima del tempo, mi pare, troppo presto - disse lui guardando il salotto vuoto. E accortosi che le sue previsioni si erano avverate, che cioè nulla gli impediva di dichiararsi, si rabbuiò in viso. 

- Oh, no - disse Kitty e sedette al tavolo. 

- Ma io volevo proprio questo, trovarvi sola - cominciò senza sedersi e senza guardarla per non perder coraggio. 

- La mamma viene subito. Ieri s'è stancata molto. Ieri\ldots{} 

Parlava senza saper lei stessa quello che pronunciavano le sue labbra e senza staccare da lui il suo sguardo supplice e carezzevole. 

Egli la guardò; ella arrossì e tacque. 

- Vi ho detto che non sono venuto per restar molto\ldots{} che questo dipende da voi\ldots{} 

Ella chinava sempre più la testa, non sapendo ella stessa che cosa avrebbe risposto a quello che stava per avverarsi. 

- Che ciò dipende da voi - ripeté lui. - Io volevo dirvi\ldots{} Per questo son venuto\ldots{} che voi\ldots{} siate mia moglie! - esclamò non sapendo egli stesso cosa diceva, ma sentiva che il peggio era stato detto; si fermò e la guardò. 

Lei respirava con affanno, senza guardarlo. Provava un certo incantamento. L'anima sua era come gonfia di felicità. Non credeva che in nessun modo il rivelarsi dell'amore di lui potesse produrle un'impressione così intensa. Ma questo durò un attimo solo. Si ricordò di Vronskij. Alzò su Levin i suoi cari occhi sinceri e, vedendo il viso disperato di lui, rispose in fretta: 

- Questo non può essere, perdonatemi. 

Come gli era stata vicina un minuto prima, tanto importante per la sua vita! E come ora gli si faceva estranea e lontana! 

- Non poteva essere altrimenti - disse lui, senza guardarla. 

S'inchinò e fece per andarsene. 

\capitolo{XIV}Ma proprio in quel momento entrò la principessa. Quando li vide soli e sconvolti, il terrore le si espresse in viso. Levin si inchinò e non disse nulla. Kitty taceva senza alzar gli occhi. ``Grazie a Dio, ha detto di no'' pensò la madre, e il viso le si schiarì nel consueto sorriso col quale accoglieva gli ospiti il giovedì. Sedette e incominciò a interrogare Levin sulla sua vita in campagna. Egli sedette di nuovo in attesa degli ospiti per andarsene inavvertito. 

Dopo cinque minuti entrò un'amica di Kitty che si era sposata l'inverno prima, la contessa Nordston. 

Era una donna secca e giallognola, con occhi neri scintillanti, malaticcia e nervosa. Voleva bene a Kitty e il suo affetto per lei, come accade sempre alle donne maritate che vogliono bene a una ragazza, si esprimeva nel desiderio di trovarle marito secondo il proprio ideale di felicità; desiderava perciò darla a Vronskij. Levin, che aveva incontrato da loro al principio dell'inverno, le era sempre riuscito antipatico. Ogni volta che lo vedeva, la sua occupazione favorita consisteva nel prenderlo in giro. 

``Mi piace quando mi guarda dall'alto della sua superiorità, o interrompe la sua saggia conversazione con me, perché sono una sciocca o quando ancora si benigna di scendere fino a me. Questo mi piace: che discenda. Sono molto contenta che non mi possa sopportare'' diceva di lui. 

Aveva ragione, perché realmente Levin non la poteva sopportare e la disprezzava per tutto quello di cui lei andava orgogliosa e vaga: per quel suo nervosismo, per quel suo sottile spregio e per quella sua indifferenza verso tutto ciò che è comune e quotidiano. 

Fra la Nordston e Levin si erano perciò venuti a stabilire quei rapporti, frequenti nel gran mondo, per cui due persone, pur rimanendo esteriormente in rapporti di cortesia, si disprezzano reciprocamente a tal punto da non riuscire non solo a trattarsi con serietà, ma da non sentirsi neppure offese l'una dall'altra\ldots{} 

La contessa Nordston investì subito Levin. 

- Ah, Konstantin Dmitric. Siete venuto di nuovo in questa nostra depravata Babilonia - disse dandogli la sua piccola mano giallognola e ripetendo le parole dette da lui in una certa occasione al principio dell'inverno, che cioè Mosca era una Babilonia. - Che forse Babilonia si è messa sulla giusta via, o siete voi ad esservi pervertito? - soggiunse, guardando Kitty con un sorriso. 

- Sono molto lusingato, contessa, che ricordiate le mie parole - rispose Levin che si era affrettato a rimettersi, entrando subito, per abitudine, nei suoi rapporti di scherzosa inimicizia con la contessa Nordston. - Evidentemente, esse hanno fatto molto effetto su di voi. 

- Oh, e come! Io prendo nota di tutto. Ebbene, Kitty, hai pattinato di nuovo? 

E cominciò a parlare con Kitty. Per quanto poco conveniente fosse ora per Levin andarsene, tuttavia gli era più facile compiere questa sgarberia che rimaner tutta la serata a osservare Kitty che ogni tanto lo guardava di sfuggita ed evitava di incontrare il suo sguardo. Stava per alzarsi, quando la principessa, avendo notato il suo silenzio, gli rivolse la parola: 

- Vi trattenete a lungo a Mosca? Perché voi, mi pare, vi occupate degli arbitrati del consiglio distrettuale e non potete assentarvi a lungo. 

- No, principessa, non mi occupo più del consiglio distrettuale - disse. - Sono venuto per pochi giorni. 

``Ha qualcosa di speciale - pensò la Nordston, osservando il viso di lui serio e severo. - Chi sa perché non si ingolfa nei suoi ragionamenti. Ma io ce lo porterò. Mi piace immensamente di fargli fare la figura dello sciocco davanti a Kitty, e ci riuscirò''. 

- Konstantin Dmitric - gli disse - spiegatemi, vi prego, voi sapete tutto ciò, che cosa mai significa che da noi, nel villaggio di Kaluga, i contadini e perfino le donne si son mangiati tutto quello che avevano e a noi non hanno dato proprio un bel nulla. Che significa? Voi non fate che cantar le lodi dei contadini. 

In quel momento entrò nella stanza una signora e Levin si alzò. 

- Perdonatemi, contessa, ma io davvero non so nulla di questo e non posso dirvene nulla - disse, e si mise a guardare un ufficiale che era entrato dopo la signora. 

``Deve essere Vronskij'' pensò Levin e, per convincersene, guardò Kitty. Ella aveva fatto appena in tempo a guardare Vronskij e s'era poi girata verso Levin. E da questo solo sguardo dei suoi occhi, involontariamente illuminati, Levin capì che ella amava quell'uomo, e lo capì così fermamente come se glielo avesse detto lei a parole. Ma che uomo era mai? 

Adesso - fosse bene o fosse male - Levin non poteva non rimanere: doveva sapere che uomo era mai quello che lei amava. 

Ci sono delle persone che, incontrando un rivale fortunato in una qualsiasi cosa, sono subito pronte a distogliere lo sguardo da tutto ciò che c'è di buono in lui e a vederne solo le manchevolezze; vi sono persone, invece, che desiderano trovare nel rivale fortunato proprio quelle qualità con le quali costui ha vinto loro, e vedono in lui, con una punta di dolore al cuore, solo le buone qualità. Levin apparteneva a queste ultime persone. Ma a lui non fu difficile trovare il lato buono e attraente di Vronskij; questo anzi gli saltò subito agli occhi. Vronskij era di statura media, ma di costituzione forte, bruno, con un viso simpatico e bello, straordinariamente calmo e deciso. Nel viso e nella persona di lui, dai capelli neri dal taglio corto e dal mento rasato di fresco fino all'uniforme ampia e nuova fiammante, tutto era semplice e nello stesso tempo elegante. Cedendo il passo alla signora che entrava, Vronskij si avvicinò alla principessa e poi a Kitty. 

Nel momento in cui si avvicinò a lei, i suoi begli occhi brillarono di una particolare tenerezza e con un impercettibile sorriso felice di trionfo discreto (così parve a Levin), chinandosi con rispetto verso di lei, le tese la mano non grande, ma larga. 

Dopo aver salutato tutti e dopo aver detto qualche parola, sedette senza guardare neppure una volta Levin che non staccava gli occhi da lui. 

- Permettete che vi presenti - disse la principessa indicando Levin. - Konstantin Dmitric Levin. Il conte Aleksej Kirillovic Vronskij. 

Vronskij si alzò e, guardando cordialmente Levin negli occhi, gli strinse la mano. 

- Questo inverno dovevo pranzare con voi, mi pare - disse, sorridendo del suo semplice e aperto sorriso. - Ma voi partiste all'improvviso per la campagna. 

- Konstantin Dmitric disprezza e odia la città e tutti noi cittadini - disse la contessa Nordston. 

- Si vede proprio che le mie parole vi hanno fatto effetto, per ricordarle tanto - disse Levin; ma pensando di averlo già detto prima, arrossì. 

Vronskij guardò Levin e la contessa Nordston e sorrise. 

- E voi siete sempre in campagna? - chiese. - Ci si annoia, penso, d'inverno. 

- No, non ci si annoia, quando si ha da fare; e poi anche a star da soli con se stessi non ci si annoia - rispose aspro Levin. 

- A me piace la campagna - disse Vronskij, avendo notato, ma fingendo di non rilevare, il tono di Levin. 

- Ma spero, conte, che non acconsentiate a vivere sempre in campagna - disse la contessa Nordston. 

- Non so, non ho mai provato a lungo. Ho provato un sentimento strano però - soggiunse. - Non ho mai provato tanta nostalgia per la campagna, per la campagna russa con i lapti e con i muziki, come dopo aver vissuto un inverno intero a Nizza con mia madre. Nizza di per sé è noiosa e anche Napoli, Sorrento, sono belle solo per poco tempo. E proprio là ci si ricorda intensamente della Russia e in particolare della campagna russa. Esse sono quasi come\ldots{} 

Egli parlava rivolto a Kitty e a Levin, passando dall'una all'altro il suo tranquillo sguardo cordiale; diceva, evidentemente, quel che gli veniva in testa. 

Avendo notato che la contessa Nordston voleva dire qualcosa, non finì ciò che aveva cominciato e prese ad ascoltarla attentamente. 

La conversazione non venne meno neppure un attimo, così che la vecchia principessa che aveva sempre di riserva, in caso fossero venuti a mancare gli argomenti, i due pezzi forti dell'istruzione classica o tecnica e del servizio militare obbligatorio, non ebbe bisogno di tirarli fuori, e la Nordston non ebbe modo di stuzzicare Levin. 

Levin avrebbe voluto entrare nella conversazione generale, ma non gli riusciva; e dicendo a se stesso ogni minuto: ``è ora d'andar via'' non se ne andava, come aspettando qualcosa. 

La conversazione si era orientata intanto verso i tavoli che girano e gli spiriti, e la Nordston, che credeva allo spiritismo, cominciò a raccontare i prodigi che aveva visto. 

- Ah, contessa, portatemi ad ogni costo, per amor di Dio, portatemi da loro! Io non ho mai visto nulla di straordinario, pur cercandolo dappertutto - disse sorridendo Vronskij. 

- Va bene, per sabato prossimo - rispose la contessa Nordston. - Ma voi, Konstantin Dmitric, ci credete? - chiese a Levin. 

- Perché me lo chiedete? Sapete già la mia risposta. 

- No, io voglio sentire la vostra opinione. 

- La mia opinione è semplicemente questa - rispose Levin - che questi tavolini che girano dimostrano che la cosiddetta società colta non è al di sopra dei contadini. Questi credono al malocchio, alla fattura e ai sortilegi, e noi\ldots{} 

- Dunque, non ci credete? 

- Non posso crederci, contessa. 

- Ma se ho visto con i miei occhi? 

- Anche le contadine raccontano di aver veduto con i loro occhi gli spiriti. 

- Così voi pensate che io non dica il vero. 

E rise senza allegria. 

- Ma no, Maša, Konstantin Dmitric dice che non ci può credere - disse Kitty, arrossendo per Levin, e questi lo capì e, irritatosi ancor più, voleva rispondere; ma Vronskij col suo sorriso aperto, cordiale, venne subito in aiuto della conversazione che minacciava di farsi spiacevole. 

- Voi non ne ammettete per nulla la possibilità? - chiese. - Perché mai? Noi ammettiamo l'esistenza dell'elettricità che pure non conosciamo; perché allora non potrebbe esistere una nuova forza ancora sconosciuta, e noi che\ldots{} 

- Quando fu scoperta l'elettricità - interruppe pronto Levin - fu scoperto soltanto un fenomeno, e non si sapeva da che cosa derivasse e che cosa producesse; e passarono secoli prima che si pensasse alla sua applicazione. Gli spiritisti, invece, hanno cominciato dalla constatazione che i tavolini scrivono e che gli spiriti vanno loro a far visita, e solo dopo si son messi a parlare di una certa forza sconosciuta. 

Vronskij ascoltava, come del resto ascoltava tutti sempre, attentamente Levin, interessandosi alle sue parole. 

- Sì, ma gli spiritisti dicono: noi non sappiamo qual forza sia questa, ma è una forza, ed ecco in quali condizioni agisce. E che gli scienziati scoprano in che cosa consiste questa forza. No, io non vedo perché questa non possa essere una nuova forza, se essa\ldots{} 

- E perché - interruppe Levin - nel campo dell'elettricità, ogniqualvolta sfregate della resina contro della lana, si manifesta un determinato fenomeno; mentre qui non sempre si manifesta, dunque non si tratta di un fenomeno naturale. 

Vronskij, accorgendosi che la conversazione stava per prendere un tono troppo serio per un salotto, non replicò e, cercando di mutar argomento, sorrise allegramente e si rivolse alle signore. 

- Su, proviamo subito, contessa - cominciò; ma Levin voleva finire di esporre quello che pensava. 

- Io penso - continuò - che questo sistema degli spiritisti di spiegare i loro prodigi con la trovata della forza nuova, sia quanto mai infelice. Essi parlano arditamente di forza spirituale e vogliono poi sottoporre questa a un'esperienza materiale. 

Tutti aspettavano che egli smettesse di parlare e lui lo sentiva. 

- Ma io penso che sareste un ottimo medium - disse la contessa Nordston - in voi c'è un certo che di esaltato. 

Levin aprì la bocca, volle dire qualcosa, arrossì e tacque. 

- Su, principessina, proviamo i tavoli, per favore - disse Vronskij. - Voi permettete, principessa? 

E Vronskij cominciò a cercar con gli occhi un tavolino. 

Kitty si alzò dal tavolo e, nel passare accanto a Levin, i suoi occhi si incontrarono con quelli di lui. Con tutta l'anima ne aveva pena, tanto più che era lei la causa della sua infelicità. ``Se mi si può perdonare, perdonatemi - diceva il suo sguardo. - Sono così felice''. 

``Odio tutti, e voi e me stesso'' rispondeva lo sguardo di Levin; e proprio in quel momento egli afferrò il cappello. Ma non era destino che dovesse andar via. Mentre gli altri volevano disporsi attorno al tavolino e Levin cercava di andarsene, entrò il principe e, salutate le signore, si rivolse a Levin. 

- Ah - cominciò festoso - è un pezzo che sei qua? Non lo sapevo neppure. Son contento di vedervi, molto. 

Il vecchio principe parlava a Levin a volte col tu, a volte col voi. Lo abbracciò e, parlando con lui, non si accorse di Vronskij che s'era alzato e aspettava tranquillamente che il principe gli rivolgesse la parola. 

Kitty sentiva che, dopo quello che era successo, l'espansione del padre doveva riuscire penosa a Levin. Ma notò pure con quanta freddezza suo padre rispondesse finalmente all'inchino di Vronskij, e come Vronskij guardasse il principe con affettuosa perplessità, cercando di capire come e perché si potesse essere maldisposti verso di lui. E arrossì. 

- Principe, lasciateci Konstantin Dmitric, - disse la contessa Nordston. - Vogliamo fare una prova. 

- Quale prova? Quella di far girare i tavolini? Su, scusatemi, signore e signori, ma per me è più divertente giocare all'anellino - disse il vecchio principe, guardando Vronskij e indovinando che era stato lui a organizzare la cosa. - Nell'anellino, ancora ancora, c'è un certo senso. 

Vronskij guardò con sorpresa il principe coi suoi occhi fermi e, sorridendo appena, cominciò subito a parlare con la Nordston del grande ballo della settimana seguente. 

- Spero che ci verrete - disse rivolto a Kitty. 

Non appena il vecchio principe si allontanò da lui, Levin uscì inosservato, riportando, quale ultima impressione della serata, il viso sorridente e felice di Kitty che rispondeva alla domanda di Vronskij a proposito del ballo. 

\capitolo{XV}Quando la serata fu finita, Kitty raccontò alla madre il suo colloquio con Levin, e, malgrado la pena che provava per lui, la rallegrava l'idea di aver avuto una ``domanda di matrimonio''. Non aveva nessun dubbio di non essersi regolata così come conveniva. Ma a letto, per molto tempo, non poté prendere sonno. Un'unica immagine la perseguitava ostinata. Era il viso di Levin con le sopracciglia aggrottate e gli occhi buoni che guardavano di sotto in su, scoraggiati e tristi, mentre, in piedi, ascoltava suo padre e guardava lei e Vronskij. E provò tanta pena per lui che le vennero le lacrime agli occhi. Ma allora pensò subito a quegli col quale lo aveva cambiato. Ricordò con vivezza il viso maschio di lui, la calma dignitosa e la benevolenza che emanavano in ogni suo gesto verso tutti; ricordò l'amore per lei dell'uomo che amava e la gioia le tornò nell'animo e con un sorriso di felicità poggiò la testa sul guanciale. ``Che pena, che pena, ma che farci? La colpa non è mia'' si andava dicendo; eppure una voce interiore le diceva il contrario. Di che cosa provasse rimorso - d'aver attratto a sé Levin o di averlo respinto - non sapeva. Ma la felicità sua era avvelenata dal dubbio. ``Signore abbi pietà, Signore abbi pietà'' diceva fra sé e sé finché si addormentò. 

Intanto giù, nello studio del principe, si svolgeva una di quelle scenate frequenti fra i genitori, a proposito della figlia preferita. 

- Ecco, ecco cosa c'è - gridava il principe agitando le braccia e incrociando subito i risvolti della vestaglia di vaio. - C'è che voi non avete né orgoglio né dignità, c'è che disonorate, rovinate la figliuola con questo stupido e indegno modo di cercarle marito. 

- Ma abbiate pazienza, per amor di Dio, principe, che ho fatto mai? - diceva la principessa, quasi piangendo. 

Dopo la conversazione con la figlia, era venuta dal principe a salutarlo, felice e soddisfatta e, pur non avendo intenzione di parlargli della proposta di Levin e del rifiuto di Kitty, aveva accennato al marito la faccenda di Vronskij che le sembrava del tutto definita, non appena fosse arrivata la madre di lui. E proprio a questo punto il principe aveva preso fuoco, e si era messo a gridare parole sconvenienti. 

- Che cosa avete fatto? Ecco cosa: in primo luogo avete adescato un giovanotto e tutta Mosca ne parlerà; e a ragione. Se volete dare una serata, invitate pure chi volete, ma non questi fidanzatelli prescelti. Invitateli pure tutti questi moscardini - così il principe chiamava i giovani brillanti di Mosca - chiamate pure uno strimpellatore e fate pure ballare, ma non mi mettete insieme, come avete fatto questa sera, tutti questi fidanzatelli. A me veder questo, fa schifo, schifo, e ci siete riuscita voi a far girar la testa alla ragazza. Levin è mille volte migliore. Questo invece è un cascamorto di Pietroburgo; li fanno a macchina questi elegantoni, son tutti d'uno stampo, e son tutti\ldots{} brodaglia. E fosse anche un principe di sangue, mia figlia non ha bisogno di nessuno! 

- Ma che cosa ho mai fatto io? 

- Questo, questo\ldots{} - gridò con rabbia il principe. 

- Lo so che a dar retta a te - interruppe la principessa - noi non dovremmo mai dar marito a nostra figlia. Ma se è così, meglio allora ritirarsi in campagna. 

- Eh sì che è meglio là. 

- Ma dimmi, che forse sono io che li adesco? Io non li attiro per nulla. Ma se un giovane, un giovane che ha tutte le qualità, s'innamora, e lei mi pare\ldots{} 

- Sì, ecco, vi pare! E se lei per l'appunto si innamorasse e lui pensasse a sposarsi tanto quanto me? Oh, che non lo vedano i miei occhi!\ldots{} ``Ah, lo spiritismo, ah, Nizza, ah, il ballo!''. - E il principe, immaginando di rifare il verso a sua moglie, faceva una riverenza ad ogni parola. 

- Ecco, quando avremo fatta l'infelicità di Katen'ka, quando si sarà davvero messa in testa\ldots{} 

- Ma perché lo pensi? 

- Io non lo penso, lo so; per questo noi uomini abbiamo gli occhi per vedere e non così le donnicciuole. Io vedo, da una parte, un uomo che ha intenzioni serie, Levin; e dall'altro un gallinaccio fanfarone come questo qua, che vuole soltanto divertirsi. 

- Eh già, ormai ti sei messo in testa certe cose\ldots{} 

- Ecco, te lo ricorderai, ma tardi, come è stato per Dašen'ka. 

- Su, va bene, non ne parliamo più - lo fermò la principessa, ricordandosi di Dolly infelice. 

- E va bene, addio! 

E fattisi scambievolmente la croce e baciatisi, i coniugi si separarono, sentendo, però, che ognuno era rimasto nella propria convinzione. 

La principessa, che prima era fermamente convinta che quella serata avrebbe deciso la sorte di Kitty e che non si dovevano avere più dubbi sulle intenzioni di Vronskij, era in questo momento turbata dalle parole del marito. E tornata in camera sua, proprio alla stessa maniera di Kitty, col terrore di un avvenire così incerto, ripeté parecchie volte in cuor suo:``Signore abbi pietà, Signore abbi pietà, Signore abbi pietà!''. 

\capitolo{XVI}Vronskij non aveva conosciuto mai la vita di famiglia. Sua madre in gioventù era stata una brillante donna di mondo, e aveva avuto, durante la sua vita coniugale, e specialmente dopo, molte avventure note a tutta la società. Di suo padre quasi non si ricordava, ed egli stesso era stato educato al corpo dei paggi. 

Uscito giovanissimo dalla scuola, brillante ufficiale, si era trovato subito nella carreggiata comune a tutti i facoltosi ufficiali di Pietroburgo. Sebbene frequentasse di tanto in tanto la società pietroburghese, i suoi interessi amorosi ne erano tutti al di fuori. Dopo la vita di Pietroburgo, lussuosa e dissoluta, a Mosca aveva provato per la prima volta l'incanto di avvicinarsi ad una graziosa ed ignara fanciulla della società, la quale aveva preso ad amarlo. Non gli era venuto neppure in mente che potesse esserci qualcosa di poco onesto nei suoi rapporti con Kitty. Nelle feste ballava soprattutto con lei, ne frequentava la casa. Le diceva quello che comunemente si dice in società: una sciocchezza qualsiasi, alla quale, senza volere, dava un significato particolare per lei. Tuttavia, pur non dicendo nulla che non fosse conveniente dire in presenza di tutti, avvertiva ch'ella sempre più subiva il suo fascino, e più egli s'accorgeva di questo più se ne compiaceva, e il suo affetto per lei diveniva sempre più tenero. Non sapeva che questo suo modo di agire nei riguardi di Kitty avrebbe potuto chiaramente essere definito un tentativo di adescare una ragazza senza avere alcuna intenzione di sposarla, e che questo adescamento era una delle cattive azioni dei giovani mondani come lui. Gli sembrava d'essere stato il primo a scoprire una simile soddisfazione e godeva della propria scoperta. 

S'egli avesse potuto ascoltare ciò che dicevano i genitori di Kitty quella sera, se egli avesse potuto mettersi dal punto di vista della famiglia e pensare che Kitty sarebbe stata infelice se egli non l'avesse sposata, si sarebbe molto sorpreso e non ci avrebbe creduto. Non avrebbe potuto credere che quello che procurava un piacere così grande e buono a lui e specialmente a lei, potesse essere un male. Ancor meno avrebbe pensato di doversi sposare. 

Il matrimonio non gli si era presentato mai come una possibilità. Non solo non amava la vita di famiglia, ma nella famiglia, e particolarmente nella figura del marito, egli vedeva, secondo l'opinione dell'ambiente di scapoli in cui viveva, qualcosa di estraneo, di ostile, e soprattutto di ridicolo. Ma pur senza sospettare la conversazione dei genitori di Kitty, Vronskij, uscendo quella sera da casa Šcerbackij, sentì che il segreto legame sentimentale che esisteva tra lui e Kitty si era così saldamente rafforzato, ch'egli doveva prendere una decisione. Ma quale precisamente non sapeva immaginare. 

``Anche questo è delizioso - pensava tornando da casa Šcerbackij, riportandone, come sempre, un senso di piacevole purità e freschezza dovuto forse, in parte, al fatto di non aver fumato per tutta la sera; ed insieme a questo un nuovo senso di tenerezza dinanzi all'amore di Kitty. - Anche questo è delizioso, che niente sia stato detto fra me e lei; ma ci siamo talmente intesi in quella invisibile conversazione fatta di sguardi e di toni di voce che oggi, in maniera più chiara che mai, ella mi ha detto che mi ama. E così teneramente, con tanta semplicità e soprattutto con fiducia. Io stesso mi sento migliore, più puro. Sento di avere un cuore e che c'è molto di buono in me. Quei cari occhi innamorati! Quando ha detto: e molto\ldots{}E allora? E allora nulla. Io sto bene e lei pure sta bene''. E si mise a pensare dove finir la serata. 

Passò in rassegna tutti i luoghi dove sarebbe potuto andare. ``Al club? Una partita a bazzica, lo champagne con Ignatov? No, non ci vado. Allo Château des fleurs e trovarci Oblonskij, le canzonette e il can can? No, m'è venuto a noia. Ecco, proprio perché mi piacciono gli Šcerbackij è segno che divento migliore. Andrò a casa''. Andò direttamente all'albergo Dussau nella sua camera, si fece servir la cena e, spogliatosi, fece appena in tempo a posar la testa sul guanciale che s'addormentò d'un sonno pesante e tranquillo come sempre. 

\capitolo{XVII}Il giorno dopo, alle undici del mattino, Vronskij andò alla stazione della ferrovia di Pietroburgo a rilevare la madre; e il primo viso in cui si imbatté sui gradini della scalinata principale fu Oblonskij che aspettava la sorella con quello stesso treno. 

- Oh, eccellenza! - gridò Oblonskij - tu qua? a prendere chi? 

- Io? a prendere la mamma - rispose Vronskij, sorridendo come tutti quelli che incontravano Oblonskij, e stringendogli la mano salì con lui la scalinata. - Deve arrivare oggi da Pietroburgo. 

- E io ti ho aspettato fino alle due! Dove sei andato dopo gli Šcerbackij? 

- A casa - rispose Vronskij. - A dir la verità, stavo così bene ieri sera dopo casa Šcerbackij che non ho avuto voglia di andare in nessun altro posto. 

- ``Conosco i cavalli focosi da certi loro segni, conosco i giovani innamorati dagli occhi'' - declamò Stepan Arkad'ic, proprio come aveva detto il giorno prima a Levin. 

Vronskij sorrise con l'aria di non negare, ma subito cambiò discorso. 

- E tu chi aspetti? - domandò. 

- Io? Una bella donna - disse Oblonskij. 

- Bene! 

- Honny soit qui mal y pense! Mia sorella Anna. 

- Ah, la Karenina. 

- La conosci, vero? 

- Mi pare di conoscerla. Forse no. A dire il vero, non ricordo - rispondeva distrattamente Vronskij, immaginandosi al nome di Karenina qualcosa di borioso e noioso. 

- Ma Aleksej Aleksandrovic, il mio famoso cognato, lo conosci probabilmente. Tutti lo conoscono. 

- Lo conosco infatti di fama e di vista. So che è molto intelligente, uno scienziato, qualcosa di superno\ldots{} Ma tu lo sai, questo non rientra nella mia\ldots{} not in my line - disse Vronskij. 

- Già, è un uomo molto interessante; un po' conservatore, ma una brava persona. 

- Be', tanto meglio per lui - disse Vronskij sorridendo. - Ah, tu sei qui - disse rivolto al servitore della madre, un vecchio di alta statura, che stava accanto alla porta. - Entra qua. 

Vronskij, oltre la simpatia che aveva, come tutti avevano, per Stepan Arkad'ic, si sentiva legato a lui in quell'ultimo tempo per il fatto che in mente sua lo associava a Kitty. 

- Ebbene, domenica, facciamo il pranzo per la diva? - gli disse prendendolo sotto braccio con un sorriso. - Io raccoglierò le quote. Ah, ieri hai conosciuto il mio amico Levin? - chiese Stepan Arkad'ic. 

- E come! Ma è andato via un po' presto. 

- È un caro ragazzo - continuò Oblonskij - non è vero? 

- Io non capisco - rispose Vronskij - perché in tutti i moscoviti, esclusi naturalmente quelli con cui parlo - intercalò scherzosamente - vi sia qualcosa di duro. Non so perché si inalberano sempre, si arrabbiano come se volessero far sempre sentire qualcosa\ldots{} 

- È così, è vero, è\ldots{} - disse ridendo allegramente Stepan Arkad'ic. 

- Arriva presto? - chiese Vronskij a un ferroviere. 

- È già partito dall'ultima stazione - rispose il ferroviere. 

L'avvicinarsi del treno si notava sempre più per il movimento dei preparativi nella stazione, per il correre dei facchini, per l'apparire dei gendarmi e dei ferrovieri e per l'arrivo di coloro che aspettavano. Attraverso la nebbia gelida si vedevano gli operai con le giubbe corte di pelliccia, le scarpe morbide di feltro, che passavano attraverso gli scambi delle curve delle linee. Si udiva il fischio di una locomotiva su rotaie lontane, e l'incedere di qualcosa di pesante. 

- No - disse Stepan Arkad'ic il quale aveva una gran voglia di raccontare a Vronskij le intenzioni di Levin nei riguardi di Kitty. - No, tu non hai apprezzato al giusto punto il mio Levin. È un uomo molto nervoso e a volte antipatico, è vero, ma in compenso è molto caro. È una natura, così onesta, così leale, e ha un cuore d'oro. Ma ieri vi erano delle ragioni particolari - continuò Stepan Arkad'ic con un sorriso d'intesa, dimenticando completamente la sincera simpatia che aveva provato il giorno prima per il suo amico e sentendone ora una simile, solo che per Vronskij. - Sì, vi era una ragione per la quale egli poteva diventare particolarmente felice o particolarmente infelice. 

Vronskij si fermò e chiese franco: 

- Cos'è, cos'è mai? Che forse ieri ha fatto domanda di matrimonio alla tua belle-soeur? 

- Può darsi - disse Stepan Arkad'ic. - M'è parso di capire qualcosa di simile, ieri. Già, se n'è andato via presto ed era anche di cattivo umore, deve essere stato così. È innamorato da tanto tempo e mi fa tanta pena. 

- Eh, già! Io penso, del resto, che lei può aspirare a un partito migliore - disse Vronskij e, raddrizzando il busto, si mise di nuovo a camminare. - Del resto, non lo conosco - soggiunse. - Già, deve essere una situazione penosa. Proprio per questo la maggioranza degli uomini preferisce far conoscenza con le donnine allegre. In questo caso un insuccesso dimostra solo che non hai avuto abbastanza quattrini, nell'altro, invece, è messo in giuoco il tuo onore. Ma ecco il treno. 

Infatti la locomotiva fischiava già. Dopo qualche secondo la piattaforma tremò e, sbuffando del vapore appesantito dal gelo, la locomotiva avanzò con lo stantuffo che si piegava e si distendeva lentamente e ritmicamente, e con il macchinista tutto imbacuccato e ricoperto di brina che salutava; e poi dietro al tender, scotendo sempre più lentamente e sempre più forte la banchina, passò il bagagliaio con un cane che guaiva; ed infine, traballando prima di fermarsi, avanzarono le carrozze dei passeggeri. 

Un capotreno aitante, fischiando, saltò giù mentre il treno era ancora in corsa, e dietro di lui cominciarono a scendere, uno ad uno, i viaggiatori impazienti: un ufficiale della guardia che si teneva dritto e guardava severamente attorno a sé, un piccolo mercante inquieto che sorrideva allegramente tenendo in mano una borsa, un contadino con un sacco sulle spalle. 

Vronskij, dritto accanto a Oblonskij, guardava le vetture e quelli che ne venivano fuori, e s'era completamente scordato di sua madre. Quello che aveva saputo proprio allora di Kitty lo eccitava e rallegrava. Il suo petto involontariamente si raddrizzava e gli occhi gli brillavano. Si sentiva vincitore. 

- La contessa Vronskaja è in questo scompartimento - disse il capotreno aitante, accostandosi a Vronskij. 

Le parole del capotreno lo scossero e lo costrinsero a ricordarsi della madre e dell'imminente incontro con lei. Egli, in fondo, non stimava sua madre e, senza rendersene conto, non l'amava neppure, sebbene, per l'ambiente in cui viveva e per la propria educazione, non sapeva immaginare altri rapporti verso di lei che quelli propriamente sottomessi e rispettosi, anzi tanto più sottomessi e rispettosi quanto meno intimamente la stimava ed amava. 

\capitolo{XVIII}Vronskij entrò nella vettura dietro al capotreno e all'ingresso dello scompartimento si fermò per cedere il passo a una signora che ne usciva. Con l'intuito abituale dell'uomo di mondo, Vronskij ne rilevò l'appartenenza al gran mondo. Si scusò e stava per entrare, quando provò il bisogno di guardarla ancora una volta non perché era molto bella, non per quella eleganza e quella grazia modesta che apparivano da tutta la sua figura, ma perché nell'espressione piacente del viso, quando gli era passata accanto, c'era qualcosa di affettuoso e di dolce. Nel momento in cui si era voltato a guardarla, ella pure aveva girato il capo. I suoi occhi grigi, luminosi, che sembravano scuri per le sopracciglia folte, si fermarono attenti con un'espressione amichevole sul viso di lui, come se lo riconoscessero, e subito si portarono sulla folla che si avvicinava, cercando qualcuno. In questo breve sguardo Vronskij riuscì a notare una vivacità contenuta che le errava sul viso e balenava tra gli occhi lucenti e un riso appena percettibile che increspava le labbra vermiglie. Come se qualcosa di esuberante colmasse tanto il suo essere da esprimersi contro il suo volere, ora nella luce degli occhi, ora nel riso. Ella aveva deliberatamente attutito la luce degli occhi, ma questa luce, contro il suo volere, si era illuminata nel riso appena percettibile. 

Vronskij entrò nello scompartimento. Sua madre, una vecchietta asciutta dai riccioli e dagli occhi neri, socchiudeva le palpebre guardando il figlio e sorrideva lieve con le labbra sottili. Alzatasi dal sedile e porgendo la borsetta alla cameriera, tese la piccola mano asciutta al figlio e, sollevandogli la testa dalla mano, lo baciò. 

- Hai ricevuto il telegramma? Stai bene? Sia lodato Iddio. 

- Avete fatto buon viaggio? - disse il figlio sedendosi accanto a lei e prestando involontariamente ascolto alla voce femminile che gli giungeva da dietro la porta. Egli sapeva che era la voce della signora che aveva incontrato nell'entrare. 

- Io non sono d'accordo con voi - diceva la voce della signora. 

- È il punto di vista pietroburghese, signora. 

- Non pietroburghese, ma semplicemente femminile - rispondeva lei. 

- Permettetemi di baciare la vostra piccola mano. 

- A rivederci, Ivan Petrovic. E guardate se mio fratello è qui, e mandatemelo - disse la signora proprio sulla porta, ed entrò di nuovo nello scompartimento. 

- Ebbene, avete trovato vostro fratello? - disse la Vronskaja rivolgendosi alla signora. 

Vronskij allora si ricordò che era la Karenina. 

- Vostro fratello è qui - disse alzandosi in piedi. - Perdonatemi, non vi ho riconosciuto; ma già, la nostra conoscenza è stata così breve - disse Vronskij inchinandosi - che probabilmente voi non vi ricordate di me. 

- Oh, no - disse lei - vi avrei riconosciuto, perché con vostra madre, per tutto il viaggio, mi pare, abbiamo parlato soltanto di voi - disse, permettendo infine a quella vivacità che le urgeva di esprimersi nel riso. - Ma com'è che mio fratello non viene? 

- Va' a chiamarlo, Alëša - disse la vecchia contessa. 

Vronskij uscì sulla piattaforma e gridò: 

- Oblonskij, qui! 

Ma la Karenina non aspettò che il fratello si avvicinasse e, non appena lo vide, col suo passo leggero e deciso scese subito dalla vettura. E non appena il fratello le fu dappresso, con un movimento che stupì Vronskij per la grazia e la prontezza, circondò con il braccio sinistro il collo di Oblonskij, l'attirò a sé con mossa rapida e lo baciò forte. 

Vronskij, senza staccare gli occhi da lei, l'osservava e, senza saper lui stesso perché, sorrideva. Ma, ricordatosi che la madre aspettava, montò in vettura. 

- Non è vero che è molto carina? - disse la contessa. - Il marito l'ha fatta sedere qui accanto a me e io ne sono stata molto contenta. Abbiamo parlato tutto il viaggio. E ora, su, a te, mi si dice\ldots{}vous filez le parfait amour. Tant mieux, mon cher, tant mieux. 

- Io non so a che cosa alludiate, maman - rispose freddo il figlio. - Dunque, maman, andiamo. 

La Karenina entrò di nuovo nello scompartimento per salutare la contessa. 

- Ed eccoci qua, contessa; voi avete trovato vostro figlio e io mio fratello - disse gaia. - E così tutte le mie storie si sono esaurite; forse più avanti non ci sarebbe stato più nulla da raccontare. 

- Eh, no - disse la contessa prendendole una mano - io con voi farei il giro del mondo e non mi annoierei mai. Voi siete una di quelle donne gentili con le quali è piacevole parlare e tacere. E non vi preoccupate di vostro figlio, vi prego; è impossibile non separarsene mai. 

La Karenina stava immobile, mantenendosi ben dritta e i suoi occhi ridevano. 

- Anna Arkad'evna - disse la contessa spiegando al figlio - ha un bimbo di otto anni, mi pare, e non s'è mai staccata da lui, e ora si tormenta d'averlo lasciato. 

- Già, con la contessa abbiamo parlato tutto il tempo io del mio e lei del suo figliuolo\ldots{} - disse la Karenina e di nuovo il riso le illuminò il volto, un riso carezzevole che riguardava lui. 

- Probabilmente questo vi avrà annoiato - disse lui afferrando al volo la pallina di civetteria ch'ella gli aveva lanciato. Ma ella evidentemente non voleva proseguire la conversazione su questo tono e si rivolse alla vecchia contessa. 

- Vi ringrazio molto. Non mi sono neppure accorta come ho passato la giornata di ieri. A rivederci, contessa. 

- Addio, mia piccola amica - rispose la contessa. - Fatemi baciare il vostro bel visino. Vi dico così, semplicemente, da vecchia, che sono innamorata di voi. 

Per quanto usuale fosse questa frase la Karenina evidentemente ci credette di cuore, e se ne rallegrò. Arrossì, si chinò leggermente porgendo il viso alle labbra della contessa, si raddrizzò, e sempre con quel riso che le balenava fra le labbra e gli occhi, dette la mano a Vronskij. Egli strinse la piccola mano offertagli e si rallegrò come di una cosa particolare per quella stretta energica con la quale ella scosse ardita e forte la sua mano. Ella uscì col passo svelto che portava con così strana leggerezza il corpo assai pieno. 

- È molto carina - disse la vecchia signora. 

La stessa cosa pensava il figlio. Egli accompagnò con lo sguardo la graziosa figura finché non sparve e il sorriso gli rimase sul volto. Dal finestrino la vide accostarsi al fratello, mettergli una mano sul braccio e cominciare a parlargli con animazione di qualcosa che evidentemente non aveva nulla in comune con lui, Vronskij, e questo gli dette fastidio. 

- E allora, maman, state proprio bene? - ripeté lui volgendosi alla madre. 

- Bene, benissimo. Alexandre è stato molto gentile. E Marie è diventata bella. È molto interessante. 

E prese a raccontare quello che più di tutto le interessava: il battesimo del nipote per cui era andata a Pietroburgo, e la particolare benevolenza dello zar verso il figlio maggiore. 

- Ecco anche Lavrentij - disse Vronskij guardando dal finestrino. - Ora andiamo, se non vi spiace. 

Il vecchio maggiordomo che aveva viaggiato con la contessa venne a dire che tutto era pronto e la contessa si alzò per andare. 

- Andiamo, ora c'è poca gente - disse Vronskij. 

La cameriera afferrò una sacca e il cagnolino, il maggiordomo e un facchino presero le valigie. Vronskij offrì il braccio alla madre; ma mentre uscivano dalla vettura, a un tratto alcune persone dal viso spaventato passarono vicino correndo. Passò anche il capostazione col berretto dal colore vivace. Doveva essere successo qualcosa d'eccezionale. La gente del treno correva in senso inverso. 

- Cos'è? Cos'è? S'è gettato sotto! L'ha schiacciato!\ldots{} - si sentiva dire fra quelli che passavano. 

Stepan Arkad'ic e la sorella ch'egli aveva al braccio, anche loro coi visi spaventati, tornarono indietro e si fermarono accanto alla vettura. 

Le signore vi salirono, mentre Vronskij e Stepan Arkad'ic seguirono la folla per informarsi dei particolari della disgrazia. 

Un guardiano, forse ubriaco o forse troppo imbacuccato per il gran gelo, non aveva sentito il treno che retrocedeva ed era rimasto schiacciato. 

Ancor prima che Vronskij e Oblonskij fossero tornati, le signore avevano saputo tutti i particolari dal maggiordomo. 

Oblonskij e Vronskij avevano tutti e due visto il corpo deformato. Oblonskij soffriva visibilmente. Corrugava la fronte e sembrava stesse per piangere. 

- Ah, che orrore! Oh, Anna, se avessi visto! Ah, che orrore! - esclamava. 

Vronskij taceva e il suo bel viso era serio, ma perfettamente tranquillo. 

- Ah, se aveste visto, contessa - diceva Stepan Arkad'ic. - E la moglie è qui\ldots{} È uno strazio a vederla. S'è gettata sul corpo. Dicono che era lui solo a dar da mangiare a una famiglia enorme. Che orrore! 

- Non si può fare qualcosa per lei? - disse la Karenina con un bisbiglio agitato. 

Vronskij la guardò e uscì dallo scompartimento. 

- Vengo subito, maman - aggiunse, voltandosi indietro sulla porta. 

Quando rientrò, dopo pochi minuti, Stepan Arkad'ic parlava già con la contessa di una nuova cantante, ma la contessa guardava impaziente verso la porta in attesa del figlio. 

- Ora andiamo - disse Vronskij entrando. 

Uscirono insieme. Vronskij andava avanti con la madre. Dietro venivano la Karenina e il fratello. All'uscita, raggiuntolo, il capostazione si avvicinò a Vronskij. 

- Voi avete consegnato duecento rubli al mio aiutante. Vogliate precisare a chi li destinate. 

- Alla vedova - disse Vronskij alzando le spalle. - Non capisco che bisogno ci sia di chiederlo. 

- Li avete dati voi? - gridò da dietro Oblonskij e, stretto il braccio alla sorella, aggiunse: - Che caro, che caro! Non è vero che è un gran bravo ragazzo? I miei rispetti contessa. 

E lui e la sorella si fermarono alla ricerca della cameriera. 

Quando uscirono, la carrozza dei Vronskij era già andata via. Le persone che entravano parlavano ancora fra di loro di quello che era accaduto. 

- Ecco una morte terribile! - diceva un signore passando accanto. - Dicono che sia stato fatto in due pezzi. 

- Io penso invece che sia la migliore: in un attimo - osservò un altro. 

- Ma come, non prendono delle misure di sicurezza? - diceva un terzo. 

La Karenina sedette nella carrozza e Stepan Arkad'ic si accorse con sorpresa che le labbra le tremavano e che a stento tratteneva le lacrime. 

- Che c'è, Anna? - chiese quando si furono allontanati di un centinaio di sazeni. 

- Un cattivo presagio - disse lei. 

- Sciocchezze! - disse Stepan Arkad'ic. - Tu sei arrivata, questo è l'importante. Tu non puoi immaginare come io speri in te. 

- È molto che conosci Vronskij? - chiese lei. 

- Sì, forse lo sai, noi speriamo che sposi Kitty. 

- Sì? - disse piano Anna. - Suvvia, dimmi ora di te - aggiunse, scotendo la testa come per scacciar via materialmente qualcosa di superfluo e di fastidioso. - Dimmi delle tue cose. Ho avuto la lettera ed eccomi qua. 

- Sì, ogni speranza è in te - disse Stepan Arkad'ic. 

- Su, raccontami tutto. 

E Stepan Arkad'ic prese a raccontare. 

Giunti a casa, Oblonskij fece scendere la sorella, sospirò, le dette la mano e si diresse in ufficio. 

\capitolo{XIX}Quando Anna entrò nella stanza, Dolly stava nel salottino con un bimbo biondo e paffuto che fin d'ora assomigliava al padre, e gli risentiva la lezione di lettura francese. Il bambino leggeva, rigirandosi in mano e cercando di strappare al giubbotto un bottone che appena appena si reggeva. La madre aveva varie volte allontanato quella mano, ma la manina grassoccia tornava di nuovo al bottone. La madre alla fine staccò il bottone e se lo mise in tasca. 

- Fermo con le mai, Griša - disse, e si mise di nuovo alla coperta, suo vecchio lavoro al quale attendeva sempre nei momenti penosi e che ora eseguiva nervosamente, intrecciando il filo con le dita e contando le maglie. Benché avesse fatto dire al marito, il giorno prima, che l'arrivo della sorella non la riguardava, aveva preparato tutto per riceverla e aspettava con ansia la cognata. 

Dolly era schiantata dal dolore, ne era tutta divorata. Ma ricordava che Anna era la moglie di uno dei personaggi più importanti di Pietroburgo e una grande dame pietroburghese. E per questo, contrariamente a quello che aveva fatto dire al marito, non aveva dimenticato che sarebbe arrivata la cognata. ``Poi, in fondo, Anna non ha nessuna colpa - pensava. - Io non so altro di lei se non quanto si può dir di meglio, e nei miei riguardi ne ho sempre ricevuto affetto ed amicizia''. Però, per quanto ricordava, l'impressione da lei riportata a Pietroburgo dei Karenin, non era stata favorevole: non le era piaciuta la loro casa; c'era qualcosa di falso in quell'ambiente di vita familiare. ``Ma perché mai non riceverla? Che non le venga in mente di consolarmi, però! - pensava Dolly. - Tutte le consolazioni, le esortazioni e i perdoni, tutto questo l'ho già pensato e ripensato mille volte, e non serve a nulla''. 

Tutti quei giorni Dolly era stata sola coi bambini. Parlare della sua pena non voleva, e con quel dolore nel cuore parlare di cose indifferenti non le riusciva. Sapeva che in un modo o nell'altro avrebbe detto tutto ad Anna; e ora la rallegrava il pensiero di come l'avrebbe detto, ora l'irritava quel bisogno di raccontare la propria umiliazione a lei, sorella del marito, e sentirne frasi fatte di esortazione e di conforto. 

L'aspettava guardando l'orologio ogni momento, ma, come spesso accade, le sfuggì proprio quello in cui l'ospite giunse, così che non sentì il campanello. 

Udito il fruscio di vesti e di passi lievi già sulla porta, si voltò e sul viso tormentato si espresse involontariamente non la gioia, ma la sorpresa. Si alzò e abbracciò la cognata. 

- Come, già qui? - disse baciandola. 

- Dolly, come sono contenta di vederti! 

- Anch'io sono contenta - disse Dolly, sorridendo debolmente e cercando di indovinare dall'espressione del viso di Anna se sapeva o no. ``Probabilmente sa'' pensò, notando una certa compassione sul viso di Anna. - Su, andiamo, ti accompagno in camera tua - continuò, cercando di allontanare, per quanto possibile, il momento della spiegazione. 

- Questo è Griša? Dio, com'è cresciuto! - disse Anna e, baciatolo, senza staccare gli occhi da Dolly, si fermò e arrossì. - No, permettimi di restare qui. 

Si tolse lo scialle, il cappello e, avendovi impigliato una ciocca di capelli neri inanellati, scotendo la testa, liberò la capigliatura. 

- Come splendi di felicità e di salute! - disse Dolly quasi con invidia. 

- Io? Sì - disse Anna. - Dio mio, Tanja! La coetanea del mio Serëza - aggiunse rivolta alla bambina che era entrata di corsa. La prese in collo e la baciò. - Una bimba deliziosa! un amore! Fammeli vedere tutti. 

Nominava e ricordava non soltanto i nomi, ma gli anni, i mesi, i caratteri, le malattie di tutti loro, e Dolly non poteva non apprezzare tutto questo. 

- Su, allora, andiamo da loro - disse lei. - Vasja dorme ora, peccato! 

Dopo aver veduto i bambini, sedettero davanti al caffè, ormai sole, nel salotto. 

Anna prese il vassoio, ma poi lo scostò. 

- Dolly - disse - lui mi ha parlato. 

Dolly guardò fredda Anna. Si aspettava ora delle frasi convenzionali di simpatia, ma Anna non disse nulla di simile. 

- Dolly, cara - disse - io non voglio parlarti in suo favore, né consolarti; non si può. Ma ho pena di te, cara, ne ho pena con tutta l'anima! 

Dietro alle ciglia dei suoi occhi comparvero le lacrime. Venne a sedersi più vicina alla cognata e le prese una mano con la sua piccola mano energica. Dolly non si ritrasse, ma il suo viso non mutò l'espressione arida. Disse: 

- Non è possibile consolarmi. Dopo quello che è avvenuto, tutto è perduto, tutto è finito! 

E non appena ebbe detto questo, il viso le si addolcì d'un tratto. Anna sollevò la mano magra di Dolly, la baciò e le disse: 

- Ma, Dolly, che fare, che fare? Quale la via migliore in questa terribile situazione? ecco quello a cui bisogna pensare. 

- Tutto è finito e non c'è più nulla da fare - disse Dolly. - E il peggio è, tu mi capisci, che io non posso abbandonarlo: ci sono i bambini, sono legata. E con lui non posso vivere, è un tormento per me vederlo. 

- Dolly, cara, lui mi ha parlato, ma io voglio sentire da te, dimmi tutto. 

Dolly la guardò interrogativamente. 

Una compassione un affetto sinceri apparivano chiaramente sul viso di Anna. 

- E sia - disse improvvisamente lei. - Ma voglio cominciare dal principio. Tu sai come mi sono sposata. Io, con l'educazione di maman, ero non solo ingenua, ma sciocca. Non sapevo nulla, io. Dicono, lo so, che i mariti raccontino alle mogli la loro vita di prima, ma Stiva\ldots{} - si corresse - Stepan Arkad'ic non mi aveva detto nulla. Tu non ci crederai, ma io fino ad ora credevo di essere la sola donna che egli avesse conosciuto. Così ho vissuto per otto anni. Tu capisci, io non solo non sospettavo un'infedeltà, ma la consideravo impossibile; e allora, figurati, con delle idee simili, venire a sapere improvvisamente tutto l'orrore, tutto il ribrezzo\ldots{} Comprendimi. Essere sicura in pieno della propria felicità e d'un tratto\ldots{} - continuò Dolly, trattenendo i singhiozzi - avere in mano la lettera, la sua lettera per l'amante, la mia governante. No, è troppo terribile! - Trasse fuori in fretta il fazzoletto e si coprì il viso. - Capirei anche un momento di capriccio - continuò, dopo una pausa. - Ma ingannarmi così meditatamente, con tanta astuzia\ldots{} E con chi? Continuare ad essere mio marito e nello stesso tempo con lei\ldots{} questo è orribile! Tu non puoi capire\ldots{} 

- Oh, no, capisco. Capisco, cara Dolly, capisco\ldots{} - diceva Anna, stringendole la mano. 

- E tu pensi ch'egli senta tutto l'orrore della mia posizione? - proseguì Dolly. - Per nulla! Lui è felice e soddisfatto. 

- Oh, no - interruppe in fretta Anna. - Fa pena, è distrutto dal rimorso. 

- E che forse è capace di rimorso? - interruppe Dolly, guardando attenta il viso della cognata. 

- Sì, io lo conosco. Non potevo guardarlo senza provarne pena. Noi lo conosciamo tutte e due. È buono, ma è orgoglioso, e ora è così umiliato. E poi quello che soprattutto mi ha commosso\ldots{} - e qui Anna indovinò quello che poteva commuovere Dolly - è che lo tormentano due cose: si vergogna dei bambini, e amandoti\ldots{} sì, sì, amandoti più di tutto al mondo - disse, interrompendo in fretta Dolly che voleva ribattere - ti ha fatto del male, ti ha uccisa. ``No, no, non mi perdonerà'' dice continuamente. 

Dolly guardava pensosa al di là della cognata, ascoltando le sue parole. 

- Sì, capisco come la sua situazione sia orribile; peggio per il colpevole che per l'innocente - disse - se sente che dalla colpa sua deriva tutto il male. Ma come perdonare, come posso essere di nuovo sua moglie dopo di lei? Per me vivere con lui sarebbe un tormento, proprio perché mi è così caro l'amore che ho avuto per lui. 

E i singhiozzi spezzarono le sue parole. 

Ma poi, come apposta, ogni volta che si raddolciva, riprendeva a parlare di ciò che la irritava. 

- Quella lì è giovane, è bella - continuò. - Ma tu capisci, Anna, da chi sono state prese la mia gioventù, la mia bellezza? da lui e dai suoi figli. Ora ho finito di servirgli, e in questo servizio ho dato tutta me stessa; ora, s'intende, gli è più gradita una persona fresca e volgare. Probabilmente, parlavano di me fra di loro, o peggio ancora, non ne parlavano proprio, capisci? - I suoi occhi si accesero di nuovo di rancore. - E poi, dopo tutto questo, mi dirà\ldots{} Come potergli credere? Mai. No; ormai è finito tutto quello che formava la consolazione, la ricompensa a tanto lavoro, al tormento\ldots{} Lo crederesti? Stavo facendo or ora lezione a Griša: prima questa era per me una gioia, ora è un tormento. Perché mi affanno, perché mi affatico? Perché i bambini? È terribile come ad un tratto l'anima mia si sia sconvolta e come invece di tenerezza io non senta per lui altro che rancore, sì, rancore. Lo ucciderei, e\ldots{} 

- Ma tesoro mio, Dolly, ti capisco, ma non tormentarti. Sei tanto offesa, tanto eccitata che molte cose le vedi come non sono. 

Dolly si calmò ed entrambe tacquero per alcuni minuti. 

- Che fare? Anna, pensaci tu, aiutami tu. Io ho riflettuto senza posa e non ho trovato niente. 

Neppure Anna sapeva trovar nulla, ma il suo cuore vibrava ad ogni parola, ad ogni espressione del viso della cognata. 

- Io dico una cosa sola - cominciò Anna - io sono sua sorella, e conosco il suo carattere, quella sua facilità a dimenticarsi di tutto, di tutto - ella fece un gesto sulla fronte - quella sua disposizione all'abbandono completo; ma, in compenso, anche al pentimento completo. Egli in questo momento non crede a quanto è accaduto, non capisce come abbia potuto fare quello che ha fatto. 

- No, lo capisce, lo ha capito - interruppe Dolly. - Ma io\ldots{} tu ti dimentichi di me\ldots{} sto forse meglio, io? 

- Lasciami dire. Quando egli ne parlava, ti confesso, non avevo ancora capito tutto lo sgomento della tua posizione. Vedevo soltanto lui e il fatto che un'intera famiglia fosse sconvolta; mi faceva pena lui; ma ora, dopo aver parlato con te, io, come donna, vedo un'altra cosa: vedo la tua sofferenza e non so dirti quanta pena ne abbia. Ma Dolly, anima mia, io capisco in pieno la tua sofferenza, ma una cosa non so. Io non so\ldots{} non so quanto amore c'è ancora nell'anima tua per lui. Sai solo tu se ve n'è tanto che sia possibile perdonare. Se ve n'è, e tu perdona! 

- No - cominciò Dolly, ma Anna la interruppe, baciandole ancora una volta la mano. 

- Io conosco il mondo più di te - disse lei. - Conosco questi uomini come Stiva, so come considerano queste cose. Tu dici che egli con quella avrà parlato di te. Questo no, non è accaduto. Questi uomini commettono delle infedeltà, ma il loro focolare domestico e la moglie, queste, per loro, sono cose sacre. Per loro, in un certo modo, quelle donne restano spregevoli, e non le confondono con la famiglia. Essi tracciano come una linea insormontabile tra la famiglia e quelle donne. Non lo capisco bene, ma so che è così. 

- Sì, ma lui la baciava\ldots{} 

- Dolly, ascolta, anima mia. Ho visto Stiva quando era innamorato di te. Mi ricordo il tempo in cui veniva a casa mia e parlando di te si commuoveva; e a quale poetica altezza ti trovavi tu per lui; e io so che più egli viveva con te e più in alto tu salivi per lui. Noi a volte ridevamo di lui che ad ogni parola ripeteva: ``Dolly è una donna sorprendente''. Tu sei sempre stata e sei rimasta per lui una cosa celeste, mentre questa è un'attrazione non certo dell'anima sua\ldots{} 

- Ma se questa attrazione si ripeterà? 

- Non è possibile, così per quanto possa intendere io\ldots{} 

- Già, ma tu perdoneresti? 

- Non so, non posso giudicare\ldots{} Sì, posso - disse Anna, dopo aver pensato un po'; e poi, abbracciata col pensiero la situazione e soppesatala sulla bilancia sua intima, aggiunse: - Sì, posso, posso, posso. Sì, lo perdonerei. Non sarei la stessa, ma perdonerei, come se non fosse accaduto affatto\ldots{} 

- Eh, s'intende - interruppe in fretta Dolly, come se stesse per dire quello che aveva pensato più di una volta. - Altrimenti non sarebbe un perdono. Su, andiamo ti accompagno in camera tua - disse, alzandosi, e durante il cammino abbracciò Anna. - Mia cara, come sono contenta che tu sia venuta! Mi sento meglio, molto meglio. 

\capitolo{XX}Tutto quel giorno Anna lo passò in casa degli Oblonskij e non volle ricevere nessuno, mentre già alcuni amici, informati del suo arrivo, erano venuti quel giorno stesso. Passò tutta la mattinata con Dolly e i bambini. Mandò soltanto un biglietto al fratello perché venisse senz'altro a pranzare a casa. ``Vieni, Dio è misericordioso'' aveva scritto. 

Oblonskij pranzò a casa; la conversazione fu generale e la moglie parlò con lui dandogli del tu, cosa che ultimamente non accadeva. Fra marito e moglie permaneva lo stesso distacco di rapporti, ma già non si parlava più di separazione e Stepan Arkad'ic vedeva già la possibilità di spiegarsi e far pace. 

Subito dopo pranzo venne Kitty. Conosceva già Anna Arkad'evna, ma molto poco, ed era venuta ora dalla sorella non senza temere come l'avrebbe accolta questa signora del gran mondo pietroburghese che tutti decantavano. Ma piacque ad Anna Arkad'evna; se ne accorse subito. Anna, evidentemente ne ammirava la grazia e la giovinezza e Kitty non fece in tempo a rassicurarsi che già si sentì non solo sotto il fascino di lei, ma addirittura innamorata di lei, così come le ragazze sono capaci di innamorarsi delle signore sposate più grandi di loro. Anna non aveva nulla di simile a una dama di mondo o a una mamma di un bimbo di otto anni; sarebbe piuttosto somigliata a una ragazza di vent'anni per l'agilità delle movenze, per la vivacità che le balenava ora nel riso ora nello sguardo, se non avesse avuto quell'espressione degli occhi seria, a volte triste, che aveva colpito e attirato a sé Kitty. Kitty sentiva che Anna era affatto spontanea e che non nascondeva nulla, ma che portava in sé un mondo di interessi più alti, inaccessibili a lei, complessi e poetici. 

Dopo pranzo, quando Dolly uscì per andare in camera sua, Anna si alzò in fretta e si accostò al fratello che aveva acceso un sigaro. 

- Stiva - disse, ammiccandogli con vivacità, accennandogli alla porta e facendogli il segno della croce: - va', e che il Signore ti aiuti. 

Egli capì, gettò via il sigaro e scomparve dietro la porta. 

Appena Stepan Arkad'ic fu uscito, Anna ritornò sul divano dove sedeva circondata dai bambini. O che i bambini avessero notato come la mamma voleva bene a questa zia, o che essi stessi si sentissero attratti verso di lei, certo è che i due più grandi, e dietro di questi i più piccoli, come spesso fanno i bambini, ancor prima del pranzo si erano attaccati alla nuova zia e non la lasciavano più. E fra di loro si era venuto a formare come una specie di giuoco che consisteva nello star seduti il più vicino possibile a lei, nel toccarla, nel tenere tra le proprie la sua piccola mano, nel baciarla, nel giocar con l'anello suo, o nel toccare almeno la gala del suo vestito. 

- Su, su, così come eravamo seduti prima - disse Anna Arkad'evna riprendendo il proprio posto. 

E di nuovo Griša ficcò la testa sotto il braccio di lei e poggiò la testina sull'abito, splendendo di gioia e trionfo. 

- E così ora, a quando un ballo? - ella disse rivolta a Kitty. 

- La settimana prossima, e un ballo bellissimo. Uno di quei balli in cui ci si diverte sempre. 

- E ce n'è di quelli in cui ci si diverte? - chiese con tenera ironia Anna. 

- È strano, ma ce n'è. Dai Bobrišcev ci si diverte sempre, dai Nikitin anche, ma dai Mezkovyj ci si annoia sempre. Non l'avete notato, forse? 

- No, cara, per me ormai non ci sono balli in cui ci si diverta - disse Anna, e Kitty vide negli occhi di lei quel suo mondo particolare a lei precluso. - Per me ci sono di quelli dove è meno noioso ed increscioso\ldots{} 

- Ma come potete annoiarvi voi a un ballo? 

- E perché non potrei annoiarmi, io, a un ballo? 

Kitty notò che Anna sapeva già quale risposta sarebbe seguita. 

- Ma perché voi siete dovunque la più bella. 

Anna sapeva ancora arrossire. Arrossì e disse: 

- In primo luogo, non è così; e in secondo luogo, anche se questo fosse vero, a che mi servirebbe? 

- Verrete a questo ballo? - chiese Kitty. 

- Credo che non potrò non venire. Ecco, prendi questo - disse a Tanja che tirava un anello che scivolava facilmente dal dito bianco affusolato. 

- Sarò molto contenta se verrete. Vorrei tanto vedervi a un ballo. 

- Almeno così, se sarà proprio necessario andare, mi consolerò al pensiero di farvi cosa gradita\ldots{} Griša, non tirare, ti prego, sono già tutta spettinata - disse, aggiustandosi una ciocca di capelli fuori di posto con la quale Griša aveva giocato. 

- Vi immagino al ballo in lilla. 

- E perché proprio in lilla? - chiese sorridendo Anna. - Su ragazzi, andate, andate. Sentite? Miss Hull chiama per il tè - disse, staccando da sé i bambini e avviandoli in sala da pranzo. 

- Ma io so perché mi invitate a venire al ballo. Voi vi aspettate molto da questo ballo e volete che tutti siano là, che tutti vi prendano parte. 

- Come lo sapete? È così. 

- Com'è bella la vostra età! - continuò Anna. - Ricordo e conosco anch'io quella nebbia azzurra simile a quella che è sulle montagne svizzere. Quella nebbia che vela tutto, in quel tempo beato in cui è appena appena finita l'infanzia, e da quel cerchio immenso, fortunato e gaio, il cammino si fa sempre e sempre più angusto; e ne vien gioia e sgomento a entrare in quella galleria, ancor che appaia e bella e chiara. Chi non è passato attraverso questo? 

Kitty sorrideva in silenzio. ``Ma come mai ella era passata attraverso questo? Come vorrei conoscere tutta la sua storia!'' pensava Kitty ricordando l'aspetto poco poetico del marito Aleksej Aleksandrovic. 

- Io so qualcosa. Stiva mi ha detto, e io mi compiaccio con voi; mi piace molto Vronskij - continuò Anna - l'ho incontrato alla stazione. 

- Ah, era là? - domandò Kitty arrossendo. - Ma che vi ha detto Stiva? 

- Stiva mi ha rivelato tutto. E io sono stata molto contenta. Ho viaggiato con la madre di Vronskij - continuò - ed essa non ha smesso un momento di parlare di lui; è il figlio preferito; io so come siano parziali le mamme, ma\ldots{} 

- E che cosa vi ha detto di lui sua madre? 

- Ah, un mondo di cose! Lo so che è il suo preferito, però, si vede che è un perfetto cavaliere\ldots{} Ecco, per esempio, mi ha raccontato che ha voluto dare tutto il suo patrimonio al fratello e che, fanciullo ancora, ha salvato una donna che annegava. Insomma, un eroe - disse Anna, sorridendo e ricordandosi di quei duecento rubli che egli aveva dato alla stazione. 

Ma nulla disse di quei duecento rubli. Chi sa perché non le piaceva rammentarsene. Sentiva che in quel gesto c'era qualcosa che riguardava lei, e così come non avrebbe dovuto essere. 

- Mi ha pregato tanto di andare da lei - continuò Anna - e io sono contenta di vedere quella vecchietta, e domani ci andrò. Però, grazie a Dio, Stiva rimane a lungo nello studio da Dolly - aggiunse Anna, cambiando discorso e alzandosi, come contrariata da qualcosa, così almeno parve a Kitty. 

- No, prima io, no, io - gridavano i bambini, dopo aver preso il tè, correndo verso la zia. 

- Tutti insieme - disse Anna e, ridendo, corse loro incontro e li abbracciò facendo cadere tutto quel mucchio di bambini brulicanti che mandavano strida di entusiasmo. 

\capitolo{XXI}Per il tè dei grandi Dolly uscì dalla sua camera: Stepan Arkad'ic non si faceva ancora vedere. Forse era uscito dalla camera della moglie per il passaggio di dietro. 

- Temo che avrai freddo di sopra - osservò Dolly rivolta ad Anna - vorrei farti venire giù, così staremo più vicine. 

- Oh, non ti preoccupare per me - rispondeva Anna, guardando il viso di Dolly e cercando di capire se v'era stata o no la riconciliazione. 

- Però qui avrai troppa luce - rispose la cognata. 

- Ti assicuro che dormo dovunque e sempre come un ghiro. 

- Che c'è - chiese Stepan Arkad'ic, venendo fuori dallo studio e rivolgendosi alla moglie. 

Dal suo tono di voce Kitty e Anna capirono che la pace era avvenuta. 

- Vorrei far passare Anna giù, ma bisogna cambiare le tende. Nessuno lo saprà fare, devo farlo da me - rispose Dolly rivolta a lui. 

``Dio lo sa se han fatto pace in pieno'' pensò Anna, sentendo il tono di lei freddo e calmo. 

- Ah, basta, Dolly, a far sempre difficoltà - disse il marito. - Su, se vuoi, faccio io tutto. 

``Sì, sì, devono aver fatto pace'' pensò Anna. 

- Sì, sì, lo so come farai tutto - rispondeva Dolly: - dirai a Matvej di fare proprio quello che è impossibile fare e te ne andrai e lui ingarbuglierà ogni cosa - e nel dir questo l'abituale sorriso canzonatorio increspò le estremità delle labbra di Dolly. 

``La pace è fatta, in pieno - pensò Anna. - Sia lodato Iddio!'' e, rallegrandosi d'essere stata la fautrice, si avvicinò a Dolly e la baciò. 

- Ma niente affatto; perché ci disprezzi tanto, me e Matvej? - disse Stepan Arkad'ic, sorridendo impercettibilmente, rivolto alla moglie. 

Tutta la serata Dolly fu, come al solito, leggermente canzonatoria col marito, e Stepan Arkad'ic contento e allegro, ma non tanto da apparire, dopo il perdono, dimentico della propria colpa. 

Alle nove e mezzo la conversazione serale in casa Oblonskij, particolarmente lieta e piacevole intorno al tavolo da tè, fu turbata da un avvenimento all'apparenza quanto mai naturale, ma che sembrò strano a tutti. Parlando di conoscenti comuni di Pietroburgo, Anna si era alzata, svelta. 

- Ce l'ho nel mio album - disse - sì, anzi, così vi mostrerò il mio Serëza - aggiunse con un materno sorriso d'orgoglio. 

Avvicinandosi le dieci, l'ora in cui era solita salutare il figlio o metterlo a letto lei stessa, prima di andare a un ballo, si era immalinconita per esserne tanto lontana; e di qualunque cosa si parlasse, non riusciva più a interessarsi, e tornava sempre col pensiero al suo Serëza riccioluto. Le era anzi venuta gran voglia di guardarne la fotografia e di parlare di lui. Approfittando del primo pretesto, si era alzata col suo passo leggero, deciso ed era andata a prendere l'album. La scala per salire in camera sua dava sul pianerottolo della grande scalinata dell'ingresso riscaldato. 

Nel momento in cui ella usciva dal salotto in anticamera il campanello squillò. 

- Chi può essere? - disse Dolly. 

- Per venire a riprendermi è presto, per una visita è tardi - osservò Kitty. 

- Forse sarà qualcuno con le carte d'ufficio - aggiunse Stepan Arkad'ic e mentre Anna passava accanto alla scala, un servo corse su per annunciare chi era venuto, mentre l'ospite era in piedi sotto la lampada. Anna, guardando giù, riconobbe subito Vronskij e una sensazione strana di piacere e insieme di paura le agitò il cuore. Egli stava lì dritto, senza togliersi il cappello, e tirava fuori qualcosa dalla tasca. Nel momento in cui ella fu a metà scala, egli alzò gli occhi, la vide e nell'espressione del suo viso ci fu qualcosa come tra la confusione ed il timore. Ella, chinato lievemente il capo, passò e, dietro di lei, si sentì la voce forte di Stepan Arkad'ic che invitava a entrare e la voce bassa, morbida e pacata di Vronskij che rifiutava. 

Quando Anna rientrò con l'album, Vronskij già non c'era più e Stepan Arkad'ic diceva che egli era venuto per informarsi del pranzo dell'indomani in onore di una celebrità straniera. 

- E per nessuna ragione è voluto entrare. È piuttosto strano. 

Kitty arrossì. Credeva di essere l'unica a capire perché egli fosse passato di là e perché non avesse voluto entrare. 

``È stato da noi - si diceva - e non mi ha trovata; ha pensato che fossi qui; ma non è entrato perché è tardi e perché sa che anche Anna è qui''. 

Tutti si scambiarono un'occhiata, senza dir nulla, e presero a guardare l'album di Anna. 

Niente di straordinario, o di strano che una persona passasse da casa di un amico a chiedere i particolari di un pranzo da offrire e che non entrasse; ma strana a tutti parve la cosa. Più che agli altri parve strana e inopportuna ad Anna. 

\capitolo{XXII}Il ballo era appena cominciato quando Kitty, accompagnata dalla madre, faceva il suo ingresso sulla scala grande inondata di luce e piena di fiori e di servitori incipriati e in giacca rossa. Dalle sale giungeva un brusio prodotto da un movimento uniforme, come di alveare; e mentre esse sul ripiano, fra le piante, si andavano acconciando allo specchio le pettinature e gli abiti, dalla sala si udirono i suoni accorti e precisi dei violini dell'orchestra che aveva attaccato il primo valzer. Un vecchietto in borghese, che esalava profumo di acqua di Colonia e che ravviava ad un altro specchio le piccole tempie grige, si imbatté in loro sulla scala e, facendosi da parte, ammirò visibilmente Kitty che non conosceva. Un giovanotto imberbe, uno di quelli che il vecchio principe Šcerbackij definiva ``moscardini'', con un panciotto esageratamente aperto e una cravatta bianca che s'andava aggiustando nel camminare, la salutò, passò oltre e tornò indietro per invitare Kitty per la quadriglia. La prima quadriglia era già stata concessa a Vronskij, fu quindi concessa al giovanotto la seconda. Un ufficiale che si abbottonava un guanto, si scansò presso la porta e, accarezzandosi i baffi, ammirò la rosea Kitty. 

Sebbene l'abito, l'acconciatura e i preparativi tutti del ballo fossero costati a Kitty grandi fatiche e riflessioni, in questo momento ella entrava nel ballo così disinvolta e naturale nel suo complicato vestito di tulle con trasparente rosa, come se tutte quelle roselline e quelle trine e i particolari dell'abbigliamento non fossero costati a lei e a quelli di casa neppure un attimo di attenzione; come se ella fosse venuta al mondo in quel tulle, in quelle trine, con quell'acconciatura alta con una rosa e due foglioline in cima. 

Quando la vecchia principessa, prima di entrare in sala, volle aggiustarle un nastro della cintura che si era spostato, Kitty si tirò leggermente indietro: sentiva che tutto andava bene e si aggraziava addosso a lei e che non c'era più nulla da ritoccare. 

Kitty era in una delle sue giornate felici. L'abito non tirava da nessuna parte, da nessuna parte pendeva la berta di pizzo, le roselline non s'erano sgualcite né staccate; le piccole scarpe rosa sui tacchi ricurvi non premevano, ma rallegravano il piedino. Le folte bande di posticci biondi si mantenevano come naturali sulla piccola testa. Tutti e tre i bottoni si erano chiusi senza staccarsi sul guanto lungo che avvolgeva il braccio rilevandone la forma. Il vellutino nero del medaglione cingeva il collo, proprio con tenerezza. Questo vellutino era un incanto e a casa, guardandosi allo specchio il collo, Kitty aveva sentito che quel nastrino parlava. Per tutto il resto avrebbe potuto sussistere ancora qualche dubbio, ma il vellutino era un incanto. Anche qui, al ballo, Kitty sorrise nel guardarlo allo specchio. Su per le spalle e le braccia nude Kitty sentiva freddo come di marmo, sensazione che amava in modo particolare. Gli occhi le scintillavano e le labbra vermiglie non potevano non sorridere della consapevolezza del proprio incanto. Non fece in tempo a entrare in sala e a giungere fino alla folla variegata, tutta tulle nastri pizzi e fiori delle signore in attesa di essere invitate (Kitty non si trovava mai fra queste), che già fu invitata al valzer, e dal migliore, dal primo cavaliere nella gerarchia dei balli, da un noto direttore di danze, gran cerimoniere, ammogliato, piacente e ben fatto, Egoruška Korsunskij. Lasciata allora allora la contessa Bonina con la quale aveva ballato un primo giro di valzer, questi aveva dato uno sguardo intorno alla sua corte di coppie danzanti, e avendo visto Kitty entrare, era corso verso di lei con quella particolare andatura disinvolta, propria dei direttori di danze, e, dopo essersi inchinato, senza neppure chiedere s'ella volesse o no, aveva alzato il braccio per cingerle la vita sottile. Kitty si voltò per consegnare a qualcuno il ventaglio e la padrona di casa glielo prese sorridendo. 

- Come avete fatto bene a venire per tempo - egli disse cingendole la vita; - che modo è quello di arrivare in ritardo! 

Piegato il braccio sinistro, ella lo poggiò sulla spalla di lui e i piccoli piedi si mossero nelle scarpette rosa, veloci e leggeri, a tempo di musica, sul pavimento levigato. 

- È un riposo ballare il valzer con voi - disse lui lanciandosi nei primi passi lenti del valzer. - Un incanto! una piuma! che précision! - diceva, ripetendo a lei quel che diceva a quasi tutte le sue brave dame. 

Ella sorrise della lode e continuò a osservare la sala al di sopra della spalla di lui. Non era entrata in società da così poco tempo che al ballo tutti i visi potessero fondersi in un'unica estatica visione; non ne era neppure un'assidua frequentatrice alla quale tutti i visi potessero essere così noti da poterne ricevere noia; era nel giusto mezzo: animata, ma nello stesso tempo padrona di sé tanto da poter osservare. Nell'angolo a sinistra vide che si era raccolto il fiore della società. Là, inverosimilmente scollata, stava la bella Lidie, moglie di Korsunskij; là c'era la padrona di casa, e là brillava con la sua calvizie Krivin, sempre presente nella cerchia migliore; là guardavano i giovanissimi, non osando accostarsi, e là ella trovò Stiva e subito dopo vide la testa e la figura di Anna, in abito di velluto nero. Anche lui era là. Kitty non l'aveva visto da quella sera in cui aveva detto di no a Levin. Con i suoi occhi presbiti lo riconobbe subito e notò che la guardava. 

- Ebbene, ancora un giro? Siete forse stanca? - disse Korsunskij, sentendola leggermente ansante. 

- No, grazie. 

- Dove volete che v'accompagni? 

- La Karenina è là, accompagnatemi da lei. 

- Ai vostri ordini. 

E Korsunskij riprese a ballare il valzer, smorzando l'andatura e dirigendosi verso il gruppo che era nell'angolo a sinistra della sala, mormorando: ``Pardon, mesdames, pardon, pardon, mesdames''. Bordeggiando fra un mare di trine, di tulle, di nastri, senza impigliarvisi neppure per un pelo, girò brusco la dama così che le si scoprirono le gambe sottili nelle calze traforate e lo strascico si aprì a ventaglio e coprì le ginocchia di Krivin. Korsunskij s'inchinò, raddrizzò il petto aperto e le diede la mano, per accompagnarla da Anna Arkad'evna. Kitty, rossa in viso, liberò lo strascico dalle ginocchia di Krivin e, ancora stordita, si voltò a cercare Anna. Anna non era in lilla, come proprio avrebbe voluto Kitty, ma aveva un abito di velluto nero, molto scollato che le scopriva le spalle piene e tornite di avorio antico, il petto e le braccia tonde dal polso minuscolo. Tutto l'abito era ornato di merletto veneziano. In testa, sui capelli neri, tutti suoi, aveva una piccola corona di violette, e un'altra simile sul nastro nero della cintura fra le trine bianche. La pettinatura era semplice: spiccavano soltanto quelle brevi anella restie di capelli ricci che, aggraziandola, si sbizzarrivano continuamente sulla nuca e sulle tempie. Al collo tornito e forte aveva un filo di perle. 

Kitty vedeva Anna ogni giorno, era incantata di lei e se l'era figurata sempre in lilla. Ma ora, vedendola in nero, sentì che non ne aveva afferrato tutto il fascino. Le appariva completamente nuova e insospettata. Capì, ora, che Anna non avrebbe potuto essere vestita in lilla e che il fascino suo consisteva nell'emergere sempre dall'abbigliamento, così che l'abito indossato da lei non venisse notato. E il vestito nero con i merletti pregiati neppure si notava; era solamente una cornice, e ne balzava fuori lei, semplice, naturale, elegante e, nello stesso tempo, gaia e viva. 

Stava in piedi, tenendosi, come sempre, straordinariamente diritta e quando Kitty si avvicinò al gruppo, parlava col padrone di casa volgendo lieve il capo verso di lui. 

- No, io non scaglierò la prima pietra - rispondeva - benché non capisca - aggiunse, alzando le spalle, e subito si rivolse a Kitty con un tenero sorriso di protezione. Colto in un fuggevole sguardo femminile tutto l'abbigliamento di Kitty, le fece con la testa un appena percettibile, ma ben comprensibile cenno d'approvazione per l'abito e per la bellezza. - Voi entrate in sala ballando - disse. 

- È una delle più fedeli collaboratrici - disse Korsunskij, salutando Anna Arkad'evna che non aveva ancora visto. - La principessina ci aiuta a rendere bello e allegro il ballo. Anna Arkad'evna, un giro di valzer - disse inchinandosi. 

- Ah, vi conoscete? - disse la padrona di casa. 

- Chi non ci conosce? Mia moglie ed io siamo come i lupi bianchi, tutti ci conoscono - rispose Korsunskij. - Un giro di valzer, Anna Arkad'evna. 

- Io non ballo, quando è possibile farne a meno - disse lei. 

- Ma oggi non se ne può fare a meno - rispose Korsunskij. In quel momento si avvicinò Vronskij. 

- Ebbene, se oggi non si può farne a meno, allora andiamo - disse lei senza notare l'inchino di Vronskij e sollevando rapida la mano sulla spalla di Korsunskij. 

``Perché è scontenta di lui?'' pensò Kitty avendo notato che Anna determinatamente non aveva risposto all'inchino di Vronskij. Vronskij si accostò a Kitty, per ricordarle la prima quadriglia, rammaricandosi di non avere avuto il piacere di vederla in tutto quel tempo. Kitty guardava, ammirata, Anna che ballava il valzer e intanto ascoltava lui. Si aspettava di essere invitata al valzer; ma egli non lo fece e lei lo guardò con sorpresa. Vronskij arrossì e si precipitò a chiederle il ballo, ma non appena ebbe abbracciata la vita sottile di lei e mosso il primo passo, la musica cessò di colpo. Kitty guardò quel viso che era a così breve distanza da lei; e in seguito, per parecchi lunghi anni, quello sguardo pieno d'amore che ella gli aveva rivolto e a cui egli non aveva risposto, le angosciò il cuore di tormentosa vergogna. 

- Pardon, pardon, un valzer, un valzer - gridava dall'altra parte della sala Korsunskij e, presa a volo la prima signorina che gli capitò, ricominciò a ballare. 

\capitolo{XXIII}Vronskij fece qualche giro di valzer con Kitty. Dopo il valzer Kitty si avvicinò alla madre ed ebbe appena il tempo di scambiare qualche parola con la Nordston, che Vronskij era già venuta a riprenderla per la prima quadriglia. Durante la quadriglia non fu detto nulla di particolare. La conversazione, smozzicata, si aggirò ora sui Korsunskij, marito e moglie, che Vronskij descriveva, con molta amenità, come cari ragazzi quarantenni, ora sul futuro teatro pubblico, e solo una volta la toccò nel vivo, quando egli le chiese se c'era Levin e soggiunse che gli era piaciuto molto. Ma Kitty non si aspettava nulla di più dalla quadriglia. Aspettava invece con trepidazione la mazurca. Le sembrava che nella mazurca si dovesse decidere tutto. Il fatto che durante la quadriglia egli non l'avesse invitata per la mazurca, non l'inquietava. Era sicura di ballare la mazurca con lui, come nelle altre feste, e rifiutò cinque cavalieri dicendo d'essere già impegnata. Tutto il ballo, fino all'ultima quadriglia, fu per Kitty una magica visione di colori gioiosi, di suoni e di movimento. Tralasciava di ballare e chiedeva un po' di riposo solo quando si sentiva troppo stanca. Ma ballando l'ultima quadriglia con uno di quei giovanotti uggiosi al quale non aveva potuto dire di no, venne a trovarsi vis-à-vis con Vronskij e Anna. Dall'inizio del ballo non si era più ritrovata con Anna; ed ecco, a un tratto, la vide ancora del tutto nuova e insospettata. Riconobbe in lei i segni dell'eccitamento dovuto al successo ch'ella stessa conosceva. Vedeva che Anna era come inebriata dall'incanto da lei suscitato. Conosceva questa sensazione, ne conosceva i segni e li vedeva in Anna. Vedeva lo scintillio degli occhi, tremulo e avvampante, e il riso di felicità e di eccitamento che senza volere le increspava le labbra; vedeva la grazia misurata, la sicurezza e la levità dei movimenti. 

``Ma per chi? Per tutti o per uno solo?'' si chiese. E, senza venire in aiuto al disgraziato giovanotto col quale ballava e che s'era lasciato sfuggire il filo di una conversazione iniziata e non riusciva a riannodarlo, e prestando apparentemente orecchio alle forti grida allegre e imperiose di Korsunskij che ora lanciava tutti in un grand rond, ora in una chaîne, Kitty osservava, e il cuore le si stringeva sempre più. ``No, non è l'ammirazione di tutti che l'ha inebriata, ma l'esaltazione di uno solo. E chi è quest'unico? Possibile che sia lui?''. Ogni volta che Vronskij parlava con Anna, negli occhi di lei si accendeva uno scintillio gioioso e un riso di felicità increspava le sue labbra vermiglie. Era come se ella volesse contenersi per non fare apparire questi segni, ma questi salivano da soli sul viso. ``E lui?''. Kitty lo guardò ed ebbe paura. Ciò che con tanta chiarezza appariva nello specchio del viso di Anna, Kitty vide anche in lui. Dove erano più quell'atteggiamento calmo e deciso e quell'espressione del viso liberamente serena? No, ora, ogni volta che egli si volgeva a lei, piegava un po' il capo, quasi desideroso di caderle ai piedi, e nello sguardo suo non vi era che un'espressione di sottomissione e di paura. 

``Non voglio offendervi - diceva ogni volta il suo sguardo - ma voglio salvarmi e non so come''. Un'espressione quale non aveva mai vista nel viso di lui. 

Parlavano di amici comuni, facevano la più insignificante delle conversazioni, ma a Kitty pareva che ogni parola pronunziata decidesse il loro e il suo destino. E lo strano era che, in realtà, pur parlando di come fosse ridicolo Ivan Ivanovic col suo francese e del fatto che per la Elackaja si sarebbe potuto trovare un partito migliore, tuttavia queste parole avevano un senso speciale per loro ed essi lo sentivano così come lo sentiva Kitty. Tutto il ballo, il mondo intero, tutto si coprì di nebbia nel cuore di Kitty. Soltanto la severa educazione ricevuta la sosteneva e l'obbligava a fare quello che da lei si pretendeva, cioè ballare, rispondere alle domande, parlare, sorridere persino. Ma, prima che cominciasse la mazurca, quando già si allontanavano le sedie e alcune coppie s'erano mosse dai salotti verso la sala grande, Kitty fu presa da un attimo di disperazione e di sgomento. Aveva rifiutato cinque cavalieri e ora non ballava la mazurca. Non c'era neppure speranza che qualcuno l'invitasse; proprio perché ella aveva un così grande successo in società, a nessuno poteva venire in mente che non fosse impegnata fino a quel momento. Occorreva dire alla madre che non stava bene e voleva tornare a casa, ma non ne aveva la forza. Era stroncata. 

Si ritirò in fondo a un piccolo salotto e si lasciò cadere su di una poltrona. La gonna lieve come un soffio si sollevò come una nuvola intorno alla vita sottile; la mano nuda, magra e delicata di fanciulla, abbandonata e senza forza affondò nelle pieghe della gonna rosa; l'altra reggeva il ventaglio e con movimento rapido rinfrescava il viso accaldato. Ma a dispetto di questa sua parvenza di farfalla attaccata appena a un filo d'erba e pronta a volar via aprendo le ali iridate, un'angoscia paurosa le stringeva il cuore. 

``Ma forse mi sbaglio, forse questo non è accaduto'' e di nuovo le tornava in mente quello che aveva visto. 

- Kitty, cos'è mai? - disse la contessa Nordston, avvicinandosi senza far rumore sul tappeto. - Non capisco. 

A Kitty tremò il labbro inferiore; si alzò in fretta. 

- Kitty, non balli la mazurca? 

- No, no - disse Kitty con voce che tremava di lacrime. 

- Lui l'ha invitata davanti a me per la mazurca - disse la Nordston, sapendo che Kitty avrebbe capito chi era lui e chi era lei. - Lei ha detto: ``Non ballate forse con la principessina Šcerbackaja?''. 

- Ah, a me che importa! - rispose Kitty. 

Nessuno, all'infuori di se stessa, poteva capire la sua situazione, nessuno sapeva ch'ella aveva detto di no il giorno prima a un uomo che forse amava, e che gli aveva detto di no perché credeva in un altro. 

La contessa Nordston trovò Korsunskij col quale doveva ballare la mazurca e gli impose di andare a invitare Kitty. Kitty ballava nella prima fila e per sua fortuna non doveva parlare perché Korsunskij correva su e giù tutto il tempo dando ordini al suo stuolo di ballerini. Vronskij ed Anna erano situati quasi di fronte a lei. Li vide da lontano con i suoi occhi presbiti, li vide poi da vicino, quando si incontrarono fra le coppie, e più li vedeva più si convinceva che la rovina sua era compiuta. Vedeva che essi si sentivano soli in quella sala piena di gente. E sul viso di Vronskij, sempre così deciso e libero, vedeva quell'espressione di smarrimento e di sottomissione che l'aveva stupita; l'espressione di un cane intelligente che si senta colpevole. 

Anna rideva e il riso si trasmetteva a lui. Anna diveniva pensosa, ed egli si faceva serio. Una forza magica attirava gli occhi di Kitty sul viso di Anna. Ella era incantevole con quel semplice vestito nero, ed incantevoli erano le braccia tonde con i bracciali, ed il collo forte col filo di perle; incantevole la capigliatura inanellata e sciolta e incantevoli le movenze lievi dei piccoli piedi graziosi e delle mani, e il viso piacente pieno di vita; eppure c'era qualcosa di pauroso e di crudele in quell'incanto. 

Kitty l'ammirava ancor più di prima, e sempre di più soffriva. Si sentiva stroncata e il suo viso lo rivelava. Quando Vronskij, scontratosi con lei nella mazurca, la vide, non la riconobbe al primo momento, tant'era mutata. 

- Splendido ballo - le disse, tanto per dire qualcosa. 

- Sì - rispose lei. 

Durante la mazurca, ripetendo una figura complicata inventata da Korsunskij, Anna uscì nel mezzo del circolo, prese due cavalieri e chiamò a sé una signora e Kitty. Kitty la guardò come spaurita e le si accostò. Anna la guardava, socchiudendo gli occhi, e sorrideva stringendole una mano. Ma, visto che il viso di Kitty rispondeva al suo sorriso con disperata sorpresa, si allontanò da lei e si mise a parlare allegramente con l'altra signora. 

``Sì, c'è qualcosa di strano, di diabolico e di affascinante in lei'' si diceva Kitty. 

Anna non voleva restare a cena, ma il padrone di casa cominciò a pregarla. 

- Su via, Anna Arkad'evna - prese a dire Korsunskij, mettendo il braccio nudo di lei sotto la manica del suo frac. - Che idea mi è venuta per il cotillon! Un bijou! 

E si spostava a poco a poco, cercando di trascinarla. Il padrone di casa sorrideva approvando. 

- No, non resterò - rispondeva Anna sorridendo e, malgrado il sorriso, Korsunskij e il padrone di casa capirono, dal tono deciso di lei, che non sarebbe rimasta. - No, anche così ho ballato più a Mosca al vostro ballo che un intero inverno a Pietroburgo - disse Anna voltandosi a guardare Vronskij che stava in piedi davanti a lei. - Bisogna riposare prima d'intraprendere il viaggio. 

- E voi partite certamente domani? 

- Sì, credo - rispose Anna, sorpresa dell'audacia della domanda; e mentre diceva queste parole l'irrefrenabile tremulo scintillio degli occhi e del riso arse lui. 

Anna Arkad'evna non rimase a cena e andò via. 

\capitolo{XXIV}``Sì, c'è qualcosa di sgradevole e di scostante in me - pensava Levin, uscendo da casa Šcerbackij e dirigendosi a piedi dal fratello. - Non piaccio alla gente. Orgoglio, dicono. Ma non è orgoglio. Se fossi stato orgoglioso, non mi sarei messo in una posizione come questa''. E si figurava Vronskij felice, buono, intelligente e calmo che, probabilmente, non s'era mai trovato nella posizione orribile nella quale s'era venuto a trovare lui quella sera. 

``Sì, certamente ella doveva preferire lui. Così doveva andare; ed io non ho da lamentarmi di niente e di nessuno. La colpa è mia. Quale diritto avevo io di credere ch'ella avrebbe voluto legare la sua vita alla mia? Chi sono io? Che cosa sono? Un uomo da nulla, che non è necessario a niente e a nessuno. - E si ricordò del fratello Nikolaj, e fu contento di fermarsi su questo pensiero. - Non ha forse ragione lui che tutto al mondo è cattivo e sleale? Noi non abbiamo giudicato con giustizia il fratello Nikolaj. Certo dal punto di vista di Prokofij, che l'ha incontrato ubriaco e con la pelliccia stracciata, egli è un uomo spregevole, ma io lo conosco sotto un altro aspetto. Conosco l'anima sua; so che ci somigliamo io e lui. Eppure, invece di andarlo a cercare sono andato a pranzo e poi sono andato là''. Levin si accostò a un fanale, lesse l'indirizzo del fratello che aveva nel portafoglio e poi chiamò un vetturino. Durante il percorso, Levin riandò con la mente a tutti gli episodi a lui noti della vita del fratello Nikolaj. Ricordò che suo fratello durante gli anni universitari e ancora un anno dopo, malgrado le irrisioni dei colleghi, aveva condotto una vita da cenobita, adempiendo rigorosamente i riti della religione, il servizio divino, i digiuni e rifuggendo da qualsiasi piacere, soprattutto dalle donne; ma dopo, come se a un tratto si fosse sbandato, s'era accostato alle persone più indegne e s'era lasciato andare alla vita più sregolata. Ricordò la storia del ragazzo che egli aveva preso dalla campagna per educarlo e che in un accesso di cattiveria aveva battuto tanto da farsi intentare un processo per lesioni. Ricordò la storia del baro col quale aveva perso i denari e al quale aveva richiesto una cambiale e sporto poi egli stesso querela, dimostrando d'essere stato ingannato (era questo il denaro che aveva sborsato Sergej Ivanyc). Ricordò ch'egli aveva passato una notte in guardina per atti di violenza. Ricordò l'ignobile processo che aveva imbastito contro il fratello Sergej Ivanyc per accusarlo di non aver pagato la quota del fondo materno; e la sua ultima impresa, quando cioè, inviato come impiegato nella regione occidentale, era stato messo sotto processo per aver percosso un collega anziano\ldots{} Tutto questo era certamente molto abietto, eppure a Levin non appariva così abietto come a coloro che non conoscevano Nikolaj Levin, che non conoscevano tutta la sua storia, che non conoscevano il suo cuore. 

Levin ricordava come nel tempo in cui Nikolaj era nella fase della mania religiosa, dei digiuni, dei monaci, delle funzioni, nel periodo in cui egli cercava nella religione un aiuto, un freno alla sua natura sensuale, non solo nessuno l'aveva mai sorretto, ma tutti, ed egli stesso, l'avevano irriso. Lo punzecchiavano, lo chiamavano Noè, il monaco; e quando s'era traviato, nessuno gli aveva dato aiuto, e tutti, con orrore e disgusto, gli avevano voltato le spalle. 

Levin sentiva che suo fratello Nikolaj, in fondo all'anima, malgrado la sregolatezza della sua vita, non era più irragionevole delle persone che lo disprezzavano. Non era colpa sua l'essere nato con quel carattere ribelle e con la mente ottenebrata da qualcosa: al contrario aveva sempre cercato d'essere buono. ``Gli esporrò tutto, lo costringerò a dirmi tutto, e gli mostrerò di volergli bene e di capirlo'' decise Levin, giungendo dopo le dieci all'albergo indicato nell'indirizzo. 

- Di sopra, numero 12 e 13 - rispose il portiere alla richiesta di Levin. 

- Ma c'è? 

- Dovrebb'esserci. 

La porta del numero 12 era semiaperta e ne usciva, in un fascio di luce, un fumo denso di tabacco cattivo e fiacco, e il suono di una voce che Levin non conosceva; ma Levin capì subito che il fratello era là: aveva sentito il suo tossicchiare. 

Quando entrò nel vano della porta, la voce sconosciuta diceva: ``Tutto dipende da come sarà condotto l'affare, se ragionevolmente e con coscienza''. 

Konstantin Levin guardò attraverso la porta e vide che quegli che parlava era un giovane intabarrato, con un'enorme capigliatura, mentre una donna giovane butterata, con un abito di lana senza polsi e senza colletto, sedeva sul divano. Il fratello non lo si arrivava a scorgere. Ma a Konstantin si strinse il cuore dalla pena nel vedere in quale ambiente di strane persone viveva suo fratello. Nessuno lo aveva sentito; e Konstantin nel togliersi le soprascarpe ascoltava quello che diceva il signore intabarrato. Parlava di una certa impresa. 

- E che il diavolo le scortichi, quelle classi privilegiate - proruppe tossendo la voce del fratello. - Maša, procurati da cena e dacci del vino se ce n'è restato; se no, manda a prendere. 

La donna si alzò e, uscendo fuori di là dell'intelaiatura, vide Konstantin. 

- C'è un signore, Nikolaj Dmitric - disse. 

- Che vuole? - chiese rabbiosa la voce di Nikolaj. 

- Sono io - rispose Konstantin Levin venendo avanti nella luce. 

- Chi io? - ripeté ancora più rabbiosa la voce di Nikolaj. Si sentì che egli si era alzato di scatto, impigliandosi in qualcosa, e Levin vide dinanzi a sé, sulla porta, la figura enorme, magra e ricurva del fratello; figura a lui nota, ma tuttavia lo sorprese per la selvatichezza, per l'aria malandata, per i grandi occhi spaventati. 

Era ancora più magro che non tre anni prima, quando Konstantin Levin l'aveva visto l'ultima volta. Portava una finanziera: le mani e l'ampia ossatura sembravano ancora più enormi. I capelli s'erano diradati, ma gli stessi baffi spioventi coprivano le labbra, gli stessi occhi guardavano strani e ingenui lui che era entrato. 

- Ah, Kostja! - esclamò subito riconoscendo il fratello, e i suoi occhi s'illuminarono di gioia. Ma, nello stesso momento, si voltò a guardare il giovane e fece quel movimento convulso, così noto a Konstantin, con la testa e il collo, come se la cravatta lo soffocasse, e tutta un'altra espressione, selvaggia, martoriata e crudele, si fermò sul suo viso scarno. 

- Io ho scritto a voi e a Sergej Ivanyc che non vi conosco e non voglio conoscervi. Di che hai\ldots{} di che avete bisogno? 

Era affatto diverso da come se l'era immaginato Konstantin. Konstantin infatti, pensando a lui, aveva dimenticato tutto quello che rendeva tanto laboriosi i rapporti con lui; ma ora, nel vedere il suo viso, e in particolare quel volger convulso del capo, gli tornò in mente tutto questo. 

- Non ho bisogno di nulla per nessuna ragione - rispose timido. - Sono venuto semplicemente per vederti. 

La timidezza del fratello ammansì evidentemente Nikolaj. Egli storse le labbra. 

- Ah, sì? - disse. - Allora entra, siedi. Vuoi cenare? Maša, porta per tre. No, aspetta. Sai chi è? - disse rivolto al fratello, indicando il signore intabarrato. - Questo è il signor Krickij, amico mio sin dal tempo di Kiev, un uomo molto notevole. La polizia, naturalmente, lo perseguita perché non è un vigliacco. 

E secondo la sua abitudine, si voltò a guardare in giro tutti quelli ch'erano nella camera. Visto che la donna sulla porta stava per uscire, le gridò: ``Aspetta, ho detto''. E con quell'imprecisione e discontinuità di discorso che Konstantin conosceva bene, guardando di nuovo tutti, cominciò a raccontare al fratello la storia di Krickij: come l'avessero cacciato dall'università perché aveva organizzato una società di soccorso per gli studenti poveri e scuole domenicali, e come poi fosse entrato in una scuola elementare quale maestro, e come anche di là l'avessero cacciato e infine processato per qualche cosa. 

- Siete dell'università di Kiev? - chiese Konstantin Levin a Krickij per interrompere il silenzio imbarazzante che si era stabilito. 

- Sì, ero a Kiev - disse Krickij stizzito e accigliato. 

- E questa donna - lo interruppe Nikolaj Levin, indicandola - è la compagna della mia vita Mar'ja Nikolaevna. L'ho presa da una casa - e nel dire ciò contrasse il collo. - Ma le voglio bene e la rispetto, e quelli che vogliono avere rapporti con me - aggiunse, alzando la voce e accigliandosi - sono pregati di amarla e di rispettarla. È come se fosse mia moglie, proprio lo stesso. Ecco, così tu sai con chi hai a che fare. E se credi di abbassarti, ecco la porta, e vattene con Dio. 

E di nuovo i suoi occhi percorsero tutti interrogativamente. 

- Non capisco perché mai dovrei abbassarmi. 

- Su, allora ordina, Maša; fa' portare da cena: tre porzioni, vodka e vino\ldots{} No, non occorre. Va'. 

\capitolo{XXV}-Allora guarda - continuò Nikolaj Levin, contraendosi e corrugando con sforzo la fronte. Evidentemente gli era difficile riflettere che cosa dire e che cosa fare. - Ecco, guarda - e mostrò nell'angolo della stanza vari spezzoni di ferro legati con funi. - Vedi questo? È il principio di una nuova impresa alla quale ci accingiamo. Quest'impresa è un'artel'. 

Konstantin non ascoltava quasi. Fissava quel viso malaticcio, tisico, e sempre più ne aveva pena, e non riusciva a seguire quello che suo fratello gli andava raccontando di quella sua artel'. Si rendeva conto che questa artel' era soltanto un espediente per salvarsi dal disgusto di se stesso. Nikolaj Levin continuò a dire: 

- Tu sai che il capitale schiaccia il lavoratore. Da noi gli operai, i contadini sostengono tutto il peso del lavoro e sono posti in una condizione tale che, per quanti sforzi facciano, non riescono ad uscire dalla loro situazione di bestie da soma. Tutto il margine del guadagno, col quale potrebbero migliorare la loro sorte, procurarsi un po' di tempo libero e con esso l'istruzione, tutto il soprappiù della paga è sottratto loro dai capitalisti. E la società è congegnata così che più quelli lavorano, più s'arricchiscono i mercanti, i proprietari di terre, mentre loro rimangono sempre bestie da soma. Quest'ordine di cose va mutato - e guardò fisso e interrogativamente il fratello. 

- Sì, s'intende - disse Konstantin, notando il rossore che era apparso sotto gli zigomi sporgenti del fratello. 

- E poi, ecco, organizziamo un'artel' di fabbriferrai, dove la produzione e il profitto, i principali attrezzi di produzione, tutto sarà in comune. 

- E dove avrà sede quest'artel'? - chiese Konstantin Levin. 

- Nel villaggio di Vozdrëm, nel governatorato di Kazan'. 

- E perché in un villaggio? Nei paesi, mi pare, c'è già tanto da fare. E perché un'artel' di fabbriferrai in un paese? 

- Ma perché anche ora i contadini sono gli stessi schiavi di prima; e appunto per questo, a te e a Sergej Ivanyc dispiace che si voglia farli uscire da questa schiavitù - disse Nikolaj Levin, irritato dall'obiezione. 

Konstantin Levin sospirò, e si mise a esaminare la camera tetra e sudicia. Questo sospiro parve irritare ancor più Nikolaj. 

- Conosco le opinioni aristocratiche tue e di Sergej Ivanyc. So che egli adopera tutte le forze dell'ingegno per giustificare il male esistente. 

- No, ma perché parli di Sergej Ivanyc? - proruppe Levin sorridendo. 

- Sergej Ivanyc? Ah, ecco perché! - gridò ad un tratto Nikolaj, sentendo pronunciare il nome di Sergej Ivanyc - ecco a che scopo\ldots{} Sì, ma a che scopo parlare? Dimmi una cosa\ldots{} Perché sei venuto da me? Tu disprezzi tutto ciò e va bene, e allora vattene con Dio, vattene! - gridò alzandosi dalla sedia - vattene, vattene! 

- Io non lo disprezzo affatto - disse timido Konstantin Levin. - Non discuto neppure. 

Nel frattempo era tornata Mar'ja Nikolaevna. Nikolaj Levin si voltò rabbioso verso di lei. Ella gli si accostò e gli mormorò qualcosa. 

- Non so bene, sto diventando irascibile - disse Nikolaj, calmandosi e respirando faticosamente - e poi tu mi parli di Sergej Ivanyc e del suo articolo. È una tale assurdità, una tale menzogna, un tale autoinganno. Che cosa mai può scrivere sulla giustizia un uomo che non la conosce nemmeno? Avete letto il suo articolo? - disse rivolto a Krickij, sedendosi di nuovo accanto al tavolo e spostando fino alla metà di esso le sigarette sparse, per far posto. 

- Non l'ho letto - disse cupo Krickij, non volendo evidentemente entrare in conversazione. 

- Perché? - si voltò ora a Krickij con irritazione Nikolaj Levin. 

- Perché non ritengo utile perdere il tempo in questo. 

- Ma, scusate, come fate a sapere che verreste a perdere il tempo? Per molti quell'articolo è inaccessibile, troppo alto. Ma per me è un'altra cosa, io vedo da parte a parte le sue idee e so perché tutto questo è debole. 

Tutti tacquero. Krickij si alzò lentamente e prese il berretto. 

- Non volete cenare? Allora, addio. Domani venite col fabbroferraio. 

Appena Krickij fu uscito, Nikolaj Levin sorrise e strizzò l'occhio. 

- Anche lui è cattivo - disse. - Perché io vedo\ldots{} 

Ma in quel momento Krickij sulla porta lo chiamò. 

- Che occorre ancora? - disse Nikolaj e uscì nel corridoio verso di lui. Rimasto solo con Mar'ja Nikolaevna, Levin si rivolse a lei. 

- E voi, è molto che vivete con mio fratello? - le chiese. 

- Ecco, è già più di un anno. La sua salute è molto peggiorata. Beve tanto. 

- E che cosa beve? 

- La vodka beve, e gli fa male! 

- Molta forse? - mormorò Levin. 

- Sì - disse lei, guardando timida la porta sulla quale era apparso Nikolaj Levin. 

- Di che stavate parlando? - domandò, aggrottando le sopracciglia e facendo passare dall'uno all'altra i suoi occhi spauriti. - Di che cosa? 

- Di nulla - rispose Konstantin confondendosi. 

- E se non volete dirlo, fate pure. Solo non c'è bisogno che tu parli con lei. Lei è una di quelle ragazze\ldots{} e tu sei un signore - disse contraendo il collo. - Tu, io lo vedo, hai capito tutto, l'hai apprezzata, e consideri con pietà i miei traviamenti - cominciò di nuovo, alzando la voce. 

- Nikolaj Dmitric, Nikolaj Dmitric - mormorò di nuovo Mar'ja Nikolaevna, accostandosi a lui. 

- Su, va bene, va bene! Già, e che ne è della cena? Ah, eccola - esclamò, vedendo un cameriere col vassoio. - Qua, metti qua - disse irritato e, presa la vodka, ne versò un bicchierino e bevve avidamente. - Bevi? ne vuoi? - disse, fattosi allegro d'un tratto, al fratello. - Su, via, basta di Sergej Ivanyc. Eppure son contento di vederti. Checché si dica, non siamo estranei tuttavia. Su, bevi, dunque. Racconta su, che cosa fai? - continuò, masticando avidamente un pezzo di pane e versando un altro bicchierino. - Come te la passi? 

- Vivo solo in campagna, così come vivevo prima, mi occupo dell'amministrazione - rispose Konstantin, guardando con terrore l'avidità con la quale il fratello beveva e mangiava e sforzandosi di nascondere la propria attenzione. 

- Perché non prendi moglie? 

- Non m'è capitato - rispose arrossendo Konstantin. 

- Come mai? Per me è finita. Me la sono sciupata la mia vita. L'ho detto e lo dirò ancora: se mi avessero dato la mia parte quando ne avevo bisogno, tutta la mia vita sarebbe stata un'altra. 

Konstantin Dmitrevic si affrettò a cambiare discorso. 

- Lo sai che il tuo Vaniuška è da me a Pokrovskoe come inserviente? - disse. 

Nikolaj contrasse il collo e divenne pensoso. 

- Su, raccontami che si fa a Pokrovskoe? La casa è sempre in piedi? E le betulle, e la nostra stanza di studio? E Filipp il giardiniere è possibile che sia vivo ancora? Come ricordo la pergola e il sedile! Bada, però, a non cambiar nulla in casa; ma prendi moglie al più presto, e assesta tutto così com'era prima. Io allora verrò da te, se tua moglie sarà una brava donna. 

- Ma vieni adesso da me - disse Levin. - Come ci sistemeremmo bene! 

- Verrei da te se sapessi di non trovare Sergej Ivanyc. 

- Ma non lo troverai. Io vivo del tutto indipendente da lui. 

- Sì; ma qualunque cosa tu dica, devi scegliere fra me e lui - disse, guardando timido il fratello negli occhi. Questa timidezza commosse Konstantin. 

- Se vuoi conoscere tutta la mia opinione a questo riguardo, ti dirò che nella questione tua con Sergej Ivanyc, io non prendo le parti né dell'uno né dell'altro. Avete torto tutti e due. Tu hai torto in un modo più formale, lui in un modo più sostanziale. 

- Ah, ah, tu hai capito questo, l'hai capito davvero? - gridò con gioia Nikolaj. 

- Ma io, personalmente, tengo più alla tua amicizia, perché\ldots{}

- Perché, perché? 

Konstantin non poteva dire che ci teneva perché Nikolaj era un disgraziato e aveva bisogno di affetto. Ma Nikolaj capì ch'egli voleva dire proprio questo e, accigliandosi, allungò di nuovo la mano verso la vodka. 

- Basta, Nikolaj Dmitric - disse Mar'ja Nikolaevna, stendendo la mano grassoccia verso la caraffa. 

- Lascia! Non seccare! Ti picchio! - gridò. 

Mar'ja Nikolaevna sorrise d'un sorriso mansueto e buono che si comunicò anche a Nikolaj e allontanò la vodka. 

- Tu credi che lei non capisca nulla? - disse Nikolaj. - Capisce tutto meglio di noi. Non è vero che in lei c'è qualcosa di buono e di caro? 

- Non siete stata mai prima a Mosca? - le disse Konstantin, tanto per dire qualche cosa. 

- Ma non darle del voi. Ne può avere soggezione. Nessuno, tranne il giudice di pace, quando l'hanno giudicata perché voleva andarsene dalla casa di corruzione, le ha mai dato del voi. Dio mio, che razza di insensatezze al mondo! - gridò improvvisamente. - Queste nuove istituzioni, questi giudici di pace, il consiglio distrettuale, che assurdità. 

E prese a raccontare i suoi contrasti con le nuove istituzioni. 

Konstantin Levin lo ascoltava, ma ora quel negare il valore di tutte le pubbliche istituzioni, cosa che egli stesso condivideva e che spesso aveva espresso, gli spiaceva sulle labbra del fratello. 

- In quell'altro mondo capiremo tutto questo - disse scherzando. 

- In quell'altro mondo? Oh, io non amo l'altro mondo! Non l'amo - disse, fermando i suoi selvaggi occhi spauriti in faccia al fratello. - Perché ora, ecco, ci sembra bello andarcene via da tutta questa turpitudine, da tutta questa confusione degli altri e nostra, ma io ho paura della morte, ho paura, tremenda paura della morte. - Rabbrividì. - Ma bevi qualcosa? Vuoi dello champagne? Oppure, andiamo in qualche posto. Andiamo dagli zigani! Sai, mi piacciono gli zigani e anche le canzoni russe. 

La sua lingua cominciò ad imbrogliarsi ed egli prese a saltare da un argomento all'altro. Konstantin, con l'aiuto di Maša, lo convinse a non muoversi di casa e lo mise a letto completamente ubriaco. 

Maša promise di scrivere a Konstantin in caso di necessità e di convincere Nikolaj Levin ad andare a vivere presso il fratello. 

\capitolo{XXVI}La mattina Konstantin Levin partì da Mosca e verso sera giunse a casa. In treno parlò con i compagni di viaggio di politica, delle nuove strade ferrate, e durante il percorso, così come durante il soggiorno a Mosca, fu sopraffatto da una certa confusione di idee, da uno scontento di sé, come da una vergogna di fronte a qualcosa. Ma quando uscì dalla stazione e riconobbe Ignat il cocchiere, orbo di un occhio, col bavero del gabbano rialzato; quando, nella luce incerta che filtrava dalle finestre della stazione, vide la slitta coi tappeti, i suoi cavalli con le code legate, le bardature ad anelli e i fiocchi, e quando Ignat il cocchiere, prima ancora di finire di sistemare i bagagli, prese a raccontargli le novità della campagna: l'arrivo dell'imprenditore, lo sgravo della Pava, egli sentiva che a poco a poco la confusione si diradava, che la vergogna e lo scontento scomparivano. Al solo vedere Ignat e i cavalli aveva provato questo; ma quando infilò il pellicciotto di montone che gli avevan portato e, sedutosi tutto imbacuccato nella slitta, partì, pensando alle imminenti disposizioni da dare in campagna e guardando il bilancino sgroppato eppur focoso, un tempo cavallo da sella del Don, cominciò a considerare in modo del tutto diverso quello che gli era successo. Sentiva di essere di nuovo se stesso e di non voler essere altri. Voleva soltanto essere migliore di come era prima. In primo luogo, da quel giorno decise di non sperare più in quella felicità straordinaria che gli doveva essere data dal matrimonio e, in conseguenza, di non disdegnare tanto il presente. In secondo luogo non avrebbe permesso a se stesso di lasciarsi trascinare dal vizio carnale il cui ricordo lo aveva tanto tormentato al momento di fare la sua proposta. 

Dopo, ricordando il fratello Nikolaj, decise con se stesso di non dimenticarlo mai più, di aiutarlo invece, di non allontanarlo mai più dalla sua mente e di essere pronto a venirgli in aiuto quando si fosse trovato in cattive condizioni. E questo sarebbe accaduto presto, lo sentiva. Poi, anche il discorso del fratello sul consumismo, che egli aveva lì per lì abbandonato con tanta leggerezza, ora lo faceva meditare. Riteneva un'assurdità il cambiamento delle condizioni economiche esistenti, ma sentiva sempre l'ingiustizia del proprio superfluo di fronte alla miseria del popolo. E decise che d'ora in poi, per sentirsi pienamente nel giusto, pur avendo sempre lavorato e vissuto senza sperpero, avrebbe lavorato ancora di più e ancora di meno si sarebbe consentito del lusso. E tutto questo gli sembrava così facile a ottenersi, che passò tutto il tempo del viaggio nei sogni più lusinghieri. Con un vigoroso senso di fiducia in una vita migliore, giunse a casa alle nove di sera. 

Dalle finestre di Agaf'ja Michajlovna, la vecchia njanja che in casa occupava il posto di governante, veniva giù la luce sulla neve del piazzale davanti alla casa. Ella non dormiva ancora. Kuz'ma, svegliato da lei, corse fuori sulla scala, assonnato e scalzo. La cagna da caccia Laska, che per poco non buttò a terra Kuz'ma, saltò fuori anche lei a guaire e a strofinarsi contro le ginocchia di Levin; si sollevava sulle zampe, desiderando, senza peraltro arrischiarvisi, mettergli le zampe anteriori sul petto. 

- Siete tornato presto, batjuška - diceva Agaf'ja Michajlovna. 

- M'è venuta addosso la noia, Agaf'ja Michajlovna. In albergo si sta bene, ma a casa è meglio - le rispose, e passò nello studio. 

Lo studio fu illuminato a poco a poco da una candela che vi portarono. Cominciarono a comparire i noti particolari; le corna di cervo, gli scaffali coi libri, lo specchio, la stufa con la bocca di calore che da tempo doveva essere riaccomodata, il divano del padre, il grande scrittoio, sullo scrittoio un libro aperto, un portacenere rotto, un quaderno con la propria scrittura. Quando egli vide tutto questo, per un attimo fu preso dal dubbio di poter costruire quella nuova vita di cui aveva sognato durante il viaggio. Era come se tutte queste impronte di vita lo afferrassero e gli dicessero: ``No, non ti libererai di noi e non sarai un altro; ma sarai così come sei sempre stato, con tutti i tuoi dubbi e con quell'eterno scontento di te, con gli inutili tentativi di ripresa e con le ricadute, con quell'eterna ansia di felicità che non ti è data e che per te è impossibile''. 

Ma questo lo dicevano le sue cose, mentre un'altra voce nell'animo suo diceva che non ci si doveva sottomettere al passato e che di se stessi si poteva fare tutto. E obbedendo a questa voce, si accostò a un angolo dove si trovavano due pesi da un pud ciascuno e cominciò a sollevarli da ginnasta qual era, cercando di mettersi in uno stato di vigore. Di là dalla porta scricchiolarono dei passi. Egli abbassò in fretta i pesi. 

Entrò il fattore e disse che tutto, grazie a Dio, andava bene; ma comunicò che il grano saraceno s'era bruciacchiato nel nuovo essiccatoio. Questa notizia esasperò Levin. Il nuovo essiccatoio era stato costruito e in parte ideato da Levin. Il fattore era sempre stato contrario al nuovo essiccatoio e ora, con celata soddisfazione, dichiarava che il grano saraceno s'era bruciato. Levin invece era fermamente convinto che s'era bruciato solo perché non erano state prese quelle misure che egli aveva cento volte disposto. Si indispettì, fece una solenne risciacquata al fattore. Ma c'era stato un avvenimento importante e lieto; s'era sgravata la Pava, la vacca più bella, più costosa, comprata a una esposizione. 

- Kuz'ma, dammi il pellicciotto. E voi, andate a prendere un po' la lanterna, voglio dare un'occhiata - disse al fattore. 

La stalla per le mucche pregiate si trovava subito dietro alla casa. Attraversando il cortile, vicino al mucchio di neve che era accanto alle serenelle, Levin raggiunse la stalla. Quando si aprì la porta coperta di gelo, si sentì una zaffata di letame caldo, fumante e le mucche, sorprese dalla luce insolita della lanterna, si agitarono sulla paglia fresca. Baluginò la groppa vasta, liscia, a macchie nere e bianche dell'olandese. Berkut, il toro, disteso con l'anello al labbro, avrebbe voluto alzarsi, ma cambiò idea, soffiò due volte quando gli passarono accanto. La bella Pava, rossa, enorme come un ippopotamo, con la schiena voltata, nascondeva a quelli che entravano la vitellina e se l'andava annusando. 

Levin entrò nel recinto, guardò la Pava e fece alzare sulle lunghe zampe traballanti la vitella bianca e rossa. La Pava, agitata, stava per mugghiare, ma quando Levin le accostò la vitellina, si acquietò e, dopo aver soffiato pesantemente, prese a leccarla con la lingua scabra. 

La vitella intanto dava dei colpi col muso, annaspando sotto l'anguinaia della madre e movendo in giro la piccola coda. 

- Su, fa' luce qua, Fëdor, qua la lanterna - diceva Levin osservando la vitella. - Tale e quale la madre! Benché per il colore somigli al padre. Bella, molto bella. Lunga e lattaiola. Vasilij Fëdorovic, è bella, eh? - si voltò al fattore, completamente in pace con lui per il grano saraceno, tanto era contento della vitella. 

- E a chi dovrebbe somigliare per essere brutta? Il giorno dopo la vostra partenza è venuto Semën l'imprenditore. Bisognerà mettersi d'accordo con lui, Konstantin Dmitric - disse il fattore. - Vi ho già parlato della macchina. 

Questa sola questione immise Levin in tutti i particolari dell'azienda, che era vasta e complessa, ed egli dalla stalla passò in ufficio, e, dopo aver parlato col fattore e con Semën l'imprenditore, rientrò in casa e andò difilato di sopra, in salotto. 

\capitolo{XXVII}La casa era grande, all'antica, e Levin, pur vivendo solo, la occupava e la riscaldava tutta. Sapeva che questo era sciocco, sapeva che era perfino malfatto e contrario ai suoi attuali nuovi propositi, ma questa casa era tutto un mondo per Levin. Era il mondo nel quale avevano vissuto ed erano morti suo padre e sua madre. Essi avevano vissuto quella vita che per Levin rappresentava l'ideale di ogni perfezione e che egli sognava di rinnovare con la propria moglie e con la propria famiglia. 

Levin ricordava appena sua madre. L'immagine di lei era sempre stata un ricordo sacro, e nella sua mente la futura sposa avrebbe dovuto essere una riproduzione di quell'ideale delicato e santo di donna che era stata sua madre. 

Egli non solo non poteva immaginare l'amore per la donna al di fuori del matrimonio, ma immaginava prima la famiglia e poi la donna che gliel'avrebbe data. Perciò le sue idee sul matrimonio non erano simili a quelle della maggioranza degli uomini che conosceva, per i quali il matrimonio era uno dei molti affari della vita sociale. Per Levin era il più grande avvenimento della vita, dal quale dipendeva tutta la felicità. E ora bisognava rinunciarvi. 

Quando entrò nel salottino dove era solito prendere il tè e si mise a sedere nella sua poltrona con un libro, e quando Agaf'ja Michajlovna gli portò il tè e, col suo solito ``e mi metto a sedere anch'io, batjuška'', si accomodò sulla sedia accanto alla finestra, Levin sentì che, per quanto ciò fosse strano, egli non aveva abbandonato il suo sogno e non poteva vivere senza esso. O con lei o con un'altra, ma questo sarebbe avvenuto. Leggeva il libro, pensava a quello che leggeva, soffermandosi a sentire Agaf'ja Michajlovna che parlottava senza posa; e intanto vari quadri della sua azienda agricola e della futura vita familiare si presentavano senza alcun legame alla sua immagine. Sentiva che in fondo all'anima qualcosa si fissava, si dimensionava, si assestava. 

Ascoltava il parlottare di Agaf'ja Michajlovna, di come Prochor avesse dimenticato Dio e con i denari che gli aveva regalato Levin per comprare il cavallo cioncasse tutto il giorno e picchiasse a morte la moglie; ascoltava e leggeva il libro, e ricordava tutto il procedimento delle sue idee risvegliato dalla lettura. Era un libro di Tyndall sul calore. Ricordava le sue critiche mosse al Tyndall per quella sua disinvoltura nel condurre gli esperimenti e per quella sua mancanza di visione filosofica. Ma d'un tratto gli affiorò alla mente un pensiero piacevole: ``Fra due anni avrò nella mia mandria due mucche olandesi, la stessa Pava potrà essere ancora viva, e alle dodici giovenche di Berkut aggiungici a far bella mostra queste tre, che meraviglia!''. E prese di nuovo il libro. 

``E va bene, l'elettricità e il calore sono una cosa sola; ma è possibile, per risolvere un problema, porre in un'equazione una grandezza in luogo di un'altra? E allora? Il collegamento di tutte le forze della natura anche così si sente per istinto\ldots{} Proprio bello quando la figlia di Pava sarà già una mucca pezzata di rosso e quando ci sarà già una mandria cui aggiungere queste tre!\ldots{} Perfetto!\ldots{} Uscire con la moglie e con gli ospiti a incontrar la mandria\ldots{} Mia moglie dirà: `Kostja ed io abbiamo tirato su questa vitella come fosse un bambino'. `Come vi può interessare tutto questo?' dirà l'ospite. `Tutto quello che interessa lui, interessa me'. Chi sarà mai lei? - Ed egli ricordava quello che era successo a Mosca\ldots{} - Ma che fare? Io non ne ho colpa. Ora tutto andrà in modo nuovo. È assurdo che la vita, che il passato non lascino raggiungere lo scopo. Bisogna lottare per vivere meglio\ldots{}''. Sollevò il capo e si fece pensoso. La vecchia Laska, che non aveva ancora completamente smaltito la gioia dell'arrivo del padrone, e che era corsa ad abbaiare in cortile, ritornò scodinzolando e portando con sé odor d'aria fresca; si accostò a Levin, gli ficcò il muso sotto il braccio, guaendo flebile e chiedendo d'essere carezzata. 

- Solo la parola non ha - disse Agaf'ja Michajlovna. - È un cane, eppure capisce che il padrone è tornato ed è di umor nero. 

- Perché d'umor nero? 

- E che forse non lo vedo, batjuška? Basta aver la salute e la coscienza pulita. 

Levin la guardava fisso, meravigliandosi che ella avesse intuito i suoi pensieri. 

- Be', devo portare dell'altro tè? - disse lei e, presa la tazza, uscì. 

Laska continuava a ficcargli il muso sotto il braccio. Egli la lisciò. Allora essa si acciambellò ai suoi piedi, poggiando il muso sulla zampa posteriore che sporgeva. E a mostrar che ora stava bene, che era contenta, aprì leggermente la bocca, schioccò un po' con le labbra e, accostate ai vecchi denti le labbra bavose, s'acquietò in una calma beata. Levin seguì attentamente quest'ultimo movimento. 

``Anch'io così - si disse - anch'io così. Non fa niente\ldots{} Tutto va bene''. 

\capitolo{XXVIII}La mattina dopo il ballo, Anna Arkad'evna inviò di buon'ora un telegramma al marito, annunziandogli la sua partenza da Mosca per quel giorno stesso. 

- No, devo partire, devo partire - diceva alla cognata, spiegando il cambiamento di programma con un tono tale che pareva si fosse ricordata di tante faccende da non poterle nemmeno elencare. 

- No, è meglio oggi stesso! 

Stepan Arkad'ic non pranzò a casa, ma promise di venire alle sette per accompagnare la sorella. Nemmeno Kitty venne: mandò un biglietto in cui diceva di aver mal di capo. 

Dolly e Anna pranzarono soltanto con i bambini e la signorina inglese. Ma i bambini, o perché incostanti e ipersensibili, o perché avvertivano che Anna quel giorno era tutt'altra da quella che essi avevano preso ad amare e che già non s'occupava più di loro, certo è che avevano smesso improvvisamente il loro giuoco con la zia, quell'attaccarsi a lei; e il fatto che lei partisse non li interessava per nulla. Tutta la mattina Anna fu presa dai preparativi per la partenza. Scrisse alcuni biglietti ad amici moscoviti, annotò alcuni conti, e preparò il bagaglio. A Dolly pareva che nell'insieme ella non fosse in tranquillità di spirito, ma in un certo stato di inquietudine che ella conosceva bene e che sorge non senza ragione e per lo più nasconde lo scontento di sé. Quando, dopo pranzo, Anna andò in camera sua a cambiarsi, Dolly la seguì. 

- Come sei strana oggi - le disse. 

- Io? Trovi? Non sono strana, sono cattiva. Mi accade talvolta. Avrei voglia di piangere. È molto sciocco, ma passa - disse svelta Anna, abbassando il viso divenuto rosso su un minuscolo sacchetto dove andava riponendo la cuffia da notte e i fazzoletti di batista. I suoi occhi brillavano in modo particolare e si riempivano continuamente di lacrime. - Mi è tanto dispiaciuto lasciare Pietroburgo, e ora invece non me ne andrei più via di qua. 

- Sei venuta ed hai fatto un'opera di bene - disse Dolly, osservandola attentamente. 

Anna la guardò con gli occhi pieni di lacrime. 

- Non dir questo, Dolly. Io non ho fatto nulla, e non potevo far nulla. Mi meraviglio, a volte, a veder come la gente sembri d'accordo nel guastarmi. Che ho fatto e che potevo fare? In te, nel tuo cuore s'è trovato tanto amore da perdonare\ldots{} 

- Senza di te, Dio lo sa cosa sarebbe stato! Come sei felice, Anna! - disse Dolly. - Nell'anima tua tutto è limpido e bello. 

- Ognuno ha nell'anima i propri skeletons, come dicono gli inglesi. 

- E tu quali skeletons hai mai? In te tutto è così chiaro. 

- Eppure ci sono - disse Anna a un tratto e, inaspettatamente, dopo le lacrime, un riso sottilmente ironico le increspò le labbra. 

- Ma certamente sono gai, i tuoi skeletons, non tenebrosi - disse sorridendo Dolly. 

- No, sono tenebrosi. Lo sai perché vado via oggi e non domani? Questa confessione che mi pesa te la voglio fare - disse Anna decisa, riversandosi sulla poltrona e guardando dritto negli occhi Dolly. 

E con sorpresa, Dolly vide che Anna era diventata rossa fino alle orecchie, fino a quelle brevi anella di capelli neri che le si sbizzarrivano sul collo. 

- Sì - continuò Anna. - Sai perché Kitty non è venuta a pranzo? È gelosa di me. Io le ho sciupato tutto\ldots{} sono stata io a renderle quel ballo un tormento e non una gioia. Ma davvero, davvero non ne ho colpa, oppure solo un poco - disse, indugiando con voce sottile sulla parola ``poco''. 

- Oh, come l'hai detto alla stessa maniera di Stiva! - disse ridendo Dolly. 

Anna si urtò. 

- Oh, no, no! io non sono Stiva - disse, accigliandosi. - Te lo racconto perché neppure un attimo mi permetto di dubitare di me stessa - disse Anna. 

Ma proprio nel momento in cui pronunciava queste parole, sentì che non erano vere: non solo dubitava di se stessa ma provava un'agitazione al pensiero di Vronskij, e partiva prima di quello che avrebbe voluto solo per non incontrarsi più con lui. 

- Già, Stiva ci diceva che hai ballato la mazurca con lui e che lui\ldots{} 

- Non puoi immaginare come ciò sia stato ridicolo. Io non pensavo che a combinare il matrimonio e a un tratto, ecco tutt'altra cosa. Forse senza volere io\ldots{} 

Arrossì e si fermò. 

- Oh, loro lo sentono subito! - disse Dolly. 

- Ma io sarei desolata se ci fosse qualcosa di serio da parte sua - l'interruppe Anna. - E sono sicura che tutto questo sarà dimenticato, e che Kitty cesserà di odiarmi. 

- Del resto, Anna, a dirti la verità, io non desidero molto questo matrimonio per Kitty. Ed è meglio che vada a monte se lui, Vronskij, ha potuto innamorarsi di te in un giorno. 

- Oh, Dio mio, questo sarebbe così sciocco! - disse Anna, e di nuovo un rossore denso di soddisfazione le apparve sul viso, nel sentire espresso in parole il pensiero che l'occupava tutta. - E così, ecco, me ne vado dopo essermi fatta una nemica di Kitty, che avevo preso ad amare. Ah, com'è cara! Ma tu appianerai tutto questo, Dolly, vero? 

Dolly poteva trattenere a stento un sorriso. Voleva bene ad Anna, ma non le spiaceva scorgere anche in lei qualche debolezza. 

- Una nemica! Non può essere. 

- Vorrei tanto che tutti voi mi voleste bene come ve ne voglio io; e ora ho preso a volervene ancora di più - disse Anna con le lacrime agli occhi. - Ah, come sono sciocca, oggi! 

Si passò il fazzoletto sul viso e cominciò a vestirsi. 

Stepan Arkad'ic giunse proprio al momento della partenza, in ritardo, col viso accaldato, allegro, fragrante di vino e di sigaro. 

L'emotività di Anna si era comunicata a Dolly e, quando abbracciò per l'ultima volta la cognata, le mormorò: 

- Sappi, Anna, che quello che hai fatto per me non lo dimenticherò mai. E ricordati che ti ho voluto bene e te ne vorrò sempre come all'amica migliore. 

- Non capisco perché - disse Anna, baciandola e nascondendo le lacrime. 

- Tu mi hai capita e mi capisci. Addio, cara! 

\capitolo{XXIX}``Finalmente tutto è finito, sia lodato Iddio!''. Fu questo il primo pensiero che venne ad Anna quando salutò per l'ultima volta il fratello che fino al terzo segnale aveva ostruito con la propria persona l'ingresso della vettura. Sedette nel piccolo sedile accanto ad Annuška e diede un'occhiata in giro nella penombra della vettura letto. ``Grazie a Dio, domani vedrò Serëza ed Aleksej Aleksandrovic, e la mia buona vita d'ogni giorno scorrerà come prima''. 

Ancora in quello stato di inquietudine che l'aveva posseduta tutto il giorno, Anna si preparò con cura e piacere per il viaggio; con le piccole mani agili aprì e richiuse il sacchetto rosso, tirò fuori un piccolo guanciale, se lo pose sulle ginocchia e, avvoltesi accuratamente le gambe, sedette tranquilla. Una signora malata si disponeva già a dormire. Altre due signore presero a parlare con lei, mentre una vecchia obesa si ravvolgeva le gambe e faceva delle osservazioni sul riscaldamento. Anna rispose qualche parola alle signore, ma non prevedendo alcun interesse dalla conversazione, chiese ad Annuška di tirar fuori la lanterna da viaggio, l'appese al bracciolo della poltrona e prese dalla borsetta un tagliacarte e un romanzo inglese. In un primo tempo non le fu possibile di leggere. Le davano noia innanzi tutto il chiasso e l'andirivieni della gente; poi, quando il treno si mise in moto, non poté non prestare orecchio ai rumori, e la neve che picchiava sul finestrino di sinistra e si attaccava al vetro, la vista di un capotreno tutto imbacuccato che passava tutto ricoperto di neve da un lato solo, i discorsi sulla tormenta che infuriava distrassero la sua attenzione. Poi tutto divenne uniforme, il traballio interrotto da scosse, la neve al finestrino, gli improvvisi passaggi da un caldo di vaporazione al freddo e poi di nuovo al caldo, il baluginare di quegli stessi volti nella penombra e il suono delle stesse voci; e Anna prese a leggere e a capire quello che leggeva. Annuška già sonnecchiava, tenendo la sacca rossa sulle ginocchia con le mani larghe nei guanti, uno dei quali era bucato. Anna Arkad'evna leggeva e capiva, ma non provava piacere a leggere e a seguire il riflesso della vita degli altri. Aveva troppa voglia di viverla lei, la vita. Leggeva che l'eroina del romanzo vegliava un malato e le veniva voglia di camminare in punta di piedi per la camera del malato; leggeva che un membro del parlamento faceva un discorso e le veniva voglia di pronunciare lei quel discorso; leggeva che lady Mary inseguiva a cavallo un branco di bestie, provocando la cognata e facendo meravigliare tutti del suo ardire, e le veniva voglia di far lei tutto questo. Non c'era nulla da fare, invece, e rigirando il coltellino liscio tra le piccole mani, si sforzava di leggere. 

L'eroe del romanzo aveva già cominciato a raggiungere la sua felicità inglese, il titolo di baronetto e una tenuta, e Anna stava per desiderare di andare con lui in questa tenuta, quando improvvisamente sentì ch'egli avrebbe dovuto vergognarsi e che anche lei avrebbe dovuto vergognarsi di quella medesima cosa. Ma di che cosa mai egli doveva vergognarsi? ``Di che cosa mai mi vergogno io?'' si domandò con meraviglia offesa. Lasciò il libro e si abbandonò sulla spalliera della poltrona, stringendo forte il tagliacarte con tutte e due le mani. Nulla da aver vergogna. Esaminò tutti i suoi ricordi di Mosca. Erano tutti buoni, piacevoli. Ricordò il ballo, ricordò Vronskij e il suo viso innamorato, sottomesso, ricordò tutti i suoi rapporti con lui; non c'era nulla di cui vergognarsi. Ma intanto proprio a questo punto il senso di vergogna diveniva più forte, come se una certa voce interiore, proprio lì, nel punto in cui si ricordava di Vronskij, le dicesse: ``Caldo, caldo, scottante''. ``Ebbene - si disse decisa, cambiando posizione nella poltrona. - Che vuol dire ciò? Possibile che fra me e quel giovane ufficiale, quel ragazzo, esistano o possano esistere altri rapporti fuorché quelli che esistono con ogni altro conoscente?''. Sorrise con sprezzo e riprese di nuovo il libro; ma ormai davvero non poteva più afferrare quello che leggeva. Passò il tagliacarte sul vetro del finestrino, ne accostò la superficie liscia e fredda alla guancia e si mise quasi a ridere di un'allegrezza che si era impossessata di lei senza ragione. Sentiva che i nervi, come corde, si tendevano sempre di più come su cavicchi avvitantisi. Sentiva che gli occhi le si dilatavano, e che le dita delle mani e dei piedi le si muovevano nervosamente, che qualche cosa dentro le soffocava il respiro e che tutte le immagini e i suoni in quella penombra vacillante la colpivano con una impressionante chiarità. Ad ogni momento era assalita da attimi di dubbio: ``La vettura va avanti o indietro, o sta del tutto ferma? È Annuška vicino a me o una donna estranea? Che cosa c'è lì, sul bracciuolo, una pelliccia o una bestia? E che cosa sono io? Sono proprio io, o un'altra?''. Aveva paura di lasciarsi andare a questo vaneggiamento. Ma qualcosa ve l'attirava e a volontà ella poteva abbandonarvisi o trattenersene. Si alzò per rientrare in sé, gettò indietro lo scialle da viaggio e tolse la pellegrina dal vestito pesante. Per un attimo si riebbe, e capì che l'uomo allampanato che era entrato col cappotto lungo di nanchino al quale mancava un bottone, era un fochista che era venuto a guardare il termometro, e capì che la neve e il vento avevano fatto irruzione dietro di lui attraverso la porta; ma poi di nuovo tutto si confuse. L'uomo allampanato si metteva a rosicchiare qualcosa appoggiato alla parete, la vecchia cominciava ad allungare le gambe per tutta la lunghezza della vettura e la riempiva di un vapore nero; poi qualcosa di pauroso strideva e picchiava come se sbranassero qualcuno, infine un fuoco rosso accecava gli occhi e tutto veniva chiuso da un muro. Ad Anna parve di sprofondare. Eppure tutto questo non era terribile, ma esilarante. La voce di un uomo imbacuccato e ricoperto di neve le gridò qualcosa all'orecchio. Si alzò e ritornò in sé; capì che erano arrivati ad una stazione e che questi era il controllore. Pregò Annuška di darle la pellegrina che s'era tolta e lo scialle, se li mise e si diresse verso lo sportello. 

- Volete scendere? - chiese Annuška. 

- Sì, voglio prendere una boccata d'aria. Qui fa troppo caldo. 

E aprì lo sportello. La tormenta e il vento le si scagliarono addosso contrastandole lo sportello. La cosa la divertì. Aprì lo sportello e venne fuori. Fu come se il vento avesse atteso proprio lei; prese a fischiare con gioia e voleva afferrarla e portarla via, ma lei si aggrappò con una mano ad una colonnina gelata e, trattenendo l'abito, scese sulla banchina e passò dietro la vettura. Il vento era forte sulla scaletta, ma sulla banchina, dietro la vettura, c'era calma. Respirò con gioia, a pieni polmoni, l'aria di neve, gelida, e, in piedi accanto alla vettura, si mise a guardare tutt'intorno la banchina e la stazione illuminata. 
\enlargethispage*{1\baselineskip}

\capitolo{XXX}Una tormenta paurosa s'era scatenata e fischiava fra le ruote della vettura, lungo le colonne, al di là dell'angolo della stazione. Vetture, colonne, uomini; tutto quello che si poteva scorgere veniva ricoperto da un sol lato di neve e sempre di più se ne ricopriva. Per un attimo la tormenta parve calmarsi, ma poi di nuovo si sferrò con raffiche tali che sembrava non si potesse resisterle. Nel frattempo alcune persone corsero e, scambiando allegramente qualche parola, fecero scricchiolare le assi della banchina aprendo e richiudendo continuamente la porta grande. L'ombra contorta di un uomo scivolò sotto i piedi di lei e si udì il rumore di un martello sul ferro\ldots{}``Telegrafa!'' echeggiò una voce irritata dall'altra parte nel buio della tormenta. ``Favorite qua, n. 28!'' gridarono ancora altre voci e delle persone imbacuccate corsero, ricoperte di neve. Due signori, con le sigarette accese in bocca, le passarono accanto. Ella respirò ancora una volta per prendere aria a sazietà e aveva già tirato fuori la mano dal manicotto per afferrarsi alla colonnina e rientrare in vettura, quando accanto a lei un individuo dal cappotto militare le intercettò la luce vacillante del fanale. Si voltò e in quell'attimo riconobbe il viso di Vronskij. Portando la mano alla visiera, egli s'inchinò e domandò se avesse bisogno di qualcosa e se potesse esserle utile. Anna lo fissò a lungo senza rispondere nulla e, malgrado l'ombra in cui era, vedeva, o le sembrava di vedere, anche l'espressione del viso e degli occhi di lui. Ancora quell'espressione di reverente ammirazione che la sera prima l'aveva tanto impressionata. Più di una volta in quei giorni, e fino a pochi momenti prima, era andata ripetendo a se stessa che Vronskij era per lei uno dei cento giovanotti eternamente identici che s'incontrano dovunque, e che ella mai avrebbe concesso a se stessa di pensare a lui; ma ora, in quel primo attimo dell'incontro, fu presa da un senso di orgoglio gioioso. Non c'era bisogno di chiedere perché fosse là. Lo sapeva così sicuramente come s'egli avesse detto che si trovava là perché voleva essere dov'era lei. 

- Non sapevo che foste in viaggio. Perché viaggiate? - disse, abbassando la mano con la quale stava aggrappata alla colonnina. E un'irrefrenabile gioia e animazione le illuminarono il viso. 

- Perché viaggio? - ripeté lui, guardandola dritto negli occhi. - Voi sapete che io viaggio per essere dove siete voi - disse - e non posso fare altrimenti. 

Nello stesso tempo, come se avesse superato degli ostacoli, il vento spazzò via la neve dai tetti delle vetture, strascinò una lamiera di ferro ch'era riuscito a strappare, e il fischio della locomotiva ruggì, lugubre e cupo. A lei ora tutto l'orrore della tormenta pareva ancora più bello. Egli aveva detto proprio quello che l'anima sua desiderava, ma che la sua ragione temeva. Ella non rispondeva nulla, e sul viso di lei egli scorgeva la lotta interiore. 

- Perdonatemi se vi spiace quello che ho detto - disse umilmente. 

Parlava con cortesia, con rispetto, ma con tanta fermezza e ostinazione che per molto tempo ella non poté rispondere nulla. 

- È male quello che dite, e vi prego, se siete un gentiluomo, dimenticate quello che avete detto; anch'io dimenticherò - disse infine. 

- Non una vostra parola, non un vostro gesto dimenticherò mai, e non posso\ldots{} 

- Basta, basta! - gridò lei, cercando invano di dare un'espressione severa al viso che egli andava scrutando avidamente. E afferratasi con la mano alla colonnina gelida, montò sul predellino ed entrò in fretta nel corridoio della vettura. Ma nel piccolo ingresso si fermò per riflettere a quello che era accaduto. Non ricordava né le parole proprie, né quelle di lui, ma ebbe la sensazione che quella conversazione di pochi istanti li avesse terribilmente avvicinati e ne era spaventata e felice. Dopo esser rimasta in piedi per qualche secondo, entrò nello scompartimento e sedette al proprio posto. Quello stato di tensione che l'aveva tormentata poco prima non solo si rinnovò, ma aumentò sino a farle temere che da un momento all'altro si spezzasse in lei qualcosa di troppo teso. Non dormì tutta la notte. Ma in quella tensione e in quel vaneggiamento che le riempivano la mente, non c'era nulla di spiacevole e di tetro, al contrario, c'era qualcosa di gioioso e di eccitante. All'alba si assopì nella poltrona, e quando si svegliò era giorno chiaro e il treno si avvicinava a Pietroburgo. Il pensiero della casa, del marito, del figlio, le faccende di quel giorno e dei seguenti s'impossessarono subito di lei. 

A Pietroburgo, non appena il treno si fu fermato ed ella uscì, il primo viso che attirò la sua attenzione fu quello del marito: ``Ah, Dio mio, perché ha le orecchie fatte a quel modo?'' pensò guardando la figura di lui fredda e rappresentativa e in particolare le cartilagini delle orecchie che sostenevano le falde del cappello tondo e che in quel momento la colpivano. Egli, appena la vide, le andò incontro, atteggiando le labbra al sorriso canzonatorio che gli era abituale e guardandola diritto con i suoi grandi occhi stanchi. Una sensazione sgradevole le strinse il cuore quando incontrò lo sguardo di lui ostinato e stanco: come se avesse voluto vederlo diverso. La colpì soprattutto quello scontento di sé che aveva provato nell'incontrarlo. Era, questa, una sensazione da tempo provata, simile a quella sua mancanza di lealtà nei rapporti col marito; ma prima questa sensazione ella non la notava, ora invece la percepiva con chiarezza e con pena. 

- Be', lo vedi, hai un marito tenero, tenero come al primo anno di matrimonio: bruciava dal desiderio di vederti - disse lui con la sua voce sottile e strascicata e con quel tono che quasi sempre usava con lei, un tono di canzonatura verso chi avesse parlato così per davvero. 

- Serëza sta bene? - domandò lei. 

- E questa è tutta la tua ricompensa per il mio ardore? - disse lui. - Sta bene, sta bene\ldots{} 

\capitolo{XXXI}Per tutta quella notte Vronskij non tentò neppure d'addormentarsi. Sedeva sulla sua poltrona, ora con gli occhi fissi davanti a sé, ora osservando quelli che entravano e uscivano; e se anche prima egli colpiva e disorientava le persone che non lo conoscevano, per quella sua aria di imperturbabile indifferenza, ora sembrava ancor più pieno e soddisfatto di sé. Guardava agli uomini come a cose. Un impiegato del tribunale del distretto, un giovanotto nervoso seduto di fronte a lui, prese a detestarlo per quella sua aria. Il giovanotto accendeva la sigaretta a quella di Vronskij, cominciava a parlare con lui, lo urtava perfino per fargli sentire che non era una cosa, ma Vronskij lo guardava così come si guarda un fanale, e il giovanotto si contorceva, sentendo di perdere il dominio di se stesso, sottoposto alla pressione di quel mancato riconoscimento umano. 

Vronskij non vedeva nulla e nessuno. Si sentiva un dominatore, non perché credesse d'aver fatto colpo su Anna (a questo egli non credeva ancora), ma perché l'impressione che ella aveva prodotto su di lui lo rendeva felice e orgoglioso. 

Che cosa sarebbe venuto fuori da tutto questo, non lo sapeva, e non lo immaginava neppure. Sentiva che tutte le sue forze, fino ad ora rilasciate e disperse, si erano fuse e orientate con spaventosa energia verso un unico fine beato. E ne era felice. Sapeva solo di averle detto la verità, dicendole che andava là dov'era lei; sapeva che tutta la felicità della sua vita, l'unico senso della vita lo trovava adesso nel veder lei, nell'ascoltar lei. E quando era uscito dalla vettura a Bologovo per bere dell'acqua di seltz, e aveva visto Anna, involontariamente la prima cosa che aveva detto era stata proprio ciò che pensava. Ed era felice di averglielo detto, era felice ch'ella lo sapesse e ci pensasse. Non dormì tutta la notte. Da quando era rientrato in vettura, senza mutar posto, non aveva cessato di riandare con la mente a tutti gli atteggiamenti in cui l'aveva vista, a tutte le sue parole, mentre nell'immaginazione volteggiavano le figurazioni di un possibile futuro che lo facevano venir meno. 

Quando, a Pietroburgo, uscì dalla vettura, si sentiva, dopo la notte insonne, vivido e fresco come dopo un bagno freddo. Si fermò presso la vettura, ad aspettarla. ``La vedrò ancora una volta - si disse, sorridendo involontariamente - vedrò la sua andatura, il suo viso: dirà qualcosa, volgerà il capo, guarderà, riderà, forse''. Ma prima ancora di veder lei, vide il marito accompagnato con deferenza dal capostazione tra la folla. ``Ah, già, il marito!''. Solo in quel momento Vronskij capì per la prima volta con chiarezza che il marito era una persona legata a lei. Sapeva ch'ella aveva un marito, ma non credeva alla sua esistenza, e ci credette in pieno solo quando lo vide, con quella sua testa, con quelle sue spalle, con le gambe nei pantaloni neri; specialmente quando vide con quale senso di proprietà egli prendeva tranquillamente il braccio di lei. 

Nel vedere Aleksej Aleksandrovic, col viso rasato di fresco alla pietroburghese e col suo aspetto rigidamente sicuro di sé, col cappello a falde larghe, la schiena un po' curva, ci credette, e provò una sensazione sgradevole, simile a quella di un uomo che, tormentato dalla sete, e pervenuto a una fonte vi trovi un cane, una pecora o un maiale che, bevendo, ne abbia intorbidita l'acqua. L'andatura di Aleksej Aleksandrovic, che dimenava tutto il bacino movendo le gambe ad angolo ottuso, dava fastidio in modo particolare a Vronskij. Riconosceva solo a se stesso l'indubitabile diritto di amare lei. Ma lei era sempre la stessa, e la sua apparizione agì su di lui, come sempre, animandolo fisicamente, eccitandolo ed empiendogli l'anima di gioia. Ordinò al servitore tedesco, che gli veniva incontro correndo dalla seconda classe, di prender la roba e di andar via, e intanto si avvicinò a lei. Vide il primo incontro fra marito e moglie e osservò, con la penetrazione di chi ama, la leggera ombra di soggezione con la quale ella parlava col marito. ``No, non lo ama e non può amarlo'' decise fra sé e sé. Mentre si accostava alle spalle di Anna Arkad'evna, notò con gioia ch'ella aveva sentito il suo avvicinarsi e che stava per girarsi, ma che poi, riconosciutolo, si era rivolta nuovamente al marito. 

- Avete passata bene la notte? - chiese, inchinandosi dinanzi a lei e al marito insieme, lasciando ad Aleksej Aleksandrovic la facoltà di prender per sé quell'inchino e d'accettarlo oppur no a suo piacimento. 

- Grazie, molto bene - rispose lei. 

Il suo viso sembrava stanco e non v'era quel giuoco d'animazione che urgeva ora nel riso ora negli occhi; solo per un attimo, mentre lo guardava, qualcosa balenò nei suoi occhi, e sebbene questo fuoco si spegnesse subito, egli fu felice di quell'attimo. Ella guardò il marito per vedere se conosceva o no Vronskij. Aleksej Aleksandrovic guardava Vronskij con disappunto, cercando distrattamente di ricordarsi chi fosse. La sicurezza e la tranquillità di Vronskij in quel momento urtarono, come la falce nella selce, contro la fredda sicurezza di Aleksej Aleksandrovic. 

- Il conte Vronskij - disse Anna. 

- Ah, ci conosciamo, mi pare - disse con indifferenza Aleksej Aleksandrovic, dandogli la mano. - All'andata hai viaggiato con la madre, al ritorno col figlio - disse, pronunciando con precisione, come se elargisse un rublo ad ogni parola. - Probabilmente, tornate da una licenza? - disse e, senza aspettar risposta, si voltò alla moglie in tono scherzoso. - Dunque, molte lacrime sono state versate a Mosca al momento della separazione? 

Col rivolgersi alla moglie aveva fatto intendere a Vronskij che desiderava restar solo e, giratosi verso di lui, si toccò il cappello; ma Vronskij disse ancora ad Anna Arkad'evna: 

- Spero di aver l'onore di venire da voi - disse. 

Aleksej Aleksandrovic lo guardò con occhi stanchi. 

- Molto lieto - disse freddo; - riceviamo il lunedì. - Poi, dopo aver lasciato andar via definitivamente Vronskij, disse alla moglie: - È stato proprio bene che io abbia avuto mezz'ora di tempo per venirti a prendere e dimostrarti la mia tenerezza - egli continuò nello stesso tono scherzoso. 

- Tu insisti troppo su questa tua tenerezza, perché io possa apprezzarla - disse lei con lo stesso tono scherzoso, prestando involontariamente orecchio al suono dei passi di Vronskij che camminava dietro di loro. ``Ma che me ne importa?'' pensò, e cominciò a chiedere al marito come era stato senza di lei Serëza . 

- Oh, benissimo! Mariette dice che è molto caro e\ldots{} devo darti un dispiacere\ldots{} non ha sentito nostalgia di te, non certo come tuo marito. Ma ancora una volta merci, amica mia, di avermi regalato una giornata. Il nostro caro samovar sarà entusiasta - egli chiamava ``samovar'' la famosa contessa Lidija Ivanovna, perché sempre e per tutto si agitava e accalorava. - Ha domandato di te. E sai, ti posso dare un consiglio? Dovresti andare da lei oggi stesso. Perché le duole il cuore per ogni cosa. Ora poi, oltre tutti i suoi affanni, è preoccupata della riconciliazione degli Oblonskij. 

La contessa Lidija Ivanovna era un'amica di suo marito e il centro di uno di quei circoli della società di Pietroburgo al quale Anna era legata più intimamente che non a tutti gli altri per mezzo di suo marito. 

- Ma se le ho scritto. 

- Ma a lei occorre saper tutto per filo e per segno. Cerca di andarci, se non sei stanca, amica mia. Ora la carrozza te la farà venire Kondratij, e io vado al comitato. Finalmente non mangerò più solo - continuò Aleksej Aleksandrovic non più in tono scherzoso. - Tu non puoi credere come sia abituato\ldots{} 

E stringendole a lungo la mano, la fece salire in carrozza con un sorriso particolare. 

\capitolo{XXXII}La prima persona che venne incontro ad Anna in casa fu il figlio. Si lanciò verso di lei giù per la scala, malgrado le grida della governante, chiamando con un entusiasmo disperato: ``Mamma, mamma!''. Giunto di corsa fino a lei, le si appese al collo. 

- Ve lo dicevo io che era la mamma! - gridava alla governante. - Lo sapevo! 

Anche il figlio, così come il marito, produsse in Anna un senso di delusione. Se lo immaginava più bello di quanto non fosse in realtà. E dovette discendere fino alla realtà per compiacersi di come era. Ma in fondo anche così era delizioso, con i riccioli biondi, gli occhi azzurri e le gambette piene e ben fatte nelle calze attillate. Anna provava una gioia quasi fisica nel sentirsi vicino a lui e una tenerezza e una calma morale quando incontrava lo sguardo suo leale, fiducioso e tenero e ne ascoltava le domande ingenue. Tirò fuori i regali che avevano mandato i bambini di Dolly e raccontò al figlio come a Mosca ci fosse una bimba, una certa Tanja, la quale sapeva già leggere e insegnare perfino agli altri bambini. 

- Ma forse io sono meno bravo di lei? - chiese Serëza. 

- Per me tu sei il più bravo di tutti al mondo. 

- Lo so - disse Serëza, sorridendo. 

Anna aveva appena fatto in tempo a prendere il caffè che le annunciarono la contessa Lidija Ivanovna. La contessa Lidija Ivanovna era una donna alta e grossa, dal colorito giallastro e malato e dagli occhi neri, belli e pensosi. Anna le voleva bene, ma quel giorno era come se la vedesse per la prima volta con tutti i suoi difetti. 

- Dunque, amica mia, avete portato il ramoscello d'olivo? - chiese la contessa Lidija Ivanovna entrando nella stanza. 

- Già, tutto si è concluso; ma la cosa non era poi così grave come credevamo - rispose Anna. - In generale, la mia belle soeur è troppo impulsiva. 

Ma la contessa Lidija Ivanovna, che si interessava di tutto quello che non la riguardava, e aveva l'abitudine di non ascoltare mai quello che avrebbe potuto interessarla, interruppe Anna. 

- Ah, c'è molto dolore e molta cattiveria nel mondo e io oggi sono così sfinita. 

- Che c'è? - domandò Anna, cercando di trattenere un sorriso. 

- Comincio a stancarmi di spezzare inutilmente lance in favore della verità, e a volte mi sento proprio snervata. L'affare delle Piccole Suore - era questa un'istituzione religioso-patriottica - andava già a meraviglia; ma con quei signori non si può far nulla - aggiunse in tono di ironica rassegnazione. - Si sono afferrati a un'idea, l'hanno travisata e, dopo tutto, ragionano con molta meschinità e piccineria. Due o tre persone sole, fra le quali vostro marito, comprendono il significato di quest'opera; ma gli altri la lasciano cadere. Ieri mi ha scritto Pravdin\ldots{} 

Pravdin era un noto panslavista all'estero e la contessa Lidija Ivanovna riferì il contenuto della sua lettera. 

Dopo di che la contessa parlò anche delle contrarietà e delle insidie contro la questione dell'unione delle chiese, e se ne andò in fretta, perché in quel giorno doveva andare alla seduta di un'associazione e al comitato slavo di beneficenza. 

``Certo questo suo modo di fare è lo stesso di prima, ma perché prima non lo notavo? - si diceva Anna. - O forse oggi è molto eccitata? Comunque, è ridicolo: il suo scopo è la virtù, è una cristiana, ma è sempre in collera e vede sempre nemici in nome della cristianità e della virtù''. 

Dopo la contessa Lidija Ivanovna venne un'amica, la moglie di un direttore, la quale raccontò tutte le novità della città. Alle tre se ne andò anche lei, promettendo di venire a pranzo. Aleksej Aleksandrovic era al ministero. Rimasta sola, Anna occupò il tempo, fino all'ora del pranzo, nell'assistere al pasto del figlio (egli pranzava separatamente), nel mettere in ordine le sue cose, nel leggere e rispondere ai biglietti e alle lettere che le si erano ammonticchiati sullo scrittoio. 

Il senso di inspiegabile vergogna che aveva provato in viaggio e l'agitazione erano completamente scomparsi. Nelle condizioni abituali di vita si sentiva di nuovo sicura e irreprensibile. 

Ricordava con stupore il suo stato del giorno prima. ``Ma cos'era mai? Nulla. Vronskij ha detto una sciocchezza alla quale è stato facile porre fine, e io ho risposto proprio come conveniva. Parlare a mio marito non si deve e non si può. Parlarne, sarebbe come dare importanza a ciò che non ne ha''. Ricordò d'aver raccontato al marito di un accenno di dichiarazione che le aveva fatto a Pietroburgo un giovane dipendente di lui. Aleksej Aleksandrovic le aveva risposto che ogni donna, vivendo nel gran mondo, poteva essere esposta a cose simili, ma ch'egli fidava completamente nel suo tatto e che mai si sarebbe permesso di abbassare lei e se stesso alla gelosia. ``Dunque, non c'è ragione di parlarne. Ma, grazie a Dio, non c'è neanche nulla da dire'' si disse. 

\capitolo{XXXIII}Aleksej Aleksandrovic tornò dal ministero alle quattro, ma, come spesso gli accadeva, non fece in tempo a passare da lei. Entrò nello studio a ricevere i sollecitatori che aspettavano e a firmare alcune carte portate dal capogabinetto. A pranzo (dai Karenin erano invitati a pranzo sempre un tre persone) vennero: una vecchia cugina di Aleksej Aleksandrovic, il direttore del dipartimento con la moglie e un giovanotto raccomandato ad Aleksej Aleksandrovic per un posto. Anna entrò in salotto per intrattenerli. Alle cinque in punto (l'orologio di bronzo in stile Pietro I non aveva finito di battere il quinto tocco) entrò Aleksej Aleksandrovic in cravatta bianca e in frac con due decorazioni, perché subito dopo pranzo doveva andar via. Ogni minuto della vita di Aleksej Aleksandrovic era impegnato e ripartito. E per riuscire a sbrigare quello che doveva fare ogni giorno, si atteneva alla più stretta puntualità. ``Senza fretta, ma senza tregua'' era il suo motto. Entrò frettoloso in sala, salutò tutti e, sorridendo alla moglie, sedette. 

- Sì, è finita la mia solitudine. Non puoi credere come sia spiacevole - egli marcò la parola ``spiacevole'' - pranzare da solo. 

A pranzo parlò con la moglie delle faccende di Mosca; con un sorriso canzonatorio chiese di Stepan Arkad'ic; ma la conversazione, prevalentemente generale, si aggirò sulle questioni amministrative e sociali di Pietroburgo. Dopo pranzo egli passò una mezz'ora con gli ospiti e, stretta di nuovo la mano alla moglie con un sorriso, uscì e andò al consiglio. Anna non andò questa volta né dalla principessa Betsy Tverskaja che, saputo del suo ritorno, l'aveva invitata per la serata, né a teatro dove quella sera aveva un palco. Non andò soprattutto perché il vestito sul quale contava non era pronto. Quando gli ospiti se ne andarono, dato uno sguardo generale al guardaroba, Anna s'indispettì molto. Prima della sua partenza per Mosca ella, che in genere era abilissima nel vestirsi senza spendere eccessivamente, aveva dato a rimodernare tre abiti alla sarta. Bisognava rifare i vestiti in modo da non farli riconoscere e dovevano essere pronti già da tre giorni. Invece due vestiti non lo erano affatto ed il terzo non era riuscito come avrebbe voluto lei. La sarta era venuta a giustificarsi e aveva sostenuto che in quel modo andava bene, e Anna si era indispettita tanto da provarne rimorso, dopo, al ricordo. Per rasserenarsi completamente era andata nella camera del bambino e aveva passato tutta la serata col figlio; lo aveva messo lei stessa a letto, gli aveva fatto il segno della croce e gli aveva rimboccato le coperte. Era felice di non essere andata in nessun posto e di aver passato così bene la serata. Si sentiva leggera e tranquilla e vedeva chiaramente che tutto quello che in viaggio le era parso così importante non era che uno degli insignificanti, comuni casi della vita mondana di cui non aveva da vergognarsi, né dinanzi a sé stessa, né dinanzi ad altri. Sedette presso il camino con in mano il romanzo inglese e aspettò il marito. Alle nove e mezzo in punto si udì la sua scampanellata ed egli entrò nella stanza. 

- Finalmente sei tu! - disse lei, tendendogli la mano. Egli le baciò la mano e le sedette accanto. 

- Vedo che il tuo viaggio è andato bene, nel complesso - disse. 

- Sì, molto - rispose lei, e cominciò a raccontargli tutto dal principio: il viaggio con la Vronskaja, l'arrivo, la disgrazia alla stazione. Dopo disse della sua impressione di pena provata prima per il fratello e poi per Dolly. 

- Io non credo che si possa scusare un uomo simile, anche se è tuo fratello - disse Aleksej Aleksandrovic severo. 

Anna sorrise. Capì che egli aveva detto ciò proprio per mostrare che le considerazioni di parentela non potevano trattenerlo dall'esprimere con franchezza la propria opinione. Conosceva questo tratto in suo marito e le piaceva. 

- Sono contento che tutto sia finito felicemente e che tu sia tornata - continuò. - Ebbene, che cosa dicono della nuova tesi che ho fatto passare al consiglio? 

Anna non aveva sentito dir nulla di questa tesi e si pentì d'aver dimenticato con tanta leggerezza quello che per lui era così importante. 

- Qui, al contrario, ha fatto molto scalpore - disse lui con un sorriso di compiacimento. 

Ella vedeva che Aleksej Aleksandrovic voleva comunicarle qualcosa che lo lusingava a proposito di quella questione, e, interrogandolo, lo portò a raccontare. 

Con lo stesso sorriso di compiacimento egli parlò delle ovazioni che gli erano state fatte in seguito all'approvazione della tesi. 

- Ne sono stato molto contento. Questo dimostra che finalmente da noi comincia a consolidarsi un'opinione ragionevole e decisiva su questa faccenda. 

Dopo aver preso il suo secondo bicchiere di tè con panna e pane, Aleksej Aleksandrovic si alzò e si diresse nello studio. 

- E tu non sei andata in nessun posto? Ti sarai annoiata, probabilmente - disse. 

- Oh, no! - rispose lei, alzandosi dietro di lui per accompagnarlo nello studio. - Cosa mai leggi ora? - domandò. 

- Sto leggendo la Poésie des enfers del Duc de Lille - rispose lui. - È un libro molto interessante. 

Anna sorrise, come si sorride alla debolezza delle persone care, e, posto il braccio sotto quello di lui, lo accompagnò fino alla soglia dello studio. Conosceva la sua abitudine, che era ormai una necessità, di leggere la sera. Sapeva che, malgrado i doveri d'ufficio che assorbivano quasi tutto il suo tempo, egli considerava doveroso seguire quanto di più notevole appariva nel mondo della cultura. Sapeva pure che in realtà lo interessavano solo i libri di politica, di filosofia e di teologia; che l'arte era del tutto estranea alla sua natura, ma che nonostante questo, o meglio per questo, Aleksej Aleksandrovic non trascurava nulla che avesse successo in questo campo e considerava suo dovere leggere tutto. Sapeva che nel campo della politica, della filosofia, della teologia Aleksej Aleksandrovic aveva dei dubbi o faceva delle ricerche; ma che nelle questioni di arte e di poesia, in particolare nella musica, del cui senso era completamente sprovvisto, aveva le più ristrette e tenaci convinzioni. Gli piaceva parlare di Shakespeare, di Raffaello, di Beethoven, del valore delle nuove correnti poetiche e musicali che venivano tutte classificate da lui con una logica molto chiara. 

- E Dio sia con te - disse lei presso la porta dello studio dove già gli erano stati preparati un paralume sulla candela e una caraffa d'acqua accanto alla poltrona. - Io intanto scriverò a Mosca. 

Egli le strinse la mano e la baciò di nuovo. 

``Però è un brav'uomo, leale, di buon cuore e notevole nel suo campo - si andava dicendo Anna, tornata in camera sua; quasi a difenderlo di fronte a qualcuno che lo accusasse e che dicesse a lei che non lo si poteva amare. - Ma come mai ha le orecchie che gli sporgono così stranamente in fuori? Forse si è tagliato i capelli''. 

A mezzanotte in punto, quando Anna era ancora seduta allo scrittoio terminando una lettera a Dolly, si udirono dei passi eguali e Aleksej Aleksandrovic, in pantofole, lavato e pettinato, col libro sotto al braccio, si accostò a lei. 

- È ora, è ora - disse, sorridendo in modo particolare, e si diresse in camera. 

``E quale diritto aveva di guardarlo così?'' pensò Anna, ricordando lo sguardo di Vronskij su di Aleksej Aleksandrovic. 

Spogliatasi, Anna entrò in camera, ma sul suo volto non solo non c'era più quell'animazione che durante il soggiorno a Mosca le balenava tra gli occhi e il riso, ma al contrario il fuoco sembrava ormai spento in lei, oppure nascosto in qualche parte, lontano. 

\capitolo{XXXIV}Partendo da Pietroburgo, Vronskij aveva lasciato il suo grande appartamento nella Morskaja all'amico e compagno carissimo Petrickij. 

Petrickij era un giovane tenente non di alto lignaggio, e non solo non ricco, ma affogato nei debiti, sempre brillo verso sera e spesso agli arresti per varie scabrose e ridicole storie, ma amato dai compagni e dai superiori. Verso le undici, tornando a casa dalla stazione, Vronskij vide dinanzi al portone una vettura da nolo a lui nota. Alla sua scampanellata, attraverso la porta, sentì un riso di uomini, il balbettio di una voce femminile e il grido di Petrickij: ``Se è qualche manigoldo, che non entri!''. Vronskij ordinò all'attendente di annunciarlo, e pian piano entrò nella prima stanza. La baronessa Shilton, l'amica di Petrickij, col viso roseo e chiaro e tutta luccicante nel raso lilla del vestito, sedeva alla tavola rotonda, intenta a far bollire il caffè, e come un canarino riempiva tutta la stanza della sua parlata parigina. Petrickij in cappotto e il capitano di cavalleria Kamerovskij in uniforme completa, reduci probabilmente dal servizio, sedevano vicino a lei. 

- Bravo! Vronskij! - gridò Petrickij, alzandosi e facendo rumore con la sedia. - Il padrone in persona! Baronessa, del caffè dalla caffettiera nuova! Ecco, non ti si aspettava proprio! Spero che tu sia contento del nuovo ornamento del tuo studio - disse indicando la baronessa. - Vi conoscete, vero? 

- Altro che - disse Vronskij sorridendo allegramente e stringendo la piccola mano della baronessa. - E come! Una vecchia amica! 

- Be', tornate a casa da un viaggio - disse la baronessa. - E allora io me ne vado via di corsa. Ah, me ne vado via in questo momento, se do fastidio. 

- Siete a casa vostra, baronessa - disse Vronskij. - Salve, Kamerovskij - soggiunse, stringendo freddamente la mano di Kamerovskij. 

- Ecco, voi non sapete mai dirmi delle cose così gentili - disse la baronessa rivolta a Petrickij. 

- No, perché? Dopo pranzo vedrete che non ne dirò di peggiori. 

- Già, dopo pranzo non c'è merito! Su, allora, vi darò del caffè; andate intanto a lavarvi e a mettervi in ordine - disse la baronessa sedendosi di nuovo e girando con premura una vite nella caffettiera nuova. 

- Pierre, datemi il caffè - disse a Petrickij che chiamava così dal cognome Petrickij, senza nascondere i suoi rapporti con lui. - Ne aggiungo dell'altro. 

- Ma lo sciupate! 

- No, che non lo sciupo. Su, e la vostra sposa? - disse subito la baronessa interrompendo la conversazione di Vronskij col compagno. - Noi qui vi abbiamo ammogliato. Avete portato vostra moglie? 

- No, baronessa. Zingaro son nato e zingaro morirò. 

- Tanto meglio, tanto meglio. Qua la mano. 

E la baronessa, senza lasciare andare Vronskij, prese a raccontargli i suoi ultimi progetti di vita, infiorandoli di scherzi, e chiedendogli consigli. 

- Lui non vuole ancora consentire al divorzio. E allora che debbo fare? - ``Lui'' era suo marito. - Voglio iniziare il processo, perché ho bisogno di un patrimonio mio. Cosa mi consigliate? Kamerovskij, badate al caffè\ldots{} esce fuori; vedete, io sto parlando d'affari. Capite forse quest'assurdità? io gli sarei infedele - diceva lei con sprezzo - e lui per questo vuole usufruire della mia proprietà. 

Vronskij ascoltava volentieri l'allegro cinguettio di quella donna carina; le diceva di sì, le dava consigli scherzando e, in complesso, andava riprendendo rapidamente il suo tono abituale con le donne di questa specie. Nel suo mondo pietroburghese tutte le persone si dividevano in due categorie perfettamente opposte. Una, la categoria inferiore, si componeva di persone comuni, sciocche e soprattutto ridicole, le quali credevano che il marito dovesse vivere soltanto con la donna con la quale s'era sposato, che una ragazza dovesse essere innocente, la donna pudica, l'uomo virile, temperato e forte, che bisognasse educare i propri figli, provvedere al proprio pane, pagare i debiti e altre sciocchezze simili. Questa era la categoria delle persone fuori moda e ridicole. Ma c'era un'altra categoria, quella delle persone alla moda, alla quale tutti loro appartenevano, e nella quale bisognava essere soprattutto belli, eleganti, spenderecci, arditi, allegri e capaci di abbandonarsi a qualsiasi passione senza arrossire e ridendosi di tutto. 

Vronskij era rimasto stordito solo il primo momento, dopo le impressioni che aveva riportato da Mosca di un mondo del tutto diverso; ma poi, subito, come se avesse infilato i piedi in un vecchio paio di pantofole, entrò nell'allegro e piacevole suo mondo di prima. 

Il caffè infatti non arrivò neanche a bollire, che schizzò tutti e andò di fuori, producendo proprio quello che occorreva: versandosi su di un tappeto di valore e sul vestito della baronessa, offrì il pretesto al chiasso e al riso. 

- Su, allora, addio, altrimenti non vi laverete mai e sulla mia coscienza graverà il più grosso delitto d'un uomo per bene: la sporcizia. Dunque, voi mi consigliate di mettergli il coltello alla gola? 

- Proprio così, e in modo tale che la vostra manina si trovi vicina alle sue labbra. Egli bacerà la vostra mano e tutto andrà nel modo migliore - disse Vronskij. 

- Allora a stasera, al Teatro Francese! - E frusciando col vestito, scomparve. 

Kamerovskij si alzò anche lui, e Vronskij, senza aspettare che fosse uscito, gli diede la mano e si diresse nel bagno. Mentre si lavava, Petrickij gli descrisse in breve la propria situazione, tanto mutata dopo la partenza di Vronskij. Denaro niente. Il padre aveva detto che non ne avrebbe dato e che non avrebbe pagato debiti. Il sarto lo voleva fare arrestare e anche un altro lo minacciava decisamente di farlo schiaffar dentro. Il comandante del reggimento aveva dichiarato che, se tutti questi scandali non fossero finiti, egli avrebbe dovuto dare le dimissioni. La baronessa gli era venuta a noia come una radica amara, e soprattutto perché voleva continuamente dargli del denaro; ma ce n'era una, che egli poi avrebbe mostrata a Vronskij, una meraviglia, un amore, di perfetto stile orientale, ``genre schiava Rebecca, capisci''. Anche con Berkošëv aveva litigato e gli voleva mandare i padrini, ma, naturalmente, non ne sarebbe venuto fuori nulla. In complesso tutto era eccellente, e straordinariamente allegro. E senza dare all'amico la possibilità di approfondire i particolari di questa situazione, Petrickij si diede a raccontargli tutte le novità interessanti. Ascoltando i racconti così noti di Petrickij, in quell'atmosfera ancor più nota dell'appartamento che occupava da tre anni, Vronskij provava un piacevole senso di ritorno all'abituale spensierata vita di Pietroburgo. 

- Non può essere! - gridò, lasciando il pedale del lavabo, nel quale bagnava il collo rosso e sano. - Non può essere! - gridò alla notizia che Lora s'era unita con Mileev e aveva piantato Fertigov. - E lui, sempre così balordo e soddisfatto? Su, e di Buzulukov che ne è? 

- Ah, con Buzulukov c'è stata una storia, una delizia! - gridò Petrickij. - Dunque, la passione sua sono i balli, e non se ne perde nemmeno uno di quelli a corte. Era andato al gran ballo con l'elmo nuovo. Hai visto gli elmi nuovi? Molto belli, leggeri. Eccolo, è lì in piedi\ldots{} Su, ascolta. 

- Sì, che ascolto - rispose Vronskij, fregandosi con un asciugamano a spugna. 

- Passa una granduchessa con un ambasciatore, e, per disgrazia sua, il discorso cade sugli elmi nuovi. La granduchessa vuole mostrare l'elmo nuovo. Guardano, e il nostro giovincello sta lì impalato - Petrickij lo rifaceva così come stava, lì ritto con l'elmo sotto al braccio. - La granduchessa gli chiede di darle l'elmo. Lui, niente. Che succede? Non fanno che ammiccargli, fargli gesti, aggrottar le sopracciglia. Dàglielo. Non lo dà. Pare un morto. Ti puoi figurare\ldots{} Ma quello\ldots{} come si chiama\ldots{} vuol prendere l'elmo\ldots{} lui niente, non lo dà! Quello glielo strappa, lo dà alla granduchessa: ``Ecco l'elmo nuovo'' dice la granduchessa. Volta l'elmo, e figurati, dall'elmo, giù una pera, dei confetti, due libbre di confetti. Li aveva raccolti, poverino! 

Vronskij scoppiò a ridere. E a lungo dopo, quando già parlavano d'altro, se gli tornava in mente l'elmo, scoppiava a ridere del suo riso sano che metteva in mostra i denti regolari e forti. 

Sapute tutte le novità, Vronskij, con l'aiuto del servitore, si mise in uniforme per andare a presentarsi. Voleva poi, dopo essersi presentato, passare dal fratello, da Betsy e fare alcune visite per cominciare a entrare in quella società nella quale avrebbe potuto incontrare la Karenina. Come sempre a Pietroburgo, uscì di casa con l'intenzione di rientrarvi a notte alta. 