\parte{PARTE OTTAVA}\label{parte-ottava} 
\pagestyle{pagina}

\capitolo{I}\label{i-7} 

Erano passati quasi due mesi. Si era già alla metà di un'estate calda, e Sergej Ivanovic, soltanto adesso, era pronto a lasciare Mosca. 

La vita di Sergej Ivanovic aveva avuto, nel frattempo, i suoi avvenimenti. Già da un anno circa egli aveva finito il suo libro, frutto di un lavoro di sei anni, intitolato Saggio di una rassegna delle basi e delle forme di stato in Europa e in Russia. Alcune parti di questo libro e l'introduzione erano state pubblicate in periodici, e altre erano state lette da Sergej Ivanovic a persone del suo ambiente, così che le idee di questo lavoro non potevano mai essere una novità assoluta per il pubblico; tuttavia Sergej Ivanovic si aspettava che il suo libro, uscendo, dovesse produrre una seria impressione sulla società e, se non proprio una rivoluzione nella scienza, in ogni caso un grande fermento nel mondo scientifico. 

Questo libro, dopo un accurato lavoro di lima, era stato pubblicato l'anno prima e spedito ai librai. 

Senza domandarne a nessuno, rispondendo svogliatamente e con finta indifferenza alle domande degli amici su come andava il libro, senza chiederne neppure ai librai se veniva comprato, Sergej Ivanovic aveva spiato con vigilanza, con ansia, la prima impressione che il suo libro produceva in società e fra i letterati. 

Ma passò una settimana, ne passarono due, tre e nella società non si notava alcuna impressione; gli amici specialisti e studiosi, a volte, evidentemente per cortesia, ne cominciavano a parlare. Ma gli altri suoi conoscenti, non interessati a un libro di contenuto scientifico, non ne parlavano affatto. E nella società, attirata in questo momento da altri interessi, vi era una completa indifferenza. Anche nelle critiche letterarie, per tutto un mese, non ci fu neppure una parola sul libro. 

Sergej Ivanovic calcolava fin nei particolari il tempo necessario per scrivere una recensione; ma passò un mese, ne passò un altro, sempre lo stesso silenzio. 

Soltanto nel ``Severnyj zuk'', in uno scherzoso articolo sul cantante Drabanti, che aveva perso la voce, erano dette, a questo proposito, alcune parole sprezzanti sul libro di Koznyšev, che mostravano che il libro, già da tempo, era condannato da tutti e abbandonato all'irrisione generale. 

Finalmente il terzo mese, in una rivista seria, apparve un articolo critico. Sergej Ivanovic conosceva l'autore dell'articolo. L'aveva incontrato una volta da Golubcov. 

L'autore dell'articolo era un giornalista molto giovane e malato, molto vivace come scrittore, ma straordinariamente incolto e timido nei rapporti personali. 

Malgrado il suo assoluto disprezzo per l'autore, Sergej Ivanovic si accinse alla lettura dell'articolo con piena considerazione. L'articolo era orribile. 

Evidentemente, l'articolista aveva inteso il libro in modo da renderne impossibile l'interpretazione. Ma aveva disposto così bene le citazioni che, per coloro i quali non avevano letto il libro (ed evidentemente quasi nessuno lo aveva letto), era del tutto chiaro che il libro non era altro che un cumulo di parole altisonanti, e per di più adoperate a sproposito (cosa che i punti interrogativi mettevano in rilievo), e che l'autore era una persona completamente ignorante. E tutto ciò era così spiritoso che neppure Sergej Ivanovic avrebbe respinto uno spirito simile; e appunto questo era orribile. 

Malgrado l'assoluta coscienziosità con cui Sergej Ivanovic controllava la giustezza degli argomenti del recensore, non si fermò neppure un attimo sui difetti e sugli errori che gli erano stati derisi, perché era troppo evidente che tutto questo era stato fatto con intenzione; tuttavia subito, involontariamente, riandò col pensiero, fin nei più piccoli particolari, al suo incontro e alla sua conversazione con l'autore dell'articolo. 

- Che l'abbia offeso in qualche modo? - si chiedeva Sergej Ivanovic. 

E ricordatosi come, nell'incontro, avesse corretto quel giovane in una parola che rivelava la sua ignoranza, Sergej Ivanovic trovò la spiegazione del senso dell'articolo. 

Dopo questo articolo seguì un silenzio di morte sul libro, sia da parte della stampa che della pubblica opinione, e Sergej Ivanovic vedeva che la sua opera, frutto di sei anni di lavoro, elaborata con tanto amore e tanta fatica, era passata senza lasciar traccia. 

La situazione di Sergej Ivanovic era ancor più penosa per il fatto che finito il libro, egli non aveva più l'occupazione dello scrivere, che prima prendeva tanta parte del suo tempo. 

Sergej Ivanovic era un uomo intelligente, colto, sano, attivo e non sapeva come adoperare la propria attività. I discorsi nei salotti, nei congressi, nelle riunioni, nei comitati, dovunque si parlasse, occupavano una parte del suo tempo; ma, vecchio abitante di città, non si concedeva di perdersi tutto in discorsi, come il suo inesperto fratello quand'era a Mosca; gli rimanevano così ancora molto tempo libero e molte energie intellettuali. 

Per sua fortuna, in quel periodo per lui penoso a causa dell'insuccesso del libro, in luogo delle questioni dei credenti di altre fedi, degli amici americani, della carestia di Samara, dell'esposizione, dello spiritismo, era sorta la questione slava, che fino allora languiva in seno alla società, e Sergej Ivanovic, che anche prima ne era stato uno dei promotori, vi si dedicò completamente. 

Nella cerchia delle persone a cui apparteneva Sergej Ivanovic, in quel momento non si scriveva altro che della guerra serba. Tutto quello che fa di solito una folla oziosa, per ammazzare il tempo, adesso si faceva a beneficio degli slavi. I balli, i concerti, i pranzi, i discorsi, le acconciature femminili, la birra, le trattorie, tutto testimoniava la simpatia per gli slavi. 

Con gran parte di quello che si diceva e si scriveva in quell'occasione, Sergej Ivanovic non era d'accordo nei particolari. Egli vedeva che la questione slava era diventata una di quelle questioni di moda che sempre, sostituendosi le une alle altre, servono alla società come materia d'interesse; vedeva che c'erano molte persone, che avevano scopi interessati, ambiziosi, che si occupavano di quella impresa. Riconosceva che i giornali stampavano molte cose inutili ed esagerate col solo scopo di richiamare l'attenzione e di gridare più degli altri. Vedeva che, in quella generale infatuazione della società, erano usciti fuori e gridavano più forte degli altri tutti i falliti e gli offesi; comandanti in capo senza eserciti, ministri senza ministero, giornalisti senza giornali, capipartito senza partito. Vedeva che in questo c'era molto di vacuo e ridicolo; ma vedeva e riconosceva un indubitabile e sempre crescente entusiasmo che aveva riunito in un tutto unico le classi della società, e per il quale non si poteva non aver simpatia. Il massacro dei correligionari e dei fratelli slavi aveva suscitato la simpatia verso coloro che soffrivano e l'indignazione contro gli oppressori. E l'eroismo dei serbi e dei montenegrini, che lottavano per una grande causa, aveva generato in tutto il popolo il desiderio di aiutare i fratelli non più con la parola, ma con l'azione. 

Inoltre c'era un altro fenomeno, soddisfacente per Sergej Ivanovic: era questo il manifestarsi di una opinione pubblica. La società aveva espresso in modo preciso il proprio desiderio. L'anima popolare aveva ricevuto un'espressione, come diceva Sergej Ivanovic. E quanto più egli si occupava di tale impresa, tanto più evidente gli appariva come questa impresa dovesse assumere proporzioni enormi, dovesse, cioè, fare epoca. 

Egli consacrò tutto se stesso al servizio di questa grande impresa, e dimenticò di pensare al suo libro. 

Adesso il suo tempo era occupato, così che non riusciva a rispondere a tutte le lettere e alle richieste che gli venivano rivolte. 

Dopo aver lavorato tutta la primavera e parte dell'estate, soltanto nel mese di luglio era pronto per andare in campagna dal fratello. 

Andava a riposarsi per un paio di settimane, e proprio nel sacrario del popolo, nella solitudine della campagna, andava a godere la visione di quel risveglio dello spirito nazionale, del quale lui e tutti gli abitanti di città erano pienamente convinti. Katavasov, che da lungo tempo voleva mantenere la promessa fatta a Levin di essere per un po' suo ospite, era partito insieme con lui. 

\capitolo{II}\label{ii-7} 

Sergej Ivanovic e Katavasov avevano appena fatto in tempo ad avvicinarsi alla stazione, quel giorno particolarmente animata di gente, della ferrovia di Kursk, scendere dalla carrozza e guardare il cameriere che li seguiva con la roba, che sopraggiunsero anche dei volontari su quattro vetture da nolo. Alcune signore con dei fasci di fiori andarono loro incontro, e i volontari, accompagnati dalla folla che s'era precipitata dietro di loro, entrarono nella stazione. 

Una delle signore che erano andate incontro ai volontari, uscendo dalla sala, si rivolse a Sergej Ivanovic. 

- Voi pure siete venuto ad accompagnarli? - domandò in francese. 

- No, parto, principessa. Vado a riposarmi da mio fratello. E voi accompagnate sempre? - disse Sergej Ivanovic con un sorriso appena percettibile. 

- Sì, non si può far diversamente! - rispose la principessa. - È vero che da noi ne sono partiti già ottocento? Malvinskij non mi credeva. 

- Più di ottocento. Se si contano quelli che sono stati inviati non direttamente da Mosca, già più di mille - disse Sergej Ivanovic. 

- Ecco. Lo dicevo, appunto! - soggiunse gioiosa la signora. - Ed è vero che adesso è stato offerto quasi un milione? 

- Di più, principessa. 

- E qual'è il comunicato di oggi? Hanno battuto di nuovo i turchi. 

- Sì, ho letto - rispose Sergej Ivanovic. Parlavano dell'ultimo bollettino, il quale confermava che per tre giorni di seguito i turchi erano stati battuti su tutti i punti e fuggivano, e che per l'indomani si prevedeva un combattimento decisivo. 

- Ah, sì, sapete, un ottimo giovane ha chiesto di andare. Non so perché abbiano fatto delle difficoltà. Vi volevo pregare, io lo conosco, scrivete un biglietto, per favore. È mandato dalla contessa Lidija Ivanovna. 

Dopo aver domandato i particolari che la principessa conosceva sul giovane che chiedeva di partire, Sergej Ivanovic, passato in prima classe, scrisse un biglietto a colui dal quale dipendeva la cosa e lo consegnò alla principessa. 

- Sapete, il conte Vronskij, il famoso\ldots{} parte con questo treno - disse la principessa con un sorriso trionfante e significativo, mentre egli, ritrovatala, le consegnava il biglietto. 

- Ho sentito che partiva, ma non sapevo quando. Con questo treno? 

- L'ho visto. È qui: la madre sola lo accompagna. Tuttavia è questa la cosa migliore che potesse fare. 

- Oh sì, s'intende. 

Mentre parlavano, la folla passò con furia accanto a loro verso la tavola da pranzo. Anche loro si avvicinarono e sentirono la voce forte di un signore che, con una coppa in mano, faceva un discorso ai volontari. ``Servire per la fede, per l'umanità, per i nostri fratelli - diceva il signore, alzando sempre più la voce. - La madre Mosca vi benedice per la grande impresa. zivio!'' - egli concluse forte e con le lacrime agli occhi. 

Tutti gridarono ``zivio'' e ancora una nuova folla irruppe nella sala e fece quasi cadere la principessa. 

- Eh, principessa, che discorso! - disse, esplodendo di un sorriso gioioso, Stepan Arkad'ic, che era comparso a un tratto in mezzo alla folla. - Non è vero che ha parlato bene? con calore? Bravo! Anche Sergej Ivanic! Ecco, sarebbe bene che anche voi, da parte vostra, parlaste così! qualche parola, sapete, un incoraggiamento, voi lo fate così bene - soggiunse, con un sorriso affabile, rispettoso e prudente, spingendo leggermente per un braccio Sergej Ivanovic. 

- No, parto subito. 

- Dove? 

- In campagna, da mio fratello - rispose Sergej Ivanovic. 

- Allora vedrete mia moglie. Le ho scritto, ma voi la vedrete prima; per favore, ditele che mi avete visto e che all right. Lei capirà. Ma del resto, ditele, siate buono, che sono stato nominato membro dell'agenzia\ldots{} Su, ma lei capirà! Sapete, les petites misères de la vie humaine - disse, rivolto alla principessa, come a scusarsi. - E la Mjagkaja però, non Liza ma Bibiche, manda mille fucili e dodici suore. Ve l'ho detto? 

- Sì, ho sentito - rispose di malavoglia Koznyšev. 

- Ma è un peccato che partiate - disse Stepan Arkad'ic. - Domani offriamo un pranzo a due parenti: Dimer-Bartnjanskij, quello di Pietroburgo, e il nostro Veselovskij, Griša. Vanno tutti e due. Veselovskij ha preso moglie da poco. Ecco un uomo coraggioso! Non è vero, principessa? - si rivolse alla signora. 

La principessa, senza rispondere, guardava Koznyšev. Ma il fatto che Sergej Ivanovic e la principessa desiderassero liberarsi di lui, non turbava per nulla Stepan Arkad'ic. Egli guardava sorridendo ora la piuma del cappello della principessa, ora di lato, come a ricordarsi di qualche cosa. Avendo scorto una signora che passava con una cassetta, la chiamò presso di sé e mise dentro un biglietto da cinque rubli. 

- Non posso veder passare tranquillamente queste cassette finché ho denaro in tasca - disse. - E com'è il bollettino di oggi? Bravi i montenegrini! 

- Cosa dite! - egli gridò quando la principessa gli disse che Vronskij partiva con quel treno. 

Per un attimo il viso di Stepan Arkad'ic espresse tristezza, ma dopo un momento, quando, molleggiando su ciascuna gamba e accomodandosi le fedine, entrò nella sala dove era Vronskij, Stepan Arkad'ic aveva già del tutto dimenticato quei propri disperati singhiozzi sul corpo della sorella, e vedeva in Vronskij solo l'eroe e il vecchio amico. 

- Con tutti i suoi difetti non gli si può non render giustizia - disse la principessa a Sergej Ivanovic, non appena Oblonskij si fu allontanato da loro. - Ecco proprio una natura veramente russa, slava! Temo soltanto che a Vronskij dispiacerà vederlo. Qualunque cosa diciate, commuove la sorte di quest'uomo. Parlate un po' con lui, in viaggio - aggiunse. 

- Sì, forse, se capiterà. 

- A me non è mai piaciuto. Ma questo riscatta molte cose. Non solo va lui stesso, ma conduce uno squadrone a proprie spese. 

- Sì, ho sentito. 

Si udì un campanello. Tutti si affollarono alla porta. 

- Eccolo! - esclamò la principessa, indicando Vronskij con un cappotto lungo e un cappello nero a larghe falde, che camminava al braccio della madre. Oblonskij camminava accanto a lui, dicendo animatamente qualcosa. 

Vronskij guardava accigliato davanti a sé, quasi senza ascoltare quello che diceva Stepan Arkad'ic. 

Probabilmente, per indicazione di Oblonskij, egli si voltò a guardare dalla parte dove stavano la principessa e Sergej Ivanovic, e sollevò il cappello in silenzio. Il suo viso invecchiato, che esprimeva la sofferenza, pareva impietrito. 

Uscito dalla banchina, Vronskij, lasciata la madre, scomparve in silenzio nello scompartimento di una vettura. 

Sulla banchina echeggiava ``Dio proteggi lo zar''; poi si udirono grida di ``urrà'' e ``zivio!''. Uno dei volontari, un uomo alto, molto giovane, dal petto incavato, salutava in modo molto clamoroso, agitando sopra il capo un cappello di feltro e un fascio di fiori. Dietro di lui, sporgevano la testa, pure salutando, due ufficiali e un uomo anziano, dalla gran barba, con un berretto sporco di grasso. 

\capitolo{III}\label{iii-7} 

Salutata la principessa, Sergej Ivanovic, insieme con Katavasov che s'era avvicinato, entrò in una vettura piena zeppa, e il treno si mosse. 

Alla stazione di Carycin il treno fu accolto da un coro di giovani, armonioso, che cantava ``Gloria a te!''. Di nuovo i volontari salutarono e sporsero le teste, ma Sergej Ivanovic non prestò loro attenzione: aveva avuto già tanto a che fare con i volontari che ne conosceva ormai le caratteristiche, e tutto questo non lo interessava più. Katavasov invece, che fra le sue occupazioni scientifiche non aveva avuto occasione d'osservare i volontari, se ne interessava molto e interrogava Sergej Ivanovic. 

Sergej Ivanovic gli consigliò di passare in seconda classe per parlare lui stesso con loro. Alla stazione seguente Katavasov seguì questo consiglio. Passò in seconda e fece conoscenza con i volontari. Stavano seduti in un angolo dello scompartimento, discorrendo forte e sapendo, evidentemente, che l'attenzione dei passeggeri e di Katavasov, che era entrato, era rivolta verso di loro. Più forte di tutti parlava l'adolescente alto dal petto incavato. Era ubriaco, si vedeva, e raccontava d'una certa storia capitata nel loro istituto. Di fronte a lui era seduto un ufficiale non più giovane con una maglia militare austriaca della divisa della Guardia. Egli ascoltava, sorridendo, il parlatore e lo interrompeva. Un terzo, in divisa d'artigliere, sedeva su di una valigia accanto a loro. Un quarto dormiva. 

Entrato in discorso con l'adolescente, Katavasov venne a sapere che era un ricco mercante moscovita, che aveva scialacquato un gran patrimonio prima di aver compiuto ventidue anni. Non piacque a Katavasov perché era effeminato, viziato e debole di salute; si vedeva che era sicuro, specialmente adesso dopo aver bevuto, di compiere un atto eroico, e si vantava nella maniera più antipatica. 

L'altro, l'ufficiale a riposo, produsse pure un'impressione sgradita su Katavasov. Era, si vedeva, un uomo che aveva provato tutto. Era stato nelle ferrovie, intendente, e lui stesso aveva fondato delle fabbriche e parlava di tutto, adoperando senza nessuna necessità, a sproposito, parole scientifiche. 

Il terzo, l'artigliere, al contrario, piacque molto a Katavasov. Era un uomo modesto, silenzioso che, evidentemente, si inchinava dinanzi alla posizione dell'ufficiale della Guardia e dinanzi all'eroica abnegazione del mercante e, per conto suo, nulla diceva di sé. Quando Katavasov gli domandò cosa lo avesse indotto ad andare in Serbia, rispose modestamente: 

- Ma cosa vuoi mai, vanno tutti, bisogna pure aiutare i serbi. Fanno pena. 

- Eh, sì, specialmente di artiglieri come voi, là ce n'è pochi - disse Katavasov. 

- Io non ho mica servito molto nell'artiglieria; può darsi anche che mi mettano in fanteria o in cavalleria. 

- Ma come in fanteria, se hanno più bisogno di tutto di artiglieri? - disse Katavasov, deducendo, dall'età dell'artigliere, ch'egli dovesse avere già un grado importante. 

- Non ho servito molto in artiglieria, sono a riposo come junker - disse e cominciò a spiegare perché non aveva superato l'esame. 

Tutto questo insieme produsse un'impressione spiacevole su Katavasov, e quando i volontari uscirono nella stazione a bere, Katavasov voleva confessare a qualcuno la propria impressione sfavorevole. Un vecchietto di passaggio, in cappotto militare, aveva prestato ascolto tutto il tempo alla conversazione di Katavasov coi volontari. Rimasto da solo con lui, Katavasov gli rivolse la parola. 

- Ma che diversità di condizione fra tutte queste persone che vanno là! - disse vagamente Katavasov, desiderando di esprimere la propria opinione e nello stesso tempo di sapere l'opinione del vecchietto. 

Il vecchietto era un militare che aveva fatto due campagne. Sapeva cos'era un militare, e dall'aspetto e dal parlare di quei signori, dal piglio con cui, in viaggio, si attaccavano alla borraccia, li giudicava cattivi militari. Inoltre, abitava in un capoluogo di distretto e aveva voglia di raccontare come, dalla sua cittadina, fosse andato un soldato in congedo illimitato, ubriacone e ladro, che più nessuno assumeva come lavoratore. Ma, sapendo per esperienza che, con l'odierno stato d'animo della società, era pericoloso esprimere un'opinione contraria a quella generale, e in particolare biasimare i volontari, anche lui osservava Katavasov. 

- Eh, là hanno bisogno di gente - egli disse, ridendo con gli occhi. E si misero a parlare delle ultime notizie militari, e tutti e due nascosero, l'uno all'altro, la propria perplessità sul fatto che s'aspettasse per l'indomani un combattimento quando i turchi, secondo l'ultima informazione, erano stati battuti su tutta la linea. E così, senza aver detto nessuno dei due la propria opinione, si separarono. 

Katavasov, entrato nel suo scompartimento, andando involontariamente contro coscienza, raccontò a Sergej Ivanovic le osservazioni sui volontari, dalle quali risultava che erano ottimi figlioli. 

A una grande stazione di una città, di nuovo canti e grida accolsero i volontari, apparvero di nuovo raccoglitrici e raccoglitori di offerte con le cassette, e le signore del capoluogo del governatorato offrirono fasci di fiori ai volontari e li seguirono al ristorante; ma tutto questo era in tono molto più debole e in proporzioni minori che non a Mosca. 

\capitolo{IV}\label{iv-7} 

Durante il tempo della fermata nella stazione del capoluogo del governatorato, Sergej Ivanovic non andò al ristorante, ma si mise a camminare avanti e indietro sulla banchina. 

Passando per la prima volta accanto allo scompartimento di Vronskij, notò che il finestrino era chiuso. Ma, passando una seconda volta, vide al finestrino la vecchia contessa. Ella chiamò a sé Koznyšev. 

- Ecco, vado, lo accompagno fino a Kursk - ella disse. 

- Sì, ho sentito - disse Sergej Ivanovic, fermandosi vicino al finestrino e dandovi un'occhiata dentro - che bel gesto da parte sua! - soggiunse dopo aver notato che Vronskij non era nello scompartimento. 

- Sì, dopo la sventura, che cosa mai doveva fare? 

- Che cosa terribile! - disse Sergej Ivanovic. 

- Ah, cosa ho passato! Ma entrate\ldots{} Ah, cosa ho passato! - ella ripeté, quando Sergej Ivanovic entrò e sedette accanto a lei sul divano. - Non si può immaginare! Per sei settimane, non ha parlato con nessuno e ha mangiato solo quando lo supplicavo io. E neppure per un momento lo si poteva lasciare solo. Avevamo portato via tutto quello con cui poteva uccidersi; stavamo a pianterreno, ma non si poteva prevedere nulla. Perché lo sapete, s'era già sparato una volta, pure per lei - disse, e le sopracciglia della vecchietta si contrassero a questo ricordo. - Sì, è finita proprio come doveva finire una donna simile. Perfino la morte ha scelto vile, bassa. 

- Non sta a noi giudicare, contessa - disse Sergej Ivanovic con un sospiro - ma capisco come questo sia stato penoso per voi. 

- Ah, non lo dite! Io stavo nella mia villa, e lui era da me. Portano un biglietto. Lui scrive la risposta e la manda via. Noi non sapevamo nulla, che lei fosse proprio lì, alla stazione. La sera me n'ero appena andata in camera mia, quando la mia Mary mi dice che alla stazione una signora s'è gettata sotto il treno. Fu come se qualcosa m'avesse colpito! Capii che era lei. La prima cosa che dissi fu: ``non lo dite a lui''. Ma gliel'avevano già detto. Il cocchiere s'era trovato là e aveva visto tutto. Quando io accorsi in camera sua, egli era già fuori di sé; era tremendo a guardarlo. Non disse nemmeno una parola e corse là. Non so più cosa ci fu, ma lo portarono come un morto. Io non l'avrei riconosciuto. Prostration complète, diceva il dottore. Poi cominciò come una frenesia. Ah, che dire! - disse la contessa, facendo un gesto sconsolato con la mano. - Un momento tremendo. No, qualunque cosa diciate, è stata una donna perversa. Ma che passioni disperate sono queste! È sempre per mostrar qualcosa di particolare. Ecco che lei proprio l'ha dimostrato. Ha rovinato se stessa e due ottime persone: suo marito e il mio povero figliolo. 

- E suo marito che fa? - chiese Sergej Ivanovic. 

- Ha preso la figlia di lei. Alësa nei primi tempi consentiva a tutto. Ma adesso lo tormenta orribilmente il fatto d'aver dato la propria figlia a una persona estranea, ma non vuol rimangiarsi la parola. Karenin è venuto al funerale. Ma noi abbiamo cercato di non farlo incontrare con Alësa. Per lui, per il marito tuttavia, la vita è più facile. Lei l'ha liberato. Ma il mio povero figliolo s'era dato tutto a lei. Aveva abbandonato tutto: la carriera, me, e lei non solo non ha avuto pietà, ma l'ha proprio distrutto, deliberatamente. No, qualunque cosa diciate, la stessa sua morte è la morte di una donna bassa, senza religione. Iddio mi perdoni, ma non posso non odiare la sua memoria, vedendo la rovina di mio figlio. 

- Ma adesso come sta? 

- È Dio che ci ha aiutato, con questa guerra serba. Io sono vecchia, non ci capisco nulla, ma questa gliel'ha mandata Iddio. S'intende che io, come madre, provo spavento; e soprattutto, dicono, ce n'est pas très bien vu à Petersbourg. Ma che fare? Questo solo poteva sollevarlo. Jašvin, un suo amico, ha perduto tutto al giuoco e s'è preparato ad andare in Serbia. È passato da lui e l'ha convinto. Adesso questo lo occupa. Voi, per favore, parlate con lui, desidero farlo distrarre. È così triste! E per disgrazia gli è sopraggiunto un gran mal di denti. Ma sarà molto contento di vedervi. Per favore, parlate un po' con lui: sta camminando da questa parte. 

Sergej Ivanovic disse ch'era molto contento, e passò dall'altra parte del treno. 

\capitolo{V}\label{v-7} 

Nell'ombra serale, obliqua, dei sacchi ammassati sulla banchina, Vronskij, nel suo cappotto lungo, col cappello abbassato e le mani in tasca, camminava, voltandosi rapidamente a ogni venti passi, come una belva in gabbia. A Sergej Ivanovic, mentre si avvicinava, parve che Vronskij lo avesse scorto, ma che fingesse di non vederlo. Per Sergej Ivanovic era indifferente. Egli era al di sopra di ogni considerazione personale nei riguardi di Vronskij. 

In quel momento Vronskij, agli occhi di Sergej Ivanovic, era un collaboratore importante di una grande impresa, e Koznyšev stimava suo dovere incoraggiarlo e approvarlo. Gli si avvicinò. 

Vronskij si fermò, lo guardò fisso, lo riconobbe e, fatti alcuni passi incontro a Sergej Ivanovic, gli strinse forte la mano. 

- Forse voi non desiderate neppure di vedermi - disse Sergej Ivanovic; - ma non posso esservi utile? 

- Non c'è nessuno che io possa vedere meno spiacevolmente di voi - disse Vronskij. - Perdonatemi. Cose piacevoli per me nella vita non esistono. 

- Capisco, volevo offrirvi i miei servigi - disse Sergej Ivanovic, esaminando il viso, evidentemente sofferente, di Vronskij. - Non avete bisogno d'una lettera per Ristic, per Milan? 

- Oh, no! - disse Vronskij, quasi stentando a capire. - Se per voi è indifferente, camminiamo. Nelle vetture c'è una afa tale. Una lettera? No, vi ringrazio; per morire non c'è bisogno di raccomandazioni. Se non ai turchi\ldots{} - diss'egli, sorridendo solo con la bocca. Gli occhi continuavano ad avere una espressione di sofferenza esasperata. 

- Sì, ma forse vi sarebbe più facile entrare in relazioni, che tuttavia sono indispensabili, con una persona preparata. Ma, come volete! Sono stato molto contento sentendo della vostra decisione. Anche così, ci sono già tanti attacchi contro i volontari, che un uomo come voi li solleva nell'opinione pubblica. 

- Io, come uomo - disse Vronskij - sono buono perché la vita per me non vale nulla. E che in me ci sia abbastanza forza fisica per sfondare un quadrato e romperlo o rimanerci, questo lo so. Sono contento che ci sia qualcosa per cui dare la mia vita, la quale non è che non mi sia necessaria, ma m'è venuta in odio. A qualcuno servirà - ed egli fece un movimento impaziente con lo zigomo per un doloroso, incessante mal di denti, che gli impediva perfino di parlare con l'espressione che voleva. 

- Tornerete a nuova vita, ve lo predìco - disse Sergej Ivanovic, sentendosi commosso. - La liberazione dei propri fratelli dal giogo è un fine degno e della morte e della vita. Che Iddio vi conceda buon successo esterno, e la pace interiore - soggiunse e tese la mano. 

Vronskij strinse forte la mano tesa di Sergej Ivanovic. 

- Sì, come strumento posso servire a qualcosa. Ma come uomo, sono un rudere - egli disse dopo una pausa. 

L'attanagliante dolore al dente robusto, che gli riempiva di saliva la bocca, gli impediva di parlare. Tacque, esaminando le ruote di un tender che scivolava lento e scorrevole sulle rotaie. 

E a un tratto non un dolore, ma un disagio interiore, tormentoso e complesso, che pure non era dolore, lo obbligò a dimenticare per un momento il mal di denti. Guardando il tender e le rotaie, sotto l'influsso della conversazione con un amico che non aveva rivisto dopo la propria sventura, gli tornò a un tratto in mente lei, cioè quello che rimaneva ancora di lei, quand'egli era entrato, correndo come un pazzo, nella caserma della stazione ferroviaria: sul tavolo della caserma il corpo insanguinato, disteso senza ritegno in mezzo agli estranei, ancora pieno di vita recente; la testa intatta reclinata indietro con le trecce pesanti e i capelli inanellati sulle tempie, e sul viso delizioso, dalla bocca rossa socchiusa, una strana espressione di pena rappresa sulle labbra e spaventosa negli occhi non chiusi e fissi, quasi stesse per pronunciare quella frase terribile che gli aveva detto durante il litigio, ch'egli se ne sarebbe pentito. 

Ed egli cercava di ricordarla come era quando l'aveva incontrata la prima volta, pure alla stazione, misteriosa e incantevole, piena d'amore, che cercava e dava la felicità, e non così crudelmente vendicativa come gli tornava alla memoria nell'ultimo istante. Egli cercava di ricordare i momenti migliori passati con lei; ma questi momenti erano avvelenati per sempre. Egli ricordava di lei solo quella minaccia trionfante, che aveva compiuto per ottenere un rimorso non necessario a nessuno, ma indistruttibile. Cessò di sentire il dolore al dente, e i singhiozzi gli contrassero il viso. 

Passando due volte davanti ai sacchi, in silenzio, e tornato padrone di sé, si voltò con calma verso Sergej Ivanovic: 

- Non si è avuto un bollettino dopo quello di ieri? Sì, li hanno battuti per la terza volta, ma per domani intanto si aspetta un combattimento decisivo. 

E, dopo aver parlato ancora della proclamazione di Milan a sovrano e delle conseguenze enormi che questo poteva avere, si separarono andando nelle rispettive vetture, al secondo squillo di campanello. 

\capitolo{VI}\label{vi-7} 

Non sapendo quando sarebbe potuto partire da Mosca, Sergej Ivanovic non aveva telegrafato al fratello di mandarlo a rilevare. Levin non era in casa quando Katavasov e Sergej Ivanovic, su di un piccolo tarantas noleggiato alla stazione, neri di polvere, alle undici passate, si avvicinarono alla scalinata della casa di Pokrovskoe. Kitty, che era seduta al balcone col padre e la sorella, riconobbe il cognato e corse giù ad accoglierlo. 

- Come, non vi vergognate di non farcelo sapere? - disse, tendendo la mano a Sergej Ivanovic e porgendogli la fronte. 

- Siamo arrivati benissimo e non v'abbiamo incomodato - rispose Sergej Ivanovic. - Son così impolverato che temo di toccarvi. Ero tanto occupato che non sapevo quando sarei fuggito. E voi come prima - diss'egli, sorridendo - vi godete una tranquilla felicità al di fuori delle correnti, nel vostro tranquillo porto. Ecco che anche il nostro Fëdor Vasil'ic finalmente si è deciso. 

- Non sono un negro, mi laverò, avrò un aspetto umano - disse Katavasov col suo abituale brio, dando la mano e sorridendo, in modo particolare, con i denti che spiccavano nel viso annerito. 

- Kostja sarà molto contento. È andato alla fattoria. Sarebbe tempo che venisse. 

- S'occupa sempre dell'azienda. Ecco, proprio come in un porto - disse Katavasov. - E noi in città, eccettuata la guerra serba, non vediamo nulla. Be', come vede la cosa il mio amico? Forse un po' diversamente dagli altri. 

- Ma lui la vede così, come tutti - rispose Kitty un po' confusa, voltandosi a guardare Sergej Ivanovic. - Allora manderò a chiamarlo. E da noi c'è ospite papà. È arrivato dall'estero da poco. 

E, dato l'ordine di mandare a chiamare Levin e di condurre gli ospiti impolverati a lavarsi, l'uno nello studio, l'altro nella camera che era stata di Dolly, e di preparare la colazione agli ospiti, compiacendosi della facoltà di muoversi agilmente di cui era stata privata durante la gravidanza, andò di corsa al balcone. 

- Ci sono Sergej Ivanovic e Katavasov, il professore - disse. 

- Oh, con questo caldo, la cosa è pesante! - disse il principe. 

- No, papà, è molto simpatico, e Kostja gli vuole molto bene - disse sorridendo Kitty, come a supplicarlo di qualcosa, dopo aver notato l'espressione di canzonatura sul viso del padre. 

- Ma io non dico nulla. 

- Va' tu, Dolly, tesoro mio - disse alla sorella - e intrattienili. Hanno visto Stiva alla stazione, sta bene. E io corro da Mitja. Neanche a farlo apposta, non gli ho dato il latte dall'ora del tè. Adesso s'è svegliato e probabilmente grida. - E, sentendo un afflusso di latte, andò nella camera del bambino a passo svelto. 

Realmente, non era che avesse indovinato (il suo legame col bambino non era ancora spezzato), ma aveva capito con certezza, all'afflusso di latte che sentiva in sé, ch'egli mancava di nutrimento. 

Sapeva che gridava, ancor prima di avvicinarsi alla camera del bambino. E realmente egli gridava. Ne sentì la voce e accelerò il passo. Ma quanto più presto ella camminava, tanto più forte egli gridava. La sua voce era buona, sana, ma affamata e impaziente. 

- È un pezzo, njanja, è un pezzo? - diceva in fretta Kitty, sedendosi su di una seggiola e preparandosi a dare il latte. - Ma datemelo presto, dunque! Ah, njanja, come siete noiosa, via, la cuffietta gliela legherete dopo! 

Il bambino si strozzava a furia di gridare per la fame. 

- Ma non si può mica, matuška - disse Agaf'ja Michajlovna, che era quasi sempre presente nella camera del bambino. - Bisogna metterlo in ordine. Ahu, ahu! - ella cantava sopra di lui, senza prestare attenzione alla madre. 

La njanja portò il bambino alla madre. Agaf'ja Michajlovna gli camminava dietro col viso rasserenato dalla tenerezza. 

- Conosce, conosce. Come è vero Iddio, matuška, Katerina Aleksandrovna, m'ha conosciuta! - gridava Agaf'ja Michajlovna più forte del bambino. 

Ma Kitty non ascoltava le parole di lei. La sua impazienza cresceva come l'impazienza del bambino. 

Per l'impazienza la faccenda, per un pezzo, non si avviò. Il bambino non afferrava quello che doveva e si arrabbiava. 

Finalmente, dopo un disperato grido soffocato, dopo un inghiottire a vuoto, la cosa si sistemò, e la madre e il bambino si sentirono contemporaneamente calmi e placati. 

- Però anche lui, poverino, è tutto sudato - disse sottovoce Kitty, palpando il bambino. - Perché pensate che conosca? - aggiunse, osservando di traverso gli occhi del bambino che le sembrava guardassero furbi di sotto alla cuffietta abbassatasi, le piccole guance che riprendevano fiato uniformemente, e la manina dalla palma rossa, con la quale faceva dei movimenti circolari. 

- Non può essere! Se conoscesse, allora conoscerebbe me - disse Kitty, all'assicurazione di Agaf'ja Michajlovna, e sorrise. 

Sorrise perché, quantunque dicesse ch'egli non poteva riconoscere, col cuore sapeva che non solo riconosceva Agaf'ja Michajlovna, ma sapeva e capiva tutto; e sapeva e capiva ancora molte cose che nessuno sapeva e che lei, madre, aveva imparato a conoscere e aveva cominciato a capire grazie a lui. Per Agaf'ja Michajlovna, per la njanja, per il nonno, perfino per il padre, Mitja era un essere vivo che esigeva per sé soltanto cure materiali; ma per la madre egli era da tempo un essere morale, con cui c'era già tutta una storia di rapporti spirituali. 

- Ma ecco che si sveglierà, se Dio vuole, lo vedrete da voi. Appena faccio così, lui si rischiara, golubcik. Si fa subito raggiante come una giornata serena - diceva Agaf'ja Michajlovna. 

- E va bene, va bene, lo vedremo, allora - mormorò Kitty. - Adesso andate, si addormenta 

\capitolo{VII}\label{vii-7} 

Agaf'ja Michajlovna uscì in punta di piedi; la njanja abbassò la tendina, scacciò le mosche di sotto alla cortina di mussola del lettuccio e un calabrone che batteva contro i vetri della finestra, e si sedette, agitando un ramo di betulla, quasi vizzo, sopra la madre e il bambino. 

- Che caldo, che caldo! Se almeno Iddio ci mandasse una pioggerella - disse. 

- Sì, sì, sst\ldots{} - rispose soltanto Kitty, dondolandosi lievemente e premendo con tenerezza il braccio paffuto, come stretto al polso da un filo, che Mitja agitava sempre più debolmente, ora chiudendo, ora aprendo gli occhietti. Questo braccino confondeva Kitty; aveva voglia di baciare quel braccino, ma aveva paura di farlo, per non svegliare il bimbo. Il braccino finalmente cessò di muoversi, e gli occhi si chiusero. Solo di tanto in tanto, il bambino sollevava le lunghe ciglia ricurve, fissava la madre con gli occhi umidi, che nella penombra sembravano neri. La njanja cessò di agitare il ramo e si assopì. Di sopra si udì uno scoppio di voce del vecchio principe e uno scroscio di risa di Katavasov. 

``Di sicuro, si son messi a parlare senza di me - pensava Kitty - tuttavia mi spiace che Kostja non ci sia. È andato di sicuro nell'arniaio. Per quanto sia triste, il fatto che vada spesso là, mi fa piacere. Ciò lo distrae. Adesso è diventato più allegro, più buono che non in primavera. S'era fatto così cupo e tormentato che cominciavo ad avere paura di lui. E come è buffo!'' sussurrò, sorridendo. 

Ella sapeva quello che tormentava suo marito. Era la propria mancanza di fede. Se avessero domandato a lei se riteneva che, nella vita futura, non credendo, egli si sarebbe perduto, avrebbe dovuto convenire che si sarebbe perduto; eppure la mancanza di fede in lui, non la tormentava; ed ella, che doveva riconoscere non esserci salvezza per un miscredente, pur amando più di tutto al mondo l'anima di suo marito, pensava con un sorriso alla miscredenza di lui e fra di sé diceva ch'egli era buffo. 

``Perché tutto l'anno non fa che leggere certe filosofie? - pensava. - Se tutto questo è scritto in quei libri, lui li può capire. Ma se c'è falsità, perché leggerli allora? Lui stesso dice che vorrebbe credere. Allora perché non crede? Probabilmente perché pensa molto. E pensa molto per la solitudine. È sempre solo, solo. Con noi non può dire tutto. Penso che questi ospiti gli facciano piacere, particolarmente Katavasov. Gli piace ragionare con lui'' ella pensò, ma subito cominciò a considerare dove sarebbe stato più comodo mettere a dormire Katavasov, separatamente o insieme a Sergej Ivanyc. E qui le venne a un tratto un pensiero che la fece trasalire di agitazione e riuscì perfino a inquietare Mitja che, per questo, la guardò severo. ``La lavandaia, mi pare, non ha ancora portato la biancheria, e per gli ospiti la biancheria da letto è tutta fuori. Se non si dànno ordini, Agaf'ja Michajlovna darà a Sergej Ivanyc della biancheria usata'' e, a questo solo pensiero, il sangue affluì al viso di Kitty. 

``Sì, darò ordini'' ella stabilì e, ritornando alle idee di prima, ricordò che qualcosa d'importante riguardo all'anima non era stato ancora finito di pensare, e cominciò a ricordare cosa. ``Sì, Kostja è miscredente'' ricordò di nuovo con un sorriso. 

``Via, miscredente! Meglio che sia sempre così, invece d'essere come la signora Stahl o come volevo essere io allora, all'estero. No, lui, poi, non si metterà a fingere''. 

E un tratto recente della sua bontà le sorse dinanzi con chiarezza. Due settimane prima era arrivata una lettera contrita di Stepan Arkad'ic per Dolly. Egli la supplicava di salvare il suo onore, di vendere i propri possessi per pagare i debiti. Dolly s'era disperata, aveva odiato il marito, l'aveva disprezzato, compianto, si era decisa a divorziare, a rifiutare, ma aveva finito con l'acconsentire a vendere una parte dei propri possessi. Dopo questo, Kitty ricordò, con un involontario sorriso di commozione, il turbamento di suo marito, il suo ripetuto imbarazzo nell'avvicinarsi alla questione che lo interessava e come finalmente, escogitato l'unico mezzo per aiutare Dolly senza offenderla, avesse proposto a Kitty di cederle la sua parte dei possedimenti, cosa che a lei non era venuta in mente prima. 

``E che miscredente? Col suo cuore, con quel terrore di addolorare chiunque, perfino un bambino! Tutto per gli altri, nulla per sé. Sergej Ivanovic pensa proprio che sia un dovere di Kostja essere il suo amministratore. Così anche la sorella. Adesso Dolly con i suoi bambini è sotto la sua tutela. E tutti questi contadini che vengono ogni giorno da lui, come se egli fosse obbligato a servirli''. 

``Sì, sii soltanto come tuo padre, soltanto così'' ella diceva, dando Mitja alla njanja e toccandone col labbro la guancia. 

\capitolo{VIII}\label{viii-7} 

Dal momento in cui, nel vedere morire il fratello amato, Levin aveva considerato per la prima volta la questione della vita e della morte attraverso le nuove convinzioni, come egli le chiamava, che insensibilmente avevano sostituito le credenze infantili e giovanili, nel periodo che per lui era andato dai venti ai trentaquattro anni, aveva provato orrore non tanto della morte, quanto di una vita senza la minima conoscenza di ciò che essa è, donde viene, a che scopo e perché. L'organismo, la sua distruzione, l'indistruttibilità della materia, la legge di conservazione della forza, l'evoluzione, erano tutte parole che, in lui, avevano preso il posto della fede d'un tempo. Queste parole, e le concezioni ad esse legate, andavano molto bene per gli scopi intellettuali; ma per la vita non davano nulla, e Levin si sentì, a un tratto, nella situazione d'un uomo che abbia scambiato una pelliccia calda per un vestito di mussola e che, per la prima volta, al gelo, si persuada in modo indubitabile, non con ragionamenti ma con tutto il suo essere, che per lui è come se fosse nudo e che deve inevitabilmente perire in modo tormentoso. 

Da quel momento, pur senza rendersene conto e continuando a vivere come prima, Levin non aveva cessato di provare il terrore della propria ignoranza. 

Inoltre, sentiva confusamente che ciò che chiamava le sue convinzioni, era non solo ignoranza, ma un modo di pensare col quale era impossibile raggiungere la conoscenza di quello che gli occorreva. 

Nel primo tempo del matrimonio, le nuove gioie, i nuovi doveri da lui conosciuti avevano completamente soffocato questi pensieri: ma negli ultimi tempi, dopo il parto della moglie, quando aveva vissuto a Mosca inoperoso, a Levin si era presentato sempre più frequente e insistente un problema che chiedeva soluzione. 

Il problema per lui consisteva in questo: ``Se io non riconosco quelle risposte che dà il cristianesimo alle domande della vita, allora quali risposte riconosco?''. E non riusciva in nessun modo a trovare in tutto l'arsenale delle proprie convinzioni non solo una qualche risposta, ma nulla che fosse simile a una risposta. 

Era nella situazione di un uomo che cerca il cibo in una bottega di giocattoli o di armi. Involontariamente, senza averne egli stesso coscienza, adesso, in ogni libro, in ogni conversazione, in ogni persona cercava i rapporti con tali questioni e la loro soluzione. 

Più di tutto in quel tempo lo stupiva e lo sconvolgeva il fatto che la maggioranza delle persone del suo ambiente e della sua età, avendo scambiato, come lui, le credenze di prima con le stesse convinzioni nuove che aveva lui, non vedevano in questo alcun danno ed erano del tutto contente e tranquille. Così che, oltre la questione principale, ancora altre questioni tormentavano Levin: erano sincere quelle persone? non fingevano? Oppure, non capivano in un qualche modo diverso, più chiaramente di lui, le risposte che dava la scienza alle questioni che lo interessavano? Ed egli studiava accuratamente e le opinioni di quelle persone e i libri che enunciavano quelle risposte. 

Una cosa sola aveva trovato, dal momento in cui tali questioni avevano cominciato a interessarlo, ed era che egli sbagliava nel supporre, dai ricordi del suo ambiente giovanile universitario, che la religione avesse già fatto il suo tempo e che non esistesse più. Tutte le persone buone, a lui vicine per rapporti di vita, credevano. E il vecchio principe, e L'vov, che gli piaceva tanto, Sergej Ivanyc, tutte le donne credevano, e sua moglie credeva, così come egli aveva creduto nella prima infanzia, e così credeva il novantanove per cento del popolo russo, tutto quel popolo la cui vita gli ispirava il più grande rispetto. 

Un'altra cosa era che, avendo letto molti libri, s'era convinto che le persone le quali condividevano le sue opinioni non intendevano null'altro e, senza spiegar nulla, negavano non soltanto le questioni, senza la soluzione delle quali egli sentiva di non poter vivere, ma cercavano di risolvere questioni del tutto diverse, che non potevano interessarlo, come, per esempio, quella dell'evoluzione degli organismi, quella della spiegazione meccanica dell'anima e simili. 

Inoltre, durante il parto della moglie, gli era accaduto un avvenimento straordinario. Lui, che non credeva, s'era messo a pregare, e, nel momento in cui aveva pregato, aveva creduto. Ma quel momento era passato, e a quello stato d'animo d'allora egli non poteva dare alcun posto nella propria vita. 

Non poteva ammettere che in quel momento aveva conosciuto la verità e che ora si sbagliava; perché, non appena cominciava a pensare con calma, tutto si frantumava in mille pezzi; non poteva riconoscere nemmeno che in quel momento si era sbagliato perché gli era caro lo stato d'animo di allora e, riconoscendolo come un risultato della propria debolezza, avrebbe contaminato quegli attimi. Era in una penosa disarmonia con se stesso e tendeva tutte le forze dell'animo per uscirne. 

\capitolo{IX}\label{ix-7} 

Questi pensieri lo facevano soffrire e lo tormentavano ora in maniera più debole, ora più forte, ma non lo abbandonavano mai. Leggeva e pensava, e quanto più leggeva e pensava, tanto più lontano si sentiva dallo scopo che perseguiva. 

Negli ultimi tempi a Mosca e in campagna, convintosi che nei materialisti non avrebbe trovato una risposta, aveva letto e riletto Platone e Spinoza, Kant e Schelling, Hegel e Schopenhauer, filosofi questi che spiegavano la vita da un punto di vista non materialistico. 

I pensieri gli sembravano fecondi quando leggeva o quando si figurava le confutazioni di altre dottrine, in particolare materialistiche; ma non appena leggeva o immaginava da sé la soluzione delle questioni, allora si ripeteva sempre la stessa cosa. Seguendo una definizione già data di parole oscure, come spirito, volontà, libertà, sostanza, entrando apposta in quella rete di parole che gli ponevano i filosofi o che egli stesso si poneva, cominciava quasi a capire qualcosa. Ma bastava dimenticare l'artificioso corso del pensiero e tornare, mentre pensava secondo il filo dato, a quello che nella vita lo soddisfaceva, che improvvisamente tutta quell'artificiosa impalcatura crollava come un castello di carte, e appariva chiaro che l'edificio era fatto con quelle stesse parole trasposte, indipendentemente da qualcosa che nella vita era più importante della ragione. 

Un certo tempo, leggendo Schopenhauer, sostituì al posto della volontà l'amore; e questa nuova filosofia per un paio di giorni, finché vi rimase dentro, lo consolò; ma crollò proprio alla stessa maniera quando la osservò dalla vita, e si rivelò un vestito di mussola che non teneva caldo. 

Suo fratello Sergej Ivanovic gli consigliò di leggere le opere teologiche di Chomjakov. Levin lesse il secondo volume delle opere di Chomjakov e, malgrado il tono polemico, elegante e spiritoso che dapprima l'aveva allontanato, fu colpito dalla dottrina sulla Chiesa da esso esposta. Lo colpì il pensiero che la comprensione delle verità divine non era data all'uomo, ma era data all'insieme degli uomini uniti dall'amore, alla Chiesa. Lo rallegrò il pensiero di come fosse più facile credere alla Chiesa esistente, presentemente viva, che costituiva tutte le credenze degli uomini e che aveva a capo Iddio e perciò era santa e infallibile, e da essa poi accogliere le credenze in Dio, nella creazione, nel peccato, nella redenzione, anziché cominciare da Dio, da un Dio lontano, misterioso, dalla creazione e via di seguito. Ma avendo poi letto una storia della Chiesa di uno scrittore cattolico e una storia della Chiesa di uno scrittore ortodosso e visto che tutte e due le Chiese, infallibili per loro natura, si negavano reciprocamente, egli si disincantò anche della dottrina di Chomjakov, e quest'edificio si dissolse in polvere come le costruzioni filosofiche. 

Tutta quella primavera visse fuori di sé ed ebbe momenti paurosi. 

``Senza la conoscenza di quel che sono qui, non si può vivere. Ma sapere questo non posso, di conseguenza non si può vivere'' si diceva Levin. 

``Nel tempo infinito, nell'infinità della materia, nello spazio infinito nasce un piccolo organismo; questa bollicina si tiene un po' in alto e poi scoppia, e questa bollicina sono io''. 

Era una tormentosa menzogna, ma era l'unico, l'ultimo risultato del secolare lavoro del pensiero umano in quella direzione. 

Era quella l'ultima credenza nella quale si sistemavano tutte le ricerche del pensiero umano in quasi tutti i campi. Era la convinzione che dominava, e Levin, fra tutte le altre spiegazioni, assimilò proprio questa, come la più chiara, tuttavia senza sapere lui stesso quando e come. 

Ma questa non solo era una menzogna, ma era la crudele irrisione di una certa forza perversa, infame, contraria e tale che non ci si poteva sottomettere. 

Bisognava liberarsi da questa forza. E la liberazione era nelle mani di ognuno. Bisognava far cessare la dipendenza dal male. E non v'era che un mezzo: la morte. 

E Levin, padre di famiglia felice, uomo sano, fu varie volte così vicino al suicidio, che nascose una corda per non impiccarsi, ed ebbe paura di andar col fucile per non spararsi. 

Ma Levin non si sparò e non si impiccò e continuò a vivere. 

\capitolo{X}\label{x-7} 

Quando Levin pensava che cosa mai egli fosse e per quale mai cosa vivesse, non trovava una risposta e si dava alla disperazione; ma quando cessava di chiederselo, pareva sapere cos'era e per che cosa vivesse, perché agiva e viveva in modo fermo e deciso; anzi in quegli ultimi tempi viveva con molta più fermezza e decisione di prima. 

Tornato in campagna al principio di giugno, era tornato anche alle sue solite occupazioni. L'azienda rurale, i rapporti con i contadini e con i vicini, l'azienda domestica, gli affari della sorella e del fratello che gli erano sulle spalle, i rapporti con la moglie, i parenti, le preoccupazioni per il bambino, la nuova caccia alle api, alla quale si era appassionato fin dalla primavera, occupavano tutto il suo tempo. 

Questi affari lo occupavano non perché egli li giustificasse da un qualche punto di vista generale, come soleva far prima; al contrario, adesso, da una parte deluso dall'insuccesso delle precedenti opere per la comune utilità, dall'altra troppo preso dai suoi pensieri e dal gran numero di faccende che gli piombavano da ogni lato, aveva completamente abbandonato ogni considerazione sull'utilità generale, e questi affari lo occupavano soltanto perché gli pareva di dover fare quello che faceva, di non poter fare altrimenti. 

Un tempo (la cosa era cominciata quasi dall'infanzia e si era accentuata sempre più fino alla piena virilità), quando egli cercava di far qualcosa che procurava il bene di tutti, dell'umanità, della Russia, di tutto il villaggio, aveva notato che i pensieri al riguardo erano piacevoli, ma l'attività stessa era sempre slegata, non c'era la piena sicurezza che la cosa fosse assolutamente necessaria; e quella stessa attività, che in principio sembrava così vasta, si restringeva sempre più, si riduceva a nulla. Adesso invece, quando, dopo il matrimonio, egli aveva cominciato a restringere sempre più la vita per se stesso, pur non provando più nessuna gioia al pensiero della propria attività, aveva la certezza che la sua opera fosse necessaria e vedeva che essa riusciva molto meglio di prima e diveniva sempre più vasta. 

Adesso, quasi contro la propria volontà, egli si conficcava sempre più nella terra come un aratro, così che ormai non poteva più uscirne senza rivoltare il solco. 

Che la famiglia vivesse come erano abituati a vivere i padri e i nonni, cioè nelle stesse condizioni di cultura, e che nelle stesse condizioni venissero educati i figli, era indubbiamente necessario così come si pranza quando si ha voglia di mangiare; e così come preparare il pranzo, era altrettanto necessario condurre la macchina economica a Pokrovskoe in modo che rendesse. Così come indubbiamente bisognava pagare un debito, bisognava pure tenere la terra patrimoniale in una situazione tale che il figlio, ricevutala in eredità, fosse grato al padre, così come Levin lo era stato al nonno per tutto quello che aveva costruito e piantato. E per questo non bisognava dare in fitto la terra, ma coltivarla da sé, tenere il bestiame, concimare i campi, piantare i boschi. 

Non si potevano non curare gli affari di Sergej Ivanovic, della sorella, di tutti i contadini che venivano a chieder consiglio e che vi si erano abituati, così come non si può abbandonare un bambino che si tenga per mano. Bisognava prendersi cura delle comodità della cognata invitata coi figliuoli, e della moglie e del bambino e non si poteva non dedicare loro una sia pur piccola parte del giorno. 

E tutto questo insieme con la caccia alla selvaggina e la nuova caccia alle api, riempiva tutta quella vita di Levin che non aveva nessun senso per lui quando pensava. 

Ma, oltre al fatto che Levin sapeva bene cosa dovesse fare, così egli sapeva pure come dovesse far tutto questo e quale faccenda fosse più importante dell'altra. 

Sapeva che si dovevano assumere dei lavoratori al minor prezzo possibile; ma che assumerli come servi, dando il denaro in anticipo, a un prezzo minore di quanto costavano, non si doveva, anche se questo era molto vantaggioso. Vender la paglia ai contadini, durante la carestia, si poteva, anche se ne veniva compassione; ma la locanda e la taverna, anche se rendevano, bisognava distruggerle. Per il taglio dei boschi bisognava punire il più severamente possibile, ma per il bestiame fatto pascolare abusivamente non si potevano prendere multe; e benché questo addolorasse i guardiani e distruggesse il timore, non si poteva non lasciare pascolare il bestiame abusivamente. 

A Pëtr, che pagava il dieci per cento al mese a uno strozzino, bisognava fare un prestito per riscattarlo; ma non si poteva condonare né differire il tributo ai contadini insolventi. Non si poteva lasciar passare all'amministratore che un praticello non fosse stato falciato e l'erba si fosse perduta per niente; ma si potevano anche non falciare le ottanta desjatiny dove era stato piantato un bosco giovane. Non si poteva perdonare un lavoratore che al tempo del lavoro se ne andasse a casa perché gli era morto il padre e, per quanta pena suscitasse, bisognava pagarlo di meno per i mesi in cui era stato assente; ma non si poteva non dare la mensilità anche ai vecchi servi che non facevano nulla. 

Levin sapeva pure che, tornando a casa, bisognava prima di tutto andar dalla moglie che stava poco bene, mentre i contadini, che lo aspettavano già da tre ore, potevano aspettare ancora; ma sapeva che, malgrado tutto il piacere da lui provato a metter dentro uno sciame, bisognava privarsi di quel piacere e, lasciato il vecchio a metter dentro lo sciame senza di lui, bisognava andare a ragionare con i contadini che l'avevano trovato nell'arniaio. 

Se agiva bene o male non lo sapeva, e non soltanto non si sarebbe messo adesso a dimostrarlo, ma evitava discorsi e pensieri in proposito. 

I ragionamenti lo portavano al dubbio e gli impedivano di vedere quel che si doveva e quel che non si doveva fare. Quando invece non pensava, ma viveva, sentiva continuamente nell'animo suo la presenza di un giudice infallibile che decideva quale delle due azioni fosse la migliore e quale peggiore, e, non appena agiva in modo diverso da come si doveva, lo sentiva immediatamente. 

Così egli viveva, non sapendo e non vedendo la possibilità di sapere che cosa mai egli fosse e perché mai fosse al mondo, tormentandosi per questa sua ignoranza fino al punto da temere il suicidio, e nello stesso tempo aprendosi nella vita con fermezza la propria strada, ben tracciata e tutta sua. 

\capitolo{XI}\label{xi-7} 

Il giorno in cui Sergej Ivanovic giunse a Pokrovskoe, Levin era in una delle sue giornate più tormentose. 

Era il periodo più fervido di lavoro, quando in tutto il popolo si manifesta una così straordinaria tensione dello spirito di sacrificio nel lavoro, come non si manifesta mai in altre occasioni di vita, e che sarebbe altamente apprezzabile se le persone che dimostrano queste qualità le apprezzassero loro stesse, se questa tensione non si ripetesse ogni anno e se le conseguenze di essa non fossero così semplici. 

Falciare e mietere la segala e l'avena e trasportarle, finire di falciare i prati, dividere a mezzo il maggese, sgranare le sementi e seminare il grano autunnale, tutto questo sembra semplice e ordinario; ma per giungere a questo bisogna che, dal grande al piccolo, tutta la gente di campagna lavori incessantemente, in quelle tre o quattro settimane, tre volte più del normale, nutrendosi di kvas, di cipolla e di pane nero, battendo e portando di notte i covoni e dedicando al sonno non più di due o tre ore sulle ventiquattro. E questo si fa ogni anno in tutta la Russia. 

Avendo vissuto la maggior parte della sua vita in campagna e in rapporti intimi con la gente di campagna, Levin, nel periodo di lavoro, sentiva sempre che quella generale eccitazione del popolo si comunicava anche a lui. 

Fin dalla mattina era andato alla prima seminagione della segala, a vedere l'avena che portavano alle biche e, tornato a casa per l'ora in cui si alzavano la moglie e la cognata, aveva bevuto il caffè con loro e se n'era andato a piedi a una fattoria dove si doveva mettere in funzione una battitrice, impiantata di recente, per la preparazione delle sementi. 

Tutto quel giorno Levin, discorrendo con l'amministratore e i contadini, e a casa con la moglie, con Dolly e i bambini e con il suocero, aveva pensato sempre all'unica cosa che l'occupava in quel tempo, oltre le cure dell'azienda; e in tutto aveva cercato un riferimento alla propria domanda: ``Che cosa sono mai io? e dove sono? e perché sono qui?''. 

Stando in piedi, al fresco di un granaio ricoperto di recente da una grata di foglie odorose di nocciuolo non ancora fissata ai freschi travicelli scortecciati di tremula dal tetto di paglia, Levin guardava attraverso il portone aperto, ora la polvere secca e amara della battitura, che si ammucchiava e saltellava, ora l'erba dell'aia illuminata dal sole caldo e la paglia fresca allora allora portata fuori da una tettoia, ora le rondini dal capo variegato e il petto bianco, che con un sibilo entravano volando sotto il tetto, e battendo le ali si fermavano nel vano del portone, ora la gente che formicolava nel granaio scuro e polveroso. Pensieri strani gli venivano in mente. 

``Perché si fa tutto questo? - pensava. - Perché io sto qui, perché li costringo a lavorare? Come mai sono tutti affaccendati e cercano di mostrami il loro zelo? Perché mai questa vecchia Matrëna si affanna, amica mia? (L'ho curata quando nell'incendio le cadde una trave addosso) - pensava, guardando la vecchia allampanata che, movendo il grano col rastrello, camminava con sforzo per l'aia diseguale e scabra coi piedi nudi abbronzati. - Allora è guarita, ma oggi o domani, fra dieci anni la metteranno sotto terra e non rimarrà nulla né di lei, né di questa elegantona con la giacchetta rossa che toglie la spiga dalla pula con un movimento così agile, delicato. Anche lei sotterreranno, e questo cavallo pezzato molto presto - pensava guardando un cavallo che trascinava il ventre a stento e respirava frequentemente con le froge rigonfie, oltrepassando una ruota ricurva che gli si moveva sotto - lo sotterreranno, e anche Fëdor il porgitore, con quella barba ricciuta piena di pula e la camicia strappata sulla spalla bianca, sotterreranno. E lui rompe i covoni, e comanda qualcosa, e sgrida le donne, e con un movimento rapido accomoda la cinghia del volante. E soprattutto, non soltanto loro, ma me sotterreranno, e non ne rimarrà più nulla. A che scopo?''. 

Pensava, e nello stesso tempo guardava l'orologio per calcolare quanto si batteva in un'ora. Aveva bisogno di saperlo per assegnare il da fare per la giornata, calcolando da questo. 

``Presto sarà un'ora, e hanno cominciato soltanto il terzo mucchio'' pensò Levin, si avvicinò al porgitore e, sovrastando con la voce il rumore della macchina, gli disse di porgere meno fitto. 

- Ne dài troppo alla volta, Fëdor! Vedi, si ostruisce, perciò non rende. Fallo pari! 

Fëdor, annerito dalla polvere appiccicatasi al viso sudato, gridò qualcosa in risposta, ma continuava a non fare come voleva Levin. 

Levin, avvicinatosi al cassone, allontanò Fëdor e si mise lui stesso a porgere. 

Dopo aver lavorato fino al pranzo dei contadini, prima del quale non rimaneva molto tempo, uscì dal granaio insieme col porgitore e si mise a parlare con lui, fermandosi accanto a una bica gialla di segala mietuta e disposta con precisione sull'aia per la sementa. Il porgitore era d'un villaggio lontano, di quello stesso in cui Levin aveva assegnato la terra al capo dell'artel'. Adesso era stata data in affitto al portiere. 

Levin si mise a parlare di questa terra con Fëdor il porgitore e domandò se per l'anno prossimo non avrebbe preso la terra Platon, un ricco e buon contadino dello stesso villaggio. 

- Il prezzo è caro, Platon non può guadagnarci, Konstantin Dmitric - rispose il contadino, tirando fuori delle spighe dal petto sudato. 

- Ma come mai Kirillov guadagna? 

- Mitjucha - così il contadino chiamò con disprezzo il portiere - come non dovrebbe guadagnare, Konstantin Dmitric! Lui pigia e tira fuori il suo. Non ha compassione d'un cristiano. Ma zio Fokanyc - così egli chiamava il vecchio Platon - si metterà forse a strappare la pelle all'uomo? Dove darà a credito, dove anche calerà sul prezzo. E non arriverà a guadagnare. Lui agisce da uomo. 

- Ma perché mai calerà sul prezzo? 

- Ma così, perché le persone sono diverse; uno vive solo per il proprio bisogno, come, per esempio, Mitjucha, pensa solo a riempirsi la pancia; ma Fokanyc è un vecchio veritiero. Vive per l'anima. Si ricorda di Dio. 

- Come si ricorda di Dio? Come vive per l'anima? - gridò, quasi, Levin. 

- Ma si sa come: secondo la verità, secondo il volere di Dio. Perché le persone sono diverse. Ecco, prendiamo magari voi, anche voi non offendereste un uomo\ldots{} 

- Sì, sì, addio! - esclamò Levin, ansimando per l'agitazione e, voltatosi, prese il bastone e andò via rapidamente verso casa. Alle parole del contadino su Fokanyc che viveva per l'anima, secondo verità, secondo il volere di Dio, pensieri confusi, densi di significato, pareva avessero fatto irruzione in folla, come se venissero da chi sa quale luogo remoto e che, tendendo tutti a una sola mèta, si fossero messi a turbinare nel suo capo, accecandolo con la loro luce. 

\capitolo{XII}\label{xii-7} 

Levin camminava a grandi passi per la strada maestra, prestando attenzione non tanto ai propri pensieri (non riusciva ancora a distinguerli), quanto allo stato d'animo, non provato mai prima. 

Le parole dette dal contadino avevano prodotto nell'animo suo l'azione di una scintilla elettrica che avesse trasformato e unito in una cosa sola l'intero sciame di pensieri disordinati, impotenti, staccati, che non avevano mai cessato di accompagnarlo. Questi pensieri lo avevano occupato senza che egli stesso se ne accorgesse anche nel momento in cui aveva parlato dell'assegnazione della terra. 

Sentiva nell'animo suo qualcosa di nuovo e palpava con godimento questa cosa nuova, non sapendo cosa fosse. 

``Vivere non per i propri bisogni, ma per Dio. Per quale Dio? E cosa si può dire di più insensato di quello ch'egli ha detto? Ha detto che non bisogna vivere per i propri bisogni, cioè che non bisogna vivere per quello che comprendiamo, verso cui siamo attratti, di cui sentiamo desiderio, ma che bisogna vivere per qualcosa di incomprensibile, per un Dio che nessuno può capire, né definire. E allora? Non ho forse inteso queste parole insensate di Fëdor? E, dopo averle intese, ho forse dubitato della loro giustezza? le ho giudicate sciocche, poco chiare, inesatte? 

No, le ho intese e proprio così come intende lui, le ho intese pienamente e con maggiore chiarezza ch'io non intenda qualunque altra cosa nella vita; e mai nella mia vita ho dubitato né posso dubitare di questo. E non io solo, ma tutto, tutto il mondo intende pienamente questa sola cosa e di questa sola cosa non dubita e vi consente sempre. 

Fëdor dice che il portiere Kirillov vive per la pancia. È comprensibile e ragionevole. Noi tutti, esseri ragionevoli, non possiamo vivere altrimenti che per la pancia. E a un tratto lo stesso Fëdor dice che vivere per la pancia è male e che bisogna vivere per la verità, per Dio, e io lo intendo da un accenno! E io, e milioni di uomini che hanno vissuto secoli fa e che vivono adesso, i contadini, i poveri di spirito e i saggi, che hanno pensato e scritto su questo e dicono la stessa cosa con il loro linguaggio confuso, noi tutti siamo d'accordo su questa sola cosa: per che cosa si debba vivere e cosa sia il bene. Io con tutte le persone non ho che una sola conoscenza ferma, indubitabile e chiara; e questa conoscenza non può essere spiegata con la ragione; è al di fuori di essa e non ha nessuna causa e non può avere nessun effetto. 

Se il bene ha una causa, non è più bene; se ha un effetto, la ricompensa, pure non è bene. Perciò, il bene è al di fuori della catena delle cause e degli effetti. 

E questo appunto io lo so e tutti lo sappiamo. 

Ma io cercavo dei miracoli, mi rammarico di non aver visto un miracolo che mi avesse persuaso. Ed ecco il miracolo, l'unico possibile, continuamente attuale, che da ogni parte mi circonda, e io non me ne accorgevo! 

Quale miracolo può essere mai più grande di questo? 

``Possibile che io abbia trovato la soluzione di tutto, che le mie pene adesso siano finite?'' pensava Levin, camminando per la strada polverosa, senza sentire né caldo, né stanchezza e provando una sensazione di sollievo dopo una lunga sofferenza. Questa sensazione era così gioiosa che gli sembrava inverosimile. Ansava per l'agitazione e, non avendo la forza di andare avanti, scese dalla strada nel bosco e sedette all'ombra delle tremule sull'erba non falciata. Tolse il cappello dalla testa sudata e si coricò, appoggiandosi a un braccio, sull'erba del bosco, densa di umori e simile alla bardana. 

``Sì, bisogna riaversi e capire'' pensava, guardando l'erba non calpestata che era dinanzi a lui, e seguendo i movimenti di un piccolo scarabeo verde che saliva su per lo stelo d'una gramigna ed era trattenuto, nella sua ascesa, da un filo d'erba egizia. ``Che cosa ho scoperto? - si domandò, voltando dall'altra parte lo stelo perché non disturbasse lo scarabeo, e piegando un altro filo d'erba perché lo scarabeo passasse su di esso. - Che cosa mi rallegra? che cosa ho mai scoperto? 

Prima io dicevo che nel mio corpo, come in questo filo d'erba e in questo piccolo scarabeo (ecco che non ha voluto andare su questo filo d'erba, ha raddrizzato le ali, ha preso il volo), si compiva, secondo le leggi fisiche, chimiche, fisiologiche, uno scambio di materia. E in tutti noi, insieme con le tremule, con le nubi e le nebulose si compiva una evoluzione. Evoluzione da che cosa? in che cosa? Una sconfinata evoluzione e lotta\ldots{} Come se ci potesse essere una qualche direzione e una lotta nell'infinito! E mi stupivo che, malgrado la più grande tensione di pensiero su questa via, non mi si scoprisse, tuttavia, il senso della vita, il senso dei miei impulsi e delle mie aspirazioni. Ma il senso dei miei impulsi è così chiaro in me, che io costantemente vivo secondo questo e mi sono sorpreso e mi sono rallegrato quando il contadino me lo ha enunciato: vivere per Dio, per l'anima. 

Io non ho scoperto nulla. Ho soltanto imparato a conoscere quello che sapevo. Ho capito quella forza che, non solo nel passato, la vita m'ha data, ma che mi dà anche adesso. Mi sono liberato da un inganno, ho imparato a conoscere il padrone''. 

E in breve ripeté a se stesso tutto il corso dei propri pensieri in quegli ultimi due anni, sorti con la chiara, evidente idea della morte alla vista del fratello caro, malato senza speranza. 

Avendo allora, per la prima volta, capito chiaramente che per ogni uomo come lui non c'era niente, all'infuori della sofferenza, della morte, dell'oblio completo, aveva deciso che così non si poteva vivere, che bisognava o spiegare la propria vita in modo che non apparisse una malvagia irrisione d'un qualche demone, o spararsi. 

Non aveva fatto né l'una né l'altra cosa, e aveva continuato a vivere, invece, a pensare e a sentire; in quel tempo aveva perfino preso moglie e aveva provato molte gioie ed era stato felice, sempre che non avesse pensato al senso della propria vita. 

Che cosa mai significava questo? Significava ch'egli aveva vissuto bene, ma aveva pensato male. 

Aveva vissuto (senza averne coscienza) di quelle verità spirituali ch'egli aveva succhiato col latte, e aveva pensato non solo senza riconoscere queste verità, ma eludendole con cura. 

Adesso gli era chiaro che egli aveva potuto vivere soltanto grazie a quella fede in cui era stato educato. 

``Che cosa sarei e come avrei vissuto la mia vita, se non avessi avuto questa fede, se non avessi saputo che bisognava vivere per Dio e non per i propri bisogni? Avrei rubato, mentito, ucciso. Nulla di quello che costituisce la gioia principale della mia vita esisterebbe per me''. E, facendo i più grandi sforzi di immaginazione, non riusciva tuttavia a figurarsi l'essere feroce che lui stesso sarebbe stato se non avesse saputo perché viveva. 

``Io cercavo una risposta alla mia domanda. E la risposta alla mia domanda non poteva darmela il pensiero: esso è incommensurabile con la domanda. La risposta me l'ha data la stessa vita nella mia conoscenza di quello che è bene e di quello che è male. E questa conoscenza non l'ho acquistata con nulla, ma mi è stata data insieme agli altri, data perché non la potevo prendere da nessuna parte. 

Di dove ho preso ciò? Sono forse giunto con la ragione a convincermi che bisogna amare il prossimo e non soffocarlo? Me l'hanno detto nell'infanzia, e io ci ho creduto con gioia, perché mi dicevano quello che avevo nell'animo. E chi l'ha scoperto? Non la ragione. La ragione ha scoperto la lotta per l'esistenza e la legge che esige che siano soffocati tutti quelli che ostacolano il soddisfacimento dei miei desideri. Questa è la deduzione della ragione. E che si debba amare un altro non poteva scoprirlo la ragione, perché è una cosa irragionevole''. 

``Sì, orgoglio'' si disse, buttandosi sul ventre e cominciando a legare in nodo gli steli delle erbe, cercando di non spezzarli. 

``E non soltanto orgoglio dell'intelletto, ma insipienza dell'intelletto. E soprattutto un inganno, proprio un inganno dell'intelletto. Vera e propria frode dell'intelletto'' ripeté. 

\capitolo{XIII}\label{xiii-7} 

E a Levin venne in mente una scena svoltasi di recente fra Dolly e i bambini. I bambini, rimasti soli, avevano cominciato a bruciare i lamponi sulle candele e a bere il latte a garganella. La madre, coltili sul fatto, in presenza di Levin aveva cominciato a predicare loro quanta fatica costasse ai grandi quello che loro distruggevano, e che questa fatica si faceva per loro, che se avessero rotto le tazze, non avrebbero avuto dove bere il tè, e se avessero versato il latte, non avrebbero avuto nulla da mangiare e sarebbero morti di fame. 

E Levin era rimasto colpito dalla calma, sommessa incredulità con la quale i bambini avevano ascoltato le parole della mamma. Erano dispiaciuti solo perché era cessato per loro un giuoco attraente, ma non credevano neppure una parola di quel che diceva la madre. E non potevano credere perché non potevano immaginare, in tutto il suo complesso, quello di cui godevano, e non potevano quindi immaginare che quello che distruggevano fosse proprio quello di cui si nutrivano. 

``Tutto questo va da sé - pensavano - e d'interessante e d'importante in questo non c'è nulla, perché questo è sempre stato e così sarà. È sempre la stessa cosa. A questo non dobbiamo pensare, è già bell'e pronto; ma noi abbiamo voglia di inventare qualcosa di nostro e di nuovo. E noi abbiamo inventato che nella tazza ci mettiamo i lamponi e poi li arrostiamo su una candela e il latte ce lo versiamo direttamente in bocca l'un l'altro. Questo è allegro e nuovo e non è peggiore del sistema di bere nelle tazze''. 

``Non facciamo forse lo stesso noi, non lo facevo io, cercando con la ragione il senso delle forze della natura e il senso della vita dell'uomo?'' continuò a pensare Levin. 

``E non fanno forse la stessa cosa tutte le teorie filosofiche, per la via del pensiero, strana e impropria dell'uomo, conducendolo alla conoscenza di quello ch'egli da lungo tempo conosce e sa con tanta giustezza, che senza di quello non potrebbe neppur vivere? Non si vede forse chiaramente, nell'evoluzione della teoria di ogni filosofo, ch'egli sa in precedenza, in modo altrettanto indubitabile quanto il contadino Fëdor, ma per nulla affatto in modo più chiaro, il senso principale della vita e soltanto per la dubbia via dell'intelletto vuole tornare a quello che tutti sanno? 

Ma, lasciamo soli i bambini, che si procurino, che si facciano da loro stessi le stoviglie, mungano il latte, ecc. Si metterebbero forse a far birichinate? Morirebbero di fame. E allora, lasciateci andare con le nostre passioni, coi nostri pensieri, senza l'idea dell'unico Dio e Creatore! O almeno senza l'idea di quel che sia bene, senza la spiegazione del male morale. 

E allora, provate a costruire qualcosa senza queste idee! 

Noi distruggiamo solo perché siamo spiritualmente sazi. Proprio come i bambini! 

Donde ricevo la conoscenza gioiosa, comune al contadino, che solo mi dà la tranquillità dell'anima? Da dove ho preso ciò? 

Io, educato nel concetto di Dio, da cristiano, dopo aver riempito tutta la mia vita di quei beni spirituali che mi ha dato il cristianesimo, ricolmo di questi beni e di essi vivente, io, come un bambino, senza capirli, distruggo, cioè voglio distruggere quello di cui vivo. E non appena incombe un momento grave della vita, come i bambini quando hanno freddo e fame, vado verso di Lui, e ancora meno dei bambini, che la madre sgrida per le loro birichinate infantili, sento che i miei infantili tentativi di agitarmi per troppo benessere non mi sono valsi. 

Sì, quello che so, non lo so con la ragione, ma mi è dato, mi è rivelato e io lo intendo attraverso il cuore, credo nella cosa principale che professa la Chiesa''. 

``La Chiesa, la Chiesa!'' ripeté Levin, mettendosi a giacere dall'altro lato e, appoggiandosi su di un braccio, cominciò a guardare lontano un gregge che dall'altra parte scendeva verso il fiume. 

``Ma posso io credere a tutto quello che professa la Chiesa? - pensava, mettendosi alla prova e immaginando tutto quello che avrebbe potuto distruggere la sua calma di adesso. Cominciò a ricordare proprio quelle dottrine della Chiesa che più gli erano parse strane e che lo inducevano in tentazione. - La creazione? E con che cosa mai spiegavo l'esistenza? Con l'essere? Col nulla? Il demone e il peccato? E con che cosa spiegavo il male?\ldots{} Il Redentore?\ldots{} 

Ma io nulla, nulla so, né posso sapere se non quello che mi è stato detto insieme agli altri''. 

E adesso gli sembrava che non ci fosse neppure una delle credenze della Chiesa che infrangesse la più grande: la fede in Dio, nel bene, come unica missione dell'uomo. A ogni credenza della Chiesa poteva essere sostituita la credenza nel soddisfare la verità in luogo delle proprie esigenze. E ognuna non solo non distruggeva queste, ma era indispensabile, perché si compisse quel miracolo essenziale, che continuamente appariva sulla terra, e che consisteva nel fatto che fosse possibile a ognuno, d'accordo con milioni di persone fra le più svariate, sagge o folli, giovani o vecchie, d'accordo con tutti, col contadino, con L'vov, con Kitty, con i mendicanti e con i sovrani, intendere indubitatamente la stessa cosa e comporre quella vita dell'anima, per la quale soltanto vale la pena di vivere e che sola apprezziamo. 

Sdraiato sul dorso, egli ora guardava il cielo alto, senza nuvole. ``Non so io forse che quello è lo spazio infinito e non già una volta rotonda? Ma per quanto socchiuda gli occhi e sforzi la vista, non posso non vederlo rotondo e limitato, e, malgrado la mia conoscenza dello spazio infinito, ho senza dubbio alcuno ragione quando vedo una solida volta azzurra, e ho più ragione che non quando mi sforzo di vedere al di là di essa''. 

Levin cessò di pensare e pareva soltanto prestare ascolto alle voci misteriose che, con gioia e con affanno, parlavano fra di loro di una certa cosa. 

``È forse questa la fede? - pensò, temendo di credere alla propria felicità. - Dio mio, Ti ringrazio!'' egli pronunciò, soffocando i singhiozzi che salivano e asciugando con tutte e due le mani le lacrime di cui gli s'erano riempiti gli occhi. 

\capitolo{XIV}\label{xiv-7} 

Levin guardava dinanzi a sé e vedeva il gregge, poi il suo calesse al quale era attaccato Voronoj, e il cocchiere che, avvicinatosi al gregge, diceva qualcosa al pastore; sentì poi vicino a sé il suono delle ruote e lo sbuffare del cavallo ben pasciuto; ma era così assorto nei suoi pensieri che non suppose neppure perché il cocchiere venisse verso di lui. 

Se ne ricordò solo quando il cocchiere gli fu proprio accosto e lo chiamò. 

- Ha mandato la signora. Sono arrivati il fratello e anche un certo signore. 

Levin salì sul calesse e prese le redini. 

Come se si fosse svegliato da un sonno, Levin stentò a tornare in sé. Esaminava il cavallo ben pasciuto che s'era ricoperto di schiuma fra le cosce e sul collo, dove fregavano le redini, esaminava il cocchiere Ivan, ch'era seduto accanto a lui, e ricordava che attendeva il fratello, che sua moglie probabilmente era inquieta per la sua lunga assenza, e cercava di indovinare chi fosse l'ospite arrivato col fratello. E il fratello, e la moglie e l'ospite ignoto adesso gli apparivano diversamente da prima. Gli sembrava che ormai i rapporti con tutte le persone sarebbero stati diversi. 

``Con mio fratello adesso non ci sarà più quel distacco che c'è sempre stato fra noi! discussioni non ce ne saranno; con Kitty non ci saranno più litigi; con l'ospite, chiunque esso sia, sarò affabile e buono; con la servitù, con Ivan, tutto sarà diverso''. 

Trattenendo con le redini tese il buon cavallo che sbuffava d'impazienza e chiedeva di camminare, Levin si voltava a guardare Ivan seduto accanto a lui, che non sapeva che cosa fare con le sue mani rimaste inoperose e premeva continuamente la camicia che si gonfiava. Cercava un pretesto per cominciare un discorso con lui. Egli voleva dire che Ivan aveva fatto male a tirare in alto la cinghia della stanga, ma questo poteva sembrare un rimprovero, e lui avrebbe voluto un discorso cordiale. Ma non gli veniva altro in mente. 

- Vi prego di prendere a destra, altrimenti troverete un ceppo - disse il cocchiere, correggendo Levin nella guida. 

- Per favore, non toccare e non farmi la lezione! - disse Levin, stizzito da questa intromissione del cocchiere. Come sempre, l'intromissione aveva causato la stizza, ed egli sentì subito, con tristezza, quanto errata fosse la supposizione che il suo stato d'animo potesse mutarlo immediatamente a contatto con la realtà. 

Circa un quarto di versta prima di giungere a casa, Levin scorse Tanja e Griša che gli correvano incontro. 

- Zio Kostja! Stanno venendo la mamma e il nonno, e Sergej Ivanyc, e ancora qualcuno - dicevano, arrampicandosi sul calesse. 

- Ma chi? 

- Uno terribilissimo! E fa così con le braccia - disse Tanja, sollevandosi sul calesse e facendo il verso a Katavasov. 

- Ma vecchio o giovane? - domandò ridendo Levin, al quale il gesticolare di Tanja aveva ricordato qualcuno. 

``Ah, purché non sia una persona spiacevole!'' pensò Levin. 

Soltanto dopo aver girato oltre la svolta della strada e dopo aver visto coloro che gli venivano incontro, Levin riconobbe Katavasov in cappello di paglia, che camminava agitando le braccia, proprio come aveva fatto Tanja. 

A Katavasov piaceva molto parlare di filosofia, avendone egli stesso ricevuta un'idea dai naturalisti che non si erano mai occupati di filosofia; e a Mosca, negli ultimi tempi, Levin aveva discusso molto con lui. 

E una di quelle conversazioni, in cui Katavasov, evidentemente, pensava d'aver avuto il sopravvento, fu la prima cosa che ricordò Levin, dopo averlo riconosciuto. 

``No, ormai discutere ed esprimere alla leggera le mie idee, non lo farò a nessun costo'' pensò. 

Sceso dal calesse e salutati il fratello e Katavasov, Levin domandò della moglie. 

- Ha portato Mitja al Kolok - era un bosco vicino casa. - Voleva sistemarlo là, in casa c'è caldo - disse Dolly. Levin aveva sempre sconsigliato la moglie di portare il bambino nel bosco, ritenendolo pericoloso, e questa notizia gli spiacque. 

- Corre con lui da una parte all'altra - disse sorridendo il principe. - Io le ho consigliato di portarlo sulla ghiacciaia. 

- Voleva venire dalle api. Pensava che tu fossi là - disse Dolly. 

- Be', che fai? - disse Sergej Ivanovic, rimanendo indietro agli altri e mettendosi all'altezza del fratello. 

- Ma, nulla di speciale. Come sempre mi occupo dell'azienda - rispose Levin. - Ebbene, tu rimani qui per molto? Ti aspettavamo da tempo. 

- Per un paio di settimane. C'è molto da fare a Mosca. 

A queste parole gli occhi dei fratelli si incontrarono e Levin, malgrado il desiderio costante e ora particolarmente forte in lui di essere in rapporti cordiali, e soprattutto semplici, con il fratello, sentì che provava un senso di disagio a guardarlo. Abbassò gli occhi, e non seppe che dire. 

Cercando argomenti di conversazione che potessero piacere a Sergej Ivanovic e distrarlo dal discorso sulla guerra serba e sulla questione slava, a cui aveva alluso accennando alle occupazioni di Mosca, Levin cominciò a parlare del libro di Sergej Ivanovic. 

- Ci sono state recensioni, dunque, del tuo libro? - domandò. 

Sergej Ivanovic sorrise della domanda premeditata. 

- Nessuno se ne occupa e io meno degli altri - disse. - Guardate, Dar'ja Aleksandrovna, ci sarà una pioggerella - soggiunse, indicando con l'ombrello alcune nuvolette bianche ch'erano apparse sopra le cime delle tremule. 

E bastarono queste parole perché quei rapporti reciproci non ostili, ma freddi, che Levin voleva tanto evitare, si stabilissero di nuovo tra i fratelli. Levin si avvicinò a Katavasov. 

- Come avete fatto bene a pensar di venire! - gli disse. 

- Mi preparavo da molto tempo. Adesso discorreremo, vedremo. Spencer l'avete letto? 

- No, non l'ho finito di leggere - disse Levin. - Del resto, non ne ho bisogno, ora. 

- Come mai? è interessante. Perché? 

- Cioè, mi sono definitivamente convinto che le soluzioni dei problemi che mi interessano non le troverò né in lui né in quelli simili a lui. Adesso\ldots{} 

Ma l'espressione calma e allegra del viso di Katavasov lo colpì, a un tratto, e gli venne pietà dello stato d'animo proprio, che, evidentemente, egli turbava con quella conversazione; si ricordò del suo proposito e si fermò. 

- Del resto, parleremo dopo - soggiunse. - Se vogliamo andare all'arniaio, allora di qua, per questo sentiero - disse rivolto a tutti. 

Giunti, per un sentiero stretto, a una prateria non falciata, coperta da una parte da violacciocche vivaci, in mezzo alle quali s'infoltivano i cespugli alti color verde scuro dell'elleboro, Levin dispose i suoi ospiti all'ombra spessa e fresca delle giovani tremule, su di una panchina e su di alcuni tronchi, preparati apposta per i visitatori dell'arniaio che temevano le api; lui stesso andò sul limitare, per portare ai bambini e ai grandi pane, cetrioli e miele fresco. 

Cercando di fare pochi movimenti rapidi e prestando ascolto alle api che gli volavano intorno sempre più frequenti, giunse per il sentiero fino all'izba. Proprio all'ingresso un'ape, impigliataglisi nella barba, cominciò a ronzare, ma egli la liberò con cautela. Entrato nell'ingresso ombroso, tolse dal muro la maschera appesa a un gancio e, messala e ficcate le mani in tasca, entrò nell'arniaio recinto. Qua, in file regolari, legate con rami di tiglio ai pali, stavano, in mezzo al luogo falciato, le vecchie arnie, ognuna con la propria storia, e lungo le pareti della siepe le nuove, messe lì quell'anno. Dinanzi alle aperture delle arnie le api e i fuchi, che giravano e si urtavano nello stesso posto, abbagliavano la vista, e in mezzo a loro, sempre nella stessa direzione, verso il bosco, su di un tiglio fiorito e indietro verso le arnie, passavano volando le api operaie cariche di polline e in cerca di bottino. 

Negli orecchi non cessavano di echeggiare suoni vari, ora d'un'ape operaia intenta al lavoro, che passava volando rapida, ora d'un fuco strombettante, ozioso, ora di api sentinelle agitate, che difendevano dal nemico il proprio bene, pronte a pungere. Dalla parte della cinta il vecchio piallava un cerchio e non aveva visto Levin che, in silenzio, si fermò in mezzo all'arniaio. 

Era contento dell'occasione di rimanere solo, per tornare in sé dalla realtà, che già aveva fatto in tempo a umiliare il suo stato d'animo. 

Ricordò che aveva già avuto il tempo di arrabbiarsi con Ivan, di mostrar freddezza verso il fratello e di parlare con leggerezza con Katavasov. 

``Possibile che sia stato solo lo stato d'animo di un istante e che tutto passi senza lasciare traccia?'' pensò. 

Ma nello stesso momento, tornato al proprio stato d'animo sentì con gioia che qualcosa di nuovo e d'importante era accaduto in lui. La realtà aveva velato solo temporaneamente quella calma dell'anima che egli aveva trovato, e che tuttavia era intatta in lui. 

Proprio come le api che adesso gli volteggiavano intorno, lo minacciavano e lo distraevano, gli toglievano la piena calma fisica, lo obbligavano a contrarsi per sfuggirle, proprio così le preoccupazioni, avvolgendolo, dal momento in cui era salito sul calesse, lo avevano privato della libertà dell'anima; ma ciò era durato soltanto finché vi era rimasto in mezzo. Come, malgrado le api, la sua forza fisica era intatta, così era anche intatta la sua forza spirituale, recentemente scoperta da lui. 

\capitolo{XV}\label{xv-7} 

- E sai, Kostja, con chi è venuto in qua Sergej Ivanovic? - disse Dolly, dopo aver distribuito cetrioli e miele ai bambini. - Con Vronskij! Va in Serbia. 

- E neppure solo, ma conduce uno squadrone a sue spese! - disse Katavasov. 

- Questo gli si addice - disse Levin. - Ma vanno forse ancora volontari? - soggiunse, dopo aver guardato Sergej Ivanovic. 

Sergej Ivanovic, senza rispondere, tirava fuori cautamente con un coltello smussato, da una tazza, nella quale c'era in un angolo un favo bianco di miele, un'ape ancora viva, appiccicatasi al miele ch'era colato di sotto. 

- E come ancora! Se aveste veduto cosa c'era ieri alla stazione! - disse Katavasov, addentando rumorosamente un cetriolo. 

- Su, e come capire ciò? In nome di Cristo, Sergej Ivanovic, spiegatemi: dove vanno tutti questi volontari, con chi sono in guerra? - domandò il vecchio principe, evidentemente continuando una conversazione avviata quando Levin ancora non c'era. 

- Coi turchi - rispose Sergej Ivanovic, sorridendo tranquillamente, dopo aver liberato l'ape che muoveva le zampine senza speranza di soccorso, annerita dal miele, e facendola scendere dal coltello su di una foglia spessa di tremula. 

- Ma chi ha mai dichiarato la guerra ai turchi? Ivan Ivanovic Ragozov e la contessa Lidija Ivanovna con la signora Stahl? 

- Nessuno ha dichiarato la guerra, ma la gente compatisce le sofferenze del prossimo e desidera aiutarlo - disse Sergej Ivanovic. 

- Ma il principe non parla d'aiuto - disse Levin, prendendo le parti del suocero - ma di guerra. Il principe dice che i privati non possono prendere parte a una guerra senza il permesso del governo. 

- Kostja, guarda, è un'ape! Davvero, ci pungerà tutti! - disse Dolly, difendendosi da una vespa. 

- Ma questa non è neanche un'ape, è una vespa - disse Levin. 

- Ebbene, qual'è la vostra teoria? - disse con un sorriso Katavasov a Levin, evidentemente per sfidarlo a una discussione. - Perché i privati non ne hanno il diritto? 

- Be', la mia teoria è questa: la guerra, da una parte è una cosa tanto bestiale, crudele e tremenda, che nessun uomo, non dico poi un cristiano, può prendersi personalmente la responsabilità di cominciare una guerra, ma lo può soltanto un governo che vi sia chiamato e vi sia condotto ineluttabilmente. Dall'altra parte, e secondo coscienza, e secondo il buon senso, negli affari di stato, in particolare nella questione di guerra, i cittadini rinunciano alla propria volontà personale. 

Sergej Ivanovic e Katavasov cominciarono a parlare nello stesso tempo con obiezioni già pronte. 

- Sta proprio lì il fatto, amico mio, che ci possono essere dei casi, in cui il governo non adempie la volontà dei cittadini, e allora la società dichiara la propria volontà - disse Katavasov. 

Ma Sergej Ivanovic, evidentemente, non approvava questa opinione. Alle parole di Katavasov aggrottò le sopracciglia e disse un'altra cosa. 

- Fai male a porre la questione così. Qui non c'è dichiarazione di guerra, ma semplicemente l'espressione di un sentimento umano, cristiano. Uccidono i fratelli, dello stesso sangue e della stessa fede. Ammettiamo che non siano né fratelli, né della stessa fede, ma semplicemente dei bambini, delle donne e dei vecchi; il sentimento s'indigna, e la gente russa corre per aiutare e far cessare questi orrori. Immagina di camminare per una strada e di vedere che degli ubriachi percuotono una donna o un bambino; io penso che tu non staresti là a domandare se sia dichiarata o no la guerra a quest'uomo, ma ti scaglieresti su di lui e difenderesti l'offeso. 

- Ma non l'ucciderei - disse Levin. 

- Sì, lo uccideresti. 

- Non so, se, vedendo questo, mi abbandonerei al mio sentimento immediato; ma in precedenza non posso dirlo. E un sentimento simile, immediato, per l'oppressione degli slavi non c'è e non può esserci. 

- Forse per te no. Ma per gli altri c'è - disse Sergej Ivanovic, aggrottando le sopracciglia, con aria scontenta. - Nel popolo è viva la tradizione sulla gente ortodossa che soffre sotto il giogo degli ``empi agareni''. Il popolo ha sentito le sofferenze dei suoi simili e ha parlato. 

- Forse - disse evasivo Levin - ma io non lo vedo; io stesso sono popolo e non lo sento. 

- Ecco, anch'io - disse il principe. - Stavo all'estero, leggevo i giornali e, confesso, ancora prima degli orrori bulgari, non capivo in nessun modo perché tutti i russi, a un tratto, avessero preso tanto ad amare i fratelli slavi, mentre io non sentivo nessun amore per loro. Mi addoloravo molto, pensavo d'essere un mostro o che fosse Karlsbad ad agire così su di me. Ma arrivato qua, mi sono tranquillizzato; vedo che oltre me ci sono anche altre persone che si interessano soltanto della Russia, e non dei fratelli slavi. Ecco, anche Konstantin. 

- Le opinioni individuali qui non significano niente - disse Sergej Ivanovic - non c'entrano le opinioni personali quando tutta la Russia, il popolo, ha espresso la sua volontà. 

- Ma scusatemi, io non lo vedo. Il popolo, quanto a saperlo, non lo sa - disse il principe. 

- No, papà\ldots{} e come no? E domenica in chiesa? - disse Dolly prestando ascolto alla conversazione.- Dammi un asciugamano, per favore - disse al vecchio che guardava i bambini con un sorriso. - Non può essere che tutti\ldots{} 

- Ma cosa domenica in chiesa? Al prete hanno dato l'ordine di leggere. Lui ha letto. Loro non hanno capito nulla, sospiravano come ad ogni predica - seguitò il principe. - Poi hanno detto loro che, ecco, si faceva una colletta in chiesa per un'impresa salutare per l'anima; ebbene, loro hanno tirato fuori una copeca per ciascuno e l'hanno data, ma per cosa, non lo sanno loro stessi. 

- Il popolo non può non sapere; la coscienza dei propri destini c'è sempre nel popolo, e in momenti come questi essa si chiarisce in lui - disse recisamente Sergej Ivanovic, gettando uno sguardo al vecchio apicultore. 

Il bel vecchio dalla barba nera con qualche pelo grigio e dai capelli folti d'argento, stava in piedi, immobile, tenendo la tazza col miele, guardando i signori con cordialità e con calma dall'alto della sua statura, evidentemente senza capire e non desiderando di capire nulla. 

- È proprio così - diss'egli alle parole di Sergej Ivanovic, scotendo significativamente il capo. 

- Ma ecco, domandate a lui. Lui non sa e non pensa nulla - disse Levin. - Hai sentito della guerra, Michajlyc? - si rivolse a lui. - Ecco, quello che hanno letto in chiesa. Tu che ne pensi mai? Dobbiamo far la guerra per i cristiani? 

- Che dobbiamo mai pensare? Aleksandr Nikolaevic, lo zar, ha pensato per noi, penserà per noi anche in tutti gli affari. Lui vede meglio\ldots{} Devo portare dell'altro pane? Posso darne ancora al bambino? - disse rivolto a Dar'ja Aleksandrovna, indicando Griša, che finiva di mangiare una crosta. 

- Io non ho bisogno di domandare - disse Sergej Ivanovic; - abbiamo visto e vediamo centinaia e centinaia di persone che abbandonano tutto per servire una causa giusta, vengono da tutte le estremità della Russia ed esprimono sinceramente e chiaramente il proprio pensiero e scopo. Portano i loro soldi o vanno loro stessi e dicono sinceramente perché. E cosa vuol dire questo? 

- Vuol dire, secondo me - disse Levin che cominciava a scaldarsi - che in un popolo di ottanta milioni di uomini si troveranno sempre non centinaia, come adesso, ma decine di migliaia di persone che hanno perduto una posizione sociale, persone turbolente, che sono sempre pronte a entrare nella banda di Pugacëv, a Chiva, in Serbia\ldots{} 

- Io ti dico che non sono centinaia e non sono persone turbolente, ma i migliori rappresentanti del popolo! - disse Sergej Ivanyc con un'irritazione tale, come se avesse difeso il suo ultimo bene. - E le offerte? In questo tutto il popolo esprime la propria volontà. 

- Questa parola ``popolo'' è indeterminata - disse Levin. - Gli scrivani comunali, i maestri e i contadini uno su mille, forse, sanno di che si tratta. Ma gli altri ottanta milioni, come Michajlyc, non solo non esprimono la propria volontà, ma non hanno neppure la minima idea di quello su cui dovrebbero esprimere la propria volontà. Che diritto abbiamo mai di dire che è la volontà del popolo? 

\capitolo{XVI}\label{xvi-7} 

Sergej Ivanovic, esperto in dialettica, senza obiettare, portò immediatamente la conversazione su di un altro campo. 

- Ma se tu vuoi conoscere lo spirito del popolo in maniera matematica, allora, s'intende, ottenere ciò è molto difficile. E il suffragio non è introdotto da noi e non può essere introdotto, perché non esprime la volontà del popolo; ma per questo ci sono altre vie. Si sente nell'aria, si sente nel cuore. Non parlo poi di quelle correnti sottomarine, che si sono mosse nel mare stagnante del popolo e che sono chiare per qualsiasi persona che non abbia prevenzioni; guarda la società in senso stretto. Tutti i più svariati partiti del mondo dell'intelligencija, tanto ostili prima, si sono fusi in una sola cosa. Ogni dissenso è finito, tutti gli organi pubblici dicono sempre la stessa cosa, tutti hanno sentito la forza naturale che li ha afferrati e li porta in una stessa direzione. 

- Ma sono i giornali che dicono tutti la stessa cosa - disse il principe. - È vero. Tutti la stessa cosa, proprio come ranocchie prima del temporale. E proprio per causa loro non si può sentir nulla. 

- Ranocchie o non ranocchie, io giornali non ne pubblico e non li voglio difendere; ma parlo dell'unità di pensiero nel mondo dell'intelligencija - disse Sergej Ivanovic, rivolto al fratello. Levin voleva rispondere, ma il vecchio principe lo interruppe. 

- Eh, via, su questa unità di pensiero si può dire ancora un'altra cosa - disse il principe. - Ecco, io ho un genero, Stepan Arkad'ic, lo conoscete. Adesso ha avuto il posto di membro del comitato di una commissione e qualcosa ancora, non ricordo. Certo là non c'è nulla da fare; ebbene, Dolly, non è mica un segreto, ma ci sono ottomila rubli di stipendio. Provate, chiedetegli se il suo impiego è utile, vi dimostrerà ch'è utilissimo. Ed è uomo sincero, ma non si può non credere all'utilità degli ottomila rubli. 

- Sì, mi ha pregato di riferire a Dar'ja Aleksandrovna che ha avuto il posto - disse scontento Sergej Ivanovic, ritenendo che il principe parlasse a sproposito. 

- Appunto così è l'unità di pensiero dei giornali. Me l'hanno spiegato: non appena c'è una guerra, hanno un reddito due volte maggiore. E come non devono ritenere che le sorti del popolo e degli slavi\ldots{} e tutto ciò? 

- A me molti giornali non piacciono, ma questo è ingiusto - disse Sergej Ivanovic. 

- Io porrei soltanto una condizione - seguitò il principe. - Alphonse Karr lo scrisse benissimo prima della guerra di Prussia: ``Voi stimate che la guerra sia indispensabile? Benissimo. Chi predica la guerra, vada in una legione speciale, d'avanguardia, e all'assalto, all'attacco, innanzi a tutti''. 

- Saranno belli i direttori! - disse Katavasov, mettendosi a ridere forte, immaginando i direttori che conosceva in quella legione scelta. 

- Macché, scapperanno via - disse Dolly - daranno soltanto noia. 

- E se scapperanno, di dietro bisognerà tirare a mitraglia o metter dei cosacchi con le fruste - disse il principe. 

- Ma questo è uno scherzo, e uno scherzo poco buono, scusatemi principe - disse Sergej Ivanovic. 

- Io non vedo come questo sia uno scherzo, che\ldots{} - voleva cominciare Levin, ma Sergej Ivanovic lo interruppe. 

- Ogni membro della società è chiamato a fare il lavoro che gli è proprio - disse egli. - Anche gli uomini di pensiero compiono il loro lavoro esprimendo l'opinione pubblica. E l'unanime e piena espressione dell'opinione pubblica è un merito della stampa e nello stesso tempo un fenomeno che rallegra. Vent'anni fa avremmo taciuto, e ora si sente la voce del popolo russo, che è pronto a levarsi come un solo uomo, ed è pronto a sacrificarsi per i fratelli oppressi; è un grande passo e un pegno di forza. 

- Ma non si tratta mica soltanto di far dei sacrifici, ma anche di uccidere i turchi - disse timido Levin. - Il popolo fa sacrifici ed è pronto a far sacrifici per la propria anima, e non per uccidere - egli soggiunse, collegando involontariamente la conversazione con le idee che lo occupavano tanto. 

- Come per l'anima? Per un naturalista, capirete, è una espressione imbarazzante. Che cos'è l'anima? - disse sorridendo Katavasov. 

- Ah, lo sapete! 

- Ecco, com'è vero Dio, non ne ho la più pallida idea! - disse con una forte risata Katavasov. 

- ``Io non sono venuto a metter pace, ma guerra'' dice Cristo - obiettò da parte sua Sergej Ivanyc, riportando semplicemente, come fosse la cosa più comprensibile, quello stesso passo del Vangelo che più di tutti tormentava Levin. 

- È proprio così - ripeté di nuovo il vecchio, che stava proprio accanto a loro, rispondendo a un'occhiata gettata per caso su di lui. 

- No, amico mio, siete battuto, battuto, completamente battuto - gridò allegramente Katavasov. 

Levin arrossì di stizza, non perché era stato battuto, ma perché s'era intrattenuto e s'era messo a discutere. 

``No, io non posso discutere con loro - pensava - loro hanno addosso una corazza impenetrabile, e io sono nudo''. 

Vedeva che convincere il fratello e Katavasov non si poteva, e ancor meno vedeva la possibilità di consentire con loro. Ciò che essi predicavano era quella stessa superbia d'intelletto che lo aveva quasi rovinato. Non poteva consentire che decine di persone, nel cui numero era anche suo fratello, avessero il diritto, in base a quello che dicevano loro le centinaia di ciarlatani volontari che giungevano nelle capitali, di affermare che essi, con i giornali, esprimevano la volontà e il pensiero del popolo, e un pensiero che s'esprimeva nella vendetta e nell'uccisione. Non poteva consentire con questo, perché non vedeva l'espressione di questi pensieri nel popolo in mezzo al quale viveva, e non trovava questi pensieri in se stesso (ed egli non poteva considerarsi nient'altro che una delle persone componenti il popolo russo), e soprattutto perché insieme col popolo non sapeva, non poteva sapere in che cosa consistesse il bene comune, ma sapeva con certezza che il raggiungimento di questo bene comune era possibile soltanto con un severo adempimento di quella legge di bontà che si rivela a ogni uomo, e perciò non poteva desiderare la guerra e predicare in favore di scopi comuni, quali che essi fossero. Egli diceva, insieme con Michajlyc e col popolo, che aveva espresso il proprio pensiero nella leggenda della chiamata dei Variaghi: ``Regnate e siate i nostri signori. Noi promettiamo con gioia una completa sottomissione. Tutto il lavoro, tutte le umiliazioni, tutti i sacrifici ce li assumiamo noi; ma che non siamo noi a giudicare e decidere''. E ora il popolo, secondo le parole di Sergej Ivanyc, rinunciava a tale diritto, comprato a così caro prezzo. 

Desiderava ancora dire che, se l'opinione pubblica era giudice infallibile, allora perché la rivoluzione, la Comune non erano altrettanto legittime come il movimento a favore degli slavi? Ma tutti questi erano pensieri che non potevano decidere nulla. Una sola cosa si poteva indubbiamente vedere: era che nel momento presente la discussione irritava Sergej Ivanovic e perciò discutere era male; e Levin tacque e rivolse l'attenzione degli ospiti sul fatto che le nuvolette s'erano addensate e che, per sfuggire alla pioggia, era meglio tornarsene a casa. 

\capitolo{XVII}\label{xvii-7} 

Il principe e Sergej Ivanyc salirono sul calesse e andarono via; gli altri affrettarono il passo, ritornarono a casa a piedi. 

Ma la nuvola ora sbiancandosi, ora oscurandosi, avanzava così rapida che bisognava accelerare ancora di velocità per giungere in tempo a casa prima della pioggia. Le nuvole che avanzavano, basse e nere, come fumo misto a fuliggine, correvano per il cielo con straordinaria rapidità. Fino a casa c'erano ancora duecento passi, e s'era già alzato il vento e, da un secondo all'altro, ci si poteva aspettare un acquazzone. 

I bambini correvano avanti con uno stridio spaventato e gioioso. Dar'ja Aleksandrovna, lottando con fatica con le sottane che le si erano come incollate alle gambe, non camminava più, ma correva, senza togliere gli occhi di dosso ai fanciulli. Gli uomini, trattenendo il cappello, camminavano a passi lunghi. Erano già proprio alla scalinata, quando una grossa goccia batté e si infranse contro l'estremità della grondaia di ferro. I bambini, e dietro loro i grandi, corsero con allegro vocìo sotto la protezione del tetto. 

- Katerina Aleksandrovna? - domandò Levin ad Agaf'ja Michajlovna che li aveva accolti nell'anticamera con fazzoletti e scialli. 

- Pensavamo fosse con voi - ella disse. 

- E Mitja? 

- Al Kolok, forse, e la njanja è con lui. 

Levin afferrò gli scialli e corse via al Kolok. 

In quel breve spazio di tempo la nuvola s'era già tanto avanzata col suo centro sul sole, che s'era fatto buio come in una eclissi. Il vento pareva insistere ostinatamente, fermava Levin e, strappando le foglie e i fiori dai tigli e rendendo spogli, informi e strani i rami bianchi delle betulle, piegava tutto da un sol lato: le acacie, i fiori, la bardana, l'erba e le cime degli alberi. Le ragazze, che lavoravano in giardino, corsero strillando sotto il tetto della camera della servitù. La cortina bianca della pioggia dirotta aveva già invaso tutta la selva lontana e metà dei campi vicini, e si avanzava rapida verso il Kolok. L'umidità della pioggia, frantumata in gocce minute, si sentiva nell'aria. 

Piegando la testa in avanti e lottando col vento che gli strappava gli scialli, Levin giungeva di corsa al Kolok e vedeva già qualcosa di bianco dietro una quercia, quando, a un tratto, tutto si infiammò, tutta la terra prese fuoco e fu come se, sopra il capo, gli si fosse spaccata la volta celeste. Aperti gli occhi abbacinati, Levin, attraverso lo spesso velo della pioggia, che adesso lo separava dal Kolok, vide con orrore prima di tutto la cima verde della quercia, a lui nota, in mezzo al bosco, che aveva mutato stranamente la sua posizione. ``Possibile che l'abbia spezzata?'' fece appena in tempo a pensare Levin, accelerando sempre più il passo, quando la cima della quercia si nascose dietro alle altre piante, ed egli sentì lo schianto del grande albero caduto sugli altri. 

La luce del fulmine, il fragore del tuono e la sensazione del corpo istantaneamente percosso dal freddo si fusero per Levin in una sola impressione d'orrore. 

- Dio mio! Dio mio! che non sia su di loro! - esclamò. 

E sebbene pensasse subito come fosse insensata la propria implorazione perché i suoi non fossero uccisi dalla quercia caduta in quel momento, la ripeté, sapendo che non poteva far nulla di meglio di quella preghiera insensata. 

Giunto di corsa al posto dove erano di solito, non li trovò. Erano all'altra estremità del bosco, sotto un vecchio tiglio, e lo chiamavano. Due figure, vestite di scuro (prima erano in chiaro), stavano curve sopra qualcosa. Erano Kitty e la njanja. La pioggia diminuiva e cominciava già a schiarire, quando Levin giunse correndo presso di loro. La njanja aveva la parte inferiore del vestito asciutto, ma addosso a Kitty il vestito s'era bagnato da parte a parte, e si era incollato tutto al corpo. Benché non piovesse più, esse rimanevano sempre nella stessa posizione in cui s'erano messe quando s'era scatenato il temporale: tutte e due stavano in piedi, chine sopra la carrozzina con un ombrellino verde. 

- Vivi? intatti? Sia lodato Iddio! - esclamò Levin, sguazzando nell'acqua rimastagli in una scarpa che gli sfuggiva, piena d'acqua, e accorrendo presso di loro. 

Il viso arrossato e bagnato di Kitty era rivolto a lui e sorrideva timido sotto il cappello che aveva cambiato foggia. 

- Ma come non ti vergogni! Non capisco come si possa essere così imprudenti! - egli assalì con stizza la moglie. 

- Com'è vero Dio, non ne ho colpa. Volevo proprio allora andarmene, quando lui ci ha fatto perder tempo. Bisognava cambiarlo. Aveva appena\ldots{} - cominciò a scusarsi Kitty. 

Mitja era incolume, asciutto e non aveva cessato di dormire. 

- Via, sia lodato Iddio! non so quello che dico. 

Raccolsero le fasce bagnate, la njanja tirò fuori il bambino e lo portò in braccio. Levin camminava accanto alla moglie, stringendole la mano, di nascosto alla njanja, con aria colpevole per la propria stizza. 

\capitolo{XVIII}\label{xviii-7} 

Durante tutto il giorno, nelle conversazioni più svariate alle quali sembrava partecipare soltanto con la parte esteriore del proprio intelletto, Levin, malgrado la delusione per il mutamento che doveva avvenire in lui, non cessava di sentire con gioia la pienezza del proprio cuore. 

Dopo la pioggia era troppo umido per andare a passeggio; inoltre anche le nuvole temporalesche non sparivano dall'orizzonte e ora là, ora qua passavano, tonando e facendosi scure all'estremità del cielo. Tutta la compagnia trascorse il resto della giornata in casa. 

Discussioni non se ne intavolarono più, e, al contrario, dopo pranzo tutti furono nella migliore disposizione d'animo. 

Katavasov, da principio, fece ridere le signore coi suoi scherzi originali, che piacevano sempre tanto appena lo si conosceva; ma poi, invitato da Sergej Ivanovic, raccontò le sue osservazioni molto interessanti sulla differenza di carattere e perfino di fisionomia tra femmine e maschi delle mosche domestiche, e sulla loro vita. Anche Sergej Ivanovic fu allegro, e al tè, invitato dal fratello, espose il suo punto di vista sull'avvenire della questione orientale, e così semplicemente e bene che tutti l'ascoltarono con piacere. 

Solamente Kitty non poté ascoltarlo fino in fondo: la chiamarono a fare il bagno a Mitja. 

Di lì a poco, dopo Kitty, chiamarono anche Levin nella camera del bambino. 

Lasciato il suo tè e rammaricandosi anche lui di interrompere una conversazione interessante, e nello stesso tempo inquieto perché lo chiamavano (giacché questo accadeva in casi rari), Levin andò nella camera del bambino. 

Sebbene le idee non finite di ascoltare di Sergej Ivanovic, su come il mondo liberato dagli slavi e forte di quaranta milioni di uomini dovesse, con la Russia, incominciare una nuova epoca nella storia, lo interessassero molto, come qualcosa di completamente nuovo per lui, e sebbene l'inquietassero la curiosità e l'ansia per la chiamata insolita, pure, quando si trovò solo, fuori dal salotto, egli ricordò immediatamente i suoi pensieri del mattino. E tutte quelle considerazioni sull'importanza dell'elemento slavo nella storia mondiale gli apparvero così insignificanti in confronto a ciò che accadeva nell'anima sua, che dimenticò subito tutto quanto, e si riportò al medesimo stato d'animo che aveva avuto la mattina. 

Adesso non ricordava, come gli accadeva prima, tutta la successione del pensiero (non ne aveva bisogno). Si riportò immediatamente al sentimento che l'aveva guidato, che era collegato a quei pensieri, e trovò nell'anima sua questo sentimento ancora più forte e definito di prima. Adesso non gli accadeva quello che accadeva nei tentativi di calmarsi che si inventava prima, quando bisognava ricostruire tutta la successione del pensiero per trovare il sentimento. Adesso, al contrario, il senso di gioia e di serenità era più vivo di prima e il pensiero non poteva tener dietro al sentimento. 

Camminava su e giù per la terrazza e guardava due stelle che apparivano nel cielo già fattosi scuro e, a un tratto, ricordò: ``Sì, guardando il cielo, pensavo che la volta che vedevo non era un'illusione\ldots{} però, non ho finito di pensare a qualcosa che devo aver nascosto a me stesso. Ma, qualunque cosa sia, non potranno esservi obiezioni. Basterà pensare un po' e tutto si chiarirà!''. 

Mentre stava per entrare nella camera del bambino, ricordò quello ch'egli aveva nascosto a se stesso. Se la dimostrazione principale della Divinità era la Sua rivelazione di quello che è bene, perché allora questa rivelazione si limitava alla sola Chiesa cristiana? Che rapporti avevano con questa rivelazione le credenze dei buddisti, dei maomettani, che anch'essi professavano e operavano il bene? 

Gli sembrava d'aver una risposta a questa domanda; ma non fece in tempo a esprimerla a se stesso, che era già entrato nella camera del bambino. 

Kitty era in piedi, con le maniche rimboccate, accanto alla vasca da bagno, curva sul bambino che veniva lavato e, sentiti i passi del marito, voltò il viso a lui, e col sorriso lo chiamò a sé. Con una mano sosteneva sotto il capo il bambino paffuto che annaspava sulla schiena e divaricava le gambette; con l'altra, tendendo il muscolo, premeva su di lui la spugna. 

- Su, ecco, guarda, guarda! - ella disse quando il marito le fu accanto. - Agaf'ja Michajlovna ha ragione: riconosce. 

Si trattava del fatto che quel giorno Mitja evidentemente, senza più alcun dubbio, riconosceva tutti i suoi. 

Non appena Levin si fu avvicinato alla vasca da bagno, gli fu offerto subito un esperimento, e l'esperimento riuscì in pieno. La cuoca, chiamata apposta per questo, si chinò sopra il bambino. Egli aggrottò le sopracciglia e scosse il capo negativamente. Si chinò Kitty sopra di lui, egli s'illuminò d'un sorriso, si appoggiò con le mani alla spugna e fece un suono con le labbra così soddisfatto e strano, che non solo Kitty e la njanja, ma anche Levin venne preso da un improvviso entusiasmo. 

Tirarono fuori il bambino dal bagno su una mano sola, gli versarono dell'acqua addosso, lo involtolarono in un lenzuolo, lo asciugarono e, dopo un acuto gridio, lo tesero alla madre. 

- Be', sono contenta che cominci a volergli bene - disse Kitty al marito, dopo che, col bambino al petto, si fu messa tranquillamente a sedere, al posto abituale. - Sono molto contenta. Se no questo cominciava già ad addolorarmi. Tu dicevi che non sentivi niente per lui. 

- No, dicevo forse di non sentire? Dicevo soltanto che ero deluso. 

- Come, di che deluso? 

- Non che fossi deluso di lui, ma del mio sentimento; m'aspettavo di più. Mi aspettavo che, come una sorpresa, sarebbe sbocciato in me un nuovo, piacevole sentimento. E a un tratto, invece di questo, ripugnanza, compassione\ldots{} 

Ella lo ascoltava attenta, curva sul bimbo, infilando nelle dita sottili gli anelli che aveva tolto per fare il bagno a Mitja. 

- E soprattutto, c'è tanta più apprensione e tanta più pena che non piacere. Oggi, dopo quello spavento, durante il temporale, ho capito come gli voglio bene. 

Kitty si illuminò d'un sorriso. 

- Ti sei spaventato molto? - ella disse. - Anch'io, ma ne sento più spavento ora che è passato. Andrò a vedere la quercia. E come è simpatico Katavasov! Ma, in generale, tutto il giorno è stato così piacevole. E tu con Sergej Ivanyc sei così caro, quando vuoi\ldots{} Su, va' da loro, dopo il bagno, qui, c'è sempre caldo e vapore. 

\capitolo{XIX}\label{xix-7} 

Uscito dalla camera del bambino e rimasto solo, Levin ricordò subito il pensiero in cui c'era qualcosa di poco chiaro. 

Invece di andare nel salotto, dal quale giungevano le voci, si fermò sulla terrazza e, appoggiatosi coi gomiti alla balaustrata, si mise a guardare il cielo. 

S'era già fatto completamente buio, e a sud, dove egli guardava, non c'erano nuvole. Le nuvole erano dalla parte opposta. Di là s'accendeva un lampeggio e si sentiva un rimbombo lontano. Levin prestava ascolto alle gocce che cadevano eguali dai tigli in giardino e guardava il triangolo di stelle a lui noto e la Via Lattea con la ramificazione che l'attraversava. A ogni guizzo di lampo non solo la Via Lattea, ma anche le stelle lucenti sparivano; non appena il lampo si spegneva, di nuovo, come gettate da una mano precisa, apparivano tutte negli stessi punti. 

``Ebbene, cos'è che mi turba?'' si disse Levin, sentendo in precedenza che la soluzione del dubbio, pur senza saperlo, era già pronta nell'animo suo. 

``Sì, l'unica evidente, indubitabile manifestazione della Divinità è la legge del bene, che è manifestata al mondo dalla rivelazione e che io sento in me, e nel riconoscimento di questa non è che mi unisca, ma, volere o no, sono unito con gli altri uomini in una sola società di credenti che si chiama la Chiesa. Già, e gli ebrei, i maomettani, i confucianisti, i buddisti, che cosa sono mai? - e si pose quella domanda che gli sembrava così pericolosa. - Possibile che queste centinaia di milioni di uomini siano privati di quel bene migliore, senza il quale la vita non ha senso?''. Si fece pensieroso, ma immediatamente si corresse. ``Ma cosa mai mi chiedo? - si disse. - Chiedo il rapporto che hanno con la Divinità le più svariate credenze dell'umanità intera. Domando della comune manifestazione di Dio per tutto l'universo con tutte queste nebulose. E che faccio? A me personalmente, al mio cuore è aperta una conoscenza indubitabile, irraggiungibile con la ragione, e io ostinatamente voglio esprimere con la ragione e a parole questa conoscenza''. 

``Non so forse che le stelle non camminano?'' egli domandò interrogando un vivido pianeta che aveva già cambiata la propria posizione passando al ramo superiore d'una betulla. 

``Ma io, guardando il moto delle stelle, non posso immaginarmi la rotazione della terra, e ho ragione di dire che le stelle camminano. 

E gli astronomi potrebbero forse capire e calcolare qualcosa, se prendessero in considerazione tutti i complessi e svariati movimenti della terra? Tutte le loro meravigliose conclusioni sulle distanze, sul peso, sui moti e le rivoluzioni dei corpi celesti sono basate soltanto sul movimento apparente degli astri intorno alla terra immobile, su quello stesso moto che adesso è dinanzi a me e che è stato così per milioni di persone durante secoli ed è stato e sarà sempre eguale e potrà essere sempre controllato. E proprio così come sarebbero oziose e incerte le conclusioni degli astronomi non basate sulle osservazioni del cielo visibile, in rapporto con un meridiano e un orizzonte, così sarebbero oziose e incerte anche le mie conclusioni non basate su quella comprensione del bene, che è stata e sarà sempre eguale per tutti e che mi viene dischiusa dal cristianesimo e può essere sempre controllata nell'anima mia. La questione poi delle altre credenze e dei loro rapporti con la Divinità, non ho io il diritto e la possibilità di risolverla''. 

- Ah, non te ne sei andato? - disse a un tratto la voce di Kitty, che tornava in salotto. - Che hai, sei agitato da qualcosa? - ella disse, osservandogli attentamente il viso al chiarore delle stelle. 

Tuttavia ella non avrebbe scorto bene il viso di lui se di nuovo un lampo, che nascose le stelle, non lo avesse illuminato. Alla luce del lampo ella guardò bene tutto il suo viso e, avendo visto ch'egli era calmo e gioioso, gli sorrise. 

``Lei capisce - egli pensava - sa a che cosa penso. Devo dirglielo o no? Sì, glielo dirò''. Ma nel momento in cui egli voleva cominciare a parlare prese a parlare anche lei. 

- Ecco, Kostja, fammi un piacere - ella disse - va' nella stanza d'angolo e guarda come hanno accomodato tutto per Sergej Ivanovic. Che ci vada io non sta bene. L'hanno messo il lavabo nuovo? 

- Va bene, ci andrò senz'altro - disse Levin, alzandosi e baciandola. 

``No, non bisogna parlare - egli pensò, quand'ella gli passò avanti. - È un segreto necessario, importante per me solo e inesprimibile a parole. 

Questo nuovo sentimento non mi ha cambiato, non mi ha reso felice, non mi ha rischiarato a un tratto, come sognavo, proprio come il sentimento per mio figlio. Anche qui non c'è stata nessuna sorpresa. E fede o non fede, non so cosa sia, ma questo sentimento è entrato in me egualmente inavvertito, attraverso la sofferenza, e si è fermato saldamente nell'anima. 

Mi arrabbierò sempre alla stessa maniera contro Ivan il cocchiere, sempre alla stessa maniera discuterò, esprimerò a sproposito le mie idee, ci sarà lo stesso muro fra il tempio dell'anima mia e quello degli altri, e perfino mia moglie accuserò sempre alla stessa maniera del mio spavento e ne proverò rimorso; sempre alla stessa maniera, non capirò con la ragione perché prego e intanto pregherò, ma la mia vita adesso, tutta la mia vita, indipendentemente da tutto quello che mi può accadere, ogni suo attimo, non solo non è più senza senso, come prima, ma ha un indubitabile senso di bene, che io ho il potere di trasfondere in essa!''. 