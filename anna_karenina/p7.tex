\parte{PARTE SETTIMA}\label{parte-settima} 
\pagestyle{pagina}

\capitolo{I}\label{i-6} 

Era il terzo mese che i Levin erano a Mosca. Era già passato da tempo il termine, in cui, secondo i calcoli più sicuri delle persone che conoscono tali cose, Kitty doveva partorire; e lei era tuttora in stato interessante, e da nulla si vedeva che il tempo si era fatto più vicino adesso che non due mesi prima. E il dottore e la levatrice e Dolly e la madre, e in particolare Levin, che non poteva pensare senza agitarsi a quello che si avvicinava, cominciavano a essere impazienti e inquieti; soltanto Kitty si sentiva pienamente tranquilla e serena. 

Adesso, aveva chiara la coscienza ch'era sorto in lei un sentimento nuovo per la creatura che sarebbe nata, già in parte presente per lei, e prestava ascolto con gioia a questo sentimento. Adesso non era più semplicemente una parte di lei, ma viveva a volte anche di vita sua propria, indipendente da lei. Spesso le accadeva di sentirsi male per questo, ma nello stesso tempo aveva voglia di ridere per una strana nuova gioia. 

Tutti quelli che ella amava erano con lei, e tutti erano così buoni, la curavano tanto, le era offerto di ogni cosa solo il lato piacevole, così che, se non avesse saputo e sentito che ciò sarebbe finito presto, non avrebbe neppur desiderato una vita migliore e più piacevole. Soltanto una cosa le sciupava l'incanto di questa vita, ed era il fatto che suo marito non era così com'ella lo amava e come era stato in campagna. 

Le piaceva il tono calmo, carezzevole, e ospitale ch'egli aveva avuto in campagna. In città, invece, sembrava continuamente inquieto e guardingo, come se temesse che qualcuno non offendesse lui, ma soprattutto lei. Là in campagna, egli evidentemente sapendosi al suo posto, non si affannava e non era mai inoperoso. Qui, in città, si affannava continuamente, come se non volesse lasciarsi sfuggire nulla, eppure non aveva nulla da fare. E a lei faceva pena. Agli altri, lo sapeva, non appariva pietoso; al contrario, quando Kitty lo guardava in società, come si guarda a volte la persona che si ama, cercando di giudicarla da estranei, per definire l'impressione che produce sugli altri, ella vedeva, quasi con apprensione per la propria gelosia, che egli non solo non era pietoso, ma era molto attraente con la sua onestà, con la sua cortesia un po' all'antica, timida con le signore, con la sua figura forte e il volto particolarmente espressivo. Ma ella l'osservava non dal di fuori, ma nell'intimo; vedeva che non era spontaneo; non altrimenti poteva definire lo stato di lui. A volte lo accusava fra sé e sé di non saper vivere in città; a volte, invece, riconosceva che realmente gli era difficile organizzarsi una vita in modo da esserne soddisfatto. 

E che cosa doveva fare, infatti? Giocare a carte non gli piaceva. Al club non andava. Frequentare dei buontemponi sul genere di Oblonskij, ella sapeva ormai che cosa significasse\ldots{} significava bere, e, dopo aver bevuto, andare chi sa dove. Non poteva pensare senza orrore dove andavano gli uomini in simili casi. Frequentare il gran mondo? Ma ella sapeva che per questo bisognava trovar piacere nell'avvicinare donne giovani, e lei questo non lo poteva desiderare. Rimanere a casa con lei, con la madre e le sorelle? Ma per quanto piacevoli e allegri fossero per lei gli stessi discorsi, le ``Aline-Nadine'' come il vecchio principe chiamava questi discorsi fra le sorelle, sapeva bene che questo lo doveva annoiare. Allora che cosa gli rimaneva da fare? Continuare a scrivere il suo libro? Aveva anche tentato di farlo, e da principio andava in biblioteca a consultare lavori monografici e d'informazione per il suo libro; ma, come egli diceva, quanto più non faceva nulla, tanto meno tempo gli rimaneva. Inoltre egli stesso si lamentava che aveva finito col discorrere troppo del proprio libro e che perciò tutte le idee gli si erano confuse e avevano perso interesse. 

L'unico vantaggio di questa vita cittadina era che, in città, litigi fra di loro non ne avvenivano. Forse perché le condizioni di vita cittadina erano diverse, o perché tutti e due s'erano fatti più accorti e più ragionevoli a questo riguardo, certo è che a Mosca non ebbero le questioni di gelosia che tanto avevano temuto, andando in città. 

A proposito di questo, accadde un avvenimento molto importante per tutti e due, l'incontro cioè di Kitty con Vronskij. 

La vecchia principessa Mar'ja Borisovna, madrina di Kitty, che sempre le aveva voluto molto bene, desiderò assolutamente di vederla. Kitty, che per il suo stato non andava in nessun posto, andò col padre dalla veneranda signora, e da lei incontrò Vronskij. 

In quell'incontro, Kitty poté soltanto rimproverarsi che, per un attimo, quando riconobbe nell'abito borghese i tratti a lei un tempo così noti, le venne meno il respiro, il sangue le affluì al cuore, e un colorito vivace, lo sentì, le apparve sul viso. Ma questo durò soltanto alcuni attimi. Il padre, che apposta aveva cominciato a parlare ad alta voce con Vronskij, non aveva ancora finito la sua conversazione, che lei era già del tutto pronta a guardar Vronskij, a parlare con lui, se ce n'era bisogno, proprio così come parlava con la principessa Mar'ja Borisovna, e, soprattutto, in modo che ogni cosa, fino all'ultima intonazione e all'ultimo sorriso, fosse approvata dal marito, la cui presenza invisibile era come sentita da lei su di sé, in quel momento. 

Disse alcune parole con lui, sorrise perfino tranquilla all'arguzia sulle elezioni, che egli chiamava ``il nostro parlamento''. (Bisognava sorridere per mostrare che se n'era capito lo spirito). Ma subito si voltò verso Mar'ja Borisovna e non lo guardò neanche più una volta, finché egli non si alzò per salutare; allora lo guardò, ma evidentemente solo perché è scortese non guardare una persona quando saluta. 

Fu riconoscente al padre che non le disse nulla sull'incontro con Vronskij; ma vedeva dalla particolare tenerezza di lui dopo la visita, durante la solita passeggiata, ch'egli era contento di lei. Lei stessa era contenta. Non si aspettava in nessun modo di trovare in sé la forza di trattenere chi sa dove, nel profondo dell'anima, tutti i ricordi del sentimento provato un tempo per Vronskij e non solo di apparire, ma anche di essere del tutto indifferente e calma verso di lui. 

Levin arrossì molto più di lei, quand'ella gli disse d'aver incontrato Vronskij dalla principessa Mar'ja Borisovna. Le fu molto difficile dirglielo, ma ancor più difficile continuare a parlare dei particolari dell'incontro, poiché egli non la interrogava, ma la guardava soltanto accigliato. 

- Mi dispiace molto, che tu non ci sia stato - ella disse. - Non che tu non ci fossi nella stanza\ldots{} non sarei stata così naturale in tua presenza. Io adesso arrossisco molto di più, molto, molto di più - diceva lei, arrossendo fino alle lacrime. - Ma che tu non potessi vedere attraverso una fessura. 

Gli occhi sinceri dissero a Levin ch'ella era contenta di sé, e lui, malgrado ella arrossisse, si calmò immediatamente e cominciò a farle delle domande, il che era proprio quello che lei voleva. Quando egli seppe tutto, fino al particolare che solo al primo momento non aveva potuto non arrossire, ma che poi s'era comportata in modo semplice e spontaneo come col primo venuto, Levin si rallegrò completamente e disse che ne era molto contento e che adesso non avrebbe agito in modo così sciocco come alle elezioni, ma che avrebbe cercato, al primo incontro con Vronskij, di essere cordiale, per quanto possibile. 

- È un tale tormento pensare che c'è un essere quasi nemico col quale è penoso incontrarsi - disse Levin. - Sono molto, molto contento. 

\capitolo{II}\label{ii-6} 

- Su via, ti prego, passa dai Bol' - disse Kitty al marito, quando egli, alle undici, prima di uscire di casa, andò da lei. - So che pranzi al club; papà ti ha iscritto. E stamattina che fai? 

- Vado soltanto da Katavasov - rispose Levin. 

- Come mai così presto? 

- Mi ha promesso di farmi conoscere Metrov. Desideravo parlare con lui del mio lavoro, è un noto studioso di Pietroburgo - disse Levin. 

- Sì, è suo quell'articolo che hai lodato tanto? Be', e poi? - disse Kitty. 

- Può darsi che vada ancora in tribunale per l'affare di mia sorella. 

- E al concerto? - ella domandò. 

- Ma perché devo andare solo? 

- No, vacci: là dànno cose nuove\ldots{} Ti interessava tanto. Io ci andrei assolutamente. 

- Be', in ogni modo passerò a casa prima di pranzo - egli disse, guardando l'orologio. 

- Mettiti la finanziera per poter andare direttamente dalla contessa Bol'. 

- È proprio necessario? 

- Ah, assolutamente! Lui è stato da noi. Via, che ti costa? Arrivi, ti siedi, parli per cinque minuti, ti alzi e te ne vai. 

- Non ci crederai, ma sono talmente disabituato a questo, che quasi me ne vergogno. Cos'è questo? Viene una persona estranea, si siede, rimane lì a sedere senza far niente, dà loro noia, si disorienta e se ne va. 

Kitty rise. 

- Ma facevi pur delle visite, da scapolo! - disse. 

- Le facevo, ma mi vergognavo sempre, e ora sono così disabituato che, lo sa Iddio, meglio non pranzare per due giorni anzi che far questa visita. Mi vergogno tanto! Mi pare sempre debbano offendersi, che debbano dire: perché sei venuto senza una ragione seria? 

- No, non si offenderanno. Per questo ne rispondo io - disse Kitty, mentre guardava ridendo il viso di lui. Lo prese per una mano. - Su, addio\ldots{} Va', per favore. 

Stava già per uscire, dopo aver baciato la mano alla moglie, quando ella lo fermò. 

- Kostja, lo sai che mi son rimasti soltanto cinquanta rubli? 

- Va bene, passerò a prenderne in banca. Quanto? - egli disse con un'espressione di scontento a lei nota. 

- No, aspetta. - Ella lo trattenne per mano. - Parliamo un po', ciò mi preoccupa. Io, mi pare, non pago nulla più caro di quel che dovrei, e i denari se ne vanno a fiumi. C'è qualcosa che non va. 

- Per nulla - disse lui, tossendo e guardandola di sotto in su. 

Questo tossicchiare ella lo conosceva. Era un segno di forte scontento, non verso di lei, ma verso se stesso. Realmente egli era contrariato, non che se ne andassero molti denari, ma che gli si ricordasse quello che lui, sapendo che v'era qualcosa che non andava, voleva dimenticare. 

- Ho ordinato a Sokolov di vendere il frumento e di prendere il denaro in anticipo per il mulino. I denari ci saranno in ogni modo. 

- No, ma io ho paura che in generale, sì, molto\ldots{} 

- Per nulla, per nulla - egli ripeteva. - Be', addio, cara. 

- No, davvero, a volte, rimpiango di avere ascoltato la mamma. Come sarebbe stato bello in campagna! Invece vi ho tormentato tutti e sperperiamo denaro\ldots{} 

- Per nulla, per nulla. Non è avvenuto neppure una volta, da che sono sposato, che io abbia detto che sarebbe stato meglio altrimenti di quello che non sia\ldots{} 

- Davvero? - disse lei, guardandolo negli occhi. 

Egli l'aveva detto senza pensare, solo per consolarla. Ma quando, guardandola, vide che quei sinceri, cari occhi lo fissavano interrogativamente, ripeté la stessa cosa con tutta l'anima. ``Decisamente la trascuro'' pensò. E ricordò quello che li attendeva tra poco. 

- Sarà presto? Come ti senti? - sussurrò lui, prendendola per tutt'e due le mani. 

- Ci ho pensato tante volte, che ora non ci penso più e non so nulla. 

- E non hai paura? 

Ella sorrise con disprezzo. 

- Neppure un briciolo - ella disse. 

- Allora se succede qualcosa, sono da Katavasov. 

- No, non succederà niente, non ci pensare. Io andrò a passeggio con papà sul viale. Passeremo da Dolly. Prima di pranzo ti aspetto. Ah, già, lo sai che la situazione di Dolly diventa proprio impossibile? È piena di debiti, denari non ne ha. Ieri abbiamo parlato, mamma ed io, con Arsenij - così ella chiamava il marito della sorella L'vova - e abbiamo deciso di lanciare te e lui contro Stiva. È proprio una cosa impossibile. Con papà non se ne può parlare\ldots{} Ma se tu e lui\ldots{} 

- Ma che possiamo fare mai? - disse Levin. 

- Comunque, va' da Arsenij, parla un po' con lui; ti dirà quello che abbiamo deciso. 

- Ma con Arsenij sono fin d'ora d'accordo su tutto. Allora passerò da casa sua. A proposito, se devo andare al concerto, ci andrò con Natalie. Be', addio. 

Sulle scale il vecchio servo Kuz'ma, ch'egli aveva ancora da quando era scapolo, e che sorvegliava la casa di città, fermò Levin. 

- Krasavcik - era un cavallo, timoniere di sinistra, portato dalla campagna - è stato ferrato di nuovo, ma zoppica sempre - disse. - Cosa comandate? 

I primi tempi a Mosca, Levin si occupava dei cavalli portati dalla campagna. Desiderava organizzare questo servizio nel modo migliore e più conveniente; ma era successo che i cavalli propri venivano a costare più cari di quelli di fitto, e le vetture da nolo venivano prese ugualmente. 

- Ordina che si mandi a chiamare il veterinario, può darsi che sia un'ammaccatura. 

- E per Katerina Aleksandrovna? - domandò Kuz'ma. 

Ormai Levin non si stupiva più, come nei primi tempi della sua vita a Mosca, che per andare dalla Vozdvizen'ka al Sivcev Vrazëk bisognasse attaccare una pariglia di cavalli forti a una carrozza pesante, condurre questa carrozza per un quarto di versta su per il fango nevoso e star lì fermi quattro ore, pagando cinque rubli per questo. Questo ormai gli sembrava naturale. 

- Ordina al vetturino di portare una pariglia per la nostra carrozza - disse. 

- Sissignore. 

E risolta così, con semplicità e disinvoltura, grazie alle condizioni di vita cittadina, una difficoltà che in campagna avrebbe richiesto tanto lavoro personale e tanta cura, Levin uscì sulla scalinata e, chiamata una vettura da nolo, vi salì e si diresse alla Nikitskaja. Per strada non pensava già più ai denari, ma rifletteva su come avrebbe fatto conoscenza con lo studioso di Pietroburgo che si occupava di sociologia, e come avrebbe parlato con lui del proprio libro. 

Soltanto nei primissimi tempi a Mosca quelle spese inusitate per un abitatore della campagna, improduttive, ma inevitabili, che da ogni parte lo circuivano, stupirono Levin. Ma adesso vi si era già abituato. A questo riguardo gli accadde quello che, si dice, accade agli ubriaconi: il primo bicchierino va giù come un palo, il secondo come un falco e, dopo il terzo, gli altri volano via come uccelletti di nido. Quando Levin aveva cambiato il primo biglietto da cento rubli per comprare le livree al cameriere e al portiere, involontariamente aveva considerato che queste livree non erano necessarie a nessuno, ma erano inevitabilmente indispensabili, a giudicare dal modo con cui si erano sorprese la principessa e Kitty a un suo accenno che, senza livree, si poteva vivere lo stesso, che queste livree sarebbero costate come due operai estivi, cioè quasi trecento giorni lavorativi, dalla settimana di Pasqua fino all'ultimo giorno di carnevale, e ogni giorno di lavoro pesante, dalla mattina presto fino alla sera tardi; e questo biglietto da cento rubli gli era andato giù come un palo. Ma il seguente, cambiato per comprar cibarie che erano costate ventotto rubli, per un pranzo ai parenti, pur destando in Levin il pensiero che ventotto rubli erano nove stai di avena, che si falciavano, accovonavano, battevano, vagliavano, stacciavano e versavano, sudando e sbuffando, questo secondo biglietto, tuttavia, se ne era andato con maggiore facilità. E adesso i biglietti cambiati non suscitavano più da lungo tempo queste considerazioni e volavano via come uccelli di nido. La considerazione se al lavoro impiegato nel procurarsi il denaro avesse corrisposto il piacere che procurava ciò che veniva comprato con esso, era sfumata già da lungo tempo. Il calcolo economico che c'era un certo prezzo al di sotto del quale non si poteva vendere una certa qualità di grano, anch'esso era stato dimenticato. La segala, il cui prezzo egli aveva tenuto su per tanto tempo, era stata venduta a cinquanta copeche lo staio in meno di quello che si dava per essa un mese prima. Perfino il calcolo che con simili spese non sarebbe stato possibile vivere tutto l'anno senza debiti, anche questo calcolo non aveva più nessuna importanza. Si voleva soltanto una cosa: aver denari in banca senza domandare donde venissero, in modo da saper sempre come comprar carne l'indomani. E questo calcolo finora era stato mantenuto; egli aveva sempre avuto denari in banca, ma adesso erano stati spesi, ed egli non sapeva bene da qual parte prenderli. E questo, proprio quando Kitty gli aveva ricordato la questione dei denari, l'aveva sconvolto; ma non aveva avuto il tempo di pensarci su. Mentre andava in vettura, ripensava a Katavasov e all'imminente incontro con Metrov. 

\capitolo{III}\label{iii-6} 

Levin, in quella sua permanenza, era di nuovo in rapporti intimi con il suo compagno di università, il professore Katavasov, che non aveva più visto dal tempo del suo matrimonio. Katavasov gli piaceva per la chiarezza e la semplicità della sua concezione del mondo. Levin pensava che la chiarezza della concezione del mondo di Katavasov derivasse dalla povertà della sua natura; Katavasov invece pensava che la mancanza di coerenza nel pensiero di Levin derivasse dalla mancanza di disciplina del suo ingegno; ma la chiarezza di Katavasov piaceva a Levin e l'abbondanza di pensieri di Levin piaceva a Katavasov, ed essi amavano incontrarsi e discutere. 

Levin aveva letto a Katavasov alcuni punti della propria opera, e questi gli erano piaciuti. Il giorno avanti, incontrando Levin a una conferenza, Katavasov gli aveva detto che il famoso Metrov, il cui articolo era tanto piaciuto a Levin, si trovava a Mosca e si era molto interessato, a dire di Katavasov, al lavoro di Levin; l'indomani alle undici sarebbe andato da lui e sarebbe stato felice di conoscerlo. 

- State decisamente migliorando, amico mio, e ciò mi fa piacere - disse Katavasov, accogliendo Levin in un piccolo salotto. - Sento una scampanellata e penso: è impossibile che arrivi in orario\ldots{} Be', come sono i montenegrini? Di razza sono guerrieri. 

- E allora? 

Katavasov in poche parole gli riferì l'ultima notizia e, entrando nello studio, presentò Levin a un uomo non alto, tarchiato, di aspetto molto simpatico. Era Metrov. La conversazione si fermò per un poco sulla politica e su come a Pietroburgo, nelle alte sfere, si considerassero gli ultimi avvenimenti. Metrov riferì le parole dette a quel proposito dall'imperatore e da uno dei ministri, apprese da fonte sicura. Katavasov invece aveva sentito, pure da fonte sicura, che l'imperatore aveva detto tutt'altra cosa. Levin cercò di immaginare una situazione in cui potessero essere state dette e le une e le altre parole, ma la conversazione su questo argomentò cessò. 

- Sì, ecco che ha quasi finito di scrivere un libro sulle condizioni naturali del lavoratore in rapporto alla terra - disse Katavasov; - io non sono uno specialista, ma m'è piaciuto, come naturalista, ch'egli non consideri l'umanità qualcosa al di fuori delle leggi zoologiche, ma, al contrario, ne veda la dipendenza dall'ambiente e in questa dipendenza ricerchi le leggi dell'evoluzione. 

- È molto interessante - disse Metrov. 

- Io veramente avevo cominciato a scrivere un libro di economia rurale, ma senza volere, occupandomi dello strumento principale dell'economia rurale, del lavoratore - disse Levin, arrossendo - sono pervenuto a risultati del tutto inaspettati. 

E Levin cominciò a esporre prudentemente, quasi a tastare il terreno, il proprio punto di vista. Sapeva che Metrov aveva scritto un articolo contro la dottrina di economia politica generalmente accettata, ma fino a che punto poteva sperare di trovar simpatia presso di lui per i suoi nuovi criteri, non lo sapeva e non poteva indovinarlo dall'intelligente e calmo volto dello studioso. 

- Ma in che cosa vedete le peculiarità del lavoratore russo? - disse Metrov. - Nelle sue caratteristiche, per così dire, zoologiche o nelle condizioni in cui esso si trova? 

Levin vedeva che in questa domanda si rivelava già un'idea alla quale egli non accedeva; ma continuò a esporre il proprio pensiero, sostenendo che il popolo russo ha un modo di considerare la terra assolutamente particolare rispetto agli altri popoli. E, per dimostrare questa idea, si affrettò ad aggiungere che, secondo la sua opinione, questa visione del popolo russo derivava dalla consapevolezza della propria vocazione di popolare enormi spazi a oriente, non occupati. 

- È facile essere indotti in errore, facendo una conclusione sulla vocazione generale di un popolo - disse Metrov, interrompendo Levin. - Lo stato del lavoratore tuttavia dipenderà sempre dai suoi rapporti con la terra e il capitale. 

E, senza consentire a Levin di esporre a fondo il proprio pensiero, Metrov cominciò a esporgli la particolarità della propria dottrina. 

In che cosa consistesse la particolarità della sua dottrina, Levin non lo capì perché non si sforzava neppure di capirlo: vedeva che Metrov, alla stregua degli altri, sebbene nel proprio articolo avesse smentito la dottrina degli economisti, tuttavia guardava la situazione del lavoratore russo soltanto rispetto al capitale, al salario e alla rendita. Pur riconoscendo che nella parte orientale, la più grande della Russia, la rendita era ancora nulla, che il salario per i nove decimi della popolazione russa di ottanta milioni consisteva soltanto nel nutrire se stessi, e che il capitale non esisteva ancora che sotto l'aspetto di strumenti primitivi, tuttavia soltanto sotto questo aspetto considerava qualsiasi lavoratore, pur senza concordare in molte cose con gli economisti e avendo una sua nuova teoria sul salario, che espose a Levin. 

Levin ascoltava svogliato, e in principio fece delle obiezioni. Voleva interrompere Metrov per dire il proprio pensiero che, secondo lui, avrebbe resa superflua un'ulteriore esposizione. Ma poi, convintosi che consideravano la cosa in modo così diverso, che mai si sarebbero capiti, non contraddisse neppure più e ascoltò soltanto. Pur senza interessarsi affatto a quello che diceva Metrov, provava un certo gusto ad ascoltarlo. Il suo amor proprio era lusingato dal fatto che un uomo così colto gli esponesse le proprie idee così volentieri, con una tale premura e fiducia nella conoscenza della materia da parte di Levin, indicando a volte tutto un aspetto della cosa con una sola allusione. Egli attribuiva ciò al proprio merito, e non sapeva che Metrov, avendone parlato con tutti gli intimi, parlava particolarmente volentieri di questa materia con ogni persona nuova, e che in generale parlava volentieri con tutti della materia che l'occupava, ancora poco chiara a lui stesso. 

- Però arriveremo in ritardo - disse Katavasov, guardando l'orologio, appena Metrov ebbe finito la sua esposizione. 

- Già, oggi c'è una seduta alla Società degli amatori per il cinquantenario di Svintic - disse Katavasov a una domanda di Levin. - Io e Pëtr Ivanovic siamo disposti ad andare. Ho promesso di parlare dei suoi lavori sulla zoologia. Venite con noi, è molto interessante. 

- Sì, è ora - disse Metrov. - Venite con noi e di là, se volete, a casa mia. Desidererei parlare del vostro lavoro. 

- Ma no. È ancora così poco finito. Ma alla seduta sì, sono molto contento. 

- Ebbene, amico mio, avete sentito? Ho presentato una soluzione a parte - disse Katavasov che indossava il frac in un'altra stanza. 

E cominciò una conversazione sulla questione universitaria. 

La questione universitaria di quell'inverno, a Mosca, era molto importante. Tre vecchi professori, in consiglio, non avevano accettato la soluzione dei docenti giovani; i giovani avevano presentato una soluzione separata. Questa soluzione, secondo il giudizio degli uni, era orribile; secondo il giudizio degli altri, era la più semplice e giusta, e i professori si erano divisi in due partiti. Gli uni, a cui apparteneva Katavasov, vedevano nella parte contraria una denuncia vile e un inganno; gli altri, una ragazzata e una mancanza di rispetto verso le autorità. Levin, pur senza appartenere all'università, nella sua permanenza a Mosca aveva già sentito e parlato parecchie volte di questa faccenda e aveva una propria opinione a questo riguardo; prese quindi parte alla conversazione che continuò anche per la strada, finché tutti e tre giunsero alla vecchia università. 

La seduta era già cominciata. Intorno a una tavola coperta d'un panno, a cui sedettero Katavasov e Metrov, sedevano sei persone, e una di loro, curva su di un manoscritto, leggeva qualcosa. Levin sedette su di una delle sedie vuote che stavano intorno alla tavola, e domandò sottovoce a uno studente, ch'era là seduto, cosa leggessero. Lo studente, dopo aver squadrato Levin con aria scontenta, disse: 

- La biografia. 

Sebbene Levin non s'interessasse della biografia dello studioso, pure, involontariamente, prestò ascolto e venne a sapere qualcosa d'interessante e di nuovo sulla vita del famoso scienziato. 

Quando il lettore ebbe finito, il presidente lo ringraziò, lesse i versi del poeta Ment, mandatigli per quella celebrazione, e alcune parole di ringraziamento al poeta. Dopo, Katavasov, con voce forte, stridente, lesse la sua nota sui lavori scientifici di Svintic. Quando Katavasov ebbe finito, Levin guardò l'orologio, vide che erano già più dell'una, e pensò che non avrebbe fatto in tempo a leggere il proprio lavoro a Metrov prima del concerto, e adesso, ormai, non lo desiderava neppure più. Durante la lettura aveva pensato alla conversazione passata. Adesso per lui era chiaro che, ammessa una certa importanza alle idee di Metrov, anche le sue erano importanti; e queste idee potevano chiarirsi e portare a qualcosa solo quando ognuno avesse lavorato separatamente sulla via scelta, ma dalla comunicazione di queste idee non poteva venir fuori nulla. Decisosi quindi a declinare l'invito di Metrov, Levin, alla fine della seduta, gli si avvicinò. Metrov lo presentò al presidente, con cui parlava di novità politiche. Allora Metrov raccontò al presidente la stessa cosa che aveva raccontato a Levin, e Levin fece le stesse osservazioni che aveva già fatto quella mattina e, per variare, espresse una nuova opinione che gli era venuta in mente proprio in quel punto. Dopo si riprese la conversazione sulla questione universitaria. Poiché Levin aveva già sentito tutto questo, si affrettò a dire a Metrov che gli rincresceva di non poter profittare dell'invito, salutò e andò da L'vov. 

\capitolo{IV}\label{iv-6} 

L'vov, che aveva sposato Natalie, la sorella di Kitty, aveva passato tutta la sua vita nelle capitali e all'estero, dove si era formato e dove era stato come diplomatico. 

Da un anno aveva lasciato la carriera diplomatica, non per dispiaceri (non aveva mai avuto dispiaceri con nessuno), ed era passato a un impiego amministrativo della casa imperiale a Mosca, per dare un'educazione migliore ai suoi due ragazzi. 

Malgrado la grande diversità di abitudini e di opinioni e malgrado L'vov fosse più anziano di Levin, quell'inverno avevano stretto grande amicizia e avevano preso a volersi bene. 

L'vov era in casa, e Levin entrò da lui senza farsi annunciare. 

L'vov, in veste da camera con cintura e scarpe scamosciate, era seduto in una poltrona e, con un pince-nez dalle lenti turchine, leggeva un libro che stava su di un leggio, tenendo con accortezza, un po' discosto, con la bella mano un sigaro incenerito a metà. 

Il suo bel viso, fine e ancora giovane, al quale i capelli d'argento lucidi e inanellati davano un'espressione ancor più nobile, s'illuminò d'un sorriso quando scorse Levin. 

- Benissimo! E io volevo mandar da voi. Be', come va Kitty? Sedetevi qua: stiamo più tranquilli\ldots{} Avete letto l'ultima circolare nel ``Journal de St.~Pétersbourg''? Io penso che vada benissimo - egli disse con un accento un po' francese. 

Levin raccontò ciò che aveva sentito da Katavasov e da Metrov su quello che si diceva a Pietroburgo, e, dopo aver parlato un po' di politica, raccontò d'aver conosciuto Metrov e di essere andato alla seduta. Questo interessò molto L'vov. 

- Ecco, vi invidio il libero ingresso in quell'interessante mondo scientifico - egli disse e, preso a parlare, passò, come al solito, immediatamente al francese, per lui più comodo. 

- È vero che io non ne ho neppure il tempo. E il mio impegno e la cura dei ragazzi me ne privano; e poi, non mi vergogno di dire che la mia cultura è insufficiente. 

- Questo non lo credo - disse Levin con un sorriso, ammirando, come sempre, quella modesta opinione di sé, non assunta per il desiderio di apparire o anche di essere modesto, ma assolutamente sincera. 

- Eh, come! Lo sento, adesso, come sono poco istruito. Perfino per l'educazione dei ragazzi devo rinfrescare molte cose nella memoria o impararle del tutto. Perché non basta che ci siano dei maestri, bisogna che ci sia chi sorveglia, così come nella vostra azienda ci vogliono i lavoratori e un sorvegliante. Ecco, sto leggendo - egli indicò la grammatica di Buslaev, che stava sul leggio; - pretendono questo da Miša, ed è così difficile\ldots{} Su, ecco, spiegatemi. Qui egli dice\ldots{} 

Levin voleva spiegargli che non si poteva capire, ma che bisognava impararlo; ma L'vov non era d'accordo con lui. 

- Sì, ecco, voi ci ridete su! 

- Al contrario, non potete immaginare come, guardando voi, io impari sempre quello che per me è imminente, l'educazione dei bambini. 

- Ma da imparare non c'è nulla - disse L'vov. 

- Io so soltanto - disse Levin - che non ho veduto ragazzi più educati dei vostri, e non potrei desiderarne migliori dei vostri. 

L'vov, evidentemente, voleva contenersi, per non esprimere la propria gioia, ma s'illuminò tutto in un sorriso. 

- Basta che siano migliori di me. Ecco quello che desidero. Voi non sapete ancora tutta la fatica - egli cominciò - con dei ragazzi che, come i miei, sono stati trascurati per causa di questa vita all'estero. 

- Tutto questo lo riguadagnerete. Sono ragazzi di grande talento. La cosa più importante è l'educazione morale. Ecco ciò che imparo, guardando i vostri figli. 

- Voi dite: l'educazione morale. Non ci si può immaginare come sia difficile! Avete appena vinto un'inclinazione, che ne vengono fuori altre, e di nuovo bisogna lottare. Se non si ha un appoggio nella religione (ricordate che ne parlavamo con voi?), nessun padre, con le sole sue forze, potrebbe educare senza questo aiuto. 

Questa conversazione, che interessava sempre Levin, fu interrotta dalla bella Natalie Aleksandrovna ch'era entrata, già pronta per uscire. 

- E io non sapevo che foste qui - ella disse, evidentemente non solo non rimpiangendo, ma rallegrandosi di aver interrotto quella conversazione a lei nota da tempo e venutale a noia. - Ebbene, come va Kitty? Pranzo da voi, oggi. Ecco, Arsenij - ella disse rivolta al marito - ti prenderai la carrozza\ldots{} 

E tra marito e moglie si cominciò a decidere come avrebbero passato la giornata. Poiché lui doveva andare a incontrare qualcuno per dovere d'ufficio e lei al concerto e alla seduta pubblica del comitato sud-orientale, bisognava risolvere e considerare molte cose. Levin, come persona di casa, doveva prender parte a questi conciliaboli. Fu deciso che Levin sarebbe andato al concerto e alla seduta pubblica con Natalie, e di là avrebbero mandato la carrozza all'ufficio a prendere Arsenij, e lui sarebbe passato a rilevarla per portarla da Kitty; oppure, s'egli non avesse sbrigato gli affari, avrebbe mandato la carrozza, e Levin sarebbe andato con lei. 

- Ecco, egli mi vizia - disse L'vov alla moglie - mi assicura che i nostri ragazzi sono buonissimi, quando io so che in loro c'è tanto di cattivo. 

- Arsenij giunge agli estremi, io dico sempre - disse la moglie. - A cercar la perfezione, non si sarà mai contenti. E dice bene papà, che quando educavano noi, si esagerava, ci tenevano nei mezzanini, e i genitori vivevano al piano nobile; adesso al contrario, i genitori in uno stambugio e i ragazzi al piano nobile. I genitori adesso non devono più vivere, ma tutto deve essere per i figli. 

- Ebbene, se questo fa più piacere? - disse L'vov, sorridendo col suo bel sorriso e toccandole il braccio. - Chi non ti conosce penserà che tu non sia un madre, ma una matrigna. 

- No, l'esagerazione non va bene in nulla - disse tranquilla Natalie, mettendo al suo posto il tagliacarte sul tavolo. 

- Su, ecco, venite qua, ragazzi perfetti - disse L'vov ai bei ragazzi che entravano, i quali, dopo aver salutato Levin, si avvicinarono al padre, desiderando evidentemente di chiedergli qualcosa. 

Levin aveva voglia di parlare un po' con loro, d'ascoltare quello che avrebbero detto al padre, ma Natalie si mise a parlare con lui, e proprio in quel momento entrò nella stanza un compagno di ufficio di L'vov, Machotin, in uniforme di corte, per andare insieme ad incontrare quella tale persona, e cominciò una conversazione senza fine sull'Erzegovina, sulla principessa Korzinskaja, sull'assemblea, sulla morte improvvisa dell'Apraksina. 

Levin s'era perfino dimenticato dell'incarico che gli avevano dato. Se ne ricordò quando era già per uscire, in anticamera. 

- Ah, Kitty mi ha incaricato di parlare un po' con voi degli Oblonskij - disse quando L'vov si fermò sulla scala, accompagnando la moglie e lui. 

- Sì, sì, maman vuole che noi, les beaux-frères, lo assaliamo - disse egli, arrossendo. - E poi, perché mai io? 

- E allora, lo assalirò io - disse, sorridendo, la L'vova, che aspettava la fine della conversazione nella sua bianca rotonde di pelo di cane. - Su, andiamo. 

\capitolo{V}\label{v-6} 

Al concerto pomeridiano venivano date due cose molto interessanti. 

Una era la fantasia Re Lear nella steppa, l'altra era un quartetto dedicato alla memoria di Bach. Tutte e due le cose erano nuove e di gusto moderno, e Levin desiderava di farsene un'opinione. Dopo aver accompagnato la cognata alla sua poltrona, si pose in piedi vicino a una colonna e decise di ascoltare nel modo più attento e coscienzioso possibile. Cercava di non distrarsi e di non sciupare l'impressione guardando il gesticolare delle mani del direttore d'orchestra dalla cravatta bianca, che sempre così spiacevolmente distrae l'attenzione musicale, le signore in cappello, che per il concerto s'erano fasciate con cura le orecchie con nastri, e tutti quei visi oziosi, o presi dagli interessi più vari, tranne quello della musica. Cercava di evitare incontri con intenditori di musica e ciarlatori, guardando in giù davanti a sé, e ascoltando. 

Ma quanto più egli ascoltava la fantasia di Re Lear, tanto più si sentiva lontano dalla possibilità di formarsene una opinione definitiva. Senza posa cominciava, come se si preparasse l'espressione musicale d'un sentimento, ma subito si scomponeva in nuovi frammenti di frasi musicali, accennate, e a volte in suoni per null'altro legati se non che per capriccio del compositore, e straordinariamente complessi. Ma anche gli stessi frammenti di queste frasi musicali, a volte buone, erano spiacevoli, perché del tutto inaspettati e non predisposti. L'allegria e la tristezza, la disperazione e la tenerezza e il trionfo comparivano senza alcun diritto, come i sentimenti di un folle. E, nello stesso tempo, come nel folle, questi sentimenti passavano inaspettatamente. 

Levin, per tutto il tempo della esecuzione, provò la sensazione d'un sordo che guardi dei danzatori. Era in un'assoluta perplessità quando il brano fu finito e sentiva una grande stanchezza per l'attenzione tesa e non ricompensata da nulla. Da tutte le parti si udirono forti applausi. Tutti si alzarono, presero a camminare, a parlare. Desiderando di chiarire la propria perplessità con le opinioni degli altri, Levin andò in giro, cercando gli intenditori, e fu contento di scorgerne uno dei più noti in colloquio con Pescov ch'egli conosceva. 

- Sorprendente! - diceva la voce piena, di basso, di Pescov. - Buon giorno, Konstantin Dmitric. È in particolar modo immaginoso e scultoreo, per così dire, ricco di colori quel punto dove si sente l'avvicinarsi di Cordelia, dove la donna, das ewig Weibliche, entra in lotta col fato. Non è vero? 

- Come dite, perché mai Cordelia qui? - domandò timido Levin, avendo completamente dimenticato che la fantasia rappresentava re Lear nella steppa. 

- Appare Cordelia\ldots{} ecco! - disse Pescov, battendo col dito sul programma di raso che teneva in mano e passandolo a Levin. 

Soltanto allora Levin si ricordò del titolo della fantasia e si affrettò a leggere i versi di Shakespeare nella traduzione russa, stampati a tergo nel programma. 

- Senza questo non si può seguire - disse Pescov, rivolgendosi a Levin, poiché il suo interlocutore se n'era andato ed egli non aveva più con chi parlare. 

Nell'intervallo si intavolò fra Levin e Pescov una discussione sui pregi e sui difetti dell'indirizzo wagneriano della musica. Levin dimostrava che l'errore di Wagner e di tutti i suoi discepoli consisteva nel fatto che la musica voleva passare nel campo di un'arte non sua, che nello stesso modo vien meno la poesia quando descrive i tratti di un volto, cosa che spetta alla pittura, e come esempio di un tale errore, citò uno scultore cui era venuto in mente di tagliare nel marmo le ombre delle immagini poetiche sorgenti intorno alla figura d'un poeta su di un piedistallo. 

- Queste ombre sono così poco ombre nell'opera dello scultore, che devono reggersi a una scala - disse Levin. 

Questa frase gli piacque, ma non ricordava se questa stessa frase non l'avesse detta già prima, e proprio a Pescov, così che, detto questo, si confuse. 

Pescov, invece, dimostrava che l'arte era una e che poteva raggiungere le sue più alte manifestazioni soltanto nell'unione di tutti i generi. 

Il secondo numero del concerto Levin non poté più ascoltarlo. Pescov, fermatoglisi accanto, parlò quasi tutto il tempo, riprovando quel pezzo per la sua eccessiva, sdolcinata, voluta, semplicità, paragonandola alla semplicità dei preraffaelliti in pittura. All'uscita, Levin incontrò ancora molti conoscenti coi quali parlò di politica, di musica e di amici comuni; fra l'altro incontrò il conte Bol', dal quale aveva completamente dimenticato di andare. 

- Su, allora andate subito - gli disse la L'vova, alla quale egli riferì questo - forse non vi riceveranno, e poi venite a prendermi alla seduta. Mi troverete ancora. 

\capitolo{VI}\label{vi-6} 

- Forse non ricevono? - disse Levin, entrando nell'ingresso della casa della contessa Bol'. 

- Ricevono, prego - disse il portiere, togliendogli decisamente la pelliccia di dosso. 

``Che rabbia! - pensava Levin, togliendosi con un sospiro un guanto e accomodando il cappello. - Ma perché ci vengo? che ragione ho mai di parlare con loro?''. 

Passando per il primo salotto, Levin incontrò sulla porta la contessa Bol' che, con viso preoccupato e severo, ordinava qualcosa a un cameriere. Visto Levin, sorrise e lo fece entrare nel piccolo salotto attiguo, in cui si sentivano delle voci. In questo salotto sedevano, in poltrona, le due figlie della contessa e un colonnello moscovita che Levin conosceva. Levin si avvicinò loro, salutò e sedette accanto al divano, tenendo il cappello sulle ginocchia. 

- Come va la salute di vostra moglie? Siete stato al concerto? Noi non abbiamo potuto. La mamma doveva andare alla messa di requiem. 

- Sì, ho sentito\ldots{} Che morte improvvisa! - disse Levin. 

Venne la contessa, sedette sul divano e domandò anche lei della moglie e del concerto. 

Levin rispose e ripeté la frase sulla morte improvvisa della Apraksina. 

- Del resto era sempre stata di salute debole. 

- Siete stato all'opera ieri? 

- Sì, ci sono stato. 

- Come è andata bene la Lucca! 

- Sì, molto bene - disse egli e, siccome gli era del tutto indifferente quel che avrebbero pensato di lui, cominciò a ripetere quel che aveva sentito centinaia di volte sulla particolarità del talento della cantante. La contessa Bol' fingeva di ascoltare. Poi, quando egli ebbe parlato abbastanza e tacque, il colonnello che fino allora aveva taciuto, cominciò a parlare. Il colonnello prese a parlare anche lui dell'opera e dell'illuminazione. Infine, dopo aver parlato della sua programmata folle journée da Tjurin, si mise a ridere e a far chiasso, poi si alzò e uscì. Levin si alzò pure, ma dal viso della contessa si accorse che per lui non era ancora tempo d'andarsene. Ci volevano ancora due minuti. Sedette. 

Ma poiché pensava di continuo come fosse sciocco tutto ciò, non trovava neppure un argomento di conversazione e taceva. 

- Non andate alla seduta pubblica? Dicono che sia molto interessante - disse la contessa. 

- No, ho promesso alla mia belle-soeur d'andarla a prendere - disse Levin. 

Seguì un silenzio. La madre e la figlia si guardarono ancora una volta. 

``Su, adesso pare che sia ora'' pensò Levin e si alzò. Le signore gli strinsero la mano e pregarono di dire mille choses alla moglie. 

Il portiere gli domandò, tendendogli la pelliccia: 

- Dove abitate, di grazia? - e lo annotò immediatamente in un gran libro ben rilegato. 

``S'intende, per me è lo stesso, tuttavia è vergognoso e orribilmente sciocco'' pensò Levin, consolandosi col dirsi che lo facevano tutti, e andò alla seduta pubblica del comitato, dove doveva trovar la cognata per andare a casa con lei. 

Alla seduta pubblica del comitato c'era molta gente e quasi tutto il gran mondo. Levin giunse ancora in tempo per sentire una relazione che, come dicevano tutti, era molto interessante. Quando questa lettura ebbe termine, la società si riunì e lì Levin incontrò Svijazskij che lo invitò assolutamente per quella sera alla società d'economia rurale, dove si sarebbe letta una famosa relazione, incontrò Stepan Arkad'ic, che era appena arrivato dalle corse e ancora molti altri conoscenti; e Levin parlò e ascoltò i giudizi più svariati sulla seduta, su una commedia nuova e su di un processo. Ma, probabilmente per la stanchezza dovuta all'attenzione, che cominciava a provare, si sbagliò parlando del processo, e questo sbaglio poi gli venne in mente varie volte con suo dispetto. Parlando della prossima condanna d'uno straniero giudicato in Russia, e di come sarebbe stato ingiusto punirlo con l'estradizione, Levin aveva ripetuto quel che aveva sentito il giorno innanzi in una conversazione da un conoscente. 

- Io penso che mandarlo all'estero è lo stesso che punire un luccio lasciandolo andare in acqua - disse Levin. Soltanto dopo si ricordò che questa idea, sentita da un conoscente e data per propria da costui, era di una favola di Krylov, e che il conoscente l'aveva ripetuta ricavandola dall'articolo di un giornale. 

Rientrato a casa con la cognata e trovatavi Kitty allegra e felice, Levin andò al club. 

\capitolo{VII}\label{vii-6} 

Levin arrivò al club all'ora giusta. Insieme con lui giungevano ospiti e soci. Levin non era stato al club da molto tempo, fin da quando, finiti gli studi universitari, viveva a Mosca e andava in società. Ricordava il club, i particolari esteriori della sua organizzazione, ma aveva completamente dimenticato l'impressione che provava prima al club. Ma appena entrato nel cortile largo, semicircolare, e appena sceso dalla vettura, salì la scalinata e incontro a lui un portiere con la bandoliera aprì la porta senza rumore e s'inchinò; appena vide nella portineria le soprascarpe e le pellicce dei soci, i quali avevano considerato che costava minor fatica togliere le soprascarpe giù che non portarle su; appena sentì la scampanellata misteriosa che lo precedeva e, salendo per la scala a dolce pendio, coperta d'un tappeto, vide la statua sul pianerottolo e, sulla porta di sopra, nella livrea del club, il terzo portiere a lui noto, invecchiato, il quale senza affrettarsi e senza indugiare apriva la porta ed esaminava l'ospite, Levin fu preso dall'antica impressione del club: un'impressione di distensione, di benessere e di decoro. 

- Favorite il cappello - disse il portiere a Levin, che aveva dimenticato la regola del club di lasciare i cappelli in portineria. - È un pezzo che non siete venuto. Il principe vi ha iscritto proprio ieri. Il principe Stepan Arkad'ic non c'è ancora. 

Il portiere conosceva non solo Levin, ma anche tutte le relazioni e la parentela, e aveva ricordato immediatamente le persone che gli erano prossime. 

Attraversata la prima sala di passaggio coi paraventi e a destra la stanza chiusa da un tramezzo dove sedeva il dispensiere della frutta, Levin, sorpassato un vecchio che camminava adagio, andò nella sala da pranzo che rumoreggiava di gente. 

Passò lungo le tavole già quasi occupate, osservando gli ospiti. Ora qua, ora là gli capitavano dinanzi le persone più disparate, e vecchie e giovani, appena conosciute e intime. Non c'era neppure un viso irritato o preoccupato. Sembrava che tutti avessero lasciato in portineria, insieme con i berretti, le agitazioni e le preoccupazioni, e si preparassero a usar senza fretta dei beni materiali della vita. Qui c'erano Svijazskij e Šcerbackij e Nevedovskij e il vecchio principe e Vronskij e Sergej Ivanovic. 

- Ah, come, sei in ritardo? - disse, sorridendo, il principe, dandogli la mano al di sopra della spalla. - Come va Kitty? - aggiunse, accomodando il tovagliolo che s'era messo dietro il bottone del panciotto. 

- Sta bene; pranzano in tre a casa. 

- Ah, le Aline-Nadine! Eh, da noi non c'è posto. Ma va' a quella tavola e occupa presto un posto - disse il principe e voltatosi, accolse con prudenza un piatto con la zuppa di pesce. 

- Levin, qua! - gridò un po' più lontano una voce bonaria. Era Turovcyn. Sedeva con un giovane militare, e accanto a loro c'erano due sedie girate. Levin si accostò a loro con gioia. Egli voleva sempre bene a quel bonaccione e gozzovigliatore di Turovcyn, al quale si univa il ricordo della spiegazione con Kitty, ma quel giorno, dopo tutte le conversazioni intellettuali che avevano richiesto un certo sforzo, l'aspetto bonario di Turovcyn gli era particolarmente gradito. 

- Questo è per voi e per Oblonskij. Verrà subito. 

Il militare dagli occhi allegri, sempre ridenti, che si teneva molto dritto, era Gagin di Pietroburgo. Turovcyn li presentò. 

- Oblonskij è certamente in ritardo. 

- Eh, ecco anche lui. 

- Sei appena arrivato? - disse Oblonskij, avvicinandosi svelto a loro. - Salve. Hai bevuto la vodka? Su, andiamo. 

Levin si alzò e andò con lui verso una grande tavola piena di bottiglie di vodka e dei più svariati antipasti. Sembrava che si potesse scegliere quello ch'era di proprio gusto fra una ventina di antipasti, ma Stepan Arkad'ic ne volle uno speciale, e uno dei camerieri in livrea, che stava lì in piedi, portò subito quello che era richiesto. Bevvero un bicchierino per uno e tornarono alla tavola. Immediatamente, mentre ancora mangiavano la zuppa di pesce, a Gagin fu servito dello champagne ed egli ordinò di versarlo in quattro bicchieri. Levin non rifiutò il vino offertogli e chiese un'altra bottiglia. Gli era venuta fame e mangiava e beveva con gran piacere, e con piacere ancora maggiore prendeva parte agli allegri e semplici discorsi degli intervenuti. Gagin, abbassando la voce, raccontò una nuova storiella di Pietroburgo, e la storiella, benché indecente e sciocca, era così comica che Levin scoppiò a ridere tanto forte da far voltare i vicini. 

- È dello stesso stampo di: ``io questo proprio non lo posso sopportare''. La sai? - chiese Stepan Arkad'ic. - Ah, è un incanto! Dammi un'altra bottiglia - disse al cameriere e prese a raccontare. 

- Pëtr Il'ic Vinovskij offre - lo interruppe un cameriere vecchio, avvicinando due bicchieri sottili pieni di champagne spumeggiante, e rivolgendosi a Stepan Arkad'ic e a Levin. Stepan Arkad'ic prese il bicchiere e, scambiato uno sguardo con un uomo rosso, calvo e baffuto, all'altro capo della tavola, gli fece un cenno col capo, sorridendo. 

- Chi è? - disse Levin. 

- L'hai incontrato da me una volta, ricordi? Un bravo ragazzo\ldots{} 

Levin imitò Stepan Arkad'ic e prese il bicchiere. 

La storiella di Stepan Arkad'ic era pure molto divertente. Levin raccontò la sua che pure piacque. Dopo, il discorso cadde sui cavalli, sulle corse di quel giorno e su come audacemente aveva vinto il primo premio Atlasnyj di Vronskij. Levin non si accorse come fosse passato il tempo del pranzo. 

- Ah, ecco anche loro! - disse alla fine del pranzo Stepan Arkad'ic, piegandosi di là dalla spalliera della sedia e tendendo la mano a Vronskij che veniva verso di lui con un colonnello alto della Guardia. Nel viso di Vronskij splendeva la stessa generale allegra bonomia del club. Egli si appoggiò allegramente col gomito alla spalla di Stepan Arkad'ic, mormorandogli qualcosa, e con lo stesso allegro sorriso tese la mano a Levin. 

- Sono molto contento d'incontrarvi - disse. - Vi avevo cercato allora alle elezioni, ma mi dissero che eravate andato via - gli disse. 

- Sì, partii il giorno stesso. Or ora parlavamo del vostro cavallo. Mi congratulo con voi - disse Levin. - È veramente una velocità notevole. 

- Ma già, anche voi avete dei cavalli. 

- No, mio padre ne aveva; ma io me ne ricordo e me ne intendo. 

- Dove hai pranzato? - domandò Stepan Arkad'ic. 

- Noi, alla seconda tavola, dietro le colonne. 

- Gli han fatto i complimenti - disse il colonnello alto. - Il secondo premio imperiale; avessi io tanta fortuna alle carte, quanta ce n'ha lui coi cavalli. 

- Ma perché perdere del tempo d'oro? Vado nell'infernale - disse il colonnello e si allontanò dalla tavola. 

- È Jašvin - rispose Vronskij a Turovcyn, e sedette a un posto che s'era fatto libero accanto a loro. Bevuta la coppa offertagli, chiese una bottiglia. Fosse sotto l'influsso del club, o del vino bevuto, Levin si mise a parlare con Vronskij della razza migliore di bestiame e fu molto contento di non sentir nessuna avversione per quell'uomo. Gli disse perfino, tra l'altro, di aver udito dalla moglie che lo aveva incontrato dalla principessa Mar'ja Borisovna. 

- Ah, la principessa Mar'ja Borisovna, che delizia! - disse Stepan Arkad'ic e raccontò su di lei una storiella che fece ridere tutti. In particolare, Vronskij scoppiò a ridere così di cuore, che Levin si sentì del tutto rappacificato con lui. 

- Ebbene, avete finito? - disse Stepan Arkad'ic, alzandosi e sorridendo. - Andiamo! 

\capitolo{VIII}\label{viii-6} 

Levin, alzatosi da tavola, sentendo nel camminare i movimenti delle braccia particolarmente regolari e liberi, andò con Gagin, attraverso le stanze alte, al biliardo. Passando per il salone, si scontrò col suocero. 

- Be', che te ne pare del nostro tempio dell'ozio? - chiese il principe, prendendolo sotto braccio. - Andiamo a fare un giro. 

- Proprio questo volevo fare, andare a guardare un po'. È interessante. 

- Già, per te è interessante. Ma per me l'interesse è un altro. Tu, ecco, guardi questi vecchietti - disse indicando un socio ingobbito, con un labbro penzoloni che, movendo appena le gambe negli stivali flosci, veniva loro incontro - e pensi che siano nati proprio šljupiki. 

- Come? 

- Ecco, non conosci neppure questa parola. È un nostro termine del club. Sai, quando fanno rotolare le uova, se le fanno rotolare molto, diventano šljupiki. Così anche noi: vai, vai al club e diventi šljupiki. Ma ecco che tu ridi, anche noi abbiamo riso un tempo come te; ora invece guardiamo già come diventeremo šljupiki. Lo conosci il principe cecenskij? - domandò il principe, e Levin vedeva dalla faccia ch'egli si accingeva a raccontar qualcosa di ameno. 

- No, non lo conosco. 

- E come mai? Via, il principe cecenskij, quello famoso. Ma fa lo stesso. Lui, ecco, giuoca sempre a biliardo. Fino a tre anni fa non era ancora tra gli šljupiki e faceva il gradasso. E dava dello šljupik agli altri. Ma ecco, una volta, arriva, e il nostro portiere\ldots{} sai, Vasilij? quello grasso. È un bello spirito. Il principe cecenskij gli domanda: ``Be', Vasilij, chi c'è e chi è venuto? E di šljupiki ce ne sono?''. E quello: ``Ecco, con voi sono tre''. E già, amico mio, proprio così. 

Discorrendo e salutando gli amici che incontravano, Levin e il principe passarono tutte le sale; quella grande dove c'erano già i tavoli e dove i soliti compagni di giuoco facevano la partita; la sala dei divani dove si giocava a scacchi e dove era seduto Sergej Ivanovic il quale discorreva con un tale; quella del biliardo, dove in un gomito della stanza, presso un divano, si era formata un'allegra compagnia con champagne, alla quale prendeva parte Gagin. Dettero un'occhiata anche in quella infernale, dove vicino a un tavolo, al quale era già seduto Jašvin, si affollavano molti sostenitori. Cercando di non far rumore, entrarono nella sala oscura di lettura, dove, sotto le lampade coi paralumi, sedevano un giovane dal viso torvo, che afferrava un giornale dietro l'altro, e un generale calvo, sprofondato nella lettura. Poi, entrarono anche nella stanza che il principe chiamava intellettuale. In questa stanza tre signori parlavano con calore dell'ultima novità politica. 

- Principe, prego, è pronto - disse uno dei suoi compagni di giuoco, trovandolo lì, e il principe andò via. Levin rimase un po' a sedere, ad ascoltare, ma ricordando tutti i discorsi della mattina, gli venne a un tratto una malinconia terribile. Si alzò in fretta e andò a cercare Oblonskij e Turovcyn, coi quali si stava allegri. 

Turovcyn era seduto nella sala del biliardo su di un divano alto con una coppa in mano, e Stepan Arkad'ic con Vronskij parlava di qualcosa, vicino alla porta, in un angolo lontano della stanza. 

- Non è che si annoi, ma questa situazione indefinita, incerta - sentì Levin e voleva allontanarsi in fretta; ma Stepan Arkad'ic lo chiamò. 

- Levin! - disse Stepan Arkad'ic, e Levin notò che negli occhi non aveva lacrime, ma un certo umidore, come gli accadeva sempre quando aveva bevuto o quando era commosso. Quel giorno era l'una e l'altra cosa. - Levin, non te ne andare - disse e gli strinse forte il braccio per il gomito, evidentemente desiderando di non lasciarlo andare per nessuna ragione al mondo. - È il mio sincero, forse il mio migliore amico - disse a Vronskij. - Tu per me sei ancora più prossimo e caro. E io voglio e so che voi dovete essere amici, molto amici, perché siete tutti e due brave persone. 

- E allora, non ci rimane che abbracciarci - disse, scherzando, bonariamente, Vronskij, mentre dava la mano. 

Egli prese rapido la mano tesa e la strinse forte. 

- Sono molto, molto contento - disse Levin, stringendogli la mano. 

- Cameriere, una bottiglia di champagne - disse Stepan Arkad'ic. 

- Anch'io sono molto contento - replicò Vronskij. 

Tuttavia, malgrado il desiderio di Stepan Arkad'ic e il loro reciproco desiderio, non avevano nulla da dirsi, e lo sentivano tutti e due. 

- Sai che lui non conosce Anna? - disse Stepan Arkad'ic a Vronskij. - E io voglio assolutamente portarlo da lei. Andiamo, Levin. 

- Davvero? - disse Vronskij. - Lei ne sarà molto lieta. Io andrei subito a casa - egli soggiunse - ma Jašvin mi preoccupa, e voglio rimanere qua finché non smette. 

- Perché, va male? 

- Sta perdendo tutto, e solo io riesco a trattenerlo. 

- E allora, una carambola? Levin, giuochi? Benissimo - disse Stepan Arkad'ic. - Metti una carambola - disse rivolto al marcatore. 

- È pronta da un pezzo - rispose il marcatore che aveva già messo a triangolo le palle e per distrarsi faceva rotolar la palla rossa. 

- Su, via. 

Dopo la partita, Vronskij e Levin sedettero vicino al tavolo di Gagin, e Levin, su proposta di Stepan Arkad'ic, si mise a puntare sugli assi. Vronskij ora sedeva presso il tavolo circondato da amici che gli si avvicinavano continuamente, ora andava nell'infernale a trovare Jašvin. Levin provava un piacevole riposo dalla stanchezza intellettuale del mattino. Lo rallegrava la fine dell'avversione verso Vronskij, e il senso di distensione, benessere e decoro non lo lasciava. 

Quando la partita fu finita, Stepan Arkad'ic prese Levin sotto braccio. 

- Su, allora, andiamo da Anna. Subito? eh? È in casa. Da tempo le ho promesso di condurti da lei. Tu, dove pensavi di andare a passare la serata? 

- Ma in nessun posto particolare. Ho promesso a Svijazskij di andare alla società di economia rurale. Ma andiamo - disse Levin. 

- Benissimo, andiamo! Informati se è venuta la mia carrozza - disse Stepan Arkad'ic a un cameriere. 

Levin si avvicinò al tavolo e pagò i quaranta rubli da lui perduti sugli assi, pagò le spese del club, note, in un certo modo misterioso, al vecchio cameriere che stava in piedi vicino alla porta, e agitando con forza le braccia attraversò le sale dirigendosi verso l'uscita. 

\capitolo{IX}\label{ix-6} 

- La carrozza di Oblonskij! - gridò il portiere con voce di basso irritata. La carrozza si accostò, e ci salirono tutti e due. Solo nel primo momento, mentre la carrozza usciva dal portone, Levin continuò a provare l'impressione di distensione del club, di benessere e di sicuro decoro di quello che lo circondava; ma non appena la carrozza uscì sulla via ed egli sentì il traballìo del veicolo sulla strada ineguale, udì il grido rabbioso di un vetturale, e vide nella luce smorta l'insegna rossa d'una bettola e di una botteguccia, quell'impressione svanì ed egli cominciò a riflettere se faceva bene, oppure no ad andare da Anna. Che avrebbe detto Kitty? Ma Stepan Arkad'ic non gli consentì di riflettere e, quasi indovinando i suoi dubbi, li disperse. 

- Come sono contento - disse - che tu la conosca! Lo sai, Dolly lo desiderava da tempo. Anche L'vov è già stato da lei e ci va. Sebbene mi sia sorella - seguitò Stepan Arkad'ic - posso dire senza timore che è una donna particolare. Ecco, vedrai. La sua situazione è molto penosa, specialmente adesso. 

- E perché proprio adesso? 

- Abbiamo in corso trattative con suo marito per il divorzio. E lui acconsente; ma ora sorgono delle difficoltà per il figlio, e questa faccenda, che doveva finire da un pezzo, si trascina da tre mesi. Non appena verrà il divorzio, ella sposerà Vronskij. Com'è sciocca quella vecchia usanza, a cui nessuno crede, di fare il giro cantando ``Isaia esulta'' che intralcia la felicità della gente! - intercalò Stepan Arkad'ic. - Bene, prima o poi la loro situazione sarà definita, come la mia e la tua. 

- E in che consiste la difficoltà? - disse Levin. 

- Ah, è una storia lunga e noiosa! Tutto questo è così mal definito da noi. Ma il fatto è che lei vive da tre mesi, aspettando questo divorzio, a Mosca, dove tutti conoscono lui e lei; non va in nessun posto; di donne non vede che Dolly, perché, capirai, non vuole che vadano da lei per compassione; quella stupida principessa Varvara, anche quella è andata via, ritenendo ciò poco conveniente. E così, in questa situazione un'altra donna non avrebbe potuto trovare risorse in se stessa. Lei invece, ecco, vedrai come ha organizzato la propria vita, come è calma, dignitosa. A sinistra, nel vicolo di fronte alla chiesa! - gridò Stepan Arkad'ic, piegandosi verso il finestrino della carrozza. - Uff! che caldo! - disse, aprendo ancor più, malgrado i dodici gradi sotto zero, la sua pelliccia già aperta. 

- Ma poiché ha una figlia, probabilmente, si occupa di lei - disse Levin. 

- Tu, a quanto pare, immagini ogni donna come una femmina, come une couveuse - disse Stepan Arkad'ic. - Se è occupata, è di sicuro coi bambini. No, la educa benissimo, a quanto pare, ma non se ne sente parlare. È occupata, in primo luogo, a scrivere. Vedo già che tu sorridi ironicamente, ma a torto. Scrive un libro per ragazzi e non ne parla a nessuno, ma a me l'ha letto, e io ho dato il manoscritto a Vorkuev\ldots{} sai quell'editore\ldots{} e anche lui è scrittore, mi pare. Lui se ne intende, e dice che è una cosa notevole. Ma tu pensi che sia una scrittrice? Per nulla. Prima di tutto è una donna di cuore, del resto lo vedrai. Adesso ha una bambina inglese e tutta una famiglia di cui si occupa. 

- Be', qualcosa di filantropico? 

- Ecco, tu vuoi subito vedere il male. Non di filantropico, ma di cuore. Loro, cioè Vronskij, aveva un allenatore inglese, maestro dell'arte sua, ma ubriacone. Costui s'è proprio dato al bere, delirium tremens; e i familiari sono abbandonati. Lei li ha visti, li ha aiutati, ci s'è affezionata, e adesso tutta la famiglia è sulle sue spalle, e non così, dall'alto in basso, a denari, ma lei stessa prepara i bambini per il russo per l'ammissione al ginnasio, e la bambina l'ha presa con sé. Ma ecco, la vedrai. 

La carrozza entrò nel cortile, e Stepan Arkad'ic sonò forte a un ingresso dove era ferma una slitta. 

E senza chiedere all'inserviente che aveva aperto la porta se erano in casa, Stepan Arkad'ic entrò nell'ingresso. Levin lo seguiva, sempre più in dubbio se faceva bene o male. 

Guardatosi nello specchio, Levin notò che era divenuto rosso; ma era sicuro di non essere ubriaco e andò su per la scala, coperta di un tappeto, dietro a Stepan Arkad'ic. Di sopra, al cameriere che s'era inchinato come a persona intima, Stepan Arkad'ic chiese chi c'era da Anna Arkad'evna e ricevette la risposta che c'era il signor Vorkuev. 

- Dove sono? 

- Nello studio. 

Attraversata una piccola sala da pranzo con pareti scure di legno, Stepan Arkad'ic e Levin, su un morbido tappeto, entrarono in uno studio semibuio, illuminato da una sola lampada con un grosso paralume scuro. Un'altra lampada a riflettere era accesa sulla parete e illuminava un gran ritratto di donna in piedi, su cui Levin rivolse involontariamente l'attenzione. Era il ritratto di Anna, fatto in Italia da Michajlov. Mentre Stepan Arkad'ic entrava di là da una grata e una voce maschile che parlava tacque, Levin guardò il ritratto, che nella luce scintillante risaltava fuori dalla cornice, e non riuscì a staccarne gli occhi. Aveva perfino dimenticato dove si trovava, e, senza ascoltare quello che si diceva, non abbassava gli occhi dal ritratto meraviglioso. Non era un ritratto, ma una deliziosa donna viva, coi capelli neri ondulati, le spalle e le braccia nude e un pensoso, appena accennato sorriso sulle labbra coperte di sottile peluria, che lo guardava trionfante e tenera con occhi che intimidivano. Non era viva solo perché era più bella di quel che possa essere una donna viva. 

- Sono molto contenta - egli sentì a un tratto accanto a sé una voce evidentemente rivolta a lui, la voce di quella stessa donna che aveva ammirato nel quadro. Anna gli era uscita incontro di là dalla grata e Levin vide, nella penombra dello studio, quella stessa donna del ritratto, in abito scuro d'un turchino cangiante, non nella posa, non con l'espressione, ma della stessa bellezza con cui era stata colta dall'artista nel ritratto. Era meno splendente nella realtà, ma in compenso in lei viva c'era un fascino nuovo che mancava nel ritratto. 

\capitolo{X}\label{x-6} 

Ella gli era andata incontro, senza nascondere la propria gioia nel vederlo. E nella calma con cui ella tese la mano piccola ed energica, e con cui lo presentò a Vorkuev e indicò la graziosa bambina rossastra che sedeva là, intenta a un lavoro, chiamandola sua allieva, Levin riconobbe le maniere a lui note e gradite della donna del gran mondo, sempre calma e naturale. 

- Molto, molto contenta - ella ripeté e sulle sue labbra, chissà perché, quelle parole acquistarono per Levin un significato particolare. - Vi conosco e vi voglio bene da lungo tempo e per l'amicizia con Stiva e per vostra moglie\ldots{} l'ho conosciuta per poco tempo, ma ha lasciato in me l'impressione di un fiore delizioso, proprio di un fiore. E sarà presto mamma! 

Ella parlava liberamente e senza fretta, qualche rara volta portando il suo sguardo da Levin al fratello, e Levin sentiva che l'impressione da lui prodotta era buona, e subito provò una sensazione lieve, semplice a star con lei, come se l'avesse conosciuta dall'infanzia. 

- Ivan Petrovic ed io ci siamo messi nello studio di Aleksej - disse ella, rispondendo a Stepan Arkad'ic alla domanda se poteva fumare - appunto per fumare - e, dopo aver guardato Levin, invece di chiedergli se fumava, avvicinò a sé un portasigari di tartaruga e ne tirò fuori una sigaretta. 

- Come va la tua salute oggi? - domandò il fratello. 

- Così. I nervi sono come sempre. 

- Non è vero che è straordinariamente bello? - disse Stepan Arkad'ic, notando che Levin guardava di tanto in tanto il ritratto. 

- Non ho veduto ritratto più bello. 

- È straordinariamente somigliante, vero? - disse Vorkuev. 

Levin dal ritratto passò a guardare l'originale. Uno splendore particolare illuminò il viso di Anna nell'attimo in cui ella sentì su di sé il suo sguardo. Levin arrossì, e, per nascondere la propria confusione, voleva domandare se era da molto che non vedeva Dolly ma nello stesso tempo Anna cominciò a parlare. 

- Si parlava or ora con Ivan Petrovic degli ultimi quadri di Vašcenkov. Li avete visti? 

- Sì, li ho visti - rispose Levin. 

- Ma perdonate, vi ho interrotto, volevate dire\ldots{} 

Levin domandò se era molto che ella non vedeva Dolly. 

- È stata ieri da me, è molto arrabbiata per il ginnasio di Griša. Il professore di latino mi pare sia ingiusto con lui. 

- Già, li ho visti questi quadri. Non mi sono piaciuti molto - disse Levin, tornando al discorso cominciato da lei. 

Levin, adesso, non parlava assolutamente più con quel modo tecnico di trattar l'argomento, che aveva usato la mattina. Ogni parola, nella conversazione con lei, acquistava un significato particolare. E parlare con lei era piacevole, ma ancor più piacevole era ascoltarla. 

Anna parlava non solo con naturalezza e intelligenza, ma con intelligenza e noncuranza, senza attribuire nessun peso alle proprie idee, e dando grande importanza alle idee dell'interlocutore. 

Si venne a parlare della nuova tendenza dell'arte, della nuova Bibbia illustrata da un pittore francese. Vorkuev accusava il pittore di un realismo spinto fino alla volgarità. Levin disse che i francesi avevano spinto il convenzionale nell'arte come nessun altro popolo e che perciò vedevano un merito particolare nel ritorno al realismo. Nel fatto di non mentire più, vedevano la poesia. 

Mai ancora nessuna cosa intelligente detta da Levin le aveva fatto tanto piacere come questa. Il viso di Anna si illuminò quando, a un tratto, apprezzò questo pensiero. Ella si mise a ridere. 

- Rido - disse - come si ride quando si vede un ritratto molto somigliante. Quello che avete detto caratterizza perfettamente l'arte francese di adesso, e la pittura e perfino la letteratura: Zola, Daudet. Ma forse, accade sempre così: si costruiscono le proprie conceptions con figure convenzionali inventate, e poi, quando tutte le combinaisons sono tentate, e le figure inventate son venute a noia, allora si cominciano a inventare figure più vicine alla natura, al vero. 

- Ma è proprio giusto! - disse Vorkuev. 

- E allora siete stati al club? - disse lei al fratello. 

``Sì, ecco una donna!'' pensava Levin, dimentico di sé e guardando ostinatamente il viso bello, mobile di lei che adesso s'era d'un tratto del tutto mutato. Levin non sentiva di cosa ella parlasse, dopo essersi tutta piegata verso il fratello, ma fu sorpreso del mutamento della sua espressione. Il viso di lei, prima tanto bello nella calma, espresse a un tratto una strana curiosità, una rabbia e un certo orgoglio. Ma durò solo un momento. Ella socchiuse gli occhi, come per ricordare qualcosa. 

- Eh sì, del resto, questo non interessa nessuno - disse e si rivolse alla inglese: - Please order the tea in the drawing-room. 

La bambina si alzò e uscì. 

- Be', l'ha superato l'esame? - domandò Stepan Arkad'ic. 

- Benissimo. È una bambina di grande ingegno e di carattere simpatico. 

- Andrai a finire che l'amerai più della tua. 

- Ecco un uomo che parla. Nell'amore non c'è più e meno. Amo mia figlia d'un amore, lei d'un altro. 

- Io, ecco, stavo dicendo ad Anna Arkad'evna - disse Vorkuev - che, qualora dedicasse sia pure una centesima parte dell'energia, che adopera per questa inglese, alla causa comune dell'educazione dei bambini russi, Anna Arkad'evna compirebbe un'opera grande e utile. 

- Sì, ecco, che volete, non potevo. Il conte Aleksej Kirillovic mi spingeva molto - pronunciando ``il conte Aleksej Kirillovic'' ella guardò interrogativamente Levin ed egli le rispose involontariamente con uno sguardo rispettoso e affermativo - mi incitava a occuparmi della scuola in campagna. Ci sono andata varie volte. Sono molto carine le bambine, ma non sono riuscita ad affezionarmi a quest'opera. Voi dite: energia. L'energia è fondata sull'amore. E donde prenderlo l'amore? non si può comandarlo. Ecco, ho preso a voler bene a questa bambina, io stessa non so perché. 

Ed ella guardò di nuovo Levin. E il sorriso, e lo sguardo di lei, tutto gli diceva che, soltanto a lui, ella rivolgeva il proprio discorso, prendendone in considerazione l'opinione e nello stesso tempo sapendo già che si capivano scambievolmente. 

- Lo capisco benissimo - rispose Levin. - Alla scuola, e in genere a simili istituzioni, non si può dare il cuore, e penso che, appunto per questo, tali istituzioni filantropiche dànno sempre così scarsi risultati. 

Ella tacque un po', poi sorrise. 

- Sì, sì - confermò. - Io non ho mai potuto. Je n'ai pas le coeur assez large, per mettermi a voler bene a tutto un asilo con delle bambine sporche. Cela ne m'a jamais réussi. Ci sono tante donne che se ne sono fatte una position sociale. E adesso tanto più - disse, rivolgendosi con una triste, confidente espressione, in apparenza al fratello ma evidentemente solo a Levin. - Anche adesso, quando ho tanto bisogno di un'occupazione, non posso. - E accigliatasi a un tratto (Levin capì che s'era accigliata verso se stessa perché parlava di sé), ella cambiò discorso. - So di voi - disse a Levin - che siete un cattivo cittadino, e vi ho difeso come potevo. 

- E come mi avete difeso? 

- Secondo gli assalti. Vi fa piacere un po' di tè? - Ella si alzò e prese in mano un libro rilegato in marocchino. 

- Datemelo, Anna Arkad'evna - disse Vorkuev, indicando il libro. - Ne vale molto la pena. 

- Oh no, è tutto così poco rifinito. 

- L'ho detto a lui - disse Stepan Arkad'ic rivolto alla sorella e indicando Levin. 

- Hai fatto male. Il mio scritto è sul genere di quei cestini intagliati che talvolta mi vendeva Liza Merkalova dalle carceri. Ella si occupava dei carcerati in quella società - disse rivolta a Levin. - E quei disgraziati facevano miracoli di pazienza. 

E Levin scoprì ancora un nuovo tratto di quella donna, che gli era piaciuta in modo così straordinario. Oltre la grazia, l'intelligenza, la bellezza, in lei c'era la sincerità. Ella non voleva nascondergli tutta la difficoltà della propria situazione. Detto questo, ella sospirò, e il suo viso, presa a un tratto una espressione severa, si fece come di pietra. Con un'espressione simile sul viso ella era ancora più bella di prima; quest'espressione era nuova, era al di fuori di quella sfera di espressioni, che splendeva di felicità ed effondeva felicità e che era stata colta dal pittore nel quadro. Levin guardò ancora una volta il quadro e la figura di lei, quando, al braccio del fratello, ella passò con lui attraverso la porta alta, e sentì per lei una tenerezza e una compassione che lo stupirono. 

Ella pregò Levin e Vorkuev di passare in salotto, e lei stessa rimase a parlare di qualcosa col fratello. ``Del divorzio, di Vronskij, di quel ch'egli fa al club, di me'' pensò Levin. E lo agitava tanto la questione di che cosa parlasse con Stepan Arkad'ic, che quasi non ascoltava quello che gli andava dicendo Vorkuev sui pregi del romanzo per ragazzi scritto da Anna Arkad'evna. 

Durante il tè, continuò quella piacevole conversazione, densa di contenuti. Non solo non c'era un attimo in cui bisognasse cercare l'argomento per discorrere, ma, al contrario, si sentiva che non c'era il tempo di dire quello che si voleva, e che volentieri ci si tratteneva ad ascoltare quello che l'altro diceva. E tutto quello che dicevano, non soltanto lei, ma Vorkuev, Stepan Arkad'ic, tutto acquistava un'importanza particolare, come pareva a Levin, grazie all'attenzione e alle osservazioni fatte da lei. 

Seguendo la conversazione interessante, Levin tutto il tempo ammirava lei e la sua bellezza e intelligenza e la cultura e insieme la semplicità e la cordialità. Egli ascoltava, parlava e tutto il tempo pensava a lei, alla sua vita interiore, cercando di indovinare i suoi sentimenti. E lui che prima la rimproverava così duramente, adesso, per un certo strano corso di pensieri, la giustificava e insieme la compiangeva, e temeva che Vronskij non la capisse in pieno. Dopo le dieci, quando Stepan Arkad'ic si alzò per andar via (Vorkuev era andato via prima), a Levin parve d'esser giunto allora. E si alzò anche lui, con rammarico. 

- Addio - ella disse, trattenendolo per la mano e guardandolo negli occhi con uno sguardo pieno di fascino. - Sono molto contenta que la glace est rompue. 

Lasciò andare la mano di lui e socchiuse gli occhi. 

- Dite a vostra moglie che le voglio bene come prima e che, se ella non mi può perdonare la mia situazione, allora le auguro di non perdonarmi mai. Per perdonare bisogna passare quello che ho passato io, e da questo la salvi Iddio. 

- Certamente, sì, lo dirò\ldots{} - diceva Levin, arrossendo. 

\capitolo{XI}\label{xi-6} 

``Che donna straordinaria, simpatica, degna di compassione!'' egli pensava, uscendo con Stepan Arkad'ic all'aria gelida. 

- Be', via, te l'avevo detto - gli disse Stepan Arkad'ic, vedendo che Levin era completamente conquistato. 

- Sì, rispose Levin - una donna straordinaria. Non solo intelligente, ma tanto cordiale! Fa una gran pena! 

- Adesso, con l'aiuto di Dio, presto s'accomoderà tutto. Eh sì, non giudichiamo prima del tempo - disse Stepan Arkad'ic, aprendo lo sportello della carrozza. - Addio, non facciamo la stessa strada. 

Senza smettere di pensare ad Anna, a tutti i discorsi più semplici che c'erano stati con lei, e ricordando intanto tutti i particolari dell'espressione del suo viso, sempre più immedesimandosi nella posizione di lei e provandone pena, Levin giunse a casa. 

A casa, Kuz'ma riferì a Levin che Katerina Aleksandrovna stava bene, e che solo da poco erano andate vie le sorelle dalla signora e gli diede due lettere. Levin, proprio lì in anticamera, per non dimenticarsene poi, le lesse. Una era di Sokolov, l'amministratore. Sokolov scriveva che il frumento non si poteva vendere, offrivano soltanto cinque rubli e mezzo, ma il denaro non si sapeva più dove prenderlo. L'altra lettera era della sorella. Lo rimproverava che il suo affare non si fosse ancora risolto. 

``E vendiamo per cinque rubli e mezzo, se non ne dànno di più - decise immediatamente Levin, con straordinaria facilità, la prima questione, che in altri tempi gli sarebbe parsa così difficile. - È sorprendente come io qui sia sempre occupato'' pensò a proposito della seconda lettera. Si sentiva colpevole di fronte alla sorella perché finora non aveva fatto quello di cui ella lo aveva pregato. ``Oggi di nuovo non sono andato in tribunale, ma oggi poi non c'era proprio tempo''. E, stabilito che l'avrebbe certamente fatto l'indomani, andò da sua moglie. Andando da lei, Levin percorse rapidamente col pensiero tutto quello che era passato in quel giorno. Tutta la giornata era passata in discorsi: discorsi che aveva ascoltato o ai quali aveva partecipato. Erano tutti su argomenti tali che, se lui fosse stato solo e in campagna, non se ne sarebbe mai occupato, ma qui erano molto interessanti. E tutti i discorsi andavano bene; solo in due punti non andavano bene. L'uno era quello in cui lui aveva parlato del luccio, l'altro quello in cui qualcosa non andava nella tenera pietà ch'egli aveva provato per Anna. 

Levin trovò la moglie triste e annoiata. Il pranzo con le sorelle sarebbe riuscito allegro, ma avevano aspettato lui, l'avevano aspettato tanto, poi tutte avevano preso ad annoiarsi, le sorelle se n'erano andate e lei era rimasta sola. 

- Be', e tu cosa hai fatto? - ella domandò, guardandolo negli occhi che brillavano in maniera particolarmente sospetta. Ma, per non impedirgli di raccontare tutto, ella nascose la propria osservazione e ascoltò con un sorriso d'approvazione il racconto di com'egli aveva passato la serata. 

- Sai, sono stato molto contento d'avere incontrato Vronskij. Mi sono comportato con disinvoltura e semplicità con lui. Capisci, ormai cercherò di non incontrarmi mai più con lui, ma perché questo disagio finisse - disse, ma, ricordatosi che egli, cercando di non incontrarsi mai più, era andato immediatamente da Anna, arrossì. - Ecco, noi diciamo che il popolo beve; non so chi beva di più, il popolo o la nostra classe; il popolo almeno alla festa, ma\ldots{} 

Ma a Kitty non interessava affatto ragionar su come bevesse il popolo. Ella vedeva ch'egli arrossiva e desiderava sapere perché. 

- E poi dove sei stato mai? 

- Stiva mi ha supplicato con straordinaria insistenza di andare da Anna Arkad'evna . 

E, detto questo, Levin arrossì ancora di più, e i suoi dubbi sul fatto se avesse fatto bene o male ad andare da Anna furono definitivamente risolti. Adesso sapeva che non avrebbe dovuto farlo. 

Gli occhi di Kitty si dischiusero e scintillarono in modo particolare al nome di Anna, ma fatto uno sforzo su di sé, ella nascose la propria agitazione e simulò. 

- Ah! - disse soltanto. 

- Tu, non è vero, non ti dispiacerai ch'io sia andato. Stiva mi ha pregato e Dolly lo desiderava - continuò Levin\ldots{} 

- Oh no - disse, ma nei suoi occhi egli vedeva uno sforzo su di sé che non gli prometteva nulla di buono. 

- È una donna molto simpatica, molto, molto da compatire, buona - egli diceva, parlando di Anna, delle sue occupazioni e riferendo quello di cui lo aveva incaricato. 

- Già, s'intende, è molto da compatire - disse Kitty, quando egli ebbe finito. - Da chi mai hai ricevuto una lettera? 

Egli glielo disse e, credendo al suo tono calmo, andò a spogliarsi. 

Tornato, trovò Kitty sulla stessa poltrona. Quando le si avvicinò, ella lo guardò e scoppiò in singhiozzi. 

- Cosa, cosa? - egli domandava, sapendo già da prima cosa. 

- Tu ti sei innamorato di quella donna disgustosa, ti ha affascinato. L'ho visto nei tuoi occhi. Sì, sì! E che ne può venir fuori? Al club hai bevuto, bevuto, hai giocato e poi sei andato\ldots{} da chi? No, partiamo\ldots{} domani io parto. 

A lungo Levin non poté calmare la moglie. Finalmente la calmò quando le confessò che il senso di pena unito al vino l'avevano messo fuor di strada e ch'egli aveva ceduto all'abile influenza di Anna, e che l'avrebbe evitata. La cosa che egli confessava più sinceramente era che, vivendo così a lungo a Mosca, passando il tempo soltanto a far discorsi, a mangiare e a bere, era divenuto un insensato. Parlarono fino alle tre di notte. Soltanto alle tre si rappacificarono e riuscirono ad addormentarsi. 

\capitolo{XII}\label{xii-6} 

Accompagnati gli ospiti, Anna, invece di sedersi, si mise ad andare avanti e indietro per la stanza. Sebbene inconsciamente (come sempre in quegli ultimi tempi nei riguardi di tutti gli uomini giovani), per tutta la sera avesse cercato, con ogni mezzo, di destare in Levin un sentimento d'amore verso di lei, e sebbene sapesse d'averlo ottenuto per quanto è possibile da parte di un uomo onesto ammogliato e in una sola serata, e sebbene egli le fosse piaciuto molto (malgrado la netta differenza, dal punto di vista di un uomo, fra Vronskij e Levin, lei, come donna, vedeva in loro quello stesso lato comune, per cui Kitty aveva amato e Vronskij e Levin), non appena egli fu uscito dalla stanza, cessò di pensare a lui. 

Un unico pensiero, sotto vari aspetti, la perseguitava con insistenza. ``Se io agisco così sugli altri, su quest'uomo che ha famiglia, che ama, perché mai lui è così freddo verso di me?\ldots{} Non che sia freddo, mi ama, lo so. Ma qualcosa di nuovo adesso ci divide. Come mai non viene per tutta la sera? Ha fatto dire da Stiva che non poteva lasciare Jašvin e che doveva sorvegliare il suo gioco. È forse un bambino Jašvin? Ma ammettiamo che sia vero. Egli non dice mai una cosa non vera. Ma in questa verità c'è qualcos'altro. È contento dell'occasione per dimostrami che lui ha altri doveri. Lo so, sono d'accordo su questo. Ma perché dimostrarmelo? Egli vuole dimostrarmi che il suo amore per me non deve intralciare la sua libertà. Ma io non ho bisogno di dimostrazioni, ho bisogno d'amore. Egli dovrebbe capire tutta la difficoltà della mia vita qui, a Mosca. Vivo io forse? Non vivo, ma aspetto lo scioglimento che si trascina e si trascina sempre. Di nuovo non c'è risposta. E Stiva dice che lui non può andare da Aleksej Aleksandrovic. Ma io non posso scrivere ancora. Io non posso fare nulla, non posso cominciar nulla, mutar nulla; mi trattengo, aspetto, mi invento dei passatempi, con la famiglia dell'inglese, con lo scrivere, e il leggere, ma tutto questo è soltanto un inganno, tutto questo è lo stesso della morfina. Egli dovrebbe aver pena di me'' ella diceva, sentendosi venire agli occhi lacrime di compassione verso se stessa. 

Sentì la scampanellata violenta di Vronskij e asciugò in fretta queste lacrime; non solo asciugò le lacrime, ma sedette vicino alla lampada e aprì un libro, fingendosi calma. Bisognava mostrargli ch'era scontenta ch'egli non fosse tornato così come aveva promesso, soltanto scontenta, ma non fargli vedere in nessun modo il proprio dolore, e, soprattutto, la compassione verso se stessa. Lei poteva avere pena di se stessa, ma non lui di lei. Ella non voleva la lotta, lo rimproverava perché egli voleva lottare, e senza volere si poneva lei stessa in posizione di lotta. 

- Be', non ti sei annoiata? - disse egli con animazione e allegramente, avvicinandosi a lei. - Che passione tremenda è il giuoco! 

- No, non mi sono annoiata e già da tempo ho imparato a non annoiarmi. Sono stati qui Stiva e Levin. 

- Già, volevano venire da te. Be', t'è piaciuto Levin? - disse, sedendosi accanto a lei. 

- Molto. Sono andati via da poco. E che ha fatto Jašvin? 

- Era in vincita, diciassettemila rubli. Io lo chiamavo, lui era proprio sul punto d'andar via. Ma è tornato di nuovo, e adesso è in perdita. 

- E allora perché sei rimasto? - ella domandò, levando a un tratto gli occhi su di lui. L'espressione del viso di lei era fredda e ostile. - Hai detto a Stiva che saresti rimasto per portar via Jašvin. E l'hai pure lasciato. 

La stessa espressione di freddezza, di fronte alla lotta, apparve anche sul viso di lui. 

- In primo luogo non gli ho chiesto di dirti niente; in secondo luogo, io non dico mai quello che non è vero. E soprattutto, volevo rimanere e son rimasto - disse egli, aggrottando le sopracciglia. - Anna, perché, perché? - diss'egli dopo un istante di silenzio, piegandosi verso di lei, e aprì la mano, sperando ch'ella vi mettesse la sua. 

Ella era contenta di questo invito alla tenerezza. Ma una certa strana forza maligna non le permetteva di abbandonarsi alla sua inclinazione, come se le condizioni della lotta non le permettessero di sottomettersi. 

- S'intende, tu volevi rimanere e sei rimasto. Tu fai tutto quello che vuoi. Ma perché non lo dici? Per che cosa? - ella diceva, accalorandosi sempre più. - Qualcuno contesta forse i tuoi diritti? Ma tu vuoi avere ragione e abbi pure ragione. 

La mano di lui si chiuse, egli si allontanò, e il suo viso prese un'espressione ancora più caparbia di prima. 

- Per te è questione di ostinazione - diss'ella, dopo averlo guardato fisso e trovando, a un tratto, un nome a quest'espressione del viso che l'irritava - proprio di ostinazione. Per te la questione è se la vincerai con me, ma per me\ldots{} - Di nuovo le venne pena di sé, e si mise quasi a piangere. - Se tu sapessi di che si tratta per me! Quando io sento, come adesso, che tu mi tratti con ostilità, sì proprio con ostilità, se tu sapessi cosa significa questo per me! Se sapessi come sono vicina a una sciagura in questi momenti, come ho paura, come ho paura di me! - e si voltò dall'altra parte per nascondere i singhiozzi. 

- Per quale ragione fai così? - disse lui, spaventato dinanzi a quell'espressione disperata, dopo essersi di nuovo piegato verso di lei e averle preso e baciato la mano. - Perché? Cerco forse delle distrazioni fuori casa? Non evito forse la compagnia delle donne? 

- Ci mancherebbe altro! - disse lei. 

- Dimmelo, via: che devo fare perché tu sia tranquilla? Io sono pronto a tutto purché tu sia felice - egli diceva, commosso dalla disperazione di lei - e che cosa non farei per liberarti da qualsiasi dolore, come adesso, Anna! - diss'egli. 

- Nulla, nulla - ella disse. - Io stessa non so: forse la vita solitaria, i nervi\ldots{} Ma non ne parliamo più. Come sono andate le corse? non me ne hai parlato - domandò cercando di nascondere il trionfo della vittoria che, comunque, era dalla parte sua. 

Egli chiese di cenare e cominciò a raccontarle i particolari delle corse; ma nel tono, negli sguardi di lui, che si facevano sempre più freddi, ella vedeva ch'egli non le aveva perdonato la vittoria, che quel sentimento di ostinazione, contro cui ella aveva lottato, si stabiliva di nuovo in lui. Egli era più freddo di prima con lei, come pentito d'essersi sottomesso. E lei, ricordando le parole che le avevano dato la vittoria, e precisamente ``io son vicina a un'orribile sciagura e ho paura di me stessa'', capì che quest'arma era pericolosa e che non si sarebbe potuto adoperarla una seconda volta. Ma sentiva che, insieme all'amore che li legava, s'era stabilito fra di loro lo spirito perverso di una certa lotta ch'ella non poteva scacciare né dal cuore di lui, né tanto meno dal proprio. 

\capitolo{XIII}\label{xiii-6} 

Non ci sono condizioni tali a cui l'uomo non possa abituarsi, in particolare se vede che tutti quelli che lo circondano vivono allo stesso modo. Levin tre mesi prima non avrebbe creduto di potersi addormentare tranquillamente nelle condizioni in cui era quel giorno; vivendo una vita senza scopo, senza senso, inoltre una vita al di sopra delle proprie possibilità, dopo una ubriacatura (non poteva chiamare diversamente quello che c'era stato al club), i rapporti stranamente amichevoli con un uomo di cui una volta era stata innamorata sua moglie, e la visita ancor più strana a una donna che non si poteva chiamare altrimenti che perduta, e dopo il proprio entusiasmo per questa donna e il dolore della moglie, non sarebbe stato possibile addormentarsi tranquillamente in queste condizioni. Eppure, preso dalla stanchezza, dalla notte insonne e dal vino bevuto, si addormentò profondamente, tranquillo. 

Alle cinque lo scricchiolio di una porta che si apriva lo svegliò. Saltò su e si girò. Kitty non era nel letto accanto a lui. Ma di là dal tramezzo c'era una luce che si moveva, ed egli sentì i passi di lei. 

- Che c'è, che c'è? - egli pronunciò nel sonno. - Kitty, cosa c'è? 

- Nulla - ella disse, uscendo fuori dal tramezzo con una candela in mano. - Non mi sentivo bene - disse, sorridendo di un sorriso particolarmente grazioso e significativo. 

- Cos'è? è cominciato? cominciato? - egli disse con spavento - bisogna mandare a chiamare - e prese a vestirsi in fretta. 

- No, no - ella disse, sorridendo e trattenendolo con la mano. - Probabilmente non è nulla. Mi sentivo indisposta, ma solo un poco. Adesso è passato. 

E, avvicinatasi al letto, spense la candela, si coricò e si calmò. Sebbene lo tenessero in ansia il silenzio del respiro di lei, quasi trattenuto e, più di tutto, l'espressione di particolare tenerezza ed eccitazione con cui, uscendo dal tramezzo, ella gli aveva detto: ``nulla'', egli aveva tanto sonno che si riaddormentò immediatamente. Soltanto dopo ricordò la sospensione del respiro di lei e capì tutto quello che era accaduto nella cara, gentile anima sua quando, senza muoversi, nell'attesa dell'avvenimento più grande nella vita d'una donna, ella si era coricata accanto a lui. Alle sette, lo svegliò il contatto della mano di lei sulla spalla e un sussurro sommesso. Era come s'ella lottasse fra il dispiacere di svegliarlo e il desiderio di parlare con lui. 

- Kostja, non spaventarti. Non è nulla. Ma mi pare\ldots{} Bisogna andare a chiamare Elizaveta Petrovna. 

La candela era accesa di nuovo. Ella era seduta sul letto e teneva in mano il lavoro a maglia di cui si occupava negli ultimi giorni. 

- Per favore, non spaventarti, non è nulla. Io non ho paura per niente - disse, vedendo il viso spaventato di lui, e premette la mano di lui al proprio petto, poi alle labbra. 

Egli saltò su in fretta, senza aver coscienza di sé e, senza levarle gli occhi di dosso, infilò la vestaglia e si fermò, guardandola sempre. Bisognava andare, ma non poteva staccare lo sguardo da lei. O che non gli piacesse quel suo viso, o che non ne conoscesse l'espressione, lo sguardo, certo non l'aveva mai vista così. Come si vedeva disgustoso e detestabile nel ricordare il dolore da poco arrecatole, dinanzi a lei così com'era adesso! Il suo viso, divenuto vermiglio, circondato dai capelli morbidi, di sotto alla cuffia da notte, splendeva di gioia e di risolutezza. 

Per quanto poca fosse la finzione e la convenzionalità nel carattere di Kitty, Levin tuttavia fu sorpreso da quello che ora gli si metteva a nudo, ora che, tolti a un tratto tutti i veli, l'essenza stessa dell'anima di lei le splendeva negli occhi. E in quella semplicità e nudità, lei, proprio quella che egli amava, si vedeva ancora meglio. Lo guardava sorridendo; ma a un tratto le tremarono le sopracciglia, levò il capo, e, avvicinatasi rapida, lo prese per mano e si strinse tutta a lui, inondandolo del proprio respiro caldo. Ella soffriva ed era come se si lamentasse con lui delle proprie sofferenze. E a lui, nel primo momento, per abitudine, parve d'essere colpevole. Ma nello sguardo di lei c'era una tenerezza la quale diceva ch'ella non solo non lo rimproverava, ma lo amava per quelle sofferenze. ``Se non sono io, chi mai è colpevole di questo?'' egli pensò senza volere, cercando il colpevole di quelle sofferenze per punirlo; ma un colpevole non c'era. Ella soffriva, si lamentava ed esultava di quelle sofferenze, ne gioiva e le amava. Egli vedeva che nell'anima di lei si compiva qualcosa di splendido, ma cosa, egli non poteva capirlo. Era al di sopra della sua comprensione. 

- Io mando a chiamar la mamma. E tu va' in fretta a prendere Lizaveta Petrovna\ldots{} Kostja!\ldots{} Non è nulla, è passato. 

Si allontanò da lui e sonò. 

- Su, ecco, adesso va'; viene Paša. Io mi sento abbastanza bene. 

E Levin vide, con stupore, ch'ella riprendeva il lavoro a maglia che aveva portato la notte e si metteva di nuovo a sferruzzare. 

Mentre Levin usciva da una porta, sentì che dall'altra entrava una donna. Si fermò vicino alla porta e sentì Kitty che dava ordini particolareggiati alla donna e che si metteva a spostare il letto con lei. 

Egli si vestì e, mentre attaccavano i cavalli, poiché vetture da nolo non ce n'erano ancora, rientrò di corsa nella stanza da letto, non in punta di piedi, ma sulle ali, come gli sembrava. Due donne cambiavano qualcosa di posto, con precauzione, nella stanza; Kitty camminava e sferruzzava, mettendo rapidamente le maglie sui ferri e dando ordini. 

- Io vado subito dal dottore. A chiamare Lizaveta Petrovna sono già andati, ma io ci passerò ancora. Non c'è bisogno di nulla? Sì, da Dolly? 

Ella lo guardò, evidentemente, senza ascoltare quello ch'egli diceva. 

- Sì, sì, va', va' - disse in fretta, accigliandosi e facendogli un gesto con la mano. 

Egli usciva già in salotto, quando a un tratto un gemito penoso, immediatamente chetatosi, echeggiò dalla stanza da letto. Egli si fermò, e a lungo non poté capire. 

``Sì, è lei'' si disse e, messosi la testa fra le mani, corse giù. 

``Signore abbi pietà, perdona, aiuta!'' egli ripeteva le parole che gli venivano, chi sa come, a un tratto, alle labbra. E lui, persona scettica, ripeteva quelle parole non con le labbra sole. Adesso, in questo momento, egli sapeva che non solo tutti i suoi dubbi, ma quell'impossibilità di credere secondo ragione che conosceva in sé, non gli impedivano per nulla di rivolgersi a Dio. Tutto questo, adesso, era volato via dall'anima sua come polvere. A chi mai rivolgersi, se non a Colui nelle cui mani egli sentiva se stesso, l'anima sua e il suo amore? 

Il cavallo non era ancora pronto, ma, sentendo in sé un particolare tendersi delle forze fisiche e dell'attenzione verso quello che bisognava fare, per non perdere neanche un minuto, senza aspettare il cavallo, uscì a piedi e ordinò a Kuz'ma di raggiungerlo. 

All'angolo incontrò una vettura da nolo notturna che andava in fretta. Nella piccola slitta, in cappa di velluto, con la testa ravvolta in un fazzoletto, sedeva Lizaveta Petrovna. ``Sia lodato Iddio, sia lodato Iddio!'' egli pronunciò, riconoscendo con entusiasmo il piccolo viso biondo di lei, che adesso aveva un'espressione particolarmente seria, perfino severa. Senza fermare la vettura, egli le corse dietro. 

- Allora, un due ore? Non di più? - ella domandò. - Troverete Pëtr Dmitric, ma non lo sollecitate. E prendete dell'oppio in farmacia. 

- Allora, voi pensate che tutto possa andar bene? Signore, abbi pietà e aiutami! - esclamò Levin, dopo aver visto che il suo cavallo usciva dal portone. Saltato nella slitta a fianco di Kuz'ma, ordinò di andare dal dottore. 

\capitolo{XIV}\label{xiv-6} 

Il dottore ancora non s'era alzato, e il cameriere disse che ``il signore s'era coricato tardi e aveva ordinato di non svegliarlo, ma che presto si sarebbe alzato''. Il cameriere puliva i vetri di una lampada e sembrava molto occupato in questa faccenda. Questa attenzione del cameriere per i vetri e l'indifferenza per quello che avveniva in lui, Levin, lo sorpresero dapprima; ma subito, riflettendo, capì che nessuno sapeva né era obbligato a sapere i suoi sentimenti e che tanto più bisognava agire con calma, con riflessione e risolutezza, per sfondare quel muro di indifferenza e raggiungere il proprio scopo. ``Non aver fretta e non trascurare nulla'' si diceva Levin, sentendo un sollevarsi sempre crescente di forze fisiche e di tensione per tutto quello che c'era da fare. 

Saputo che il dottore non s'era ancora alzato, Levin, fra i vari piani che gli si presentarono, si fermò sul seguente: che Kuz'ma andasse con un biglietto da un altro dottore mentre lui sarebbe andato in farmacia a prendere l'oppio, e se, al suo ritorno, il dottore non si fosse alzato ancora, o corrompendo il cameriere o con la forza, se non avesse consentito, avrebbe svegliato il dottore a qualunque costo. 

In farmacia uno sparuto aiutante farmacista, con la stessa indifferenza con cui il cameriere puliva i vetri, suggellava con un'ostia le polverine per un cocchiere che aspettava, e rifiutò l'oppio. Cercando di non avere fretta e di non accalorarsi, fatto il nome del dottore e quello della levatrice, e spiegato perché serviva l'oppio, Levin cominciò a persuaderlo. L'aiutante chiese consiglio in tedesco se dovesse darlo, e, ricevutane da dietro il tramezzo l'approvazione, tirò fuori una fiala e un imbuto; versò lentamente dal recipiente grande nel piccolo, incollò l'etichetta, suggellò, malgrado la preghiera di Levin di lasciar stare e voleva ancora avvolgerla. Questo Levin non poté più sopportarlo: gli strappò decisamente dalle mani la fiala e corse via per la grande porta a vetri. Il dottore non s'era ancora alzato e il cameriere, occupato adesso a stendere un tappeto, si rifiutò di svegliarlo. Levin, senza affrettarsi, tirò fuori un biglietto da dieci rubli, e, pronunciando piano le parole, ma anche senza perder tempo, gli tese il biglietto e spiegò che Pëtr Dmitric (come sembrava a Levin grande e importante Pëtr Dmitric, finora così trascurabile!) aveva promesso di venire in qualunque momento, che certo non si sarebbe arrabbiato e che perciò lo svegliasse subito. 

Il cameriere accettò, andò di sopra e fece entrare Levin nella sala di ricevimento. 

Levin sentiva, dietro la porta, il dottore che tossiva, camminava, si lavava e diceva qualcosa. Passarono circa tre minuti; a Levin parve che fosse passata un'ora. Non poteva più aspettare. 

- Pëtr Dmitric, Pëtr Dmitric - cominciò a dire, attraverso la porta aperta, con voce supplichevole. - In nome di Dio, scusatemi. Ricevetemi così come siete. Son già più di due ore. 

- Subito, subito! - rispondeva la voce, e Levin sentiva con meraviglia che il dottore diceva questo sorridendo. 

- Per un momento solo\ldots{} 

- Subito. 

Passarono ancora due minuti, prima che il dottore avesse infilato gli stivali e altri due minuti, prima che il dottore avesse messo il vestito e si fosse pettinato il capo. 

- Pëtr Dmitric! - cominciò di nuovo la voce pietosa di Levin, ma intanto venne fuori il dottore, vestito e pettinato. ``Non hanno coscienza queste persone - pensò Levin. - Pettinarsi, mentre noi moriamo''. 

- Buon giorno! - gli disse il dottore, dandogli la mano e quasi stuzzicandolo con la propria calma. - Non abbiate fretta. Be'? 

Cercando d'essere il più preciso possibile, Levin cominciò a raccontare i particolari inutili sullo stato della moglie, interrompendo di continuo il racconto con preghiere, perché il dottore andasse immediatamente con lui. 

- Ma non abbiate fretta. Voi non sapete nulla. Io certamente non sono necessario, ma ho promesso, e, vi assicuro, verrò. Ma fretta non ce n'è. Sedetevi per favore; non desiderate forse del caffè? 

Levin lo guardò, domandando con lo sguardo s'egli volesse prendersi giuoco di lui. Ma il dottore non pensava neppure a scherzare. 

- Lo so, lo so - disse il dottore, sorridendo - io stesso ho famiglia; ma noi mariti, in questi momenti, siamo le persone più pietose. Io ho una paziente che ha un marito il quale, quando comincia questo, se ne scappa nella scuderia. 

- Ma voi cosa pensate, Pëtr Dmitric? Credete che possa andar tutto bene? 

- Tutti i sintomi sono per un esito felice. 

- E allora, verrete subito? - disse Levin, guardando ostilmente il cameriere che entrava portando il caffè. 

- Fra un'oretta. 

- No, in nome di Dio! 

- Be', almeno lasciatemi bere il caffè. 

Il dottore si mise a sorbire il caffè. Tutti e due tacquero alquanto. 

- Però i turchi si battono risolutamente. Avete letto il comunicato di ieri? - disse il dottore, masticando un panino. 

- No, non posso! - disse Levin, saltando su. - Allora fra un quarto d'ora sarete da noi? 

- Fra mezz'ora. 

- Parola d'onore? 

Quando Levin tornò a casa, s'incontrò con la principessa e si accostarono insieme alla porta della stanza da letto. La principessa aveva le lacrime agli occhi, e le tremavano le mani. Visto Levin, lo abbracciò e si mise a piangere. 

- Ebbene, carissima Lizaveta Petrovna? - disse, afferrando per un braccio Lizaveta Petrovna ch'era uscita loro incontro con un viso luminoso e preoccupato. 

- Va bene - ella disse - persuadetela a coricarsi. Si sentirà meglio. 

Dal momento in cui s'era svegliato e aveva capito di che cosa si trattava, Levin s'era disposto a sopportare tutto quello cui andava incontro, senza riflettere, senza prevedere nulla, fermando tutti i pensieri e i sentimenti, con forza, senza agitare la moglie, al contrario, calmandola e sostenendone il coraggio. Senza concedersi neppure di pensare quello che sarebbe avvenuto, ma come sarebbe finito, giudicando dall'inchiesta che aveva svolto su quanto durava di solito la cosa, Levin, nella sua immaginazione, s'era preparato a pazientare e a tenere il cuore in mano per circa cinque ore, e questo gli sembrava possibile. Ma quando tornò dal dottore e vide di nuovo le sofferenze di lei, si mise a ripetere più spesso: ``Signore, perdona, aiuta'', a sospirare e a levare la testa in alto; e provò il terrore di non poter sopportare, d'essere costretto a piangere o a fuggire; tanto tormento provava. Ed era passata soltanto un'ora. Ma dopo quest'ora, ne passò ancora un'altra, poi ne passarono due, tre, tutte e cinque le ore, e le cose erano sempre allo stesso punto; e lui sopportava ancora, perché non c'era più niente da fare se non pazientare, pensando, ogni momento, d'essere giunto al limite della sopportazione e che il cuore, subito, da un momento all'altro, si sarebbe spezzato dalla pena. 

Ma passavano ancora minuti, ore e ancora ore, e i suoi sentimenti di pena e di angoscia crescevano e si tendevano ancora di più. 

Tutte quelle solite condizioni di vita, senza le quali non ci si può immaginare nulla, non esistevano più per Levin. Egli aveva perduto la nozione del tempo. A tratti i minuti, quei minuti in cui lei lo chiamava accanto a sé ed egli le teneva la mano sudata, che ora stringeva con una forza straordinaria, ora respingeva, gli sembravano ore; a tratti invece le ore gli sembravano minuti. Fu stupito quando Lizaveta Petrovna lo pregò di accendere una candela dietro il paravento e venne a sapere che erano le cinque di sera. Se gli avessero detto che erano soltanto le dieci del mattino, non si sarebbe stupito maggiormente. Dove fosse in quel momento, lo sapeva così poco come poco sapeva quando ciò sarebbe avvenuto. Vedeva il viso infiammato di lei, che a momenti era ansioso e sofferente, a momenti gli sorrideva, rasserenandolo. Vedeva anche la principessa, rossa, indaffarata, coi riccioli di capelli bianchi disfatti e con le lacrime ch'ella si affrettava a inghiottire, mordendosi le labbra; vedeva Dolly e il dottore che fumava delle grosse sigarette, e Lizaveta Petrovna dal viso deciso e forte e rasserenante, e il vecchio principe che passeggiava per la sala accigliato. Ma come essi venissero e uscissero, dove fossero, non lo sapeva. La principessa era ora col dottore nella stanza da letto, ora nello studio, dove si trovava una tavola imbandita; ora non c'era lei, ma Dolly. Poi Levin ricordava che l'avevano mandato chi sa dove. Una volta a trasportare un divano e una tavola. L'aveva fatto con cura, pensando che questo fosse necessario per lei, e soltanto dopo seppe che era stato preparato un letto per lui. Poi lo avevano mandato nello studio, dal dottore, a chiedere qualcosa. Il dottore aveva risposto, ma poi s'era messo a parlare sui disordini del consiglio di stato. Poi l'avevano mandato nella camera della principessa a prendere un'icona dalla cornice d'argento, e lui, insieme con la vecchia cameriera della principessa, s'era arrampicato su di un armadio per raggiungerla e aveva rotto la lampada, e la cameriera della principessa l'aveva tranquillizzato per la moglie e per la lampada, e lui aveva portato l'icona e l'aveva posta al capezzale di Kitty, ficcandola con cura dietro i guanciali. Ma dove, quando e perché avvenisse tutto questo, non lo sapeva. Non capiva neppure perché la principessa lo prendesse per mano, e, guardandolo pietosamente, lo pregasse di calmarsi, e Dolly lo persuadesse di mangiare un po' e lo portasse via dalla stanza, e perfino il dottore lo guardasse seriamente e con compassione, mentre gli offriva delle gocce. 

Egli sapeva e sentiva soltanto che quello che avveniva era simile a quello che s'era compiuto un anno prima nell'albergo della città di provincia, sul letto di morte di suo fratello Nikolaj. Ma quello era un dolore, questa una gioia. Ma sia quel dolore che questa gioia erano egualmente al di fuori di tutte le solite contingenze della vita ed erano, nella consuetudine della vita, come uno spiraglio attraverso il quale appariva qualcosa di ultraterreno. Ed egualmente con pena, con tormento avanzava quello che si compiva, ed egualmente in modo impenetrabile dinanzi a questo avvenimento di ordine superiore, l'anima si sollevava a un'altezza quale egli mai prima aveva neppure concepito e dove la ragione ormai non poteva tenerle dietro. 

``Signore, perdona e aiuta'' egli ripeteva continuamente, sentendo, malgrado il suo lungo e, in apparenza, completo allontanamento, di rivolgersi a Dio con la stessa confidenza e semplicità del tempo della fanciullezza e della prima gioventù. 

In tutto questo periodo ebbe due stati d'animo distinti. Uno, fuori della presenza di lei, col dottore, che fumava, una dopo l'altra, grosse sigarette e le spegneva contro l'orlo del portacenere pieno, con Dolly e col principe quando, mentre parlava del pranzo, di politica, della malattia di Mar'ja Petrovna, Levin a un tratto dimenticava completamente, per un poco, quello che accadeva, e si sentiva come risvegliato; e l'altro stato d'animo, in presenza di lei, al suo capezzale, quando il cuore voleva spezzarsi, eppure non si spezzava dalla pietà, ed egli pregava Dio continuamente. E ogni volta che un grido, giungendo dalla camera, lo tirava fuori da un momento di oblio, egli cadeva sempre in quello strano smarrimento che gli era piombato addosso il primo momento: ogni volta, sentito il grido, saltava su, correva a scolparsi, si ricordava per via che non era colpevole, e voleva proteggere, aiutare. Ma guardando lei, di nuovo vedeva che non si poteva darle aiuto, e cadeva nel terrore e diceva: ``Signore, perdona e aiuta''. E quanto più tempo passava, più forti si facevano tutti e due gli stati d'animo: tanto più calmo egli diveniva, dimenticandola completamente, lontano da lei, e tanto più tormentose diventavano e le sofferenze di lei e la sensazione d'impotenza di fronte ad esse. Egli saltava su, voleva correre via, in qualche parte, e correva da lei. 

A volte, quand'ella lo chiamava ancora e poi ancora, egli accusava lei. Ma, vedendo quel viso docile e sorridente e sentendo le parole ``Ti sto tormentando'', egli ne accusava Dio, e, ricordandosi di Dio, lo pregava subito di perdonare e di avere pietà. 

\capitolo{XV}\label{xv-6} 

Non sapeva se era tardi, se era presto. Le candele stavano tutte per spegnersi. Dolly era stata allora allora nello studio e aveva proposto al dottore di coricarsi un po'. Levin sedeva, ascoltando i racconti del dottore su di un ciarlatano ipnotizzatore e guardava la cenere della sua sigaretta. Era in un momento di tregua, come smemorato. Aveva completamente dimenticato quel che accadeva, adesso. Ascoltava il racconto del dottore e lo seguiva. A un tratto echeggiò un grido, che non era simile a nulla. Il grido era così terribile che Levin non saltò neanche su, ma, senza respirare, guardò il dottore con uno spavento interrogativo. Il dottore piegò il capo da un lato, ascoltando, e sorrise con approvazione. Tutto era così fuori dell'ordinario, che Levin non si stupiva più di nulla. ``Probabilmente deve essere così'' pensò e continuò a stare seduto. Di chi era quel grido? Saltò su, entrò di corsa in punta di piedi nella stanza da letto, sorpassò Lizaveta Petrovna, la principessa e si pose al suo posto, al capezzale. Il grido era finito, ma adesso qualcosa era cambiato. Cosa, non lo vedeva e non lo capiva e non voleva capirlo. Ma lo vedeva dal viso di Lizaveta Petrovna: il viso di Lizaveta Petrovna era severo e pallido e sempre egualmente deciso, sebbene le mascelle le tremassero un poco e i suoi occhi fossero diretti con fissità su Kitty. Il volto di Kitty, infiammato e sfinito, con una ciocca di capelli appiccicata al viso sudato, era rivolto verso di lui e cercava il suo sguardo. Le mani alzate chiedevano le sue mani. Afferrate con le mani sudate le mani fredde di lui, ella si mise a premerle contro il proprio viso. 

- Non te ne andare, non te ne andare! Io non ho paura, non ho paura! - diceva in fretta. - Mamma, toglietemi gli orecchini. Mi dànno noia. Tu non hai paura. Presto, presto, Lizaveta Petrovna. 

Ella parlava in fretta, molto in fretta e voleva sorridere. Ma, a un tratto, il suo viso si alterò, ella lo respinse da sé. 

- No, è tremendo! Morirò, morirò. Va', va'! - ella gridò e sentì di nuovo quello stesso grido che non era simile a nulla. 

Levin si afferrò la testa tra le mani e fuggì dalla camera. 

- Nulla, nulla, va tutto bene - gli disse dietro Dolly. 

Ma, qualunque cosa dicessero, egli sapeva che ormai tutto era perduto. Col capo contro lo stipite della porta, stava in piedi nella stanza attigua e sentiva uno stridio, un mugghio da lui non mai prima sentito, e sapeva che gridava quella cosa informe che prima era Kitty. Il bambino non lo desiderava più, già da tempo. Adesso odiava quell'essere. Adesso non desiderava neanche più la vita di lei, desiderava solo la fine di quelle orribili sofferenze. 

- Dottore, cos'è mai questo? cos'è mai questo? Dio mio! - disse, afferrando per il braccio il dottore che era entrato. 

- Finisce - disse il dottore. E il viso del dottore era così serio, mentre diceva questo, che Levin capì ``finisce'' nel senso di ``muore''. 

Fuori di sé, entrò di corsa nella stanza da letto. La prima cosa che vide fu il viso di Lizaveta Petrovna. Esso era ancora più agitato e più severo. Il viso di Kitty non c'era più. Nel posto dov'era prima, c'era qualcosa di mostruoso e per l'aspetto di tensione e per il suono che ne usciva. Egli cadde con la testa sul legno del letto, sentendo che il cuore gli si spezzava. L'orribile grido non finiva, s'era fatto ancora più orribile, ma poi, come se fosse giunto al limite estremo dell'orrore, si calmò a un tratto. Levin non credeva al proprio udito, ma non si poteva dubitare: il grido s'era calmato e si sentiva un silenzioso affaccendarsi, un fruscio, un respirare ansioso, e la voce di lei felice e affannata, viva e tenera che pronunciava piano: ``È finito''. 

Egli sollevò il capo. Abbassate sulla coperta le braccia senza forza, straordinariamente bella e calma, ella lo guardava senza parole e voleva, ma non poteva sorridere. 

E a un tratto da quel mondo misterioso e orribile, estraneo, in cui aveva vissuto in quelle ventidue ore, Levin si sentì trasportato in un attimo nel mondo solito di prima, ma splendente, adesso, d'una tale luce nuova di felicità, ch'egli non la sopportò. Le corde tese si strapparono tutte. Singhiozzi e lacrime di gioia, ch'egli non aveva in nessun modo preveduto, si sollevarono in lui con una forza tale, agitando tutto il suo corpo, che per lungo tempo gli impedirono di parlare. 

Caduto in ginocchio davanti al letto, egli teneva dinanzi alle labbra la mano della moglie e la baciava, e questa mano con un debole movimento delle dita rispondeva ai suoi baci. E intanto, là, ai piedi del letto, nelle abili mani di Lizaveta Petrovna, come la fiammella d'una lampada, oscillava la vita d'un essere umano che prima non c'era mai stato e che avrebbe vissuto e creato degli altri esseri nello stesso modo, con lo stesso diritto, con la stessa importanza di sé. 

- Vivo! vivo! È pure un maschio! Non vi agitate! - Levin sentì la voce di Lizaveta Petrovna, che batteva con la mano tremante la schiena del bambino. 

- Mamma, è vero? - disse la voce di Kitty. Le risposero i singhiozzi della principessa. 

E in mezzo al silenzio, come una risposta indubitabile alla domanda della madre, si sentì una voce affatto diversa da tutte le voci che parlavano nella camera. Era il grido ardito, temerario, che non voleva considerare nulla, d'un nuovo essere umano, che non si capiva donde fosse venuto fuori. 

Prima, se avessero detto a Levin che Kitty era morta e che lui era morto insieme con lei, e che avevano per bambini gli angeli, e che Dio era lì dinanzi a loro, non si sarebbe stupito di nulla; ma adesso, tornato nel mondo della realtà, faceva grandi sforzi per capire ch'ella era viva, sana e che l'essere che strideva in modo così disperato era suo figlio. Kitty era viva, le sofferenze erano finite. Ed egli era inesprimibilmente felice. Questo lo capiva e ne era pienamente soddisfatto. Ma il bambino? Donde veniva, perché, chi era? Non poteva in nessun modo abituarsi a questo pensiero. Gli sembrava qualcosa di superfluo, una sovrabbondanza a cui per lungo tempo non poté abituarsi. 

\capitolo{XVI}\label{xvi-6} 

Dopo le nove il vecchio principe, Sergej Ivanovic e Stepan Arkad'ic erano dai Levin e, dopo aver parlato della puerpera, si misero a discorrere anche di argomenti estranei. Levin li ascoltava, ricollegando involontariamente a questi discorsi il passato, cioè quello che era stato fatto fino a quella mattina, ricordava anche se stesso, così come era stato fino a quell'avvenimento. Erano passati proprio cent'anni da allora. Si sentiva ad un'altezza irraggiungibile, dalla quale discendeva con sforzo, per non offendere quelli con cui parlava. Parlava e pensava continuamente alla moglie, ai particolari del suo stato presente e al figlio, all'esistenza del quale cercava di abituare il proprio pensiero. Tutto il mondo femminile, che per lui da quando s'era sposato aveva acquistato un'importanza nuova, sconosciuta, ora si sollevava così in alto nella sua concezione, ch'egli non poteva abbracciarlo con la mente. Sentiva una conversazione sul pranzo del giorno prima al club e pensava: ``Cosa starà facendo adesso? si sarà addormentata? Come sta? cosa pensa? Grida mio figlio Dmitrij?''. E nel bel mezzo della conversazione, nel mezzo di una frase saltò su e uscì dalla stanza. 

- Mandami a dire se si può andare da lei - disse il principe. 

- Va bene, subito - rispose Levin e, senza fermarsi, andò da lei. 

Ella non dormiva e discorreva piano con la madre, facendo progetti per il prossimo battesimo. 

Tutta in ordine, pettinata, con una cuffietta elegante con qualcosa d'azzurro, le braccia stese sulla coperta, giaceva supina, e, incontrato lui con lo sguardo, con lo sguardo l'attirò a sé. I suoi occhi, già luminosi, divennero sempre più splendenti a misura ch'egli si avvicinava a lei. Sul viso c'era quello stesso passaggio dal terreno all'ultraterreno che c'è nel viso dei morenti; ma là c'è un distacco, qui era come un ritrovarsi. Di nuovo un'agitazione, simile a quella ch'egli aveva provato al momento del parto, gli venne al cuore. Ella gli prese la mano e chiese se avesse dormito. Lui non poteva rispondere e si voltava dall'altra parte, convinto della propria debolezza. 

- Io invece mi sono assopita, Kostja! - ella disse. - E adesso sto così bene. 

Lo guardava, ma a un tratto la sua espressione cambiò. 

- Datemelo - ella disse, sentendo il vagito del bambino. - Date qua, Lizaveta Petrovna, che anche lui lo veda. 

- Su, ecco, che il papà lo veda - disse Lizaveta Petrovna, sollevando e tendendo qualcosa di rosso, strano e oscillante. - Aspettate, prima ci sistemiamo - e Lizaveta Petrovna poggiò quella cosa rossa e oscillante sul letto, e si mise a sfasciare e a fasciare il bambino, sollevandolo e rivoltolandolo con un dito e cospargendolo di qualcosa. 

Levin, guardando quel minuscolo essere pietoso, faceva sforzi vani per trovare nell'animo suo il segno di un qualche sentimento paterno verso di lui. Sentiva per lui soltanto disgusto. Ma quando lo spogliarono e guizzarono i braccini sottili, i piedini color zafferano, anch'essi coi ditini e perfino con l'alluce che si distingueva dagli altri, e quando vide come Lizaveta Petrovna premeva, quasi fossero morbide molle, quei braccini che si protendevano, rinchiudendoli in panni di tela, gli venne una tale pietà per quell'essere e un tale terrore ch'ella gli facesse male, che la trattenne per un braccio. 

Lizaveta Petrovna si mise a ridere. 

- Non abbiate paura, non abbiate paura! 

Quando il bambino fu sistemato e trasformato in una pupattola dura, Lizaveta Petrovna lo dondolò, come inorgoglita del proprio lavoro, e si fece da parte, perché Levin potesse vedere il figlio in tutta la sua bellezza. 

Kitty, senza staccar gli occhi, guardava là, di sbieco. 

- Date, date! - disse, e si sollevò perfino. 

- Che fate, Katerina Aleksandrovna? Non si possono fare dei movimenti simili! Aspettate, ve lo darò io. Ecco che faremo vedere a papà, che bel giovanotto siamo! 

E Lizaveta Petrovna sollevò verso Levin, su di una sola mano (l'altra sosteneva solo con le dita la nuca oscillante), quello strano essere rosso che ciondolava e nascondeva il capo tra gli orli delle fasce. Ma c'era anche il naso, e c'eran gli occhi che guardavano di traverso e le labbra che succhiavano. 

- Un bellissimo bambino! - disse Lizaveta Petrovna. 

Levin sospirò con pena. Quel bellissimo bambino gli ispirava soltanto disgusto e pena. Era tutt'altro il sentimento che si aspettava. 

Egli si voltò mentre Lizaveta Petrovna lo accomodò al petto inesperto. 

A un tratto un riso gli fece sollevare il capo. Era Kitty che s'era messa a ridere. Il bambino s'era attaccato al petto. 

- Su, basta, basta! - diceva Lizaveta Petrovna, ma Kitty non lo lasciò andare. Egli si addormentò fra le sue braccia. 

- Guarda, adesso - disse Kitty, voltando verso di lui il bambino perché egli potesse vederlo. Il visino da vecchio, a un tratto, si corrugò ancora di più e il bimbo starnutì. 

Sorridendo e trattenendo appena le lacrime per l'emozione, Levin baciò la moglie e uscì dalla camera buia. 

Quello ch'egli provava per quel piccolo essere era proprio tutt'altra cosa da quello che si aspettava. Non c'era nulla di allegro e di gioioso in questo sentimento; al contrario, un nuovo senso di paura. Era la coscienza di un nuovo campo di vulnerabilità. E questa coscienza era così tormentosa nei primi tempi, il terrore che quell'essere impotente soffrisse era così forte, che proprio per questo non avvertiva lo strano sentimento di spensierata gioia e perfino di orgoglio ch'egli aveva provato proprio nel momento in cui il bambino aveva starnutito. 

\capitolo{XVII}\label{xvii-6} 

Gli affari di Stepan Arkad'ic erano in pessimo stato. 

Il denaro del legname per i due terzi era stato speso, e, detrattone il dieci per cento, egli aveva preso dal compratore, quasi tutto in anticipo, l'ultimo terzo. Il compratore non dava più denaro tanto più che quell'inverno Dar'ja Aleksandrovna, dichiarando per la prima volta in modo esplicito i propri diritti sul patrimonio, aveva rifiutato di quietanzare sul contratto la ricevuta del denaro per l'ultimo terzo del legname. Tutto lo stipendio se ne andava per le spese di casa e per il pagamento dei piccoli debiti insopportabili. Denaro non ce n'era affatto. 

Questo era spiacevole, disgustoso e non doveva prolungarsi tanto, secondo Stepan Arkad'ic. La ragione di questo, secondo il suo criterio, consisteva nel fatto ch'egli riceveva uno stipendio troppo basso. Il posto che occupava era stato, evidentemente, buono cinque anni prima, ma adesso non era più la stessa cosa. Petrov, come direttore di banca, riceveva 12.000 rubli; Sventickij, membro d'una società, ne riceveva 17.000; Mitin, dopo aver fondato una banca, ne ricavava 50.000. ``Evidentemente io mi sono addormentato e gli altri mi hanno dimenticato'' pensava di sé Stepan Arkad'ic. E cominciò a prestare ascolto, a guardarsi in giro, e, verso la fine dell'inverno, notò un posto molto buono e vi mosse all'attacco, prima da Mosca, per mezzo delle zie, degli zii, degli amici e, poi, quando la faccenda fu a buon punto, in primavera andò lui stesso a Pietroburgo. Era uno di quei posti di cui ce n'è adesso, di tutte le misure, dai 1.000 ai 50.000 rubli all'anno di stipendio, che erano disponibili più di quanti non ce ne fossero prima, posti comodi, venali: il posto di membro della commissione dell'agenzia unita del bilancio mutuo-credito delle ferrovie e degli istituti bancari meridionali. Questo posto, come tutti i posti simili, esigeva così enormi conoscenze e attività, che era difficile riunirle in una sola persona. E poiché l'uomo che riunisse queste qualità non c'era, era meglio allora che il posto lo occupasse un uomo onesto piuttosto che un disonesto. E Stepan Arkad'ic era non solo un onest'uomo (senza accento), ma era veramente un uomo onesto (con l'accento), con quel significato particolare che questa parola assume a Mosca, quando si dice uomo politico onesto, scrittore onesto, rivista onesta, istituto onesto, tendenza onesta, e che significa non soltanto che la persona o l'istituto non sono disonesti, ma che sono anche capaci, all'occasione, di scagliare una frecciata al governo. Stepan Arkad'ic frequentava a Mosca gli ambienti dove questa parola era introdotta, era considerato proprio là un uomo onesto e perciò aveva più diritti degli altri a quel posto. 

Quel posto dava dai sette ai diecimila rubli all'anno e Oblonskij poteva occuparlo senza lasciare il proprio posto statale. Esso dipendeva da due ministeri, da una signora e da due ebrei, e tutte queste persone, benché fossero già ben disposte, Stepan Arkad'ic doveva vederle a Pietroburgo. Inoltre, Stepan Arkad'ic aveva promesso a sua sorella Anna di ottenere da Karenin una risposta decisiva sul divorzio. E, ottenuti cinquanta rubli da Dolly, partì per Pietroburgo. 

Sedendo nello studio di Karenin e ascoltando il piano per lo studio delle cause del cattivo stato delle finanze russe, Stepan Arkad'ic aspettava soltanto ch'egli finisse, per cominciare a parlare del proprio affare personale e di Anna. 

- Sì, è molto giusto - egli disse, quando Aleksej Aleksandrovic, toltosi il pince-nez, senza il quale adesso non poteva leggere, guardò interrogativamente l'ex-cognato - è molto giusto nei particolari, ma tuttavia il principio del nostro tempo è la libertà. 

- Sì, ma io espongo un altro principio, che abbraccia il principio della libertà - disse Aleksej Aleksandrovic, accentuando le parole ``che abbraccia'' e mettendo di nuovo il pince-nez, per leggere di nuovo quel passo in cui ciò era scritto. 

E, sfogliato il manoscritto steso in bella grafia, dai margini enormi, Aleksej Aleksandrovic lesse di nuovo il passo convincente. 

- Io non voglio il sistema protezionistico per l'utilità dei privati, ma per il bene comune, per le classi inferiori e per quelle superiori allo stesso modo - egli diceva, guardando Oblonskij al di sopra del pince-nez. - Ma loro non possono capirlo, loro sono occupati soltanto dagli interessi personali e corrono dietro alle frasi. 

Stepan Arkad'ic sapeva che quando Karenin cominciava a parlare di quello che pensano e fanno loro, quegli stessi che non volevano accogliere i suoi piani ed erano tutto il male della Russia, allora si era vicini alla fine, e perciò adesso rinunciò volentieri al principio di libertà e consentì pienamente. Aleksej Aleksandrovic tacque, sfogliando pensieroso il manoscritto. 

- Ah, a proposito - disse Stepan Arkad'ic - volevo pregarti, se ne hai l'occasione, quando vedrai Pomorskoj, di dirgli una parola nel senso che io desidererei molto di occupare il posto che si è reso vacante di membro della commissione dell'agenzia unita del bilancio mutuo-credito delle ferrovie meridionali. 

Per Stepan Arkad'ic il nome di questo posto, tanto vicino al suo cuore, era ormai abituale, ed egli lo pronunciava in fretta senza sbagliarsi. 

Aleksej Aleksandrovic chiese in che consisteva l'attività di questa nuova commissione, e si fece pensieroso. Esaminava se nell'attività di questa commissione non ci fosse qualcosa di contrario ai propri piani. Ma siccome l'attività della nuova istituzione era molto complessa e i suoi piani abbracciavano un campo molto vasto, non poteva esaminarlo subito e, togliendo il pince-nez, disse: 

- Senza dubbio, posso dirglielo; ma perché ci tieni proprio a occupare questo posto? 

- Lo stipendio è buono, fino a novemila rubli, e le mie condizioni\ldots{} 

- Novemila - ripeté Aleksej Aleksandrovic e aggrottò le sopracciglia. 

L'alta cifra di questo stipendio gli ricordava che, da questo lato, l'attività eventuale di Stepan Arkad'ic era contraria al senso principale dei propri piani che tendevano sempre all'economia. 

- Io credo, e ci ho scritto su un memoriale, che nel nostro tempo questi stipendi enormi siano l'indice della falsa assiette economica della nostra amministrazione. 

- Ma come vorresti dire? - disse Stepan Arkad'ic. - Su, ammettiamo che un direttore di banca riceva diecimila rubli; significa che li vale. Oppure che un ingegnere ne riceva ventimila. È lavoro vivo, comunque tu la pensi. 

- Io considero che lo stipendio sia il pagamento di una merce e che esso debba sottostare alla legge della domanda e dell'offerta. Se invece il criterio dello stipendio si allontana da questa legge, come, ad esempio, nel caso di due ingegneri laureati dallo stesso istituto, tutti e due con le stesse nozioni e capacità, uno riceve quattromila rubli e l'altro si accontenta di duemila; o di direttori di banca che assumono con uno stipendio enorme degli studentelli di legge, degli ussari che non hanno nessuna nozione particolare, allora concludo che lo stipendio si fissa non secondo la legge della domanda e dell'offerta, ma direttamente, per compiacere le persone. E qui c'è un abuso, grave per se stesso e che si ripercuote dannosamente sul servizio dello stato. Io considero\ldots{} 

Stepan Arkad'ic si affrettò ad interrompere il cognato. 

- Sì, ma convieni che si apre un'istituzione nuova indubbiamente utile. Checché tu dica è un lavoro vivo! Apprezzano in modo particolare che il lavoro sia condotto con onestà - disse Stepan Arkad'ic con intenzione. 

Ma il significato moscovita di ``onestà'' era incomprensibile ad Aleksej Aleksandrovic. 

- L'onesta è soltanto una qualità negativa - disse. 

- Tuttavia mi farai gran piacere - disse Stepan Arkad'ic - se vorrai dire una parola a Pomorskoj. Così nel discorso\ldots{} 

- Eppure questo dipende maggiormente da Bolgarinov, sembra - disse Aleksej Aleksandrovic. 

- Bolgarinov per parte sua è del tutto consenziente - disse Stepan Arkad'ic, arrossendo. 

Stepan Arkad'ic arrossiva al ricordo di Bolgarinov, perché quello stesso giorno di mattina era stato dall'ebreo Bolgarinov e questa visita gli aveva lasciato una impressione spiacevole. Stepan Arkad'ic sapeva bene che il lavoro che voleva intraprendere era un lavoro nuovo, vivo e onesto; ma quando quella mattina Bolgarinov, evidentemente apposta, lo aveva fatto attendere due ore con gli altri sollecitatori nella sala di ricevimento, a un tratto, si era sentito a disagio. 

Ch'egli si sentisse a disagio perché lui, principe Oblonskij, discendente di Rjurik, aveva aspettato due ore nella sala d'aspetto di un giudeo, o perché, per la prima volta nella vita, non aveva seguito l'esempio degli antenati col servire il governo, e aveva voluto entrare in un'amministrazione nuova, certo è che si era sentito molto a disagio. In quelle due ore di attesa da Bolgarinov Stepan Arkad'ic, passeggiando agile per la sala d'aspetto, accomodandosi le fedine, mettendosi a discorrere con gli altri sollecitatori ed escogitando un giuoco di parole col quale poter dire come egli avesse aspettato dal giudeo, aveva nascosto con cura agli altri e a se stesso la sensazione provata. 

Ma tutto quel tempo s'era sentito a disagio e irritato, senza saper lui stesso perché: forse perché dal giuoco di parole non veniva fuori nulla o per qualcos'altro. Quando poi alla fine Bolgarinov l'aveva ricevuto con eccessiva cortesia, evidentemente trionfando dell'umiliazione inflittagli, e gli aveva quasi detto di no, Stepan Arkad'ic s'era affrettato a dimenticarlo il più presto possibile. E soltanto adesso, ricordandosene, s'era fatto rosso. 

\capitolo{XVIII}\label{xviii-6} 

- Adesso ho ancora una cosa da dirti, e tu sai quale\ldots{} a proposito di Anna - disse Stepan Arkad'ic dopo un certo silenzio e dopo avere scosso via da sé quell'impressione spiacevole. 

Non appena Oblonskij ebbe pronunciato il nome di Anna il viso di Aleksej Aleksandrovic cambiò completamente: in luogo dell'animazione di prima espresse una stanchezza mortale. 

- Che cosa volete da me? - disse, voltandosi sulla poltrona e chiudendo il pince-nez. 

- Una decisione, una qualsiasi decisione, Aleksej Aleksandrovic. Mi rivolgo a te adesso - ``non come al marito offeso'' voleva dire Stepan Arkad'ic, ma, temendo di rovinare la faccenda, cambiò - non come all'uomo di stato - il che risultò fuor di proposito - ma semplicemente all'uomo, all'uomo buono, al cristiano. Tu devi aver pietà di lei - egli disse. 

- Cioè, in che cosa propriamente - disse piano Karenin. 

- Sì, aver pietà di lei. Se tu la vedessi come l'ho vista io (ho passato tutto l'inverno con lei), ne avresti pena. La sua situazione è orribile, proprio orribile. 

- Mi sembrava - rispose Aleksej Aleksandrovic, con voce più sottile, quasi stridente - che Anna Arkad'evna avesse tutto quello che lei proprio aveva voluto. 

- Ah, Aleksej Aleksandrovic in nome di Dio, niente recriminazioni! Quel ch'è passato è passato, e tu sai quello che lei desidera e attende: il divorzio. 

- Ma io ritenevo che Anna Arkad'evna rinunciasse al divorzio nel caso che io esigessi l'obbligo di lasciarmi il figlio. Io ho risposto proprio così e pensavo che questa faccenda fosse finita. Io la ritengo finita - stridette Aleksej Aleksandrovic. 

- Ma in nome di Dio, non ti accalorare - disse Stepan Arkad'ic, toccando il cognato su un ginocchio. - La faccenda non è finita. Se tu mi permetti di ricapitolare, le cose stavano così: quando vi siete separati, tu sei stato grande, generoso come più non si poteva essere; le davi tutto, la libertà, perfino il divorzio. Lei ha apprezzato ciò. No, non credere. L'ha proprio apprezzato. Fino al punto che, in quel primo momento, sentendo la propria colpa dinanzi a te, non ha riflettuto e non poteva riflettere. Ha rifiutato tutto. Ma la realtà, il tempo hanno dimostrato che la situazione è tormentosa e impossibile. 

- La vita di Anna Arkad'evna non può interessarmi - interruppe Aleksej Aleksandrovic, sollevando le sopracciglia. 
\enlargethispage*{1\baselineskip}

- Permettimi di non crederci - obiettò dolcemente Stepan Arkad'ic. - La sua situazione è tormentosa per lei e di nessun vantaggio per gli altri. Ella l'ha meritata, tu dirai. Lei lo sa e non ti chiede nulla; dice apertamente che non osa chieder nulla. Ma io, noi tutti parenti, tutti quelli che le vogliono bene, ti preghiamo, ti supplichiamo. Perché si tormenta? Chi se ne avvantaggia? 

- Ma scusate, voi, a quanto pare, mi mettete nella situazione di un accusato - pronunciò Aleksej Aleksandrovic. 

- Ma no, ma no, per nulla affatto, comprendimi - disse Stepan Arkad'ic, toccandolo di nuovo, nel braccio, come se fosse convinto che questo contatto avrebbe raddolcito il cognato. - Io dico solo una cosa: la sua situazione è tormentosa e può essere alleviata da te, e tu non ci perderai nulla. Io ti accomoderò tutto in modo che non te ne accorgerai. Perché l'hai promesso. 

- La promessa era stata fatta prima. E io credevo che la questione del figlio decidesse la cosa. Inoltre speravo che Anna Arkad'evna avrebbe avuto sufficiente generosità\ldots{} - pronunciò a stento, pallido, con le labbra che gli tremavano, Aleksej Aleksandrovic. 

- Lei affida tutto alla tua generosità. Ti prega, ti supplica d'una cosa sola: di trarla da questa situazione impossibile in cui si trova. Ella ormai non chiede il figlio. Aleksej Aleksandrovic, tu sei un uomo buono. Mettiti al posto suo per un attimo. La questione del divorzio per lei, nella sua situazione, è una questione di vita o di morte. Se tu non avessi promesso prima, lei si sarebbe adattata alla sua situazione, avrebbe vissuto in campagna. Ma tu hai promesso, lei ti ha scritto ed è andata a vivere a Mosca. Ed ecco a Mosca, dove ogni incontro per lei è una coltellata al cuore, vive da sei mesi, aspettando la decisione da un giorno all'altro. E questo è come tenere un condannato a morte per sei mesi col laccio al collo, promettendo ora la morte, ora la grazia. Abbi pietà di lei, e io mi assumo di accomodar tutto così. Vos scrupules\ldots{} 

- Io non parlo di questo\ldots{} - lo interruppe con disgusto Aleksej Aleksandrovic. - Ma forse, io ho promesso quello che non avevo il diritto di promettere. 

- Allora tu rifiuti quello che hai promesso? 

- Io non ho mai respinto l'effettuazione di ciò che è possibile, ma desidero avere il tempo di riflettere fino a che punto sia possibile quello che è stato promesso. 

- No, Aleksej Aleksandrovic - cominciò a dire Oblonskij, saltando su - io non voglio crederci. Lei è così infelice, come solo può essere infelice una donna, e tu non puoi rifiutare una tale\ldots{} 

- Per quanto è possibile quello che è stato promesso. Vous professez d'être un libre penseur. Ma io, come persona credente, non posso agire contro la legge cristiana in una cosa tanto importante. 

- Ma nelle comunità cristiane e da noi, a quanto io sappia, il divorzio è ammesso - disse Stepan Arkad'ic. - Il divorzio è ammesso anche dalla nostra Chiesa. E noi vediamo\ldots{} 

- Ammesso, ma non in questo senso. 

- Aleksej Aleksandrovic, io non ti riconosco - disse Oblonskij, dopo aver taciuto un po'. - Non sei tu (e forse noi non l'abbiamo apprezzato?) che hai perdonato tutto e, mosso proprio dal sentimento cristiano, eri pronto a sacrificare tutto? Tu stesso hai detto: dare la tunica, quando ti han preso la camicia, e adesso\ldots{} 

- Io vi prego - cominciò a dire a un tratto Aleksej Aleksandrovic con voce stridula, alzandosi in piedi, pallido e con la mascella che gli tremava - vi prego di smettere, di smettere\ldots{} questo discorso. 
\enlargethispage*{1\baselineskip}

- Ah, no! Via, scusa, scusami se ti ho addolorato - cominciò a dire, sorridendo confuso Stepan Arkad'ic, tendendo la mano - ma tuttavia, come ambasciatore, ho riferito solo la mia ambasciata. 

Aleksej Aleksandrovic dette la mano, si fece pensieroso e pronunciò: 

- Devo riflettere e informarmi. Domani l'altro vi darò la risposta decisiva - disse, dopo aver riflettuto. 

\capitolo{XIX}\label{xix-6} 

Stepan Arkad'ic voleva andarsene, quando Kornej venne ad annunciare: 

- Sergej Alekseevic! 

``Chi è Sergej Alekseevic?'' stava per chiedere Stepan Arkad'ic, ma se ne ricordò immediatamente. 

- Ah, Serëza! - egli disse. ``Sergej Alekseevic. Io credevo fosse un capodivisione. Anna m'aveva appunto pregato di vederlo'' ricordò. 

E ricordò l'espressione timida, pietosa con cui, lasciandolo andar via, aveva detto: ``Tuttavia lo vedrai. Cerca di sapere dettagliatamente dove si trova, chi lo cura. E, Stiva\ldots{} se è possibile! Perché è possibile?''. Stepan Arkad'ic aveva capito quel che significava questo: ``se fosse possibile!'': se fosse possibile fare il divorzio in modo da darle il figlio\ldots{} Adesso Stepan Arkad'ic vedeva che non c'era neanche da pensarci, tuttavia fu contento di vedere il nipote. 

Aleksej Aleksandrovic ricordò al cognato che al ragazzo non parlavano mai della madre e che lo pregava di non accennarvi neanche con una parola. 

- È stato molto male dopo quell'incontro con la madre, che non avevamo preveduto - disse Aleksej Aleksandrovic. - Abbiamo perfino temuto per la sua vita. Ma una cura appropriata e i bagni di mare d'estate gli han ridato la salute, e ora, per consiglio del dottore, l'ho messo a scuola. In effetti l'influsso dei compagni ha avuto una buona azione su di lui, ed egli è in perfetta salute e studia bene. 

- Che bel giovane che è diventato! E non è Serëza, ma un vero e proprio Sergej Alekseevic! - disse, sorridendo, Stepan Arkad'ic, guardando il bel ragazzo dalle spalle larghe che entrava ardito e franco con una giacchetta turchina e i pantaloni lunghi. Il ragazzo aveva un aspetto sano e allegro. S'inchinò allo zio come a un estraneo, ma poi, riconosciutolo, arrossì e, come offeso e urtato da qualcosa, si voltò frettolosamente in là. Il ragazzo si avvicinò al padre e gli diede un biglietto in cui erano segnati i voti ricevuti a scuola. 

- Be', è passabile - disse il padre - puoi andare. 

- È dimagrito e s'è slanciato, ha finito d'essere un bambino, è diventato un monello; mi piace - disse Stepan Arkad'ic. - Ma ti ricordi di me? 

Il ragazzo si voltò rapido a guardare il padre. 

- Me ne ricordo, mon oncle - rispose, dopo aver guardato lo zio, e di nuovo ficcò gli occhi a terra. 
\enlargethispage*{1\baselineskip}

Lo zio chiamò a sé il ragazzo e lo prese per mano. 

- Be', come vanno le cose? - disse, desiderando chiacchierare, pur senza sapere cosa dire. 

Il ragazzo, arrossendo senza rispondere, tirava fuori dalla mano dello zio la propria. Non appena Stepan Arkad'ic lasciò andare la mano, come un uccello messo in libertà, dopo aver guardato il padre, uscì dalla stanza a passo svelto. 

Era passato un anno da che Serëza aveva visto sua madre per l'ultima volta. Da quel tempo non aveva mai più sentito parlare di lei. E in quello stesso anno era stato messo a scuola e aveva imparato a conoscere e a voler bene ai compagni. Quei sogni e il ricordo della madre, che, dopo l'incontro con lei, lo avevano fatto ammalare, adesso non l'occupavano più. Quando venivano, egli li scacciava con cura da sé, ritenendoli vergognosi, propri soltanto delle bambine, non di un ragazzo e d'un ragazzo che va a scuola. Sapeva che fra il padre e la madre c'era stato un litigio che li aveva separati, sapeva di esser destinato a rimanere col padre, e cercava di abituarsi a questo pensiero. 

Nel vedere lo zio che assomigliava alla madre, aveva provato una sensazione spiacevole, perché questo suscitava in lui quegli stessi ricordi ch'egli riteneva vergognosi. La cosa era stata tanto più spiacevole, in quanto da alcune parole che aveva sentito, aspettando, vicino alla porta dello studio, e in particolare dall'espressione del viso del padre e dello zio, indovinava che fra di loro si era dovuto parlare della madre. E per non giudicare il padre con il quale viveva e da cui dipendeva, e soprattutto per non abbandonarsi alla sensibilità che egli riteneva così umiliante, Serëza aveva cercato di non guardare quello zio, ch'era venuto a sconvolgere la sua calma, e di non pensare a quello ch'egli ricordava. 

Ma quando Stepan Arkad'ic, ch'era uscito dopo di lui, vistolo sulla scala, lo chiamò accanto a sé e domandò come passasse il tempo fra la scuola e le lezioni, Serëza, fuori della presenza del padre, si mise a parlare con lui. 

- Adesso qui abbiamo la ferrovia - disse, rispondendo alla sua domanda. - È così, vedete: due si siedono su di una panca. Sono i passeggeri. E uno si mette in piedi sempre sulla panca. E tutti si attaccano. Si può fare con le braccia, si può fare con le cinture, e si fanno andare per tutte le sale. Le porte si aprono già da prima. Eh, qui è molto difficile fare il conduttore! 

- È quello che sta in piedi? - domandò Stepan Arkad'ic, sorridendo. 

- Sì, qui ci vuole coraggio e sveltezza, specialmente quando si fermano tutt'a un tratto o quando qualcuno cade. 

- Già, non è mica uno scherzo - disse Stepan Arkad'ic, esaminando con tristezza quegli occhi animati che ricordavano la madre, adesso non più infantili, non più del tutto innocenti. E, sebbene avesse promesso ad Aleksej Aleksandrovic di non parlare di Anna, non resistette. 

- E ti ricordi di tua madre? - domandò a un tratto. 

- No, non me ne ricordo - pronunciò svelto Serëza e, fattosi rosso di fuoco, chinò gli occhi a terra. E lo zio non poté cavarne più nulla. 

L'istitutore slavo, dopo mezz'ora, trovò il suo allievo sulla scala, e per lungo tempo non poté capire se fosse irritato o piangesse. 
\enlargethispage*{1\baselineskip}

- Ma vi siete forse fatto male quando siete caduto? - disse l'istitutore. - Lo dicevo io ch'era un giuoco pericoloso. E bisogna dirlo al direttore. 

- Se mi fossi anche fatto male, nessuno se ne sarebbe accorto. Questo è sicuro. 

- E allora cosa mai? 

- Lasciatemi!\ldots{} Mi ricordo, non mi ricordo\ldots{} Che gliene importa? Perché devo ricordarmene? Lasciatemi in pace! - disse rivolto ormai non più all'istitutore, ma a tutto il mondo. 

\capitolo{XX}\label{xx-6} 

Stepan Arkad'ic, come sempre del resto, non passava oziosamente il tempo a Pietroburgo. A Pietroburgo, oltre agli affari, il divorzio della sorella e il posto, aveva bisogno, come sempre, di rinfrescarsi, come diceva, dopo l'odor di chiuso moscovita. 

Mosca, malgrado i suoi cafés chantants e gli omnibus, era pur sempre una palude stagnante. Stepan Arkad'ic lo avvertiva sempre. Dopo aver vissuto un po' a Mosca, specialmente vicino alla famiglia, sentiva che andava giù di umore. Vivendo a lungo a Mosca, senza allontanarsene mai, giungeva al punto da cominciare a irritarsi con la moglie per il cattivo umore e i rimproveri di lei, per la salute, per l'educazione dei figli, per i piccoli interessi del suo ufficio; perfino i debiti l'inquietavano. Ma gli bastava soltanto arrivare e restare un poco a Pietroburgo, nell'ambiente che frequentava, dove si viveva, si viveva proprio e non ci si congelava, come a Mosca, e subito quei pensieri sparivano e si dissolvevano, come cera al fuoco. 

La moglie?\ldots{} Proprio quel giorno aveva parlato col principe cecenskij. Il principe cecenskij aveva una moglie e una famiglia, e dei figlioli che erano paggi; e aveva un'altra famiglia, illegittima, dalla quale pure aveva dei figli. Sebbene anche la prima famiglia fosse buona, il principe cecenskij si sentiva più felice nella seconda famiglia. E portava il primogenito nella seconda famiglia e raccontava a Stepan Arkad'ic che riteneva ciò utile e adatto a sviluppare il figlio. Che ne avrebbero detto a Mosca? 

I figli? A Pietroburgo i figli non impedivano di vivere ai padri. I figli venivano educati negli istituti e non c'era quel barbaro concetto che si diffondeva a Mosca (L'vov ne era un esempio), che i figli dovessero avere tutto il bello della vita, e i genitori solo il lavoro e le preoccupazioni. Qui si capiva che un uomo aveva il dovere di vivere per sé, così come deve vivere un uomo colto. 

L'impiego? Ma anche l'impiego qui non era quel tirar la carretta ostinato, senza speranze, come a Mosca; qui c'era interesse nell'impiego. Un incontro, un favore, una parola incisiva, il saper rappresentare in dialogo vari scherzi, e un uomo, a un tratto, faceva carriera, come Brjancev che Stepan Arkad'ic aveva incontrato il giorno prima e che adesso era primo dignitario. Questo impiego aveva un interesse. 

Il modo poi di Pietroburgo di considerare gli affari pecuniari agiva in maniera particolarmente rasserenante su Stepan Arkad'ic. Bartnjanskij, che spendeva almeno cinquantamila rubli per il train che conduceva, il giorno innanzi gli aveva detto a questo proposito una parola significativa. 

Prima di pranzo, messisi a parlare, Stepan Arkad'ic aveva detto a Bartnjanskij: 

- Tu, mi pare, sei intimo di Mordvinskij; mi puoi fare un favore, digli una parola per me, ti prego. C'è un posto che vorrei occupare. Membro dell'agenzia\ldots{} 

- Via, tanto non me lo ricordo\ldots{} Soltanto, che gusto c'è ad andare in queste imprese ferroviarie insieme ai giudei? Sia come vuoi, ma è un obbrobrio! 

Stepan Arkad'ic non gli disse che era un lavoro vivo: Bartnjanskij non l'avrebbe capito. 

- C'è bisogno di denaro per vivere. 

- E non vivi? 

- Vivo, ma ho dei debiti. 

- Davvero? per molto?- disse Bartnjanskij, compassionevole. 

- Per moltissimo, per ventimila rubli. 

Bartnjanskij scoppiò a ridere allegramente. 

- Oh, un uomo fortunato! - disse. - Io ne ho per mezzo milione e non ho nulla, e, come vedi, si può ancora campare. 

E Stepan Arkad'ic non soltanto a parole, ma nei fatti vedeva la giustezza di questo. zivachov aveva trecentomila rubli di debiti e neanche una copeca di suo, e pure viveva e ancora come! Il conte Krivcov, da tempo, tutti gli avevano già fatto il funerale, e lui ne manteneva due. Petrovskij aveva speso cinque milioni e viveva sempre allo stesso modo e soprintendeva alle finanze e riceveva ventimila rubli di stipendio. Ma, oltre a questo, Pietroburgo agiva su Stepan Arkad'ic in maniera fisicamente piacevole. Lo ringiovaniva. A Mosca qualche volta si guardava i capelli bianchi, si addormentava dopo pranzo, si stirava, saliva la scala al passo, respirando con affanno, si annoiava con le donne giovani, non ballava nelle feste. A Pietroburgo invece sentiva sempre che gli andavano via dieci anni dal groppone. 

Egli provava a Pietroburgo la stessa cosa che gli aveva detto, ancora il giorno prima, il sessantenne principe Pëtr Oblonskij, appena tornato dall'estero. 

- Noi qui non sappiamo vivere - diceva Pëtr Oblonskij. - Ci credi, io ho passato l'estate a Baden; ebbene, mi son sentito proprio un giovanotto. Vedo una donna giovane, e i pensieri\ldots{} Pranzi, bevi un poco e si ha forza, coraggio. Sono arrivato in Russia; dovevo andare da mia moglie e poi in campagna; non ci crederai, dopo due settimane, mi son messo la vestaglia indosso, ho smesso di vestirmi per il pranzo. Altro che pensare alle donne giovani! Son diventato proprio un vecchio. Mi restava solo da pensare all'anima. Sono andato a Parigi e mi son ripreso di nuovo. 

Stepan Arkad'ic sentiva proprio la stessa cosa, come Pëtr Oblonskij. A Mosca egli si lasciava andare così che davvero, a viverci molto, sarebbe giunto perfino a qualcosa di buono, quasi alla salvezza dell'anima; a Pietroburgo invece si sentiva di nuovo una persona in gamba. 

Fra la principessa Betsy Tverskaja e Stepan Arkad'ic esistevano vecchi rapporti, molto strani. Stepan Arkad'ic le faceva sempre scherzosamente la corte e le diceva, sia pure scherzosamente, le cose più sconvenienti, sapendo che questo le piaceva più di tutto. Il giorno dopo la sua conversazione con Karenin, Stepan Arkad'ic, essendo passato da lei, si sentiva tanto giovane, che nel farle la corte e nel celiare, era andato, senza volerlo, così lontano, da non saper più come tornare indietro, perché, per sua disgrazia, non solo non gli piaceva, ma gli era disgustosa. Questo tono s'era stabilito perché lui piaceva molto a lei. Quindi egli fu molto contento dell'arrivo della principessa Mjagkaja che interruppe la loro solitudine a due. 

- Ah, anche voi siete qui - ella disse nel vederlo. - Be', come va la vostra povera sorella? Non mi guardate così - soggiunse. - Da che le si sono scagliati contro tutti quelli che sono centomila volte peggiori di lei, io penso che ha fatto benissimo. Non posso perdonare a Vronskij di non avermi fatto sapere quando erano a Pietroburgo. Sarei andata da lei e con lei dovunque. Per favore, ditele da parte mia il mio affetto. Ma raccontatemi di lei. 

- Sì, la sua situazione è penosa, ella\ldots{} - voleva cominciare a raccontare Stepan Arkad'ic, prendendo ingenuamente per moneta contante le parole della Mjagkaja: ``raccontate di vostra sorella''. La principessa Mjagkaja subito, secondo la sua abitudine, lo interruppe e si mise a raccontare lei stessa. 

- Lei ha fatto quello che tutte, tranne me, fanno, ma nascondono; ma lei non voleva ingannare e ha fatto benissimo. E ha fatto ancora meglio, perché ha abbandonato quel pazzo di vostro cognato. Voi mi scusate. Tutti dicevano ch'era intelligente, io sola dicevo ch'era scemo. Adesso, da che s'è messo insieme con Lidija Ivanovna e con Landau, tutti dicono che è un pazzo, e io sarei magari contenta di non esser d'accordo con tutti, ma questa volta non posso. 

- Ma spiegatemi, per favore - disse Stepan Arkad'ic - che cosa significa questo? Ieri sono stato da lui per l'affare di mia sorella e ho chiesto una risposta decisiva. Lui non m'ha dato una risposta e mi ha detto che ci avrebbe pensato, e questa mattina invece della risposta, ho ricevuto un invito per questa sera dalla contessa Lidija Ivanovna. 

- Eh già, è così! - cominciò a dire con gioia la principessa Mjagkaja. - Chiederanno a Landau che cosa ne dice. 

- Come a Landau? Perché? Chi è mai Landau? 

- Come, non conoscete Jules Landau? Le fameux Jules Landau, le clairvoyant? Anche lui è un pazzo, ma la sorte di vostra sorella dipende da lui. Ecco le conseguenze della vita in provincia, voi non sapete niente. Landau, vedete, era commis in un magazzino di Parigi ed era andato dal dottore. Dal dottore, nella sala d'aspetto, si addormentò e nel sonno cominciò a dare consigli a tutti i malati. E consigli sorprendenti. Poi la moglie di Jurij Meldinskij, sapete, quello malato, è venuta a sapere di questo Landau e lo ha messo accanto al marito. Egli cura il marito. E, secondo me, non gli ha portato nessun miglioramento, perché è sempre debole allo stesso modo, ma loro credono in lui e se lo portano dietro. E l'hanno portato in Russia. Qui tutti si sono gettati su di lui, e lui s'è messo a curar tutti. Ha guarito la contessa Bezzubova, e lei ha preso a volergli tanto bene che l'ha adottato. 

- Come adottato? 

- Sì, adottato. Adesso non è più Landau, ma il conte Bezzubov. Ma la questione non sta qui. Lidija Ivanovna, io le voglio molto bene, ma non ha la testa a posto e naturalmente adesso s'è gettata su questo Landau, così che senza di lui né lei né Aleksej Aleksandrovic decidono nulla, e perciò la sorte di vostra sorella è ora nelle mani di Landau, o conte Bezzubov. 

\capitolo{XXI}\label{xxi-6} 

Dopo un ottimo pranzo e una grande quantità di cognac bevuto da Bartnjanskij, Stepan Arkad'ic, solo un poco in ritardo rispetto all'ora fissata, entrava dalla contessa Lidija Ivanovna. 

- Chi c'è ancora dalla contessa, il francese? - domandò Stepan Arkad'ic al portiere, osservando il noto cappotto di Aleksej Aleksandrovic e uno strano, primitivo cappotto con le fibbie. 

- Aleksej Aleksandrovic Karenin e il conte Bezzubov - rispose sostenuto il portiere. 

``La principessa Mjagkaja ha indovinato - pensò Stepan Arkad'ic salendo la scala. - Strano! Però sarebbe bene farsi intimo con lei. Ha un'influenza enorme. Se dirà una parola a Pomorskoj, allora la cosa è sicura''. 

Fuori era ancora giorno, ma nel piccolo salotto della contessa Lidija Ivanovna dalle tende abbassate erano già accese le luci. Presso una tavola rotonda, sotto la lampada, sedevano la contessa e Aleksej Aleksandrovic, e discorrevano sottovoce di qualcosa. Un uomo non alto, magro, con un bacino da donna, le gambe ripiegate in dentro all'altezza delle ginocchia, molto pallido, bello, con gli occhi luminosi, bellissimi, e i capelli lunghi, spioventi sul colletto del soprabito, stava in piedi all'altra estremità, osservando la parete con i ritratti. Dopo aver salutato la padrona di casa e Aleksej Aleksandrovic, Stepan Arkad'ic, involontariamente, guardò un'altra volta lo sconosciuto. 

- Monsieur Landau! - disse, rivolta a costui, la contessa con una dolcezza e una precauzione che colpirono Oblonskij. E li presentò. 

Landau si voltò in fretta, si avvicinò e, dopo aver sorriso, mise nella mano di Stepan Arkad'ic la sua immobile mano sudata e subito si allontanò di nuovo e si mise a guardare i ritratti. La contessa e Aleksej Aleksandrovic si scambiarono uno sguardo significativo. 

- Sono molto contenta di vedervi, in particolar modo oggi - disse la contessa Lidija Ivanovna, indicando a Stepan Arkad'ic un posto accanto a Karenin. 

- Vi ho presentato a lui come a Landau - disse lei con voce sommessa, dopo aver dato un'occhiata al francese e poi subito ad Aleksej Aleksandrovic - ma invero egli è il conte Bezzubov, come forse saprete. Ma egli non ama questo titolo. 

- Già, ho sentito - rispose Stepan Arkad'ic. - Dicono che abbia del tutto guarito la contessa Bezzubova. 

- Quest'oggi è stata da me, fa proprio pena! - disse rivolta ad Aleksej Aleksandrovic. - Questa separazione è tremenda per lei. Per lei è un tale colpo! 

- E lui va via definitivamente? - domandò Aleksej Aleksandrovic. 

- Sì, va a Parigi. Ieri ha sentito una voce - disse la contessa Lidija Ivanovna, guardando Stepan Arkad'ic. 

- Ah, una voce! - ripeté Oblonskij, sentendo che bisognava essere quanto più possibile prudenti in quella compagnia, dove accadeva o stava per accadere qualcosa di speciale, di cui non possedeva ancora la chiave. 

Seguì un momento di silenzio, dopo il quale la contessa Lidija Ivanovna, come avvicinandosi all'argomento principale della conversazione, disse a Oblonskij con un sorriso sottile: 

- Io vi conosco da tempo e sono molto contenta di conoscervi più da vicino. Les amis de nos amis sont nos amis. Ma per essere amici bisogna penetrare col pensiero lo stato d'animo dell'amico, e io ho paura che voi non lo facciate riguardo ad Aleksej Aleksandrovic. Voi capite di che cosa parlo - disse, alzando i suoi bellissimi occhi pensosi. 

- In parte, contessa, capisco che la posizione di Aleksej Aleksandrovic\ldots{} - disse Oblonskij senza ben capire di che cosa si trattasse, e desiderando perciò di restare sulle generali. 

- Il cambiamento non è nella situazione esterna - disse severa la contessa Lidija Ivanovna, seguendo nello stesso tempo con lo sguardo innamorato Aleksej Aleksandrovic, che s'era alzato e s'era portato presso Landau - il suo cuore s'è cambiato, gli è stato dato un cuore nuovo, e io temo che voi non abbiate pienamente penetrato il cambiamento che è avvenuto in lui. 

- Ma nelle linee generali posso immaginarmi questo cambiamento. Noi siamo sempre stati amici, e adesso\ldots{} - disse, rispondendo con uno sguardo mellifluo allo sguardo della contessa, Stepan Arkad'ic, mentre andava considerando quale dei due ministri ella conoscesse più intimamente, per sapere a quale dei due avrebbe dovuto pregarla di parlare. 

- Quel cambiamento che è avvenuto in lui non può indebolire i suoi sentimenti d'amore verso il prossimo; al contrario, il cambiamento che è avvenuto in lui deve aumentare l'amore. Ma io ho paura che voi non mi comprendiate. Volete del tè? - disse, indicando con gli occhi un cameriere che serviva il tè su di un vassoio. 

- No, grazie contessa. S'intende, la sua sventura\ldots{} 

- Sì, la sventura, che è divenuta la più grande fortuna poiché il cuore s'è fatto nuovo, si è riempito di Lui - diss'ella guardando con trasporto Stepan Arkad'ic. 

``Io penso che si potrà chiederle di dirlo a tutti e due'' pensava Stepan Arkad'ic. 

- Oh, certamente, contessa - diss'egli - ma io penso che questi cambiamenti sono così intimi, che nessuno, neppure la persona più vicina, ama parlarne. 

- Al contrario! Dobbiamo parlarne, e aiutarci l'un l'altro. 

- Sì, senza dubbio, ma può esserci una tale differenza di opinioni, e inoltre\ldots{} - disse Oblonskij con un sorriso mellifluo. 

- Non ci può essere differenza nell'opera della santa verità. 

- Oh sì, certamente, ma\ldots{} - e, essendosi confuso, Stepan Arkad'ic tacque. Aveva capito che si trattava di religione. 

- Mi pare che stia per addormentarsi - pronunciò con un mormorio significativo Aleksej Aleksandrovic, avvicinandosi a Lidija Ivanovna. 

Stepan Arkad'ic si guardò in giro. Landau era seduto vicino a una finestra, appoggiato al bracciuolo della spalliera di una poltrona, col capo chino. Avendo notato gli sguardi rivolti su di lui, alzò il capo e sorrise d'un sorriso infantilmente ingenuo. 

- Non ci badate - disse Lidija Ivanovna, e con un movimento leggero accostò una sedia ad Aleksej Aleksandrovic. - Ho notato\ldots{} - cominciò a dire qualcosa, quando nella stanza entrò un cameriere con una lettera. Lidija Ivanovna scorse rapidamente il biglietto e, chiesto scusa, con una straordinaria sveltezza scrisse, consegnò la risposta e tornò accanto alla tavola. - Ho notato - continuò il discorso incominciato - che i moscoviti, soprattutto gli uomini, sono le persone più indifferenti verso la religione. 

- Oh no, contessa, mi pare che i moscoviti abbiano la fama d'essere i più tenaci - rispose Stepan Arkad'ic. 

- Ma per quanto sappia, voi, disgraziatamente, siete fra gli indifferenti - disse con un sorriso stanco Aleksej Aleksandrovic, rivolgendosi a lui. 

- Come si può essere indifferenti! - disse Lidija Ivanovna. 

- Io, riguardo a questo, non che sia indifferente, ma sono in attesa - disse Stepan Arkad'ic con il più dolce dei suoi sorrisi. - Io penso che per me non sia giunto ancora il tempo per codeste questioni. 

Aleksej Aleksandrovic e Lidija Ivanovna si scambiarono un'occhiata. 

- Noi non possiamo mai sapere se sia venuto o no il tempo per noi - disse severo Aleksej Aleksandrovic. - Noi non dobbiamo pensare al fatto se siamo pronti o non siamo pronti: la grazia non può essere guidata da considerazioni umane; a volte non scende su chi opera e scende sugli impreparati, come su Saul. 

- No, mi pare, non ancora adesso - disse Lidija Ivanovna, che intanto aveva sorvegliato i movimenti del francese. Landau si alzò e si avvicinò a loro. 

- Mi permettete di ascoltare? - chiese. 

- Oh sì, non volevo disturbarvi - disse Lidija Ivanovna, guardandolo con tenerezza - sedetevi con noi. 

- Bisogna soltanto non chiudere gli occhi per non rimanere privi della luce - continuò Aleksej Aleksandrovic. 

- Ah, se voi sapeste la felicità che noi proviamo, sentendo la Sua eterna presenza nell'anima nostra! - disse la contessa Lidija Ivanovna, sorridendo beata. 

- Ma l'uomo a volte può sentirsi incapace di elevarsi a quest'altezza - disse Stepan Arkad'ic, sentendo di mancar di lealtà nel riconoscer l'altezza della religione, ma nello stesso tempo senza decidersi a confessare la propria libertà di pensiero dinanzi a una persona che, con una sola parola a Pomorskoj, poteva fargli avere il posto desiderato. 

- Con questo volete dire che il peccato glielo impedisce? - disse Lidija Ivanovna. - Ma è un'opinione falsa. Non c'è peccato per i credenti, il peccato è già riscattato. Pardon - ella soggiunse, guardando il cameriere che era entrato di nuovo con un altro biglietto. Lesse e rispose a voce: ``dite domani, dalla granduchessa\ldots{}''. - Per il credente non c'è peccato - ella proseguì il discorso. 

- Sì, ma la fede senza opere è morta - disse Stepan Arkad'ic, ricordandosi questa frase del catechismo e difendendo ormai la propria indipendenza col solo sorriso. 

- Eccolo, è dell'epistola dell'apostolo Giacomo - disse Aleksej Aleksandrovic, rivolgendosi con un certo rimprovero a Lidija Ivanovna, evidentemente come per una cosa di cui avevano già parlato più di una volta. - Quanto danno ha fatto la falsa interpretazione di questo passo! Nulla allontana tanto dalla fede come questa interpretazione. ``Io non ho opere, non posso credere'', mentre questo non è detto in nessun posto. È detto il contrario. 

- Lavorare per Dio, salvar l'anima con le fatiche, col digiuno - disse la contessa Lidija Ivanovna con disprezzo e disgusto - sono le idee barbare dei nostri monaci\ldots{} Mentre questo non è detto in nessun passo. È molto più semplice e facile - ella soggiunse, guardando Oblonskij con quello stesso sorriso d'approvazione con cui a corte incoraggiava le giovani damigelle d'onore, confuse dall'ambiente nuovo. 

- Noi siamo salvati dal Cristo che ha sofferto per noi. Siamo salvati dalla fede - confermò Aleksej Aleksandrovic, approvando con lo sguardo le parole di lei. 

- Vous comprenez l'anglais? - chiese Lidija Ivanovna e, ricevutane risposta affermativa, si alzò e cominciò a scegliere dei libri su di uno scaffaletto. 

- Voglio leggere Safe and Happy o Under the wing - ella disse, guardando interrogativamente Karenin. E trovato il libro e sedutasi di nuovo al proprio posto, lo aprì. - È molto breve. Qui è descritta la via attraverso la quale si acquista la fede, e la felicità, al di sopra di ogni cosa terrestre, che allora riempie l'anima. Il credente non può essere infelice, perché non è solo. Ma ecco, vedrete. - Ella s'era già disposta a leggere, quando entrò di nuovo il cameriere. - La Borozdina? Dite domani alle due. Sì - ella disse, segnando con un dito un punto nel libro e guardando con un sospiro dinanzi a sé coi bellissimi occhi pensosi. - Ecco come agisce la fede vera. Conoscete la Sanina? Marie? Sapete la sua sventura? Ha perduto il suo unico bambino. Era disperata. Ebbene? Ha trovato quest'amico, e adesso ringrazia Dio per la morte del bambino. Ecco la felicità che dà la fede! 

- Oh sì, è molto\ldots{} - disse Stepan Arkad'ic, contento del fatto che avrebbero letto e gli avrebbero dato il tempo di riaversi un poco. ``No, ormai si vede che è meglio non chiedere nulla per oggi - pensava - basta uscire di qua senza avere ingarbugliato le cose''. 

- Vi annoierete - disse la contessa Lidija Ivanovna, rivolta a Landau - voi non sapete l'inglese, ma è una cosa breve. 

- Oh, capirò - disse Landau con lo stesso sorriso e chiuse gli occhi. 

Aleksej Aleksandrovic e Lidija Ivanovna si scambiarono un'occhiata significativa e la lettura cominciò. 

\capitolo{XXII}\label{xxii-6} 

Stepan Arkad'ic si sentiva completamente confuso nell'ascoltare quegli strani discorsi, per lui nuovi. La complessità della vita pietroburghese agiva, in genere, da eccitante su di lui, traendolo dal ristagno moscovita; ma quella complessità egli l'amava e capiva nei campi a lui vicini e noti; in quell'ambiente estraneo si sentiva confuso, stordito e non riusciva ad abbracciare tutto. Ascoltando la contessa Lidija Ivanovna e sentendo fissi su di sé gli occhi belli, ingenui o furbi, non lo sapeva lui stesso, di Landau, Stepan Arkad'ic cominciava a provare una certa pesantezza di testa. 

I pensieri più vaghi gli si confondevano nella testa. ``Marie Sanina è contenta che le sia morto il bambino\ldots{} Sarebbe bene fumare adesso\ldots{} Per salvarsi bisogna soltanto credere, e i monaci non sanno come bisogna fare; lo sa la contessa Lidija Ivanovna\ldots{} E perché ho un tal peso sulla testa? Per il cognac o perché tutto questo è molto strano? Tuttavia finora non ho fatto nulla di poco conveniente, mi pare, comunque, non posso più chiederglielo. Dicono che essi obblighino a pregare. Purché non obblighino me. Questo poi sarebbe troppo sciocco. E quali assurdità sta leggendo, ma pronuncia bene. Landau è Bezzubov, perché è Bezzubov?''. A un tratto Stepan Arkad'ic sentì che la mascella inferiore cominciava a torcerglisi in uno sbadiglio irrefrenabile. Accomodò le fedine, nascondendo lo sbadiglio, e si scosse. Ritornò in sé nel momento in cui la voce della contessa Lidija Ivanovna diceva: ``dorme''. 

Stepan Arkad'ic ritornò in sé con spavento, sentendosi colpevole e colto in fallo. Ma si consolò subito vedendo che la parola ``dorme'' non riguardava lui, ma Landau. Il francese s'era addormentato proprio come Stepan Arkad'ic. Ma il sonno di Stepan Arkad'ic, come egli pensava, li avrebbe offesi (ormai non pensava neanche più questo, tanto ormai gli sembrava tutto strano), e il sonno di Landau, invece, li rallegrò straordinariamente, in particolare rallegrò la contessa Lidija Ivanovna. 

- Mon ami - disse Lidija Ivanovna, sollevando con precauzione, per non far rumore, le pieghe del suo vestito di seta e chiamando, ormai eccitata, Karenin non Aleksej Aleksandrovic ma mon ami - donnez lui la main. Vous voyez? Sst! - ella zittì il cameriere che era entrato di nuovo. - Non si riceve. 

Il francese dormiva o fingeva di dormire, appoggiando la testa alla spalliera della seggiola, e con la mano sudata, che teneva su di un ginocchio, faceva dei movimenti deboli come se afferrasse qualcosa. Aleksej Aleksandrovic si alzò, voleva proceder cauto, ma inciampò nella tavola, si avvicinò e mise la mano nella mano del francese. Stepan Arkad'ic si alzò anche lui e, spalancando gli occhi, desiderando di svegliarsi nel caso si fosse addormentato, guardava ora l'uno ora l'altro. Tutto ciò accadeva nella realtà. Stepan Arkad'ic sentiva che nella sua testa le cose andavano sempre peggio. 

- Que la personne qui est arrivée la dernière, celle qui demande, qu'elle sorte! Qu'elle sorte! - pronunciò il francese senza aprire gli occhi. 

- Vous m'excuserez, mais vous voyez\ldots{} Revenez vers dix heures, encore mieux demain. 

- Qu'elle sorte! - ripeté il francese con impazienza. 

- C'est moi, n'est-pas? - E, ricevutane risposta affermativa, Stepan Arkad'ic, dimentico anche di quello che voleva chiedere a Lidija Ivanovna, dimentico dell'affare della sorella, col solo desiderio di andar via di là al più presto, uscì in punta di piedi e corse fuori in istrada come da una casa infestata, e discorse e scherzò a lungo con i vetturini, desiderando di riprendere i sensi al più presto. 

Al Teatro Francese, dove giunse alla fine, e poi dai tartari dove bevve lo champagne, Stepan Arkad'ic respirò un poco nell'atmosfera che gli era propria. Tuttavia quella sera non si sentiva bene affatto. 

Tornato a casa da Pëtr Oblonskij, presso il quale abitava a Pietroburgo, Stepan Arkad'ic trovò un biglietto di Betsy. Ella gli scriveva che desiderava molto terminare la conversazione cominciata, e lo pregava di passare l'indomani. Fece appena in tempo a leggere questo biglietto e a farci su una smorfia, che di sotto si sentirono i passi strascicati di persone che trasportavano qualcosa di pesante. 

Stepan Arkad'ic uscì a vedere. Era il ringiovanito Pëtr Oblonskij. Era così ubriaco da non poter salire le scale; ma, nel vedere Stepan Arkad'ic, ordinò che lo mettessero in piedi e, avvinghiatosi a lui, andò in camera sua e là cominciò a raccontargli come aveva passato la serata, e proprio lì si addormentò. 

Stepan Arkad'ic era avvilito, il che gli accadeva di rado, e per lungo tempo non poté prender sonno. Qualunque cosa ricordasse era disgustosa, e la cosa più disgustosa di tutte, quasi vergognosa, era il ricordo della serata dalla contessa Lidija Ivanovna. 

Il giorno dopo ricevette da Aleksej Aleksandrovic un rifiuto definitivo per il divorzio di Anna e capì che questa decisione era stata basata su quello che aveva detto il giorno prima il francese nel suo sonno vero o finto. 

\capitolo{XXIII}\label{xxiii-6} 

Per intraprendere qualcosa nella vita familiare, sono indispensabili o un completo dissidio fra i coniugi o un amorevole accordo. Quando invece i rapporti fra i coniugi sono indefiniti e non c'è né l'uno né l'altro, nulla può essere intrapreso. 

Molte famiglie rimangono per anni in vecchi luoghi, uggiosi ormai per entrambi i coniugi, solo perché non c'è un assoluto dissidio, né un pieno accordo. 

E per Vronskij e per Anna la vita moscovita con il caldo e con la polvere, quando il sole splendeva ormai non più primaverile, ma estivo, e tutti gli alberi dei viali erano già da tempo con le foglie, e le foglie erano ricoperte di polvere, era insopportabile; ma essi, senza trasferirsi a Vozdvizenskoe, come era stato deciso da tempo, continuavano a stare a Mosca, ormai uggiosa a entrambi, perché negli ultimi tempi non c'era accordo fra di loro. 

L'irritazione che li divideva non aveva nessuna causa esteriore, e tutti i tentativi di spiegazione non solo non la eliminavano, ma l'accrescevano. Era un'irritazione interna, che per lei aveva come base l'affievolirsi dell'amore di lui, per lui, il pentimento d'essersi messo per amore di lei in una posizione penosa, ch'ella, invece di alleviare, rendeva sempre più penosa. Né l'una né l'altro dichiaravano le cause della propria irritazione, ma si consideravano scambievolmente in fallo e ad ogni pretesto cercavano di dimostrarselo a vicenda. 

Per Anna, lui, con le sue abitudini, coi suoi pensieri, coi suoi desideri, col suo complesso spirituale e fisico, significava una cosa sola: l'amore per le donne che, secondo il suo sentimento, doveva concentrarsi unicamente su di lei. Ma questo amore era diminuito e, di conseguenza, pensava che egli aveva dovuto riversarlo su altre o su di un'altra donna e ciò la rendeva gelosa. Non era però gelosa di una determinata donna; lo era solamente perché l'amore per lei si affievoliva. Non avendo ancora trovato un oggetto per la propria gelosia, lo andava cercando. Ogni minima allusione era buona; ora erano quelle donne volgari con le quali egli per la sua vita di scapolo poteva venire in rapporti così facilmente; ora erano le donne che incontrava nell'alta società; ora era una ragazza immaginaria che egli avrebbe voluto sposare, rompendo i rapporti con lei. Quest'ultima gelosia la tormentava più di tutte, in particolare perché egli stesso, in un momento di sincerità, le aveva detto, imprudentemente, che sua madre lo capiva tanto poco da permettersi di esortarlo a sposare la principessina Sorokina. 

E, essendone gelosa, Anna era indignata contro di lui e cercava in tutto i motivi per indignarsi. Di tutto quello che c'era di penoso nella propria situazione ella accusava lui. Il tormentoso stato di attesa che aveva vissuto a Mosca, fra cielo e terra, la lentezza e l'indecisione di Aleksej Aleksandrovic, la propria solitudine, tutto ella attribuiva a lui. Se egli l'avesse amata, avrebbe capito tutta la difficoltà della sua situazione, e l'avrebbe tratta fuori da questa. Del fatto che ella stesse a Mosca e non in campagna, anche lui era colpevole. Egli non poteva vivere sepolto in campagna, come voleva lei. Gli era indispensabile la società, e aveva messo lei in quella orribile posizione, la cui difficoltà non voleva intendere. E, di nuovo, era anche lui colpevole ch'ella fosse divisa per sempre dal figlio. 

Perfino quei pochi momenti di effusione che avvenivano fra di loro non la calmavano più: nell'amore di lui ella intravedeva una certa calma, un tono di sicurezza che prima non c'erano e che la irritavano. 

Era già il crepuscolo. Anna, sola, aspettava il ritorno di lui da un pranzo di scapoli a cui era andato, camminava avanti e indietro per il suo studio (la stanza dove si sentiva meno il rumore del selciato), e ripensava in tutti i particolari le espressioni del litigio del giorno prima. Riandando sempre più indietro, dalle parole offensive della discussione che le tornavano in mente a quello che ne era stato il motivo, ella giunse finalmente all'inizio della conversazione. Per lungo tempo non poté credere che il dissidio fosse cominciato da una conversazione così innocua, così poco vicina al cuore di chiunque. E realmente era stato così. Tutto era cominciato dal fatto che egli aveva preso in giro i ginnasi femminili, ritenendoli inutili, e lei ne aveva preso le difese. Egli aveva trattato irriverentemente l'istruzione femminile, in generale, e aveva detto che Hanna, l'inglese protetta da Anna, non aveva nessun bisogno di conoscere la fisica. Questo aveva irritato Anna. 

- Io non m'aspetto che vi ricordiate di me, dei miei sentimenti, come se ne può ricordare una persona che ama, ma mi aspetto soltanto un po' di delicatezza - ella aveva detto. 

E invero, egli era diventato rosso di collera e aveva detto qualcosa di spiacevole. Ella non ricordava cosa gli avesse risposto, ma soltanto che a questo punto, a proposito di qualche cosa, egli, evidentemente col desiderio di farle male, aveva detto: 

- Mi spiace il vostro entusiasmo per questa bambina, è vero, perché vedo che esso è innaturale. 

La crudezza con la quale egli distruggeva il mondo da lei creato con tanta fatica per sopportare la propria vita penosa, l'ingiustizia con la quale l'accusava di finzione, di mancanza di naturalezza, l'avevano indignata. 

- Sono molto spiacente che solo ciò che è volgare e materiale sia comprensibile e naturale per voi - ella aveva detto, ed era uscita dalla stanza. 

Quando la sera prima egli era venuto da lei, essi non avevano ricordato il diverbio che c'era stato, ma tutti e due avevano sentito che esso era appianato, ma non scomparso. 

Tutto quel giorno egli non era stato in casa e lei provava una sensazione di solitudine e di pena nel sentirsi in urto con lui; così che voleva dimenticare e perdonare tutto e far la pace, voleva accusare se stessa e assolvere lui. 

``Io stessa sono colpevole. Sono irritabile, sono insensatamente gelosa. Farò la pace con lui, e partiremo per la campagna, là sarò più calma'' ella si diceva. 

``Innaturale - ella ricordò a un tratto non tanto la parola quanto l'intenzione di farle del male, che più di tutto l'aveva offesa. - Lo so quello che voleva dire; voleva dire: è innaturale amare una creatura estranea quando non si ama la propria figlia. Cosa capisce lui dell'amore per i bambini, del mio amore per Serëza, che ho sacrificato per lui? Ma questo desiderio di farmi del male! No, ama un'altra donna, non può essere altrimenti''. 

E visto che, desiderando calmarsi, aveva compiuto di nuovo il giro dei pensieri da lei fatto tante volte ed era tornata all'irritazione di prima, inorridì di se stessa. ``Davvero non è possibile? Davvero non posso prender la cosa su di me? - si disse e cominciò di nuovo, daccapo. - È sincero, è onesto, mi ama. Io lo amo, a giorni uscirà il divorzio. E di che cosa c'è bisogno ancora? C'è bisogno di calma, di fiducia, e io prenderò la cosa su di me. Sì, adesso, quando verrà, dirò che sono io colpevole, quantunque non lo sia, e partiremo''. 

E, per non pensare più e per non cedere all'irritazione, sonò e fece portar dentro i bauli per mettervi la roba da mandare in campagna. 

Alle dieci venne Vronskij. 

\capitolo{XXIV}\label{xxiv-5} 

- Be', c'è stata allegria? - chiese lei, uscendogli incontro con un'espressione colpevole e mansueta nel viso. 

- Come al solito - egli rispose, comprendendo immediatamente da un solo sguardo, che ella era in buona disposizione d'animo. S'era già abituato a questi passaggi, e quel giorno ne era particolarmente contento, perché anche lui era nella migliore disposizione d'animo. 

- Che vedo! Questo sì, che va bene! - disse, indicando i bauli in anticamera. 

- Sì, bisogna partire. Sono andata a passeggio in vettura e si stava tanto bene, che m'è venuta voglia di andare in campagna. Perché non ti trattiene nulla, vero? 

- Non desidero che questo. Vengo subito e parleremo, mi cambio soltanto. Fa' portare il tè. 

Ed egli andò nello studio. 

C'era qualcosa di offensivo nell'aver egli detto: ``Questo sì, che va bene'' così come si dice a un bambino che abbia cessato di far capricci. E ancora più offensivo era quel contrasto fra il tono colpevole di lei e quello di lui sicuro di sé; e per un attimo ella sentì il desiderio di lotta che insorgeva in lei; ma, fatto uno sforzo su di sé, lo soffocò e accolse Vronskij ancora allegra. 

Quand'egli venne da lei, ella gli raccontò, ripetendo in parte delle parole preparate, la propria giornata e i suoi progetti per la partenza. 

- Sai, m'è venuta quasi un'ispirazione - ella diceva. - Perché aspettare il divorzio qui? Io non posso aspettare più. Non voglio sperare, non voglio sentir dire nulla del divorzio. Ho stabilito che questo non avrà più influenza sulla mia vita. Anche tu sei d'accordo? 

- Oh sì - egli disse, dopo aver guardato con inquietudine il viso di lei agitato. 

- E che avete fatto là? chi c'era? - diss'ella, dopo essere stata un poco zitta. 

Vronskij nominò gli ospiti. 

- Il pranzo è stato ottimo, e la regata delle imbarcazioni e tutto è stato abbastanza simpatico, ma a Mosca non possono vivere senza il ridicule. È apparsa una certa signora, la maestra di nuoto della regina di Svezia, e ha fatto sfoggio della propria abilità. 

- Come? ha nuotato? - domandò Anna, accigliandosi. 

- In un certo costume de natation rosso, lei vecchia e deforme. E allora quando andiamo via? 

- Che fantasia sciocca! Be', nuota in modo particolare? - disse Anna senza rispondere. 

- Niente di speciale. Lo sto dicendo, una cosa tremendamente sciocca. E allora quando pensi di andar via? 

Anna scosse il capo, come desiderando di scacciare un pensiero spiacevole. 

- Quando andar via? Ma quanto prima, tanto meglio. Domani non faremo in tempo. Domani l'altro. 

- Sì\ldots{} no, aspetta. Domani l'altro è domenica, devo andare da maman - disse Vronskij, confuso, perché appena pronunciato il nome della madre, aveva sentito su di sé uno sguardo sospettoso e fisso. La confusione di lui confermò i sospetti di lei. Si accese in viso e si allontanò. Adesso non era più la maestra di nuoto della regina di Svezia che appariva ad Anna, ma la principessina Sorokina, che abitava in campagna nei dintorni di Mosca, insieme alla contessa Vronskaja. 

- Puoi andare domani - disse. 

- Ma no. Per l'affare per cui vado, le procure e i denari non si riscuotono domani - egli rispose. 

- Se è così, non partiremo affatto. 

- Ma perché? 

- Io non vado via più tardi. O lunedì o mai più. 

- E perché mai? - disse Vronskij, quasi con stupore. - Non ha mica senso questo! 

- Per te questo non ha senso, perché non ti importa nulla di me. Tu non vuoi capire la mia vita. L'unica cosa che mi occupava qui era Hanna. Tu dici che è una finzione. Hai pure detto ieri che non amavo mia figlia, ma che fingevo di amare questa inglese, il che era innaturale; io vorrei sapere quale vita qui può essere naturale per me. 

Per un attimo ella ritornò in sé e inorridì d'esser venuta meno alla propria intenzione. Ma, anche sapendo di rovinarsi, non riusciva a trattenersi, non poteva non fargli vedere come egli avesse torto, non poteva sottomettersi a lui. 

- Io non ho mai detto questo, ho detto che non avevo simpatia per quest'amore improvviso. 

- Perché tu, che ti vanti della tua dirittura, non dici la verità? 

- Io non mi vanto mai e non dico mai quello che non è vero - egli disse piano, trattenendo l'ira che si sollevava in lui. - È un gran peccato, se tu non rispetti\ldots{} 

- Il rispetto l'hanno inventato per nascondere il vuoto là dove dovrebbe essere l'amore\ldots{} E se tu non mi ami più è meglio ed è più onesto dirlo. 

- No, questo diventa insopportabile! - gridò Vronskij, alzandosi dalla sedia. E, fermatosi dinanzi a lei, pronunciò adagio: - Perché metti a prova la mia pazienza? - disse con un tono tale come se avesse voluto dire molte cose, e si contenesse. - Essa ha dei limiti. 

- Che volete dire con questo? - ella gridò, esaminando con orrore la esplicita espressione di odio che era su tutto il viso di lui e in particolare negli occhi crudeli, minacciosi. 

- Voglio dire\ldots{} - egli voleva cominciare, ma si fermò. - Posso sapere, che cosa volete da me? 

- Che cosa posso volere? Posso volere soltanto che non mi abbandoniate, come avete in mente - diss'ella, comprendendo tutto quello ch'egli non aveva detto fino in fondo. - Ma questo non lo voglio, è secondario. Io voglio l'amore e l'amore non c'è. Perciò tutto è finito. 

Ella si diresse verso la porta. 

- Aspetta! Aspetta! - disse Vronskij, senza distendere la piega cupa delle sopracciglia e fermandola per un braccio. - Di che si tratta? Io ho detto che bisogna rimandare la partenza di tre giorni, tu in risposta a questo hai detto che mento, che sono un uomo disonesto. 

- Sì, e ripeto che l'uomo che mi rinfaccia di aver sacrificato tutto per me - disse ella ricordando le parole amare della lite di prima - è peggiore di un uomo disonesto, è un uomo senza cuore. 

- No, ci sono dei limiti alla pazienza - egli gridò e lasciò andare rapidamente il braccio di lei. 

``Egli mi odia, è chiaro'' ella pensò, e in silenzio, senza voltarsi, a passi incerti uscì dalla stanza. 

``Ama un'altra donna, è ancor più chiaro - ella si diceva, entrando in camera sua. - Io voglio amore e amore non c'è. Perciò tutto è finito - ella ripeteva le parole già dette - e bisogna definire''. 

``Ma come?'' si domandò e sedette su di una poltrona dinanzi allo specchio. 

I pensieri su dove sarebbe andata adesso, se dalla zia presso la quale era stata allevata, da Dolly, o semplicemente sola all'estero, e su quello che faceva adesso lui nello studio, solo, se questo era un litigio definitivo, o se era possibile ancora far la pace, e su quello che adesso avrebbero detto di lei tutte le antiche conoscenti di Pietroburgo, come avrebbe visto la cosa Aleksej Aleksandrovic, e molti altri pensieri su quello che sarebbe accaduto, dopo la rottura, le venivano in mente, ma ella non si abbandonava con tutta l'anima a questi pensieri. Nella sua anima c'era un certo pensiero confuso, che la interessava unicamente, ma di cui non riusciva a rendersi conto. Ricordando ancora una volta Aleksej Aleksandrovic, ricordò anche il tempo della propria malattia, dopo il parto, e quel sentimento che allora non la lasciava. ``Perché non sono morta?''. Le tornavano in mente le sue parole di allora e il sentimento di allora. E a un tratto capì quello che c'era nell'anima sua. Sì, era quel pensiero solo che risolveva tutto. ``Sì, morire! E la vergogna e l'infamia di Aleksej Aleksandrovic e di Serëza, e la mia orribile vergogna, tutto si salva con la morte. Morire, e lui si pentirà, avrà pietà, amerà, soffrirà per me''. Con un sorriso di compassione verso se stessa fisso sul viso, ella sedeva nella poltrona, togliendo e infilando gli anelli dalla mano destra, figurandosi con chiarezza, sotto vari aspetti, i sentimenti di lui dopo la sua morte. 

Dei passi che si avvicinavano, i passi di lui, la distrassero. Come se fosse occupata nel mettere a posto gli anelli, ella non si voltò neppure verso di lui. 

Egli le si accostò e, presala per una mano, disse piano: 

- Anna, andiamo via domani l'altro, se vuoi. Acconsento a tutto. 

Ella taceva. 

- Ebbene? - egli domandò. 

- Lo sai tu stesso - diss'ella e nello stesso momento, non avendo più la forza di contenersi, si mise a singhiozzare. 

- Lasciami, lasciami! - ella diceva fra i singhiozzi. - Domani parto\ldots{} Farò di più. Chi sono? sono una donna perduta. Una pietra al tuo collo. Non voglio tormentarti, non voglio! Ti libererò. Tu non mi ami, tu ami un'altra! 

Vronskij la supplicava di calmarsi e la rassicurava che la sua gelosia non aveva un'ombra di fondamento, che non aveva mai cessato di amarla e che l'amava più di prima. 

- Anna, perché tormentare te e me? - egli diceva, baciandole le mani. Sul viso di lui, adesso, si esprimeva la tenerezza, e a lei sembrava di sentire con l'orecchio il suono delle lacrime nella voce di lui e sulla propria mano ne sentiva l'umidore. E in un attimo, la disperata gelosia di Anna si cambiò in una disperata, appassionata tenerezza: lo abbracciò, gli coprì di baci la testa, il collo, le mani. 

\capitolo{XXV}\label{xxv-5} 

Sentendo che la riconciliazione era avvenuta in pieno, Anna fin dalla mattina si mise con lena a fare i preparativi per la partenza. Sebbene non si fosse deciso se partivano il lunedì o il martedì (il giorno prima avevano ceduto l'una all'altro), Anna si preparava con cura alla partenza, sentendosi ormai del tutto indifferente al fatto che andassero via un giorno prima o dopo. Era in piedi nella stanza, curva su di un baule aperto, scegliendo le sue cose, quando egli entrò da lei, già vestito, prima del solito. 

- Vado subito da maman, il denaro me lo può mandare per mezzo di Egor. E domani sono pronto ad andare via - egli disse. 

Per quanto ella fosse di buonumore, il ricordo della gita in campagna la punse. 

- No, tanto neanch'io farò in tempo - ella disse subito, e pensò: ``allora si potevano disporre le cose in modo da fare come volevo io''. - No, fa' come volevi. Va' in sala da pranzo, vengo subito, devo scegliere fra questa roba inutile - diss'ella passando ancora qualcosa sul braccio di Annuška, sul quale c'era già una montagna di roba. 

Vronskij mangiava la sua bistecca, quand'ella entrò in sala da pranzo. 

- Non puoi immaginare come siano diventate prive di attrattiva per me queste stanze - ella disse, sedendosi accanto a lui davanti al proprio caffè. - Non c'è niente di più detestabile di queste chambres garnies. Non c'è espressione, non c'è anima. Quest'orologio, le tende e soprattutto le tappezzerie sono un incubo. Penso a Vozdvizenskoe come alla terra promessa. Non mandi via anche i cavalli? 

- No, andranno via dopo di noi. E tu vai in qualche posto? 

- Volevo andare dalla Wilson. Ho portato dei vestiti da lei. Allora proprio domani - disse con voce gaia; ma a un tratto il suo viso cambiò. 

Il cameriere di Vronskij venne a chiedere la ricevuta di un telegramma da Pietroburgo. Non c'era nulla di speciale che Vronskij ricevesse un telegramma, ma egli, come desiderando di nasconderle qualcosa, disse che la ricevuta era nello studio e si voltò con premura verso di lei. 

- Domani finirò tutto assolutamente. 

- Di chi è il telegramma? - ella domandò, senz'ascoltarlo. 

- Di Stiva - egli rispose controvoglia. 

- E perché non me l'hai fatto vedere? Che mistero può esistere tra Stiva e me? 

Vronskij fece tornare il cameriere e ordinò di portare il telegramma. 

- Non te lo volevo far vedere perché Stiva ha la mania dei telegrammi; perché telegrafare, quando nulla è deciso? 

- Per il divorzio? 

- Sì, ma lui scrive: ``Non ho potuto ancora ottenere nulla. A giorni ha promesso una risposta definitiva''. Ma ecco, leggi. 

Con le mani tremanti Anna prese il telegramma e lesse quelle stesse cose che Vronskij aveva detto. Alla fine, era ancora aggiunto: ``C'è poca speranza, ma farò il possibile e l'impossibile''. 

- Ieri ho detto che per me era proprio lo stesso ottenere e anche non ottenere il divorzio - diss'ella, arrossendo. - Non c'era nessun bisogno di nascondermelo. - ``Così egli può nascondere e nasconde la sua corrispondenza con le donne'' ella pensò. 

- E Jašvin voleva venire stamattina con Vojtov - disse Vronskij; - pare che abbia vinto a Pevcov tutto, e anche più di quello che lui può pagare, intorno ai sessantamila rubli. 

- No - ella disse, irritata perché lui, con questo mutar di discorso, le mostrava chiaramente ch'ella era irritata - perché mai pensi che questa notizia mi interessi tanto da dovermela perfino nascondere? Io ho detto che non voglio pensarci, e desidero che tu te ne interessi tanto poco quanto me. 

- Io me ne interesso perché mi piace la chiarezza - egli disse. 

- La chiarezza non è nella forma, ma nell'amore - ella disse, irritandosi sempre più non per le parole ma per il tono di fredda calma con cui egli parlava. - Perché lo desideri? 

``Dio mio, di nuovo a parlare dell'amore'' egli pensò, facendo una smorfia. 

- Ma lo sai perché: per te e per i figli che ci saranno - egli disse. 

- Figli non ce ne saranno. 

- È un gran peccato - egli disse. 

- Tu hai bisogno dei figli: e a me non pensi? - ella disse, avendo completamente dimenticato o non avendo sentito ch'egli aveva detto: per te e per i figli. 

La questione della possibilità di avere figli da lungo tempo era in discussione e la irritava. Il desiderio di lui di avere figli ella lo attribuiva al fatto ch'egli non apprezzasse la sua bellezza. 

- Ah, io ho detto: per te. Soprattutto per te - egli replicò, facendo una smorfia quasi di dolore - perché sono sicuro che la maggior parte della tua irritazione proviene dalla indeterminatezza della situazione. 

``Sì, ecco, adesso ha smesso di fingere, e si vede tutto il suo freddo odio verso di me'' ella pensò, senza ascoltare le parole, ma esaminando con orrore quel freddo e crudele giudice che, stuzzicandola, guardava dagli occhi di lui. 

- La ragione non è questa - ella disse - e io non capisco neppure come la causa di quella che tu chiami mia irritazione possa essere il fatto ch'io sia completamente in tuo potere. Che indeterminatezza di situazione c'è mai qui? al contrario. 

- Mi spiace molto che tu non voglia capire - la interruppe lui, desiderando d'esprimere il proprio pensiero: - l'indeterminatezza consiste nel fatto che a te pare ch'io sia libero. 

- Riguardo a questo puoi essere completamente tranquillo - ella disse e, voltategli le spalle, si mise a bere il caffè. 

Sollevò la tazza, staccando il mignolo, e l'accostò alla bocca. Dopo averne bevuti alcuni sorsi ella lo guardò e, dall'espressione del viso di lui, capì chiaramente che gli erano disgustosi la mano e il gesto e il suono ch'ella produceva con le labbra. 

- Per me è proprio indifferente quello che pensa tua madre e in quale maniera voglia darti moglie - ella disse, deponendo la tazza con la mano tremante. 

- Ma noi non parliamo di questo. 

- No, proprio di questo. E credi pure che per me una donna senza cuore, sia vecchia o no, tua madre o un'estranea, non m'interessa, e io non ne voglio sapere. 

- Anna, ti prego, non parlare senza rispetto di mia madre. 

- Una donna che non ha indovinato col cuore in che cosa consista la felicità e l'onore di suo figlio, quella donna non ha cuore. 

- Ti ripeto la mia preghiera: non parlare senza rispetto d'una madre che io rispetto - diss'egli, alzando la voce e guardandola severo. 

Ella non rispondeva. Guardando fisso lui, il suo viso, le sue mani, ricordò tutti i particolari della riconciliazione del giorno prima, e le carezze appassionate di lui. ``Queste carezze, proprio le stesse, le ha prodigate, le prodigherà e le vuol prodigare ad altre donne!'' ella pensava. 

- Tu non ami tua madre. Sono tutte frasi, frasi e frasi! - disse lei, guardandolo con odio 

- E se è così, allora bisogna\ldots{} 

- Bisogna decidersi e io mi son decisa - ella disse, e voleva andarsene, ma intanto entrò nella stanza Jašvin. Anna lo salutò e si fermò. 

Perché in quel momento, in cui nell'anima sua c'era tempesta ed ella sentiva d'essere a una svolta della vita che poteva avere orribili conseguenze, perché proprio in quel momento ella avesse bisogno di fingere davanti a un essere estraneo, che presto o tardi avrebbe pur saputo tutto, non lo sapeva; ma, calmata immediatamente in sé la tempesta interiore, si sedette e prese a parlare con l'ospite. 

- Be', come va il vostro affare? avete avuto il vostro credito? - disse a Jašvin. 

- Ma nulla; pare che non riceverò tutto, e mercoledì bisogna andar via. E voi quando? - disse Jašvin, guardando accigliato Vronskij e indovinando, evidentemente, la lite avvenuta. 

- Sembra domani l'altro - disse Vronskij. 

- Voi, del resto, vi preparate da un pezzo. 

- Ma ormai decisamente - disse Anna, guardando diritto negli occhi Vronskij con uno sguardo tale che gli diceva di non pensare neppure alla possibilità di una riconciliazione. - Possibile che non vi faccia pena quel disgraziato di Pevcov? - continuò la conversazione con Jašvin. 

- Non mi sono mai domandato, Anna Arkad'evna, se mi faceva pena o non mi faceva pena. Perché tutto il mio patrimonio è qui - egli mostrò la tasca laterale - e adesso sono un uomo ricco; ma oggi andrò al club e forse ne uscirò pezzente. Perché quegli che siede al tavolo con me vuol lasciarmi senza la camicia, e io lui. Lottiamo, in questo sta il gusto. 

- Via, e se foste ammogliato? - disse Anna - come farebbe vostra moglie? 

Jašvin si mise a ridere. 

- Proprio per questo, si vede, non mi sono ammogliato, e non ne ho mai avuta l'intenzione. 

- E Helsingfors? - disse Vronskij, entrando nella conversazione e guardando Anna che aveva sorriso. Nell'incontrare lo sguardo di lui, il viso di Anna, d'un tratto, prese un'espressione dura, come a dirgli: ``Non è dimenticato. È sempre lo stesso''. 

- Possibile che non siate stato mai innamorato? - ella disse a Jašvin. 

- Oh Signore! quante volte! Ma, capirete, uno può sedersi a giocare a carte, pronto ad alzarsi quando è l'ora d'un rendez-vous. Io, invece, posso occuparmi d'amore solo per quel tempo che mi consenta di non arrivare in ritardo la sera alla partita. Sistemo sempre così le cose. 

- No, non domando di questo, ma di quello che è stato. - Ella avrebbe voluto dire ``Helsingfors'', ma non voleva ripetere una parola detta da Vronskij. 

Venne Vojtov, che aveva comprato uno stallone; Anna si alzò e uscì dalla stanza. 

Prima di andar via, Vronskij passò da lei. Ella voleva fingere di cercar qualcosa sulla tavola, ma, vergognandosi di fingere, lo guardò diritto in faccia con uno sguardo freddo. 

- Di che cosa avete bisogno? - gli domandò in francese. 

- Di prendere il certificato per Gambetta, l'ho venduto - egli disse con un tono tale che esprimeva più chiaramente delle parole: ``per spiegarmi non ho tempo e non porterebbe a nulla''. 

``Io non sono colpevole in nulla verso di lei - egli pensava. - Se vuole punirsi, tant pis pour elle''. Ma uscendo, gli sembrò ch'ella avesse detto qualcosa, e il suo cuore tremò di pena per lei. 

- Cosa, Anna? - egli domandò. 

- Io, nulla - ella rispose con altrettanta freddezza e calma. 

``Ebbene, se è nulla, allora tant pis'' egli pensò, divenuto di nuovo freddo, si voltò e uscì. Uscendo vide nello specchio il viso di lei, pallido, con le labbra tremanti. Voleva fermarsi e dirle una parola per consolarla, ma le gambe lo portarono via dalla stanza, prima che avesse pensato cosa dire. Tutta quella giornata la passò fuori di casa e, quando venne la sera tardi, la donna gli disse che Anna Arkad'evna aveva mal di capo e lo pregava di non entrare da lei. 

\capitolo{XXVI}\label{xxvi-5} 

Non era ancora mai passato un intero giorno in lite. Quel giorno era la prima volta. E non era una lite. Era l'evidente ammissione d'un definitivo raffreddamento. Le si poteva forse lanciare uno sguardo quale egli le aveva lanciato quando era entrato nella stanza a prendere il certificato? Guardarla, vedere che il suo cuore si spezzava di disperazione e passarle accanto con quel viso impassibile e calmo? Non solo egli si era raffreddato verso di lei, ma la odiava, perché amava un'altra donna, era chiaro. 

E, ricordando tutte le parole crudeli ch'egli le aveva detto, Anna inventava ancora le parole che, evidentemente, egli avrebbe desiderato e potuto dirle, e s'irritava ancora di più. 

``Io non vi trattengo - egli poteva dirle. - Potete andare dove volete. Non avete voluto divorziare da vostro marito per tornare a lui, probabilmente. Tornate. Se avete bisogno di denaro, ve ne darò. Di quanti rubli avete bisogno?''. 

Tutte le parole più crudeli che può dire un uomo volgare, egli le diceva a lei nell'immaginazione sua, e lei non gliele perdonava, come se realmente egli gliele avesse dette. 

``E non è appena ieri che m'ha giurato amore, lui, uomo sincero e onesto? Non mi son forse disperata senza ragione già altre volte?'' si diceva dopo. 

Tutta quella giornata, tranne le due ore che passò dalla Wilson, Anna visse nel dubbio se tutto era finito o se c'era speranza di rappacificarsi, se doveva partire subito o vederlo ancora una volta. L'aveva aspettato tutto il giorno, e la sera, ritirandosi in camera sua, dopo aver ordinato di dire che aveva mal di capo, aveva pensato: ``Se egli verrà, malgrado le parole della cameriera, allora vuol dire che mi ama ancora. Altrimenti vuol dire che tutto è finito e allora deciderò quello che devo fare!''. 

La sera sentì il rumore del carrozzino di lui che si fermava, sentì la sua scampanellata, i suoi passi e la conversazione con la donna: egli aveva creduto quanto gli dicevano, non aveva voluto indagare ed era andato in camera sua. Tutto era finito, dunque. 

E la morte, come l'unico mezzo per far tornare nel cuore di lui l'amore, per punirlo e riportare vittoria in quella lotta che lo spirito del male, stabilitosi nel cuore di lei, conduceva con lui, le apparve chiaramente e con vivezza. 

Adesso era indifferente: andare o non andare a Vozdvizenskoe, ricevere o non ricevere il divorzio dal marito, tutto era inutile. Una cosa sola era necessaria: punirlo. 

Quando ebbe versato la solita dose d'oppio ed ebbe pensato che bastava soltanto bere tutta la fiala per morire, questo le parve così facile e semplice, che si mise a pensare di nuovo con piacere come egli si sarebbe tormentato, pentito, come avrebbe amato la sua memoria, quando sarebbe stato ormai troppo tardi. Ella giaceva nel letto con gli occhi aperti, guardando, alla luce di una candela che stava per spegnersi, la cornice modellata del soffitto e l'ombra di un paravento che ne invadeva una parte, e immaginava con chiarezza quello ch'egli avrebbe provato quando lei non ci sarebbe stata più e sarebbe stata soltanto un ricordo per lui. ``Come ho potuto dirle quelle parole crudeli? - avrebbe detto. - Come ho potuto uscir dalla stanza senza dirle nulla? Ma adesso lei non c'è più. Se n'è andata per sempre da noi. È là\ldots{}''. A un tratto l'ombra del paravento tentennò, invase tutta la cornice, tutto il soffitto, altre ombre dall'altra parte le si precipitarono incontro, per un attimo le ombre corsero via, ma poi avanzarono con rinnovata rapidità, tentennarono un po', si confusero, e tutto si fece buio. ``La morte!'' pensò. E un tale terrore la prese, che a lungo non poté capire dov'era e a lungo non poté trovare con le mani tremanti i fiammiferi e accendere un'altra candela al posto di quella che s'era consumata e spenta. ``No, tutto pur di vivere! Perché io l'amo. Perché lui mi ama! Questo è stato e passerà'' ella diceva, sentendo che le lacrime della gioia del ritorno alla vita le scorrevano per le guance. E, per liberarsi dal terrore, andò in fretta da lui nello studio. 

Nello studio egli dormiva di un sonno profondo. Gli si avvicinò e, illuminandogli il viso dall'alto, lo guardò a lungo. Adesso, quando dormiva, lo amava tanto che nel vederlo non poteva trattenere le lacrime di tenerezza; ma sapeva che, svegliandosi, l'avrebbe guardata con uno sguardo freddo, cosciente di aver ragione, e che, prima di parlargli del proprio amore, ella non avrebbe potuto non dimostrargli come egli fosse colpevole dinanzi a lei. Tornò in camera sua senza svegliarlo, e dopo una seconda dose di oppio, verso l'alba, si addormentò di un sonno pesante, non pieno, durante il quale non cessò di sentire se stessa. 

La mattina un incubo pauroso, che le era apparso varie volte nei sogni, ancora prima del suo legame con Vronskij, le apparve di nuovo e la fece svegliare. Un vecchietto con la barba arruffata faceva qualcosa, chino su di un ferro, mentre diceva delle parole francesi senza senso, e lei come sempre in quell'incubo (ciò che ne formava proprio l'orrore), sentiva che quel vecchio non faceva nessun caso a lei. E si svegliò coperta di un sudore freddo. 

Quando si fu alzata, le venne in mente, come in una nebbia, la giornata precedente. 

``C'è stata una lite. C'è stato quello che è già accaduto altre volte. Io ho detto che avevo mal di capo, e lui non è entrato. Domani andiamo via, bisogna vederlo e prepararsi per la partenza'' ella si disse. E avendo saputo ch'egli era già nello studio, andò da lui. Passando per il salotto sentì fermarsi all'ingresso una vettura, e, guardando dalla finestra, vide una vettura dalla quale si affacciava una fanciulla con un cappellino lilla, che ordinava qualcosa al cameriere che bussava. Dopo un parlottio in anticamera, qualcuno andò su, e, accanto al salotto, si sentirono i passi di Vronskij. Egli scendeva le scale a passo svelto. Ecco, era uscito senza cappello sulla scalinata e s'era avvicinato alla vettura. La fanciulla col cappellino lilla gli consegnò un pacchetto. Vronskij le disse qualcosa sorridendo. La vettura si allontanò; lui tornò, correndo rapido su per la scala. 

La nebbia, che avvolgeva tutto nell'animo di lei, si dissipò a un tratto. I sentimenti del giorno prima strinsero con rinnovato dolore il cuore malato. Adesso non poteva capire come avesse potuto umiliarsi tanto da passare tutta una giornata con lui, in casa sua. Ella entrò nello studio per annunciargli la propria decisione. 

- È la Sorokina con la figlia che è passata e m'ha portato i denari e le carte da parte di Ma. Ieri non ho potuto riceverli. Come va il tuo mal di capo, meglio? - egli disse tranquillo, senza desiderar di scorgere e intendere l'espressione cupa e grave del viso di lei. 

Ella lo guardava in silenzio, fissa, rimanendo in piedi al centro della stanza. Egli la guardò, si accigliò per un attimo e seguitò a leggere una lettera. Lei si voltò e andò via lentamente dalla stanza. Egli poteva ancora farla tornare, ma ella giunse fino alla porta, e lui taceva sempre, e si sentiva soltanto il fruscio del foglio di carta girato. 

- Sì, a proposito - disse egli, mentre lei era già sulla porta - domani andiamo via decisamente, vero? 

- Voi, ma non io - ella disse, voltandosi verso di lui. 

- Anna, così è impossibile vivere\ldots{} 

- Voi, ma non io - ella ripeté. 

- Diventa insopportabile! 

- Voi\ldots{} voi ve ne pentirete - ella disse e uscì. 

Spaventato dall'espressione disperata con cui erano state dette queste parole, egli saltò su e voleva correrle dietro, ma, ritornato in sé, sedette di nuovo e, stretti fortemente i denti, aggrottò le sopracciglia. Quella minaccia, informe, com'egli la riteneva, d'un qualche cosa, lo irritò. ``Ho provato tutto - pensò - rimane una cosa sola: non farci caso'' e cominciò a prepararsi ad andare in città e di nuovo dalla madre, dalla quale bisognava ricevere la firma per le procure. 

Ella sentì il suono dei passi di lui nello studio e nella sala da pranzo. Vicino al salotto egli si fermò. Ma non voltò per andare da lei, diede soltanto l'ordine che consegnassero lo stallone a Vojtov in sua assenza. Poi ella sentì come facevano venire avanti il carrozzino, come si apriva la porta ed egli ne usciva di nuovo. Ma ecco, egli rientrava nel vestibolo, e qualcuno veniva su di corsa. Era il cameriere che veniva a prendere i guanti dimenticati. Ella si avvicinò alla finestra e vide che, senza guardare, egli prendeva i guanti e, toccata con la mano la schiena del cocchiere, diceva qualcosa. Poi, senza guardare le finestre, sedette nella sua solita posa nel carrozzino, poggiando una gamba sull'altra, e, infilando un guanto, scomparve dietro l'angolo. 

\capitolo{XXVII}\label{xxvii-5} 

``È andato via! È finita!'' si disse Anna, in piedi accanto alla finestra, e, in risposta a questo problema, le impressioni del buio, per la candela che s'era spenta, e del sogno terribile si fusero in una, riempiendole il cuore di fredda paura. 

``No, questo è impossibile!'' gridò e, attraversata la stanza, sonò. Le sembrava così pauroso restar sola ora, che, senza aspettare che giungesse il cameriere, gli andò incontro. 

- Informatevi dove è andato il conte - ella disse. 

L'uomo rispose che il conte era andato alle scuderie. 

- Ha ordinato di dirvi che se desiderate uscire, il carrozzino ritornerà subito. 

- Va bene. Aspettate. Scrivo subito un biglietto. Mandate Michajla col biglietto alle scuderie. Presto. 

Sedette e scrisse: 

``Sono colpevole. Torna a casa, dobbiamo spiegarci. In nome di Dio vieni, sono spaventata''. 

Suggellò e consegnò all'uomo. 

Aveva paura di rimaner sola, adesso, e, dietro all'uomo, uscì dalla stanza e andò in quella dei bambini. 

``Ma non è lui, non è lui! Dove sono i suoi occhi azzurri, il caro e timido sorriso?'' questo fu il primo suo pensiero, quando vide la bambina, rossa e paffuta, con i capelli neri ondulati, invece di Serëza che, nella confusione delle idee, ella s'aspettava di vedere nella camera dei bambini. La bimba, sedendo alla tavola, la batteva con forza e ostinazione con un turacciolo, e guardava senza espressione la madre con due occhi neri simili a more. Dopo aver risposto all'inglese che si sentiva bene e che l'indomani partiva per la campagna, Anna sedette accanto alla piccina e cominciò a far girare davanti a lei il turacciolo della caraffa. Ma il riso forte e sonoro della bambina e un movimento ch'ella fece con un sopracciglio le ricordarono con tanta vivezza Vronskij che, trattenendo i singhiozzi, si alzò in fretta e uscì. ``Possibile che tutto sia finito? No, non può essere - ella pensava - egli tornerà. Ma come mi spiegherà quel sorriso, quell'animazione dopo aver parlato con lei? Ma anche se non lo spiegherà, tuttavia ci crederò. Se non ci crederò, allora mi rimane una cosa sola\ldots{} e non voglio''. 

Guardò l'orologio. Erano passati dodici minuti. ``Adesso ha già ricevuto il biglietto e torna indietro. Ma è poco, ancora dieci minuti\ldots{} Ma cosa sarà se non viene? No, questo non può essere. Bisogna che non mi veda con gli occhi rossi di pianto. Andrò a lavarmi. Sì, sì, mi sono pettinata o no? - si domandò. Tastò la testa con la mano. - Sì, mi sono pettinata, ma quando, non lo ricordo assolutamente''. Non credeva neppure alla propria mano e si avvicinò alla specchiera per veder se era veramente pettinata o no. Era pettinata e non poteva ricordare quando l'avesse fatto. 

``Chi è?'' pensava, guardando nello specchio il proprio viso infiammato, con gli occhi stranamente scintillanti, che la fissavano con spavento. ``Ma sono io'' ella capì a un tratto e, osservandosi tutta, sentì su di sé i baci di lui e, rabbrividendo, scosse le spalle. Poi sollevò una mano alle labbra e la baciò. 

``Cos'è impazzisco?'' e andò nella stanza da letto dove c'era Annuška che rassettava la camera. 

- Annuška - disse, fermandosi davanti a lei, guardando la cameriera, senza sapere lei stessa quel che le avrebbe detto. 

- Volevate andare da Dar'ja Aleksandrovna - disse la cameriera come se avesse capito. 

- Da Dar'ja Aleksandrovna? Sì, andrò. 

``Quindici minuti per andare, quindici per tornare indietro. Egli sta per venire, arriverà subito - e tirò fuori l'orologio e lo guardò. - Ma come ha potuto andar via, lasciandomi in uno stato simile? Come può vivere senza far pace con me?''. Si avvicinò alla finestra e si mise a guardare per la strada. Come tempo, egli avrebbe potuto già tornare. Ma il calcolo poteva non essere giusto, ed ella si diede a ricordare nuovamente quando era andato via, e a calcolare i minuti. 

Mentre si allontanava verso la pendola grande per controllare l'ora, qualcuno giunse in vettura. Guardando dalla finestra vide il carrozzino di lui. Ma nessuno saliva la scala e giù si sentivano delle voci. Era l'inserviente che tornava col carrozzino. Ella gli scese incontro. 

- Il conte non s'è trovato. Era partito per la linea di Niznij-Novgorod. 

- Che c'è? cosa?\ldots{} - ella disse rivolta al rubizzo e allegro Michajla che le rendeva il biglietto. ``Ma lui dunque non l'ha ricevuto'' ella ricordò. 

- Va' con questo stesso biglietto, in campagna, dalla contessa Vronskaja, sai? E porta immediatamente la risposta - disse all'inserviente. 

``E io che farò mai? - pensò. - Sì, andrò da Dolly, è vero, se no impazzisco. Sì, posso ancora telegrafare''. E scrisse un telegramma: 

``Mi è indispensabile parlarvi, venite subito''. 

Spedito il telegramma andò a vestirsi. Già pronta e col cappello, guardò negli occhi Annuška, grassa e calma. Si vedeva una compassione manifesta in quei piccoli, buoni occhi grigi. 

- Annuška, cara, che devo fare? - disse Anna, singhiozzando, lasciandosi andare sopra una poltrona. 

- Ma perché vi inquietate tanto, Anna Arkad'evna! Questo succede. Andate, vi distrarrete - disse la cameriera. 

- Sì, andrò - disse Anna tornando in sé e alzandosi. - E se verrà un telegramma quando non ci sarò, mandatemelo da Dar'ja Aleksandrovna. No, tornerò io stessa. 

``Sì, non bisogna stare a pensare, bisogna fare qualcosa, andare, soprattutto, andar via da questa casa - ella disse, prestando ascolto con orrore al terribile ribollimento che avveniva nel suo cuore, e in fretta uscì e salì nel carrozzino. 

- Dove comandate? - Domandò Pëtr, prima di sedersi a cassetta. 

- Alla Znamenka, dagli Oblonskij. 

\capitolo{XXVIII}\label{xxviii-5} 

Il tempo era chiaro. Tutta la mattina era caduta giù una pioggerella fitta, minuta, e adesso s'era schiarito da poco. I tetti di ferro, le lastre dei marciapiedi, i ciottoli del selciato, le ruote e il cuoio, il rame e lo stagno delle carrozze, tutto luccicava vivido al sole di maggio. Erano le tre ed era l'ora più animata per le strade. 

Sedendo in un angolo del comodo carrozzino, che si dondolava appena sulle molle elastiche all'andatura veloce dei cavalli grigi, Anna, in mezzo al frastuono incessante delle ruote e alle impressioni che si succedevano rapide all'aria aperta, esaminando di nuovo uno dopo l'altro gli avvenimenti degli ultimi giorni, vide la propria situazione completamente diversa da come le era sembrata a casa. Adesso anche il pensiero della morte non le sembrava più così terribile e chiaro, e la morte stessa non le appariva più inevitabile. Adesso si rimproverava l'umiliazione alla quale s'era lasciata andare. ``Lo supplico di perdonarmi. Mi sono sottomessa a lui. Mi sono riconosciuta colpevole. Perché? Non posso forse vivere senza di lui?''. E senza rispondere alla domanda come avrebbe vissuto senza di lui, si mise a leggere le insegne. ``Ufficio e deposito. Dentista\ldots{} Sì, dirò tutto a Dolly. Vronskij non le piace. Proverò vergogna, dolore, ma le dirò tutto. Lei mi vuole bene, e io seguirò il suo consiglio. Non mi assoggetterò a lui; non gli permetterò di plasmarmi. Filippov, ciambelle\ldots{} Dicono che portino la pasta a Pietroburgo. L'acqua di Mosca è così buona. E i pozzi e i biscotti di Mytišci''. 

E ricordò come molto tempo addietro, quando aveva appena diciassette anni, c'era andata con la zia per la Pentecoste. ``Ancora coi cavalli. Possibile che fossi io, con le mani rosse? Tante cose di quelle che allora mi sembravano così splendide e irraggiungibili sono diventate insignificanti, e quello che c'era allora, adesso è irraggiungibile per sempre. Avrei creduto, allora, di poter arrivare a tanta umiliazione? Come sarà orgoglioso e soddisfatto, per aver ricevuto il mio biglietto! Ma io gli dimostrerò\ldots{} Che cattivo odore ha questa vernice! Perché non fanno che verniciare e costruire? Mode e confezioni'' ella leggeva. Un uomo la salutò. Era il marito di Annuška. ``I nostri parassiti - ella ricordò come diceva Vronskij. - I nostri? Perché i nostri? È orribile che non si possa estirpare dalla radice il passato. Non si può estirpare, ma se ne può sperdere la memoria. E io la sperderò''. E a questo punto ricordò il suo passato con Aleksej Aleksandrovic, come l'avesse cancellato dalla propria memoria. ``Dolly penserà che io abbandono il secondo marito e che perciò certamente ho torto. Voglio forse aver ragione, io? Non posso!'' si disse, e le venne voglia di piangere. Ma si mise immediatamente a pensare di che cosa potessero sorridere tanto quelle due ragazze. ``Forse a proposito dell'amore? Non sanno come sia poco allegro, come sia vile\ldots{} Il viale e i bambini. Tre ragazzi corrono, giuocano ai cavalli. Serëza! E io perderò tutto e non farò tornare lui. Sì, tutto è perduto, s'egli non torna. Forse è arrivato in ritardo per il treno, e adesso è già tornato. Ecco, vuoi un'altra umiliazione! - disse a se stessa. - No, io andrò da Dolly e le dirò apertamente: sono infelice, me lo merito, sono colpevole, ma sono così infelice, aiutami! Questi cavalli, questo carrozzino, come mi vedo ripugnante in questo carrozzino. Tutto è suo, ma non vedrò più nulla''. 

Immaginando le parole con le quali avrebbe detto tutto a Dolly e avvelenandosi deliberatamente il cuore, Anna cominciò a salire la scala. 

- C'è qualcuno? - chiese in anticamera. 

- Katerina Aleksandrovna Levina - rispose il cameriere. 

``Kitty, quella stessa Kitty di cui è stato innamorato Vronskij! - pensò Anna. - Quella stessa che egli ricordava con amore. Si rammarica di non averla sposata. E di me si ricorda con odio, e si rammarica d'essersi unito a me''. 

Fra le sorelle, quando Anna arrivò, si parlava dell'allattamento. Dolly uscì sola incontro all'ospite, che in quel momento disturbava la loro conversazione. 

- Ah, non sei ancora partita? Volevo venire da te - disse - oggi ho ricevuto una lettera da Stiva. 

- Anche noi abbiamo ricevuto un telegramma - rispose Anna, voltandosi per vedere Kitty. 

- Scrive che non riesce a capire che cosa precisamente voglia Aleksej Aleksandrovic, ma che non partirà senza una risposta. 

- Pensavo che da te vi fosse qualcuno. Si può leggere la lettera? 

- Sì, Kitty - disse Dolly confusa - è rimasta nella camera dei bambini. È stata molto malata. 

- L'ho sentito. Si può leggere la lettera? 

- La porto subito. Ma egli non rifiuta; al contrario Stiva spera - disse Dolly, fermandosi sulla porta. 

- Io non spero, e non lo desidero neanche - disse Anna. 

``Cos'è mai questo? Kitty considera umiliante per lei incontrarmi? - pensava Anna rimasta sola. - E forse ha ragione. Ma non è lei, che è stata innamorata di Vronskij, non è lei che deve dimostrarmelo, anche se è vero. Lo so che, nella mia situazione, non mi può ricevere nessuna donna per bene. Lo so che da quel primo momento gli ho sacrificato tutto. Ed ecco la ricompensa! Oh, come lo odio! E perché son venuta qua? Sto ancora peggio, mi è ancora più penoso''. Ella sentiva nell'altra stanza le voci delle sorelle che parlavano fra di loro. ``E che cosa dirò a Dolly adesso? Devo consolare Kitty con la mia infelicità, sottomettendomi alla sua protezione? No, ma anche Dolly non capirà nulla. Ed è inutile che le parli. Sarebbe interessante soltanto veder Kitty e farle vedere come disprezzo tutti e tutto, come per me, adesso, tutto sia indifferente''. 

Dolly entrò con la lettera. Anna la lesse e la consegnò in silenzio. 

- Tutto questo lo sapevo - disse. - E non mi interessa affatto. 

- Ma perché poi? Io, al contrario, spero - disse Dolly, guardando Anna con curiosità. Non l'aveva mai vista in uno stato così strano, irritato. - Tu quando vai via? - ella domandò. 

Anna, socchiusi gli occhi, guardava davanti a sé e non le rispondeva. 

- Ebbene, Kitty si nasconde per non vedermi? - disse, guardando la porta e arrossendo. 

- Oh, che sciocchezze! Dà il latte e la cosa non procede bene, le stavo consigliando\ldots{} È molto contenta. Verrà subito - diceva Dolly con imbarazzo, non sapendo dire quello che non era vero. - Ma eccola. 

Avendo saputo che era venuta Anna, Kitty non voleva venir fuori; ma Dolly l'aveva persuasa. Raccolte le proprie forze, Kitty comparve, e, arrossendo, si avvicinò e le diede la mano. 

- Sono molto contenta - disse con voce tremante. 

Kitty era sconcertata dalla lotta che avveniva in lei fra l'inimicizia verso quella donna perversa e il desiderio di esserle indulgente; ma non appena vide il viso bello, simpatico di Anna, tutta l'inimicizia scomparve immediatamente. 

- Non mi sarei sorpresa se non aveste neppure voluto incontrarvi con me. Sono abituata a tutto. Siete stata malata? Sì, siete cambiata - disse Anna. 

Kitty sentiva che Anna la guardava con ostilità. Ella spiegò questa ostilità con la situazione di disagio in cui si sentiva adesso, di fronte a lei, Anna che prima la proteggeva, e ne provò pena. 

Parlarono della malattia, del bambino, di Stiva, ma evidentemente nulla interessava Anna. 

- Sono passata a salutarti - diss'ella, alzandosi. 

- E quando andate via? 

Ma Anna si voltò di nuovo verso Kitty, senza rispondere. 

- Sì, sono molto contenta d'avervi vista - ella disse con un sorriso. - Ho tanto sentito parlare di voi da tutte le parti, perfino da vostro marito. È stato da me, e m'è piaciuto molto - soggiunse con un'evidente intenzione perversa. - Dov'è? 

- È andato in campagna - disse Kitty, arrossendo. 

- Salutatelo da parte mia, salutatelo senza meno. 

- Senza meno! - ripeté ingenuamente Kitty, guardandola negli occhi con pena. 

- Allora addio, Dolly - e, baciata Dolly e stretta la mano a Kitty, Anna uscì frettolosa. 

- Sempre la stessa e sempre così affascinante. È molto bella! - disse Kitty , rimasta sola con la sorella. - Ma c'è qualcosa in lei che fa pena. Tanta pena! 

- No, oggi in lei c'è qualcosa di strano - disse Dolly. - Quando l'ho accompagnata in anticamera, m'è parso che avesse voglia di piangere. 

\capitolo{XXIX}\label{xxix-5} 

Anna sedette nel carrozzino in uno stato peggiore di quello in cui era uscita di casa. Ai tormenti di prima s'era unito adesso un senso di offesa e di ripulsione che aveva chiaramente avvertito nell'incontro con Kitty. 

- Dove andate? a casa? - chiese Pëtr. 

- Sì a casa - disse lei, senza neppur pensare dove andasse ora. 

``Come mi guardavano, come qualcosa di terribile, di incomprensibile e di curioso! Che cosa può raccontare quello lì all'altro con tanto calore? - ella pensava, guardando due passanti. - Si può forse raccontare ciò che si sente a un altro? Io volevo raccontarlo a Dolly ed è stato bene che non l'abbia fatto. Come sarebbe stata contenta della mia sventura! L'avrebbe nascosto; ma il sentimento principale sarebbe stata la gioia che io fossi punita per quel piacere che lei mi invidiava. Kitty poi sarebbe stata ancor più contenta. Come la vedo tutta da parte a parte! Sa che io sono stata più gentile del solito verso suo marito. Ed è gelosa di me e mi odia. E mi disprezza, per giunta. Ai suoi occhi io sono una donna immorale. Se fossi una donna immorale avrei potuto fare innamorare di me suo marito\ldots{} se avessi voluto. Ma io non lo volevo neanche. Quello lì è contento di sé - pensò di un signore grasso, rosso in viso, che veniva verso di lei in carrozza, il quale, scambiandola per una conoscente, aveva sollevato il cappello lucido sopra la lucida testa calva e poi s'era convinto d'essersi sbagliato. - Pensava di conoscermi. E mi conosce così poco, come poco mi conosce chiunque altro al mondo. Io stessa non mi conosco. Conosco i miei appetiti, come dicono i francesi. Ecco, loro desiderano questo gelato sporco. Questo, loro lo sanno con sicurezza - pensava, guardando due ragazzini che avevano fermato un gelataio, che si toglieva di capo il recipiente e s'asciugava con l'orlo dell'asciugamano il viso sudato. - Tutti noi desideriamo roba dolce, buona. Se non ci sono confetti, allora gelato sporco. E Kitty lo stesso: se non Vronskij, allora Levin. E mi invidia. E mi odia. E tutti noi ci odiamo a vicenda. Io Kitty, Kitty me. Ecco, questa è la verità. Tjut'kin, coiffeur\ldots{} Je me fais coiffer par Tjut'kin\ldots{} Glielo dirò, quando arriverà - pensò e sorrise. Ma nello stesso momento si ricordò che non aveva nessuno cui dire qualcosa di divertente. - E poi non c'è nulla di ameno, di allegro. Tutto è disgustoso. Suonano a vespro, e questo mercante si fa il segno della croce con tanta cura come se temesse di lasciarsi sfuggire qualcosa. Perché queste chiese, questo suono, questa menzogna? Soltanto per nascondere che ci odiamo tutti a vicenda, come questi vetturali che si ingiuriano con tanta cattiveria. Jašvin dice: `lui vuol lasciare me senza camicia, e io lui'. Ecco, questa è la verità!''. 

In questi pensieri, che l'avevano tanto presa da non farla pensare più alla propria situazione, la sorprese l'arrestarsi della carrozza vicino ai gradini di casa sua. Visto il cocchiere che le veniva incontro, ricordò soltanto allora d'aver spedito il telegramma e il biglietto. 

- C'è risposta? - domandò. 

- Guardo subito - rispose il portiere e, data un'occhiata al banco, tirò fuori e le porse la busta sottile, quadrata di un telegramma ``Non posso arrivare prima delle dieci. Vronskij'' ella lesse. 

- E l'inserviente non è tornato? 

- Nossignora - rispose il portiere. 

``E se è così, so quello che devo fare - ella disse e, sentendo insorgere in sé un'ira indefinita e un bisogno di vendetta, andò sopra di corsa. - Andrò io stessa da lui. Prima di partire per sempre gli dirò tutto. Non ho mai odiato nessuno come quest'uomo!'' ella pensava. Visto il cappello di lui all'attaccapanni, rabbrividì di repulsione. Non considerava che il telegramma di lui era la risposta al suo telegramma, e ch'egli non aveva ancora ricevuto il biglietto. Lo immaginava mentre con la madre e con la Sorokina discorreva tranquillo e gioiva delle sofferenze di lei. ``Sì, bisogna andare presto'' si disse senza pensare dove andare. Desiderava staccarsi al più presto dalle sensazioni che provava in quell'orribile casa. La servitù, i muri, gli oggetti, qui tutto suscitava in lei repulsione e rancore e l'opprimeva come un peso. 

``Sì, bisogna andare alla stazione ferroviaria, e se no, allora, andare là e coglierlo sul fatto''. Anna guardò nei giornali l'orario dei treni. La sera, il treno partiva alle otto e due minuti. ``Sì, farò in tempo''. Ordinò di attaccare altri cavalli e si occupò di mettere in una sacca da viaggio le cose indispensabili per qualche giorno. Sapeva che non sarebbe più tornata lì. Aveva deciso confusamente, fra i progetti che le erano venuti in mente, anche questo, che, dopo quanto sarebbe accaduto, alla stazione o nella tenuta della contessa Vronskaja, sarebbe andata per la linea di Niznij-Novgorod fino alla prima stazione e sarebbe rimasta là. 

Il pranzo era in tavola; ella si avvicinò, annusò il pane e il formaggio, e, convintasi che l'odore di tutti i cibi le riusciva nauseante, ordinò di far venire la vettura e uscì. La casa gettava un'ombra che attraversava ormai tutta la strada, ed era una serata chiara, ancora tiepida al sole. E Annuška che l'accompagnava con la roba, e Pëtr che riponeva la roba nel carrozzino e il cocchiere, evidentemente scontento, tutti la nauseavano e la irritavano con le loro parole e i loro gesti. 

- Non ho bisogno di te, Pëtr. 

- E come si fa per il biglietto? 

- Be', come vuoi, per me è lo stesso - disse lei con stizza. 

Pëtr saltò a cassetta e, messosi le mani sui fianchi, ordinò di andare alla stazione. 

\capitolo{XXX}\label{xxx-5} 

``Ecco, di nuovo! Di nuovo capisco tutto'' si disse Anna, non appena il carrozzino si mosse e sobbalzando rintronò sul lastrico, e di nuovo, una dopo l'altra, cominciarono a succedersi le impressioni. 

``Sì, qual'è l'ultima cosa a cui pensavo così chiaramente? - cercava di ricordare. - Tjut'kin coiffeur? No, non è quello. Sì, quel che dice Jašvin: la lotta per l'esistenza e l'odio sono le uniche cose che leghino gli uomini. No, andate inutilmente - disse rivolta col pensiero a una compagnia, in un calessino dal tiro a quattro, che, evidentemente, andava a divertirsi fuori città. - E il cane che portate con voi non vi aiuterà. Non sfuggirete a voi stessi''. Gettato uno sguardo dalla parte verso la quale Pëtr si voltava, vide un operaio ubriaco fradicio, con la testa ciondoloni, che una guardia portava chi sa dove. ``Ecco, questo piuttosto - ella pensò. - Io e il conte Vronskij però non l'abbiamo provato questo piacere, sebbene ci aspettassimo molto da esso''. E per la prima volta rivolse quella luce chiara, in cui vedeva tutto, verso i propri rapporti con lui, ai quali prima aveva evitato di pensare. ``Che cercava egli in me? Non tanto l'amore quanto la soddisfazione della vanità''. Ricordò le parole di lui, l'espressione del viso, che somigliava a un docile cane da caccia, nei primi tempi del loro legame. E tutto adesso lo confermava. ``Sì, in lui c'era il trionfo del successo, della vanità. S'intende, c'era anche l'amore, ma la parte maggiore era l'orgoglio del successo. Egli si vantava di me. Adesso è passato. Non c'è di che essere orgoglioso, ma di che vergognarsi. M'ha preso tutto quello che poteva, e adesso non gli sono più necessaria. Sente il peso di me e cerca di non essere disonesto nei miei riguardi. Ieri se l'è lasciato sfuggire: vuole il divorzio e il matrimonio per bruciare le sue navi. Mi ama, ma come? The zest is gone. Questo qui vuole far colpo su tutti ed è molto soddisfatto di sé - pensava, guardando un commesso rosso in faccia che andava su di un cavallo da corsa. - Sì, quel gusto in me non lo trova più. Se andrò via da lui, in fondo all'anima sarà contento''. 

Non era una sua supposizione; ella vedeva ciò chiaramente in quella luce penetrante che le scopriva adesso il senso della vita e dei rapporti umani. 

``Il mio amore si fa sempre più appassionato ed egoistico, e il suo non fa che spegnersi, ecco perché ci dividiamo - ella seguitò a pensare. - E non vi si può rimediare. Io ho tutto in lui solo, e pretendo che egli mi si dia sempre di più. E lui sempre di più vuole allontanarsi da me. Noi, prima di giungere al nostro legame, ci siamo proprio andati incontro, così ora ci dividiamo andando irresistibilmente verso parti opposte. E cambiare questo non si può. Lui mi ha detto che sono insensatamente gelosa e io stessa mi sono detta che sono insensatamente gelosa; ma non è vero. Non sono gelosa, sono scontenta, invece. Ma\ldots{} - aprì la bocca e cambiò posto nel carrozzino per l'agitazione suscitata in lei dal pensiero che le era venuto a un tratto. - S'io potessi essere qualcos'altro, invece dell'amante che ama appassionatamente le sole sue carezze; ma io non posso e non voglio essere null'altro. E con questo desiderio io suscito in lui la repulsione, e lui in me il rancore, e non può essere altrimenti. Non so io, forse, che egli non si metterebbe a ingannarmi, che non ha intenzioni circa la Sorokina, che non è innamorato di Kitty, che non mi tradirà? Tutto questo lo so, ma per questo non sto meglio. Se lui, senza amarmi, sarà buono, tenero con me per dovere, e non ci sarà quello che io voglio, questo è mille volte peggiore anche dell'odio! Questo è l'inferno! Ed è proprio così. Lui non mi ama già più da tempo. E dove finisce l'amore, comincia l'odio\ldots{} Queste strade non le conosco per nulla. Vi sono delle montagnole, e poi sempre case, case\ldots{} E in queste case sempre uomini, uomini\ldots{} Quanti ce ne sono, e sono senza fine e tutti si odiano a vicenda. Ebbene, ammettiamo che io trovi quello che voglio per essere felice. Ecco. Ottengo il divorzio, Aleksej Aleksandrovic mi dà Serëza, e io sposo Vronskij''. Ricordatasi di Aleksej Aleksandrovic, immediatamente, con una straordinaria chiarezza, se lo raffigurò davanti a sé come vivo, con i suoi occhi mansueti, senza vita, spenti, le vene turchine sulle mani bianche, le intonazioni di voce e lo scricchiolio delle dita e, ricordatasi di quel sentimento che c'era stato fra di loro e che pure s'era chiamato amore, rabbrividì di repulsione. ``Allora dunque, otterrò il divorzio e sarò moglie di Vronskij. Ebbene, Kitty smetterà di guardarmi come mi guardava oggi? No. E Serëza smetterà di chiedere e di pensare ai miei due mariti? E fra me e Vronskij che sentimento nuovo inventerò mai? È possibile, non pure una qualche felicità, ma la fine del tormento? No e no! - ella si rispose adesso, senza la più piccola esitazione. - È impossibile! Noi siamo separati dalla vita, e io faccio la sua infelicità, lui la mia, e non si può rifare né lui, né me. Tutti i tentativi sono stati fatti, la vite s'è spanata\ldots{} Già, una mendicante con un bambino. Pensa che si provi pena di lei. Non siamo forse tutti gettati nel mondo per odiarci a vicenda, e poi tormentare noi stessi e gli altri? Passano degli studenti di ginnasio, ridono. Serëza? - si ricordò. - Anch'io pensavo di volergli bene, e mi commovevo dinanzi alla mia tenerezza. E ho vissuto senza di lui, e l'ho scambiato con un altro amore, e non mi sono lamentata di questo baratto finché mi sono contentata di quest'altro amore''. E ricordò con repulsione quello che chiamava ``quest'amore''. E la lucidità con cui ora vedeva la propria vita e quella di tutte le persone la rallegrava. ``Così siamo e io, e Pëtr, e il cocchiere Fëdor, e quel mercante, e tutte quelle persone che vivono là lungo la Volga, dove quegli avvisi invitano ad andare, e dappertutto e sempre'' ella pensava, mentre si era avvicinata alla costruzione bassa della ferrovia di Niznij-Novgorod e le erano corsi incontro i facchini. 

- Comandate il biglietto fino a Obiralovka? - disse Pëtr. 

Lei aveva completamente dimenticato dove e perché andava, e soltanto con un grande sforzo poté capire la domanda. 

- Sì - disse, tendendo il portamonete col denaro e, infilato al braccio il sacchetto rosso, uscì dal carrozzino. 

Dirigendosi fra la folla verso la sala d'aspetto di prima classe, ella riandava con la mente a tutti i particolari della sua situazione, a tutte le decisioni fra cui ondeggiava. E di nuovo ora la speranza, ora la disperazione cominciarono, nei soliti punti dolenti, ad avvelenare le ferite del suo cuore tormentato, che batteva paurosamente. Seduta su di un divano a forma di stella, in attesa del treno, guardando con ripugnanza quelli che entravano e uscivano (tutti erano disgustosi per lei), pensava ora come sarebbe arrivata alla stazione, o come gli avrebbe scritto un biglietto e cosa gli avrebbe scritto, ora come adesso egli si lamentasse con la madre della propria situazione (senza rendersi conto delle sofferenze di lei) e come lei sarebbe entrata nella stanza e cosa gli avrebbe detto. Ora pensava come avrebbe potuto essere ancora felice la vita e come lo amasse e lo odiasse tormentosamente, e come paurosamente le battesse il cuore. 

\capitolo{XXXI}\label{xxxi-5} 

Squillò un campanello, e passarono alcuni giovani, orribili, insolenti e frettolosi, nello stesso tempo intenti a cogliere l'impressione che producevano. Passò anche Pëtr attraverso la sala, con le ghette e la livrea, il viso ottuso e animalesco, e si avvicinò a lei per accompagnarla al treno. Gli uomini rumorosi fecero silenzio, mentre ella passava accanto a loro sulla banchina, e uno di loro mormorò qualcosa dietro di lei a un altro, qualcosa, si intende, di volgare. Ella salì sull'alto predellino e sedette sola in uno scompartimento su di un sudicio divano a molle che una volta era stato bianco. Il sacchetto rimbalzò sulle molle, e poi si fermò. Pëtr, in segno di addio, si tolse, presso il finestrino e con un sorriso ebete, il suo cappello gallonato; un capotreno insolente sbatté la porta e abbassò la maniglia. Una signora sgraziata, con un vestito ridicolmente ampio dietro (Anna col pensiero denudò quella donna e inorridì della sua deformità), con una bambina che rideva forzatamente, passarono di corsa lì sotto. 

- Da Katerina Andreevna, sempre da lei, ma tante! - gridò la bambina. 

``La bambina, anche quella è sfigurata e smorfiosa'' pensò Anna. Per non vedere nessuno si alzò svelta e sedette accanto al finestrino opposto nello scompartimento vuoto. Un informe contadino sudicio, con un berretto di sotto al quale spuntavano dei capelli arruffati, passò vicino a quel finestrino, chino verso le ruote della vettura. ``C'è qualcosa di noto in questo informe contadino'' pensò Anna. E, ricordatasi del sogno, si allontanò, tremando di paura, verso la parte opposta. Un capotreno apriva la porta, per fare entrare una coppia. 

- Desiderate uscire? 

Anna non rispose. Il capotreno e quelli che erano entrati non notarono, sotto il velo, il terrore sul viso di lei. Ella tornò nel suo angolo e sedette. La coppia sedette dalla parte opposta, esaminando con attenzione, sott'occhio, il vestito di lei. Sia il marito che la moglie sembravano ripugnanti ad Anna. Il marito domandò il permesso di fumare, evidentemente, non per fumare, ma per intavolare discorso con lei. Ottenutone il consenso, si mise a dire in francese alla moglie che ancor più che di fumare, aveva bisogno di parlare. Parlavano, fingendo, di sciocchezze, sol perché lei sentisse. Anna vedeva chiaramente che erano annoiati l'uno dell'altra e che si odiavano a vicenda. E non potevano non odiarsi simili pietosi esseri deformi. 

Si sentì un secondo campanello seguìto da un trasportar di bagagli, da grida e da risate. Per Anna era così chiaro che nessuno aveva di che rallegrarsi, che quelle risate la irritarono fino a farla soffrire e le venne la voglia di tapparsi le orecchie per non sentirle. Finalmente squillò un terzo campanello, echeggiò un fischio, si sentì uno stridio di catene, una forte scossa e il marito si fece il segno della croce. ``Sarebbe interessante chiedergli cosa intende con questo'' pensò Anna con cattiveria. Guardando di fianco alla moglie, ella osservava attraverso il finestrino le persone che avevano accompagnato i congiunti al treno e che stavano in piedi sulla banchina, e pareva proprio che andassero all'indietro. Scotendosi regolarmente sui binari, la vettura in cui era seduta Anna scivolò lungo la banchina, accanto a un muro di pietra, a un disco e ad altre vetture; con un suono sottile le ruote risonarono più scorrevoli e più oleate sulle rotaie; il finestrino s'illuminò del sole vivido della sera e un vento leggero si mise a giocare con la tendina. Anna dimenticò i suoi vicini di vagone e, al leggero dondolio della corsa, aspirando l'aria fresca, si mise di nuovo a pensare. 

``Sì, a che punto mi son fermata? Al fatto che non riesco a inventare una situazione in cui la vita non sia un tormento, che noi tutti siamo creati per tormentarci, e che noi tutti lo sappiamo e tutti inventiamo dei mezzi per ingannarci. E quando si vede la verità, che mai si può fare?''. 

- La ragione è data all'uomo per liberarsi di quello che lo inquieta - disse in francese la signora, evidentemente soddisfatta della propria frase e facendo smorfie con la lingua. 

Queste parole parvero rispondere al pensiero di Anna. 

``Liberarsi di quello che lo inquieta'' ripeté Anna. E, guardando il marito dalle guance rosse e la moglie magra, ella capì che la moglie malaticcia si considerava una donna incompresa e che il marito la ingannava, mantenendo in lei questa opinione su se stessa. Ad Anna pareva di vedere la loro storia e tutti gli angoli remoti dell'anima loro, mentre spostava su di essi la sua luce. Ma lì non c'era nulla di interessante, e continuò il suo pensiero. 

``Sì, mi agita molto, e la ragione è data per liberarsene; perciò bisogna liberarsene. E perché non spegnere la candela, quando non c'è più nulla da guardare, quando fa ribrezzo guardare tutto? Ma come? Perché questo capotreno è passato di corsa sulla traversa? perché gridano quei giovani, in quello scompartimento? Perché parlano, perché ridono? Tutto è menzogna, tutto inganno, tutto malvagità\ldots{}''. 

Quando il treno entrò in stazione, Anna uscì tra la folla degli altri passeggeri e, allontanandosi da loro come da lebbrosi, si fermò sulla banchina, cercando di ricordare perché era arrivata là e cosa avesse intenzione di fare. Tutto quello che prima le sembrava possibile, adesso era così difficile a considerarsi, specialmente tra la folla rumoreggiante di tutte quelle persone deformi, che non la lasciavano in pace. Ora i facchini accorrevano da lei, offrendole i loro servigi, ora dei giovani, battendo coi tacchi le assi della banchina e discorrendo forte, la esaminavano, ora quelli che venivano incontro si facevano di lato non dalla parte giusta. Ricordatasi che voleva proseguire, se non ci fosse stata risposta, fermò un facchino e domandò se era venuto un cocchiere con un biglietto per il conte Vronskij. 

- Il conte Vronskij? Per incarico suo sono stati qui proprio ora. Venivano incontro alla principessa Sorokina con la figlia. E il cocchiere com'è? 

Mentre ella parlava col facchino, Michajla, rosso e allegro, con un elegante pastrano turchino e la catena, evidentemente orgoglioso d'avere eseguito così bene la commissione, le si avvicinò e le porse un biglietto. Ella aprì e il cuore le si strinse ancor prima di leggere. 

``Mi dispiace molto che il biglietto non m'abbia trovato. Verrò alle dieci'' scriveva Vronskij con una scrittura trascurata. 

``Ecco! Me l'aspettavo!'' si disse con un sorriso cattivo. 

- Va bene, allora va' a casa - disse piano, rivolta a Michajla. Ella parlava piano perché la rapidità dei battiti del cuore le impediva di respirare. ``No, non ti permetterò di tormentarmi'' ella pensò, rivolta con minaccia, non a lui, né a se stessa, ma a chi le imponeva di tormentarsi, e si incamminò per la banchina lungo la stazione. 

Due cameriere che camminavano sulla banchina si voltarono a guardarla, facendo ad alta voce qualche apprezzamento sul suo vestito: ``sono veri'' dissero dei pezzi ch'ella aveva addosso. I giovani non la lasciavano in pace. Di nuovo le passarono accanto, guardandola in viso e gridando fra le risa qualcosa con voce contraffatta. Il capostazione, passando, le domandò se partiva. Un ragazzo, venditore di kvas, non le toglieva gli occhi di dosso. ``Dio mio, dove andare?'' ella pensava, allontanandosi sempre più sulla banchina. Alla fine si fermò. Le signore e i bambini, che erano venuti a incontrare un signore con gli occhiali e che ridevano e parlavano forte, tacquero, esaminandola, quand'ella giunse alla loro altezza. Ella affrettò il passo e si allontanò da loro verso l'orlo della banchina. Si avvicinava un treno merci. La banchina si mise a tremare e a lei parve d'essere di nuovo in viaggio. 

E a un tratto si ricordò dell'uomo schiacciato al suo primo incontro con Vronskij e capì quello che doveva fare. Dopo essere scesa con passo veloce, leggero, per i gradini che andavano verso le rotaie, si fermò accanto al treno che le passava vicinissimo. Guardava la parte sottostante dei carri, le viti e le catene e le ruote alte di ghisa del primo carro che scivolava lento, e cercava di stabilire con l'occhio il punto mediano fra le ruote anteriori e le posteriori e il momento in cui questo punto mediano sarebbe stato di fronte a lei. 

``Là - si diceva, guardando nell'ombra del carro la sabbia mista a carbone di cui erano sparse le traverse - là, proprio nel mezzo, e lo punirò, e mi libererò da tutti e da me stessa''. 

Voleva cadere sotto il primo vagone che giungesse alla sua altezza nel punto mediano; ma il sacchetto rosso che aveva preso a togliere dal braccio, la trattenne, ed era già tardi; il punto mediano le era passato accanto. Bisognava aspettare il vagone seguente. Un sentimento simile a quello che provava quando, facendo il bagno, si preparava a entrar nell'acqua, la prese, ed ella si fece il segno della croce. Il gesto abituale della croce suscitò nell'anima sua tutta una serie di ricordi verginali e infantili, e a un tratto l'oscurità che per lei copriva tutto si lacerò, e la vita le apparve per un attimo con tutte le sue luminose gioie passate. Ma ella non staccava gli occhi dalle ruote del secondo vagone che si avvicinava. E proprio nel momento in cui il punto mediano fra le ruote giunse alla sua altezza, ella gettò indietro il sacchetto rosso, ritirò la testa fra le spalle, cadde sulle mani sotto il vagone e con movimento leggero, quasi preparandosi a rialzarsi subito, si lasciò andare in ginocchio. E in quell'attimo stesso inorridì di quello che faceva. ``Dove sono? che faccio? perché?''. Voleva sollevarsi, ripiegarsi all'indietro, ma qualcosa di enorme, di inesorabile le dette un urto nel capo e la trascinò per la schiena. ``Signore, perdonami tutto!'' ella disse, sentendo l'impossibilità della lotta. Un contadino, dicendo qualcosa, lavorava su del ferro. E la candela, alla cui luce aveva letto il libro pieno di ansie e di inganni, di dolore e di male, avvampò di una luce più viva che mai, le schiarì tutto quello che prima era nelle tenebre, crepitò, prese ad oscurarsi e si spense per sempre. 
 
