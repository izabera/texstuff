\parte{PARTE SECONDA}\label{parte-seconda} 
\pagestyle{pagina}

\capitolo{I}\label{i-1} 

Alla fine dell'inverno si tenne un consulto in casa Šcerbackij per accertare quali fossero le condizioni di salute di Kitty e decidere cosa fare per ristabilirne le forze sempre più deboli. Il medico curante le aveva prescritto l'olio di fegato di merluzzo, poi il ferro, poi il nitrato di argento; ma poiché né questo, né quello, né l'altro avevano giovato ed egli consigliava di condurla all'estero, nella primavera fu fatto venire un medico di grido. Costui, bell'uomo ancor giovane, volle visitare l'ammalata. Insisteva con particolare compiacimento sul fatto che il pudore verginale è solo un residuo di barbarie, e che non vi è nulla di sconveniente che un medico, se pur non del tutto vecchio, visiti una ragazza tastandone il corpo svestito. Gli pareva del tutto naturale, gli capitava ogni giorno, e non sentiva e non pensava che potesse esservi nulla di male: e perciò considerava il pudore di una fanciulla non solo un residuo di barbarie, ma un'offesa alla propria persona. 

Era necessario piegarvisi, perché, sebbene anche gli altri medici avessero frequentato la stessa scuola e studiato sugli stessi libri e tutti fossero in possesso di una stessa scienza, e pur avendo costui presso alcuni fama di medico inetto, tuttavia nella casa della principessa e nella sua cerchia, chi sa perché, si riteneva che solo questo medico famoso sapesse qualcosa di speciale e solo lui potesse salvare Kitty. Dopo aver visitato e tastato attentamente l'ammalata, smarrita e stordita per la vergogna, il medico famoso, lavatesi accuratamente le mani, rimase in piedi nel salotto a parlare col principe. Il principe aggrottava le sopracciglia e tossiva nell'ascoltarlo. Come uomo vissuto, non sciocco e di sana costituzione, non credeva alla medicina e nell'animo suo si irritava contro tutta quella commedia, tanto più che egli era forse il solo a capire in pieno la causa del malanno di Kitty. ``Eccolo, lo spadellatore!'' pensava, adattando nel pensiero il termine venatorio al medico di grido e ascoltandone le dissertazioni sui sintomi della malattia della figlia. Il medico, da parte sua, tratteneva a stento un'espressione di dispregio verso il vecchio gentiluomo e si abbassava con degnazione al livello dell'intelligenza di lui. Capiva che col vecchio non c'era nulla da fare e che in quella casa il capo era la madre. Dinanzi a costei si proponeva quindi di profondere le sue millanterie. In quel momento la principessa entrò in salotto col medico curante. Il principe si allontanò, cercando di non far notare quanto per lui fosse ridicola tutta quella commedia. La principessa era smarrita e non sapeva cosa fare. Si sentiva colpevole di fronte a Kitty. 

- Ebbene, dottore, decidete la nostra sorte - disse la principessa. - Ditemi tutto. - E voleva dire: ``C'è speranza?'', ma le labbra le tremarono, e non poté pronunciare la domanda. - Dunque, dottore? - 

- Subito, principessa; conferirò con il mio collega e poi avrò l'onore di dirvi la mia opinione. 

- Allora vi dobbiamo lasciare? 

- Come volete. 

La principessa, dopo aver sospirato, uscì. 

Quando i dottori rimasero soli, il medico curante cominciò timidamente a sottoporre la sua opinione che consisteva nell'ammettere un principio di processo tubercolare, ma\ldots{} e via di seguito. Il medico famoso lo ascoltava e, nel mezzo del discorso, guardò l'orologio d'oro massiccio. 

- Già, - disse - ma\ldots{} 

Il medico curante tacque rispettosamente, a metà discorso. 

- Come voi sapete, un principio di processo tubercolare noi non possiamo diagnosticarlo; fino alla comparsa delle caverne non vi è nulla di positivo. Possiamo fare solamente delle ipotesi. E sintomi ce ne sono: denutrizione, eccitamento nervoso, ecc. La questione si pone in questi termini: supposto un processo tubercolare, che cosa bisogna fare per sostenere la nutrizione? 

- Ma voi sapete, del resto, come in questi casi si nascondano sempre ragioni morali, spirituali - si permise di far presente, con un sorriso delicato, il medico curante. 

- Già, s'intende - rispose il medico famoso, dopo aver guardato di nuovo l'orologio. - Ditemi, vi prego, è stato rimesso il ponte Jauzskij o bisogna ancora fare il giro? - chiese. - Ah, è a posto. Allora potrò trovarmi là in venti minuti. Dunque, dicevamo, la questione si pone in questi termini: sostenere la nutrizione e sistemare i nervi. L'una cosa è legata all'altra; bisogna battere sulle due parti del cerchio. 

- E il viaggio all'estero? - chiese il medico curante. 

- Io son nemico dei viaggi all'estero. E guardate un po': se c'è un principio di processo tubercolare, cosa che non possiamo sapere, il viaggio all'estero non aiuta. È indispensabile un mezzo che sostenga la nutrizione senza far danno. 

Il medico curante ascoltava attento e deferente. 

- Ma in favore del viaggio all'estero io farei notare il cambiamento di abitudini, l'allontanamento da quanto può suscitare ricordi. E poi la madre lo desidera - disse. 

- Ah, allora, in tal caso, che vadano pure; badino, però, che quei ciarlatani di tedeschi non abbiano a nuocere loro. Che si attengano\ldots{} Ma che vadano pure. 

E guardò di nuovo l'orologio. 

- Oh, è già ora - e andò verso la porta. 

Il medico famoso annunciò alla principessa (un senso di convenienza glielo suggeriva) che aveva bisogno di visitare ancora una volta l'ammalata. 

- Come, osservarla ancora una volta? - esclamò la madre spaventata. 

- Oh, no, mi occorrono alcuni particolari principessa. 

- Prego, favorisca. 

E la madre, seguita dal dottore, entrò nel salottino di Kitty. 

Smagrita e arrossata, con un particolare luccichio negli occhi pel suo pudore violato, Kitty stava al centro della stanza. Quando il dottore entrò, avvampò tutta e gli occhi le si empirono di lacrime. La malattia e le cure le sembravano una così sciocca e risibile cosa! La cura poi le sembrava ridicola tanto quanto la ricomposizione di un vaso rotto. Il suo cuore era spezzato. Perché la volevano curare con polverine e pillole? Ma non si poteva dispiacere la mamma che del suo malessere si considerava colpevole. 

- Abbiate la compiacenza di sedervi, principessina - disse il medico famoso. 

Sedette di fronte a lei, con un sorriso le prese il polso e di nuovo cominciò a far domande oziose. Ella gli rispondeva, ma a un tratto, indispettita, si alzò. 

- Scusatemi, dottore, ma tutto questo è davvero inconcludente. Per tre volte mi avete chiesto la stessa cosa. 

Il medico famoso non si offese. 

- Irritazione morbosa - disse alla principessa quando Kitty fu uscita. - Del resto, ho finito\ldots{} 

E il dottore, come a una donna eccezionalmente intelligente, definì alla madre in termini scientifici lo stato della principessina, e concluse col prescrivere quelle acque che non erano necessarie. Alla domanda se si dovesse andare o no all'estero, si sprofondò in meditazioni, come se dovesse decidere una questione difficile. La decisione infine venne fuori: andare e non prestar fede ai ciarlatani, e rivolgersi per tutto a lui. 

Andato via il dottore, si ebbe la sensazione che fosse successo qualcosa di piacevole. La madre si mise di buon umore nel rientrare nella stanza della figlia, e Kitty finse di essere allegra. Le accadeva ormai spesso, anzi quasi sempre, di fingere. 

- Sto bene, maman, davvero. Ma se voi volete andare, andiamo - disse e, cercando di prendere interesse al prossimo viaggio, cominciò a parlare dei preparativi per la partenza. 

\capitolo{II}\label{ii-1} 

Dopo il dottore giunse Dolly. Sapeva che in quel giorno si sarebbe tenuto il consulto e, pur avendo di recente lasciato il letto (aveva dato alla luce una bambina alla fine dell'inverno), pur avendo molte pene e affanni da parte sua, lasciata la neonata e una bambina che si era ammalata, era venuta per sapere della sorte di Kitty che in quel giorno si decideva. 

- E allora? - chiese, entrando nel salotto e togliendosi il cappello. - Siete tutti di buon umore. Probabilmente, va bene? 

Provarono a riferirle quello che aveva detto il dottore, ma si accorsero che, sebbene il dottore avesse parlato diffusamente e a lungo, in nessun modo si riusciva a ripetere quello che aveva detto. Risultava chiaro solo il fatto che era stato deciso di andare all'estero. 

Dolly sospirò involontariamente. La sua amica migliore, la sorella, partiva. E la sua vita non era allegra. I rapporti con Stepan Arkad'ic, dopo la riconciliazione, erano divenuti umilianti. La saldatura fatta da Anna era risultata precaria, e l'accordo familiare si era spezzato di nuovo nello stesso preciso punto. Non v'era nulla di concreto, ma Stepan Arkad'ic non era mai in casa; e quasi mai c'era denaro, e i sospetti delle infedeltà tormentavano continuamente Dolly, ed ella li allontanava, temendo le pene già provate della gelosia. Il primo accesso di gelosia, una volta superato, non poteva più ripetersi, e anche la scoperta di un'altra infedeltà non avrebbe prodotto su di lei lo stesso effetto della prima. Una scoperta di questo genere avrebbe soltanto sconvolto le sue abitudini familiari, e perciò si lasciava ingannare, disprezzando lui e più di tutto se stessa per la propria debolezza. Oltre a questo, le cure di una famiglia numerosa la tormentavano incessantemente: ora l'allattamento della neonata non andava bene, ora la balia si licenziava, ora infine, come in quel momento, s'ammalava uno dei bambini. 

- Be', come stanno i tuoi? - chiese la madre. 

- Ah, maman, di pena da noi ce n'è sempre tanta. Lily s'è ammalata e io temo che sia scarlattina. Sono venuta solo ad informarmi, ma poi mi chiuderò in casa senza più uscire se, Dio ne liberi, dovesse essere scarlattina. 

Il vecchio principe, dopo che il dottore se ne era andato, era uscito anche lui dal suo studio e, dopo aver offerta la guancia a Dolly e aver parlato con lei, si era rivolto alla moglie: 

- Cosa è stato deciso, allora, andate? Be', e di me che ne volete fare? 

- Io ritengo che tu debba restare, Aleksandr - disse la moglie. 

- Maman, e perché papà non può venire con noi? - disse Kitty. - E per lui e per noi sarà più piacevole. 

Il vecchio principe si alzò e carezzò con la mano i capelli di Kitty. Ella aveva sollevato il viso e, sorridendo forzatamente, lo guardava. Le sembrava sempre ch'egli la capisse meglio degli altri in famiglia, benché parlasse poco con lei. Come ultima figlia era la preferita del padre, e a lei sembrava che quel grande affetto lo rendesse perspicace. E quando il suo sguardo incontrò quei suoi buoni occhi azzurri che la guardavano fissi, le sembrò ch'egli la vedesse da parte a parte e che capisse tutto il tormento che avveniva in lei. Arrossendo si protese verso di lui, aspettando un bacio, ma egli le batté soltanto sui capelli e disse: 

- Questi stupidi chignons! Non carezzi i capelli di tua figlia, ma quelli di femmine già morte. Be', Dolin'ka, che fa il tuo bel tomo? 

- Nulla, papà - rispose Dolly, comprendendo che l'allusione si riferiva al marito. - È sempre fuori e non lo vedo quasi - aggiunse con un sorriso ironico. 

- E che, non è ancora partito per la campagna a vendere il legname? 

- No, si prepara sempre. 

- Ecco - disse il principe. - Così allora anch'io devo prepararmi? Ai vostri ordini, signora - disse alla moglie, sedendosi. - E tu, ecco cosa devi fare, Katja - aggiunse, rivolgendosi alla figlia minore: - un bel mattino, quando parrà a te, svegliati e di' a te stessa: ecco io sto perfettamente bene e sono di ottimo umore; andiamo di nuovo con papà a spasso sul ghiaccio di buon'ora. Eh? 

Sembrava molto semplice quello che diceva il padre, ma Kitty a queste parole si confuse e si smarrì, come un delinquente colto in fallo. ``Sì, egli sa tutto, capisce tutto e con queste parole mi dice che, per quanto sia vergognoso quello che mi è accaduto, tuttavia bisogna sopravvivere alla propria vergogna''. Non riuscì a riprendersi per rispondere qualcosa. Stava incominciando quando improvvisamente scoppiò a piangere e scappò via dalla stanza. 

- Ecco, tu con i tuoi scherzi - disse la principessa, investendo il marito. - Sei sempre\ldots{} - e cominciò a rimproverarlo. 

Il principe ascoltò a lungo le recriminazioni della principessa e tacque, ma il viso gli si faceva sempre più scuro. 

- Fa tanta pena, la poveretta, tanta pena, e tu non ti accorgi che le fa male ogni accenno a quello che ne è la causa. Ah, sbagliarsi così sul conto della gente! - disse la principessa e, dal cambiamento di tono, Dolly e il principe capirono che alludeva a Vronskij. - Non capisco come non vi siano delle leggi contro esseri così disgustosi e ignobili. 

- Ah, se aveste dato retta a me! - esclamò cupo il principe, alzandosi dalla poltrona e desiderando andarsene; ma, fermandosi poi sulla porta: - Le leggi! ci sono, matuška, e giacché tu mi stai provocando, ti dirò che la colpa di tutto questo è tua, tua, tua soltanto. Leggi contro questi bellimbusti ci sono sempre state e ci sono. Sissignora; se non ci fosse stato da parte vostra quello che non ci sarebbe dovuto essere, io, anche vecchio, l'avrei sfidato a duello, quel cascamorto. Sì: e adesso curate pure la ragazza, fate venire in casa questi ciarlatani. 

Il principe sembrava avesse da dire ancora molte cose, ma non appena la principessa sentì il tono irato di lui, si calmò e si pentì subito come accadeva sempre nelle questioni serie. 

- Alexandre, Alexandre - mormorava, agitandosi e scoppiando in pianto. 

Non appena cominciò a piangere, anche il principe si calmò e le si avvicinò. 

- Su, basta, basta! Anche per te è penoso, lo so. Che fare? Non è un grosso guaio. Dio è misericordioso\ldots{} grazie\ldots{} - disse, non sapendo già più neppur lui cosa dire e, rispondendo al bacio umido di lei sulla sua mano, uscì dalla stanza. 

Già da quando Kitty in lacrime era uscita dalla stanza, Dolly, con la sua esperienza materna, aveva sentito subito che c'era un'opera femminile da compiere, e si era accinta a compierla. Si levò il cappello e, rimboccate moralmente le maniche, si preparò ad agire. Durante l'aggressione materna contro il padre, aveva cercato di trattenere la madre per quanto lo consentiva il suo rispetto filiale. Durante lo scoppio d'ira del padre aveva taciuto, provando vergogna per la madre e tenerezza per il padre, per quella sua bontà immediatamente sopraggiunta; ma appena il padre fu uscito, si apprestò a fare la cosa più urgente: andare da Kitty a calmarla. 

- Ve lo volevo dire da tempo, maman. Sapete che Levin voleva far domanda di matrimonio a Kitty, quando è stato qui l'ultima volta? L'ha detto a Stiva. 

- E allora? Non capisco\ldots{} 

- Allora, forse, Kitty l'avrà respinto. Non ve ne ha parlato? 

- No, non ha detto niente né di questo né dell'altro: è troppo orgogliosa. Ma io lo so che tutto dipende dal fatto che\ldots{} 

- Certamente. Immaginate\ldots{} se ha detto di no a Levin\ldots{} e non l'avrebbe mai respinto se non ci fosse stato l'altro, lo so\ldots{} E invece poi l'altro l'ha ingannata così orribilmente. 

La principessa si sentiva sgomenta a pensare quanto ella fosse colpevole verso la figlia, e montò in collera. 

- Ah, non capisco più nulla! Oggigiorno vogliono fare di testa loro, non dicono nulla alla mamma, e poi, ecco\ldots{} 

- Maman, io vado da lei. 

- Va', te lo proibisco forse? - disse la madre. 

\capitolo{III}\label{iii-1} 

Entrando nello studiolo di Kitty, una graziosa stanza color rosa, giovanile, rosea e gaia come la stessa Kitty fino a due mesi addietro, disseminata di figurine vieux saxe, Dolly ricordò con quanta gioia e con quanto amore avevano arredata insieme, l'anno prima, quella stanzetta. Le si gelò il cuore quando vide Kitty seduta su di una seggiola bassa, la più vicina alla porta, con gli occhi fissi immobili su di un angolo del tappeto. 

Kitty guardò la sorella e l'espressione fredda, un po' dura del viso non mutò. 

- Adesso me ne vado e dovrò chiudermi in casa, neanche tu potrai venire da me - disse Dar'ja Aleksandrovna, sedendosi accanto a lei. - Volevo parlare un po' con te. 

- Di che? - domandò Kitty in fretta, alzando spaventata la testa. 

- Di che, se non della tua pena? 

- Ma io non ho nessuna pena. 

- Basta, Kitty. Davvero pensi che io possa non capire? Io so tutto! E credimi, questo non è nulla. Ci siamo passate tutte. 

Kitty taceva, e il suo viso aveva un'espressione dura. 

- Non merita che tu soffra per lui - continuò Dar'ja Aleksandrovna andando dritta allo scopo. 

- Già, perché mi ha disdegnata - disse Kitty con voce tremante. - Non me ne parlare, ti prego, non me ne parlare! 

- Ma chi ti ha detto questo? Nessuno ha detto questo. Sono sicura che lui era innamorato di te ed è rimasto innamorato ma\ldots{} 

- Ah, la cosa più tremenda per me sono questi compatimenti! - gridò Kitty, irritandosi a un tratto. Si girò sulla seggiola, arrossì e prese a muovere rapidamente le dita, stringendo ora con una mano, ora con l'altra la fibbia della cintura. Dolly conosceva quel tratto della sorella, di afferrar qualcosa con le mani quando si eccitava: sapeva Kitty capace, in un momento d'ira, di trascendere e di pronunciare molte cose inutili e spiacevoli, e voleva calmarla, ma era già troppo tardi. - Cosa, cosa mi vuoi far sentire? - diceva con furia. - Che io ero innamorata di un uomo che non voleva saperne di me, e che muoio di amore per lui? E questo me lo dice una sorella che crede così di\ldots{} compatirmi! Non ne voglio di questi compatimenti e di queste mistificazioni! 

- Kitty, sei ingiusta. 

- E tu perché mi tormenti? 

- Ma al contrario\ldots{} Vedo che soffri\ldots{} 

Ma Kitty nella sua collera non l'ascoltava. 

- Non ho nulla di cui debba affliggermi o consolarmi. Sono tanto orgogliosa da non permettermi mai di amare un uomo che non mi ama. 

- Sì, ma io non dico\ldots{} Solo\ldots{} dimmi la verità - disse prendendole la mano Dar'ja Aleksandrovna. - Dimmi, Levin ti ha parlato? 

L'accenno a Levin fece perdere del tutto a Kitty il dominio di sé; scattò su dalla seggiola e, gettata via la fibbia, e agitando rapida le mani, si mise a dire: 

- E che c'entra, ora, anche Levin? Non capisco che bisogno abbia tu di tormentarmi. Ti ho detto e ti ripeto che sono orgogliosa e che mai, mai farò quello che fai tu: di ritornare a un uomo che ti ha tradito; che si è innamorato di un'altra. Io questo non lo capisco. Tu puoi, e io non posso! 

Dette queste parole, guardò la sorella e, vedendo che Dolly taceva, abbassando tristemente il capo, invece di uscire dalla stanza come stava per fare, Kitty ristette presso la porta e chinò la testa, nascondendo il viso nel fazzoletto. 

Il silenzio durò circa due minuti. Dolly pensava a sé. L'umiliazione che sempre sentiva, risonava in maniera particolarmente dolorosa in lei, ora che gliela rinfacciava la sorella. Non si aspettava tanta crudeltà da lei e ne provò sdegno. Ma improvvisamente sentì il fruscio di un abito e insieme il suono di un singhiozzo trattenuto che prorompeva e due braccia che dal basso le circondavano il collo. Kitty era davanti a lei in ginocchio. 

- Dolin'ka, sono tanto, tanto infelice! - mormorò in tono colpevole. 

E il viso gentile, coperto di lacrime, si nascose nella gonna di Dar'ja Aleksandrovna. Come se le lacrime fossero state l'olio indispensabile senza il quale non poteva muoversi la macchina delle reciproche confidenze fra sorelle, dopo le lacrime esse non parlarono più di quello che loro stava a cuore, ma anche conversando di altro, si intesero scambievolmente. Kitty capì che le parole pronunziate nella furia sull'infedeltà del cognato e sulla posizione umiliante della sorella avevano sì, ferito la poveretta in fondo al cuore, ma ch'ella aveva perdonato. Dolly da parte sua seppe quello che voleva sapere: si convinse cioè che le sue supposizioni erano fondate, che il dolore, l'inguaribile dolore di Kitty, consisteva proprio in questo: che Levin aveva fatto la sua proposta di matrimonio, e Kitty gli aveva detto di no, mentre Vronskij l'aveva ingannata; e ch'ella avrebbe amato Levin e odiato Vronskij. Ma Kitty non disse neppure una parola di questo. Parlava solo delle sue condizioni di spirito. 

- Non ho nessun male - diceva, dopo essersi calmata; - ma non puoi credere come per me tutto sia diventato brutto, ripugnante, volgare e prima di tutto me stessa. Tu non puoi immaginare quali brutti pensieri io abbia su tutto. 

- Ma quali brutti pensieri puoi mai avere tu? - chiese Dolly, sorridendo. 

- I più disgustosi e volgari, non te li posso dire. Non è malinconia, né stanchezza, ma qualcosa di molto peggiore. È come se tutto quello che c'era di buono in me si fosse nascosto e fosse rimasta solo la parte più ignobile. Ma come dirti? - continuò vedendo la perplessità negli occhi della sorella. - Papà comincia a parlare\ldots{} e a me sembra ch'egli pensi soltanto che io debba prender marito. Mamma mi accompagna a un ballo: e a me pare che mi ci conduca soltanto per darmi un marito al più presto e liberarsi di me. Lo so che questo non è vero, ma non posso scacciar via questi pensieri. I cosiddetti pretendenti non li posso più vedere. Mi sembra che mi prendan le misure. Prima per me andare in qualche posto, in abito da ballo, era un vero godimento, mi compiacevo di me stessa; ora mi vergogno, sono impacciata. Ma che vuoi! Il dottore\ldots{} e poi\ldots{} 

Kitty s'ingarbugliò; voleva dire ancora che, da quando era avvenuto in lei questo cambiamento, Stepan Arkad'ic le era divenuto insopportabilmente odioso, e che non poteva guardarlo senza associargli le immagini più volgari e sconvenienti. 

- Già, tutto mi appare nell'aspetto più volgare e più disgustoso - continuò. - Questa è la malattia, forse passerà. 

- Ma cerca di non pensare. 

- Non posso. Soltanto coi bambini sto bene. Soltanto da te. 

- Peccato che non ci potrai più venire. 

- Sì che verrò. Ho avuto già la scarlattina, e convincerò maman. 

Kitty insistette nel suo proposito e andò a stare dalla sorella, e per tutto il tempo della scarlattina, che realmente si manifestò, curò i bambini. Tutte e due le sorelle portarono felicemente a guarigione i sei piccoli, ma la salute di Kitty non migliorò, e durante la quaresima gli Šcerbackij partirono per l'estero. 

\capitolo{IV}\label{iv-1} 

Una sola è la cerchia mondana di Pietroburgo; tutti si conoscono e si scambiano visite. Ma in questa vasta cerchia vi sono delle suddivisioni. Anna Arkad'evna Karenina aveva amici e relazioni in tre circoli diversi. Il primo era quello burocratico, cioè il circolo ufficiale del marito composto di colleghi e di dipendenti, legati e divisi tra di loro dalle varie condizioni sociali nel modo più strano e capriccioso. Anna, ora, stentava assai a ricordarsi di quel senso di considerazione quasi devota che nei primi tempi aveva provato per questi personaggi. Ora li conosceva tutti come ci si conosce in un capoluogo di provincia: conosceva le abitudini, le debolezze e le insofferenze di ognuno, conosceva i rapporti fra di loro e i rapporti di ciascuno col centro, sapeva a chi precisamente ciascuno fosse legato e per mezzo di che cosa si congiungesse e si distaccasse dagli altri; ma a questo circolo di interessi burocratici maschili non era riuscita mai a interessarsi e, malgrado i suggerimenti della contessa Lidija Ivanovna, ne rifuggiva. 

Un altro circolo molto vicino ad Anna era quello attraverso il quale Aleksej Aleksandrovic aveva fatto carriera. Centro ne era la contessa Lidija Ivanovna. Era un circolo di donne vecchie e brutte, virtuose e bigotte, di uomini intelligenti, colti e ambiziosi. Una persona intelligente che ne faceva parte lo aveva definito: ``la coscienza della società di Pietroburgo''. Aleksej Aleksandrovic amava molto questo circolo, e Anna che sapeva trattare tutti, nei primi tempi della sua vita a Pietroburgo, si era fatta degli amici anche qui. 

Il terzo circolo, infine, che Anna frequentava, era proprio il cosiddetto gran mondo, il gran mondo dei balli, dei pranzi, delle toilettes, il mondo che si appoggiava alla corte per non scendere al livello di quel mondo equivoco che i membri di questo circolo credevano di poter disprezzare, pur avendo con esso gusti, più che simili, identici. Anna era legata a questo circolo per mezzo della principessa Betsy Tverskaja, moglie di un suo cugino, che aveva centoventimila rubli di rendita e che, dal suo primo apparire nel gran mondo, aveva preso a volerle bene, a circuirla e attrarla nel suo ambiente deridendo quello della contessa Lidija Ivanovna. 

- Quando sarò vecchia e brutta diventerò anch'io come loro - diceva Betsy. - Ma per voi, per una donna giovane e bella come voi, è prematuro un simile ospizio di vecchi. 

Anna, nei primi tempi, evitava, per quanto poteva, questo circolo della principessa Tverskaja, e perché la vita che vi svolgeva esigeva delle spese superiori alle sue possibilità e perché poi, in fondo all'animo, preferiva l'altro; ma dopo il viaggio a Mosca era avvenuto il contrario. Sfuggiva i suoi amici morali e frequentava il gran mondo. Là incontrava Vronskij, e provava una gioia conturbante in questi incontri. Incontrava Vronskij soprattutto da Betsy che era nata Vronskaja e gli era cugina. Vronskij si trovava ovunque potesse incontrare Anna, e le parlava, appena poteva, del suo amore. Ella non gliene dava pretesto, ma ogni volta che si incontrava con lui, le si accendeva nell'animo quella stessa esaltazione che l'aveva presa quel giorno in treno, quando l'aveva visto per la prima volta. Sentiva che, nel vederlo, la gioia le luceva negli occhi e le labbra le si increspavano nel riso e non riusciva ad attutire le manifestazioni di questa gioia. 

Nei primi tempi, Anna credeva in buona fede d'essere contrariata da lui che si permetteva di perseguitarla; ma poco dopo il ritorno da Mosca, una sera, in un ricevimento in cui pensava d'incontrarlo ed egli non c'era, dalla tristezza che s'impossessò di lei, capì che ingannava se stessa e che questa persecuzione non solo non le era spiacevole, ma costituiva tutto l'interesse della sua vita. 

La cantante famosa cantava per la seconda volta e tutto il gran mondo era a teatro. Vista dalla sua poltrona la cugina in prima fila, Vronskij, senza aspettare l'intervallo, entrò nel palco. 

- Com'è che non siete venuto a pranzo? - ella chiese. - Resto meravigliata di fronte a questa chiaroveggenza da innamorati - aggiunse con un sorriso e in modo ch'egli solo potesse sentire: - lei non c'era. Ma venite dopo l'opera. 

Vronskij la guardò interrogativamente. Ella chinò il capo, ed egli la ringraziò con un sorriso, sedendo vicino a lei. 

- E come ricordo le vostre beffe! - continuò la principessa Betsy che trovava un particolare piacere nel seguire l'accendersi di questa passione. - Dov'è andato a finire tutto quello che dicevate? Siete preso al laccio, mio caro! 

- È quel che desidero, d'esser preso - disse Vronskij col suo tranquillo sorriso cordiale. - Se mi lamento, è perché son troppo poco ``preso'', a dir il vero. Comincio a perdere la speranza. 

- Che speranza potete mai avere? - disse Betsy offesa per l'amica - entendons nous. - Ma nei suoi occhi saltellava un focherello che diceva come ella capisse molto bene, e proprio alla stessa guisa di lui, quale fosse la sua speranza. 

- Nessuna - disse Vronskij, ridendo e mettendo in mostra la sua bella dentatura. - Scusate - disse, prendendo il binocolo dalle mani di lei e osservando, al di là della sua spalla nuda, l'ordine opposto dei palchi. - Temo di diventar ridicolo. 

Egli sapeva molto bene che, agli occhi di Betsy e di tutte le persone di mondo, non rischiava di diventar ridicolo. Sapeva molto bene che agli occhi di queste persone la parte dell'amante infelice di una ragazza e in generale di una donna libera poteva parer ridicola; ma la parte del corteggiatore di una donna maritata, che, qualunque cosa accada, pone la propria vita in giuoco per trascinarla all'adulterio, questa parte aveva qualcosa di bello e di grande e non poteva mai apparire ridicola; e perciò con un sorriso d'orgoglio e di felicità che gli errava sotto i baffi, abbassò il binocolo e guardò la cugina. 

- E perché non siete venuto a pranzo? - disse lei, compiaciuta. 

- Questo proprio ve lo devo raccontare. Sono stato occupato, in che cosa? Ve lo do a indovinare su cento\ldots{} su mille. Non l'indovinerete mai. Ho fatto rappacificare un marito con l'offensore della propria moglie. Sì, davvero! 

- Be', e han fatto pace? 

- Quasi. 

- Me lo dovete raccontare - disse lei, alzandosi. - Venite nell'altro intervallo. 

- Non posso, vado al Teatro Francese. 

- E non ascoltate la Nilsson? - chiese con orrore Betsy che non avrebbe saputo in nessun modo distinguere la Nilsson da una qualsiasi corista. 

- Che fare? Ho un appuntamento là, sempre per questa mia opera di pace. 

- Beati i pacificatori, essi si salveranno - disse Betsy, ricordando qualcosa di simile, sentito dire da qualcuno. - Su, allora sedetevi e raccontate, cos'è? 

E riprese il proprio posto. 

\capitolo{V}\label{v-1} 

- È un po' scabrosa, ma è così carina che ho una voglia matta di raccontarla - disse Vronskij, guardandola con gli occhi ridenti. - Non farò nomi. 

- Tanto meglio, indovinerò. 

- Allora ascoltate: due giovani allegri vanno in carrozza\ldots{} 

- S'intende, ufficiali del vostro reggimento. 

- Non ho detto ufficiali, semplicemente due giovani che hanno fatto colazione\ldots{} 

- Traducete: che hanno bevuto. 

- Forse. Vanno in carrozza a pranzo da un amico, nella più allegra disposizione di spirito. E vedono una bella signora che li sorpassa in vettura, si volta e, così almeno a loro sembra, fa cenno e ride. Quelli, naturalmente, subito dietro a lei. Galoppano a tutta forza. Con sorpresa la bella si ferma all'ingresso di quella stessa casa dove vanno loro. La bella corre al piano di sopra. Essi scorgono solo le labbruzze vermiglie di sotto al velo corto e i deliziosi piccoli piedi. 

- Raccontate con tale sentimento che par proprio che siate voi uno dei due. 

- Be', a che cosa avete accennato or ora?\ldots{} Dunque i giovani entrano in casa del compagno; c'è un pranzo di addio. Qui forse appunto bevono un po' più del necessario, come sempre avviene nei pranzi di addio. E a tavola chiedono chi abita su in quella casa. Nessuno lo sa, ma quando chiedono se al piano di sopra ci sono delle mamzel', il servo del padrone risponde che lì ce n'è tante. Dopo pranzo i giovani vanno nello studio del padrone di casa e scrivono una lettera alla sconosciuta, una dichiarazione, e portano loro stessi la lettera di sopra per spiegare quello che nella lettera non sarebbe apparso del tutto comprensibile. 

- Ma perché mi raccontate tutte queste sciocchezze? E poi? 

- Bussano. Vien fuori una cameriera. Consegnano la lettera e assicurano la cameriera che sono tutti e due così innamorati che stanno lì lì per morire sulla porta. La cameriera, perplessa, conduce delle trattative. Ed ecco, a un tratto compare il padrone di casa con le fedine a salsicciotto, rosso come un gambero, il quale spiega che in casa non c'è nessuno all'infuori di sua moglie, e li caccia via tutti e due. 

- E come fate a sapere che ha le fedine, così come avete detto, a salsicciotto? 

- Ecco, ascoltate. Non sono forse andato oggi a far da paciere? 

- E allora? 

- E qui viene il bello. Viene in chiaro che si tratta di una coppia felice: un consigliere titolare e una consiglieressa titolare. Il consigliere titolare sporge querela e io faccio da paciere; e quale paciere!\ldots{} Vi assicuro, Talleyrand non è nulla a petto mio. 

- Ma in che consiste la vostra abilità? 

- Ecco, ascoltate. Noi ci siamo scusati a questo modo: ``siamo desolati, chiediamo venga perdonato il disgraziato equivoco''. Il consigliere titolare dai salsicciotti comincia a rabbonirsi, ma vuole anche lui esprimere i suoi sentimenti, e, non appena comincia a esprimerli, ecco che prende fuoco, si riscalda e dice villanie, e io devo di nuovo mettere in moto tutto il mio talento diplomatico. ``Sono d'accordo che l'azione non è punto lodevole, ma vi prego prendere in considerazione l'equivoco, l'età giovanile e il fatto che i ragazzi avevano allora allora finito di mangiare. Voi comprenderete! Essi sono pentiti con tutta l'anima, chiedono il vostro perdono''. Il consigliere titolare si rabbonisce di nuovo: ``D'accordo, conte, sono pronto a perdonare, ma capirete che mia moglie, mia moglie, una donna onesta, è stata sottoposta a un inseguimento, alle villanie ed alle impertinenze di due ragazzacci qualsiasi, masc\ldots{}''. E pensate che intanto uno di quei ragazzacci sta lì, e io devo far fare la pace. Metto di nuovo in moto tutta la mia diplomazia, ma appena l'affare si avvia alla conclusione, il mio consigliere titolare si scalda ancora, si fa rosso, solleva i salsicciotti e allora, di nuovo, io mi effondo in sottigliezze diplomatiche. 

- Ah, questa bisogna raccontarvela! - disse Betsy alla signora che entrava nel palco. - Mi ha fatto tanto ridere. 

- Su, bonne chance! - aggiunse, dando a Vronskij un dito libero della mano che teneva il ventaglio e abbassando, con un movimento delle spalle, il corpetto del vestito che si era sollevato per apparire interamente scollata quando si sarebbe accostata, secondo l'uso, al parapetto del palco, alla luce del gas e agli sguardi di tutti. 

Vronskij andò al Teatro Francese, dove realmente doveva vedere il comandante del reggimento, che non perdeva neanche una rappresentazione, per parlargli della sua opera di pace che lo occupava e lo divertiva da due giorni. In questo affare era implicato Petrickij, cui egli voleva bene, e un altro, entrato da poco nel reggimento, un buon ragazzo, un ottimo compagno, il giovane principe Kedrov. Ma era l'onore del reggimento principalmente in giuoco. Tutti e due erano dello squadrone di Vronskij. Al comandante del reggimento si era presentato un impiegato, il consigliere titolare Venden, con una querela contro gli ufficiali che gli avevano offeso la moglie. La sua giovane moglie (come raccontava Venden che era ammogliato da sei mesi appena) stava in chiesa con la mamma, quando, avvertito a un tratto un certo malessere dovuto a un suo particolare stato, e non potendo più stare in piedi, era andata a casa con la prima vettura che le era capitata. A questo punto le si erano messi dietro gli ufficiali, lei s'era spaventata e, sentendosi sempre peggio, era corsa su per le scale a casa. Lo stesso Venden, tornato dal tribunale, aveva sentito la scampanellata e il vocio e, visti gli ufficiali ubriachi con la lettera in mano, era uscito e li aveva scaraventati fuori. 

- Ma, dite quel che volete - diceva il comandante del reggimento a Vronskij dopo averlo fatto accostare a sé - Petrickij diventa impossibile. Non passa una settimana senza una storia. Questo funzionario non farà passar liscia la cosa, la manderà avanti. 

Vronskij vedeva quanto fosse incresciosa la faccenda, come si dovesse evitare un duello e far di tutto per rabbonire il consigliere e mettere a tacere la cosa. Il comandante del reggimento si era rivolto a Vronskij proprio perché egli apparteneva all'aristocrazia ed era persona intelligente e soprattutto gelosa dell'onore del reggimento. Discussero un po' e decisero di fare andare Petrickij e Kedrov con Vronskij da questo consigliere titolare a chiedere scusa. Il comandante del reggimento e Vronskij capivano entrambi che il nome di Vronskij e la sua qualifica di aiutante di campo dovevano contribuire non poco a rabbonire il consigliere titolare. E in realtà queste due prerogative risultarono in parte efficienti, ma la conclusione era rimasta dubbia, come del resto stava raccontando lo stesso Vronskij. 

Giunto al Teatro Francese, Vronskij si era appartato insieme con il comandante del reggimento nel ridotto e gli andava raccontando il suo successo o insuccesso. Dopo aver riflettuto, il comandante del reggimento decise di lasciar cadere la faccenda; ma poi, per divertirsi, cominciò a interrogare Vronskij sui particolari dell'incontro, e a lungo non poté trattenersi dal ridere ascoltando quel che Vronskij diceva del consigliere titolare che, quando stava per calmarsi, si accendeva di nuovo al ricordo dei particolari dell'offesa, e sul fatto che Vronskij, alla prima mezza parola conciliante, aveva battuto in ritirata, spingendo avanti a sé Petrickij. 

- È un brutto affare, ma esilarante. Kedrov non può certo battersi con quel signore. Ma si scaldava proprio così furiosamente? - tornava a chiedere, ridendo, il comandante. - E come vi pare questa sera Claire? Una meraviglia! - disse, alludendo alla nuova attrice francese. - Per quanto la si veda, ogni volta è nuova. Solo i francesi sanno essere così. 

\capitolo{VI}\label{vi-1} 

La principessa Betsy, senza aspettare la fine dell'ultimo atto, uscì dal teatro. Aveva fatto appena in tempo ad entrare nello spogliatoio, cospargere il lungo viso pallido di cipria e spalmarvela, ricomporsi e ordinare il tè nel salotto grande, che già una dietro l'altra cominciarono ad arrivare le carrozze alla sua enorme casa nella Bol'šaja Morskaja. Gli invitati raggiungevano la grande scala e il portiere imponente, che la mattina leggeva i giornali dietro la porta di vetro a edificazione dei passanti, apriva in silenzio la grande porta e faceva passare quelli che arrivavano. 

Entrarono quindi, nello stesso tempo, la padrona di casa, da una porta, con la pettinatura racconciata e il viso rinfrescato, e gli ospiti dall'altra nel salotto grande, dalle pareti scure e i tappeti lanosi, con la tavola illuminata a giorno su cui risplendevano, alla luce delle candele, il bianco della tovaglia, l'argento del samovar e la porcellana trasparente del servizio da tè. 

La padrona di casa sedette al samovar e si tolse i guanti. Spostando le sedie con l'aiuto dei camerieri che non si facevano notare, la compagnia si distribuì in due gruppi, uno accanto al samovar intorno alla padrona di casa, l'altro all'estremo opposto del salotto, intorno alla bella moglie di un ambasciatore, dalle sopracciglia scure marcate, in abito di velluto nero. La conversazione nei due gruppi, come del resto avviene sempre sulle prime in un ricevimento, oscillava interrotta dagli incontri, dai saluti, dal tè, come se cercasse un argomento su cui fissarsi. 

- È straordinaria come attrice: evidentemente si è studiata Kaulbach - diceva un diplomatico nel gruppo dell'ambasciatrice - avete notato con che arte è caduta\ldots{} 

- Ah, vi prego, non parliamo più della Nilsson! Di lei ormai non si può dire nulla di nuovo - disse una signora grassa, rossa, senza sopracciglia e senza chignons, coi capelli bianchi e un vecchio vestito di seta. Era la principessa Mjagkaja, nota per la sua semplicità e ruvidezza di tratto, e soprannominata l'enfant terrible. La Mjagkaja sedeva tra i due gruppi e, tendendo l'orecchio, prendeva parte ora a questo ora a quello. - Oggi tre persone mi hanno detto questa stessa frase su Kaulbach, proprio come se si fossero messe d'accordo. E non so capire perché la frase fosse loro piaciuta tanto. 

La conversazione fu interrotta da questa osservazione, e bisognò trovare un altro tema. 

- Raccontateci qualcosa di divertente, ma non di maligno - disse la moglie dell'ambasciatore, grande maestra di quella conversazione elegante che gli inglesi chiamano small-talk, rivolta al diplomatico che in quel momento non sapeva neanche lui che cosa dire. 

- Sembra che non sia facile, perché solo quello che è maligno fa ridere - cominciò lui con un sorriso. - Ma mi ci proverò. Datemi un tema. Tutto sta nel tema. Quando è dato il tema è più facile ricamarci su. Spesso penso che i famosi parlatori del secolo scorso si troverebbero oggigiorno in difficoltà a conversare con intelligenza. Tutto quello che è intelligente è così noioso\ldots{} 

- Già detto da tempo - lo interruppe, ridendo, la moglie dell'ambasciatore. 

La conversazione, incominciata piacevolmente, proprio perché già troppo cordiale, si arrestò di nuovo. Era il caso di ricorrere al mezzo sicuro che non viene mai meno: la maldicenza. 

- Non trovate che in Tuškevic c'è qualcosa alla Louis XV? - disse il diplomatico indicando con gli occhi un bel giovane biondo che era in piedi accanto alla tavola. 

- Oh, sì! È nello stesso stile del salotto; proprio per questo ci viene così spesso. 

Questo tema di conversazione attecchì, proprio perché alludeva a quello di cui non si sarebbe dovuto parlare in quel salotto, dei rapporti, cioè, di Tuškevic con la padrona di casa. 

Intanto, anche intorno al samovar e alla padrona di casa, la conversazione, dopo aver oscillato allo stesso modo per un po' fra i tre temi inevitabili: l'ultima novità mondana, il teatro e la maldicenza, si era fatta stabile, appena toccato l'ultimo tema, quello della maldicenza. 

- Avete sentito, anche la Maltišceva, non la figlia, ma la madre, si fa un vestito diable rose. 

- È impossibile! No, questa è bella! 

- Mi meraviglio come con la sua intelligenza, non è mica sciocca, non s'accorga di quanto sia ridicola. 

Ognuno aveva qualcosa da dire per criticare e prendere in giro la povera Maltišceva, e la conversazione scoppiettò allegra come un fastello di legna che prenda fuoco. 

Il marito della principessa Betsy, un panciuto bonaccione, appassionato raccoglitore di stampe, saputo che la moglie aveva ospiti, era entrato in salotto prima di andare al circolo. Silenziosamente, sul tappeto soffice, si era accostato alla Mjagkaja. 

- V'è piaciuta la Nilsson? - disse. 

- Ah!\ldots{} ma è forse permesso avvicinarsi così? Come mi avete spaventata! - disse lei. - Con me, vi prego, non parlate dell'opera; voi non capite nulla di musica. Piuttosto discenderò io fino a voi a parlar delle vostre maioliche e delle vostre stampe. Dunque, qual'è l'ultimo tesoro che avete comprato dal rigattiere? 

- Volete che ve lo mostri? Ma voi non capite nulla. 

- Mostratemelo. Ho imparato da quei tali, come si chiamano\ldots{} da quei banchieri\ldots{} hanno delle stampe bellissime. Ce le han fatte vedere. 

- Come, siete stata dagli Schützburg? - domandò la padrona di casa di là dal samovar. 

- Ci siamo stati, ma chère. Ci hanno invitato, me e mio marito, a pranzo, e m'han detto che la salsa a quel pranzo era costata mille rubli - diceva a gran voce la Mjagkaja, sentendo che tutti l'ascoltavano - e per giunta una salsa pessima, una certa broda verdastra. Poi ho dovuto invitarli a casa mia, e io ho fatto preparare una salsa da ottantacinque copeche, e tutti sono rimasti molto soddisfatti. Io non posso far mica sempre salse da mille rubli! 

- È unica! - disse la padrona di casa. 

- Sorprendente! - disse qualcuno. 

L'effetto prodotto dai discorsi della principessa Mjagkaja era sempre lo stesso, e il segreto di questo effetto consisteva nel dire, anche se non del tutto a proposito, come adesso, delle cose semplici che avevano un certo senso. Nella società in cui viveva queste parole producevano l'effetto dello scherzo più spiritoso. La Mjagkaja non riusciva a capire perché ciò accadesse, ma sapeva che così era, e ne approfittava. 

Dal momento che durante il discorso della Mjagkaja tutti avevano ascoltato lei e la conversazione intorno alla moglie dell'ambasciatore era cessata, la padrona di casa volle riunire i due gruppi e si rivolse all'ambasciatrice. 

- Ma proprio non volete del tè? Dovreste passare dalla parte nostra. 

- No, stiamo tanto bene qui - rispose con un sorriso la moglie dell'ambasciatore, e riprese la conversazione di poco prima. 

La conversazione era molto piacevole. Si criticavano i Karenin, marito e moglie. 

- Anna è molto cambiata dopo il viaggio a Mosca. C'è in lei qualcosa di strano - diceva una sua amica. 

- Il cambiamento di maggior rilievo è che ha portato con sé l'ombra di Aleksej Vronskij - disse l'ambasciatrice. 

- E che c'è di strano? C'è una favola di Grimm: l'uomo senza ombra, l'uomo privato dell'ombra. E questo gli è dato in castigo di qualcosa. Non ho mai capito in che cosa consistesse il castigo. Ma per una donna, sì che deve essere triste non aver l'ombra. 

- Sì, ma le donne con l'ombra, di solito, vanno a finir male - disse l'amica di Anna. 

- Che vi si secchi la lingua! - disse di botto la principessa Mjagkaja a queste parole. - La Karenina è un'ottima donna. Il marito non mi piace, ma a lei voglio un gran bene. 

- Perché non vi piace il marito? È un uomo così notevole - disse l'ambasciatrice. - Mio marito dice che uomini di stato come lui ce ne sono pochi in Europa. 

- Anche mio marito dice questo, ma io non ci credo - disse la Mjagkaja. - Se i nostri mariti non avessero detto ciò, noi vedremmo quello che è; e Aleksej Aleksandrovic, secondo me, è semplicemente scemo. Io lo dico sottovoce\ldots{} Ma non è vero che così tutto diventa chiaro? Prima, quando m'imponevano di ritenerlo intelligente, non facevo che cercare, e trovavo che ero io la sciocca che non vedeva la sua intelligenza; non appena mi son detta: ``è scemo'', ma sottovoce, tutto è diventato così chiaro; non è vero, forse? 

- Come siete cattiva, oggi! 

- Per nulla affatto. Non c'è altra soluzione. Uno dei due è scemo. Certo, voi lo sapete, di se stessi non si arriva mai a dirlo. 

- Nessuno è contento del proprio stato e ciascuno è contento della propria intelligenza - disse il diplomatico con un verso francese. 

- Ecco, ecco, proprio così - si voltò a lui la Mjagkaja. - Ma il fatto è che io Anna non ve la do in pasto. È così simpatica, gentile. Che fare se tutti si innamorano di lei e le corrono dietro come ombre? 

- Ma io non penso affatto di criticarla - si andava giustificando l'amica di Anna. 

- Se a noi non c'è nessuno che ci vien dietro come l'ombra, questa non è una ragione per aver il diritto di condannare. 

E dopo aver conciato per le feste, così come si conveniva, l'amica di Anna, la principessa Mjagkaja s'alzò e, insieme con la moglie dell'ambasciatore, si unì a quelli della tavola dove era avviata una conversazione di ordine generale sul re di Prussia. 

- Di chi stavate parlando male? - chiese Betsy. 

- Dei Karenin. La principessa ci ha dipinto le caratteristiche di Aleksej Aleksandrovic - rispose con un sorriso l'ambasciatrice, sedendosi a tavola. 

- Peccato che non abbiamo sentito - disse la padrona di casa, guardando la porta d'ingresso. - Oh, eccovi, ci siete anche voi, finalmente! - disse rivolta con un sorriso a Vronskij che entrava. 

Vronskij non solo conosceva tutti, ma s'incontrava ogni giorno con tutti quelli ch'erano lì; entrò quindi con quel suo fare calmo, così come si entra nella stanza di persone che si sono allora allora lasciate. 

- Di dove vengo? - rispose ad una domanda dell'ambasciatrice. - Non c'è scampo, bisogna confessarlo: dai Bouffes. Per la centesima volta e sempre con piacere nuovo, a quanto pare. Un incanto! Lo so che è vergognoso, ma all'opera dormo, mentre ai Bouffes rimango a sedere fino all'ultimo momento e mi diverto. Oggi\ldots{} 

Nominò un'attrice francese e voleva raccontare qualcosa su di lei, ma l'ambasciatrice l'interruppe con scherzoso raccapriccio. 

- Vi prego, non parlate di quell'orrore. 

- E sia, ve ne dispenserò, tanto più che tutti conoscono questi orrori. 

- E tutti ci andrebbero, se questo fosse di moda come andare all'opera - aggiunse la Mjagkaja. 

\capitolo{VII}\label{vii-1} 

Si udirono dei passi alla porta e la principessa Betsy, sapendo che era la Karenina, guardò Vronskij. Egli guardava l'uscio e il suo viso aveva un'espressione strana, nuova. Guardava fisso, con gioia e insieme con timidezza, colei che entrava, e nello stesso tempo si alzava lentamente. Anna entrava nel salotto. Straordinariamente diritta come sempre, con quel suo passo agile, sicuro e leggero che la distingueva dall'andatura delle altre donne del suo mondo, fece i pochi passi che la separavano dalla padrona di casa e, senza cambiare direzione allo sguardo, le porse la mano, sorrise e con quello stesso sorriso si voltò a guardare Vronskij. Vronskij s'inchinò profondamente e le accostò una sedia. 

Ella rispose con un semplice chinar del capo, arrossì e aggrottò le sopracciglia. Ma poi, facendo subito un cenno della testa agli amici e stringendo le mani tese, si rivolse alla padrona di casa: 

- Sono stata dalla contessa Lidija ed avrei voluto venir via prima. Ma c'era da lei sir John. È molto interessante. 

- Ah, quel missionario? 

- Sì, ha raccontato delle cose molto interessanti sulla vita degli indiani. 

La conversazione, interrotta dall'arrivo, si animò come la fiamma di una lampada avvivata. 

- Sir John, già, sir John. L'ho visto. Parla bene. La Vlas'eva è innamorata pazza di lui. 

- È vero che la Vlas'eva più piccola sposa Topov? 

- Già, dicono che sia tutto deciso. 

- Mi meraviglio dei genitori. Dicono che sia un matrimonio d'amore. 

- D'amore? Che idee antidiluviane che avete! Chi mai al giorno d'oggi parla ancora d'amore? - disse l'ambasciatrice. 

- Che fare? Questa stupida vecchia moda non è ancora passata - disse Vronskij. 

- Tanto peggio per quelli che vi si attengono. Io di matrimoni felici non conosco che quelli d'interesse. 

- Già, ma in cambio, quante volte la felicità di questi matrimoni d'interesse si polverizza proprio perché insorge quella tale passione che non si è voluta ammettere! - disse Vronskij. 

- Ma noi per matrimoni d'interesse intendiamo quelli in cui tutt'e due le parti si siano già ammansite. L'amore è come la scarlattina, bisogna passarci. 

- Allora bisogna imparare a inocularlo artificialmente, l'amore, come il vaiolo. 

- Io in gioventù mi sono innamorata di un sacrestano - disse la principessa Mjagkaja; - non so se questo mi abbia aiutato. 

- No, io penso, a parte gli scherzi, che per conoscere l'amore sia necessario sbagliare e poi correggersi - disse la principessa Betsy. 

- Anche dopo il matrimonio? - disse scherzosa l'ambasciatrice. 

- Non è mai troppo tardi per pentirsi - disse il diplomatico con un proverbio inglese. 

- Davvero - replicò a volo Betsy: - bisogna sbagliarsi e correggersi. Cosa ne pensate? - chiese rivolta ad Anna che ascoltava in silenzio questo discorso con un sorriso fisso, appena percettibile sulle labbra. 

- Io penso - disse Anna, giocando con un guanto che si era tolto - io penso\ldots{} se è vero che ci sono tante sentenze quante teste, così pure tante specie d'amore quanti cuori. 

Vronskij guardava Anna e, col cuore che gli veniva meno, aspettava quello che avrebbe detto. Respirò come dopo un pericolo, quando ella ebbe pronunciato queste parole. 

Anna a un tratto si voltò verso di lui. 

- Ho ricevuto una lettera da Mosca. Mi dicono che Kitty Šcerbackaja stia molto male. 

- Davvero? - disse Vronskij, aggrottando le sopracciglia. Anna lo guardò severa. 

- Non vi interessa questo? 

- Al contrario, molto. Cosa vi scrivono precisamente, se è lecito sapere? - chiese. 

Anna si alzò e si accostò a Betsy. 

- Datemi una tazza di tè - disse, fermandosi dietro la sedia di lei. 

Mentre Betsy le versava il tè, Vronskij si avvicinò ad Anna. 

- Cosa vi scrivono dunque? - ripeté. 

- Io penso molto spesso che gli uomini non capiscono quello che è ignobile, anche parlandone continuamente - disse Anna senza rispondergli. - Ve lo volevo dire da tempo - aggiunse, e, fatti alcuni passi, sedette a una tavola in angolo, sulla quale erano degli album. 

- Non capisco per nulla il senso delle vostre parole - disse lui, dandole la tazza. 

Ella accennò il divano vicino a sé ed egli sedette subito. 

- Sì, ve lo volevo dire - disse lei senza guardarlo. - Avete agito male, male, molto male. 

- Forse non lo so di aver agito male? Ma chi mi ha fatto agire male? 

- Perché mi dite questo? - disse lei, guardandolo severa. 

- Voi lo sapete perché - rispose lui franco e felice, incontrando lo sguardo di lei e senza staccarne gli occhi. 

Non lui, ma lei si turbò. 

- Questo dimostra soltanto che siete senza cuore - disse lei. Ma il suo sguardo diceva che sapeva bene come egli avesse un cuore e che per questo lo temeva. 

- Quello di cui parlavate poc'anzi è stato un abbaglio, non un amore. 

- Ricordatevi che vi ho proibito di pronunciare questa parola, questa parola disgustosa - disse Anna, rabbrividendo; ma in quello stesso attimo sentì che con la sola parola ``proibito'' dava prova di attribuirsi dei diritti su di lui, e che con questo lo eccitava a parlare d'amore. - Da tempo volevo dirvi questo - continuò guardandolo decisa negli occhi e tutta accesa dal rossore che le scottava il viso; - ma oggi sono venuta apposta, sapendo di incontrarvi. Sono venuta per dirvi che questo deve finire. Io non ho mai arrossito davanti a nessuno, e voi mi costringete a sentirmi colpevole di qualche cosa. 

Egli la guardava ed era colpito dalla nuova bellezza, tutta spirituale, del volto di lei. 

- Che volete da me? - disse semplice e serio. 

- Voglio che andiate a Mosca e chiediate perdono a Kitty - disse lei. 

- Voi questo non lo volete - disse lui. 

Egli sentiva che Anna diceva quello che s'era imposta di dire, non quello che avrebbe voluto dire. 

- Se mi amate come dite - ella mormorò - fate che io abbia pace. 

Il viso di lui s'illuminò. 

- Non sapete forse che siete per me tutta la vita? Questa pace io non conosco e non posso darvi. Tutto me stesso, l'amore\ldots{} sì. Non riesco a pensare a voi e a me separatamente. Per me, voi ed io siamo una cosa sola. E io non vedo davanti a me possibilità di pace, né per me, né per voi. Vedo una possibilità di disperazione, di infelicità\ldots{} o la possibilità di una gioia, quale gioia!\ldots{} È forse impossibile? - aggiunse a fior di labbra, ma lei sentì. 

Ella tese tutte le forze del suo spirito per dire quello che si sarebbe dovuto dire; ma, in luogo di questo, fermò il suo sguardo pieno d'amore su di lui, e tacque. 

``Ecco - pensò lui con esaltazione. - Mentre già mi disperavo e credevo dovesse venir la fine, ecco: mi ama. Lo confessa''. 

- Allora fate questo per me, non mi parlate mai più di queste cose e rimaniamo buoni amici - disse lei con le labbra, ma il suo sguardo diceva tutt'altra cosa. 

- Amici non saremo mai, questo lo sapete. Saremo gli esseri più felici o gli esseri più infelici della terra, questo dipende da voi. 

Ella voleva dire qualcosa, ma lui l'interruppe. 

- Perché io chiedo una cosa sola, chiedo il diritto di sperare, di tormentarmi come adesso; ma se anche questo non si può, ditemi allora di scomparire, e io scomparirò. Se la mia presenza vi è di peso, non mi vedrete più. 

- Io non voglio scacciarvi. 

- E allora non cambiate nulla. Lasciate tutto così com'è - disse lui con voce tremante. - Ecco vostro marito. - Infatti, proprio in quel momento, Aleksej Aleksandrovic con la sua andatura molle e sgraziata entrava nel salotto. 

Visti la moglie e Vronskij, si avvicinò alla padrona di casa e, sedutosi a bere una tazza di tè, prese a parlare con quella sua voce lenta e penetrante, con quel suo tono abitualmente scherzoso, come se prendesse in giro qualcuno. 

- Il vostro Rambouillet è al completo - disse, esaminando tutta la compagnia; - le Grazie e le Muse. 

Ma la principessa Betsy non tollerava questo suo tono, sneering come lo chiamava lei, e, da padrona di casa intelligente, lo avviò subito a una conversazione seria sul servizio militare obbligatorio. Aleksej Aleksandrovic fu subito preso dall'argomento e cominciò a difendere la nuova disposizione contro la principessa Betsy che la avversava. 

Vronskij e Anna continuavano a star seduti alla tavola piccola. 

- La cosa diventa scandalosa - mormorò una signora, indicando con gli occhi la Karenina, Vronskij e il marito di lei. 

- Cosa vi ho detto io? - rispondeva l'amica di Anna. 

Non solo queste signore, ma quasi tutti quelli che erano nel salotto, perfino la principessa Mjagkaja e la stessa Betsy, guardarono parecchie volte i due che si erano staccati dalla cerchia generale come se ne fossero infastiditi. Solo Aleksej Aleksandrovic non guardò neppure una volta da quella parte e non si distrasse dall'interesse della conversazione iniziata. 

Notando la cattiva impressione prodotta su tutti, la principessa Betsy mise al proprio posto un'altra persona ad ascoltare Karenin, e si accostò ad Anna. 

- Sono sempre sorpresa dalla chiarezza ed esattezza di esposizione di vostro marito - disse. - I concetti più trascendentali mi diventano accessibili quando parla lui. 

- Oh, sì - disse Anna, illuminandosi di un sorriso di felicità e senza capire una parola di quello che le andava dicendo Betsy. Si avvicinò alla tavola grande e prese parte alla conversazione generale. 

Aleksej Aleksandrovic, dopo essere rimasto mezz'ora, si avvicinò alla moglie e le propose di andare a casa; ma lei, senza guardarlo, rispose che rimaneva a cena. Aleksej Aleksandrovic salutò ed uscì. 

Il cocchiere della Karenina, un vecchio tartaro panciuto, con una giacca lustra di pelle, tratteneva a stento il cavallo grigio di sinistra che, intirizzito, s'impennava all'ingresso. Un servitore, diritto impalato, apriva lo sportello, mentre il portiere, in piedi, teneva la porta esterna. Anna Arkad'evna con la mano piccola e agile andava staccando i pizzi della manica da un gancio della pelliccia e, chinando la testa, ascoltava incantata quello che Vronskij le andava dicendo nell'accompagnarla. 

- Voi non avete detto nulla; va bene, neanche io pretendo nulla - diceva - ma voi sapete che non è l'amicizia di cui ho bisogno; per me è possibile una sola felicità nella vita, quella parola che tanto vi spiace\ldots{} sì, l'amore\ldots{} 

- L'amore\ldots{} - ripeté lentamente lei con una voce che proveniva dall'intimo del suo essere, e a un tratto, proprio nel momento in cui si staccava il pizzo, aggiunse: - Non mi piace questa parola anche perché significa qualcosa di troppo grande per me, molto più grande di quello che voi possiate immaginare - e lo guardò in viso. - A rivederci. 

Gli tese la mano, e col passo svelto ed elastico passò accanto al portiere e scomparve nella carrozza. 

Lo sguardo di lei, il contatto della sua mano, lo bruciarono. Baciò la palma nel punto in cui era stata toccata da lei, andò a casa felice, convinto d'essersi accostato al suo scopo, in quella sera, molto più che negli ultimi due mesi. 

\capitolo{VIII}\label{viii-1} 

Aleksej Aleksandrovic non aveva trovato nulla di singolare e di sconveniente nel fatto che sua moglie fosse rimasta insieme con Vronskij a una tavola separata, parlando animatamente di qualche cosa; ma aveva notato che a tutti nel salotto questo era parso singolare e sconveniente, perciò era parso sconveniente pure a lui. Decise di parlarne alla moglie. 

Tornato a casa, Aleksej Aleksandrovic, come al solito, andò nel suo studio e sedette in una poltrona, aprendo, nel punto segnato dal tagliacarte, un libro sul cattolicesimo e, come al solito, rimase a leggere fino all'una; soltanto, di quando in quando, si passava una mano sulla fronte alta e scoteva il capo come ad allontanare qualcosa. All'ora solita si alzò, e fece la sua toletta notturna. Anna Arkad'evna non c'era ancora. Con il libro sotto il braccio andò su; ma quella sera, invece dei soliti pensieri e delle solite considerazioni sugli affari di ufficio, la sua testa era piena della moglie e di qualcosa di spiacevole che la riguardava. Contrariamente alle proprie abitudini non si mise a letto ma, incrociate le mani dietro la schiena, cominciò ad andare su e giù per le stanze. Non poteva coricarsi, sentiva di dover prima riflettere su di una circostanza sorta di recente. 

Gli era sembrato facile e semplice decidere di parlare a sua moglie; ma ora che aveva preso a riflettere sulla circostanza sorta di recente, la cosa gli appariva complessa e difficile. 

Aleksej Aleksandrovic non era geloso. La gelosia, secondo lui, offendeva la moglie e nella moglie si doveva aver fiducia. Perché egli dovesse aver fiducia, perché, cioè, dovesse avere la sicurezza piena che la sua giovane moglie lo avrebbe sempre amato, non se lo chiedeva; ma non provava sfiducia perché aveva fiducia, e diceva a se stesso che si dovesse averne. Ora invece, benché la sua convinzione, che la gelosia è un sentimento riprovevole e che si doveva aver fiducia, non fosse stata distrutta, sentiva di trovarsi di fronte a qualcosa di illogico e di assurdo, e non sapeva cosa fare. Aleksej Aleksandrovic veniva a trovarsi di fronte alla vita, di fronte alla possibilità che sua moglie si innamorasse di qualcun altro che non fosse lui, e ciò gli sembrava assurdo e incomprensibile, proprio perché questo era la vita stessa. Aleksej Aleksandrovic aveva vissuto e lavorato tutta la vita negli ambienti burocratici che hanno a che fare con i riflessi della vita. E ogniqualvolta si era imbattuto nella vita vissuta, se ne era scostato. In questo momento provava una sensazione simile a quella di un uomo che, traversato tranquillamente un precipizio su di un ponte, si accorgesse improvvisamente che il ponte è crollato e che sotto c'era un abisso. L'abisso era la vita così come è; il ponte quella vita artificiale che aveva vissuta. Per la prima volta gli si affacciava alla mente l'ipotesi che sua moglie potesse amare un altro, ed egli inorridiva di fronte a questo. 

Senza essersi spogliato, andava avanti e indietro, con passo eguale, sul pavimento di legno scricchiolante della sala da pranzo illuminata da un'unica lampada, sul tappeto del salotto oscuro in cui la luce si rifletteva solo sul suo grande ritratto fatto da poco, appeso sopra il divano, e attraverso lo studiolo di lei, dove ardevano due candele che davan luce ai ritratti dei parenti e delle amiche e agli oggetti belli della scrivania a lui così noti da tempo. Attraversando lo studiolo giungeva alla porta della stanza da letto e voltava di nuovo indietro. 

A ogni giro del suo percorso, e soprattutto quando giungeva sul pavimento di legno della stanza da pranzo illuminata, si fermava e diceva a se stesso: ``Sì; è assolutamente necessario risolvere e far cessare tutto, esprimere la propria idea e la propria decisione''. E si voltava indietro. ``Ma esprimere che cosa? quale decisione?'' diceva a se stesso nel salotto, e non trovava risposta. ``Ma, dopo tutto - si chiedeva prima di voltare nello studiolo - che cosa è mai successo? Nulla. Ha parlato a lungo con lui, ebbene?\ldots{} Con chi non può parlare una donna in società? E, poi, essere geloso vuol dire umiliare se stesso e lei'' si diceva, entrando nello studiolo; ma questa convinzione che prima aveva tanto peso per lui, ora non ne aveva alcuno e non significava nulla. E dalla porta della camera tornava di nuovo verso la sala da pranzo; ma non appena rientrava nel salotto oscuro, una voce gli diceva che non era così e che se gli altri l'avevano notato, voleva dire che qualcosa c'era. E di nuovo, in sala da pranzo, si diceva: ``Sì, è assolutamente necessario risolvere e far cessare tutto ed esporre il proprio punto di vista\ldots{}''. E di nuovo, nel salotto, prima di voltare, si domandava: ``Ma in che modo decidere?''. E dopo, ancora: ``Che cosa è successo?''. E rispondeva: ``Nulla'' e tornava a ripetere a se stesso che la gelosia è un sentimento che avvilisce la moglie, mentre di nuovo, nel salotto, tornava a convincersi che qualcosa c'era stato. I suoi pensieri, così come la sua persona, compivano un intero giro, senza imbattersi in nulla di nuovo. Egli notò questo, si passò una mano sulla fronte e sedette nello studiolo di lei. 

Qui, guardando sullo scrittoio dove c'erano un tampone di malachite e un biglietto cominciato, i suoi pensieri cambiarono improvvisamente corso. Cominciò a pensare a lei, a quello che ella avrebbe potuto pensare e sentire. Per la prima volta si rappresentò con chiarezza la vita intima di lei, i suoi pensieri, i suoi desideri; e l'idea che ella potesse avere una vita tutta propria gli sembrò così spaventosa che s'affrettò a scacciarla. Era questo l'abisso nel quale era così pauroso guardare. Trasferirsi col pensiero e col sentimento in un altro essere era un'azione spirituale estranea ad Aleksej Aleksandrovic. Egli la considerava come dannosa e pericolosa fantasticheria. 

``E la cosa più terribile - pensava - è che ora, proprio quando la mia questione si approssima alla conclusione - alludeva al progetto che stava facendo passare - quando ho bisogno di tutta la serenità e di tutte le forze dello spirito, proprio ora mi si scaraventa addosso questa insensata inquietudine. Ma, che fare? Io non sono di quegli uomini che soffrono agitazioni e inquietudini senza aver la forza di affrontarle''. 

- Bisogna riflettere, decidere e sistemare tutto - disse ad alta voce. 

``La questione dei suoi sentimenti, di quello che avviene e può avvenire nell'anima sua non è affar mio; riguarda la sua coscienza e riguarda la religione'' si diceva, provando sollievo nel trovare il lato normativo al quale soggiaceva la circostanza che era sorta. 

``È così - si disse Aleksej Aleksandrovic - la questione dei suoi sentimenti e il resto sono questioni della sua coscienza con la quale io non ho nulla da spartire. Il mio dovere, d'altra parte, è chiaramente determinato. Come capo della famiglia, e come persona tenuta a guidarla e perciò in parte responsabile, devo prospettarle il pericolo che vedo, metterla in guardia e adoperare perfino la mia autorità. Devo parlarle''. 

E nella mente di Aleksej Aleksandrovic si andò formulando chiaramente tutto quello ch'egli avrebbe detto alla moglie. Riflettendo a quello che avrebbe detto, rimpiangeva di dover adoperare, a scopi domestici e in maniera così insignificante, il proprio tempo e le proprie facoltà intellettuali; nonostante ciò, nella testa gli si vennero a comporre, chiari e distinti, così come in una relazione ministeriale, la forma e lo svolgimento del discorso da fare. ``Devo esprimermi in questo ordine: in primo luogo, dimostrare l'importanza dell'opinione pubblica e delle convenienze sociali; in secondo luogo, precisare i valori religiosi del matrimonio; in terzo luogo, se necessario, indicare il danno che potrebbe derivare al figlio; in quarto luogo, prospettarle la sua stessa infelicità''. E incrociate le dita le une nelle altre, con le palme all'ingiù, Aleksej Aleksandrovic le stiracchiò e le dita scricchiolarono nelle giunture. 

Questo gesto, questa cattiva abitudine di riunire le mani e far scricchiolare le dita, lo tranquillizzava sempre, e gli dava quel senso di precisione che in questo momento gli era tanto necessario. Si sentì il rumore di una carrozza che giungeva all'ingresso. Aleksej Aleksandrovic si fermò in mezzo alla sala. 

Sulla scala risonarono dei passi femminili. Aleksej Aleksandrovic, pronto per il suo discorso, stava in piedi, stringendo le dita incrociate e provando se in qualche giuntura volessero ancora scricchiolare. Una giuntura scricchiolò. 

Dal suono dei passi leggeri su per la scala, egli sentì l'approssimarsi di lei; e, pur essendo soddisfatto del proprio discorso, ebbe paura della spiegazione imminente\ldots{} 

\capitolo{IX}\label{ix-1} 

Anna camminava a testa china, giocherellando con le nappine del cappuccio. Il suo viso emanava un bagliore vivo; ma questo bagliore non era gaio, ricordava il bagliore sinistro di un incendio in una notte oscura. Visto il marito, Anna alzò il capo e, come svegliandosi, sorrise. 

- Non sei a letto? Oh, ma questo è un miracolo! - disse, togliendosi il cappuccio e, senza fermarsi, proseguì verso lo spogliatoio. - È ora, Aleksej Aleksandrovic - disse di là dalla porta. 

- Anna, ho bisogno di parlare con te. 

- Con me? - disse lei sorpresa, uscendo dalla porta e guardandolo. - Cos'è mai? Di che si tratta? - chiese, sedendosi. - Parliamo pure, se è proprio tanto necessario. Sarebbe meglio dormire, però. 

Anna diceva quel che le veniva sulle labbra e, nell'ascoltarsi, stupiva della propria capacità di mentire. Come erano semplici e naturali le sue parole e come era verosimile il fatto ch'ella avesse proprio sonno! Si sentiva rivestita d'un'impenetrabile maglia d'inganno. Sentiva che una forza invisibile l'aiutava e la sosteneva. 

- Anna, devo metterti in guardia - egli disse. 

- Mettermi in guardia? - rispose lei. Ella appariva così schietta e allegra che chiunque non l'avesse conosciuta non avrebbe notato nulla di straordinario nel suono e nel senso delle sue parole. Ma per lui che la conosceva, che sapeva come ella notasse perfino se egli andava a letto cinque minuti più tardi e ne chiedeva la ragione; per lui che sapeva come ella gli confidasse ogni sua gioia, ogni allegrezza e ogni suo dispiacere, per lui vedere come in questo momento ella non volesse accorgersi dello stato suo e nulla volesse dire di sé, significava molto. Sentiva che il fondo dell'animo suo, che un tempo gli si offriva, gli veniva ora precluso. Non solo, ma dal suo tono sentiva che tutto questo non turbava minimamente lei, ed era come se gli dicesse sul viso: ``sì, è precluso, e così sarà d'ora in poi''. Provava una sensazione simile a quella di un uomo che nel tornare a casa trovi la propria casa chiusa. 

``Ma forse se ne troverà ancora la chiave'' pensava Aleksej Aleksandrovic. 

- Ti voglio mettere in guardia - disse a voce bassa - perché tu non dia, per incoscienza o leggerezza, motivo di far parlare di te in società. Il tuo colloquio di oggi troppo vivace con il conte Vronskij - pronunciò fermamente e dopo una tranquilla pausa questo nome - ha attirato su di te l'attenzione. 

Egli parlava e guardava gli occhi ridenti di lei, ormai paurosi per la loro impenetrabilità, e parlando sentiva tutta la vanità e l'oziosità delle proprie parole. 

- Tu sei sempre così - rispondeva lei, come se non riuscisse a capirlo in nessun modo e come se di tutto quello ch'egli aveva detto avesse afferrato solo l'ultima cosa. - Un momento ti spiace che io mi annoi, un momento che io sia allegra. Non mi sono annoiata. Questo forse ti offende? 

Aleksej Aleksandrovic ebbe un brivido, piegò le mani per farle scricchiolare. 

- Ah, ti prego, non le fare scricchiolare, non mi piace - disse lei. 

- Ma, Anna, sei proprio tu? - disse Aleksej Aleksandrovic piano, facendo uno sforzo su di sé per trattenersi dal gesto abituale delle mani. 

- Ma cos'è mai? - disse lei con uno stupore comicamente sincero. - Che vuoi da me? 

Aleksej Aleksandrovic tacque, si fregò la fronte e gli occhi con una mano. Si accorgeva che invece di quello che voleva fare, mettere cioè in guardia la moglie da quello che poteva apparire un errore agli occhi del mondo, si agitava involontariamente per quello che riguardava la coscienza di lei, e lottava contro un muro creato dalla sua stessa immaginazione. 

- Ecco quello che intendo dirti - continuò freddo e tranquillo - e ti chiedo di ascoltarmi. Come sai, io ritengo che la gelosia offenda e umilii, e non mi permetterò mai di lasciarmi andare a questo sentimento; ma ci sono certe leggi di convenienza che non possono essere impunemente trasgredite. Non sono stato io a notarlo quest'oggi, ma è l'impressione generale prodotta sulla compagnia; tutti hanno notato che il tuo contegno e il tuo comportamento non erano quali precisamente si potevano desiderare. 

- Non capisco proprio nulla - disse Anna, stringendosi nelle spalle. ``A lui personalmente non importa alcun che, ma la compagnia lo ha notato, e lui se ne inquieta''. - Tu stai poco bene, Aleksej Aleksandrovic - aggiunse, alzandosi per uscire dalla porta; ma egli le si parò innanzi, quasi a fermarla. 

Il suo viso era torvo e tetro come Anna non l'aveva mai veduto. Ella si fermò e, buttando il capo all'indietro, da un lato, prese a toglier via le forcine con la mano agile. 

- Ebbene, io ascolto quel che devi dirmi - disse con calma e irrisione. - E ascolto anzi con interesse, perché vorrei capire di che cosa si tratta. 

Parlava, e si stupiva del tono calmo e sincero che le veniva naturale e della scelta delle parole che adoperava. 

- Io non ho alcun diritto di entrare in fondo ai tuoi sentimenti, anzi in genere ritengo ciò inutile e perfino dannoso - cominciò Aleksej Aleksandrovic . - Tante volte, scavando nell'anima nostra, ne facciamo venir fuori qualcosa che sarebbe rimasto inosservato. I tuoi sentimenti riguardano la tua coscienza; ma io ho l'obbligo verso di te, verso di me e verso Dio di indicarti i tuoi doveri. La nostra vita è stata legata non dagli uomini, ma da Dio. Solo un delitto può infrangere questo legame, e un delitto di tal genere porta con sé una pena. 

- Non capisco nulla. Ah, Dio mio! e, per mia disgrazia, ho tanta voglia di dormire! - disse lei in fretta, toccando con la mano i capelli per cercarvi le forcine rimaste. 

- Anna, in nome di Dio, non parlare così - disse lui sommesso. - Può darsi che io mi sbagli, ma credimi, quello che dico lo dico tanto per me come per te. Io sono tuo marito e ti amo. 

Per un attimo la testa di lei si chinò e la luce ironica degli occhi si spense; ma la parola ``amo'' la irritò di nuovo. Pensò: ``Ama? Può forse amare lui? Se non avesse sentito dire che esiste l'amore, non avrebbe neanche mai usato questa parola. Ma lui non sa neppure cosa sia l'amore!''. 

- Aleksej Aleksandrovic, davvero, non capisco - disse. - Precisa quello che pensi\ldots{} 

- Lasciami parlare, ti prego. Io ti amo. Ma io non parlo di me; qui le persone principali siete voi, tu e nostro figlio. Può darsi benissimo, ripeto, che le parole ti sembrino del tutto inutili e fuori posto; forse sono provocate da un mio smarrimento. In questo caso ti prego di perdonarmi. Ma se tu stessa senti che c'è anche il più piccolo fondamento, allora, ti prego, pensaci, e, se il cuore te lo dice, confidati\ldots{} 

Aleksej Aleksandrovic, senza rendersene conto, diceva cose affatto diverse da quelle che aveva preparate. 

- Non ho nulla da dire. E poi\ldots{} - ella disse in fretta, trattenendo a stento un sorriso - davvero è ora di dormire. 

Aleksej Aleksandrovic sospirò e, senza dir più nulla, si diresse in camera. 

Quando ella entrò, egli era già a letto. Le sue labbra erano severamente strette e gli occhi non la guardavano. Anna si coricò nel suo letto, aspettando ch'egli da un momento all'altro riprendesse a parlare. Ne aveva insieme paura e desiderio. Ma egli taceva. Ella attese a lungo, immobile, ma già lo aveva dimenticato: pensava all'altro, vedeva l'altro e sentiva che il cuore a questo pensiero le si riempiva di ansia e di gioia colpevole. A un tratto sentì un ronfio nasale, eguale e calmo. Dapprima Aleksej Aleksandrovic si spaventò quasi del proprio russare e si fermò, ma, dopo due respiri, il ronfio si fece sentire calmo e cadenzato. 

- È tardi, è tardi ormai - mormorò lei con un sorriso. Rimase a lungo immobile con gli occhi aperti e le sembrava di vedere lei stessa, nel buio, il loro bagliore. 

\capitolo{X}\label{x-1} 

Da quella sera cominciò una nuova vita per Aleksej Aleksandrovic e sua moglie. Non accadde nulla di straordinario. Anna continuò a frequentare il gran mondo, andava spesso, più che altrove, dalla principessa Betsy, e s'incontrava con Vronskij dovunque. Aleksej Aleksandrovic rilevava tutto questo, ma non poteva farci nulla. A tutti i tentativi per portarla ad una spiegazione, ella opponeva il muro impenetrabile del suo allegro stupore. Esteriormente tutto era come prima, ma i loro rapporti intimi si erano completamente mutati. Aleksej Aleksandrovic, l'uomo così energico negli affari di stato, si sentiva impotente. Come un bue, aspettava, con il capo abbassato, la mazza che sentiva sospesa su di sé. Ogni qualvolta ci pensava, sentiva che era necessario tentare qualcosa, sentiva che, con la bontà, la tenerezza, la persuasione, c'era ancora la speranza di salvarla, di farla rientrare in sé, e ogni giorno si disponeva a parlare. Ma appena cominciava a parlare con lei, sentiva che lo spirito del male e dell'inganno che la possedeva s'impossessava anche di lui, ed egli parlava di cose del tutto diverse e con un tono contrario a quello che avrebbe voluto usare. Suo malgrado, parlava con lei con quell'abituale tono di canzonatura, come se proprio così volesse parlare. E con questo tono non si poteva dire ciò che era necessario dire. 

\capitolo{XI}\label{xi-1} 

Quello che per Vronskij era stato, per quasi un anno, l'unico, esclusivo desiderio che si era sostituito a tutti i desideri della sua vita, quello che per Anna era un impossibile, pauroso e così fascinoso sogno di felicità, quel desiderio era soddisfatto. Pallido, con la mascella inferiore che tremava, egli stava in piedi, chino su di lei, e la supplicava di calmarsi, non sapendo egli stesso di che, di che cosa. 

- Anna, Anna - diceva, con voce tremante - Anna, in nome di Dio! 

Ma quanto più forte egli parlava, tanto più bassa ella chinava la testa, un tempo orgogliosa e gaia, ora vergognosa; e si piegava tutta e scivolava dal divano sul quale era poggiata verso terra, ai piedi di lui; sarebbe caduta sul tappeto s'egli non l'avesse sorretta. 

- Dio mio, perdonami! - diceva, singhiozzando, stringendo al petto le mani di lui. 

Si sentiva così colpevole e peccatrice che non le restava che prostrarsi e chiedere perdono; ma adesso, nella sua vita, all'infuori di lui, non c'era più nessuno, e a lui volgeva la sua preghiera di perdono. Guardandolo, sentiva fisicamente la propria abiezione, e non poteva più parlare. Egli, invece, sentiva quello che deve sentire l'assassino quando vede il corpo da lui privato della vita. Questo corpo da lui privato della vita era il loro amore, il primo tempo del loro amore. C'era orrore e ripugnanza nel ricordare quello ch'era stato pagato a un così pauroso prezzo di vergogna. La vergogna dinanzi alla propria nudità spirituale soffocava lei e si comunicava a lui. Ma nonostante tutto l'orrore dell'assassino dinanzi al corpo assassinato, occorre fare a pezzi questo corpo, nasconderlo, valersi di ciò che l'assassino, uccidendo, ha conquistato. 

E con accanimento, con furore quasi, colui che ha ucciso si getta su questo corpo, e lo trascina e smembra: così anch'egli copriva di baci il viso e le spalle di lei. Ella gli teneva stretta una mano e non si moveva. Ecco, questi baci sono il prezzo di questa vergogna. Anche questa mano che sarà sempre mia, è la mano del mio complice. Sollevò la mano e la baciò. Egli si piegò sulle ginocchia e voleva scoprirle il viso, ma lei si nascondeva e non diceva nulla. Finalmente, facendo uno sforzo, si sollevò e lo respinse. Il suo viso era sempre bello, ma faceva tanta più pena. 

- Tutto è finito - disse. - Non ho nessuno all'infuori di te. Ricordalo. 

- Io non posso non ricordare quello che è la mia vita. Per me, un attimo di questa felicità\ldots{} 

- Quale felicità! - disse lei con ribrezzo e orrore; e l'orrore si comunicò a lui. - Per amor di Dio, non una parola, non una parola di più. 

Si alzò in fretta e si scostò. 

- Non una parola di più - ripeté e, con un'espressione strana, a lui sconosciuta, di fredda disperazione, andò via. Sentiva di non poter dire la vergogna, la gioia e l'orrore che provava nell'entrare in quella nuova vita, e non voleva parlarne e non voleva rendere volgare, con parole inadatte, quel che sentiva. Ma anche dopo, l'indomani, e il giorno seguente, non trovò le parole adatte a dire tutto il complesso delle sue sensazioni, e così neppure le idee adatte a mettere ordine nell'animo suo. 

``No, adesso non posso pensare - si diceva - dopo, quando sarò tranquilla''. Ma questa tranquillità per riflettere non veniva mai; ogni volta che le tornava in mente quello che aveva fatto, quello che sarebbe stato di lei e quello che doveva fare, era presa dallo sgomento e allontanava questi pensieri. 

``Dopo, dopo - diceva - quando sarò più tranquilla''. 

Nel sonno, invece, quando non aveva il dominio dei suoi pensieri, la situazione le appariva in tutta la sua informe nudità. Un unico identico sogno la visitava quasi ogni notte. Sognava che tutti e due erano nello stesso tempo suoi mariti, che tutti e due le prodigavano le loro carezze. Aleksej Aleksandrovic piangeva, baciandole le mani, e diceva: ``Come si sta bene, ora!''. E Aleksej Vronskij era là, e anche lui era suo marito. Ed ella stupiva come questo le fosse apparso prima impossibile, e spiegava loro, ridendo, che era molto più semplice, e che ora entrambi erano felici e contenti. Ma questo sogno la soffocava come un incubo. 

\capitolo{XII}\label{xii-1} 

Ancora nei primi tempi dopo il suo ritorno da Mosca, Levin, fremendo ed arrossendo ogni volta che ricordava l'offesa del rifiuto, finiva col dire a se stesso: ``Arrossivo e fremevo proprio così giudicando tutto perduto, quando presi uno in fisica e dovetti ripetere l'anno; così pure mi considerai fallito quando persi la causa affidatami da mia sorella. Ebbene?\ldots{} ora che gli anni sono passati, ricordo e stupisco come abbia potuto addolorarmene tanto. Sarà lo stesso anche per questo dispiacere. Passerà il tempo, e diverrò indifferente anche a questo''. 

Ma erano passati tre mesi e non diventava indifferente, e gli doleva, come nei primi giorni, questo ricordo. Non riusciva a rasserenarsene, perché, dopo aver sognato così a lungo una vita di famiglia, e sentendosi ormai maturo per essa, non s'era sposato, e s'era più che mai allontanato dal matrimonio. Sentiva, come lo sentivano tutti quelli che lo circondavano, che per un uomo della sua età rimaner celibe era un male. Ricordava che prima di partire per Mosca, aveva detto un giorno a Nikolaj il bovaro, un brav'uomo col quale amava parlare: ``Ehi, Nikolaj, voglio prender moglie'', e Nikolaj aveva risposto senza indugio, come di una cosa di cui non s'avesse a dubitare: ``È tempo da un pezzo, Konstantin Dmitric''. Ma il matrimonio s'era fatto più lontano che mai. Il posto nel suo cuore era occupato, e quando gli capitava di sostituirvi nell'immaginazione qualcuna delle ragazze di sua conoscenza, sentiva che tale sostituzione era assolutamente impossibile. Inoltre il ricordo del rifiuto e della parte che aveva recitato in quell'occasione, lo tormentava di vergogna. Per quanto si dicesse che non era per nulla colpevole, questo ricordo, al pari degli altri ricordi umilianti di tal genere, lo costringeva a rabbrividire e ad arrossire. Nel suo passato, come in quello di ogni uomo, c'erano delle cattive azioni da lui riconosciute come tali, per le quali la coscienza avrebbe dovuto rimordergli; ma il ricordo di queste cattive azioni era ben lungi dal tormentarlo allo stesso modo di questi inconsistenti, ma umilianti ricordi. Questa ferita non si rimarginava mai. E nel ricordo venivano a trovarsi adesso, sullo stesso piano, e il rifiuto e quella situazione penosa in cui era apparso agli altri in quella sera. Ma il tempo e le occupazioni facevano l'opera loro. I ricordi penosi venivano sempre più velati dagli impercettibili, ma significativi avvenimenti della vita di campagna. Di settimana in settimana ricordava sempre più di rado Kitty. Aspettava con ansia la notizia che si fosse sposata o stesse per sposarsi a giorni; sperava che una notizia simile, come l'estirpazione di un dente, finisse col guarirlo. 

Sopraggiunse intanto la primavera, splendida, improvvisa, senza le attese e gli inganni delle primavere; una di quelle primavere di cui si rallegrano insieme e piante e bestie e uomini. Questa primavera bellissima rianimò ancor più Levin e lo confermò nel suo proposito di rinunciare a tutti i suoi sogni precedenti per costruire, salda e indipendente, la sua vita di uomo solo. Pur non avendo mantenuto fede a molti propositi che aveva formulato nel viaggio di ritorno, tuttavia, l'aspirazione prima, la continenza di vita, egli l'aveva osservata. Non provava la vergogna che di solito lo tormentava dopo ogni caduta, e poteva coraggiosamente guardare in faccia agli uomini. Inoltre, in febbraio, aveva ricevuta da Mar'ja Nikolaevna una lettera in cui si diceva che le condizioni di salute del fratello erano peggiorate, e che egli non voleva curarsi; in seguito a questa lettera, Levin era andato a Mosca e aveva fatto in tempo a persuadere il fratello a consigliarsi con un medico e ad andare all'estero per la cura delle acque. Gli era riuscito così bene di convincere il fratello e di dargli in prestito, senza irritarlo, del denaro per il viaggio, che, sotto questo rapporto, era soddisfatto di sé. Oltre l'azienda che esigeva cure particolari in primavera, Levin aveva anche cominciato a scrivere, in quell'inverno, un libro di economia, la cui tesi consisteva nell'assumere in economia il temperamento del lavoratore come un dato assoluto, così come il suolo e il clima, e nel sostenere che tutte le tesi dell'economia dovessero essere di conseguenza dedotte non dai soli dati del suolo e del clima, ma da quelli del suolo, del clima e di un certo immutabile temperamento del lavoratore. Così che, malgrado la solitudine, e anzi proprio per la solitudine, la sua vita era straordinariamente ricca, e solo di rado sentiva il bisogno insoddisfatto di comunicare i pensieri che gli passavano per la testa a qualcuno che non fosse Agaf'ja Michajlovna, benché anche con lei gli accadesse di ragionar di fisica, di agraria e in particolare di filosofia; la filosofia, anzi, era l'argomento preferito da Agaf'ja Michajlovna. 

La primavera aveva tardato ad arrivare. Nelle ultime settimane della quaresima il tempo era stato sereno, gelido. Di giorno, al sole, sgelava; di notte la temperatura scendeva a sette gradi sotto lo zero. La neve era così indurita che i carri non seguivano più la strada. Per Pasqua c'era ancora la neve. Ma due giorni dopo la settimana santa, si levò a un tratto un vento tiepido, le nuvole si addensarono, e per tre notti cadde una pioggia burrascosa e calda. Il giovedì, il vento si calmò e, quasi a nascondere il mistero dei cambiamenti che si operavano nella natura, avanzò una nebbia fitta e grigia. Nella nebbia si sciolsero le acque, crepitarono e si smossero i ghiacci, più rapidi corsero i torrenti torbidi e schiumosi, e proprio per la domenica in Albis, la sera si squarciò la nebbia, le nuvole corsero via a pecorelle, si rasserenò, e si schiuse la primavera. Al mattino il sole, levatosi splendidamente, divorò in fretta il ghiaccio sottile che aveva coperto le acque, e l'aria trepidò dei vapori che si sprigionavano dalla terra rianimata, invadendola tutta. Verzicò l'erba vecchia e la novella che spuntava ad aghi; si gonfiarono le gemme del viburno, del ribes e della betulla viscosa e inebriante, e su di un ramo di salice, soffuso di fiori d'oro, prese a ronzare un'ape rimasta fuori che vagava all'intorno. Allodole invisibili presero a trillare sul velluto delle verzure e sulla stoppia gelata; piansero le pavoncelle sulle bassure e sulle paludi piene d'acqua nera non ancora riassorbita, e in alto, a volo, con un gridìo di primavera, passarono cicogne e oche. Gli armenti, che non avevano ancora del tutto mutato il pelo, presero a muggire nei pascoli, e gli agnelli dalla zampe ritorte ruzzarono intorno alle madri belanti che mutavano il vello, mentre i ragazzi dalle gambe agili presero a correre per i tratturi che, rasciugandosi, conservavano le impronte dei piedi scalzi; accanto allo stagno crepitarono le voci allegre delle comari intente a candeggiar le tele, e sulle aie risonarono le accette dei contadini che racconciavano aratri ed erpici. Era venuta la vera primavera. 

\capitolo{XIII}\label{xiii-1} 

Levin infilò gli stivali alti e, per la prima volta, indossò, invece della pelliccia, un giubbotto di panno, e s'avviò per il podere, saltando fra i rigagnoli che ferivano gli occhi luccicando al sole, e mettendo il piede ora su un ghiacciolo ora sul fango viscido. 

La primavera è il tempo dei progetti e dei propositi. Uscendo fuori, Levin, come un albero che non sa ancora, in primavera, dove e come spunteranno i germogli e i rami racchiusi nelle gemme turgide, non sapeva egli stesso bene a quali imprese si sarebbe particolarmente accinto ora, nella sua cara azienda; sentiva solo d'aver dentro di sé un mondo di pensieri e i migliori propositi. Per prima cosa andò a dare un'occhiata al bestiame. Le mucche erano state sospinte nel recinto e, luccicanti nel pelo liscio or ora mutato, riscaldatesi al sole, muggivano chiedendo di andare nei prati. Compiaciuto delle mucche che conosceva fin nei più piccoli particolari, Levin ordinò che venissero condotte al pascolo, e che nel recinto si lasciassero circolare i vitelli. Il mandriano corse allegro a prepararsi per andar nei campi. Le donne, sollevando le gonne e guazzando nel fango con i bianchi piedi nudi, non ancora abbronzati, correvano tenendo in mano frasche secche dietro i vitelli che muggivano e ruzzavano di gioia primaverile, e li sospingevano nel cortile. 

Soddisfatto dell'incremento del bestiame, che quell'anno era stato eccezionalmente fecondo, (i vitelli, precoci, erano come vacche da lavoro, la figlia di Pava, di tre mesi appena, sembrava già di un anno), Levin fece portar fuori la mangiatoia e dare il fieno fuori dalle greppie. Ma nel recinto chiuso, non adoperato nell'inverno, constatò che le greppie costruite nell'autunno erano rotte. Fece chiamare il falegname a cui era stato dato l'ordine di lavorare ad una trebbiatrice. Gli dissero che il falegname, invece, stava riparando gli erpici che avrebbero dovuti essere pronti fin da carnevale. Questo spiacque molto a Levin. Era infatti spiacevole che si ripetesse l'eterno disordine dell'azienda, contro il quale da tanti anni lottava con tutte le sue forze. Venne a sapere che le greppie, inutilizzabili d'inverno, erano state trasferite nella stalla dei cavalli da tiro, e là s'erano spezzate perché, costruite per i vitelli, erano risultate troppo leggere per i cavalli. Inoltre, era ormai chiaro che gli erpici e tutti gli strumenti agricoli che egli aveva ordinato di esaminare e di riparare durante l'inverno (lavoro pel quale erano stati assunti tre falegnami), non erano stati riparati, e che agli erpici si andava provvedendo ora che era già tempo di erpicare. Levin mandò a chiamare l'amministratore, e poi andò a cercarlo egli stesso. L'amministratore, risplendente, come ogni cosa in quel giorno, in un pellicciotto di montone guarnito d'agnina, veniva dall'aia, sminuzzando nelle mani una pagliuzza. 

- Perché il falegname non lavora alla trebbiatrice? 

- Eh, già, ve lo volevo dire ieri; era necessario accomodare gli erpici. Ecco che è già tempo d'arare. 

- E allora d'inverno che s'è fatto? 

- Ma perché vi occorre il falegname? 

- Dove sono le greppie del recinto dei vitelli? 

- Ho detto di portarle al posto loro. Che volete fare, con questa gente\ldots{} - disse l'amministratore, con un gesto della mano. 

- Altro che con questa gente! Con questo amministratore! - disse Levin, riscaldandosi. - Ma allora che vi tengo a fare? - gridò. Ma poi, ricordandosi che così non riparava a nulla, si fermò a mezzo il discorso e sospirò. - Su via, si può seminare? - domandò dopo essere rimasto per un po' in silenzio. 

- Al di là di Turkin sì, che si potrà, domani o domani l'altro. 

- E il trifoglio? 

- Ho mandato Vasilij e Miška a seminare. Ma non so se riusciranno a passare: c'è fango. 

- Su quante desjatiny? 

- Su sei. 

- E perché non su tutte? - urlò Levin. 

Che il trifoglio venisse seminato soltanto su sei e non su venti desjatiny, era ancora più increscioso. La seminagione del trifoglio, e teoricamente, e per sua personale esperienza, rendeva solo se fatta al più presto possibile e quasi sulla neve. E Levin non riusciva mai a ottenere che così si facesse. 

- Non ci sono gli operai; cosa mai volete che faccia con questa gente? Tre non sono venuti. Ma ecco Semën\ldots{} 

- Ma via, avreste dovuto toglierne dal lavoro della paglia. 

- Ma ne ho tolti anche di là. 

- Dove sono gli operai? 

- Cinque fanno lo sconcio - voleva dire ``il concio''. - Quattro trasportano l'avena\ldots{} ma anche quella, purché non prenda a ``sguigliare'', Konstantin Dmitric! 

Levin intendeva bene che ``purché non prenda a sguigliare'' significava che l'avena inglese da semenza l'avevano già fatta marcire; ancora una volta non era stato fatto quello che aveva ordinato. 

- Ma se l'ho detto che era ancora quaresima, trombone! - gridò. 

- Non v'inquietate, faremo tutto in tempo! 

Levin agitò con rabbia la mano, andò in granaio a dare un'occhiata all'avena, e tornò alla stalla. L'avena non era ancora andata a male; ma gli operai la rimovevano con le pale, quando sarebbe stato più facile farla scendere direttamente nella rimessa sottostante. Dati gli ordini in proposito, e tolti di lì due operai per la semina del trifoglio, Levin, rabbonito, si liberò della collera contro l'amministratore. Il tempo era così bello che non c'era modo di arrabbiarsi. 

- Ignat! - gridò al cocchiere che, con le maniche rimboccate, lavava una carrozza accanto al pozzo. - Metti la sella a\ldots{} 

- Chi volete? 

- Su, magari, vada per Kolpik. 

- Sissignore. 

Mentre sellavano il cavallo, Levin chiamò di nuovo l'amministratore che gli gironzolava intorno con l'evidente intenzione di far pace, e prese a parlargli dei lavori da farsi in primavera e dei suoi progetti agricoli. 

Bisognava cominciare al più presto il trasporto del concio, in modo da finire alla prima falciatura. E arare senza interruzione il campo più lontano per serbarlo come maggese nero. Il fieno bisognava falciarlo tutto, non a mezzadria, ma coi braccianti. 

L'amministratore ascoltava attento, ma era evidente che faceva uno sforzo per dare a intendere che approvava i progetti del padrone, e aveva, suo malgrado, quell'aria sfiduciata e rassegnata, ben nota a Levin, che sempre se ne irritava. Sembrava dire: ``tutto va bene, ma sarà come Dio vorrà''. 

Nulla amareggiava Levin più di questo atteggiamento. Ma era l'atteggiamento comune a tutti gli amministratori, quanti gliene erano passati per le mani. Tutti si comportavano allo stesso modo verso le sue nuove idee, perciò egli non se ne adirava più, ma se ne amareggiava e si sentiva ancor più spinto a lottare contro questa forza primordiale che gli si opponeva continuamente e che egli non sapeva definire altrimenti che ``come Dio vorrà''. 

- Se ce la faremo, Konstantin Dmitric - disse l'amministratore. 

- Perché non si dovrebbe farcela? 

- Bisogna ancora assumere almeno altri quindici operai. Ed ecco che non vengono. Oggi qualcuno è venuto, ma chiedono settanta rubli per l'estate. 

Levin tacque. Di nuovo gli si parava di fronte quella forza. Sapeva che, per quanto si cercasse, non si sarebbe potuto assumere più di quaranta, trentasette, trentotto operai al prezzo giusto: forse anche quaranta se ne potevano assumere, ma certamente non di più; tuttavia non poteva non lottare. 

- Mandateli a cercare a Sury, a cefirovka, se non vengono. Bisogna cercare. 

- Per cercare io cerco - disse sommessamente Vasilij Fëdorovic. - Ma poi, anche i cavalli si sono infiacchiti. 

- Ne compreremo degli altri. Perché io lo so - aggiunse, ridendo - quando fate voi, ne vien fuori sempre il meno e sempre il peggio; ma quest'anno non vi permetterò di fare a modo vostro. Farò tutto io. 

- Ma voi, del resto, anche ora, mi pare, non state dormendo. Del resto, noi viviamo più contenti sotto l'occhio del padrone. 

- Dunque, di là dal Berëzovyj Dol, si semina il trifoglio? Vado a vedere - disse, assestandosi sul piccolo Kolpik, il sauro che era stato condotto dal garzone. 

- Per il ruscello non passerete Konstantin Dmitric - gridò il garzone. 

- Su via, allora, per il bosco. 

E sull'arzilla andatura del buon cavallino che era rimasto a lungo a riposo, e che sbruffava sulle pozzanghere, chiedendo le briglie, Levin si avviò attraverso il fango del cortile, oltre il portone, verso i campi. 

Se Levin si rallegrava nel cortile del bestiame e in quello delle mucche, si rallegrava ancor più nei campi. Dondolandosi alla cadenza dell'ambio del buon cavallino, aspirando l'odore tiepido e fresco dell'aria e della neve, attraversava il bosco sul nevischio rimasto qua e là, sulla neve sfaldata sulla quale le impronte si andavano sciogliendo. Godeva di ogni pianta rigonfia di gemme, avvivata dal musco sulla corteccia. Quando uscì di là dal bosco, dinanzi a lui si distendevano, per uno spazio enorme, i prati verdi, come un liscio tappeto di velluto, senza piazzuole né pozzanghere, macchiati solo qua e là negli avvallamenti dai resti della neve che andava sciogliendosi. Levin non si turbò né alla vista di un cavallo da tiro e di uno stallone che calpestavano i suoi prati (ordinò a un contadino col quale s'era imbattuto di cacciarli via), né alla risposta canzonatoria e sciocca di Ipat, il contadino incontrato, il quale alla sua domanda: ``Ohi, Ipat, si semina presto?'' aveva risposto: ``S'ha prima da arare, Konstantin Dmitric!''. Quanto più andava avanti, tanto più gioiva, e i suoi piani di amministrazione gli sembravano l'uno migliore dell'altro: recingere di giunchi tutti i campi in linee meridiane, di modo che la neve non vi rimanesse a lungo; dividerli in sei campi da concio e in tre di riserva per la coltura delle erbe, costruire una stalla sull'estremo limite del campo e scavare una fossa per l'avena e per il concio, costruire dei recinti trasportabili per il bestiame al pascolo. E così avrebbe avuto trecento desjatiny di frumento, cento di patate, centocinquanta di trifoglio e neanche una desjatiny incolta. 

Con questi sogni, conducendo accorto il cavallo sui viottoli terminali per non calpestare i suoi prati, si avvicinò agli operai che seminavano il trifoglio. Il carro con la semenza era fermo, non sul limite, ma sul campo arato, e il frumento autunnale era solcato dalle ruote e scavato dalle zampe del cavallo. Tutti e due gli operai sedevano sulla proda, fumando la pipa, probabilmente a turno. La terra che era sul carro, frammischiata ai semi, non era impastata, ma tutta impiastricciata e a pallottole. Scorgendo il padrone, l'operaio Vasilij si mosse verso il carro e Miška si diede a seminare. Anche questo non andava bene, ma Levin si adirava di rado con gli operai. Quando Vasilij si avvicinò, Levin gli ordinò di portare il cavallo sulla proda. 

- Non fa nulla, padrone, si rimargina - rispose Vasilij. 

- Ti prego, non stare a discutere - disse Levin - ma fa' quello che ti vien detto. 

- Sissignore - rispose Vasilij e prese il cavallo per la cavezza. - Ma la semenza, Konstantin Dmitric - disse, adulando - è di prima qualità. Solo che camminare è un guaio! Tiri su un pud con un solo piede. 

- E perché non avete setacciato la terra? - disse Levin. 

- Ma la gramoliamo noi - rispose Vasilij, prendendo su della semenza e impastandovi un po' di terra nelle mani. 

Vasilij non aveva colpa lui, se gli avevano messo della terra non setacciata, tuttavia ciò era spiacevole. 

Ma Levin, avendo sperimentato più di una volta, con profitto, un mezzo sicuro per soffocare il proprio dispetto e per far tornare ad andar bene quel che sembrava andar male, lo provò anche in questo momento. Vide che Miška camminava a grandi passi, facendo rotolare enormi zolle di terreno che gli si appiccicavano ai piedi; scese da cavallo, tolse a Vasilij il sacco della semenza e andò a seminare. 

- Dove ti sei fermato? 

Vasilij fece un segno col piede, e Levin andò a seminare, così come sapeva far lui, il terreno misto alla semenza. Andare avanti era difficile, proprio come in un pantano; e Levin, seminato che ebbe un solco, cominciò a sudare e, fermatosi, restituì il sacco con la semenza. 

- Ohi, padrone, bada bene a non prendertela con me questa estate, per questo solco qua! - disse Vasilij. 

- E che c'è - disse allegro Levin, scorgendo già l'effetto del mezzo adoperato. 

- Sì, ecco, vedrete poi quest'estate. Si vedrà la differenza. Date un'occhiata dove ho seminato io la primavera scorsa. Come ho dato la semenza! Ecco, Konstantin Dmitric, io mi adopero, ecco, proprio come se foste il padre mio carnale. A me stesso non piace il lavoro fatto male, e non permetto che gli altri lo facciano male. Se va bene per il padrone, va bene anche per noi. Se dai un'occhiata laggiù - disse Vasilij, mostrando il campo - ti si rallegra il cuore. 

- Che bella primavera, Vasilij! 

- È una primavera che i vecchi non ricordano più bella. Io, ecco, sono stato a casa mia; anche là da noi il vecchietto ha seminato tre stai di frumento. Dice che non lo si distingue dalla segala. 

- E voi, è un pezzo che avete preso a seminare il frumento? 

- Ma se siete stato voi a insegnarcelo l'anno scorso! E me ne avete regalate pure due misure. Un quarto l'abbiamo venduto e tre stai l'abbiamo seminati. 

- Su, guarda, sfarina le pallottole - disse Levin, avvicinandosi al cavallo - e da' un occhio a Miška. E se verrà su bene, ti darò cinquanta copeche per desjatina. 

- Ringrazio umilmente! Noi, mi pare, anche così siamo molto contenti di voi. 

Levin montò a cavallo e andò nel campo dove c'era il trifoglio dell'anno precedente, e in quello arato, pronto per il grano marzuolo. 

Il trifoglio da stoppia veniva su magnificamente. S'era già tutto avvivato e verzicava dietro gli steli del frumento dell'anno prima. Il cavallo vi affondava fino al ginocchio e ogni sua zampata provocava uno scroscio quando si liberava dalla terra mezzo disgelata. Per i solchi arati non si poteva proprio passare; solo dove c'era un po' di ghiaccio il terreno sosteneva, ma nei solchi disgelati la zampa affondava fino a sopra il ginocchio. Ottima l'aratura; fra due giorni si sarebbe potuto erpicare e seminare. Tutto era bello, tutto era festoso. Levin decise di tornare indietro attraverso il ruscello, sperando che l'acqua vi fosse più bassa. E in effetti lo passò a guado, spaventando due anitre. ``Ci devono essere anche le beccacce'' pensò, e, proprio alla svolta per tornare a casa, incontrò il guardaboschi che lo confermò nella sua supposizione. 

Levin tornò a casa al trotto, per fare in tempo a mangiare e a preparare il fucile per la sera. 

\capitolo{XIV}\label{xiv-1} 

Mentre nella migliore disposizione d'animo si avvicinava a casa, Levin sentì un tinnir di sonagli dalla parte principale dell'ingresso della casa. 

``Ma è qualcuno che viene dalla stazione - pensò - è proprio l'ora del treno di Mosca\ldots{} Chi può essere? Che sia Nikolaj? L'ha detto del resto: `Può darsi che vada a fare la cura delle acque, ma chi sa che non venga da te'\,''. Sulle prime provò sgomento e rammarico al pensiero che la presenza del fratello Nikolaj non avesse a turbare quella sua felice disposizione d'animo. Ma poi si vergognò di questo suo sentimento, e subito gli aprì, per così dire, spiritualmente le braccia, e con gioia intenerita s'aspettò e desiderò con tutta l'anima che fosse il fratello. Stimolò il cavallo e, oltrepassata l'acacia, vide la trojka postale della stazione ferroviaria e un signore in pelliccia. Non era il fratello. ``Ah, se fosse qualche persona simpatica con la quale poter parlare!'' pensò. 

- Ah - gridò con gioia Levin, alzando tutte e due le braccia. - Ecco un ospite gradito! Ah, come sono felice di vederti! - gridò, riconoscendo Stepan Arkad'ic. 

``Così probabilmente saprò se si è sposata o quando si sposerà'' pensò. 

E in quella magnifica giornata di primavera, sentì che il ricordo di lei non gli faceva più alcun male. 

- Forse non m'aspettavi? - disse Stepan Arkad'ic, uscendo dalla slitta con vari schizzi di fango alla radice del naso, sulla guancia e sul sopracciglio, ma splendente di buonumore e di salute. - Sono venuto, prima di tutto, per vederti - disse, abbracciandolo e baciandolo; - poi, per fermarmi un po' per la caccia, ed infine anche per vendere il bosco di Ergušovo. 

- Benone! Ma che primavera! com'è che sei arrivato fin qui in slitta? 

- In carrozza è anche peggio, Konstantin Dmitric - rispose il postiglione che lo conosceva. 

- Be', sono molto contento di vederti - disse Levin, sorridendo sinceramente di un riso infantile e festoso. 

Levin guidò l'ospite nella camera dei forestieri, dove appunto erano state portate le cose di Stepan Arkad'ic: un sacco, un fucile nel fodero, una borsa per i sigari; e, lasciatolo a lavarsi e a cambiarsi, passò nel frattempo in amministrazione a dare gli ordini per l'aratura e per il trifoglio. Agaf'ja Michajlovna, sempre molto preoccupata del prestigio della casa, gli venne incontro in anticamera con alcune domande intorno al pranzo. 

- Fate come volete, purché al più presto - disse lui, e andò dall'amministratore. 

Quando tornò, Stepan Arkad'ic, lavato, pettinato e raggiante, usciva dalla sua camera, e insieme salirono. 

- Ma come son contento d'essere arrivato fin qui da te! Ora capirò in che cosa consistono i prodigi che tu compi qua! Ma, davvero, ti invidio. Che casa, come tutto è eccellente! - disse Stepan Arkad'ic, dimenticando che non sempre c'erano la primavera e le giornate chiare come quella. - E la tua governante che delizia! Forse sarebbe più desiderabile una graziosa cameriera in grembiulino, ma per il tuo cenobitismo e la tua austerità questo va proprio bene. 

Stepan Arkad'ic raccontò molte cose interessanti e gli diede la notizia, che riguardava in particolare Levin, che il fratello Sergej Ivanovic si preparava ad andare da lui in campagna per l'estate. 

Stepan Arkad'ic non disse neppure una parola di Kitty, né in generale degli Šcerbackij; riferì solo i saluti di sua moglie. Levin gli fu grato di questa delicatezza e fu molto contento dell'ospite. In genere, nel periodo del suo isolamento, gli si accumulavano un'infinità di pensieri e di sentimenti che non poteva comunicare a quelli che lo circondavano, e invece ora egli poteva riversare in Stepan Arkad'ic la gioia poetica della primavera, le vicende e i progetti per l'azienda, le idee e le osservazioni sui libri che aveva letto, e in particolare lo schema della sua opera che aveva a base, sebbene egli stesso non lo notasse, la critica di tutte le vecchie opere di economia. Stepan Arkad'ic, sempre simpatico, che afferrava tutto da un accenno, fu particolarmente cordiale in questo suo soggiorno, e Levin notò anche un nuovo tratto di considerazione e quasi di tenerezza verso di lui, che lo lusingò. 

Gli sforzi di Agaf'ja Michajlovna e del cuoco perché il pranzo fosse in tutto e per tutto ben fatto, produssero l'effetto che i due amici, affamati com'erano, seduti davanti all'antipasto, si rimpinzassero di pane e di burro, di uccelletti e di funghi sotto sale; inoltre, che Levin finisse con l'ordinare di servir la minestra senza gli sfogliantini con i quali il cuoco avrebbe voluto in particolar modo stupire l'ospite. Ma Stepan Arkad'ic, pur abituato a pranzi d'altro genere, trovava tutto eccellente; la salsa verde e il pane e il burro, gli uccelletti e i funghi, la minestra d'ortiche e la gallina in salsa bianca e il vino bianco di Crimea, tutto per lui era straordinario ed eccellente. 

- Ottimo, ottimo - diceva accendendo una grossa sigaretta dopo l'arrosto. - Sono arrivato da te proprio come chi, uscendo dal frastuono e dal rollio di un piroscafo, giunga ad una spiaggia silenziosa. Così, allora, tu dici che anche l'elemento ``lavoratore'' dev'essere preso in considerazione e deve guidare nella scelta dei sistemi economici. Io, già, in questo sono un profano; ma mi sembra che la tua teoria e la sua applicazione potranno incidere sul lavoratore. 

- Sì, ma aspetta: io non parlo di economia politica, parlo di scienza agraria. Questa, come le scienze naturali, deve prendere in esame i fenomeni dati e il lavoratore con le sue caratteristiche economiche, etnografiche\ldots{} 

In quel momento entrò Agaf'ja Michajlovna con la marmellata. 

- Ehi, Agaf'ja Michajlovna - le disse Stepan Arkad'ic baciandosi la punta delle dita grassocce - che uccelletti che avete!\ldots{} E che, non è ora Kostja? - aggiunse. 

Levin guardò dalla finestra il sole che scendeva dietro le cime del bosco che si riuscivano a scorgere. 

- È ora, è ora - disse. - Kuz'ma, fa' attaccare il calesse - e corse giù. 

Stepan Arkad'ic, disceso, tolse egli stesso con cura la fodera di tela dall'astuccio verniciato e, apertolo, cominciò a montare il suo costoso fucile di nuovo modello. Kuz'ma, che fiutava fin d'ora una grossa mancia, non si allontanava da Stepan Arkad'ic, e gli infilava calze e stivali, mentre Stepan Arkad'ic lasciava fare volentieri. 

- Ti prego, Kostja, se viene Rjabinin il compratore, gli ho detto di venire quest'oggi, fallo ricevere, e che mi aspetti\ldots{} 

- Ma forse vendi il bosco a Rjabinin? 

- Sì, lo conosci, per caso? 

- Altro se lo conosco. Ho avuto un affare con lui ``positivamente e definitivamente''. 

Stepan Arkad'ic rise. ``Definitivamente e positivamente'' erano gli intercalari del compratore. 

- Già, parla in modo proprio buffo. Ha capito dove va il padrone! - aggiunse, tastando con la mano Laska che gironzolava intorno a Levin mugolando e leccandogli ora una mano, ora gli stivali, ora il fucile. 

La vettura era già accanto alla scalinata, quando uscirono. 

- Ho ordinato di attaccare, sebbene non sia lontano: vogliamo andare a piedi? 

- No, meglio in carrozza - disse Stepan Arkad'ic, accostandosi alla vettura. Sedette, avvoltolò le gambe in uno scialle tigrato, e accese un sigaro. - Com'è che non fumi? Il sigaro non è proprio un godimento, ma il coronamento, il segno del godimento. Ecco, questa è vita! Come si sta bene! Ecco come vorrei vivere io! 

- E che cosa te lo impedisce? - disse Levin, sorridendo. 

- No, tu sei un uomo felice. Hai tutto quello che ti piace. Ti piacciono i cavalli\ldots{} ne hai, i cani\ldots{} ne hai, la caccia\ldots{} ce l'hai, l'azienda\ldots{} ce l'hai. 

- Forse perché mi contento di quello che ho, e non rimpiango quello che non ho - disse Levin, ricordandosi di Kitty. 

Stepan Arkad'ic capì, lo guardò, ma non disse nulla. 

Levin era grato a Oblonskij di aver notato, con il suo tatto abituale, ch'egli temeva il discorso sugli Šcerbackij, e di non avere detto nulla di loro; ora però Levin cominciava a desiderare di sapere quello che lo tormentava; ma non osava avviare il discorso. 

- Be', i tuoi affari come vanno? - disse Levin, dopo aver pensato che fosse poco gentile da parte sua pensare solo a se stesso. 

Gli occhi di Stepan Arkad'ic brillarono allegramente. 

- Tu, è vero, non ammetti che possano piacere le ciambelle, quando si ha la razione assegnata; questo per te è un delitto; ma io non so comprendere la vita senza amore - disse, interpretando a modo suo la domanda di Levin. - Che farci, son fatto così. E invero, con questo si fa tanto poco male a qualcuno e tanto piacere a se stesso. 

- Be', c'è forse qualcosa di nuovo? - disse Levin. 

- C'è, amico mio! Ecco, vedi: conosci il tipo delle donne ossianesche\ldots{} delle donne che vedi in sogno\ldots{} Queste donne vivono nella realtà\ldots{} e queste donne sono fatali. La donna, vedi, per quanto tu la studi, è un soggetto sempre nuovo. 

- Allora è meglio non studiarlo. 

- No, un matematico ha detto che la gioia non consiste nella scoperta della verità, ma nella ricerca di essa. 

Levin ascoltava in silenzio e, pur facendo tutti gli sforzi su se stesso, non riusciva in nessun modo a trasferirsi nell'animo dell'amico, non riusciva a capire i suoi sentimenti e il piacere ch'egli provava nello studio di donne siffatte. 

\capitolo{XV}\label{xv-1} 

Il passo non era lontano, al di sopra del fiume, in un boschetto di tremule. Giunti al bosco, Levin accompagnò Oblonskij all'angolo di una radura coperta di musco e di fango, già sgombra di neve. Egli stesso tornò indietro, all'altro estremo, verso una betulla doppia, e, appoggiato il fucile alla biforcazione del ramo inferiore secco, si tolse il pastrano, si mise la cintura e provò la scioltezza dei movimenti delle braccia. 

Lanka che gli andava dietro passo passo, grigia e vecchiotta, s'accucciò guardinga di fronte a lui, e tese le orecchie. Il sole scendeva dietro al bosco grande, e nella luce del tramonto le giovani betulle sparse fra le tremule si disegnavano nette coi loro rami pendenti dalle gemme gonfie, pronte a scoppiare. 

Dal bosco fitto, dove era rimasta ancora neve, scorreva appena percettibile l'acqua in rigagnoli stretti e tortuosi. Uccelli piccoli cinguettavano e di tanto in tanto frullavano da un albero all'altro. 

Negli intervalli di calma completa, si poteva udire il crepitar delle foglie dell'anno prima, smosse dallo sgelo della terra e dal germinare delle erbe. 

``Che meraviglia! Si sente e si vede come cresce l'erba!'' si disse Levin, notando una foglia bagnata di tremula color lavagna che si moveva accanto a un filo d'erba nuova. Egli stava in piedi, in ascolto, e guardava ora la terra umida muscosa, ora Laska tutt'orecchi, ora il mare delle cime spoglie degli alberi che si stendeva dinanzi a lui ai piedi della montagna, ora il cielo che scolorava velato da strati bianchi di nuvole. Un falco, battendo le ali lentamente, volò alto sul bosco lontano; un secondo, con moto eguale, volò nella stessa direzione e scomparve. Gli uccelli presero a cinguettare ancor più chiassosi e insistenti nel fitto del bosco. Non lontano urlò un gufo, e Laska, rabbrividendo, fece alcuni passi accorti e, piegata la testa da un lato, si mise in ascolto. Di là dal fiume si udì il cuculo. Per due volte lanciò il solito verso, poi s'arrochì, abborracciò, barbugliò. 

- Che bellezza! di già il cuculo! - disse Stepan Arkad'ic uscendo di dietro a un cespuglio. 

- Già, ho sentito - rispose Levin, rammaricandosi di rompere il silenzio del bosco con la propria voce, sgradita a lui stesso. - Ecco, arrivano! 

La figura di Stepan Arkad'ic passò di nuovo dietro al cespuglio e Levin vide solo la fiammella viva di un fiammifero seguìta subito dopo dal fuoco rosso della sigaretta e da un piccolo fumo turchino. 

Cik! cik!, scattarono i cani del fucile alzati da Stepan Arkad'ic. 

- Che cos'è che stride? - domandò Oblonskij, attirando l'attenzione di Levin su di uno stridio prolungato, come di un puledro che, ruzzando, nitrisse con voce acuta. 

- Ah, non sai? È una lepre, un maschio. Ma stiamo zitti! Senti?\ldots{} passano! - gridò quasi Levin, alzando i cani del fucile. 

Si udì un fischio lontano e, proprio all'intervallo regolare di due secondi così noto al cacciatore, un secondo, un terzo fischio e, dopo il terzo, lo zirlio era già percettibile. 

Levin girò gli occhi a destra e a sinistra, ed ecco, dinanzi a lui, nel cielo azzurro cupo, al di sopra dei germogli teneri e gonfi delle tremule, apparve l'uccello in volo. Volava diritto verso di lui: lo zirlio ormai vicino, simile allo squarciarsi a intervalli regolari di una grossa tela, gli risonò proprio sopra l'orecchio; si scorgeva già il becco lungo e il collo dell'uccello, ma nel momento in cui Levin prendeva la mira, di dietro al cespuglio dov'era Oblonskij, guizzò un lampo rosso; l'uccello, come una freccia, s'abbassò e salì di nuovo in alto. Guizzò un altro lampo e si udì un colpo, e sbattendo le ali, quasi cercando di reggersi nell'aria, l'uccello si fermò, rimase un attimo sospeso e precipitò pesantemente sul terreno fangoso. 

- Possibile che abbia fatto padella? - gridò Stepan Arkad'ic che non riusciva a vederci per il fumo. 

- Eccola! - disse Levin, indicando Laska che, con un orecchio alzato e agitando la punta della coda lanosa, a passi lenti, come se sorridesse e volesse prolungarsene il piacere, portava l'uccello ucciso al padrone. - Via, son contento che sia riuscito a te - disse Levin, pur provando un certo senso di invidia a non essere stato lui ad ammazzar la beccaccia. 

- Una brutta padella dalla canna destra - rispose Stepan Arkad'ic, ricaricando il fucile. - Sst\ldots{} passano\ldots{} 

Si udivano infatti fischi acuti susseguirsi l'uno all'altro, rapidi. Due beccacce, giocando a rincorrersi e fischiando solo, senza zirlare, volarono sopra le teste dei cacciatori. Risonarono quattro colpi, ma le beccacce, quasi rondini, compirono una voluta rapida e scomparvero dalla vista. 

\begin{center}\rule{3in}{0.4pt}\end{center} 

Il passo fu ottimo. Stepan Arkad'ic uccise due uccelli e Levin due, di cui uno non si trovò. Cominciava a imbrunire. In basso, al di là delle betulle, Venere con la sua luce tenue splendeva chiara d'argento; mentre in alto, a levante, il corrusco Arturo spandeva già la sua luce rossastra. Proprio sopra il suo capo, Levin ora scorgeva, ora smarriva le stelle dell'Orsa. Le beccacce avevano già cessato il volo; ma Levin decise di aspettare che Venere, ch'egli vedeva al di sotto di un piccolo ramo di betulla, passasse al di sopra, e che le stelle dell'Orsa apparissero chiare in ogni punto. Ma Venere aveva già oltrepassato il ramo, il carro dell'Orsa col suo timone era già tutto chiaro nel cielo azzurro fondo, e Levin aspettava ancora. 

- Non è ora? - chiese Stepan Arkad'ic. 

Nel bosco c'era già quiete e neppure il più piccolo uccello si moveva. 

- Restiamo ancora - rispose Levin. 

- Come vuoi. 

Adesso stavano in piedi, a quindici passi l'uno dall'altro. 

- Stiva! - disse a un tratto, inaspettatamente, Levin - come mai non mi dici se tua cognata s'è sposata o sta per sposarsi? 

Si sentiva così sicuro e sereno da ritenere che nessuna risposta potesse turbarlo. Ma proprio non si aspettava quello che rispose Stepan Arkad'ic. 

- Non ci ha pensato e neppure ci pensa a sposarsi; ma è molto malata, e i medici l'hanno mandata all'estero. Si teme persino per la sua vita. 

- Ma che dici? - gridò Levin. - Molto malata? E cosa mai le è accaduto? Come è\ldots{} 

Mentre dicevano questo, Laska, drizzando le orecchie, guardò in alto, verso il cielo, e poi verso di loro con aria di rampogna. ``Ecco, hanno scelto proprio il momento buono per chiacchierare\ldots{} e lei intanto se ne vola\ldots{} Eccola, è proprio così. Se la lasceranno scappare\ldots{}'' pensava Laska. 

Ma in quello stesso momento tutti e due sentirono a un tratto un fischio penetrante frustar loro l'orecchio, e tutti e due imbracciarono il fucile e due colpi risonarono nello stesso istante. La beccaccia, che volava in alto, piegò le ali e cadde nel fitto di un cespuglio curvandone i germogli sottili. 

- Ecco, perfetto! Insieme! - gridò Levin e corse con Laska nel cespuglio a cercare la beccaccia. ``Ah, sì, ma cos'è che m'ha fatto dispiacere? - andava ricordando. - Già, Kitty, che è malata. Ma non c'è nulla da fare; è un gran peccato'' pensava. 

- Ah, l'hai trovata. Ecco, l'intelligentona! - disse prendendo dalla bocca di Laska l'uccello ancora caldo e ponendolo nel carniere quasi pieno. - L'ho trovata, Stiva! - gridò. 

\capitolo{XVI}\label{xvi-1} 

Tornando a casa, Levin chiese tutti i particolari della malattia di Kitty e i progetti degli Šcerbackij, e in fondo (se ne vergognava persino nel confessarlo a se stesso) quello che aveva saputo gli faceva piacere. Gli faceva piacere e perché c'era ancora una speranza e ancor più perché soffriva chi aveva fatto soffrire tanto lui. Ma quando Stepan Arkad'ic cominciò a parlare della cause della malattia di Kitty e fece il nome di Vronskij, Levin lo interruppe. 

- Io non ho alcun diritto di sapere i particolari di famiglia, e, a dire il vero, neanche nessun interesse. 

Stepan Arkad'ic sorrise appena percettibilmente, cogliendo il mutamento subitaneo, e a lui così noto, del viso di Levin, divenuto tanto scuro quanto allegro era stato un momento prima. 

- Hai concluso del tutto il taglio del bosco con Rjabinin? - chiese Levin. 

- Sì, ho concluso. Il prezzo è ottimo, trentottomila rubli: otto anticipati e il resto in sei anni. Ho dovuto faticare per averlo. Nessuno mi offriva di più. 

- In conclusione, l'hai regalato il bosco - disse torvo Levin. 

- Come regalato? - disse Stepan Arkad'ic con un sorriso bonario, sapendo che ormai Levin avrebbe trovato tutto mal fatto. 

- Perché quel bosco vale almeno un cinquecento rubli a desjatina - rispose Levin. 

- Ah, eccoli questi proprietari di terre! - disse Stepan Arkad'ic scherzando. - Questo vostro tono di disprezzo verso noi cittadini!\ldots{} Intanto, quando c'è da concludere un affare, siamo noi a far meglio. Credimi, ho calcolato tutto - disse - e il bosco è stato venduto a condizioni molto vantaggiose: temo persino che egli rifiuti. Certo non è un bosco conveniente - disse Stepan Arkad'ic desiderando con la parola ``conveniente'' convincere Levin dell'infondatezza dei suoi dubbi - più che altro è legna da ardere. E ce ne saranno non più di trenta sazeni per desjatina e lui me ne dà duecento rubli. 

Levin sorrise sprezzante. ``Conosco - pensò - questo modo di fare, non solo suo, ma di tutti gli abitanti di città; vengono in campagna due volte in dieci anni, annotano due o tre termini campagnoli, e li usano a proposito e a sproposito, fermamente convinti di sapere tutto. `Conveniente, ce ne saranno trenta sazeni'. Ripete delle parole, ma non ne conosce il senso''. 

- Io non starò a insegnarti quel che scrivi là al tuo ufficio - disse - ma se fosse necessario chiederei di apprendere da te. E tu, invece, sei così sicuro di capire tutto in materia di legname. È difficile. Hai dato una contata agli alberi? 

- Come, la contata degli alberi? - disse, ridendo, Stepan Arkad'ic, desiderando sempre di far uscire Levin dal suo cattivo umore. - ``Contar le sabbie, i raggi dei pianeti, potrebbe ancora un alto ingegno\ldots{}''. 

- Eh, già, ma intanto l'alto ingegno di Rjabinin, sì, che lo può. E nessun compratore compra un taglio di bosco senza contare, a meno che non glielo regalino, così come hai fatto tu. Conosco il tuo bosco. Ci vado ogni anno a caccia e vale cinquecento rubli contanti a desjatina; mentre lui te ne dà duecento a rate. Il che significa che tu, a lui, ne regali trentamila. 

- Su, via, non esageriamo - disse con pena Stepan Arkad'ic - e allora perché nessuno me li offriva? 

- Perché lui è d'accordo con gli altri. Ho avuto a che fare con tutti loro, li conosco. Non sono dei compratori, ma degli accaparratori. Se non c'è da guadagnare il dieci, il quindici per cento egli non avvia neppure l'affare; aspetta a comprare un rublo con venti copeche. 

- Lascia andare! Tu vedi tutto nero. 

- Niente affatto - disse cupo Levin, mentre si avvicinavano a casa. 

All'ingresso c'era già una carretta tutta ricoperta di ferro e cuoio, con un cavallo ben pasciuto attaccato con corregge larghe ben tese. Nella carretta sedeva un inserviente che faceva da cocchiere a Rjabinin, fortemente stretto da una cintura, con una faccia turgida e iniettata di sangue. Lo stesso Rjabinin era già in casa, e venne incontro agli amici nell'anticamera. Era un uomo di mezza età, alto e rinsecchito, con i baffi, il mento raso sporgente e gli occhi torbidi all'infuori. Vestiva un soprabito turchino a lunghe falde, con i bottoni più in basso del dorso, e sopra agli alti stivali raggrinziti alle caviglie e tirati sui polpacci portava delle grosse calosce. Si asciugò tutto il viso in giro col fazzoletto e, allacciatosi il soprabito, che anche senza di questo chiudeva bene, salutò con un sorriso quelli che erano entrati, tendendo la mano a Stepan Arkad'ic come se volesse afferrare qualcosa. 

- Ah, siete arrivato - disse Stepan Arkad'ic, dandogli la mano. - Benone. 

- Non ho osato mancare a un ordine di vostra eccellenza, benché la strada fosse cattiva. Ho fatto tutta la strada positivamente a piedi, ma sono arrivato in tempo. Konstantin Dmitric, i miei rispetti - disse rivolto a Levin, cercando di afferrare anche a lui la mano. Ma Levin, accigliato, fingeva di non vedere e tirava fuori le beccacce. - I signori si sono divertiti a caccia? Ma che uccelli son codesti? - aggiunse Rjabinin guardando sprezzante le beccacce. - Be', un certo sapore lo avranno! - E scosse il capo, disapprovando e dubitando assai che il giuoco valesse la candela. 

- Vuoi andar nello studio? - disse Levin in francese, sempre scuro in viso, a Stepan Arkad'ic. - Accomodatevi nello studio, parlerete là. 

- Ma dove a voi piace, signore - disse con aria dignitosa e altera Rjabinin per far intendere che per gli altri potevano esserci difficoltà, sul come e con chi trattare l'affare, ma per lui mai e per nessuna cosa. 

Entrando nello studio, Rjabinin, per abitudine, guardò in giro a cercare l'icona, ma, trovatala, non si segnò. Esaminò gli armadi e gli scaffali coi libri, e con lo stesso atteggiamento di diffidenza assunto per le beccacce, sorrise sprezzante e scosse il capo disapprovando, deciso ormai a non ammettere che il giuoco valesse la candela. 

- Be', il denaro lo avete portato? - disse Oblonskij. - Sedete. 

- Noi non ci faremo certo attendere pel denaro. Sono venuto per vedervi, per discorrere un po'. 

- Discorrere di che? Ma sedetevi. 

- Questo sì - disse Rjabinin, sedendosi e appoggiandosi, nel modo più scomodo per lui, alla spalliera della poltrona. - Bisogna che abbassiate un po' il prezzo, principe. Sarebbe un peccato mandare a monte. E i denari son pronti, prontissimi. Fino all'ultima copeca. Pel pagamento non ci sarà ritardo. 

Levin, che, nel frattempo, aveva riposto il fucile nell'armadio e già stava per uscire, udite le parole del compratore, si fermò sulla porta. 

- E così, voi avete preso il bosco per niente - disse. - È giunto tardi da me, altrimenti il prezzo l'avrei fatto io. 

Rjabinin si alzò e guardò Levin in silenzio con un sorriso di sotto in su. 

- Siete molto attaccato al denaro, Konstantin Dmitric - disse, volgendosi a Stepan Arkad'ic con un sorriso. - Decisamente da lui non ci si può comprar nulla. Stavo trattando per il frumento, offrivo dei bei soldi, io. 

- Perché dovrei regalarvi il mio? Non l'ho mica trovato per terra, né rubato. 

- Vi prego. Al giorno d'oggi rubare è definitivamente impossibile. Tutto, al giorno d'oggi è definitivamente di dominio pubblico, oggi tutto è onesto; altro che rubare! Ma via, trattiamo onestamente. È caro questo legname, non ci esco neanche con le spese. Chiedo di cedere almeno di una piccolezza. 

- Ma voi l'affare lo avete concluso sì o no? Se è concluso non c'è più nulla da trattare, se non è concluso - disse Levin - il bosco lo compro io. 

Il sorriso scomparve a un tratto dal viso di Rjabinin. Una espressione di sparviero, rapace e dura, vi si fissò. Con le dita ossute e rapide sbottonò il soprabito, scoprendo la camicia che ne usciva fuori, i bottoni di rame del panciotto e la catena dell'orologio, e in fretta cavò fuori un grosso e vecchio portafoglio. 

- Vi prego, il bosco è mio - pronunciò, dopo essersi fatto in fretta il segno della croce e tendendo la mano. - Ecco il denaro, il bosco è mio. Ecco come fa gli affari Rjabinin, e non bada agli spiccioli - disse accigliato, agitando il portafoglio. 

- Io al posto tuo non avrei fretta - disse Levin. 

- Ti prego - disse sorpreso Oblonskij. - Ho dato la parola. 

Levin uscì dalla stanza, sbattendo la porta. Rjabinin, guardando verso questa, scosse la testa con un sorriso. 

- Gioventù, definitivamente; anzi fanciullaggine. Compro per tanto, credetemi sull'onore, solo per potermi vantare, ecco, che Rjabinin, e nessun altro, ha comprato il bosco da un Oblonskij. E anche, se Dio vuole, per guadagnarci su. Credete a Dio. Vi prego, signore. Scriviamo il contrattino. 

Un'ora dopo il compratore, incrociatasi accuratamente la veste e agganciati gli uncini del soprabito, col contratto in tasca, sedette nella sua carretta ben ferrata per tornarsene a casa. 

- Oh, questi signori! - disse all'inserviente - tutti a un modo! 

- Proprio così - rispose l'inserviente, dandogli le briglie e abbottonando il grembiule di cuoio. - E il vostro affaruccio com'è andato, Michail Ignat'ic? 

- Be', be'\ldots{} 

\capitolo{XVII}\label{xvii-1} 

Stepan Arkad'ic salì con la tasca piena di titoli che gli aveva dato il compratore come rata di tre mesi. L'affare del bosco era concluso, il denaro era in tasca, la caccia era stata magnifica, e Stepan Arkad'ic si trovava nella più amena disposizione d'animo; voleva perciò in particolar modo disperdere il cattivo umore che era piombato su Levin. Aveva voglia di chiudere piacevolmente con la cena la giornata che piacevolmente era cominciata. 

Levin era davvero di cattivo umore e, malgrado il desiderio suo di mostrarsi affettuoso, cordiale con l'ospite simpatico, non riusciva a dominarsi. Lo stordimento della notizia che Kitty non s'era sposata cominciava a prenderlo a poco a poco 

Kitty tuttora nubile e malata, malata d'amore per l'uomo che l'aveva disdegnata! Quest'offesa gli pareva ricadesse su di lui. Vronskij aveva disdegnato lei, e lei aveva respinto lui. Per conseguenza, Vronskij aveva il diritto di disprezzarlo, e perciò Levin sentiva per lui dell'avversione. Ma Levin non aveva chiaro nella mente tutto ciò. Sentiva solo in maniera confusa che in questo c'era qualcosa d'offensivo per lui e se ne irritava: non proprio per quello che l'aveva sconvolto, ma per ogni cosa che intanto gli accadeva. La vendita insensata del bosco, l'inganno in cui era caduto Oblonskij, e che si era perpetrato in casa sua, lo irritavano. 

- Be', hai finito? - disse, incontrando di sopra Stepan Arkad'ic. - Vuoi cenare? 

- Certo, non mi rifiuterò. Che appetito m'è venuto in campagna, un prodigio! Perché mai non hai offerto da mangiare a Rjabinin? 

- Che il diavolo se lo pigli! 

- Ma come lo maltratti! - disse Oblonskij. - Non gli hai dato neppure la mano. Perché non gli dai la mano? 

- Perché io non do la mano a un lacchè, e un lacchè è cento volte migliore di lui. 

- Ma che retrogrado che sei! E la fusione delle classi? - disse Oblonskij. 

- Buon pro' gli faccia a chi la vuole la fusione; a me non va. 

- Vedo che sei decisamente retrogrado. 

- Davvero non ho mai pensato a quel che sono. Io sono Konstantin Levin e nient'altro. 

- Un Konstantin Levin che è di pessimo umore! - disse, sorridendo, Stepan Arkad'ic. 

- Sì, sono di cattivo umore, e tu lo sai perché; per questa tua stupida, scusami, vendita. 

Stepan Arkad'ic corrugò bonariamente le sopracciglia come un uomo che viene offeso e mortificato senza ragione. 

- Su, basta - disse. - Quando mai s'è visto che un uomo vende qualcosa e non gli dicano subito, dopo la vendita: ``questo vale molto di più''? E finché la cosa è in vendita nessuno offre\ldots{} No, vedo che tu ce l'hai proprio con quel disgraziato di Rjabinin. 

- Può anche darsi. Ma sai perché? Tu dirai di nuovo che io sono un retrogrado o qualche altra strana cosa; tuttavia ti dico che mi spiace e mi offende assistere a questo impoverimento che d'ogni lato si va compiendo della nobiltà alla quale appartengo e alla quale, malgrado la fusione delle classi, sono molto lieto di appartenere\ldots{} E questo impoverimento non è oggetto di sperpero che sarebbe cosa di poco conto: lo sperpero da gran signore è affare da signori; solo i signori sanno sperperare così. Adesso i contadini, così come facciamo noi, si accaparrano le terre\ldots{} e nemmeno questo mi offende. Il signore non fa nulla, il contadino lavora e soppianta l'uomo ozioso. Così deve essere. Anzi sono molto contento per il contadino. Ma mi offende assistere a questo impoverimento dovuto a una certa tal quale, non so come chiamarla, dabbenaggine. Qua un affittuario polacco ha comprato a metà prezzo un magnifico podere della padrona che vive a Nizza, là danno in fitto a un mercante per un rublo una desjatina di terra che ne vale dieci. Qua tu, senza nessuna ragione al mondo, dài in regalo trentamila rubli a questo furfante. 

- E allora? Bisognava contare ogni albero? 

- Contare, sì, assolutamente. Ed ecco, tu non hai contato, ma Rjabinin ha contato. I figli di Rjabinin avranno i mezzi per vivere ed educarsi, e i tuoi, perdonami, non ne avranno. 

- Su, via, scusami, ma c'è qualcosa di meschino in questo contare. Noi abbiamo le nostre occupazioni, loro le loro, ed essi hanno bisogno di grossi profitti. Via, del resto l'affare è fatto, è concluso. Ma ecco le uova in tegame, le uova che più mi piacciono! E Agaf'ja Michajlovna ci darà quel suo meraviglioso sughetto di erbe\ldots{} 

Stepan Arkad'ic sedette a tavola e cominciò a scherzare con Agaf'ja Michajlovna, assicurando che un pranzo ed una cena simili da lungo tempo non li aveva mangiati. 

- Ecco, voi almeno mi fate un elogio - disse Agaf'ja Michajlovna - ma Konstantin Dmitric, qualunque cosa gli si dia, sia pure una crosta di pane, mangia e se ne va. 

Per quanto Levin cercasse di dominarsi, era cupo e silenzioso. Aveva gran voglia di fare una domanda a Stepan Arkad'ic, e non riusciva a decidersi e non trovava né il modo, né il momento di farla. 

Stepan Arkad'ic era già disceso in camera sua, s'era spogliato, lavato di nuovo, s'era messo in dosso una camicia da notte pieghettata, s'era coricato, e Levin era sempre con lui in camera, parlando di varie sciocchezze, senza avere il coraggio di chiedere quel che voleva sapere. 

- Ma in che modo meraviglioso fanno il sapone! - diceva, scartocciando e guardando un pezzo di sapone profumato che Agaf'ja Michajlovna aveva preparato per l'ospite, e che Oblonskij non aveva usato. - Guarda, è proprio un'opera d'arte. 

- Già, adesso si è raggiunto in tutto la perfezione - disse Stepan Arkad'ic, sbadigliando molle e beato. - I teatri per esempio e quei luoghi di divertimento\ldots{} ah ah ah! - sbadigliava. - La luce elettrica dovunque\ldots{} ah ah\ldots{} 

- Già, la luce elettrica - disse Levin. - Già\ldots{} e dov'è Vronskij adesso? - chiese dopo aver posato a un tratto il sapone. 

- Vronskij - disse Stepan Arkad'ic, arrestando uno sbadiglio - è a Pietroburgo. È andato via poco dopo di te, e poi non è venuto neppure una volta a Mosca. E sai, Kostja, ti dirò la verità - continuò, appoggiandosi col gomito sul tavolo e posando sulla mano il bel viso arrossato sul quale brillavano come stelle i languidi, buoni occhi assonnati. - La colpa è proprio tua. Tu hai avuto paura del rivale. E io, come ti avevo detto anche allora, so da parte di chi c'erano maggiori probabilità. Perché non ti sei fatto avanti? Io ti avevo detto allora che\ldots{} - sbadigliò con le sole mascelle senza aprire la bocca. 

``Lo sa o non lo sa che io ho fatto la mia domanda di matrimonio? - pensò Levin, guardandolo. - Già; c'è qualcosa di accorto, di diplomatico nel suo viso'' e, sentendo di arrossire, guardò in silenzio, diritto negli occhi, Stepan Arkad'ic. 

- Se allora da parte di lei c'era qualcosa, era un'attrazione per l'esteriorità - continuò Oblonskij. - Sai, quel perfetto aristocraticismo e la futura posizione nella società hanno agito, non su di lei, ma sulla madre. 

Levin si accigliò. L'affronto del rifiuto, attraverso il quale era passato, gli bruciava nel cuore come una ferita fresca, appena ricevuta. Ma era a casa sua e le mura della casa aiutano. 

- Aspetta, aspetta - prese a dire, interrompendo Oblonskij. - Tu dici: aristocraticismo. Ma permettimi di chiederti in che cosa consiste questo aristocraticismo di Vronskij, o di chiunque altro sia; questo tale aristocraticismo per il quale si possa disdegnare me. Tu consideri Vronskij un aristocratico, ma io no. Un uomo il cui padre è venuto su dal nulla con l'intrigo, la cui madre Dio sa con chi non ha avuto legami\ldots{} No, scusami, io considero aristocratico me stesso e le persone simili a me che possono vantare tre o quattro oneste generazioni di famiglie che hanno conseguito il più alto grado di cultura (il talento e l'ingegno sono un'altra cosa), che non si sono mai umiliate dinanzi a nessuno, che non hanno mai avuto bisogno di nessuno; così come hanno vissuto mio padre, mio nonno. E io ne conosco molti fatti così. A te sembra cosa meschina che io conti gli alberi nel bosco, mentre tu fai regalo di trentamila rubli a Rjabinin; ma tu riceverai un'indennità o non so cos'altro ancora, mentre io non la ricevo, e perciò mi tengo caro quello che m'han lasciato i miei e il frutto delle mie fatiche. Noi siamo i veri aristocratici, e non quelli che possono viver solo delle elargizioni dei potenti di questo mondo o quelli che si possono comprare con venti copeche. 

- Ma con chi te la prendi? Io sono d'accordo con te - disse Stepan Arkad'ic con sincerità e allegria, sebbene sentisse che Levin, accennando a quelli che si possono comprare con venti copeche, si riferisse anche a lui. L'eccitazione di Levin gli piaceva davvero. - Con chi ce l'hai? Benché gran parte di quel che dici di Vronskij non sia vero, io non mi riferivo a questo. Io ti dico francamente, al posto tuo\ldots{} insomma dovresti venire con me a Mosca e \ldots{} 

- No, io non so se tu sia al corrente o no, ma per me è lo stesso. E te lo dico subito: io ho fatto la mia domanda di matrimonio e ho ricevuto un rifiuto; e per me in questo momento, Katerina Aleksandrovna è un ricordo umiliante e molesto. 

- E perché? Ma guarda che sciocchezza! 

- Non ne parliamo più. Scusami, ti prego, se sono stato villano con te. - disse Levin. Ora, dopo aver messo fuori tutto, era diventato di nuovo quello che era stato la mattina. - Non sei mica in collera con me, Stiva? Ti prego, non ti arrabbiare - disse e, sorridendo, gli prese la mano. 

- Ma no, per nulla affatto, e non ce ne sarebbe ragione! Sono contento che ci siamo spiegati. E sai, la caccia del mattino di solito è fruttuosa. Non ci si potrebbe andare? Così non dormirei neppure, e dal posto di caccia andrei di filato alla stazione. 

- Benissimo! 

\capitolo{XVIII}\label{xviii-1} 

Sebbene la vita intima di Vronskij fosse tutta piena della sua passione, la sua vita esteriore si svolgeva immutata e immutabile sull'abituale carreggiata di prima, tra i rapporti e gli interessi del gran mondo e del reggimento. Gli interessi del reggimento occupavano un posto importante nella vita di Vronskij, e perché egli amava il reggimento e ancor più perché ne era amato. Al reggimento non solo gli volevano bene, ma lo stimavano ed erano orgogliosi di lui; erano orgogliosi che quest'uomo enormemente ricco, che possedeva una cultura ed aveva splendide attitudini, che aveva un avvenire aperto a ogni genere di successi, di ambizioni e di vanità, disdegnasse tutto questo e prendesse a cuore più di ogni altra cosa gli interessi del reggimento e dei compagni. Vronskij era cosciente di questa sua posizione presso i compagni, e oltre al fatto che amava quella vita, si sentiva impegnato a mantenere l'opinione che si aveva di lui. 

Naturalmente egli non parlava del suo amore con nessuno dei compagni; non si tradiva neppure nelle più grosse sbornie (del resto non era mai così ubriaco da perdere il controllo di se stesso), e tappava la bocca a quei compagni meno prudenti che tentavano di fare delle allusioni al suo legame. Malgrado ciò, il suo amore era noto a tutta la città: tutti indovinavano più o meno esattamente i suoi rapporti con la Karenina; la maggioranza dei giovani gli invidiava proprio quello che c'era di più penoso nel suo amore: l'alta posizione di Karenin, e perciò lo scalpore di quel legame nel gran mondo. 

La maggior parte delle giovani donne, gelose di Anna e che da tempo erano annoiate di sentirla definire ``irreprensibile'', si rallegravano di queste supposizioni sul suo conto, e aspettavano solo che l'opinione pubblica mutasse per piombarle addosso con tutto il peso del loro disprezzo. Preparavano già il fango da scagliare su di lei, non appena fosse giunto il momento. La maggior parte delle persone anziane di condizione sociale elevata erano spiacenti di questo scandalo che si preparava in società. 

La madre di Vronskij, conosciuta la relazione di lui, in principio ne era stata contenta perché nulla, secondo lei, dava l'ultima finitura a un giovane brillante quanto una relazione nel gran mondo, e perché la Karenina, che tanto le era piaciuta, che tanto aveva parlato del proprio figlio, aveva finito con l'essere così come, secondo la contessa Vronskaja, doveva essere ogni bella donna del gran mondo. Ma negli ultimi tempi aveva saputo che suo figlio aveva rifiutato l'offerta di un posto importante per la carriera, solo per voler rimanere al reggimento, ed aver così modo di vedere la Karenina; aveva saputo che i superiori erano scontenti di lui per questo rifiuto, e allora aveva cambiato idea. Le dispiaceva inoltre che la relazione, da quanto aveva saputo, non fosse la brillante, graziosa relazione mondana ch'ella aveva approvata, ma una certa passione alla Werther, disperata, come le riferivano, capace di trascinare lui a fare delle sciocchezze. Non vedeva il figlio dal tempo della sua partenza improvvisa da Mosca e, per mezzo del primogenito, aveva preteso che egli venisse da lei. 

Il fratello maggiore anch'egli era scontento del più giovane. Non capiva che specie di amore fosse quello: grande o piccolo, appassionato o non appassionato, vizioso o non vizioso (egli stesso, pur avendo dei figli, manteneva una ballerina ed era indulgente in tale materia), ma sapeva che quest'amore spiaceva a coloro ai quali era necessario piacere, perciò non approvava la condotta del fratello. 

Oltre le occupazioni del servizio e quelle mondane, Vronskij aveva un'altra occupazione: i cavalli, di cui era un appassionato. 

Proprio quell'anno erano state indette le corse a ostacoli per ufficiali. Vronskij vi si era iscritto, aveva comprato un purosangue inglese e, nonostante il suo amore, era tutto preso, anche se con riserbo, dalle corse imminenti. 

Queste due passioni non si contrastavano. Anzi egli aveva bisogno di trovare interesse e svago in qualcosa di diverso dal suo amore, in qualcosa in cui potersi rinnovare e riposare dalle impressioni che lo agitavano troppo. 

\capitolo{XIX}\label{xix-1} 

Il giorno delle corse di Krasnoe Selo, Vronskij andò prima del solito a mangiare una bistecca nella sala grande della mensa degli ufficiali. Non aveva bisogno di osservare una dieta rigorosa per mantenersi in forma; il suo peso era quello stabilito, di quattro pudy e mezzo; ma non doveva ingrassare, perciò evitava i farinacei e i dolciumi. Sedette col soprabito aperto sul panciotto bianco, appoggiando tutte e due le braccia sulla tavola, e, in attesa della bistecca ordinata, guardava in un romanzo francese poggiato sul piatto. Guardava nel libro solo per evitare di parlare con i colleghi che entravano ed uscivano, e pensava. 

Pensava che Anna gli aveva promesso un appuntamento per quel giorno, dopo le corse. Ma non la vedeva da tre giorni e, dopo il ritorno del marito dall'estero, non sapeva se ciò sarebbe stato possibile quel giorno o no, e non sapeva come fare per saperlo. S'era visto con lei l'ultima volta nella villa della cugina Betsy. Alla villa dei Karenin invece egli andava il meno possibile. Ora però voleva andarci e ne cercava il modo. 

``Dirò naturalmente che Betsy mi ha incaricato di chiederle se verrà o no alle corse. Certo che andrò'' decise fra sé, sollevando la testa dal libro. E, raffiguratasi la gioia nel vederla, si illuminò nel viso. 

- Manda a casa mia, perché attacchino al più presto la trojka - disse al cameriere che gli aveva servito la bistecca su un piatto d'argento caldo e, avvicinato il piatto, cominciò a mangiare. 

Dalla sala accanto, dei biliardi, si sentivano colpi di palle, vocìo e risa. Alla porta d'ingresso apparvero due ufficiali: uno giovanissimo con un viso delicato, magro, che da poco era entrato nel reggimento dal corpo dei paggi; l'altro grasso, anziano, con un braccialetto al polso e i piccoli occhi infossati. 

Vronskij li guardò, aggrottò le sopracciglia e, come se non li avesse notati, sbirciando il libro di traverso, prese a mangiare e a leggere insieme. 

- O che, ti metti in forza per il lavoro? - disse l'ufficiale grasso, sedendosi accanto a lui. 

- Lo vedi - rispose Vronskij, accigliandosi e asciugandosi le labbra senza guardarlo. 

- Ma non hai paura d'ingrassare? - disse quello, girando una sedia per l'ufficiale giovane. 

- Cosa? - disse Vronskij irritato, facendo una smorfia di disgusto che mostrò i suoi denti regolari. 

- Non hai paura d'ingrassare? 

- Cameriere, del Xeres - disse Vronskij senza rispondere e, posato il libro dall'altra parte, continuò a leggere. 

L'ufficiale grasso prese la carta dei vini e si voltò verso il giovane. 

- Scegli tu stesso quello che vuoi - disse, dandogli la carta e guardandolo. 

- Prego, del Reno - disse l'ufficiale giovane, guardando timido Vronskij e cercando di lisciare con le dita i baffi incipienti. Visto che Vronskij non gli dava retta, l'ufficiale giovane si alzò. 

- Andiamo al biliardo. 

Il grassone si alzò docile, e si diressero insieme verso la porta. 

In quel momento entrò nella sala il capitano Jašvin, alto e ben fatto, che, con un cenno sprezzante all'insù del capo verso i due ufficiali, s'accostò a Vronskij. 

- Ah, eccolo! - gridò, battendogli con forza sulla spallina con la mano grande. Vronskij si voltò a guardare con stizza, ma subito il viso gli si illuminò della cordialità calma e decisa che gli era propria. 

- Hai agito con intelligenza, Alëša - disse il capitano con una forte voce baritonale. - Adesso mangia e bevi un bicchierino. 

- Ma non ne ho voglia. 

- Ecco gl'inseparabili - aggiunse Jašvin, guardando con aria canzonatoria i due ufficiali che in quel momento uscivano dalla sala. E sedette accanto a Vronskij, piegando ad angolo acuto i suoi femori troppo lunghi per l'altezza delle sedie e le gambe strette nei pantaloni da cavallerizzo. - Come mai ieri sera non sei passato al teatro di Krasnoe Selo? La Numerova non andava mica male. Dove sei stato? 

- Ho fatto tardi dai Tverskij - rispose Vronskij. 

- Ah - ripeté Jašvin. 

Jašvin, giocatore, uomo sregolato, senza principi fuorché quelli immorali, era il miglior amico di Vronskij nel reggimento. Vronskij gli voleva bene e per la sua non comune forza fisica, della quale dava prova in particolar modo bevendo come un otre e facendo a meno di dormire pur rimanendo sempre presente a se stesso, e per la sua grande forza morale che dimostrava nei rapporti con i superiori e i compagni, suscitando timore e stima, e che nel giuoco (per il quale si impegnava per decine di migliaia di rubli e che sempre conduceva, malgrado il vino bevuto, con grande abilità e freddezza) lo faceva considerare il miglior giocatore del club inglese. Vronskij lo stimava e gli voleva bene soprattutto perché sentiva che Jašvin gliene voleva non per il suo nome o per la sua ricchezza, ma per la sua persona. E tra tutti gli uomini solo con lui Vronskij avrebbe voluto parlare del suo amore. Sentiva che solamente Jašvin, pur ostentando disprezzo verso qualsiasi sentimento, lui solo, così sembrava a Vronskij, poteva capire quella passione prepotente, che riempiva tutta la sua vita. Inoltre era sicuro che Jašvin non godeva del pettegolezzo e dello scandalo, ma intendeva quel sentimento così come andava inteso, e capiva e credeva cioè che quel suo amore non era un giuoco, né uno svago, ma qualcosa di molto più grave ed importante. 

Vronskij non aveva mai parlato con lui del suo amore, ma sentiva che egli sapeva tutto, che intendeva tutto così come andava inteso, e gli faceva piacere scorgere ciò dentro i suoi occhi. 

- Ah, sì - disse Jašvin, alludendo al fatto che Vronskij era stato dai Tverskij e, dopo aver lasciato sfuggire un guizzo dai suoi occhi neri, afferrò il baffo sinistro e cominciò a ficcarselo in bocca, secondo una cattiva abitudine. 

- Su, via, e tu ieri che hai fatto? Hai vinto? - chiese Vronskij. 

- Ottomila rubli. Ma tre non sono sicuri: difficile che li dia. 

- Su, così puoi anche perdere puntando su di me - disse Vronskij ridendo. (Jašvin aveva fatto una grossa scommessa su Vronskij). 

- Non perderò per nulla affatto. Solo Machotin è pericoloso. 

E la conversazione si aggirò sull'attesa delle corse di quel giorno alle quali solamente poteva pensare, ora, Vronskij. 

- Andiamo, ho finito - disse Vronskij e, alzatosi, si diresse verso la porta. Jašvin si alzò anche lui, allungando l'enormi gambe e la lunga schiena. 

- Per me è ancora presto per pranzare, ma non per bere. Vengo subito. Ehi, del vino! - gridò con la sua voce nota nel comando, tanto robusta da far tremare i vetri. - No, non occorre - gridò subito di nuovo. - Tu va' a casa, vengo anch'io con te. 

E andarono via insieme. 

\capitolo{XX}\label{xx-1} 

Vronskij alloggiava in un'izba finnica, spaziosa e pulita, divisa in due da un tramezzo. Petrickij viveva con lui. Petrickij dormiva quando Vronskij e Jašvin entrarono nell'izba. 

- Alzati, su, finiscila di dormire - disse Jašvin, entrando di là dal tramezzo e dando un colpo sulla spalla di quell'arruffone di Petrickij che s'era ficcato col naso nel guanciale. 

Petrickij saltò su a un tratto e si voltò a guardare. 

- È stato qui tuo fratello - disse a Vronskij. - Mi ha svegliato, che il diavolo se lo pigli; ha detto che verrà ancora. - E si gettò nuovamente sul guanciale, tirando su la coperta. - Ma smettila, Jašvin! - disse, arrabbiandosi con Jašvin che gli tirava via la coperta. - Basta! - Si girò e aprì gli occhi. - Di' piuttosto, cosa c'è da bere; ho una tale porcheria in bocca, che\ldots{} 

- Della vodka, è meglio di tutto - disse Jašvin con voce di basso. - Terešcenko! Vodka al signore e dei cetrioli - gridò, evidentemente compiaciuto d'ascoltare la propria voce. 

- Della vodka, pensi? Eh? - chiese Petrickij, facendo smorfie e fregandosi gli occhi. - E tu bevi? Beviamo insieme così. Vronskij, bevi anche tu? - disse Petrickij, alzandosi e avviluppandosi in una coperta tigrata. Uscì sulla porta del tramezzo, alzò le braccia e prese a cantare in francese: ``A Tule c'era un re''. - Vronskij, vuoi bere? 

- Fila via - disse Vronskij che metteva un soprabito tesogli dal servitore 

- Dove vai? - gli chiese Jašvin. - Ecco anche la trojka - aggiunse, dopo aver visto la vettura che si avvicinava. 

- Alla scuderia, e devo anche passare da Brjanskij per i cavalli - disse Vronskij. 

Vronskij aveva davvero promesso di andare da Brjanskij a dieci verste da Petergof, a portargli il denaro per i cavalli; voleva trovare il tempo di andare anche là. Ma i compagni capirono subito che non andava soltanto là. 

Petrickij, continuando a canterellare, ammiccò con un occhio e gonfiò le labbra come per dire: ``Lo sappiamo che Brjanskij è mai questo''. 

Jašvin disse soltanto: 

- Bada a non far tardi - e, per cambiar discorso: - Di', su, che forse il mio lupacchiotto fa il suo servizio tuttora? - chiese, guardando dalla finestra, a proposito di un cavallo da tiro che gli aveva venduto. 

- Fermati - disse Petrickij a Vronskij che stava già per uscire. - Tuo fratello ha lasciato per te una lettera e un biglietto. Aspetta un po', dove sono? 

Vronskij si fermò. 

- Su, dove sono? 

- Dove sono? Ecco, qui sta la questione! - disse solennemente Petrickij, facendo passare sul naso il dito indice. 

- Su parla ancora, non fare lo stupido - disse Vronskij, sorridendo. 

- Non ci ho mica acceso il camino. Devono essere qui in qualche parte. 

- Su, basta, amico. Dov'è la lettera? 

- No, davvero non ricordo. O che forse l'ho visto in sogno? Aspetta, aspetta! Ma perché arrabbiarsi? Se tu avessi bevuto quattro bottiglie, come me ieri, alla salute di tuo fratello, anche tu avresti dimenticato dove eri steso\ldots{} Aspetta, me lo ricordo subito! 

Petrickij andò di là dal tramezzo e si sdraiò sul letto. 

- Fermati! Ero sdraiato così io, così in piedi stava lui. Sì, sì, sì\ldots{} Eccola! - E Petrickij tirò fuori di sotto al materasso la lettera che aveva nascosta. 

Vronskij prese la lettera e il biglietto. Era proprio quel che si aspettava: una lettera della madre coi rimproveri perché non andava da lei e un biglietto del fratello che gli diceva di dovergli parlare. Vronskij sapeva che era sempre la stessa cosa. ``Che gliene importa a loro!'' pensò e, spiegazzate le lettere, se le ficcò tra i bottoni del soprabito per leggerle con calma per via. Nell'ingresso dell'izba incontrò due ufficiali, uno del proprio e l'altro di un altro reggimento. 

L'abitazione di Vronskij era sempre il ritrovo di tutti gli ufficiali. 

- Dove vai? 

- Devo andare a Petergof. 

- E il cavallo è venuto da Carskoe? 

- È arrivato, ma non l'ho visto ancora. 

- Dicono che Gladiator di Machotin si sia azzoppato. 

- Sciocchezze! Ma, come farete a saltare su questo fango? - disse l'altro. 

- Ecco i miei salvatori! - gridò Petrickij, vedendo quelli che erano entrati, mentre l'attendente gli stava davanti con la vodka e i cetrioli salati sopra un vassoio. - Ecco, è Jašvin che mi ordina di bere per rinfrescarmi. 

- Su, stanotte l'avete fatta bella - disse uno di quelli che erano entrati - tutta la notte non ci avete fatto dormire. 

- Già, ma sapete come è andata a finire? - raccontava Petrickij. - Volkov s'è arrampicato sul tetto e s'è messo a dire che si sentiva triste. Io dico: attacca la musica, una marcia funebre! E lui s'è addormentato proprio così sul tetto, al suono della marcia funebre! 

- Bevi, bevi assolutamente la vodka, e dopo l'acqua di seltz e molto limone - disse Jašvin, curvandosi sopra Petrickij come una madre che obblighi un bimbo a prendere la medicina - e dopo, anche un po' di champagne, così, una bottiglietta. 

- Ecco, questa è una cosa intelligente. Aspetta, Vronskij, beviamo. 

- No, addio, signori miei, adesso non bevo. 

- E che mai, diventi uggioso? Su, allora, da solo. Dammi dell'acqua di seltz e il limone. 

- Vronskij! - gridò qualcuno mentre egli usciva già nell'ingresso. 

- Che c'è? 

- Dovresti tagliarti i capelli, se no ti pesano, specie sulla zucca. 

Vronskij infatti cominciava a diventar calvo prima del tempo. Egli rise allegramente, mostrando i bei denti allineati e abbassando il berretto sulla calvizie; uscì e montò in carrozza. 

- Alla scuderia! - disse, e voleva tirar fuori le lettere per finire di leggerle, ma poi cambiò idea, per non distrarsi prima della visita al cavallo. ``Dopo!\ldots{}''

\capitolo{XXI}\label{xxi-1} 

La scuderia provvisoria era una baracca di assi costruita proprio accanto all'ippodromo, e là doveva essere stato condotto il giorno prima il suo cavallo. Non l'aveva ancora visto. In quegli ultimi giorni non l'aveva montato neppure per una breve passeggiata, ma l'aveva affidato all'allenatore, e ora non sapeva proprio in quale condizioni fosse giunto e si trovasse. Appena scese dalla carrozza, il garzone di scuderia (il groom, come era chiamato il ragazzo), che ne aveva già riconosciuto da lontano la vettura, chiamò l'allenatore. L'inglese magro, con gli stivali alti, la giacchetta corta e un ciuffo di peli lasciati crescere solo sotto il mento, gli venne incontro con il passo ondeggiante dei fantini, allargando i gomiti e dondolandosi. 

- Ebbene, che ne è di Frou-Frou? - disse Vronskij in inglese. 

- All right, sir\ldots{} tutto va bene, signore - pronunciò, chi sa in quale parte della gola, la voce dell'inglese. - È meglio che non andiate - aggiunse, togliendosi il berretto. - Le ho messo la musoliera e la cavalla è inquieta. È meglio che non andiate. 

- No, voglio entrare. Voglio dare un'occhiata. 

- Andiamo - disse l'inglese, sempre senza muover le labbra, e, tutto accigliato e dondolando i gomiti, andò avanti col suo passo dinoccolato. 

Entrarono nel piccolo cortile che era dinanzi alla baracca. Il mozzo di stalla, snello e robusto, con la giacchetta pulita, e una scopa in mano, si fece incontro a quelli che entravano, e li seguì. Nella baracca c'erano i cavalli nei loro recinti, e Vronskij sapeva che quel giorno doveva certo esservi stato condotto e trovarsi là il suo più grande antagonista, il sauro di Machotin, Gladiator, venti centimetri più lungo dell'ordinario. Più che il cavallo suo Vronskij avrebbe voluto vedere Gladiator che non aveva mai visto; ma sapeva che, secondo il codice dell'ippica, non era corretto volerlo vedere, e neppure chiederne. Mentre camminava per il corridoio, il mozzo aprì la porta del secondo recinto a sinistra e Vronskij intravide un sauro grosso dalle zampe bianche. Capì che era Gladiator, ma con l'istinto dell'uomo che distoglie lo sguardo da una lettera aperta che appartenga ad altri, distolse lo sguardo e si accostò al recinto di Frou-Frou. 

- Qui è il cavallo di Mak\ldots{} Mak\ldots{} non riesco mai a pronunciare questo nome - disse l'inglese al di sopra della spalla, indicando, col dito pollice dall'unghia sporca, il recinto di Gladiator. 

- Di Machotin? Già, questo è il mio temibile antagonista - disse Vronskij. 

- Se montaste voi questo - disse l'inglese - scommetterei su di voi. 

- Frou-Frou è più nervosa, questo qui è più forte - disse Vronskij, sorridendo per la lode alla sua abilità di cavallerizzo. 

- Nella corsa ad ostacoli tutto sta nell'arte di inforcare e nel pluck - disse l'inglese. 

Di pluck, di energia e di coraggio, cioè, Vronskij sentiva di averne ad usura, ma, quel che più contava, era fermamente convinto che nessuno al mondo di questo pluck potesse averne più di lui. 

- Siete sicuro che non occorra per la cavalla una grande sudata? 

- Non occorre - rispose l'inglese. - Per favore non parlate forte. Il cavallo si agita - aggiunse accennando col capo verso lo stallo davanti al quale stavano fermi e dal quale si udiva lo scalpitare della bestia sulla paglia. 

Aprì la porta e Vronskij entrò in un recinto debolmente illuminato da una grata. Nello stallo calpestava la paglia fresca una giumenta baia dal manto scuro con la musoliera. Abituato l'occhio alla penombra, Vronskij avvolse di nuovo con un solo sguardo le forme della sua cavalla favorita. Frou-Frou era una cavalla di media altezza, non senza difetti; era tutta stretta di ossatura ed anche il petto molto sporgente era stretto. Aveva la groppa un po' cascante e le zampe anteriori e soprattutto le posteriori alquanto ritorte. I muscoli delle posteriori e quelli delle anteriori non erano particolarmente grossi; ma in compenso, nel sottopancia, la cavalla era eccezionalmente larga, cosa che balzava agli occhi ora che, tenuta a regime, aveva la pancia incavata. Le ossa delle zampe, al di sotto dei ginocchi, viste di fronte, sembravano non più grosse di un dito, ma, in compenso, erano straordinariamente larghe viste di lato. Tutta la bestia, eccettuate le costole, era come se l'avessero schiacciata nei fianchi e tirata in lungo. Ma possedeva al massimo una qualità che faceva dimenticare tutti i suoi difetti; aveva il sangue, sangue ``che si fa sentire'', come dicono gli inglesi. I muscoli fortemente rilevati al di sotto della rete delle vene, distesi sotto la pelle sottile, mobile e liscia come raso, sembravano duri come ossa. La testa asciutta, con gli occhi in rilievo, luminosi e vivi, si allargava verso le froge prominenti dalle membrane iniettate di sangue all'interno. In tutta la linea della cavalla, e in particolare nella testa, c'era qualcosa di volitivo e nello stesso tempo di dolce. Era una di quelle bestie che sembra non parlino solo perché la conformazione della loro bocca non lo permette. 

A Vronskij, almeno, sembrò che essa capisse tutto quello che egli provava, ora, nel guardarla. Non appena Vronskij era entrato, essa aveva aspirato profondamente l'aria e, torcendo l'occhio sporgente tanto da iniettar di sangue la cornea, aveva guardato dalla parte opposta a quella da cui erano entrati, scotendo la musoliera e appoggiandosi elastica ora su una zampa ora su di un'altra. 

-Ecco, vedete come è agitata - disse l'inglese. 

- Oh, cara! - diceva Vronskij, accostandosi alla cavalla ed esortandola. 

Ma quanto più si accostava, tanto più essa si agitava. Solo quando egli si accostò alla testa, si calmò a un tratto, mentre i muscoli trasalivano sotto il pelame sottile, delicato. Vronskij le carezzò il collo forte, aggiustò sul garrese erto un ciuffo della criniera caduto dall'altra parte, ed accostò il viso alle narici dilatate e sottili come ala di pipistrello. Essa aspirò ed emise sonoramente l'aria dalle narici tese; rabbrividendo, drizzò l'occhio aguzzo e protese il labbro forte e nero verso Vronskij come se volesse afferrarlo per la manica. Ma, ricordatasi della musoliera, la scosse e cominciò di nuovo a far cambiare di posto, una dopo l'altra, le sue zampe tornite. 

- Calma, cara, calma - diceva Vronskij carezzandola ancora con la mano su per la groppa e, con la gioiosa convinzione che la cavalla fosse in forma perfetta, uscì dallo stallo. 

L'agitazione della cavalla si era comunicata a Vronskij; sentiva che il sangue gli affluiva al cuore e che anche lui, come l'animale, aveva voglia di muoversi, di mordere; c'erano in lui orgasmo e gioia. 

- Su, allora conto su di voi - disse all'inglese - alle sei e mezzo sul posto. 

- Tutto è in ordine - disse l'inglese. - E voi dove andate, milord? - chiese inaspettatamente, adoperando questa denominazione di my-Lord che non usava quasi mai. 

Vronskij sollevò il capo, sorpreso, e guardò così come sapeva guardare lui, non negli occhi, ma in fronte, l'inglese, meravigliandosi della temerarietà della sua domanda. Ma, capito che l'inglese, nel far la domanda, non l'aveva rivolta a lui come signore, ma come cavallerizzo, rispose: 

- Devo andare da Brjanskij, fra un'ora sarò di ritorno. 

``Quante volte mi fanno questa domanda, oggi!'' si disse, e arrossì, cosa che gli capitava di rado. L'inglese lo guardò attento, e, come se avesse saputo dove Vronskij andava, aggiunse: 

- Ciò che più conta è l'essere calmi prima di montare - disse. - Non vi contrariate per nessuna cosa, ed evitate ogni emozione. 

- All right - rispose sorridendo Vronskij e, saltato nella vettura, ordinò di andare a Petergof. 

Allontanatosi appena di qualche passo, una nuvola, che dalla mattina aveva minacciato la pioggia, si addensò e venne giù un acquazzone. 

``Male - pensò Vronskij, alzando il mantice della vettura. - C'era già fango, e ora sarà proprio un pantano''. Trovandosi solo nella vettura, tirò fuori la lettera della madre e il biglietto del fratello e ne terminò la lettura. 

Sì, sempre la stessa cosa. Tutti, sua madre, suo fratello, tutti ritenevano necessario immischiarsi nei suoi affari di cuore. Questa ingerenza suscitava in lui un rancore, sentimento che di rado provava. ``Cosa importa loro? Perché ognuno ritiene di doversi occupare di me? Ma perché non mi lasciano in pace? Perché è una cosa che non possono capire. Se fosse la solita volgare relazione mondana, mi avrebbero lasciato in pace. Sentono invece che è qualcosa di diverso, che non è uno svago, e che questa donna mi è più cara della vita. E proprio questo è incomprensibile per loro, e perciò inquietante. Qualunque sia e sarà il nostro destino, noi ce lo siamo fatto e non ce ne lamentiamo - diceva unendo nella parola `noi' se stesso e Anna. - E invece no, sentono il bisogno d'insegnarci a vivere. Non hanno neppure un'idea di che cosa sia la felicità, non sanno che senza questo amore per noi non c'è felicità, né infelicità, non c'è vita'' pensava. 

S'irritava contro tutti per questa intrusione nei fatti suoi, proprio perché sentiva in fondo all'anima che loro, tutti gli altri, avevano ragione. Sentiva che l'amore che lo legava ad Anna non era una distrazione momentanea che passa come passano tutte le relazioni mondane senza lasciare altra traccia nella vita dell'uno e dell'altra che un ricordo grato o increscioso. Sentiva tutto il tormento della posizione sua e di quella di lei, l'imbarazzo creato dalla necessità, esposti com'erano agli occhi del mondo, di dover nascondere il loro amore, e di mentire e di ingannare, di dover usare mille astuzie e doversi preoccupare continuamente degli altri, mentre la loro passione era così grande che per entrambi null'altro v'era al di fuori del loro amore. 

Ricordava con chiarezza le circostanze così frequenti nelle quali era stato necessario usare l'inganno e la falsità, così avversi alla natura sua; ricordava in modo particolarmente vivo il senso di vergogna più di una volta notato in lei per la necessità di mentire e di ingannare. Dal tempo della sua relazione con Anna egli provava una sensazione strana, che lo afferrava ogni tanto. Era come un senso di nausea per qualche cosa: per Aleksej Aleksandrovic, per se stesso o per il mondo intero, non sapeva bene. Ma allontanava sempre questa sensazione strana. E anche ora, dopo essersene liberato, continuava il corso dei suoi pensieri. 

``Già, lei prima era infelice, ma orgogliosa e tranquilla; ora, invece, non può essere orgogliosa e tranquilla, pur fingendo di esserlo. Sì, tutto questo deve finire'' decise da ultimo. 

Così per la prima volta gli apparve chiaro nella mente il pensiero che fosse indispensabile porre termine a quella menzogna, e che quanto prima ciò sarebbe accaduto tanto meglio sarebbe stato. 

``Lei ed io dobbiamo abbandonare tutto e andarci a nascondere in qualche luogo, noi due, con il nostro amore'' disse a se stesso. 

\capitolo{XXII}\label{xxii-1} 

L'acquazzone non durò a lungo, e mentre Vronskij si avvicinava a gran trotto col cavallo di centro che tirava e i due di lato che galoppavano liberi, senza redini, nel fango, il sole era già ricomparso, e i tetti delle ville e i vecchi tigli dei giardini, dall'una e dall'altra parte della strada maestra, scintillavano di un luccichio umido, mentre l'acqua gocciolava allegramente dai rami e grondava giù dai tetti. Non pensava ormai più che quell'acquazzone avrebbe potuto guastare l'ippodromo, si rallegrava invece che, grazie alla pioggia, avrebbe trovato certamente lei in casa e sola, poiché aveva saputo che Aleksej Aleksandrovic, rientrato da poco dalla cura delle acque, era rimasto a Pietroburgo. 

Con la speranza di trovarla sola, Vronskij, come del resto faceva sempre per non essere notato, smontò prima di arrivare al ponte, e andò a piedi. Evitò l'ingresso che dava sulla strada ed entrò per il cortile. 

- Il signore è arrivato? - chiese al giardiniere. 

- Nossignore. La signora è in casa. Ma vi prego, passate per la scala; là c'è gente, vi apriranno - rispose il giardiniere. 

- No, passerò dal giardino. 

Sicuro ormai di trovarla sola, e desideroso di coglierla di sorpresa (non aveva promesso di andare quel giorno, e probabilmente ella non sospettava di vederlo là prima delle corse), proseguì, trattenendo la sciabola e camminando cauto sulla ghiaia del viottolo fiancheggiato da fiori, verso la terrazza che dava sul giardino. Vronskij in quel momento non pensava più alla gravità ed alla difficoltà della situazione: pensava solo che l'avrebbe veduta, non già immagine, ma viva, tutta, così com'era nella realtà. Stava per entrare, poggiando per intero il piede per non far rumore, sugli scalini inclinati della terrazza, quando gli balenò nella mente il ricordo di quel che rappresentava il lato più tormentoso della sua relazione con Anna: il ricordo del figlio di lei, con quel suo sguardo indagatore, che gli sembrava ostile. 

Questo ragazzo, più di chiunque altro, rappresentava un intralcio alla loro relazione. Quando egli era presente, sia Vronskij che Anna non solo non si permettevano di parlare se non di cose da potersi dire dinanzi a tutti, ma non si concedevano neppure di fare allusioni a cose che il ragazzo non avrebbe potuto capire. E ciò non per averne parlato insieme, ma spontaneamente si era prodotto da sé. Sentivano come un'offesa a se stessi ingannare quel fanciullo. In sua presenza parlavano tra di loro come semplici conoscenti. Malgrado quest'accortezza, Vronskij scorgeva spesso fisso su di sé lo sguardo attento e perplesso del bambino e una strana timidezza, una discontinuità di atteggiamento, ora tenero, ora ritroso e riservato, nel modo di comportarsi del ragazzo nei suoi riguardi. Come se il ragazzo sentisse che tra sua madre e lui c'era un rapporto importante del quale non poteva penetrare la natura. 

Infatti il fanciullo sentiva di non poter intendere, per quanto ci si sforzasse, quel rapporto, e non sapeva rendersi conto di ciò che sentiva verso quell'uomo. Con la particolare sensibilità dei bambini, vedeva chiaramente che il padre, la governante, la njanja, tutti, non solo non amavano, ma pur senza parlarne, guardavano con avversione e timore Vronskij, che la madre invece considerava come il suo migliore amico. 

``Che cosa vuol dire questo? Chi è quell'uomo? Come debbo volergli bene? Se non lo capisco, la colpa è mia che sono un ragazzo sciocco e cattivo'' pensava il bambino; e da ciò derivavano la sua espressione indagatrice e quasi ostile, e quella discontinuità che tanto turbava Vronskij. La presenza di questo bambino suscitava sempre in Vronskij lo strano senso di nausea irragionevole che egli provava in quegli ultimi tempi. La presenza del bambino suscitava in Vronskij e in Anna una sensazione simile a quella del navigante che veda dalla bussola che la direzione, nella quale si muove rapido, si allontana da quella dovuta, ma che arrestare il moto non è più nelle sue forze, che ogni attimo lo allontana sempre più dalla giusta direzione e che confessare a se stesso la deviazione è lo stesso che confessare la propria rovina. 

Il bambino con la sua innocente visione della vita era la bussola che mostrava loro il grado di deviazione dalla rotta che conoscevano, ma che volevano ignorare. 

Questa volta Serëza non era in casa, e Anna, completamente sola, stava seduta sulla terrazza ad aspettare il ritorno del figlio uscito a spasso e sorpreso dalla pioggia. Aveva mandato un domestico e una cameriera a cercarlo e stava lì ad attenderlo. Vestiva un abito bianco con un largo ricamo; sedeva in un angolo della terrazza di là dai fiori e non aveva avvertito l'avvicinarsi di lui. Abbassata la testa nera e inanellata, premeva la fronte contro un freddo annaffiatoio che era sulla ringhiera, trattenendolo con entrambe le mani bellissime dagli anelli a lui noti. La bellezza di tutta la sua figura, della testa, del collo, delle braccia, colpiva ogni volta Vronskij come una cosa inattesa. Si fermò a guardarla incantato. Ma appena volle fare un passo per avvicinarsi, ella sentì subito l'appressarsi di lui, scostò l'annaffiatoio e voltò verso di lui il viso infiammato. 

- Che vi è accaduto? Non state bene? - disse egli in francese, avvicinandosi. Avrebbe voluto correre a lei, ma ricordando che potevano esserci estranei, si voltò a guardare verso la porta della terrazza e arrossì come arrossiva ogni volta che doveva temere ed essere guardingo. 

- No, sto bene - disse lei, alzandosi e stringendo con forza la mano ch'egli le tendeva. - Non ti aspettavo\ldots{} 

- Dio mio che mani fredde! - egli disse. 

- Mi hai spaventata. Sono sola e aspettavo Serëza che è andato a spasso: verranno di qui 

Ma, pur sforzandosi d'essere calma, le labbra le tremavano. 

- Perdonatemi se son venuto, ma non potevo far passare un altro giorno senza vedervi - continuò in quel francese che usava sempre per evitare il voi russo freddo fino all'impossibile fra di loro e il tu troppo pericoloso. 

- E perché perdonare? Sono così felice! 

- Ma voi non state bene, o siete rattristata - continuò senza lasciarle la mano e chinandosi su di lei. - A che pensavate? 

- Sempre alla stessa cosa - disse lei con un sorriso. Diceva la verità. Ogni volta, in qualunque momento le si fosse chiesto a cosa pensasse, poteva rispondere senza errore: a una cosa sola, alla sua felicità e alla sua infelicità. Pensava proprio questo nel momento in cui egli l'aveva sorpresa: pensava perché per gli altri, per Betsy, ad esempio (ella sapeva la sua relazione, tenuta segreta per il mondo, con Tuškevic), era facile ciò che per lei era tanto tormentoso. Quel giorno, questo pensiero, per varie ragioni, la tormentava in modo particolare. Gli domandò delle corse. Egli, vedendola agitata, prese a raccontare, per distrarla, con tono semplice, i particolari dei preparativi delle corse. 

``Dirlo o non dirlo? - pensava intanto lei, guardando negli occhi calmi e carezzevoli di lui. - È così felice, così preso dalle sue corse, che non gli darà il peso che si deve, non capirà tutta l'importanza per noi di questo avvenimento''. 

- Ma voi non avete detto a cosa pensavate quando sono entrato - egli disse, interrompendo il racconto - vi prego, ditemelo! 

Ella non rispondeva e, chinato un poco il capo, lo guardò interrogativamente di sotto in su con i suoi occhi luminosi, dietro le lunghe ciglia. La mano che giocava con una foglia strappata, tremò. Egli notò questo e il suo viso espresse quella sottomissione, quella dedizione da schiavo che tanto la seduceva. 

- Vedo che è accaduto qualcosa. Posso mai esser tranquillo, sapendo che avete una pena che io non divido? Parlate, per amor di Dio - ripeté supplichevole. 

``Non gli perdonerei se non capisse tutto il significato della cosa. Meglio non dirglielo: perché metterlo alla prova?'' pensava lei, continuando a guardarlo e sentendo che la mano che tratteneva la foglia tremava sempre di più. 

- Per amor di Dio - ripeté lui, prendendole la mano. 

- Devo dirlo? 

- Sì, sì, sì\ldots{} 

- Sono incinta - disse lei a voce bassa. La foglia tremò ancora di più nella mano, ma ella non distolse gli occhi da lui per scorgere come accogliesse la notizia. Egli impallidì, volle dire qualcosa, ma si fermò, lasciò cadere la mano di lei e chinò il capo. 

``Sì, ha capito tutta l'importanza di questo avvenimento'' ella pensò e gli strinse la mano con gratitudine. 

Ma si era sbagliata nel credere ch'egli intendesse il significato della notizia così come lei, donna, l'intendeva. A quella notizia egli aveva sentito dieci volte più intenso un attacco di quella strana sensazione che l'afferrava come una nausea di qualcosa; ma insieme a questo egli aveva sentito che la crisi desiderata era ormai giunta, che non si poteva più nascondere la cosa al marito e che era indispensabile rompere in un modo o nell'altro quella situazione equivoca. Oltre a ciò, l'agitazione di lei gli si era comunicata fisicamente. La guardò con uno sguardo intenerito, sottomesso, le baciò la mano, si alzò e si mise a camminare in silenzio per la terrazza. 

- Sì - disse poi, avvicinandosi a lei con decisione. - Né io né voi abbiamo considerato i nostri rapporti come un giuoco, e ora la nostra sorte è decisa. È indispensabile porre termine alla menzogna in cui viviamo - disse, guardandosi in giro. 

- Porre termine? E come, Aleksej? - ella disse piano. 

Era calma adesso, e il suo viso splendeva d'un sorriso tenero. 

- Lasciare vostro marito e unire la nostra vita. 

- È unita anche così - ella rispose in modo appena percettibile. 

- Sì, ma del tutto, del tutto. 

- Ma come, Aleksej, dimmi come? - disse con triste irrisione verso il suo caso senza via d'uscita. - Vi è forse una via d'uscita da una posizione come la nostra? Non sono forse la moglie di mio marito? 

- Da qualsiasi situazione c'è una via d'uscita. Bisogna decidersi - egli disse. - Qualunque cosa è migliore della posizione in cui vivi. Perché io vedo come ti tormenti per tutto, e per il mondo, e per tuo figlio e per tuo marito. 

- Ah, per mio marito, no - ella disse con un riso schietto. - Non so, non penso a lui, non esiste. 

- Tu non parli con sincerità. Ti conosco. Ti tormenti anche per lui. 

- Ma egli non lo sa neppure - ella disse e, a un tratto, un rossore vivo cominciò a salirle al viso; le guance, la fronte, il collo si arrossarono, e lacrime di vergogna le salirono agli occhi. - Ma non parliamo di lui. 

\capitolo{XXIII}\label{xxiii-1} 

Vronskij già altra volta aveva tentato, anche se non in maniera così decisa come ora, di indurla a esaminare la situazione, e ogni volta s'era imbattuto in quella superficialità e leggerezza di giudizio con la quale ella ora rispondeva al suo invito. Pareva esserci qualcosa ch'ella non potesse o non volesse chiarire a se stessa, pareva che non appena se ne cominciava a parlare, lei, la vera Anna, se ne andasse chi sa in qual parte di sé, e venisse fuori un'altra donna strana, estranea a lui, ch'egli non amava, anzi temeva, e che gli opponeva resistenza. Ma egli ora decise di chiarire tutto. 

- Ch'egli lo sappia o no - disse Vronskij col suo solito tono fermo e calmo - lo sappia o no, a noi questo non importa. Noi non possiamo, voi non potete continuare a stare così, specialmente ora. 

- E che fare, secondo voi? - chiese lei con la stessa sottile irrisione. A lei, che aveva tanto temuto ch'egli prendesse alla leggera la sua gravidanza, spiaceva, ora, che da questa egli traesse la necessità d'intraprendere qualcosa. 

- Rivelargli tutto, e lasciarlo. 

- Molto bene, ammettiamo che io lo faccia - ella disse. - Sapete cosa ne verrà fuori? Ve lo dico io fin d'ora - e una luce cattiva s'accese nei suoi occhi, un momento prima teneri. - ``Ah, voi amate un altro e avete contratto un legame peccaminoso con lui?'' - e, rifacendo il marito, metteva esattamente l'accento sulla parola ``peccaminoso'', così come faceva Aleksej Aleksandrovic. - ``Vi ho avvertito delle conseguenze dal punto di vista religioso, civile e familiare. Voi non m'avete dato ascolto. Adesso io non posso esporre al disonore il mio nome\ldots{}'' - ``e mio figlio'' ella avrebbe voluto dire, ma sul figlio non si poteva scherzare\ldots{} - ``al disonore il mio nome'' e ancora qualcosa del genere - aggiunse. - Si terrà sulle generali, parlerà col suo tono di uomo di stato e con chiarezza e precisione, dirà che non può lasciarmi andare, ma che prenderà le misure che dipendono da lui per arrestare lo scandalo. E farà tranquillamente, accuratamente tutto quello che avrà detto. Ecco quello che accadrà. Non è un uomo, ma una macchina, e una macchina cattiva, quando si arrabbia - ella aggiunse, ricordandosi intanto di Aleksej Aleksandrovic in tutti i particolari della sua figura, del suo modo di parlare e del suo carattere, e facendogli colpa di tutto quello che ella poteva trovare di cattivo in lui, senza perdonargli nulla, proprio per quella terribile colpa che essa aveva verso di lui. 

- Ma, Anna - disse Vronskij con voce suadente, dolce, cercando di calmarla - ma è indispensabile dirglielo e poi regolarsi secondo quello ch'egli farà. 

- E allora, fuggire? 

- E perché anche non fuggire? Non vedo la possibilità di continuare così\ldots{} E non per me\ldots{} vedo che voi ne soffrite. 

- Già, fuggire e diventare la vostra amante? - disse Anna con cattiveria. 

- Anna! - egli disse con rimprovero e tenerezza. 

- Già - continuò lei - diventare la vostra amante e perdere tutto\ldots{} 

Ella voleva di nuovo dire: ``mio figlio'', ma non poté pronunciarla, questa parola. 

Vronskij non riusciva a capire come lei, con la sua forte natura onesta, potesse sopportare quella situazione d'inganno e non desiderasse uscirne; ma egli non ne indovinava la ragione principale, che cioè era la parola ``figlio'' ch'ella non poteva pronunciare. Quando pensava al figlio e ai suoi futuri rapporti con la madre che avesse abbandonato il padre suo, era presa da un tale terrore di quello che aveva fatto da non ragionare più e, come ogni donna, si sforzava solo di calmarsi con ragionamenti mendaci e parole vane, desiderando che tutto rimanesse così com'era e che si potesse dimenticare la terribile questione: che cosa ne sarebbe stato del figlio. 

- Ti prego, ti supplico - diss'ella a un tratto con tono del tutto diverso, sincero e tenero, prendendolo per mano - non parlarmene mai più. 

- Ma Anna\ldots{} 

- Mai. Lascia a me tutto questo. Conosco tutta la bassezza, tutto l'orrore della mia posizione, ma la cosa non è così facile a decidersi come tu credi. E lascia fare a me e ascoltami. Non parlar mai più con me di questo. Me lo prometti? No, no, prometti! 

- Io prometto tutto, ma non posso esser tranquillo, specialmente dopo quello che mi hai detto. Non posso esser tranquillo quando non puoi esserlo tu\ldots{} 

- Io - ella ripeté - sì, io mi tormento a volte, ma passerà, se tu non parlerai mai più con me di questo. Quando tu ne parli, allora soltanto me ne tormento. 

- Non capisco - disse lui. 

- Io so - ella lo interruppe - quanto sia penoso per la tua natura onesta mentire, e mi fai pena. Spesso penso che tu, per me, hai rovinato la tua vita. 

- Anch'io, or ora, pensavo la stessa cosa - egli disse - come hai potuto sacrificare tutto a me? Io non posso perdonarmi d'averti resa infelice. 

- Io infelice? - ella disse, accostandosi a lui e guardandolo con un entusiastico riso d'amore - io sono come un essere affamato al quale abbiano dato da mangiare. Avrà forse freddo, e avrà il vestito lacero; avrà vergogna, forse, ma non è infelice. Io infelice? No, eccola, la mia felicità\ldots{} 

Aveva sentito la voce del figlio che era tornato e, data una rapida occhiata alla terrazza, si alzò di scatto. Il suo sguardo si accese del fuoco a lui ben noto, sollevò con un movimento rapido le belle mani coperte d'anelli, gli afferrò il capo, lo guardò con un lungo sguardo, e, avvicinando a lui il viso con le labbra aperte, ridenti, gli baciò in fretta la bocca e gli occhi, e lo respinse. Voleva andar via, ma egli la trattenne. 

- A quando? - bisbigliò in un sussurro, guardandola rapito. 

- Stanotte, all'una - ella mormorò e, con un sospiro profondo, andò incontro al figlio col suo passo svelto e leggero. 

La pioggia aveva sorpreso Serëza nel giardino grande, e lui e la njanja erano rimasti a sedere sotto una pergola. 

- Ebbene, arrivederci - diss'ella a Vronskij. - È necessario affrettarsi per le corse. Betsy ha promesso di passare a prendermi. 

Vronskij, guardato l'orologio, se ne andò in fretta. 
\enlargethispage*{1\baselineskip}

\capitolo{XXIV}\label{xxiv-1} 

Nel momento in cui Vronskij aveva guardato l'orologio sulla balconata dei Karenin era così turbato e preoccupato che aveva visto, sì, le lancette del quadrante, ma non aveva potuto capire che ora fosse. Uscì in strada e si diresse, camminando cauto nel fango, verso la vettura. Era dominato dal sentimento suo verso Anna, così che non pensava neppure più che ora fosse e se gli restasse il tempo per andare da Brjanskij. Gli rimaneva ora, come spesso accade, solo una certa memoria istintiva, quella che serve a indicare in quale ordine si è stabilito di fare le cose. Si accostò al cocchiere che s'era messo a dormire stando a cassetta, all'ombra già obliqua di un tiglio folto, e fu attratto per un attimo dai nugoli cangianti dei moscerini che volteggiavano sui cavalli sudati. Svegliato il cocchiere, saltò in vettura e ordinò di andare a Brjanskij. Solo dopo essersi allontanato di sette verste, tornò in sé, guardò l'ora, e questa volta capì che erano le cinque e mezzo e che era in ritardo. 

C'erano, quel giorno, varie corse: una per gli uomini di scorta, e una, su due verste, per gli ufficiali; un'altra su quattro verste e infine la corsa alla quale avrebbe partecipato lui. Per questa sarebbe giunto in tempo ma, passando prima da Brjanskij, sarebbe giunto dopo l'arrivo della corte. Non era ben fatto; ma aveva dato la sua parola a Brjanskij, e perciò decise di proseguire, dopo aver detto al cocchiere di non risparmiare la trojka. 

Arrivò da Brjanskij, rimase da lui cinque minuti e rifece la strada di galoppo. La corsa veloce lo calmò. Tutto quello che c'era di penoso nei suoi rapporti con Anna, tutta l'incertezza che era restata dopo la loro conversazione, tutto gli uscì di mente; ora pensava solo alla corsa con gioia e con orgasmo; pensava che sarebbe pure arrivato in tempo, e solo di tanto in tanto, l'attesa del convegno di quella notte si accendeva nella sua immaginazione di luce viva. 

La passione della corsa imminente lo prendeva sempre più a misura che si avvicinava all'ippodromo, nell'atmosfera delle corse, sorpassando le vetture di coloro che vi si recavano dai dintorni e da Pietroburgo. 

Nella sua abitazione non c'era più nessuno: tutti erano alle corse e il servo l'aspettava accanto al portone. Mentre egli si cambiava d'abito, il servo gli comunicò che era già cominciata la seconda gara e che molti signori erano venuti a chiedere di lui e che già due volte era venuto di corsa il garzone della scuderia. 

Cambiatosi senza fretta (egli non s'affrettava mai e non perdeva mai il dominio di sé), Vronskij ordinò di andare verso le baracche. Dalle baracche poteva vedere già quel mare di carrozze, di pedoni, di soldati che circondavano l'ippodromo, e le tribune piene di gente. Si correva, probabilmente, la seconda gara, perché entrando nella baracca, udì il suono della campana. Nell'avvicinarsi alla scuderia incontrò Gladatior, il sauro di Machotin dalle zampe bianche, condotto all'ippodromo con una groppiera arancione e azzurra, con le orecchie che sembravano enormi, anch'esse orlate di turchino. 

- Dov'è Kord? - domandò allo stalliere. 

- Nella scuderia, sta sellando. 

Nel recinto all'aperto, Frou-Frou era già sellata. Stava per essere portata fuori. 

- Non sono in ritardo? 

- All right! all right! Tutto, tutto bene - disse l'inglese - non vi agitate. 

Vronskij avvolse ancora una volta con lo sguardo le forme deliziose, a lui così care, della cavalla che vibrava per tutto il corpo e, staccatosene con rincrescimento, uscì dalla baracca. Si avvicinò alle tribune nel momento più opportuno per non attirare su di sé l'attenzione. Stava per terminare la corsa su due verste e tutti gli occhi erano fissi su di un cavalleggero della guardia che era in testa e su di un ussaro, a breve distanza da lui, che incitavano i cavalli all'ultimo sforzo nell'avvicinarsi al traguardo. Dal centro e dall'esterno dell'emiciclo tutti si affollavano verso il traguardo, e un gruppo di cavalleggeri, soldati e ufficiali, esprimeva, con rumorose acclamazioni, la gioia per l'atteso trionfo del loro ufficiale e compagno. Vronskij di soppiatto entrò nel mezzo della folla quasi nello stesso momento in cui sonava la campanella che annunciava la fine della corsa, e il cavalleggero della Guardia che era arrivato primo, alto, spruzzato di fango, abbandonatosi sulla sella, andava allentando le briglie allo stallone grigio, scurito dal sudore, ansante. 

Lo stallone, puntando le zampe con sforzo trattenne I'andatura veloce del corpo, e l'ufficiale dei cavalleggeri guardò intorno come un uomo risvegliato da un sonno pesante, e sorrise a stento. Una folla di amici e di estranei lo circondò. 

Vronskij evitava di proposito quella folla scelta del gran mondo che si moveva e discorreva con discrezione e disinvoltura dinanzi alle tribune. Sapeva che Ià c'erano la Karenina e Betsy e la moglie di suo fratello e, proprio per non distrarsi, non si avvicinava a loro. Ma gli amici che incontrava lo fermavano continuamente, gli raccontavano i particolari delle gare già corse, gli chiedevano perché fosse arrivato in ritardo. 

Mentre coloro che avevano già terminate le gare eran chiamati sulle tribune a ricevere i premi e tutti si volgevano verso quella parte, il fratello maggiore di Vronskij, Aleksandr, in alta uniforme da colonnello, non alto, robusto come Aleksej, ma più bello e colorito, col naso rosso e la faccia da ubriacone, aperta, si accostò a Iui. 

- Hai ricevuto il mio biglietto? - disse. - Non ti si trova mai. 

Aleksandr Vronskij, malgrado la sua vita dissipata e la sua fama di gran bevitore, era un perfetto gentiluomo di corte. 

Adesso, parlando col fratello di una cosa molto spiacevole per lui, sapendo che gli occhi di molti potevano esser fissi su di loro, assumeva un atteggiamento sorridente, come se scherzasse col fratello per cosa del tutto futile. 

- L'ho ricevuto e, davvero, non capisco di che mai tu voglia darti pensiero - disse Aleksej. 

- Mi preoccupo del fatto che proprio ora mi è stato fatto notare che non c'eri e che lunedì ti hanno incontrato a Petergof. 

- Ci sono delle cose che vanno giudicate solo da quelli che vi sono direttamente interessati, e proprio tale è la cosa di cui ti preoccupi tanto. 

- Sì, ma allora non si resta in servizio, non\ldots{} 

- Ti prego soltanto di non immischiarti. 

Il volto accigliato di Aleksej Vronskij si fece pallido, e la mascella inferiore sporgente tremò, il che accadeva di rado. Come uomo di cuore, di rado s'arrabbiava, ma quando si arrabbiava e gli tremava il labbro, allora, e Aleksandr Vronskij lo sapeva bene, era pericoloso. Aleksandr Vronskij sorrise allegro. 

- Io ti volevo unicamente consegnare la lettera della mamma. Rispondi a lei e non agitarti prima della corsa. Bonne chance - disse sorridendo, e si staccò da lui. 

Ma dopo di lui di nuovo un saluto amichevole lo fermò. 

- Non vuoi riconoscer gli amici! Buon giorno, mon cher! - cominciò a dire Stepan Arkad'ic brillando anche qui, fra lo splendore di Pietroburgo, non meno che a Mosca, col viso colorito e le fedine lucenti, ben ravviate. - Sono arrivato ieri, e sono molto contento di assistere al tuo trionfo. Quando ci vediamo? 

- Passa domani alla mensa - disse Vronskij e, strettagli, scusandosi, una manica del cappotto, si allontanò verso il centro dell'ippodromo dove facevano già entrare i cavalli per la grande corsa a ostacoli. 

I cavalli che avevano corso, sudati, sfiniti, accompagnati dagli stallieri, tornavano alla scuderia e, uno dopo l'altro, apparivano i cavalli per la corsa seguente, riposati, freschi, in gran parte inglesi, incappucciati, dal ventre asciutto, simili a strani enormi uccelli. Sulla destra conducevano Frou-Frou, magra e bella, che procedeva sulle giunture elastiche, piuttosto allungate, come su delle molle. Non lontano da lei toglievan la groppiera a Gladiator dalle orecchie lunghe. Le forme grandi, stupende, del tutto regolari dello stallone dal dorso magnifico e le giunture straordinariamente corte proprio al di sopra degli zoccoli, fermarono involontariamente l'attenzione di Vronskij. Voleva accostarsi alla sua cavalla, ma un amico lo trattenne di nuovo. 

- Ah, ecco Karenin! - gli disse l'amico col quale discorreva. - Cerca la moglie, e lei è al centro delle tribune. Non l'avete vista? 

- No, non l'ho vista - rispose Vronskij e, senza neppure voltarsi a guardare la tribuna nella quale gli avevano indicato la Karenina, si avvicinò alla cavalla. 

Vronskij non fece in tempo a osservare la sella per la quale aveva dato delle disposizioni, che chiamarono verso la tribuna i corridori per l'estrazione dei numeri. Diciassette ufficiali dal viso serio, severo, molti anche pallido, si ammassarono presso la tribuna ed estrassero i numeri. A Vronskij capitò il numero sette. Poi si udì: ``in sella!''. 

Avendo la sensazione di formare, insieme con gli altri che erano in gara, il centro dell'attenzione di tutti, Vronskij, in quel certo stato di tensione nel quale d'abitudine diveniva più calmo e lento nei movimenti, si avvicinò alla cavalla. Kord, in omaggio alle corse, si era messo l'abito di gala: un soprabito nero abbottonato, un solino inamidato che gli sosteneva le guance, un cappello tondo, nero, e gli stivaloni alla scudiera. Era, come sempre, calmo e grave e reggeva egli stesso tutte e due le briglie del cavallo, standogli ritto dinanzi. Frou-Frou continuava a tremare come se avesse la febbre. Il suo occhio pieno di fuoco guardava di traverso Vronskij che s'accostava. Vronskij le passò un dito nel sottopancia. La cavalla guardò ancor più di sbieco, mostrò i denti e drizzò l'orecchio. L'inglese fece una smorfia con le labbra, per esprimere un sorriso sul favorevole controllo alla sua abilità nel sellare. 

- Montate, sarete meno agitato. 

Vronskij si girò a guardare i suoi antagonisti per l'ultima volta. Sapeva che nella corsa non li avrebbe più visti. Due andavano avanti verso il luogo donde dovevano partire. Gal'cin, uno degli antagonisti temibili, amico di Vronskij, si aggirava intorno a uno stallone che non si lasciava montare. Un piccolo ussaro della guardia coi pantaloni stretti andava a galoppo, piegato come un gatto sul cavallo, per la mania di imitare gli inglesi. Il principe Kuzovlev montava, pallido, la sua giumenta purosangue della scuderia di Grabov, e un inglese la conduceva per il morso. Vronskij e tutti i suoi compagni conoscevano Kuzovlev, la sua particolare debolezza di nervi e il suo tremendo amor proprio. Sapevano che egli aveva paura di tutto, paura di montare un cavallo di classe; ma ora, proprio perché c'era da aver paura, proprio perché la gente si rompeva il collo e perché ad ogni ostacolo c'era un medico, l'ambulanza con la croce cucitavi sopra e una suora di carità, s'era deciso a correre. S'incontrarono con lo sguardo e Vronskij gli ammiccò con simpatia e approvazione. Soltanto uno non vide, l'antagonista principale, Machotin su Gladiator. 

- Non abbiate fretta - disse Kord a Vronskij - e ricordate una cosa sola: non la trattenete e non la spingete negli ostacoli; fatele fare quello che vuole. 

- Bene, bene - disse Vronskij, prendendo le redini. 

- Se è possibile, conducete voi la corsa, ma non perdete la speranza fino all'ultimo momento, anche foste in coda. 

La cavalla non fece in tempo a muoversi che Vronskij, con un movimento agile e forte, montò sulla staffa dentata d'acciaio e con disinvoltura e fermezza assestò il corpo ben fatto sulla sella di cuoio cigolante. Afferrando la staffa col piede destro, con gesto abituale, eguagliò fra le dita le redini che Kord lasciò scivolare dalle mani. Come non sapesse con quale zampa cominciare, Frou-Frou, distendendo col lungo collo le redini, si mosse come su delle molle, facendo oscillare il cavaliere sul dorso pieghevole. Kord, accelerando il passo, le teneva dietro. La cavalla, agitata, tirava le redini ora da una parte ora dall'altra, cercando di sfuggire al cavaliere, e Vronskij invano cercava di calmarla con la voce e con la mano. 

Si avvicinavano già al fiume sbarrato con la diga, in direzione del luogo dove avrebbero dato il via. Molti di quelli che prendevan parte alla gara erano avanti, molti indietro, quando a un tratto Vronskij udì dietro di sé, sul fango della via, il rumore del galoppare di un cavallo, e Machotin, sul suo Gladiator dalle orecchie lunghe e dalle zampe bianche, lo sorpassò. Machotin sorrise mostrando i denti lunghi, ma Vronskij lo guardò irritato. Non gli era simpatico, ora poi lo considerava il suo più temibile avversario, e gli dava fastidio il fatto che gli fosse passato accanto di galoppo, irritando la sua cavalla. Frou-Frou sollevò la zampa sinistra per mettersi al galoppo e fece due piccoli salti, ma, irritata dalla tensione delle redini, passò ad un trotto traballante che faceva sobbalzare il cavaliere. Anche Kord si accigliò e correva quasi per tener dietro a Vronskij. 

\capitolo{XXV}\label{xxv-1} 

Gli ufficiali che prendevano parte a questa corsa erano in tutto diciassette. La corsa doveva svolgersi su di un grande circuito a forma ellittica di quattro verste che si trovava dinanzi alla tribuna. Lungo questo circuito si trovavano nove ostacoli: un fiume, una grande barriera massiccia di circa due aršiny proprio davanti alla tribuna, un fosso asciutto e un altro con l'acqua, una scarpata, una banchina irlandese (uno degli ostacoli più difficili), che consisteva in un bastone ricoperto di ramaglie, dietro al quale, invisibile al cavallo, c'era ancora un fossato, così che il cavallo o doveva saltare tutti e due gli ostacoli insieme o ammazzarsi; poi ancora due fossati, uno con l'acqua e l'altro asciutto. Il traguardo era davanti alla tribuna. La corsa non iniziava dal circuito ma a cento sazeni da esso, di lato, e a questa distanza c'era il primo ostacolo, il fiume sbarrato da una diga di tre aršiny e mezzo di larghezza che i cavalieri potevano a loro piacere saltare o passare a guado. 

Per tre volte i cavalieri si misero in riga, ma ogni volta il cavallo di qualcuno ne usciva fuori e bisognava ricominciare daccapo. L'esperto di partenze, il colonnello Sestrin, cominciava già ad irritarsi, quando, finalmente, gridando per la quarta volta ``via'', i cavalieri si mossero. 

Tutti gli occhi, tutti i binocoli erano rivolti verso il gruppo multicolore dei cavalieri nel momento in cui si mettevano in riga. 

- Hanno dato il via, corrono! - si sentì da ogni parte, dopo il silenzio dell'attesa. 

E gli spettatori, a gruppi e isolati, cominciarono a correre da un posto all'altro per vedere meglio. Fin dal primo momento il gruppo dei cavalieri si allungò, e si vide come essi, a due a due, a tre a tre e uno dietro l'altro si avvicinassero al fiume. Agli spettatori pareva che fossero scattati tutti insieme; ma tra i corridori v'erano stati dei secondi di distacco che per loro avevano grande importanza. 

Frou-Frou, agitata e troppo nervosa, aveva perso il primo attimo, e alcuni cavalli si erano mossi prima di lei; ma ancor prima di arrivare al fiume Vronskij, trattenendo con tutte le forze il cavallo che tirava le briglie, ne sorpassò con facilità tre, e dinanzi a lui non rimase che Gladiator, il sauro di Machotin, che alzava con regolarità e leggerezza le zampe posteriori proprio davanti a Vronskij, e ancora, in testa a tutti, la splendida Diana che portava Kuzovlev più morto che vivo. 

Nei primi momenti Vronskij non riuscì a dominare se stesso, né la cavalla. Fino al primo ostacolo, il fiume, non poté dirigere i movimenti dell'animale. 

Gladiator e Diana si avvicinarono insieme e, quasi nello stesso momento, si sollevarono pari pari sul fiume e volarono dall'altra parte; inavvertita, quasi volando, Frou-Frou si sollevò dietro di loro; ma nello stesso attimo in cui Vronskij si sentiva sospeso in aria, vide quasi sotto le zampe della cavalla Kuzovlev che si dibatteva insieme a Diana sull'altra riva del fiume (Kuzovlev, dopo il salto, aveva abbandonato le briglie e il cavallo era capitombolato su di lui). Questi particolari Vronskij li venne a sapere dopo; in quell'attimo vide solo che proprio là dove sarebbero venute a cadere le zampe di Frou-Frou, poteva capitare una zampa o la testa di Diana. Ma Frou-Frou, come una gatta che cade, fece nel salto uno sforzo di zampe e di reni e, evitando il cavallo, galoppò oltre. 

``Oh, cara!'' pensò Vronskij. 

Dopo il fiume, Vronskij riacquistò il dominio pieno della cavalla e cominciò a trattenerla, pensando di saltare la grande barriera dietro a Machotin e di tentare di superarlo nella successiva distanza di duecento sazeni, non interrotta da ostacoli. 

La grande barriera era situata proprio dinanzi alla tribuna dello zar. L'imperatore e la corte e una folla di gente, tutti guardavano lui e Machotin, in testa per la lunghezza d'un cavallo, mentre si avvicinavano al ``diavolo'' (così veniva chiamata la barriera massiccia). Vronskij sentiva quegli sguardi rivolti su di lui da ogni parte, ma non vedeva nulla all'infuori della terra che gli correva incontro, e della groppa e delle zampe bianche di Gladiator che battevano veloci il tempo dinanzi a lui, rimanendo sempre alla stessa distanza. Gladiator saltò, senza urtare in nulla, agitò la coda e sparve agli occhi di Vronskij. 

- Bravo! - disse una voce. 

In quello stesso momento sotto gli occhi di Vronskij, proprio davanti a lui, balenarono le assi della barriera. Senza il più piccolo mutamento di andatura, la cavalla saltò sotto di lui; le assi scomparvero, ma dietro qualcosa picchiò. Eccitata da Gladiator che era in testa, la cavalla si era sollevata troppo presto sulla barriera e l'aveva urtata con lo zoccolo posteriore. Ma l'andatura non era mutata e Vronskij, nel ricevere in faccia uno schizzo di fango, capì che era sempre alla stessa distanza da Gladiator. Vide di nuovo dinanzi a sé la groppa, la coda corta e di nuovo quelle zampe bianche che si movevano rapide, ma senza allontanarsi. 

Proprio nel momento in cui Vronskij pensava di oltrepassare Machotin, Frou-Frou stessa, intuendone il pensiero, senza essere stimolata, accelerò notevolmente il galoppo, e cominciò ad avvicinarsi a Machotin dal lato più conveniente, cioè rasente la corda. Machotin però non lasciava andare la corda. Vronskij aveva appena pensato di oltrepassarlo dal lato esterno, che Frou-Frou aveva già cambiato piede e si era spinta ad oltrepassarlo proprio da questo lato. La spalla di Frou-Frou che aveva cominciato a scurirsi per il sudore, si portò alla stessa altezza del dorso di Gladiator. Per un po' galopparono insieme, ma davanti all'ostacolo al quale si avvicinavano, Vronskij, per non compiere un gran giro, si mise a lavorar di redini, e velocemente, sul pendio, oltrepassò Machotin. Vide di sfuggita la faccia di lui, inzaccherata di fango. Gli parve persino che sorridesse. Vronskij aveva superato Machotin, ma sentiva vicino e senza interruzioni, proprio dietro la schiena, il galoppo uguale e il respiro mozzato, ma ancora del tutto fresco, delle narici di Gladiator. 

I due ostacoli successivi, il fossato e la barriera, furono oltrepassati facilmente, ma Vronskij cominciò a sentire più vicini l'ansito e il galoppo di Gladiator. Lasciò andare la cavalla e con gioia sentì che essa con facilità aumentava l'andatura e che il suono degli zoccoli di Gladiator si faceva sentire di nuovo alla distanza di prima. 

Vronskij conduceva la corsa, cosa che egli stesso voleva fare e che gli aveva consigliato Kord, ed era ormai sicuro del successo. La sua agitazione, la gioia e la tenerezza per Frou-Frou divennero sempre maggiori. Voleva voltarsi indietro a guardare, ma non osava, e cercava di calmarsi e di non lanciare la cavalla per non sciupare in essa una riserva di forze eguale a quella che sentiva in Gladiator. Rimaneva un solo ostacolo e il più difficile: se egli l'avesse superato in testa, sarebbe giunto primo. Si avvicinava di galoppo alla banchina, e nello stesso momento tutti e due, lui e la cavalla, ebbero un attimo di esitazione. Egli notò nelle orecchie della cavalla indecisione e sollevò lo scudiscio, ma subito s'accorse che indecisione non c'era: la cavalla sapeva quello che occorreva fare. Accelerò l'andatura, e a tempo, proprio così come egli desiderava, si sollevò e, spintasi su da terra, si abbandonò alla forza d'inerzia che la trasportò lontano, di là dal fossato, e con la stessa cadenza, senza sforzo, senza cambiar passo, Frou-Frou riprese la corsa. 

- Bravo, Vronskij! - gli giunse da un gruppo di persone, ch'egli riconobbe come amici del reggimento, in piedi presso l'ostacolo. Non poté non distinguere la voce di Jašvin, ma non lo scorse. 

``Oh, tesoro mio!'' pensava di Frou-Frou, tendendo l'orecchio a quello che avveniva dietro di lui. ``Ha saltato'' pensò sentendo vicino il galoppo di Gladiator. Rimaneva solo l'ultimo fossato, pieno d'acqua e largo circa due aršiny. Vronskij non lo guardò neppure e, desiderando giungere di gran lunga primo, prese a lavorar di redini, alzando e abbassando la testa della cavalla. Sentiva che la cavalla sfruttava l'ultima riserva; non solo il collo e le spalle erano bagnati, ma sul garrese, sulla testa, sulle orecchie appuntite le veniva fuori il sudore, e aveva il respiro aspro e breve. Ma egli sapeva che questa riserva sarebbe stata più che sufficiente per gli ulteriori duecento sazeni. Solo da quel suo sentirsi più radente la terra e da quella particolare morbidezza dell'andatura, Vronskij poteva arguire di quanto la cavalla avesse aumentato la velocità. Essa volò sul fossato quasi senza avvedersene. Lo sorvolò come un uccello. Ma in quell'attimo stesso Vronskij sentì con orrore che, senza saper come, non era riuscito a secondare il movimento della cavalla, e, ricadendo pesantemente sulla sella, aveva fatto una mossa sbagliata, imperdonabile. E di colpo la sua posizione mutò ed egli sentì che qualcosa di spaventoso era accaduto. Prima ancora di rendersene conto gli balenarono di lato le zampe bianche dello stallone sauro, e Machotin gli passò dappresso a galoppo serrato. Vronskij si trovò a toccar terra con una gamba e la cavalla stava per abbattervisi sopra. Fece appena in tempo a liberar la gamba che quella cadde, riversa su di un fianco, rantolando pesantemente e facendo sforzi vani per rialzarsi con il sottile collo in sudore: come un uccello ferito a morte si dibatteva a terra ai piedi di lui. Il movimento malfatto di Vronskij le aveva spezzato le reni, ma egli lo capì molto tempo dopo. In quel momento vedeva solo che Machotin si allontanava veloce, e lui, barcollante, era rimasto solo sulla terra immota, fangosa; lì davanti, respirando greve, giaceva Frou-Frou che, piegando la testa verso di lui, lo guardava con i suoi occhi splendidi. Senza capire ancora quello che era avvenuto, Vronskij tirava la bestia per la briglia. Essa guizzò di nuovo tutta, come un pesciolino, facendo cricchiare le ali della sella; poggiò sulle zampe anteriori, ma non avendo la forza di sollevare il dorso, annaspò e cadde di nuovo sul fianco. Col volto sfigurato dall'emozione, pallido e col labbro inferiore che gli tremava, Vronskij la colpì col tacco nel ventre e prese di nuovo a tirarla per le briglie. Ma essa non si moveva e, ficcando il muso nel terreno, guardava il padrone con il suo sguardo parlante. 

- Aah! - muggì Vronskij, afferrandosi la testa. - Aah! Che ho fatto! - gridò. - E la corsa è perduta! E la colpa è mia, vergognosa, imperdonabile. E questa povera cara bestia perduta! Aah, che ho fatto! 

Un dottore e un infermiere, gli ufficiali del reggimento corsero, insieme con altra gente, verso di lui. Per sua disgrazia sentiva d'essere incolume e sano. La cavalla s'era spezzata la schiena, e fu deciso di finirla. Vronskij non poteva rispondere alle domande, non poteva parlare con nessuno. Si voltò e, senza raccattare il berretto che gli era saltato di testa, se ne andò via dall'ippodromo, non sapendo egli stesso dove. Si sentiva infelice. Per la prima volta in vita sua provava una pena così grande, così irreparabile, di cui la colpa era tutta sua. 

Jašvin lo raggiunse, portandogli il berretto e lo accompagnò fino a casa, e dopo mezz'ora Vronskij ritornò in sé. Ma il ricordo di questa corsa rimase per lungo tempo nell'animo suo come il ricordo più penoso e tormentoso della sua vita. 

\capitolo{XXVI}\label{xxvi-1} 

I rapporti esteriori di Aleksej Aleksandrovic con la moglie permanevano invariati. L'unica differenza consisteva nel fatto che egli era più occupato di prima. All'inizio della primavera andò all'estero per fare una cura di acque termali che ristabilisse la salute sua debilitata ogni anno dallo sforzo invernale. E, come al solito, tornò in luglio, e immediatamente, con aumentata energia, si dedicò alle occupazioni abituali. Come al solito sua moglie andò in campagna ed egli rimase a Pietroburgo. 

Dal tempo della conversazione avvenuta dopo la serata in casa della principessa Tverskaja, egli non aveva mai più parlato con Anna dei suoi sospetti e della sua gelosia; e quel suo solito tono di chi sente di essere qualcuno era quanto mai comodo per i presenti rapporti con la moglie. Era soltanto un po' freddo. Dava a vedere come fosse rimasta in lui una certa piccola scontentezza verso di lei per quella prima conversazione notturna ch'ella aveva voluto evitare. C'era pertanto nei suoi rapporti verso di lei, come un'ombra di dispetto, ma nulla di più. ``Tu non hai voluto avere una spiegazione - era come se le dicesse rivolgendosi a lei col pensiero - tanto peggio per te. Ormai sarai tu a pregarmene, ma io spiegazioni non ne darò. Tanto peggio per te'' diceva nel pensiero, come un uomo che abbia invano tentato di spegnere un incendio e, irritato contro i propri inutili sforzi, finisca col dire: ``Tanto peggio! Che bruci pure!''. 

Egli, intelligente e sottile negli affari di ufficio, non capiva tutta l'aberrazione di un simile comportamento. Non la capiva perché era troppo terribile per lui veder chiara la sua vera posizione, ed egli intanto nell'animo suo aveva nascosta, chiusa e sigillata quella tale cassetta nella quale si trovavano riposti i sentimenti suoi per la famiglia, per la moglie e per il figlio. Padre premuroso, dalla fine dell'inverno era diventato particolarmente freddo verso il figlio, e aveva verso di lui quello stesso tono canzonatorio che assumeva verso la moglie. ``Ohi, giovanotto'' gli diceva. 

Aleksej Aleksandrovic pensava e diceva che mai, come in quell'anno, aveva avuto tanto lavoro d'ufficio; e non voleva accorgersi che il lavoro se l'era creato lui stesso in quell'anno, che era stato uno dei mezzi per non aprire quella tale cassetta dove stavano rinchiusi i sentimenti suoi per la moglie e la famiglia: mentre il pensiero di costoro tanto più sgomentoso diveniva quanto più a lungo egli lo relegava là. E se qualcuno avesse avuto il diritto di chiedere ad Aleksej Aleksandrovic che cosa egli pensasse della condotta di sua moglie, quel pacifico, calmo Aleksej Aleksandrovic non avrebbe risposto nulla, e si sarebbe molto sdegnato contro la persona che gliene avesse chiesto. Proprio per questo vi era nell'espressione del viso di Aleksej Aleksandrovic qualcosa di sostenuto e di severo quando gli domandavano della salute di sua moglie. Aleksej Aleksandrovic non voleva pensare nulla circa la condotta di sua moglie, e realmente non ne pensava nulla. 

La dimora estiva consueta di Aleksej Aleksandrovic era a Petergof, dove abitualmente anche la contessa Lidija Ivanovna passava l'estate in compagnia e in continui rapporti con Anna. Quell'anno la contessa Lidija Ivanovna non aveva voluto soggiornare a Petergof, non era stata da Anna Arkad'evna neppure una volta, e aveva accennato ad Aleksej Aleksandrovic la sconvenienza dell'assiduità di Anna con Betsy e Vronskij. Aleksej Aleksandrovic le aveva chiuso la bocca, affermando con fermezza che sua moglie era al disopra di ogni sospetto; ma da allora aveva cercato di evitare la contessa Lidija Ivanovna. Non voleva vedere, e non vedeva che in società già molti guardavano di traverso sua moglie; non voleva capire e non capiva perché sua moglie insistesse per andare a Carskoe dove viveva Betsy e dove non sarebbe stata lontana dal campo del reggimento di Vronskij. Non si permetteva di pensare questo, e non lo pensava; tuttavia in cuor suo, pur senza dirselo mai, e pur senza averne non solo prova alcuna, ma neppure fondato sospetto, sapeva con certezza d'essere un marito ingannato, ed era per questo profondamente infelice. 

Quante volte durante i suoi otto anni di vita coniugale felicemente trascorsi, vedendo mogli infedeli e mariti ingannati, Aleksej Aleksandrovic si era detto: ``Ma come si può giungere a questo? Perché non troncare una situazione sconveniente?''. Ora, invece, che la disgrazia era piombata sul suo capo, non solo non pensava al modo di provvedere alla situazione, ma non voleva riconoscerla affatto, non voleva vederla, proprio perché era troppo penosa, troppo innaturale. 

Dal tempo del suo ritorno dall'estero, Aleksej Aleksandrovic era stato due volte in campagna. Una volta vi aveva pranzato, un'altra volta aveva passato la serata con ospiti, ma non vi aveva neanche una volta passato la notte, come era solito fare gli anni precedenti. 

Il giorno delle corse era un giorno pieno di lavoro per Aleksej Aleksandrovic; ma, predisposto fin dal mattino il programma della giornata, aveva deciso di andare, subito dopo colazione, in campagna dalla moglie, e di là alle corse, dove si sarebbe trovata tutta la corte e dove egli doveva andare. E dalla moglie sarebbe passato perché aveva deciso di andarle a far visita una volta alla settimana, per convenienza. Inoltre doveva consegnare alla moglie, proprio quel giorno che era il 15 del mese, secondo l'ordine da lui stabilito, il denaro per le spese. 

Dopo aver pensato tutto questo circa la moglie, con l'abituale dominio che aveva sui suoi pensieri, non permise loro di girovagare oltre, intorno a quanto la riguardava. 

La mattina fu tutta presa per Aleksej Aleksandrovic. Il giorno innanzi, la contessa Lidija Ivanovna gli aveva mandato un opuscolo di un noto viaggiatore della Cina, attualmente a Pietroburgo, con una lettera in cui lo pregava di ricevere il viaggiatore, uomo per varie considerazioni sempre interessante e utile. Aleksej Aleksandrovic non aveva fatto in tempo a leggere l'opuscolo la sera, e ne terminò la lettura la mattina. Dopo, s'erano presentati i consueti sollecitatori, erano cominciati i rapporti, i ricevimenti, le nomine, le rimozioni, le distribuzioni delle ricompense, delle pensioni, degli stipendi, la corrispondenza, quel lavoro quotidiano, infine, come lo chiamava Aleksej Aleksandrovic, che portava via tanto tempo. Poi c'erano state le occupazioni che lo riguardavano personalmente: la visita del dottore e dell'amministratore. L'amministratore non gli aveva preso molto tempo. Aveva consegnato solo il denaro necessario ad Aleksej Aleksandrovic ed aveva fatto un breve resoconto dello stato non troppo buono delle cose, giacché, in quell'anno, per i frequenti viaggi, si era speso di più, e c'era stato un certo dissesto. Ma il dottore, un celebre medico di Pietroburgo, che era in rapporti amichevoli con Aleksej Aleksandrovic, gli portò via molto tempo. Aleksej Aleksandrovic non lo aspettava quel giorno e fu stupito del suo arrivo e, ancor più, che il dottore lo interrogasse molto minutamente circa le sue condizioni di salute, gli ascoltasse il petto, picchiasse e tastasse il fegato. Non sapeva Aleksej Aleksandrovic che la sua amica Lidija Ivanovna, avendo notato che la salute di Aleksej Aleksandrovic quell'anno non era buona, aveva pregato il dottore di andare e di osservare il malato. ``Fatelo per me'' gli aveva detto la contessa Lidija Ivanovna. 

- Lo farò per la Russia, contessa - aveva risposto il dottore. 

- Un uomo inestimabile - aveva ribattuto la contessa Lidija Ivanovna. 

Il dottore era rimasto molto scontento di Aleksej Aleksandrovic. Aveva trovato il fegato notevolmente ingrossato, un certo esaurimento, nessun effetto della cura delle acque. Aveva ordinato molto esercizio fisico e poco sforzo intellettuale e, soprattutto, di guardarsi dai dispiaceri, il che per Aleksej Aleksandrovic era impossibile, così come è impossibile non respirare; e se n'era andato, lasciando in Aleksej Aleksandrovic la spiacevole consapevolezza che in lui qualcosa non andava e non si poteva aggiustare. 

Uscendo dalla camera di Aleksej Aleksandrovic il dottore si era imbattuto sulla scala con Šljudin, a lui ben noto, capogabinetto di Aleksej Aleksandrovic. Erano stati compagni di università e, sebbene si incontrassero di rado, si stimavano ed erano buoni amici; a nessuno perciò meglio che a Šljudin il dottore avrebbe detto tutta la sua sincera opinione sull'ammalato. 

- Come son contento che siate stato da lui - disse Šljudin. - Non sta bene, mi sembra. Che cos'ha? 

- Ecco, cos'ha - disse il dottore facendo un cenno al cocchiere di avanzare, al di sopra della testa di Šljudin. - Ecco vedete - disse il dottore prendendo nelle sue mani bianche il dito di un guanto di pelle e tirandolo. - Provate a spezzare una corda senza tenderla\ldots{} è molto difficile; tendetela invece fino all'estrema possibilità e poggiatevi sopra il peso di un dito\ldots{} si spezzerà. Per la sua assiduità, la sua scrupolosità nel lavoro, egli è teso fino all'estremo limite; e la pressione esterna c'è, e forte - concluse il dottore, aggrottando significativamente le sopracciglia. - Andate alle corse? - aggiunse, scendendo verso la carrozza che era stata fatta avanzare. - Sì, sì, s'intende, sarà una cosa lunga - rispose il dottore o rispose qualcosa di simile, a quello che aveva detto Šljudin e che egli non aveva afferrato. 

Dopo il dottore che gli aveva preso tanto tempo, si presentò il noto viaggiatore e Aleksej Aleksandrovic, profittando dell'opuscolo letto proprio allora e di una precedente conoscenza dell'argomento, stupì il viaggiatore con la profondità delle sue conoscenze e la larghezza delle sue vedute. 

Insieme al viaggiatore fu annunciato l'arrivo di un maresciallo della nobiltà giunto da poco a Pietroburgo e con il quale si doveva avere un colloquio. Dopo che questi se ne fu andato, dovette sbrigare le pratiche quotidiane col capo di gabinetto e dovette inoltre andare da un personaggio autorevole per un affare grave e importante. Aleksej Aleksandrovic fece appena in tempo a rientrare alle cinque, ora del suo pranzo, e dopo aver mangiato in compagnia del capogabinetto, lo invitò ad andare con lui in campagna e alle corse. 

Senza rendersene conto Aleksej Aleksandrovic cercava ormai l'occasione di avere una terza persona presente ai suoi incontri con la moglie. 

\capitolo{XXVII}\label{xxvii-1} 

Anna stava davanti allo specchio, appuntando, con l'aiuto di Annuška, l'ultimo nastro al vestito, quando sentì un rumore di ruote che calpestavano la ghiaia dell'ingresso. 

``Per Betsy è ancora presto - pensò e, guardando dalla finestra, vide una carrozza dalla quale uscirono il cappello nero e le ben note orecchie di Aleksej Aleksandrovic. - Che disdetta! Possibile che venga qui a passar la notte?'' e le parve così orribile e pauroso quello che poteva venirne fuori che, senza riflettere un attimo, gli uscì incontro col volto allegro e luminoso, e, sentendo vivo in sé lo spirito della menzogna e dell'inganno, subito vi si abbandonò, e cominciò a parlare senza sapere neppure lei cosa diceva. 

- Oh, come ciò è gentile! - disse, dando la mano al marito e salutando con un sorriso Šljudin che era persona di casa. - Passerai la notte qua, spero? - le suggerì per prima cosa lo spirito dell'inganno - e ora andiamo insieme alle corse. Peccato che abbia già promesso a Betsy. Deve passare a prendermi. 

Aleksej Aleksandrovic si accigliò al nome di Betsy. 

- Oh, non starò a separare le inseparabili - disse col suo abituale tono canzonatorio. - Andrò con Michail Vasil'evic. Anche i dottori mi ordinano di camminare. Farò la strada a piedi e immaginerò di essere alla stazione termale. 

- Non c'è bisogno di affrettarsi - disse Anna. - Volete il tè? - e sonò. 

- Servite il tè, e dite a Serëza che Aleksej Aleksandrovic è qui. Be', come va la tua salute? Michail Vasil'evic, voi non siete mai stato da me; guardate come si sta bene sul mio balcone - diceva rivolgendosi ora all'uno, ora all'altro. 

Parlava con semplicità e naturalezza, ma troppo e troppo in fretta. Lo sentiva lei stessa, tanto più che, nello sguardo incuriosito col quale la guardava Michail Vasil'evic, notava di essere osservata. 

Michail Vasil'evic uscì subito sulla terrazza. 

Anna sedette accanto al marito. 

- Non hai un buon aspetto - disse. 

- Già - disse egli - oggi il dottore è stato da me e mi ha portato via un'ora di tempo. Sospetto che qualcuno dei miei amici me lo abbia mandato: la mia salute è così preziosa\ldots{} 

- E cosa ha detto? 

Ella gli chiedeva della sua salute e delle sue occupazioni, voleva convincerlo a riposarsi e venire a stare da lei. 

Tutto questo lo diceva con vivacità, in fretta e con un particolare luccichio negli occhi; ma Aleksej Aleksandrovic, ora, non rilevava questo tono. Ascoltava le parole e dava loro solo il significato che avevano. E le rispondeva semplicemente, sia pure scherzando. In questa conversazione non ci fu nulla di particolare, ma in seguito Anna non poté mai ricordare questa breve scena senza un tormentoso senso di vergogna. 

Entrò Serëza, preceduto dalla governante. Se Aleksej Aleksandrovic avesse permesso a se stesso di osservare avrebbe notato lo sguardo timido, spaurito col quale Serëza guardava il padre e poi la madre. Ma egli non voleva vedere, e non vedeva. 

- Ohi, giovanotto! È cresciuto. Davvero, sta diventando un uomo. Buongiorno, giovanotto. 

E diede la mano a Serëza spaventato. 

Serëza, anche prima timido nei rapporti col padre, ora, da quando Aleksej Aleksandrovic aveva preso a chiamarlo giovanotto e da quando gli si era posto nella mente il dilemma se Vronskij fosse un amico o un nemico, sfuggiva il padre. Quasi per averne protezione, si rivolse alla madre. Con la madre stava bene quando era sola. Intanto Aleksej Aleksandrovic, parlando con la governante, teneva il figlio per la spalla. Serëza era così tormentosamente a disagio che Anna si accorse che stava lì lì per piangere. 

Anna, che era diventata rossa nel momento in cui era entrato il figlio, ne notò il disagio, si alzò in fretta, tolse la mano di Aleksej Aleksandrovic dalla spalla del figlio e, dopo averlo baciato, lo condusse sulla terrazza; e subito dopo rientrò. 

- Ma è già ora - disse, guardando l'orologio - come mai Betsy non viene?\ldots{} 

- Già - disse Aleksej Aleksandrovic e, alzatosi, intrecciò le mani e le fece scricchiolare. - Sono passato da te, anche per portarti del denaro, dal momento che gli usignuoli non vivono di fiabe - disse. - Ti occorre, penso. 

- No, non mi occorre\ldots{} sì, mi occorre - diss'ella senza guardarlo e arrossendo fino alla radice dei capelli. - Ma tu passerai di qua, spero, tornando dalle corse. 

- Oh, sì - rispose Aleksej Aleksandrovic. - Ed ecco ora la stella di Petergof, la principessa Tverskaja - soggiunse dopo aver guardato dalla finestra il tiro inglese con le bardature, che si avvicinava con una minuscola carrozzetta straordinariamente alta. - Che eleganza! Un incanto! Su, allora, andiamo anche noi. 

La principessa Tverskaja non uscì dalla vettura, ma solo il servitore in ghette, pellegrina e cappello nero, saltò giù all'ingresso. 

- Io vado, addio - disse Anna e, baciato il figlio, si avvicinò ad Aleksej Aleksandrovic, dandogli la mano. 

- Su, allora, arrivederci. Tu passerai a prendere il tè, benissimo! - ella disse, e uscì splendente e gaia. Ma appena non lo vide più, sentì sulla mano il punto preciso che le labbra di lui avevano toccato e rabbrividì di disgusto. 

\capitolo{XXVIII}\label{xxviii-1} 

Quando Aleksej Aleksandrovic apparve alle corse, Anna era già seduta nella tribuna accanto a Betsy, in quella tribuna dove era raccolta tutta l'alta società. Scorse il marito da lontano. Due esseri, il marito e l'amante, erano per lei i due centri della sua vita, ed ella ne avvertiva la vicinanza senza bisogno dei sensi esterni. Avvertì ancora da lontano l'avvicinarsi del marito, e suo malgrado lo seguì nella marea di folla fra la quale avanzava. Lo vide avvicinarsi alla tribuna, ora rispondendo con indulgenza ai saluti adulatori, ora salutando con cordialità e distrazione i colleghi, ora aspettando con desiderio lo sguardo dei potenti e sollevando il gran cappello tondo che gli premeva l'estremità delle orecchie. Conosceva tutti gli atteggiamenti di lui, e tutti le erano odiosi. ``Soltanto falsità, soltanto ambizione, ecco tutto quello che c'è nell'animo suo - pensava - e le idee di ordine superiore, l'amore per la cultura, la religione, tutte queste cose non sono altro che mezzi per affermarsi''. 

Dalla direzione del suo sguardo verso la tribuna delle signore (egli guardava dritto in questa, ma non riconosceva la moglie in quel mare di stoffe, nastri, piume, ombrellini e fiori), ella capì che la cercava ma finse di non accorgersene. 

- Aleksej Aleksandrovic! - gli gridò la principessa Betsy - voi probabilmente non vedete vostra moglie: eccola! 

Egli sorrise col suo sorriso freddo. 

- Qui c'è tanto splendore che gli occhi ne restano abbagliati - disse, e andò verso la tribuna. Sorrise alla moglie, come deve sorridere un marito che ritrova la moglie dopo averla vista un momento prima, e salutò la principessa e gli altri amici, dando a ciascuno il suo, scherzando, cioè, con le signore e scambiando dei convenevoli con gli uomini. Giù, accanto alla tribuna, stava in piedi un generale, un aiutante di campo che Aleksej Aleksandrovic stimava e che era noto per il suo ingegno e la sua cultura. Aleksej Aleksandrovic si mise a discorrere con lui. 

C'era intervallo fra una corsa e l'altra, e perciò nulla disturbava la loro conversazione. Il grande generale deprecava le corse. Aleksej Aleksandrovic ribatteva, prendendone le difese. Anna ascoltava la voce stridula, eguale di lui, senza perderne neppure una parola, e ogni parola le sembrava falsa e le colpiva dolorosamente l'orecchio. 

Quando cominciò la corsa a ostacoli su quattro verste, ella si sporse in avanti e, gli occhi fissi su Vronskij, prese a seguirlo mentre si avvicinava al cavallo e lo montava, e nello stesso tempo ascoltava l'odiosa, instancabile voce del marito. Era tormentata dal timore per Vronskij, ma ancora più dalla instancabile voce stridula del marito della quale conosceva tutte le intonazioni. 

``Sono una donna cattiva, sono una donna perduta - pensava - ma non mi piace mentire e non sopporto la menzogna, mentre Aleksej Aleksandrovic si pasce di menzogna. Egli sa tutto, vede tutto; che cosa mai c'è in lui, dunque, se può così tranquillamente parlare? Uccidesse me, uccidesse Vronskij, lo stimerei. Ma no, a lui bastano soltanto la menzogna e il rispetto delle convenienze'' si diceva Anna, senza pensare con precisione a quello che avrebbe voluto che il marito facesse, e sotto qual luce avrebbe voluto vederlo. Non capiva che anche quell'eccessiva verbosità di Aleksej Aleksandrovic, che tanto la irritava, era, in quel momento, l'espressione dell'inquietudine e dell'intima agitazione di lui. Come un bambino che, dopo aver urtato in qualche cosa, mette in moto, saltando, i propri muscoli per soffocare il dolore, così Aleksej Aleksandrovic aveva bisogno di un moto intellettuale per soffocare quei suoi pensieri sulla moglie che ora, alla presenza di lei e alla presenza di Vronskij, e alla continua ripetizione del nome di lui, urgevano perché si prestasse loro attenzione. E come al bambino vien naturale di saltare, così a lui veniva fatto di parlare bene e con intelligenza. Egli diceva: 

- Il pericolo nelle corse dell'arma di cavalleria, è un rischio che non si può eliminare in ogni corsa. Se l'Inghilterra può vantare nella sua storia militare le più brillanti azioni della cavalleria, è solo grazie al fatto che essa ha sviluppato, evolvendola nella storia, questa forza e di animali e di uomini. Lo sport, secondo la mia opinione, ha un grande valore, e, come sempre, noi ne vediamo soltanto il lato più superficiale. 

- Non tanto superficiale - disse la principessa Tverskaja. - Un ufficiale, dicono, si è rotto le costole! 

Aleksej Aleksandrovic sorrise col suo sorriso che gli scopriva soltanto i denti ma che non diceva nulla. 

- Ammettiamo, principessa, che questo non sia superficiale - egli disse - ma profondo. Ma non è qui la questione - ed egli si rivolse di nuovo al generale col quale parlava seriamente. - Non dimenticate che corrono dei militari i quali hanno scelto questa attività, e convenite che ogni attività ha il rovescio della medaglia. Questo rientra proprio nei doveri del militare. Lo sport scandaloso del pugilato o delle corride spagnole è un segno di barbarie. Ma uno sport specializzato è un segno di progresso. 

- No, non ci verrò più; tutto questo mi agita troppo - diceva la principessa Betsy. - Non è vero, Anna? Agita, ma non se ne possono distaccare gli occhi. Se fossi stata una romana, non avrei tralasciato un solo spettacolo del circo. 

Anna non parlava e senza abbandonare il binocolo, guardava in un punto solo. 

In quel momento, attraverso la tribuna, passò un ufficiale di alto grado. Interrotto il discorso, Aleksej Aleksandrovic si alzò in fretta, ma con dignità, e salutò profondamente l'ufficiale che passava. 

- Voi non correte? - gli disse, scherzando, l'ufficiale. 

- La mia corsa è più difficile - rispose rispettoso Aleksej Aleksandrovic. 

E sebbene la risposta non significasse nulla, l'ufficiale fece finta di aver colto una battuta di spirito intelligente, detta da un uomo d'ingegno e di aver capito in pieno la pointe de la sauce. 

- Qui vi sono due categorie di persone - riprese a dire Aleksej Aleksandrovic - quella dei partecipanti e quella degli spettatori. L'amore per questi spettacoli è il segno più sicuro del basso livello degli spettatori, ne convengo, ma\ldots{} 

- Principessa, una scommessa! - si sentì da basso la voce di Stepan Arkad'ic che si rivolgeva a Betsy. - Per chi tenete? 

- Io e Anna per il principe Kuzovlev - rispose Betsy. 

- Io per Vronskij. Un paio di guanti. 

- Vada pure! 

- Che bello spettacolo, non è vero? 

Aleksej Aleksandrovic tacque per un po' finché non finirono di parlare intorno a lui, ma poi ricominciò subito. 

- Ne convengo, non sono giuochi da uomini - e voleva continuare. 

Ma intanto davano il via ai cavalieri, e tutte le conversazioni cessarono. Aleksej Aleksandrovic tacque anche lui e tutti si alzarono e si volsero verso il fiume. Aleksej Aleksandrovic non si interessava alle corse e perciò non badava a quelli che correvano, ma distrattamente cominciò a girare intorno, sugli spettatori, i suoi occhi stanchi. Il suo sguardo si fermò su Anna. 

Il viso di lei era pallido e teso: ella evidentemente non vedeva niente e nessuno, tranne uno. Tratteneva il respiro, e la sua mano stringeva convulsa il ventaglio. Aleksej Aleksandrovic la guardò e si voltò in fretta a osservare altri visi. 

``Ma ecco, anche questa signora e le altre ancora sono agitate; ciò è molto naturale'' si diceva Aleksej Aleksandrovic. Non voleva guardare più; ma gli occhi erano involontariamente attratti verso di lei. Esaminava quel viso sforzandosi di non leggervi ciò che così chiaramente vi era scritto; e contro la sua volontà vi leggeva con terrore quello che non voleva sapere. 

La prima caduta di Kuzovlev nel compiere il salto del fiume impressionò tutti, ma Aleksej Aleksandrovic vide chiaramente sul pallido viso trionfante di Anna che quegli ch'ella seguiva non era caduto. Quando poi Machotin e Vronskij ebbero saltato la barriera, e l'ufficiale che veniva dopo cadde con la testa in giù e si abbatté come morto e un brivido di orrore percorse tutto il pubblico, Aleksej Aleksandrovic vide che Anna non aveva neppure notato questo, e che a stento capiva di che si parlasse intorno. E la osservava sempre di più e con maggiore ostinazione. Anna, tutta presa dalla vista di Vronskij che correva, sentiva di lato lo sguardo freddo del marito fisso su di lei. 

Si voltò per un attimo, lo fissò interrogativamente e, accigliandosi lievemente, si girò di nuovo. 

``Oh, non mi importa più!'' era come se gli avesse detto e non guardò più neppure una volta. 

La corsa fu disgraziata e su diciassette persone ne caddero e si fecero male più della metà. 

Alla fine delle corse tutti erano in uno stato di agitazione, tanto più che il sovrano se ne era mostrato scontento. 

\capitolo{XXIX}\label{xxix-1} 

Tutti esprimevano ad alta voce la loro disapprovazione, tutti ripetevano la frase messa in giro da qualcuno: ``non ci manca che il circo con i leoni''. Il terrore era sentito da tutti, sì che quando Vronskij cadde ed Anna emise un gemito, non ci fu nulla di straordinario. Ma subito dopo nel volto di Anna apparve un turbamento già troppo sconveniente. S'era smarrita del tutto; si dibatteva come un uccello al laccio; ora voleva alzarsi e andare chi sa dove, ora si volgeva a Betsy. 

- Andiamo, andiamo - diceva. 

Ma Betsy non l'ascoltava. Parlava, sporgendosi in giù, con un generale che le si era avvicinato. 

Aleksej Aleksandrovic si avvicinò ad Anna e le porse cortesemente la mano. 

- Andiamo, se vi fa piacere - disse in francese, ma Anna era intenta ad ascoltare quello che diceva il generale e non si curò del marito. 

- Anche lui si è rotto una gamba, dicono - diceva il generale. - Ma che senso c'è in tutto questo? 

Anna, senza rispondere al marito, aveva sollevato il binocolo e guardava il punto dove era caduto Vronskij: ma era così lontano e vi si era affollata così tanta gente che nulla di distingueva. Abbassò il binocolo e fece per andarsene; ma in quel momento giunse un ufficiale a cavallo a riferire qualcosa allo zar. Anna si sporse in avanti per ascoltarlo. 

- Stiva! Stiva! - gridò al fratello. 

Ma il fratello non la udì. Ella di nuovo voleva andar via. 

- Vi offro ancora una volta il braccio, se volete andare - disse Aleksej Aleksandrovic, toccandole il braccio. 

Ella si scostò da lui con ribrezzo e, senza guardarlo in viso, rispose: 

- No, no, lasciatemi, rimango. 

Vedeva adesso che dal punto dove era caduto Vronskij correva, attraversando tutto il circuito, un ufficiale diretto alla tribuna. Betsy gli faceva cenno col fazzoletto. L'ufficiale portò la notizia che il cavaliere era salvo, ma il cavallo si era rotto la schiena. 

Udito questo, Anna si sedette di colpo e si coprì il viso col ventaglio. Vedendo che ella piangeva e che, non solo non riusciva a trattenere le lacrime, ma neanche i singhiozzi che le sollevavano il petto, Aleksej Aleksandrovic la coprì con la propria persona, dandole il tempo di rimettersi. 

- Per la terza volta vi offro il mio braccio - disse dopo un po' di tempo, rivolgendosi a lei. Anna lo guardava e non sapeva cosa dire. La principessa Betsy venne in suo aiuto. 

- No, Aleksej Aleksandrovic, ho accompagnato io Anna, e io ho promesso di riaccompagnarla - s'intromise. 

- Perdonatemi, principessa - egli disse, sorridendo con cortesia, ma guardandola fermo negli occhi - io vedo che Anna non sta del tutto bene e desidero che venga con me. 

Anna si voltò a guardarlo spaventata, si alzò sottomessa e poggiò la mano sul braccio del marito. 

- Manderò da lui, m'informerò e poi farò sapere - le sussurrò Betsy. 

All'uscita della tribuna, Aleksej Aleksandrovic, come sempre, parlava con quelli che incontrava e Anna doveva come sempre rispondere e parlare; ma era proprio fuori di sé e come in sogno andava sotto il braccio del marito. 

``Si è ammazzato o no? È vero? Verrà o no? Lo vedrò stasera?'' pensava. 

In silenzio prese posto nella vettura di Aleksej Aleksandrovic e in silenzio rimase anche quando si furono allontanati dalla calca degli equipaggi. Malgrado tutto quello che aveva visto, Aleksej Aleksandrovic non si permetteva di pensare alla reale posizione della moglie. Egli coglieva solo i segni esteriori, aveva visto ch'ella si comportava in modo poco conveniente, e riteneva suo dovere dirglielo. Ma era molto difficile non dire nulla di più, dirle soltanto questo. Aprì la bocca per dirle che si era comportata in modo sconveniente, e invece, senza volere, disse tutt'altra cosa. 

- Ma come siamo tutti inclini a questi spettacoli feroci - disse. - Io noto\ldots{} 

- Cosa? Non capisco - disse Anna in tono sprezzante. 

Egli si offese e cominciò subito a dirle quello che voleva. 

- Devo dirvi\ldots{} - cominciò. 

``Eccola, la spiegazione'' pensò lei, e n'ebbe paura. 

- Devo dirvi che vi siete comportata in modo del tutto sconveniente - egli disse in francese. 

- In che cosa mi sono comportata in modo sconveniente? - ella disse forte, voltandosi rapida verso di lui e guardandolo dritto negli occhi, non più con quella sua allegria mordace di prima, ma con un'aria decisa che nascondeva a stento il terrore provato. 

- Non dimenticate - egli disse, indicandole il vetro aperto di contro al cocchiere. 

E si alzò e lo tirò su. 

- Che cosa avete trovato di sconveniente? - ella ripeté. 

- Quella disperazione che non avete saputo nascondere per la caduta di uno dei cavalieri. 

S'aspettava che ella ribattesse. Ma ella taceva, guardando davanti a sé. 

- Vi ho già pregata di comportarvi in modo che anche le male lingue non abbiano a dire nulla contro di voi. Un tempo vi ho parlato di rapporti interiori; ora non ne parlo più. Ora vi parlo solo dei rapporti esteriori. Vi siete comportata in modo sconveniente, e desidero che ciò non si ripeta. 

Ella non sentiva nemmeno metà delle sue parole; aveva paura di lui, ma intanto pensava se era vero che Vronskij non era rimasto ucciso. Era di lui che dicevano che era rimasto illeso, mentre il cavallo s'era spezzata la schiena? Appena egli ebbe finito di parlare, ella sorrise in quella sua maniera beffarda e falsa, e non rispose perché non aveva sentito quello che aveva detto. Aleksej Aleksandrovic allora riprese a parlare arditamente, ma appena ebbe coscienza di quello che diceva, il terrore di Anna si comunicò a lui. Notò quel riso, e una strana aberrazione lo prese. 

``Ride dei miei sospetti. Ecco, ora mi dirà subito quello che ha già detto l'altra volta: che i miei sospetti sono infondati, che tutto ciò è ridicolo''. 

Ora che era sospesa su di lui la scoperta di tutto, nulla desiderava tanto quanto ch'ella rispondesse beffarda, così come l'altra volta, che i suoi sospetti erano infondati e ridicoli. Così spaventoso era quello che sapeva che era pronto a credere a tutto. Ma l'espressione del viso di lei, atterrito e torvo, non prometteva ora neppure l'inganno. 

- Forse io mi sbaglio - disse. - In tal caso vogliate perdonarmi. 

- No, non vi siete sbagliato - ella disse lentamente, guardando con disperazione il suo viso impassibile. - Voi non vi siete sbagliato. Sono sconvolta e non posso non esserlo ancora. Io ascolto voi, e penso a lui. Io amo lui, sono la sua amante, e non posso più resistere. Ho paura, vi odio\ldots{} Fate di me quel che volete. 

E riversatasi all'indietro in un angolo della carrozza, scoppiò in singhiozzi, coprendosi il viso con le mani. Aleksej Aleksandrovic non si mosse e non mutò la direzione del suo sguardo, fisso davanti a sé. Ma tutto il suo viso prese ad un tratto l'immobilità solenne di un cadavere e questa espressione permase tale per tutto il tempo del percorso fino alla villa. Avvicinandosi alla casa, egli girò il capo verso di lei, sempre con la stessa espressione del viso. 

- Già, ma io pretendo l'osservanza delle forme esteriori fino al momento in cui - e qui la voce gli tremò - non avrò prese le misure necessarie per difendere il mio onore e ve le avrò comunicate. 

Uscì dalla carrozza e l'aiutò a discendere. In presenza della servitù le strinse in silenzio la mano, risalì in vettura e partì per Pietroburgo. 

Subito dopo venne un cameriere da parte della principessa Betsy e recò un biglietto per Anna. 

``Ho mandato da Aleksej per sapere della sua salute, ed egli mi scrive che è sano e salvo, ma desolato''. 

``Allora verrà - pensò. - Come ho fatto bene a dirgli tutto!''. 

Guardò l'orologio. Mancavano ancora tre ore, e il ricordo dei particolari dell'ultimo incontro le accese il sangue. 

``Dio mio, come è chiaro ancora! È terribile, ma io amo vederlo quel suo viso, e amo questa luce fantastica\ldots{} Mio marito, ah, già\ldots{} Ma, grazie a Dio, con lui tutto è finito''. 

\capitolo{XXX}\label{xxx-1} 

Come in tutti i luoghi dove si riunisce gente varia, così pure nella piccola stazione termale tedesca dove erano arrivati gli Šcerbackij, si era venuta a formare quella tale, per così dire, cristallizzazione della società che ad ogni suo membro fissa un posto definito e immutabile. Ogni nuovo personaggio che arrivava nel luogo di cura, si fissava nel posto che gli era proprio, così come una goccia d'acqua riceve dal freddo, definita e immutabile, una determinata forma di ghiacciuolo. 

Fürst Šcerbackij sammt Gemahlin und Tochter per il nome e per l'appartamento che occupavano e per gli amici che avevano trovato, si cristallizzarono nel loro posto definito e ad essi destinato. 

Quell'anno, alla stazione termale, c'era un'autentica Fürstin tedesca; perciò la ``cristallizzazione'' si operava in maniera ancor più rigida. La principessa Šcerbackaja volle assolutamente presentare sua figlia alla principessa di sangue reale, e fin dal giorno dopo l'arrivo compì questo rito. Kitty fece una profonda e graziosa riverenza nel suo vestito estivo molto semplice, e perciò molto elegante, ordinato a Parigi. La principessa reale disse: ``Spero che le rose torneranno presto a fiorire sul quel bel visino''. E da quel momento per gli Šcerbackij si fissò saldamente, e subito, un determinato tenore di vita al quale non era possibile sottrarsi. Fecero amicizia con la famiglia di una lady inglese, con una contessa tedesca e con il figlio che era stato ferito nell'ultima guerra, con uno scienziato svedese e con mr. Canut e la sorella. Ma la compagnia degli Šcerbackij si compose soprattutto, involontariamente, di una signora di Mosca, Mar'ja Evgenevna Rtišceva, con la figlia (che non piaceva a Kitty perché si era ammalata di amore come lei), e di un colonnello moscovita che Kitty ricordava fin dall'infanzia, in divisa e spalline, ma che qui, con i suoi piccoli occhi e il collo scoperto, la cravattina a colori, era straordinariamente ridicolo; e noioso poi, perché non si riusciva a liberarsi di lui. Quando tutto si assestò in modo preciso, Kitty cominciò ad annoiarsi molto, ancor più perché il principe era partito per Karlsbad, ed ella era rimasta sola con la madre. Non si interessava a quelli che conosceva, perché sentiva che non ne avrebbe cavato nulla di nuovo. Il suo più grande e intimo interesse consisteva invece nell'osservare quelli che non conosceva, o nel fare supposizioni circa il loro carattere. Per una particolare inclinazione del suo carattere, Kitty supponeva sempre negli altri quanto può esserci di più bello, e soprattutto in coloro che non conosceva. E ora, fantasticando così intorno alle persone, ai rapporti che intercorrevano fra di loro ed alla loro appartenenza a una o all'altra categoria, Kitty si figurava i più belli e meravigliosi caratteri, e cercava riconferma alle sue supposizioni. 

Tra queste persone le interessava in modo particolare una ragazza russa, arrivata al luogo di cura con una signora russa ammalata, la signora Stahl, come la chiamavano tutti. La signora Stahl apparteneva al gran mondo, ma era così malata da non poter camminare, e la si vedeva alle acque soltanto in qualche rara bella giornata, portatavi in una carrozzina. Ma non tanto per la malattia, quanto per alterigia, così spiegava la principessa, la signora Stahl non trattava nessuno dei russi. 

La ragazza russa curava la signora Stahl e, oltre a ciò, Kitty aveva notato ch'ella andava d'accordo con tutti i malati gravi, che erano ben numerosi nella stazione termale, e si occupava di loro con grande semplicità. Questa ragazza, secondo le supposizioni di Kitty, non era parente della signora Stahl, ma non era nemmeno un'infermiera retribuita. La signora Stahl la chiamava Varen'ka e gli altri la chiamavano ``m.lle Varen'ka''. Kitty amava non solo fantasticare intorno ai rapporti fra questa ragazza e la signora Stahl e le altre persone a lei sconosciute, ma provava, come talvolta accade, un'istintiva simpatia per m.lle Varen'ka e sentiva, nei loro sguardi che s'incontravano, d'esserne ricambiata. M.lle Varen'ka, pur essendo certamente giovane, sembrava un essere senza giovinezza: le si potevano dare diciannove o trenta anni indifferentemente. Malgrado il suo colorito malato, a giudicare dai tratti del viso, era piuttosto bella che brutta. E poteva sembrar anche ben fatta se non vi fosse stata in lei un'eccessiva magrezza del corpo e una certa sproporzione tra la testa e la sua figura di media altezza; ma certamente non poteva piacere agli uomini. Somigliava a un bellissimo fiore ancor pieno di petali, ma già sfiorito, senza profumo. Non poteva piacere agli uomini, anche perché le mancava quello che abbondava in Kitty: un fuoco di vitalità contenuta e la coscienza del proprio fascino. 

Sembrava tutta raccolta in qualche cosa di cui fosse certa in modo assoluto e non potesse pertanto interessarsi a nulla che ne fosse al di fuori. Ciò contrastava con quello che era nell'animo di Kitty e attirava questa verso di lei. Kitty sentiva che nell'altra, nel suo modo di vivere, avrebbe trovato un esempio di quanto ora tormentosamente cercava: l'interesse alla vita, il valore della vita all'infuori e al di là delle relazioni mondane tra una ragazza e gli uomini, relazioni che erano ormai odiose a Kitty e che le apparivano come un'umiliante esposizione di merce in attesa del compratore. Quanto più Kitty osservava l'amica sconosciuta, tanto più si convinceva che questa ragazza era proprio l'essere perfetto ch'ella immaginava, e tanto più desiderava di conoscerla. 

Le due ragazze si incontravano varie volte al giorno, e ad ogni incontro gli occhi di Kitty dicevano: ``Chi siete? Cosa mai siete? È vero che siete quell'essere delizioso che io mi figuro? Ma, per amor di Dio, non pensate - aggiungeva il suo sguardo - che io mi permetta di imporvi la mia conoscenza. Vi ammiro semplicemente e vi voglio bene''. ``Io pure vi voglio bene e voi siete molto carina. E vi vorrei ancora più bene, se avessi tempo'' rispondeva lo sguardo della ragazza sconosciuta. E invero Kitty vedeva ch'ella era sempre occupata: accompagnava fuori i bambini di una famiglia russa, o portava uno scialle per la malata e ve l'avviluppava dentro, o cercava di distrarre un malato inasprito, sceglieva e comprava per qualcuno i pasticcini per il caffè. 

Ben presto, dopo l'arrivo degli Šcerbackij alla stazione termale, apparvero altri due personaggi che attirarono l'attenzione, poco benevola, di tutti. Erano: un uomo alto, un po' curvo, con delle mani enormi, un cappotto corto non fatto su misura, degli occhi neri ingenui e insieme terribili, e una donna graziosa, butterata, vestita male e senza gusto. Riconosciute queste persone per russi, Kitty aveva cominciato a comporre su di loro, nella sua immaginazione, un bellissimo e commovente romanzo. Ma la principessa, scoperto nella Kurliste che erano Levin Nikolaj e Mar'ja Nikolaevna, spiegò a Kitty quale pessimo soggetto fosse questo Levin, e tutti i sogni su questi due esseri scomparvero. Non tanto perché la madre glielo avesse detto, quanto per il fatto che si trattava del fratello di Konstantin, queste due persone parvero a Kitty molto antipatiche. Anzi questo Levin, con quella sua abitudine di scuotere il capo, suscitava addirittura in lei un senso di repulsione. 

Le sembrava che in quei due grandi occhi terribili che la seguivano ostinatamente, ci fosse un sentimento di odio e di irrisione, ed ella cercava di evitare un incontro con lui. 

\capitolo{XXXI}\label{xxxi-1} 

Era una brutta giornata, la pioggia era caduta per tutta la mattina e i malati si affollavano con gli ombrelli sotto il portico. 

Kitty passeggiava insieme con la madre e il colonnello moscovita, che faceva allegramente l'elegantone con il suo soprabito all'europea, comprato già bell'e fatto a Francoforte. Camminavano da un lato del porticato, cercando di evitare Levin che camminava nell'altro senso. Varen'ka con il suo abito scuro, il cappello nero dalla falda ripiegata in giù, accompagnava una francese cieca lungo tutto il porticato, e ogni volta che s'incontrava con Kitty, scambiava con lei uno sguardo di simpatia. 

- Mamma, posso rivolgerle la parola? - disse Kitty, seguendo con gli occhi l'amica sconosciuta che si avviava alla fonte dove avrebbero potuto incontrarsi. 

- Se lo desideri tanto, prenderò informazioni sul suo conto, e l'avvicinerò io stessa - rispose la madre. - Cosa ci trovi di particolare? Deve essere una dama di compagnia. Se vuoi farò conoscenza con la signora Stahl. Conosco la sua belle-soeur - aggiunse la principessa, sollevando con orgoglio il capo. 

Kitty sapeva che la principessa era offesa dal fatto che la signora Stahl sembrava evitare di fare la sua conoscenza. 

- È un incanto, com'è cara! - ella disse, guardando Varen'ka, nel momento in cui porgeva un bicchiere alla francese. - Guardate come in lei tutto è schietto e grazioso. 

- Mi fanno ridere i tuoi engouements - disse la principessa. - No, torniamo indietro piuttosto - aggiunse poi, avendo notato Levin che moveva loro incontro con la sua donna e con un medico tedesco al quale andava dicendo qualcosa ad alta voce, con irritazione. 

Si voltarono per tornare indietro, quando improvvisamente sentirono, non più un parlare ad alta voce, ma un gridare. Levin, fermatosi, urlava, ed anche il dottore si accalorava. La folla si riuniva intorno a loro. La principessa e Kitty si allontanarono in fretta, mentre il colonnello si unì alla folla per sapere di che si trattasse. Dopo qualche minuto il colonnello le raggiunse. 

- Che cosa è successo? - domandò la principessa. 

- Un'infamia, un'ignominia - rispose il colonnello. - Una cosa sola c'è da temere: incontrare dei russi all'estero. Quel signore alto ha leticato col dottore, gli ha detto un sacco di insolenze perché non lo cura come si deve e ha levato il bastone su di lui. È proprio un'ignominia! 

- Ah, che cosa spiacevole! - disse la principessa. - E come è andata a finire? 

- Grazie a quella lì\ldots{} ci si è messa in mezzo, quella lì\ldots{} quella col cappello a fungo. Una russa, mi pare - disse il colonnello. 

- M.lle Varen'ka? - chiese Kitty con gioia. 

- Sì, sì. S'è trovata prima di tutti; ha preso quel signore sotto braccio e l'ha portato via. 

- Ecco, mamma - disse Kitty alla madre - voi vi meravigliate che io mi entusiasmi per lei! 

Fin dal giorno seguente, osservando la sua amica sconosciuta, Kitty notò che m.lle Varen'ka anche con Levin e la sua compagna usava già quei rapporti che usava con gli altri suoi protégés. Si avvicinava loro, conversava, faceva da interprete alla donna che non parlava nessuna lingua straniera. 

Kitty cominciò a supplicare ancora di più la madre perché le permettesse di conoscere Varen'ka. E per quanto dispiacesse alla principessa di fare, per così dire, il primo passo verso la signora Stahl, che si permetteva di essere orgogliosa di chi sa che cosa, ella assunse informazioni su Varen'ka. Ottenutele e concluso che non c'era nulla di male, pur non essendovi nulla di buono, in questa conoscenza, si avvicinò ella stessa per prima a Varen'ka, e si presentò. 

Nel momento in cui la figlia era andata alla fonte e Varen'ka era ferma dinanzi ad una panetteria, la principessa le si avvicinò. 

- Permettetemi di fare la vostra conoscenza - disse con il suo sorriso sostenuto. - Mia figlia è innamorata di voi. Voi forse non mi conoscete. Io\ldots{} 

- È una simpatia più che scambievole, principessa - rispose in fretta Varen'ka. 

- Che buona azione avete fatto ieri verso quel nostro povero compatriota! - disse la principessa. 

Varen'ka arrossì. 

- Non so, mi pare di non aver fatto nulla - ella disse. 

- Come! Avete salvato quel Levin da un incidente increscioso. 

- Sì, sa compagne mi ha chiamato ed io ho cercato di calmarlo: è molto malato e non è contento del dottore. Ma io sono avvezza a curare questi malati. 

- Sì, ho sentito che vivete a Mentone con vostra zia, mi pare, m.me Stahl. Conosco la sua belle-soeur. 

- No, non è mia zia. La chiamo maman, ma non le sono parente. Sono stata allevata da lei - rispose Varen'ka, arrossendo di nuovo. 

La cosa era stata detta con tanta semplicità, ed era così piena di grazia l'espressione sincera e aperta del suo viso, che la principessa capì perché Kitty avesse preso a voler bene a questa Varen'ka. 

- E ora che cosa fa qui quel Levin? - chiese la principessa. 

- Se ne parte - rispose Varen'ka. 

In quel momento, tornando dalla fonte, raggiante di gioia perché sua madre aveva fatto la conoscenza con l'amica sconosciuta, Kitty si avvicinò. 

- Ebbene, ecco Kitty, il tuo gran desiderio di far la conoscenza con m.lle\ldots{} 

- Varen'ka - suggerì, sorridendo, Varen'ka - mi chiamano tutti così. 

Kitty arrossì di gioia e tenne stretta a lungo, tacendo, la mano della nuova amica, che non rispondeva alla sua stretta, ma rimaneva immobile nella mano di lei. La mano non rispondeva alla stretta, ma il viso di m.lle Varen'ka si illuminò di un sorriso tranquillo, dolce anche se un po' triste, che scopriva i denti grandi, ma belli. 

- Anch'io lo desideravo da tempo - ella disse. 

- Ma voi siete così occupata\ldots{} 

- Oh, al contrario, non sono per nulla occupata - rispose Varen'ka, e intanto, proprio in quel momento, dovette lasciare le nuove conoscenti, perché due bambine russe, figlie di un malato, correvano verso di lei. 

- Varen'ka, la mamma chiama! - gridavano. 

E Varen'ka se ne andò con loro. 

\capitolo{XXXII}\label{xxxii-1} 

I particolari che la principessa era venuta a sapere sul passato di Varen'ka e sui suoi rapporti con la signora Stahl erano i seguenti. 

La signora Stahl, della quale alcuni dicevano che aveva tormentato il marito, e altri che costui aveva tormentato lei con la sua condotta immorale, era una donna eternamente ammalata ed esaltata. Il suo primo bambino, nato dopo il divorzio, era morto appena venuto al mondo, e i parenti della signora Stahl, conoscendo la sua sensibilità e temendo che questa notizia potesse ucciderla, sostituirono il bambino con la figlia del cuoco di corte, nata in quella stessa notte e nella stessa casa, a Pietroburgo. Era questa Varen'ka. La signora Stahl, in seguito, aveva saputo che Varen'ka non era sua figlia, ma aveva continuato ad allevarla, tanto più che, non molto dopo, Varen'ka era rimasta orfana di padre e di madre. 

La signora Stahl viveva da più di dieci anni continuamente all'estero, al sud, senza mai alzarsi dal letto. Alcuni dicevano che la signora Stahl si era creata in società la fama di donna virtuosa, profondamente religiosa; altri dicevano ch'ella era tale nell'anima quale appariva: un essere altamente morale che viveva solo per il bene del prossimo. Nessuno sapeva di quale religione fosse: cattolica, protestante o ortodossa; ma una cosa era fuor di dubbio: che era in relazioni amichevoli con i personaggi più alti di tutte le chiese e di tutte le confessioni. 

Varen'ka viveva sempre con lei all'estero e tutti quelli che conoscevano la signora Stahl conoscevano e amavano m.lle Varen'ka, così come era chiamata. 

Conosciuti questi particolari, la principessa non trovò nulla di riprovevole nel fare avvicinare la propria figliuola a Varen'ka, tanto più che Varen'ka aveva modi di educazione eccellenti: parlava perfettamente il francese e l'inglese; e poi, e ciò contava più di tutto, ella aveva riferito il rammarico della signora Stahl di essere privata, a causa della sua malattia, del piacere di fare la conoscenza della principessa. 

Conosciuta Varen'ka, Kitty ne fu sempre più entusiasta, e ogni giorno scopriva in lei nuove qualità. 

La principessa, avendo saputo che Varen'ka aveva una bella voce, la pregò di venire a cantare da loro una sera. 

- Kitty suona\ldots{} non abbiamo un buon pianoforte, è vero, ma voi ci farete un gran piacere - disse la principessa con il suo sorriso di occasione che ora spiaceva in modo particolare a Kitty perché aveva notato come Varen'ka non avesse nessuna voglia di cantare. Varen'ka, tuttavia, venne una sera e portò con sé gli spartiti. La principessa aveva invitato Mar'ja Evgenevna con la figlia e il colonnello. 

Varen'ka sembrava completamente indifferente al fatto che ci fossero persone a lei sconosciute, e subito si accostò al piano. Non sapeva accompagnarsi; ma leggeva benissimo le note con la voce. Kitty, che suonava bene, l'accompagnava. 

- Avete un talento straordinario - le disse la principessa dopo che Varen'ka ebbe cantato il primo pezzo. Mar'ja Evgenevna e la figlia la ringraziarono e si complimentarono. 

- Guardate un po' - disse il colonnello osservando dalla finestra - che pubblico s'è raccolto ad ascoltarvi. - Infatti sotto le finestre s'era raccolto un gruppo abbastanza folto. 

- Sono molto contenta che questo vi faccia piacere - rispose semplicemente Varen'ka. 

Kitty guardava l'amica con orgoglio. Era entusiasta dell'arte, della voce e del viso di lei, ma più di tutto era entusiasta del suo modo di fare, del fatto che Varen'ka, evidentemente, non dava alcun peso al proprio canto ed era del tutto indifferente alle lodi; pareva solo domandare se dovesse cantare ancora, o se bastasse. 

``Se fossi io - pensava Kitty fra sé - come andrei orgogliosa di questo! Come mi rallegrerei a guardar questa folla sotto le finestre! E a lei tutto è indifferente. La preoccupa solo il desiderio di non rifiutare e di far cosa gradita a maman. Che c'è mai in lei? Cos'è che le dà questa forza di rinunciare a tutto, di essere imperturbabilmente serena? Come vorrei sapere ciò e impararlo da lei!'' pensava Kitty guardando fisso quel viso calmo. La principessa pregò Varen'ka di cantare ancora e Varen'ka cantò un altro pezzo con eguale calma, con precisione ed accuratezza, stando in piedi accanto al piano e battendo il tempo su di esso con la sua mano magra e abbronzata. 

Fra gli spartiti, il pezzo che veniva dopo era una canzone italiana. Kitty ne accennò le prime battute e si voltò a guardare Varen'ka. 

- Saltiamola, questa - disse Varen'ka, arrossendo. 

Kitty fissò i suoi occhi turbati e interrogativi nel viso di Varen'ka. 

- Allora via, un'altra cosa - aggiunse in fretta, svolgendo i fogli e comprendendo subito che a questa canzone era legato un qualche ricordo. 

- No - rispose Varen'ka, poggiando la mano sulla musica e sorridendo - no, cantiamo questa - e cantò altrettanto bene, tranquilla e pacata come prima. 

Quando ebbe finito, tutti la ringraziarono ancora e uscirono a prendere il tè: Kitty e Varen'ka uscirono nel piccolo giardino che era accanto alla casa. 

- È vero che qualche vostro ricordo è legato a quella canzone? - disse Kitty. - Non me ne parlate - aggiunse in fretta, - ditemi soltanto se è vero. 

- Perché non dirlo? Io parlerò - disse semplicemente Varen'ka, e, senza aspettare la risposta, continuò: - Già, è un ricordo, ed è stato penoso un tempo. Ho amato un uomo e ho cantato per lui quella canzone. 

Kitty coi grandi occhi intenti taceva e guardava Varen'ka con tenerezza. 

- Lo amavo, anche lui mi amava, ma sua madre non volle, e lui ha sposato un'altra. Ora vive non lontano da noi, ed io lo vedo ogni tanto. Non pensavate che anch'io potevo avere una storia d'amore? - disse, e nel bel viso brillò appena appena quella fiammella che, Kitty lo sentiva, aveva dovuto, un tempo, illuminarla tutta. 

- Perché dovrei non pensarlo? Se fossi un uomo non avrei potuto amare nessun'altra dopo aver conosciuto voi. Non capisco, però, come egli abbia potuto dimenticare voi per compiacere sua madre e fare di voi un'infelice; non aveva cuore. 

- Oh, no, è un uomo molto buono, e io non sono infelice; al contrario, sono molto felice. Su, non canteremo più stasera? - aggiunse, dirigendosi verso casa. 

- Come siete buona, come siete buona! - esclamò Kitty e, fermatasi, la baciò. - Potessi assomigliarvi almeno un po'! 

- Perché mai dovreste assomigliare a qualcuno? Voi siete buona così come siete - disse Varen'ka, sorridendo col suo sorriso mite e stanco. 

- No, io non sono buona affatto. Su, ditemi\ldots{} Aspettate, sediamoci un po' - disse Kitty, facendola di nuovo sedere su di una panchina accanto a sé. - Ditemi, è possibile sentire come un'offesa il fatto che un uomo ha disdegnato il vostro amore, che non l'ha voluto? 

- Ma lui non l'ha disdegnato il mio amore; io credo che mi amasse, ma era un figlio sottomesso\ldots{} 

- Già, ma se lui si fosse comportato così non per volere della madre, ma per proprio volere? - disse Kitty, sentendo di aver rivelato il proprio segreto e che il suo viso, rosso di fiamma per la vergogna, la tradiva. 

- In tal caso egli avrebbe agito male, ed io avrei avuto pena di lui - rispose Varen'ka, comprendendo che ormai non si trattava più di lei, ma di Kitty. 

- E l'offesa? - disse Kitty. - L'offesa non si può dimenticare - diceva, ricordando quel suo sguardo all'ultimo ballo, mentre la musica taceva. 

- E in che cosa consiste quest'offesa? Che forse avete agito male voi? 

- Peggio che male, vergognosamente. 

Varen'ka scosse il capo e mise la mano su quella di Kitty. 

- Ma perché mai vergognosamente? - ella disse. - Non potevate certo dire a un uomo, cui voi eravate indifferente, che l'amavate? 

- S'intende che non l'ho detto; non ho detto neppure una parola, ma egli ha capito. No, no, ci sono degli sguardi, ci sono degli atteggiamenti!\ldots{} Vivessi cento anni, non potrò dimenticare. 

- Ebbene, allora? Non capisco. La questione è tutta in questo: se voi ora l'amate o no - disse Varen'ka, parlando chiaro. 

- Lo odio, e non riesco a perdonarmelo. 

- Ma che cosa dunque? 

- La vergogna mia, l'offesa ricevutane. 

- Ah, se tutti fossero sensibili come voi! - disse Varen'ka. - Non vi è una ragazza cui ciò non sia accaduto. Ma tutto questo è così poco importante! 

- E che cosa mai è importante? - chiese Kitty, guardando il viso di lei con curiosa attenzione. 

- Ah, molte cose sono importanti - disse sorridendo Varen'ka. 

- E che cosa mai? 

- Ah, molte cose sono più importanti - rispose Varen'ka, non sapendo cosa dire. Ma in quel momento dalla finestra si udì la voce della principessa: 

- Kitty, fa fresco! O prendi uno scialle o rientra in casa. 

- È vero, è ora - disse Varen'ka, alzandosi. - Devo ancora passare da m.me Berthe, me l'ha chiesto. 

Kitty le teneva la mano e con appassionata ansietà e preghiera le domandava con lo sguardo: ``Cos'è, cos'è mai questa cosa più importante di tutto che dà una simile pace? Voi la sapete, ditemela!''. Ma Varen'ka non capiva quello che le domandava lo sguardo di Kitty. Ricordava solo che quel giorno doveva ancora passare da m.me Berthe e che doveva arrivare in tempo a casa per il tè di maman, verso mezzanotte. Entrò nelle stanze, riunì la musica e, dopo aver salutato tutti, si preparò ad andar via. 

- Permettete che vi accompagni - disse il colonnello. 

- Ma certo; e come andar sola, di notte? - replicò la principessa. - Altrimenti vi farò accompagnare da Paraša. 

Kitty vedeva che Varen'ka tratteneva a stento un sorriso per questa convinzione che si dovesse accompagnarla. 

- No, io vado sempre da sola, e non mi accade mai nulla - disse, prendendo il cappello. E dopo aver baciato ancora una volta Kitty, e senza averle detto quale fosse la cosa importante, con passo svelto e con le carte sotto il braccio, scomparve nella penombra della notte estiva, portando con sé il segreto di quello che era importante e che le conferiva quella invidiabile, dignitosa pace. 

\capitolo{XXXIII}\label{xxxiii-1} 

Kitty aveva conosciuto anche la signora Stahl e questa conoscenza, unita all'amicizia di Varen'ka, non solo aveva avuto una grande influenza su di lei, ma l'aveva consolata della sua pena. Aveva trovato sollievo perché, grazie a questa conoscenza, le si era aperto nell'anima un mondo del tutto nuovo, che non aveva nulla di comune col suo passato, un mondo elevato, bellissimo, dall'alto del quale si poteva guardare con serenità al passato. Ebbe la rivelazione che oltre alla vita istintiva, alla quale ella si era finora abbandonata, esisteva anche una vita dello spirito. Questa vita era rivelata dalla religione, ma da una religione che non aveva nulla di comune con quella che Kitty praticava dall'infanzia e che tutta si esprimeva ed esauriva nell'assistere alla messa e ai vespri, nel recarsi alla ``Casa delle vedove'' dove si potevano incontrare dei conoscenti, e nello studiare a memoria col batjuška testi in slavo antico: quest'altra era una religione altissima, misteriosa, legata a una serie di pensieri e di sentimenti splendidi, in cui non solo si poteva credere, perché così era comandato, ma che si poteva amare. 

Kitty non apprese tutto ciò dalle parole. La signora Stahl parlava con Kitty come una bambina graziosa di cui ci si compiace quasi in ricordo della propria giovinezza, e soltanto una volta aveva detto che tutti i dolori umani traggono conforto soltanto dall'amore e dalla fede, e che nessun dolore è trascurato dalla compassione di Cristo per noi: ma subito aveva avviato il discorso su un altro argomento. Eppure Kitty in ogni movimento di lei, in ogni sua parola, in ogni suo sguardo che Kitty definiva celestiale, e in particolare in tutta la storia della vita di lei che ella conosceva attraverso Varen'ka, in tutto infine, riconosceva ``quello che è importante'', e che finora non aveva conosciuto. 

Ma per quanto elevato fosse il carattere della signora Stahl, per quanto commovente fosse tutta la sua storia, per quanto elevata e tenera la sua parola, Kitty notò un lei, e con disapprovazione, alcuni tratti che la sconcertarono. Aveva notato che, chiedendole dei suoi parenti, la signora Stahl aveva sorriso sprezzantemente, il che era contrario alla carità cristiana. Inoltre un giorno che aveva trovato da lei un prete cattolico aveva notato che la signora Stahl aveva tenuto con cura il viso nell'ombra del paralume e aveva sorriso in modo strano. Per quanto insignificanti, queste due osservazioni la sconcertarono ed ella dubitava ora della signora Stahl. In compenso Varen'ka, sola al mondo, senza parenti, senza amici, con la sua triste delusione nel cuore, Varen'ka che non desiderava nulla e di nulla si rammaricava, costituiva quella perfezione che Kitty soltanto in sogno aveva intravisto. Osservando Varen'ka aveva compreso che bastava solo dimenticare se stessi e amare gli altri per essere calmi, felici e sereni. Tale voleva essere Kitty. Avendo adesso chiaramente conosciuto quale fosse la cosa più importante, Kitty non si accontentò di ammirare, ma subito si diede con tutta l'anima a praticare questa nuova vita che le si era dischiusa. Seguendo i racconti di Varen'ka sull'attività della signora Stahl e di altre persone che ella nominava, Kitty si tracciò un piano di vita per l'avvenire. Dovunque avesse vissuto, ella avrebbe cercato, come Aline, la nipote della signora Stahl di cui Varen'ka parlava tanto, gli sventurati, li avrebbe aiutati per quanto possibile, avrebbe distribuito il Vangelo, lo avrebbe letto ai malati, ai delinquenti, ai moribondi. L'idea di leggere il Vangelo ai delinquenti, così come faceva Aline, tentava in modo particolare Kitty. Ma tutti questi erano segreti, dei quali Kitty non faceva parte né alla madre, né a Varen'ka. 

E, in attesa di poter eseguire su vasta scala i suoi piani, Kitty anche ora nella stazione termale, dove c'erano tanti malati e tanti disgraziati, imitando Varen'ka, trovò facile attuazione alle sue nuove direttive. 

Dapprima la principessa notò che Kitty si trovava sotto un forte influsso del suo engouement, così come lo chiamava lei, per la signora Stahl e in particolare per Varen'ka. Vedeva che Kitty, non solo imitava Varen'ka nella sua attività, ma senza accorgersene l'imitava anche nella maniera di camminare, di parlare e di battere le palpebre. Ma in seguito la principessa notò che nella figlia, a parte questo incantamento, si compiva una vera trasformazione spirituale. 

La principessa notava che Kitty, di sera, leggeva un Vangelo francese che le aveva regalato la signora Stahl, cosa che prima non faceva; sfuggiva le relazioni mondane e si accostava ai malati che erano sotto la protezione di Varen'ka, ed in particolare a una povera famiglia di un certo pittore, Petrov. Kitty evidentemente era orgogliosa di compiere i doveri di una suora di carità in questa famiglia. Tutto questo era bene e la principessa non trovava nulla da ridire, tanto più che la moglie di Petrov era una donna perfettamente a posto, e che la principessa reale, notando l'attività di Kitty, ne aveva fatto le lodi chiamandola l'angelo consolatore. Tutto questo sarebbe andato molto bene se non avesse raggiunto l'eccesso. E la principessa, vedendo che la figlia cadeva nell'eccesso, glielo faceva notare. 

- Il ne faut jamais rien outrer - le diceva. Ma la figlia non rispondeva nulla. In cuor suo pensava che non si può parlare di eccesso nell'attività cristiana. Quale eccesso poteva esserci in una dottrina che insegnava a porgere la guancia sinistra quando avessero percosso la destra, e a dar via la camicia, quando avessero tolto il mantello? Ma alla principessa questo eccesso non piaceva e ancor più le spiaceva il fatto che Kitty, ella lo sentiva, non le aprisse tutta l'anima sua. In realtà Kitty nascondeva alla madre le sue nuove visioni e i suoi sentimenti. Li nascondeva, non perché non stimasse o non amasse sua madre, ma solo perché era sua madre; li avrebbe svelati a chiunque anziché alla madre. 

- È un bel po' che Anna Pavlovna non è venuta da noi - disse un giorno la principessa a proposito della Petrova. - L'ho invitata; ma mi pare offesa. 

- No, non l'ho notato, maman - disse Kitty, avvampando. 

- È da molto che manchi da loro? 

- Pensiamo di fare domani una passeggiata in montagna - rispose Kitty. 

- Ebbene, andate - disse la principessa, notando la confusione apparsa sul viso della figlia e cercando di indovinarne la causa. 

Quel giorno stesso Varen'ka venne a pranzo e riferì che Anna Pavlovna aveva rinunciato ad andare l'indomani in montagna. E la principessa notò che Kitty era improvvisamente diventata rossa. 

- Kitty, non è mica successo qualcosa di spiacevole tra te e i Petrov? - chiese la principessa quando restarono sole. - Perché non ha più mandato le bambine da noi? 

Kitty rispose che nulla era successo fra di loro e che proprio non capiva perché Anna Pavlovna sembrasse scontenta di lei. Kitty aveva detto tutta la verità. Non conosceva le cause del cambiamento di Anna Pavlovna nei suoi riguardi, ma indovinava. Indovinava una tal cosa che non poteva dire alla madre, che non poteva dire nemmeno a se stessa. Era una di quelle cose che si intuiscono, ma che non si possono dire neanche a se stessi: tanto è terribile e vergognoso lo sbagliarsi. 

Riesaminò ancora una volta nel ricordo tutti i suoi rapporti con quella famiglia. Ricordò la gioia ingenua che si esprimeva sul viso tondo, bonario di Anna Pavlovna nei loro incontri; ricordò i loro discorsi segreti a proposito del malato, le congiure per distrarlo dal lavoro che gli era stato proibito, e per portarlo a passeggio; l'attaccamento del bambino più piccolo che la chiamava ``la mia Kitty'' e che non voleva andare a letto senza di lei. Come tutto era bello! Poi ricordò la figura magra di Petrov, il suo collo lungo, il soprabito marrone, i radi capelli ondulati, gli occhi azzurri che sembravano interrogare e che impressionavano Kitty nei primi tempi, e gli sforzi morbosi di lui per sembrare valido e vivace in sua presenza. Ricordò il proprio sforzo per vincere nei primi tempi la ripugnanza che provava per lui come per tutti i tisici, e lo sforzo per escogitare cosa dirgli. Ricordò quello sguardo timido, commosso col quale egli la guardava, e lo strano senso di compassione e di imbarazzo, seguìto alla coscienza della propria virtù, ch'ella provava in quel momento. Come tutto ciò era bello! Ma tutto questo era accaduto nei primi tempi. Ora invece, da alcuni giorni, tutto si era improvvisamente sciupato. Anna Pavlovna l'accoglieva con una cortesia finta e non cessava d'osservare lei e il marito. 

Possibile che quella commovente gioia di lui al suo avvicinarsi fosse la causa del raffreddamento di Anna Pavlovna? 

``Sì - ricordava - c'era qualcosa di poco naturale in Anna Pavlovna, del tutto diverso dalla sua bontà, quando l'altro giorno ha detto con rancore: `Ecco, tutto per aspettare voi, non ha voluto prendere il caffè senza di voi, pur essendo spaventosamente debole'\,''. 

``Sì, forse le è spiaciuto anche quando gli ho dato lo scialle. Tutto questo è così semplice, ma lui l'ha accolto con tanto impaccio, ha ringraziato così a lungo che io ero a disagio. E quel mio ritratto che ha dipinto così bene! E poi ancora, soprattutto, quello sguardo, confuso e tenero! Sì, sì, è così! - si ripeteva con orrore. - No, questo non può, non deve essere! Fa tanta pena!'' diceva a se stessa subito dopo. 

E questo dubbio le avvelenava l'incanto della nuova vita. 

\capitolo{XXXIV}\label{xxxiv-1} 

Prima della chiusura della stagione termale, il principe Šcerbackij che, dopo Karlsbad, era stato a Baden e Kissingen, da conoscenti russi per fare, come egli diceva, provvista di spirito russo, tornò dai suoi. 

Le opinioni del principe e della principessa sulla vita all'estero erano completamente opposte. La principessa trovava tutto bellissimo e, malgrado la sua salda posizione nella società russa, all'estero faceva di tutto per sembrare una dama europea, quale non era, dal momento che era una vera signora russa, e in questo suo voler essere diversa da quello che era, si sentiva un po' a disagio. Il principe, al contrario, all'estero criticava tutto, si sentiva oppresso dalla vita europea, conservava le sue abitudini russe, sforzandosi di mostrarsi all'estero meno europeo di quanto non lo fosse in realtà. 

Il principe era tornato dimagrito, con le borse sotto gli occhi, ma di ottimo umore. E questo suo buon umore aumentò quando vide Kitty completamente ristabilita. La notizia dell'amicizia di Kitty con la signora Stahl e Varen'ka e le osservazioni della principessa su di un certo cambiamento prodottosi in Kitty, sconcertarono il principe e ridestarono in lui il solito senso di gelosia verso tutto quello che appassionava la figlia a sua insaputa, e la paura che la figlia sfuggisse alla sua influenza, rifugiandosi in qualche regione a lui inaccessibile. Ma queste notizie poco piacevoli affondarono in quel mare di bonarietà e di allegria che sempre era in lui e che la cura di Karlsbad aveva accresciuto. 

Il giorno dopo il suo arrivo, il principe, di ottimo umore, nel suo lungo cappotto, con le sue rughe tipicamente russe e le guance gonfie sostenute dal colletto inamidato, andò alla fonte in compagnia della figlia. 

La mattina era splendida: le case linde e allegre con i giardinetti, le cameriere tedesche dal viso rosso, dalle mani rosse, sature di birra e allegramente intente al lavoro, il sole gagliardo, rallegravano il cuore; ma più si avvicinavano alla fonte e più numerosi incontravano i malati, e il loro aspetto sembrava ancor più desolante sullo sfondo di vita tedesca solitamente ben organizzata. Questo contrasto non colpiva ormai più Kitty. Il sole splendente, l'allegro luccichio del verde, i suoni della musica erano per lei una cornice naturale di tutti quei visi ormai noti e dei loro mutamenti in peggio o in meglio ch'ella notava; ma al principe la luce e lo splendore di quella mattina di luglio, i suoni dell'orchestra che eseguiva un allegro valzer di moda e soprattutto la vista della rubiconde, robuste cameriere facevan l'effetto di cosa disadatta e innaturale ad accogliere quelle larve umane convenute da ogni parte d'Europa, lentamente deambulanti. 

Malgrado il senso d'orgoglio e quasi di rinnovata giovinezza ch'egli provava quando la figliuola preferita camminava al suo braccio, sentiva ora quasi un senso di disagio e di mortificazione per il proprio passo deciso, per le proprie membra robuste, ricoperte di carne. Provava la sensazione di un uomo che andasse svestito in società. 

- Presentami, presentami ai tuoi nuovi amici - chiedeva alla figliuola, premendole il braccio col gomito. - Ho finito col voler bene anche a questo tuo sudicio Soden che ti ha fatto rimettere così. Solo che è triste, triste qui da voi. Questo chi è? 

Kitty gli veniva nominando le persone conosciute e quelle non conosciute che incontrava. Proprio all'ingresso del giardino incontrarono m.me Berthe, la cieca, l'accompagnatrice, e il principe si rallegrò dell'espressione commossa della vecchia francese nel sentir la voce di Kitty. Ella subito si mise a parlar con lui, con quell'eccessiva cortesia francese, felicitandosi per la figliola così straordinaria e innalzando al cielo Kitty che chiamava tesoro, perla, angelo consolatore. 

- Via, però è sempre l'angelo numero due - disse il principe sorridendo. - Perché l'angelo numero uno è m.lle Varen'ka, a dir di mia figlia. 

- Oh, m.lle Varen'ka è un angelo del cielo, allez - replicò m.me Berthe. 

Sotto il portico incontrarono Varen'ka in persona. Veniva svelta incontro a loro, con un'elegante borsetta rossa. 

- Ecco, è arrivato anche papà! - le disse Kitty. 

Varen'ka fece con semplicità e naturalezza, come del resto faceva tutto, un movimento fra l'inchino e la riverenza, e cominciò subito a parlare col principe come parlava con tutti, in maniera semplice e spontanea. 

- Ma io vi conosco, naturalmente, e vi conosco da molto - le disse il principe con un sorriso dal quale Kitty capì con gioia che l'amica sua era piaciuta al padre. - Dove vi affrettate tanto? 

- Maman è qui - ella disse, volgendosi a Kitty. - Non ha dormito tutta la notte e il dottore le ha consigliato di uscire. Le porto il lavoro. 

- Così questo è l'angelo numero uno - disse il principe, quando Varen'ka se ne fu andata. 

Kitty vedeva ch'egli avrebbe voluto scherzare su Varen'ka, ma che non poteva riuscirci in nessun modo, perché Varen'ka gli era piaciuta. 

- Sì, ecco che vedremo tutti i tuoi amici - aggiunse - anche la signora Stahl, se mi concederà l'onore di riconoscermi. 

- Ma tu l'hai forse conosciuta, papà? - chiese Kitty con terrore, avendo notato un lampo di irrisione negli occhi del principe al ricordo della signora Stahl. 

- Conoscevo suo marito e lei, ancora prima che si iscrivesse fra le pietiste. 

- Che cosa vuol dire pietista, papà? - chiese Kitty, già spaventata del fatto che quello che ella apprezzava così altamente nella signora Stahl avesse un nome. 

- Neanche io lo so con precisione. So soltanto ch'ella ringrazia Dio di tutto; di ogni sventura, e anche della morte del marito ringrazia Iddio. Ebbene, questo fa ridere, perché loro due non andavano d'accordo. 

- Chi è quello là? Che viso da far pena! - chiese dopo aver notato un malato non alto, seduto su di una panchina, con un cappotto marrone e dei pantaloni bianchi che facevano delle strane pieghe sulle ossa scarnite delle gambe. 

Il signore sollevò il cappello di paglia sui radi capelli ondulati, scoprendo una fronte alta, arrossata dal cappello. 

- È Petrov, il pittore - rispose Kitty, arrossendo. - E questa è sua moglie - aggiunse indicando Anna Pavlovna la quale, come apposta, nel momento in cui essi si avvicinavano, si era messa a rincorrere un bambino scappato via per un viale. 

- Come fa pena, ma che viso simpatico che ha! - disse il principe. - Come mai non ti sei avvicinata? Non ti voleva forse dire qualcosa? 

- Su, via, andiamo! - disse Kitty voltandosi risoluta. - Come state oggi? - chiese a Petrov. 

Petrov si alzò, appoggiandosi al bastone e guardando timidamente il principe. 

- È mia figlia - disse il principe. - Permettetemi di fare la vostra conoscenza. 

Il pittore si inchinò e sorrise, scoprendo i denti bianchi straordinariamente lucidi. 

- Vi abbiamo aspettato ieri, principessina - disse egli a Kitty. 

Vacillò, dicendo questo, ma, ripetendo il movimento, si sforzava di far parere che l'avesse fatto apposta. 

- Io volevo venire ma Anna Pavlovna mi ha fatto sapere per mezzo di Varen'ka che non sareste andati. 

- Come non saremmo andati! - disse Petrov, arrossendo e tossendo subito, cercando con gli occhi la moglie. - Aneta! Aneta! - chiamò con voce aspra e sul collo bianco si tesero, come corde, le grosse vene. 

Anna Pavlovna si avvicinò. 

- Come mai hai mandato a dire alla principessina che non saremmo andati? - mormorò irritato, già senza voce. 

- Buon giorno, principessina - disse Anna Pavlovna, con un sorriso finto, affatto dissimile dalle sue maniere d'una volta. - Piacere di conoscervi - disse rivolta al principe. - Vi aspettavamo da lungo tempo, principe. 

- Come mai hai mandato a dire alla principessina che non saremmo andati? - mormorò rauco, una seconda volta, il pittore ancor più irritato, perché la voce gli veniva a mancare e non riusciva a dare alle parole l'intonazione che avrebbe voluto. 

- Ah, Dio mio! Pensavo che non saremmo andati - rispose la moglie con dispetto. 

- Ma, come se\ldots{} - e cominciò a tossire e a far un gesto con la mano. 

Il principe sollevò il cappello e si allontanò con la figlia. 

- Oh, oh - sospirò penosamente; - oh, che disgraziati! 

- Sì, papà - ripose Kitty. - E devi sapere che hanno tre bambini, e sono senza donna di servizio e quasi senza mezzi. Egli riceve qualcosa dall'Accademia - raccontò vivacemente Kitty sforzandosi di soffocare l'agitazione dalla quale era stata presa per lo strano mutamento di Anna Pavlovna nei suoi riguardi. 

- Ed ecco anche la signora Stahl! - disse Kitty, indicando una carrozzina nella quale, avvolta fra i cuscini e in un groviglio grigio-azzurro, sotto un ombrellino, giaceva una certa cosa. 

Era la signora Stahl! Dietro di lei stava dritto un robusto lavoratore tedesco dall'aria burbera che la trasportava. Accanto veniva un biondo conte svedese che Kitty conosceva di nome. Alcuni malati si fermarono attorno alla carrozzina, guardando questa signora come una cosa rara. 

Il principe si avvicinò. E subito negli occhi di lui Kitty notò la piccola luce di irrisione che l'aveva sconcertata. Si avvicinò alla signora Stahl e cominciò a parlare in quell'ottimo francese che ormai così pochi parlano, straordinariamente cortese e gentile. 

- Non so se vi ricordate di me, ma io devo richiamarmi alla vostra memoria per ringraziarvi della bontà usata verso la mia figliuola - egli disse, dopo essersi tolto il cappello e senza rimetterlo. 

- Il principe Aleksandr Šcerbackij - disse la signora Stahl alzando su di lui i suoi occhi celesti, nei quali Kitty notò lo scontento. - Molto lieta. Io voglio molto bene alla vostra figliuola. 

- La vostra salute è sempre poco buona? 

- Sì, ormai mi ci sono abituata - disse la signora Stahl e presentò al principe il conte svedese. 

-Ma voi siete molto poco cambiata - disse il principe. - Io non ho avuto l'onore di vedervi da dieci o undici anni. 

- Sì, Dio dà la croce e Dio dà la forza per portarla. Spesso ci si meraviglia perché si prolunga questa vita\ldots{} Dall'altra parte! - disse con stizza a Varen'ka che le avvolgeva le gambe nello scialle non precisamente come voleva lei. 

- Per far del bene, probabilmente - disse il principe, ridendo con gli occhi. 

- Questo non spetta a noi giudicare - disse la signora Stahl, che aveva colto la sfumatura di irrisione nel viso del principe. - Così voi mi manderete questo libro, caro conte? Vi ringrazio molto - disse rivolta al giovane svedese. 

- Ah - esclamò il principe, vedendo il colonnello di Mosca che era in piedi lì accanto e, salutata la signora Stahl, si allontanò con la figlia e con il colonnello moscovita che si era unito a loro. 

- Questa è la nostra aristocrazia, principe - disse, cercando d'essere ironico, il colonnello moscovita, che ce l'aveva con la signora Stahl perché non aveva fatto amicizia con lui. 

- Sempre la stessa - rispose il principe. 

- Ma voi l'avete conosciuta ancora prima della sua malattia, cioè prima che si fosse messa a letto? 

- Già, s'è messa a letto quando già la conoscevo. 

- Dicono che non si alzi da dieci anni. 

- Non si alza perché ha una gamba più corta dell'altra. È fatta molto male\ldots{} 

- Papà, ma non può essere! - gridò Kitty. 

- Le cattive lingue dicono così, figlia mia. E la tua Varen'ka deve saperne abbastanza - aggiunse. - Oh queste signore malate! 

- Oh, no, papà! - ribatté Kitty con calore. - Varen'ka l'adora. E poi è una donna che fa tanto bene. Domanda a chi vuoi. Lei ed Aline Stahl sono conosciute da tutti. 

- Può darsi - disse egli, stringendole il braccio col gomito. - Ma vale di più quando si fa in modo che, a chiunque si chieda, nessuno lo sappia. 

Kitty tacque, non perché non avesse nulla da ribattere, ma perché non voleva svelare neanche al padre i suoi segreti pensieri. Però, cosa strana, pur preparandosi a non sottostare all'introspezione del padre, a non dargli accesso nel suo santuario, sentì che quella immagine sublime della signora Stahl, che per un mese intero aveva portato nell'anima, era irrimediabilmente scomparsa, così come scompare la figura formata da un vestito abbandonato, quando ci si accorge che è solo un vestito. Era rimasta ormai una donna con una gamba più corta dell'altra che stava a letto perché era fatta male e tormentava la docile Varen'ka perché non ravvolgeva lo scialle così come andava fatto. E ormai nessuno sforzo dell'immaginazione poteva far rivivere la signora Stahl di prima. 

\capitolo{XXXV}\label{xxxv} 

Il principe aveva trasmesso il suo buon umore ai familiari e agli amici e persino all'albergatore tedesco presso il quale stavano gli Šcerbackij . 

Tornando dalla fonte con Kitty e invitati per il caffè il colonnello, Mar'ja Evgenevna e Varen'ka, il principe ordinò di portare il tavolo e le poltrone nel giardino, sotto il castagno, e di apparecchiare là per la colazione. L'albergatore e la servitù si rianimarono per effetto del suo buon umore. Essi conoscevano la sua liberalità; mezz'ora dopo un dottore d'Amburgo, ammalato, che era alloggiato al piano superiore, guardava con invidia dalla finestra quell'allegra brigata di russi, formata di persone sane, raccolta sotto il castagno. All'ombra tremula, in cerchi, delle foglie, vicino a una tavola coperta da una tovaglia bianca e cosparsa di caffettiere, pane, burro, formaggio, selvaggina fredda, sedeva la principessa con un'acconciatura ornata di nastri lilla, che distribuiva tazze e tartine. All'altra estremità sedeva il principe che mangiava abbondantemente e discorreva a voce alta, allegra. Aveva disposto accanto a sé le compere fatte in grande quantità nei vari luoghi di cura: cofanetti scolpiti, gingilli, coltellini intagliati d'ogni specie, e li andava regalando a tutti, compresa Lischen, la cameriera, e l'albergatore, col quale scherzava in quel suo comico, pessimo tedesco, assicurandolo che non erano le acque che avevano guarito Kitty, ma la sua ottima cucina, in particolare la zuppa con le prugne secche. La principessa prendeva in giro il marito per le sue abitudini russe, ma era così vivace e allegra come non lo era mai stata in tutto il suo soggiorno nel luogo di cura. Il colonnello, come sempre, sorrideva agli scherzi del principe; ma in quanto all'Europa, che egli credeva di aver studiato a fondo, teneva dalla parte della principessa. La buona Mar'ja Evgenevna scoppiava a ridere a ogni facezia che diceva il principe, e perfino Varen'ka, cosa che Kitty non aveva notato mai, si sfiniva in un debole, ma contagioso riso suscitatole dagli scherzi del principe. 

Tutto questo rallegrava Kitty, ma ella non riusciva a superare le sue preoccupazioni. Non poteva risolvere il problema che involontariamente le aveva posto il padre con la propria scherzosa opinione sui suoi amici e su quella vita che ella tanto aveva preso ad amare. A questo problema si aggiungeva inoltre il mutamento dei suoi rapporti coi Petrov che quel giorno si era rivelato così evidente e spiacevole. Tutti erano allegri, ma Kitty non poteva esserlo, e questo ancor più la tormentava. Provava una sensazione simile a quella che aveva provato nell'infanzia quando, chiusa in castigo in camera sua, sentiva il riso allegro delle sorelle. 

- Ebbene, perché l'hai comprata tutta questa roba? - diceva la principessa, sorridendo e porgendo al marito una tazza di caffè 

- Che vuoi fare? Vai a passeggio, ti avvicini a una botteguccia, ti pregano di comprare: ``Erlaucht Excellenz, Durchlaucht''. Ecco, quando hanno detto Durchlaucht, io non resisto più, ed ecco, dieci talleri sono andati via. 

- Così, solo per sfuggire alla noia - disse la principessa. 

- Si sa, per la noia. Una noia tale, moglie mia, che non sai dove batter la testa. 

- Ma come ci si può annoiare, principe? Ci sono tante cose interessanti, ora, in Germania - disse Mar'ja Evgenevna. 

- Sì, lo so tutto quello che c'è d'interessante: la zuppa con le prugne secche, lo so, le salsicce coi piselli, lo so. 

- Ma no, vi prego, principe, le loro istituzioni sono interessanti - disse il colonnello. 

- Che c'è di interessante? Sono tutti contenti come tanti soldoni di rame; hanno vinto tutti gli altri. Be', e io perché dovrei essere contento? Io non ho vinto nessuno; e là anche gli stivali te li devi togliere da solo e poi metterli dietro la porta. La mattina alzati, vestiti subito, vai nel salone a bere un pessimo tè. Ben altra cosa a casa! Ti svegli senza fretta, t'arrabbi contro qualcosa, brontoli un po', ritorni in te per benino, rifletti a tutto, non ti affanni. 

- Ma il tempo è denaro, voi dimenticate ciò - disse il colonnello. 

- Ma che tempo e tempo! A volte è tale, che dareste via tutto un mese per mezzo rublo, e altre volte non c'è denaro bastante per una mezz'ora. È così, Katen'ka? Che hai, così triste? 

- Io, nulla. 

- Ma dove andate? Restate ancora un po' - disse rivolto a Varen'ka. 

- Devo andare a casa - disse Varen'ka, alzandosi e scoppiando di nuovo a ridere. 

Ricompostasi, salutò ed entrò a prendere il cappello. Kitty la seguì. Perfino Varen'ka pareva ora un'altra. Non era peggiore, ma era un'altra da quella ch'ella aveva immaginato. 

- Ah, da tempo non ridevo così - disse Varen'ka, raccogliendo ombrellino e borsa. - Com'è simpatico il vostro papà! 

Kitty taceva. 

- Quando ci vediamo? - chiese Varen'ka. 

- Maman voleva passare dai Petrov. Voi non sarete là? - disse Kitty, mettendo Varen'ka alla prova. 

- Sì, ci sarò - rispose Varen'ka. - Si preparano a partire e io ho promesso di aiutare a fare le valigie. 

- Su, verrò anch'io. 

- No, che ve ne importa? 

- Perché, perché, perché? - si mise a dire Kitty, dilatando gli occhi e afferrando l'ombrellino per non lasciare andar via Varen'ka. - No, aspettate, perché? 

- Ma dicevo così; è arrivato vostro padre, e poi hanno soggezione di voi. 

- No, ditemi perché non volete che io vada spesso dai Petrov. Voi non volete, dunque? Perché? 

- Io non ho detto questo - disse tranquilla Varen'ka. 

- No, vi prego, ditelo! 

- Devo dir tutto? - chiese Varen'ka. 

- Tutto, tutto! - replicò Kitty. 

- Ma non c'è nulla di particolare, c'è solo questo, che Michail Alekseevic - così si chiamava il pittore - prima voleva partir subito, e ora non vuole più partire - disse Varen'ka, sorridendo. 

- Ebbene, ebbene - sollecitava Kitty, guardando torva Varen'ka. 

- Ebbene, chi sa perché Anna Pavlovna ha detto che egli non vuole partire perché voi siete qui. Certo era inopportuno dir questo, ma a causa di questo, a causa vostra, ne è venuto fuori un litigio. E voi sapete come questi malati siano irritabili. 

Kitty, accigliatasi sempre più, taceva e Varen'ka parlava da sola cercando di placarla e di calmarla, prevedendo la crisi che si andava preparando, non sapeva bene se di lacrime o di parole. 

- Così è meglio che non andiate\ldots{} Dovete capire, e non offendervi. 

- E mi sta bene e mi sta bene - cominciò a dire in fretta Kitty, afferrando l'ombrellino dalle mani di Varen'ka e guardando al di là degli occhi dell'amica. 

Varen'ka voleva sorridere, vedendo l'arrabbiatura da bimba dell'amica, ma temeva di offenderla. 

- Come, vi sta bene? Non capisco - disse. 

- Mi sta bene perché tutto questo era una finzione, perché tutto questo è artificioso, e non viene dal cuore. Che me ne importa a me di un estraneo! Ed ecco che per colpa mia è venuto fuori un litigio, perché ho fatto quello che nessuno mi ha chiesto di fare. Perché tutto è finzione, finzione, finzione! 

- Ma a quale scopo fingere? - disse piano Varen'ka. 

- Ah, che cosa brutta, stupida! Io non avevo alcun bisogno\ldots{} Tutto è finzione! - diceva, aprendo e chiudendo l'ombrellino. 

- Ma a quale scopo mai? 

- Per parer migliori agli occhi della gente, a se stessi, per ingannare tutti. No, adesso non mi sottometterò più a questo. Esser cattiva, sia pure, ma almeno bugiarda, falsa, no! 

- Ma chi mai è falsa? - disse Varen'ka con rimprovero. - Voi parlate come se\ldots{} 

Ma Kitty era tutta presa dall'ira. Non le dava modo di finir di parlare. 

- Non parlo di voi, non parlo affatto di voi, voi siete la perfezione. Sì, sì, io lo so che voi siete la perfezione; ma che fare, se io sono cattiva? Questo non sarebbe accaduto se io non fossi cattiva. Che io sia quale sono, ma non falsa. Che me ne importa di Anna Pavlovna? Che vivano pure come piace loro, e io come piace a me. Io non posso esser diversa\ldots{} E tutto questo non è quel che dovrebbe essere, non è! 

- Ma cosa mai non è quel che dovrebbe essere? - diceva Varen'ka perplessa. 

- Tutto non è come dovrebbe essere. Io non posso vivere altrimenti che secondo il cuore, e voi vivete secondo le regole. Io ho preso ad amarvi semplicemente, e voi, forse, solo per salvarmi e istruirmi! 

- Siete ingiusta! - disse Varen'ka. 

- Ma io non dico nulla degli altri, parlo di me. 

- Kitty - si udì la voce della madre, - vieni, mostra a papà i tuoi coralli. 

Kitty con aria sdegnosa, senza far pace con l'amica, prese dalla tavola i coralli nella scatolina e andò dalla madre. 

- Che ti è successo, che sei così rossa? - le dissero padre e madre a una voce. 

- Nulla - ella rispose - vengo subito - e corse via. 

``È ancora qui! - pensò. - Cosa le dirò, Dio mio! Che ho fatto, che ho detto! Perché l'ho offesa? Cosa fare? Cosa dirle?'' pensava Kitty, e si fermò presso la porta. 

Varen'ka col cappello e con l'ombrellino in mano sedeva vicino alla tavola, esaminando una molla che Kitty aveva spezzato. Ella alzò il capo. 

- Varen'ka, perdonatemi, perdonate! - sussurrò Kitty, avvicinandosi a lei. - Io non mi ricordo quello che ho detto. Io\ldots{} 

- Non volevo addolorarvi, proprio no - disse Varen'ka, sorridendo. 

La pace fu conclusa. Ma da quando era arrivato suo padre, tutto quel mondo in cui ella aveva vissuto le parve cambiato. Non rinnegò tutto quello che aveva ultimamente conosciuto, ma capì che ingannava se stessa, illudendosi di poter essere quello che voleva essere. Come se fosse tornata in sé, sentì tutta la difficoltà di mantenersi, senza finzione e senza vanteria, all'altezza alla quale aspirava; inoltre sentì tutto il peso di quel mondo di dolore, di malattie, di moribondi in cui viveva; le parvero tormentosi gli sforzi che faceva su di sé per amare tutto questo, e desiderò di andare al più presto via, all'aria fresca, in Russia, ad Ergušovo, dove, come aveva saputo da una lettera, era già andata Dolly coi bambini. 

Ma il suo amore per Varen'ka non si affievolì. Nel congedarsi, Kitty la pregò di venire da loro in Russia. 

- Verrò quando vi sposerete - disse Varen'ka. 

- Io non mi sposerò. 

- E allora non verrò mai. 

- E allora mi sposerò, soltanto perché possiate venire. Badate, dunque, di non dimenticare la promessa! - disse Kitty. 

Le previsioni del medico curante si erano avverate. Kitty ritornò a casa, in Russia, guarita. Non era più spensierata e allegra come una volta, ma era tranquilla. I suoi dolori di Mosca erano diventati un ricordo. 