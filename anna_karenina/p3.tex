\parte{PARTE TERZA}\label{parte-terza} 
\pagestyle{pagina}

\capitolo{I}Sergej Ivanovic Koznyšev voleva prendersi un po' di riposo dal lavoro intellettuale e, invece di andarsene, come al solito, all'estero, verso la fine di maggio, si recò in campagna dal fratello. Secondo la sua convinzione, la vita di campagna era la migliore. Era quindi venuto dal fratello a godersela questa vita. Konstantin Levin ne fu molto contento; tanto più che per quell'estate non aspettava suo fratello Nikolaj. Ma, pur avendo stima ed affetto per Sergej Ivanovic, in campagna Konstantin Levin non si trovava a suo agio con lui. Non si sentiva a suo agio, e perfino gli spiaceva l'atteggiamento del fratello verso la vita di campagna. Per Konstantin Levin la campagna era un luogo di vita, cioè di gioia, di sofferenza e di lavoro; per Sergej Ivanovic la campagna era, da una parte, il riposo dal lavoro, dall'altra un utile controveleno alla corruzione, ch'egli prendeva con piacere, consapevole della sua efficacia. Per Konstantin Levin la campagna era tanto bella perché rappresentava il campo di azione per un lavoro indubbiamente utile; per Sergej Ivanovic la campagna era bella perché vi si poteva e vi si doveva restare oziosi. Inoltre anche l'atteggiamento di Sergej Ivanovic verso la gente di campagna offendeva un po' Konstantin Levin. Sergej Ivanovic diceva di amarla e di conoscerla, quella gente, e spesso se ne stava a discorrere con i contadini, cosa che faceva con garbo, senza infingimenti o affettazioni, e da ognuna di queste conversazioni ricavava dei dati generali in favore del popolo e a conferma della conoscenza che diceva di averne. Un simile atteggiamento non piaceva a Konstantin Levin. Per lui il contadino era solo il collaboratore primo al lavoro comune, e malgrado tutta la considerazione che gli accordava e un certo amore che aveva probabilmente succhiato, come egli stesso diceva, insieme al latte della balia contadina, tuttavia egli, come collaboratore al lavoro comune, pure estasiandosi talvolta dinanzi alla forza, all'umiltà, alla verità di quella gente, molto spesso, quando il lavoro comune richiedeva altre attitudini, inveiva contro il contadino per la sua trascurataggine e sporcizia, per la tendenza all'ubriachezza e l'abitudine a mentire. Se avessero chiesto a Konstantin Levin se amasse o no quella gente, egli invero non avrebbe saputo rispondere. L'amava e non l'amava, così come gli uomini in generale. Istintivamente di animo buono, era più incline ad amare anziché a non amare gli uomini, e così pure quella gente. Ma amarla o non amarla come qualcosa a sé, non poteva, perché non solo viveva con essa, non solo tutti i suoi interessi erano con essa collegati, ma riteneva di farne parte egli stesso, e non vedeva fra se stesso e quella gente nessuna differenza positiva o negativa, e perciò non poteva contrapporsi ad essa. Inoltre, pur vivendo da tempo nei più stretti rapporti coi contadini, e come padrone e come arbitro e soprattutto come consigliere (i contadini avevano fiducia in lui e venivano a lui per consiglio sin da quaranta verste all'intorno), non era riuscito a formarsene, peraltro, un concetto preciso, e si sarebbe trovato imbarazzato a rispondere alla domanda se li amasse oppure no. Dire di conoscere il contadino sarebbe stato per lui come dire di conoscere gli uomini. Conosceva e osservava continuamente uomini di ogni categoria e contadini, che considerava come gli uomini migliori e più interessanti, ma continuamente notava tratti nuovi per cui mutava i giudizi precedenti e ne formulava altri. Sergej Ivanovic, invece, aveva idee del tutto diverse. Come amava e lodava la vita di campagna, contrapponendola a quella che non amava, così pure amava la gente di campagna, contrapponendola a quella categoria di persone che egli non amava: considerava, dunque, il contadino qualcosa di diverso dagli uomini in genere. Nella sua mente ordinata si erano chiaramente fissate le forme definite della vita rurale, tratte, in parte, dalla stessa vita del contadino, ma in prevalenza da quella contrapposizione. Egli non cambiava mai la sua opinione e il suo atteggiamento di simpatia verso i contadini. 

Nella discussione fra i due fratelli sul giudizio sui contadini, Sergej Ivanovic vinceva sempre il fratello, proprio perché Sergej Ivanovic aveva idee precise sul contadino e sul suo carattere, sulle sue peculiarità e usanze; Konstantin Levin, invece, non aveva nessuna idea definita, così che in queste discussioni finiva per convincersi della propria incongruenza. 

Per Sergej Ivanovic il fratello minore era un buon ragazzo, dal cuore ben formato (così egli si esprimeva in francese), dalla mente sia pure abbastanza sveglia, ma influenzabile dalle impressioni del momento, e perciò piena di contraddizioni. Con la condiscendenza di fratello maggiore verso il minore, gli spiegava il senso delle cose, ma non trovava gusto a discutere con lui perché con troppa facilità lo metteva fuori combattimento. 

Konstantin Levin giudicava il fratello un uomo di straordinario ingegno e cultura, nobile nel più alto senso della parola e dotato della facoltà di agire per il bene generale. Ma in fondo all'anima sua, quanto più gli appariva grande e quanto più nell'intimo lo conosceva, tanto più spesso gli veniva in mente che questa facoltà di lavorare per il bene collettivo, della quale egli si sentiva assolutamente sprovvisto, poteva anche non essere un valore concreto, ma piuttosto l'indice dell'insufficienza di qualche cosa; non già di buoni, onesti e nobili propositi e aspirazioni, ma di slancio vitale, di quello che si chiamava ``cuore'', di quell'anelito che costringe l'uomo, fra le innumerevoli vie della vita che gli si parano davanti, a sceglierne una, e a questa sola dedicarsi. Quanto più conosceva il fratello tanto più notava che Sergej Ivanovic e molte altre persone che agivano per il bene comune, non erano stati portati dal cuore verso questo amore per la collettività, ma dal cervello che aveva giudicato esser bene occuparsene, e solo per questo se ne occupavano. Levin si confermò ancor più in questa supposizione nel notare che il fratello si interessava alle questioni sul bene comune o sull'immortalità dell'anima, così come si interessava a una partita a scacchi o al complicato congegno di una macchina nuova. 

Oltre a ciò Konstantin Levin non si trovava a suo agio, in campagna, col fratello, anche perché, specie d'estate, egli era continuamente occupato per l'azienda e non gli bastava neppure la lunga giornata estiva per compiere quanto era necessario, mentre Sergej Ivanovic era in ferie. Ma anche in vacanze, anche senza attendere, cioè, al proprio lavoro, egli era così abituato all'attività intellettuale, che amava esporre in bella e precisa forma le idee che gli venivano in mente, e amava che ci fosse qualcuno ad ascoltarle. E il suo più abituale e naturale ascoltatore era il fratello. Perciò, malgrado l'amichevole semplicità dei loro rapporti, Levin si sentiva imbarazzato a lasciarlo solo. Sergej Ivanovic amava sdraiarsi sull'erba al sole e rimanere a crogiolarsi e a chiacchierare oziosamente. 

- Tu non crederai - diceva al fratello - che piacere è per me quest'ozio degno di un chochol. Neppure un'idea nel cervello, neanche a cercarla col lumicino. 

Ma Konstantin Levin si angustiava a star lì seduto ad ascoltarlo, tanto più che sapeva che proprio in quel momento trasportavano, lui assente, il letame su di un campo non arato e, non sorvegliati, i contadini l'avrebbero ammucchiato Dio sa come; e i dentali negli aratri non li avrebbero svitati, ma strappati e dopo avrebbero detto che gli aratri sono una sciocca invenzione da non potersi paragonare con l'aratro di legno di mastro Andrej, e via di seguito. 

- Ma finiscila di andare su e giù con questo caldo - gli diceva Sergej Ivanovic. 

- No, devo fare una cosa in amministrazione, un attimo solo - diceva Levin e scappava verso i campi. 

\capitolo{II}Nei primi giorni di giugno accadde che Agaf'ja Michajlovna, la njanja e ora governante, portando in cantina un vasetto di funghi allora da lei salati, scivolò e cadde, slogandosi un braccio. Venne il medico condotto, un giovane chiacchierone che da poco aveva terminato gli studi universitari. Osservò il braccio, disse che non s'era affatto slogato, ordinò delle compresse e, rimasto a pranzo, ebbe il piacere di conversare con il famoso Sergej Ivanovic. Gli raccontò, per far mostra del proprio illuminato punto di vista, tutti i pettegolezzi del distretto, lamentando la cattiva condizione degli affari dell'amministrazione distrettuale. Sergej Ivanovic ascoltava attento, faceva delle domande e, eccitato dalla circostanza di avere un nuovo ascoltatore, prese a parlare ed esporre alcune sue giuste e ponderate osservazioni, apprezzate con deferenza dal giovane dottore, ponendosi così in quella lieta disposizione d'animo, nota al fratello, alla quale egli abitualmente perveniva dopo una conversazione brillante e vivace. Quando il dottore se ne fu andato, Sergej Ivanovic manifestò il desiderio di andare sul fiume a pescare con la lenza. Gli piaceva pescare con la lenza, ed era quasi orgoglioso di provar piacere in un'occupazione così sciocca. 

Konstantin Levin, che doveva andare a sorvegliare l'aratura e sui prati, si offrì di accompagnarlo in calesse. 

Si era al colmo dell'estate, quando il raccolto dell'annata in corso è già assicurato e cominciano le cure della semina per l'anno nuovo e si avvicina la fienagione; quando la segale grigioverde, tutta in spighe, ma non turgida, con la pannocchia ancora leggera, ondeggia al vento; quando le avene verdi, coi cespi d'erba gialla sparsa qua e là, spiccano fra le seminagioni tardive; quando il grano saraceno primaticcio già matura, ricoprendo il terreno; quando i maggesi, calpestati dal bestiame fino a diventar di pietra e coi viottoli rimasti intatti perché il vomere non li addenta, sono arati fino a metà; quando i mucchi del concio disseccato, all'aperto, odorano all'alba insieme alle erbe mielate, e quando sui pianori, simili a un mare ininterrotto, si distendono, in attesa della falce, i prati circondati dai mucchi nereggianti degli steli dell'acetosella estirpata. 

Era il tempo in cui nel lavoro dei campi subentra una breve pausa prima di iniziare il raccolto che ogni anno ridesta tutte le energie dei campagnoli. Il raccolto si presentava splendido e le giornate estive erano chiare, calde, con brevi notti rugiadose. 

I fratelli dovevano attraversare il bosco per giungere ai prati. Sergej Ivanovic lungo il percorso non si stancava di ammirare la bellezza del bosco soffocato dal fogliame, e mostrava al fratello ora un vecchio tiglio, scurito nella parte ombrosa, screziato di stipole gialle già pronte a fiorire, ora i giovani germogli verde smeraldo, rilucenti sugli alberi. Konstantin Levin non amava parlare, né sentir parlare della bellezza della natura. Le parole, per lui, toglievano l'incanto di quello che vedeva. Faceva eco al fratello, ma istintivamente pensava ad altro. Quando ebbero attraversato il bosco, tutta la sua attenzione fu attratta da un maggese su di una collina, ricoperto in un punto di chiazze gialle d'erba secca, in un altro battuto e tagliato a riquadri, in un altro ricoperto di mucchi di letame, e in un altro ancora perfino arato. Attraverso il campo andavano in fila dei carri. Levin li contò e fu contento pensando che così sarebbe stato portato via tutto quello che si doveva, e alla vista dei prati i suoi pensieri si rivolsero alla questione della falciatura. Quando si doveva provvedere alla raccolta del fieno, egli provava sempre qualcosa che lo toccava nel vivo. Accostandosi al prato, fermò il cavallo. 

C'era ancora guazza nel folto del prato e Sergej Ivanovic, per non bagnarsi i piedi, chiese d'esser portato in calesse fino al cespuglio di citiso presso cui si pescava il pesce persico. Per quanto dispiacesse a Konstantin Levin di calpestare l'erba, entrò nel prato. L'erba alta si avvinceva morbida intorno alle ruote del calesse e alle zampe del cavallo, lasciando i semi sui raggi bagnati e sui mozzi. 

Sergej Ivanovic, approntate le lenze, sedette sotto il cespuglio e Levin allontanò il cavallo, lo legò, ed entrò nell'immenso mare grigioverde del prato non mosso dal vento. L'erba, morbida come seta, coi semi maturi, gli arrivava fin quasi alla cintola nel luogo fecondato dalle piene. 

Attraversato di sghembo il prato, Konstantin Levin uscì sulla strada e incontrò un vecchio con un occhio gonfio che portava uno sciame di api. 

- Oh che, ne hai prese delle altre, Formic? - chiese. 

- Altro che prendere, Konstantin Dmitric! A stento ti restano le tue! Ecco che è scappata per la seconda volta la regina\ldots{} Grazie, i ragazzi sono arrivati di galoppo. Da voi arano. Hanno staccato il cavallo, sono arrivati di galoppo\ldots{} 

- Be', che ne dici, Formic, si deve falciare ora o aspettare ancora? 

- Macché! Da noi si deve aspettare fino al giorno di san Pietro. Voi invece falciate sempre prima. Ma se Dio vuole, le erbe son buone. Il bestiame ne avrà a sazietà. 

- E il tempo, come credi che sia? 

- Questo è affar di Dio. Può darsi che anche il tempo sia buono. 

Levin si avvicinò al fratello. Non un pesce abboccava, ma Sergej Ivanovic non s'annoiava, e sembrava nella più lieta disposizione di spirito. Levin si accorse che, eccitato dalla conversazione col dottore, avrebbe voluto parlare un po'; egli, invece, voleva tornarsene a casa a convocare i falciatori per l'indomani e risolvere la questione della falciatura che lo occupava tanto. 

- Be', andiamo - disse. 

- Affrettarsi per andar dove? Rimaniamo a sedere un po'. Anche senza pescar nulla, si sta bene qui. Ogni caccia è buona perché mette a contatto con la natura. Eh, che delizia quest'acqua d'acciaio! - egli disse. - I bordi di questi prati - continuò - mi ricordano sempre un vecchio indovinello, lo conosci? ``L'erba dice all'acqua: e noi ondeggeremo, ondeggeremo''. 

- No, non lo conosco - rispose Levin con tristezza. 

\capitolo{III}-E sai, ho pensato a te - disse Sergej Ivanovic. - Non c'è nulla di paragonabile a quello che avviene nel vostro distretto, a quanto dice quel dottore; ma non è mica sciocco quel giovane. E io ti ho detto e ti ripeto: non è bene che tu non vada alle riunioni e che in genere ti renda estraneo all'attività del consiglio distrettuale. Se le persone dabbene se ne allontanano, tutto andrà, s'intende, Dio sa come. Le tasse che si pagano, servono per gli stipendi, ma non vi sono scuole, né infermieri, né levatrici, né farmacie, non c'è nulla. 

- Ma io ho provato - rispose piano e svogliato Levin - non posso! Ebbene, che fare? 

- Che cosa non puoi? Io, confesso, non capisco. L'indifferenza, l'inesperienza, non le ammetto; possibile che sia solo pigrizia? 

- Né la prima, né la seconda, e nemmeno la terza. Ho provato e credo di non poterci far nulla - disse Levin. 

Egli non prestava attenzione a quello che diceva il fratello. Guardava l'aratura di là dal fiume, e scorgeva qualcosa di scuro senza riuscire a distinguere se fosse un cavallo o l'amministratore a cavallo. 

- Perché non puoi farci nulla? Hai fatto una prova e secondo te non è andata bene e ti rassegni. Ma com'è che non hai amor proprio? 

- L'amor proprio - disse Levin, punto nel vivo dalle parole del fratello - io non lo capisco. Se all'università mi avessero detto che gli altri capivano il calcolo integrale e io no, allora ci sarebbe entrato l'amor proprio. Ma qui bisogna prima esser convinti di avere delle speciali attitudini a queste cose e, quel che più conta, esser convinti che queste cose siano molto importanti. 

- Eh, già! Che forse tutto ciò non è importante? - disse Sergej Ivanovic, tocco nel vivo perché il fratello non trovava importante quel che interessava lui e perché, evidentemente, non lo ascoltava quasi. 

- Non mi sembra importante, non mi tocca, che vuoi mai?\ldots{} - rispose Levin mentre s'accorgeva che quel che vedeva era l'amministratore, e l'amministratore, probabilmente, aveva mandato via gli operai dall'aratura. Essi voltavano gli aratri. ``Possibile che abbiano già arato?'' pensò. 

- Su, ma ascolta - disse il fratello maggiore, corrugando il suo bel viso intelligente - vi sono dei limiti a tutto. È molto bello essere un originale e un uomo schietto e spregiare ogni falsità, questo lo so; ma ecco, quello che tu dici, o non ha senso, o ha un senso tutt'altro che buono. Come puoi trovare poco importante che questa popolazione che tu ami, come mi assicuri\ldots{} 

``Io non l'ho mai assicurato'' pensò Konstantin Levin. 

- \ldots{} muoia senza aiuti? Queste mammane fanno morir di fame i bambini e il popolo marcisce nell'ignoranza e rimane in potere di un qualsiasi scribacchino, mentre tu hai in mano i mezzi per riparare a questo, e non te ne dài pensiero perché, secondo te, la cosa non è importante. - E Sergej Ivanovic gli pose il dilemma: - O sei così poco evoluto da non riuscire a intravedere tutto quello che puoi fare, o non vuoi rinunciare alla tua tranquillità, alla tua vanità o che so io, per fare ciò. 

Konstantin Levin sentiva che non gli restava ormai che dichiararsi vinto e confessare la mancanza di interesse per una causa comune. E questo lo offendeva e lo addolorava. 

- E l'uno e l'altro - disse reciso - non vedo proprio come si possa\ldots{} 

- Come? Non si può, ripartendo bene il denaro, creare un'assistenza medica? 

- Non si può, a quanto pare. Per le quattromila verste quadrate del nostro distretto, con le nostre zazory, con le tempeste di neve, con la stagione dei lavori, non vedo la possibilità di dare in ogni luogo un'assistenza medica. E poi, in genere, io non credo alla medicina. 

- Via, permettimi, questo è ingiusto\ldots{} Io ti porterò migliaia di esempi\ldots{} via, e le scuole? 

- Perché le scuole? 

- Che dici? Può esservi mai dubbio sull'utilità delle scuole? Se la scuola è buona per te, lo è anche per gli altri. 

Konstantin Levin si sentiva moralmente messo con le spalle al muro e perciò si accalorava dando prova, senza volerlo, della sua indifferenza al benessere collettivo. 

- Può darsi che tutto questo vada bene; ma io, perché devo curarmi di istituire dei posti di assistenza medica di cui non farò mai uso, e delle scuole dove non manderò certo i miei figli, dove neanche i contadini vorranno mandare i loro e dove non credo ancora che proprio ci si debbano mandare? - disse. 

Questo modo inatteso di vedere la questione disorientò Sergej Ivanovic per un attimo; ma subito egli preparò un nuovo piano di attacco. 

Stette un po' di tempo in silenzio, tirò fuori un amo, lo gettò in acqua e, sorridendo, si volse al fratello. 

- Su, permettimi\ldots{} In primo luogo, il posto di assistenza medica è servito anche a te. Ecco, noi per Agaf'ja Michajlovna abbiamo mandato a chiamare il medico condotto. 

- Già ma io penso che il braccio resterà storto. 

- Questo è ancora da vedere\ldots{} Poi un contadino, un lavoratore istruito ti è più utile e più accetto. 

- No, domanda a chi vuoi - rispose deciso Konstantin Levin - uno che sappia leggere e scrivere, come lavoratore, è peggiore degli altri. E le strade non si possono fare aggiustare; e i ponti, appena messi a posto, li portano via. 

- Del resto - disse, aggrottando le sopracciglia Sergej Ivanovic, che non amava le contraddizioni e particolarmente quelle che saltavano continuamente di palo in frasca e senza alcuna connessione introducevano nella discussione elementi nuovi, così che non si poteva sapere a quali di essi rispondere - del resto non si tratta di questo. Permetti. Riconosci che l'istruzione è un bene per il popolo? 

- Lo riconosco - disse Levin senza riflettere, e subito pensò di non aver detto quello che pensava. Sentiva che, riconoscendo ciò, gli sarebbe stato dimostrato che diceva delle sciocchezze che non avevano alcun senso. Come questo gli sarebbe stato dimostrato, non lo sapeva, ma sapeva che, senza dubbio, gli sarebbe stato dimostrato, a fil di logica, e aspettava questa dimostrazione. 

La dimostrazione fu più semplice di quella che Levin si aspettava. 

- Se riconosci come un bene l'istruzione - disse Sergej Ivanovic - allora tu, come uomo onesto, non puoi non amare e non aderire a quest'opera e non desiderare di lavorare per essa. 

- Ma io ancora non la riconosco buona - disse arrossendo Levin. 

- Come? O ora hai detto di sì\ldots{} 

- Cioè, non la riconosco né buona né possibile. 

- Questo non lo puoi sapere, senza aver prima fatto tutti i tentativi. 

- Su, ammettiamo - disse Levin, sebbene non lo ammettesse per nulla - ammettiamo pure che sia così; ma io tuttavia non vedo la necessità di dovermi affannare per questo. 

- Sarebbe a dire? 

- No, giacché abbiamo preso a parlarne, spiegamelo dal lato filosofico - disse Levin. 

- Non capisco cosa c'entri qui la filosofia - disse Sergej Ivanovic, con un tono che a Levin parve tale da non volergli riconoscere il diritto di discutere di filosofia, e questo lo irritò. 

- Ecco come - disse, accalorandosi. - Io penso che il movente di tutte le nostre azioni sia l'interesse personale. Ora nelle istituzioni provinciali io, nella mia qualità di nobile, non ci vedo nulla che cooperi al mio benessere. Le strade non diventano migliori e, se pure rimangono quali sono, i miei cavalli mi portano anche per quelle cattive. Del dottore e del posto di assistenza medica non ho bisogno; il giudice conciliatore non mi occorre; io non mi rivolgo e non mi rivolgerò mai a lui. Le scuole non solo non mi occorrono, ma mi sono persino dannose, come ti ho detto. Per me le istituzioni distrettuali hanno il solo scopo di obbligarmi a pagare diciotto copeche per desjatina, e farmi andare in città a pernottare con le cimici e ascoltare ogni sorta di sciocchezze e brutture: ed in questo l'interesse personale non mi stimola affatto. 

- Permettimi - interruppe con un sorriso Sergej Ivanovic - l'interesse personale non ci stimolava a lavorare per la liberazione dei contadini, eppure noi abbiamo lavorato! 

- No - interruppe, sempre più accalorandosi, Konstantin. - La liberazione dei contadini era un'altra cosa. Lì c'era, sì, un interesse personale. Volevamo scrollar da noi questo giogo che opprimeva tutti noi uomini giusti. Ma essere delegato, discutere sulla quantità necessaria di cloache e sulla maniera di far passare le fogne in una città in cui non vivo; essere giurato e giudicare un contadino che ha rubato un prosciutto e ascoltare per sei ore di seguito tutte le sciocchezze che inventano i difensori e i procuratori e star lì a sentire come il presidente interroga il vecchio Alëška, lo scemo che sta da me: ``Confessate, voi, signor imputato, il furto del prosciutto?''. ``Eh?''. 

Konstantin Levin aveva già smarrito il filo del discorso e s'era messo a rifare il presidente e Alëška lo scemo, e gli pareva che tutto questo riguardasse la questione. 

Ma Sergej Ivanovic alzò le spalle. 

- Ebbene, con questo che vuoi dire? 

- Io voglio dire che quei diritti che mi\ldots{} che toccano il mio interesse personale, io li difenderò sempre con tutte le mie forze; che quando eravamo studenti e i gendarmi facevano le perquisizioni e leggevano le nostre lettere, io ero pronto con tutte le mie forze a difendere i miei diritti, a difendere il mio diritto alla libertà e alla cultura. Capisco il servizio militare perché interessa la sorte dei miei figli, dei miei fratelli e di me stesso; sono pronto a giudicare tutto quanto mi riguarda; ma giudicare se e come distribuire quarantamila rubli di denaro del distretto o giudicare Alëška lo scemo, io questo non lo capisco e non posso farlo. 

Konstantin Levin parlava come se si fosse rotta la diga che tratteneva la sua loquela; Sergej Ivanovic sorrideva. 

- E domani potrai essere giudicato tu stesso: ti piacerebbe forse essere giudicato dalla vecchia Camera criminale? 

- Io non sarò giudicato. Io non sgozzerò nessuno, e non ne avrò bisogno. Su via! - continuò, passando di nuovo a cosa che non riguardava affatto la questione - le nostre istituzioni distrettuali e tutto il resto somigliano alle piccole betulle che noi ficchiamo in terra dovunque il giorno di Pentecoste, perché sembrino un bosco venuto su spontaneamente in Europa; ma io non posso innaffiare e credere in queste piccole betulle con tutta l'anima. 

Sergej Ivanovic alzò le spalle, esprimendo con questo gesto la sua meraviglia per queste betulle spuntate ora nella questione chi sa mai da quale parte; mentre aveva capito subito a cosa volesse alludere il fratello. 

- Scusami, ma così non si può ragionare - osservò. 

Ma Konstantin Levin voleva giustificare quella manchevolezza che riconosceva in se stesso, l'indifferenza cioè verso il bene comune e continuò. 

- Io penso - disse che nessuna attività può essere salda se non ha le radici nell'interesse personale. Questa è una verità d'ordine generale, filosofico - disse, ripetendo con intenzione la parola ``filosofico'', quasi desiderasse mostrare che anche lui aveva il diritto, come tutti, di parlare di filosofia. 

Sergej Ivanovic ancora una volta sorrise. ``E anche lui - pensò - ha una certa filosofia al servizio delle proprie tendenze''. 

- Su via, la filosofia lasciala stare - disse. - Il compito della filosofia di tutti i secoli consiste proprio nel trovare il legame indispensabile fra l'interesse personale e quello generale. Ma questo non riguarda la questione, mentre, per quel che la concerne, io devo soltanto correggere il tuo paragone. Le betulle non sono conficcate, ma alcune sono piantate e altre seminate; e a queste ultime ci si deve rivolgere con maggior cura. Soltanto i popoli che guardano all'avvenire, soltanto quelli si possono chiamare storici, quelli che sentono ciò che è importante e significativo nelle loro istituzioni, e ne hanno cura. 

E Sergej Ivanovic trasportò la questione sul terreno storico-filosofico inaccessibile a Konstantin Levin, dimostrandogli tutta l'infondatezza del suo punto di vista. 

- Che questo poi non ti piaccia, questo, perdonami, fa parte della nostra pigrizia russa e del barstvo, e io sono sicuro che, quanto a te, si tratta di una deviazione momentanea che passerà. 

Konstantin taceva. Sentiva d'essere sconfitto da ogni lato, ma nello stesso tempo sentiva che quello che egli intendeva dire non era stato capito dal fratello, non sapeva bene perché: perché non aveva saputo esporlo lui chiaramente o perché il fratello non aveva voluto o non aveva potuto capirlo? Ma non stette a riflettere e, senza replicare, cominciò a pensare a una faccenda del tutto diversa, tutta sua personale. 

Sergej Ivanovic avvolse l'ultimo amo, slegò il cavallo e insieme si avviarono. 

\capitolo{IV}La faccenda personale che era venuta in mente a Levin durante la conversazione col fratello, era questa: l'anno precedente, recatosi un giorno ad assistere alla fienagione, e irritatosi con l'amministratore, aveva adoperato, per riconquistare la propria calma, il solito suo sistema: aveva tolto dalle mani di un contadino la falce e s'era messo a falciare. 

Questo lavoro gli era piaciuto tanto che diverse altre volte aveva falciato; aveva falciato tutto il prato davanti alla casa, e per questo, fin dalla primavera, si era proposto di falciare insieme con i contadini per giornate intere. Da quando era arrivato il fratello era in dubbio: falciare o no? Gli rincresceva lasciare il fratello solo per giornate intere, e poi temeva che non avesse a prendersi giuoco di lui per questo. Ma, camminando per il prato, ricordando le impressioni della falciatura, aveva deciso di falciare. Ora, dopo il colloquio irritante avuto col fratello, s'era nuovamente ricordato della decisione. 

``Ho bisogno di movimento fisico, altrimenti il mio carattere si guasta'' pensò e decise di andare a falciare, pur rincrescendogli di fronte al fratello e alla gente. 

La sera Konstantin passò in amministrazione, diede le disposizioni per i lavori e mandò in giro per i villaggi a convocare per l'indomani i falciatori per il prato Kalinovyj, il più grande e il migliore. 

- E la mia falce mandatela a Tit perché me l'affili e me la porti domani; forse falcerò anch'io - disse, cercando di non turbarsi. 

L'amministratore sorrise e disse: 

- Sissignore. 

La sera, al tè, Levin lo disse anche al fratello. 

- Sembra che il tempo si sia messo al bello. Domani comincio a falciare. 

- Mi piace molto questo lavoro - disse Sergej Ivanovic. 

- A me straordinariamente. Io stesso ho falciato qualche volta insieme con i contadini, e domani voglio falciare tutta la giornata. 

Sergej Ivanovic alzò la testa e guardò con curiosità il fratello. 

- E così, al pari dei contadini, tutta la giornata? 

- Sì, è una cosa piacevole - disse Levin. 

- È bellissimo come esercizio fisico, ma è difficile che tu possa farcela - disse Sergej Ivanovic, senza alcuna ironia. 

- Ho provato. In principio è duro, poi ci si abitua. Io penso che non resterò indietro\ldots{} 

- Ecco, ma di' un po', che ne pensano i contadini? Probabilmente rideranno della stramberia del signore. 

- No, non credo; ma è un lavoro così piacevole e nello stesso tempo così difficile che non si ha il tempo di pensare. 

- E così tu pranzerai con loro? Mandarti là del Lafite e un tacchino arrosto non sta mica bene. 

- No, ma io, durante la sosta del lavoro, verrò a casa. 

La mattina dopo Konstantin Levin si alzò più presto del solito, ma le disposizioni da dare per l'azienda lo trattennero e, quando giunse, i falciatori andavano già per la seconda falciata. 

Sin dall'alto della collina gli si era rivelata la parte in ombra del prato, quella già tagliata, con le falciate d'erba grigiastra e i mucchi neri dei gabbani dei falciatori tolti nel punto dal quale avevano preso l'avvio per la prima falciatura. 

A misura che si avvicinava, scorgeva i contadini in fila, uno dietro l'altro, alcuni coi gabbani, altri con la sola camicia, che menavano la falce in modo vario. Ne contò quarantadue. 

Si movevano lentamente per il fondo ineguale del campo dove c'era una vecchia diga. Levin riconosceva già qualcuno di loro. C'era il vecchio Ermil con la camicia bianca molto lunga che menava la falce stando curvo; c'era Vas'ka, il giovane che stava da Levin come cocchiere, e che prendeva la falciata con tutta la forza del braccio. C'era anche Tit, un contadino piccolo e asciutto, che aveva iniziato Levin alla fienagione. Andava avanti senza curvarsi, come se giocasse con la falce nel tagliare la sua larga falciata. 

Levin scese dal cavallo e, legatolo presso la strada, raggiunse Tit che, presa da un cespuglio un'altra falce, gliela diede. 

- È pronta, padrone, taglia come un rasoio, falcia da sé - disse Tit con un sorriso, togliendosi il berretto e dandogli la falce. 

Levin prese la falce e cominciò a provare. I falciatori che avevano finito la loro fila, uscivano sudati e allegri, uno dopo l'altro, sulla strada e salutavano, sorridendo, il padrone. Tutti lo guardavano, ma nessuno aprì bocca finché un vecchio, uscendo sulla strada, alto, col viso rugoso e glabro, con un giubbotto di montone, si rivolse a lui. 

- Attento a te, padrone. Se hai preso l'avvio, non restare addietro! - disse, e Levin udì un riso contenuto fra i falciatori. 

- Cercherò di non restare addietro - disse, mettendosi accanto a Tit e aspettando il momento per cominciare. 

- Bada a te - ripeté il vecchio. 

Tit fece posto a Levin che gli tenne dietro. L'erba era bassa, vicino alla strada, e Levin, che da tempo non falciava e si sentiva confuso sotto gli sguardi di tutti, falciò male al primo momento, pur agitando con forza la falce. Dietro di lui si sentirono delle voci. 

- È impostata male, il manico è troppo alto; guarda come deve abbassarsi - disse uno. 

- Pòggiati di più col tallone - disse un altro. 

- Non fa niente, va bene, taglia lo stesso - continuò il vecchio. - Guarda\ldots{} è andata\ldots{} Stai prendendo la falciata troppo larga, ti stancherai\ldots{} Il padrone, non c'è che dire, si sforza per sé. Ma guarda che sgorbio! Per una cosa simile a noi ce la danno sul groppone. 

L'erba diventò più morbida, e Levin, ascoltando senza rispondere, cercando di falciare come meglio poteva, teneva dietro a Tit. Erano andati avanti di cento passi. Tit procedeva senza fermarsi: ma Levin aveva già il terrore di non resistere, tanto era stanco. 

Sentiva che ormai falciava con le sue ultime riserve, e decise di pregare Tit di fermarsi. Ma proprio in quel momento Tit si fermò per conto suo e, chinatosi, prese dell'erba, asciugò la falce e si mise ad affilarla. Levin si raddrizzò e, dopo aver respirato, si guardò in giro. Dietro di lui procedeva un contadino che, evidentemente, era stanco anche lui, perché subito, senza raggiungere Levin, si fermò e prese ad affilare. Tit finì di affilare la falce sua e quella di Levin, e insieme proseguirono. 

Alla seconda ripresa fu lo stesso. Tit procedeva, un colpo dietro l'altro, senza fermarsi e senza stancarsi. Levin lo seguiva, sforzandosi di non restare indietro, ma gli era sempre più difficile: veniva il momento in cui sentiva di non avere più forze, ma proprio in quel momento Tit si fermava e si metteva ad affilare. 

Così passarono la prima falciata. E questa lunga falciata parve particolarmente difficile a Levin; in compenso quando fu terminata e Tit, gettandosi la falce sulla spalla, si mise a passo lento a percorrere, sulle orme lasciate dai tacchi, la falciata, anche Levin s'incamminò sulla propria. E sebbene il sudore gli scendesse a rivoli per il viso e gocciolasse giù dal naso e tutta la schiena fosse bagnata, come immersa nell'acqua, egli si sentiva bene. Lo rallegrava in modo particolare la sicurezza di poter resistere. 

La sua soddisfazione era amareggiata solo dal fatto che la falciata non gli riusciva bene. ``Moverò meno la mano e più il torso'' pensava, confrontando la falciata di Tit come tesa su di un filo, con la sua sparpagliata e disposta in modo ineguale. 

Nel passare la prima falciata, Tit, come aveva notato Levin, era andato particolarmente in fretta, forse per mettere alla prova il padrone e la falciata era capitata lunga. Le altre erano già più facili; Levin tuttavia doveva tendere tutte le sue forze per non restare indietro ai contadini. 

Egli non pensava a nulla, non desiderava nulla, altro che non restare indietro ai contadini e terminare nel modo migliore. Sentiva solo lo stridere delle falci e vedeva dinanzi a sé la figura diritta di Tit che si allontanava, il semicerchio curvo del terreno falciato, le erbe e le corolle dei fiori che si chinavano lente, a onda, intorno alla lama della falce e dinanzi a sé il termine della falciata, là dove sarebbe giunto il riposo. 

Nel mezzo del lavoro, senza capir che fosse e donde venisse, provò improvvisamente una piacevole sensazione di fresco giù per le spalle accaldate e sudate. Guardò il cielo mentre affilava la falce. Una nuvola bianca e greve s'era addensata e ne veniva giù una pioggia pesante. Alcuni contadini corsero ai gabbani e se li infilarono; altri, come Levin, si strinsero nelle spalle con gioia sotto la piacevole rinfrescata. 

Passarono ancora una falciata e poi ancora un'altra. Passavano falciate lunghe e corte, con l'erba buona e con l'erba cattiva. Levin aveva perso ogni nozione del tempo e proprio non sapeva se fosse tardi o presto. Nel suo lavoro si era verificato un cambiamento che gli fece grande piacere. Mentre lavorava, aveva dei momenti nei quali dimenticava quello che faceva, si sentiva leggero, e proprio in quei momenti la falciata gli veniva fuori uguale e bella quasi come quella di Tit. Ma appena si ricordava di quello che faceva, e si sforzava di far meglio, provava subito tutta la pesantezza del lavoro e la falciata gli riusciva male. 

Passata un'altra falciata, egli voleva di nuovo riprendere a camminare, ma Tit si fermò, e accostandosi al vecchio, gli disse qualcosa sottovoce. Guardarono insieme il sole. ``Di che stanno a parlare, e perché non continua a falciare?'' pensò Levin, senza rendersi conto che i contadini avevano falciato ininterrottamente non meno di quattro ore e che per loro era tempo di far colazione. 

- A colazione, padrone - disse il vecchio. 

- È forse ora? Di già a colazione? 

Levin rese la falce a Tit e, insieme coi contadini, che si erano avviati verso i gabbani a prendere il pane, si avviò verso il cavallo in mezzo alle falciate leggermente spruzzate di pioggia del lungo spazio lavorato. Ora soltanto capì che non aveva indovinato il tempo giusto e che la pioggia avrebbe rovinato il fieno. 

- Sciuperà il fieno - disse. 

- Non fa nulla, padrone: con la pioggia falcia, col bel tempo rastrella! - disse il vecchio. 

Levin sciolse il cavallo e andò a casa a prendere il caffè. 

Sergej Ivanovic s'era appena alzato. Preso il caffè, Levin tornò a falciare, prima che Sergej Ivanovic facesse in tempo a vestirsi e a venire in sala da pranzo. 

\capitolo{V}Dopo la colazione, Levin non capitò più, nella fila, al posto di prima, ma fra il vecchio scherzoso che l'aveva invitato ad essere suo vicino e il contadino giovane, sposato solo dall'autunno, e che era venuto a falciare per la prima volta. 

Il vecchio, tenendosi diritto, andava avanti con un movimento eguale ed ampio delle gambe ricurve, e con un gesto preciso e uniforme, che ormai non gli costava, evidentemente, più che il dimenar delle braccia nel camminare, tagliava una falciata eguale, alta, come se giocasse. Proprio come se non lui, ma la falce affilata tagliasse da sola l'erba sugosa. 

Dietro a Levin andava il giovane Miška. Il giovane dal viso simpatico, coi capelli stretti da un laccio d'erba fresca, lavorava sempre con sforzo; ma appena lo guardavano, sorrideva. Evidentemente era pronto a morire anzi che confessare di far fatica. 

Levin camminava fra loro due. Nel pieno del caldo la falciatura non gli parve tanto difficile. Il sudore che lo inondava lo rinfrescava, e il sole che gli bruciava la schiena, la testa e il braccio dalla manica rimboccata fino al gomito, dava vigore e tenacia al lavoro; e sempre più spesso gli capitavano quei tali momenti di incoscienza, in cui si può non pensare a quello che si fa. La falce allora tagliava da sola. Erano questi i momenti felici. Ancora più felici quelli in cui, avvicinandosi al fiume verso il quale andavano a finire le falciate, il vecchio puliva con l'erba umida e folta la falce, ne sciacquava l'acciaio nell'onda fresca, vi immergeva un barattolo e lo offriva a Levin. 

- Su, ecco il mio kvas! Buono, eh? - diceva, ammiccando. 

E invero Levin non aveva mai bevuto una bevanda simile a quell'acqua tiepida con l'erba che ci sguazzava dentro e il senso di ruggine della latta del barattolo. E subito dopo seguiva una beata, lenta passeggiata con la mano sulla falce, durante la quale ci si poteva asciugare il sudore che scorreva a rivoli, si poteva respirare a pieni polmoni e si poteva guardare la schiera disseminata dei falciatori e tutto quello che avveniva in giro nel bosco e nel campo. 

Quanto più a lungo Levin falciava, tanto più spesso sentiva dei momenti di oblio durante i quali non eran le mani che menavano la falce, ma la falce stessa che trascinava con sé tutto il corpo di lui, cosciente e pieno di vita; e allora, come per incanto, senza pensarci, il lavoro si compiva da sé, regolare e preciso. Erano questi i momenti più beati. 

La cosa diveniva difficile solo quando si doveva far cessare questo moto inconsapevole e bisognava riflettere: quando cioè si doveva o falciare intorno a un monticello o intorno all'acetosella non estirpata. Il vecchio lo faceva con facilità. S'imbatteva in un monticello, ed ecco cambiava movimento, e dove col tallone, dove con l'estremità della falce abbatteva il monticello da tutte e due le parti a piccoli colpi. E nel far questo guardava e osservava sempre quello che gli si parava innanzi; ora strappava una radichetta, la mangiava o l'offriva a Levin, ora gettava via con la punta della falce un ramo, ora osservava un piccolo nido di quaglie, dal quale, proprio di sotto alla falce, volava via la femmina, ora afferrava una vipera capitatagli sul cammino, e alzandola con la falce, come su di una forchetta, la mostrava a Levin e la buttava via. 

A Levin invece e al giovane dietro di lui, queste variazioni di movimento riuscivano difficili. Tutti e due, dato l'avvio ad un unico movimento di tensione, si trovavano presi nella foga del lavoro e non erano in grado di mutar movimento e di osservare nel tempo stesso quello che si parava innanzi. 

Levin non s'accorgeva dello scorrer del tempo. Se gli avessero chiesto quanto tempo era che falciava, avrebbe risposto da una mezz'ora, e invece s'era già avvicinata l'ora del desinare. Avviandosi per la falciata, il vecchio richiamò l'attenzione di Levin su alcune bambine e alcuni ragazzetti che da varie parti, appena visibili, camminavano fra l'erba alta e sulla strada verso i falciatori, portando il pane e le brocche di kvas, chiusi in fagotti di stracci, che stiravano loro le piccole braccia. 

- Guarda, i moscerini che strisciano! - disse, indicandoli e, facendosi schermo con la mano, guardò il sole. 

Passarono altre due falciate e il vecchio si fermò. 

- Su, via, padrone, a mangiare! - disse deciso. E, avviandosi al fiume, i falciatori si diressero in mezzo alle falciate, verso i gabbani, accanto ai quali sedevano, aspettandoli, i bambini che avevano portato il desinare. I contadini si riunirono, alcuni lontani sotto i carri, altri vicino presso un ciuffo di citiso sotto il quale avevano gettato dell'erba. 

Levin sedeva accanto a loro; non aveva voglia di andarsene. 

Ogni imbarazzo di fronte al padrone era ormai scomparso da un pezzo. I contadini si preparavano a mangiare. Alcuni si lavavano, i giovani facevano il bagno nel fiume, altri si accomodavano un posto per la siesta, scioglievano gli involti col pane e aprivan le brocche col kvas. Il vecchio sbriciolò del pane nella ciotola, l'impastò col manico del cucchiaio, versò dell'acqua dalla brocca, tagliò ancora del pane, e, sparsovi sopra del sale, si volse verso oriente per pregare. 

- Ecco, barin, prendi la mia tjur'ka - disse, mettendosi in ginocchio davanti alla ciotola. 

La zuppa era così gustosa che Levin decise di non andare a casa a pranzare. Pranzò col vecchio e si mise a discorrere con lui delle sue faccende di casa, prendendovi il più vivo interesse; gli parlò poi di tutte le proprie cose e con tutti i particolari che potevano interessare il vecchio. Si sentiva più vicino a lui che al fratello, e involontariamente sorrideva per la tenerezza che provava per quell'uomo. Quando il vecchio si alzò di nuovo e, dopo aver pregato e dopo essersi approntato un fascio d'erbe, si sdraiò lì sotto al cespuglio, Levin fece lo stesso, e malgrado le mosche e i moscerini appiccicosi e molesti che gli solleticavano il viso e il corpo sudati, si addormentò immediatamente, e si svegliò solo quando il sole, dall'altra parte del cespuglio, cominciò a raggiungerlo. Il vecchio già da tempo era sveglio e sedeva affilando le falci dei giovani. 

Levin guardò attorno e non riconobbe il luogo; tanto era cambiato tutto. Un enorme spazio del campo era stato falciato e brillava di uno splendore particolare, nuovo, con le falciate che odoravano sotto i raggi obliqui del sole calante. E i cespugli intorno ai quali s'era falciato, vicino al fiume, e lo stesso fiume prima invisibile e ora risplendente d'acciaio nelle sue anse, e i contadini che si movevano e si sollevavano e la parete erta dell'erba del campo non ancora falciato, e gli sparvieri che roteavano sul prato spoglio, tutto questo era affatto nuovo. Risvegliatosi, Levin cominciò a considerare quanto era stato falciato e quanto ancora si poteva falciare nella giornata. 

S'era lavorato proprio di buona lena, tenendo conto che gli operai erano quarantadue. Tutto il prato grande, che al tempo della servitù si falciava in due giorni con trenta opre, era già stato falciato. Restavano solo gli angoli delle falciate corte. Ma Levin voleva falciare quanto più era possibile per quel giorno, e se la prendeva col sole che calava così presto. Non sentiva più nessuna stanchezza; voleva solo lavorare sempre più svelto e sempre di più. 

- E ce la faremo a falciare anche il Maškin Verch? che ne dici? - disse al vecchio. 

- Come Dio vuole, il sole non è alto. Posso promettere un po' di vodka ai ragazzi? 

Così durante la refezione, quando di nuovo si furon seduti e i fumatori si erano messi a fumare, il vecchio fece intendere ai ragazzi che ``a falciare il Maškin Verch ci sarebbe stata la vodka''. 

- E che, non falciarlo? Via, Tit! Sbrighiamoci alla svelta. Finirai di mangiare stanotte! va', va'! - si sentirono delle voci e, terminando di mangiare il pane, i falciatori si misero subito in cammino. 

- Su, ragazzi, forza! - disse Tit e, quasi al trotto, andò avanti. 

- Va', va', - diceva il vecchio, canterellando dietro di lui, e, dopo averlo raggiunto facilmente: - taglio! Bada! 

E giovani e vecchi falciavano come a gara. Ma pur facendo in fretta, non sciupavano l'erba e le falciate si adagiavano in modo preciso e netto. Il tratto di campo che era rimasto in angolo fu tagliato in cinque minuti. Non ancora gli ultimi falciatori tagliavano la falciata, che già quelli avanti avevano gettato i gabbani sulle spalle e si avviavano sulla strada verso il Maškin Verch. 

Il sole inclinava già verso gli alberi, quando i falciatori, con rumor di ciotole, entrarono nel piccolo burrone boscoso del Maškin Verch. L'erba al centro del vallone arrivava alla cintola, ed era tenera e morbida, soffice, colorata qua e là di violacciocche. 

Dopo un breve parlottare: se andare in lungo o in largo, Prochor Ermilin, un bravo falciatore anche lui, un contadino enorme, abbronzato, andò avanti. Andò avanti per una falciata, si voltò indietro e fece largo, e tutti cominciarono ad allinearsi dietro di lui, procedendo in discesa per il vallone, e in salita accanto al margine del bosco. Il sole era calato dietro il bosco. Cadeva già la brina: soltanto i falciatori che erano sull'altura erano esposti al sole, ma in basso, dove si era levata la nebbia, e di lato, procedevano all'ombra fresca, rugiadosa. Il lavoro ferveva. L'erba tagliata con un suono pieno ed esalante un odore acuto, si adagiava nelle falciate alte. I falciatori che si stringevano da ogni parte perché le falciate erano corte, si sollecitavano l'un l'altro con grida allegre, facendo rumore con le ciotole, e risonando col cozzar delle falci e lo stridere dell'acciarino sulla lama. 

Levin camminava sempre fra il giovane e il vecchio. Il vecchio, rivestito di un giubbotto di montone, era sempre allegro, scherzoso e agile nei movimenti. Nel bosco capitavano continuamente dei funghi, gonfiatisi nell'erba sugosa, che venivan tagliati via dalle falci. Ma il vecchio, incontrando i funghi, si chinava ogni volta, tirava su e metteva in petto: ``Ancora un regalo per la vecchia'' diceva. 

Per quanto fosse facile falciare l'erba umida e tenera, era però difficile scendere e salire per i ripidi pendii del burrone. Ma il vecchio non era in imbarazzo. Menava la falce sempre allo stesso modo, col piccolo passo fermo dei suoi piedi infilati nei grandi lapti, s'arrampicava lentamente lungo il pendio, e pur traballando con tutto il corpo e coi pantaloni che pendevano di sotto la camicia, non tralasciava nel cammino neppure un filo d'erba, né un fungo, e scherzava allo stesso modo coi contadini e con Levin. Levin gli teneva dietro e spesso temeva di cadere nel salir con la falce su di un'erta così ripida dove anche senza falce era difficile arrampicarsi; ma s'arrampicava e faceva quello che doveva. Si sentiva sospinto da una forza esterna. 

\capitolo{VI}Falciarono il Maškin Verch, terminarono le ultime file, indossarono i gabbani e andarono allegramente verso casa. Levin montò a cavallo e, congedatosi con rammarico dai contadini, prese la via del ritorno. Dall'alto si voltò a guardarli; non si vedevano più nella nebbia che saliva dal basso; si udivano solo le grosse voci allegre, il riso e il suono delle falci che si cozzavano. 

Sergej Ivanovic da tempo aveva finito di pranzare e stava sorbendo acqua e limone e ghiaccio in camera sua, guardando le riviste e i giornali ricevuti proprio allora con la posta, quando Levin, coi capelli arruffati e appiccicati, la schiena e il petto anneriti e bagnati dal sudore, irruppe in camera con un allegro vociare. 

- Abbiamo finito tutto il prato! Ah, com'è bello, meraviglioso! E tu come te la sei passata? - disse Levin del tutto dimentico della conversazione poco piacevole della sera prima. 

- Dio mio! cosa sembri - disse Sergej Ivanovic, voltandosi a guardare, scontento sulle prime, il fratello. - Sì, la porta, la porta, chiudila! - gridò.- Ne avrai fatte entrare certamente una dozzina. 

Sergej Ivanovic non poteva sopportare le mosche e nella sua stanza apriva le finestre solo di notte e chiudeva con cura le porte. 

- Eh, via, neppure una. E se le ho fatte entrare, le acchiapperò. Tu non puoi immaginare, che piacere! E tu come hai passato la giornata? 

- Bene. Ma hai forse falciato tutto il giorno? Avrai una fame da lupo, penso. Kuz'ma ti ha preparato tutto. 

- No, non ho voglia di mangiare. Ho mangiato là. Ma ecco, vado a lavarmi. 

- Be', vai, vai; vengo subito da te - disse Sergej Ivanovic, scotendo il capo nel guardare il fratello. - E fai presto - aggiunse sorridendo e, riuniti i suoi libri, si preparò a muoversi. A un tratto anche egli era diventato allegro e non voleva separarsi dal fratello. - Dimmi, quando ha piovuto, dov'eri? 

- Ma quale pioggia? Appena poche gocce. Allora vengo subito. Così hai passato bene la giornata? Su, benissimo. - E Levin andò a vestirsi. 

Dopo cinque minuti i due fratelli si ritrovavano in sala da pranzo. Sebbene a Levin sembrasse di non aver appetito, e si fosse seduto a tavola solo per non dispiacere Kuz'ma, quando cominciò a mangiare, il pranzo gli parve straordinariamente gustoso. Sergej Ivanovic guardava sorridendo. 

- Ah, già, c'è una lettera per te - disse. - Kuz'ma, portala giù, per piacere. E guarda di chiudere la porta. 

La lettera era di Oblonskij. Levin la lesse ad alta voce. Oblonskij scriveva da Pietroburgo: ``Ho ricevuto una lettera da Dolly; è a Ergušovo e là le cose non vanno troppo bene. Ti prego, va' da lei e aiutala un po' con il tuo consiglio; tu sai tutto. Sarà lieta di vederti. È proprio sola, poverina. Mia suocera con gli altri è ancora all'estero''. 

- Bene! Andrò certamente da loro - disse Levin. - Anzi, andremo insieme. Lei è così simpatica. Non è vero? 

- Stanno lontano da qui? 

- Trenta verste. Forse anche quaranta. Ma la strada è ottima. Andremo comodamente. 

- Sono molto contento - disse Sergej Ivanovic, sempre sorridendo. La presenza del fratello minore lo predisponeva subito all'allegria 

- Eh, che appetito che hai! - disse, guardando il volto abbronzato rosso-scuro e il collo di lui chino sul piatto. 

- Ottimo! Non puoi credere che cura utile contro ogni balordaggine. Voglio arricchire la medicina di un termine nuovo: Arbeitskur. 

- Su, questo a te non occorre, mi pare. 

- Già, ma alle varie specie di malati di nervi, sì. 

- E già, bisognerebbe provarlo. Avrei voluto venire alla fienagione per vederti, ma il caldo era così insopportabile che non sono andato più in là del bosco. Son rimasto un po' a sedere e attraverso il bosco sono andato al villaggio, ho incontrato la tua governante e l'ho saggiata un po' circa l'opinione che i contadini hanno di te. A quanto ho potuto capire, non approvano questo. Ha detto: ``non è affar da signori''. In genere, mi pare che, nella concezione popolare, la manifestazione di una certa attività che essi chiamano ``da signori'' sia molto ben delimitata. E non ammettono che i signori escano fuori dal quadro delle loro concezioni. 

- Forse; ma questo è un tale godimento, quale non avevo mai provato in tutta la vita. E non c'è nulla di male. Vero? - rispose Levin. - Che farci se a loro non va? Del resto, io credo che non importi nulla. Eh? 

- In generale - proseguì Sergej Ivanovic - come vedo, sei soddisfatto della tua giornata. 

- Molto soddisfatto. Abbiamo falciato l'intero prato. E sapessi con che razza di vecchietto ho fatto amicizia! Non te lo puoi immaginare: un incanto! 

- Dunque, sei contento della tua giornata. E io pure. Per primo, ho risolto due mosse di scacchi di cui una è molto carina, si apre con un pedone, te la mostrerò. E poi ho pensato alla nostra conversazione di iersera. 

- Cosa? Alla conversazione di ieri? - disse Levin, socchiudendo beatamente gli occhi e riprendendo fiato dopo la fine del pasto, nell'impossibilità assoluta di ricordare quale fosse stata la conversazione del giorno innanzi. 

- Ti dimostrerò che hai ragione solo in parte. Il nostro disaccordo consiste in questo: che tu poni come movente l'interesse personale, e io suppongo che ogni uomo che abbia un certo grado di cultura debba interessarsi del bene comune. Può anche darsi che tu abbia ragione, che sia più desiderabile un'attività materiale spronata dall'interesse. In generale tu sei una natura troppo primesautière, come dicono i francesi; per te o un'attività appassionata, energica, o niente. 

Levin udiva le parole del fratello, ma non capiva proprio nulla e non voleva capire. Temeva solo che il fratello gli rivolgesse qualche domanda, perché allora sarebbe subito apparso che non lo ascoltava affatto. 

- Così è, amico mio - disse Sergej Ivanovic, toccandogli la spalla. 

- Sì, s'intende. Ma cosa mai! Io non mi intestardisco mica - rispose Levin con un colpevole sorriso infantile. 

``Ma di che si discuteva? - pensava. - S'intende, ho ragione io ed ha ragione lui e tutto va benissimo. Ma debbo passare in amministrazione a dare gli ordini''. Si alzò stirandosi e sorridendo. 

Anche Sergej Ivanovic sorrise. 

- Vuoi fare una passeggiata, andiamo insieme - disse, desideroso di non staccarsi dal fratello che emanava freschezza e vigore. - Andiamo, passiamo pure in amministrazione, se ci devi andare. 

- Ah, Dio mio! - gridò Levin così forte che Sergej Ivanovic si spaventò. 

- Che hai? 

- E il braccio di Agaf'ja Michajlovna! - disse Levin, battendosi la fronte. - Me n'ero proprio scordato! 

- Va molto meglio. 

- Via, faccio una corsa da lei. Non farai in tempo a metterti il cappello che sarò qui. 

E, correndo giù per la scala, fece risonare i tacchi come una raganella. 

\capitolo{VII}Stepan Arkad'ic era andato a Pietroburgo per compiere il più elementare dei doveri, i doveri di tutti i funzionari, il più necessario dei doveri, anche se incomprensibile a chi non è funzionario, omesso il quale non c'è modo di mantenere un impiego, far notare, cioè la propria esistenza al ministero. E mentre per compiere questo dovere, dopo aver preso con sé quasi tutto il denaro di casa, passava allegramente e piacevolmente il tempo alle corse e nei dintorni, Dolly coi bambini era andata a starsene in campagna, per diminuire, quanto più era possibile, le spese. Era andata nella sua proprietà dotale di Ergušovo, quella stessa dove in primavera era stato venduto il legname e che distava cinquanta verste da Pokrovskoe di Levin. 

La grande vecchia casa di Ergušovo era da tempo mal ridotta, ed era stata riparata dal vecchio principe che ne aveva anche ingrandita un'ala. Quest'ala, venti anni prima, quando Dolly era ancora ragazza, era spaziosa e comoda, pur rimanendo di lato, come tutte le ali, rispetto al viale d'ingresso e all'esposizione a mezzogiorno. Ma ormai anch'essa era cadente e ammuffita. Quando in primavera Stepan Arkad'ic vi si era recato a vendere il legname, Dolly gli aveva raccomandato di guardare con attenzione la casa e di ordinare e fare eseguire le riparazioni più necessarie. Stepan Arkad'ic che, come tutti i mariti colpevoli, si adoperava molto perché la moglie si trovasse a suo agio, aveva egli stesso dato un'occhiata alla casa e aveva impartito ordini per tutto quello che, secondo lui, era necessario. Secondo lui era necessario rivestire tutto il mobilio di cretonne, mettere le tende, ripulire il giardino, costruire un ponticello vicino allo stagno e piantare dei fiori: ma aveva dimenticato molte altre cose indispensabili, la cui mancanza costituiva ora un tormento per Dar'ja Aleksandrovna. 

Per quanto Stepan Arkad'ic si sforzasse di essere un marito e un padre premuroso, non riusciva in nessun modo a ricordarsi d'aver moglie e figli. Aveva gusti da celibe, e solo ad essi si conformava. Tornato a Mosca, aveva detto alla moglie che la casa era pronta, che era proprio un gioiello e che le consigliava di andarvi. Sotto tutti i riguardi a Stepan Arkad'ic piaceva molto che la moglie partisse; era salutare per i ragazzi, le spese sarebbero state minori, ed egli sarebbe stato più libero. Dar'ja Aleksandrovna a sua volta considerava il soggiorno in campagna indispensabile per i ragazzi, soprattutto per la bambina che non riusciva a riaversi dai postumi della scarlattina, e vedeva in esso anche la liberazione dalle piccole umiliazioni, dai piccoli debiti col legnaiuolo, col pescivendolo, col calzolaio, che la tormentavano tanto. Inoltre, la partenza l'attraeva perché sognava di far venire presso di sé la sorella Kitty, che sarebbe dovuta rientrare a mezza estate e alla quale avevano consigliato i bagni. Kitty le aveva già scritto dalla stazione termale che nulla le sorrideva tanto quanto passare l'estate a Ergušovo, così pieno di ricordi d'infanzia per tutte e due. 

Il primo periodo della vita in campagna fu molto difficile per Dolly. Aveva vissuto in campagna nell'infanzia e ne aveva serbata l'impressione che la campagna fosse una specie di liberazione da tutti i dispiaceri cittadini, che là, malgrado la vita non brillante (e a tutto questo Dolly si rassegnava presto), tutto fosse almeno accessibile e comodo; che tutto fosse a buon prezzo, che vi si potesse trovare tutto, e che ai bambini la campagna facesse bene. Ma ora, giuntavi come padrona di casa, vide che non era così come pensava. 

Il giorno dopo il loro arrivo, venne giù una pioggia dirotta e la notte cominciò a gocciolare nel corridoio e anche nella camera dei bambini, sì che si dovettero trasportare i lettini nel salotto. La sguattera non c'era; di nove mucche, a stare a sentire la donna addettavi, alcune erano pregne, altre al primo vitello, altre erano vecchie, altre avevano i capezzoli stretti; latte e burro non bastavano per i bambini. Uova non ce n'era. Una gallina non si poteva trovare, venivan serviti arrosto certi galli vecchi color viola, filamentosi. Non si trovavano donne per lavare i pavimenti; erano tutte a raccogliere le patate. Non si poteva uscire in carrozza perché il cavallo s'impennava e strapazzava il timone. Non c'era dove fare i bagni: tutta la sponda del fiume veniva calpestata dal bestiame ed era tutta aperta dal lato della strada; non si poteva neppure passeggiare nel giardino, perché il bestiame vi entrava attraverso uno squarcio dello steccato, e c'era un terribile toro che muggiva e sembrava pronto a dare cornate. Non c'erano armadi per i vestiti. Quelli che c'erano non chiudevano e s'aprivano da soli quando ci si passava accanto. Non c'erano né pentole né tegami; non c'era la caldaia per la lavanderia, e nella stanza delle donne nemmeno la tavola da stiro. 

In quel primo periodo, Dar'ja Aleksandrovna, che sperava di trovare tranquillità e riposo, capitata tra tutti questi guai, per lei enormi, si sentiva disperata; si dava da fare con tutte le sue energie, ma sentiva che non c'era via d'uscita, e ogni momento tratteneva le lacrime che le spuntavano negli occhi. L'amministratore, un ex sottufficiale, che Stepan Arkad'ic aveva preso a benvolere e che aveva promosso, per la sua prestanza e il suo fare ossequioso, dall'ufficio di portiere, non prese parte alcuna alle pene di Dar'ja Aleksandrovna; si limitava a dire rispettosamente: ``Non è proprio possibile, è gente così cattiva'' e non l'aiutava in nulla. 

La situazione sembrava senza via d'uscita. Ma anche in casa Oblonskij, come in tutte le case dove ci sono molti membri di famiglia, c'era la persona che non si faceva notare, ma che era tanto importante e utile: Matrëna Filimonovna. Costei calmava la signora, la rassicurava che tutto si ``sarebbe appianato'' (era questo il suo intercalare e da lei l'aveva preso Matvej), ed ella stessa, senza affrettarsi e senza agitarsi, operava. 

Andò subito d'accordo con la moglie dell'amministratore, e sin dal primo giorno bevve con lei e con l'amministratore il tè sotto le acacie e prese in esame tutte le questioni. Ben presto, lì, sotto le acacie, si venne a formare il circolo formato dalla moglie dell'amministratore, dallo starosta e dall'impiegato d'ufficio; cominciarono ad appianarsi a poco a poco tutte le difficoltà della vita: dopo una settimana, infatti, realmente tutto s'era ``appianato''. Il tetto fu accomodato, si trovò la cuoca, una comare dello starosta, le galline furono comprate; le mucche ripresero a dare il latte, il giardino fu recinto, furono messi dei ganci agli armadi che non si aprirono più arbitrariamente, e la tavola da stiro, ravvolta in un panno da soldato, fu distesa dal bracciuolo di una poltrona al cassettone, sì che nella stanza delle donne si sentì odor di stiro. 

- Su, ecco, e voi non facevate altro che disperarvi! - diceva Matrëna Filimonovna, mostrando la tavola. 

Fu costruito perfino un recinto per fare i bagni con paraventi di paglia. Lily cominciò a fare il bagno e per Dar'ja Aleksandrovna si avverarono, almeno in parte, le sue aspirazioni a una vita di campagna, se non tranquilla, almeno comoda. Dar'ja Aleksandrovna, con sei bambini, tranquilla non poteva mai essere. Uno si ammalava, l'altro rischiava di ammalarsi, al terzo mancava qualcosa, il quarto mostrava i segni d'un cattivo carattere e così via di seguito. Di rado, molto di rado, venivano brevi periodi di tranquillità. Ma queste cure e questi affanni erano per Dar'ja Aleksandrovna l'unica felicità possibile. Se non vi fossero stati, sarebbe rimasta sola col pensiero rivolto al marito che non l'amava. Ma, a parte ciò, per quanto fossero penosi per la madre la paura delle malattie, le malattie stesse e il dolore suo nel constatare le cattive inclinazioni dei figli, questi stessi figliuoli già adesso, con tante piccole gioie, la ripagavano delle sue pene. Queste gioie però eran così piccole che non si notavano, così come non si nota l'oro fra la sabbia; e nei momenti cattivi ella vedeva solo i dolori, cioè la sabbia, mentre c'erano pure i momenti buoni in cui vedeva solo le gioie, solo l'oro. 

Adesso, nella solitudine della campagna, ella sempre più spesso si rendeva conto di queste gioie. Spesso, guardando i figli, faceva tutti gli sforzi possibili per convincersi che si sbagliava, che come madre era parziale verso di loro; tuttavia non poteva non dirsi che aveva dei bambini deliziosi, tutti e sei, tutti così diversi, ma come non è facile trovarne, e ne era felice e orgogliosa. 

\capitolo{VIII}Verso la fine di maggio, quando già tutto era più o meno in ordine, ebbe risposta dal marito alle sue lamentele sui disagi campestri. Le scriveva chiedendole venia di non aver pensato a tutto e promettendo di venire alla prima occasione. Questa occasione non si era presentata, e fino ai primi di giugno Dar'ja Aleksandrovna visse sola in campagna. 

La vigilia di san Pietro, di domenica, Dar'ja Aleksandrovna era andata alla messa per far fare la comunione a tutti i suoi ragazzi. Dar'ja Aleksandrovna, nei suoi discorsi intimi, filosofici con la sorella, con la madre, con gli amici, molto spesso meravigliava per la sua libertà di pensiero in materia di religione. Credeva stranamente nella metempsicosi, poco preoccupandosi dei dogmi della Chiesa. Ma nella famiglia, e non solo per dare l'esempio, ma con tutta l'anima, adempiva rigorosamente tutti i precetti della Chiesa; e il fatto che i ragazzi per quasi un anno non si fossero comunicati l'agitava molto; sì che, con la piena approvazione e partecipazione di Matrëna Filimonovna, aveva deciso di far avvenire ciò in quella estate. 

Alcuni giorni prima aveva pensato all'abbigliamento di tutti i bambini. Furono cuciti, rifatti e lavati i vestiti, messi fuori gli orli e le finte; cuciti i bottoni e preparati i nastri. Un vestito per Tanja, che la signorina inglese s'era incaricata di cucire, fece masticar veleno a Dar'ja Aleksandrovna. L'inglese, nel ricucire, non aveva rifatto le pieghe al posto giusto, aveva tirato le maniche troppo in fuori e sembrava aver completamente sciupato l'abito. E in tal modo Tanja aveva delle spalle così strette che faceva male a guardarla. Ma Matrëna Filimonovna pensò di far delle riprese e di aggiungere una pellegrina. Alla cosa si pose riparo, ma ne venne fuori una baruffa con l'inglese. La mattina però tutto era in ordine e verso le nove, il termine fino al quale avevano pregato il sacerdote di attendere prima di iniziare la messa, i bambini, raggianti di gioia, tutti agghindati, erano presso la scalinata davanti alla carrozza, in attesa della madre. 

Alla carrozza, invece di Voron che s'impennava, era stato attaccato, per raccomandazione di Matrëna Filimonovna, Buryj dell'amministratore; e Dar'ja Aleksandrovna, che s'era trattenuta per curare il proprio abbigliamento, montò in carrozza in abito bianco di mussolina. 

Si adornava e vestiva con ansia e preoccupazione. Un tempo s'era vestita per sé, per essere bella e piacente; poi, con l'andar degli anni, l'abbigliarsi le era divenuto increscioso; s'accorgeva di non esser più bella. Ma ora di nuovo si vestiva con gusto e trepidazione. Ora non si abbigliava più per sé, né per la sua bellezza, ma per non sciupare, come madre di quei tesori di figli, l'impressione generale. Guardatasi nello specchio l'ultima volta, rimase contenta di sé. Stava bene. Non così bene come quando, ai suoi tempi, voleva figurare a un ballo, ma stava bene per lo scopo che perseguiva. 

In chiesa non c'era nessuno all'infuori dei contadini, dei portieri e delle donne. Ma a Dar'ja Aleksandrovna parve di scorgere l'incanto suscitato dai suoi bambini e da lei. I bambini non solo erano splendidi nei loro vestitini di gala, ma erano graziosi perché si comportavano proprio bene. È vero che Alëša non stava proprio del tutto composto: non faceva che voltarsi per rimirarsi il dietro del giubbetto; tuttavia era straordinariamente aggraziato. Tanja stava lì composta come una personcina grande e badava ai piccoli. Ma la più piccola, Lily, era deliziosa nel suo ingenuo stupore di tutto e fu difficile non sorridere quando, ricevuta la comunione, disse: ``please, some more''. 

Tornando a casa, i bambini sentivano che qualcosa di solenne era stato compiuto, ed erano tranquilli. 

Tutto andò bene anche a casa; ma a colazione Griša cominciò a fischiare, e quel che fu peggio, non obbedì all'inglese, così che fu privato del dolce. Dar'ja Aleksandrovna, se si fosse trovata presente, non avrebbe inflitto, proprio in quel giorno, una punizione, ma ormai era necessario sostenere la punizione dell'inglese, e la decisione che per Griša non ci sarebbe stato il dolce, venne riconfermata. Questo sciupò un po' la felicità generale. 

Griša piangeva, e diceva che anche Nikolen'ka aveva fischiato, eppure non era stato punito, e che egli non piangeva per la torta\ldots{} tanto era lo stesso\ldots{} ma perché si era ingiusti con lui. Questo era troppo triste, e Dar'ja Aleksandrovna aveva deciso di perdonare Griša e, per interpellare l'inglese, andò da lei. Ma, attraversando la sala, vide una scena che le riempì il cuore di una tale gioia che le lacrime le vennero agli occhi ed ella stessa perdonò senz'altro il malandrino. 

Il colpevole sedeva nella sala sul davanzale della finestra ad angolo; vicino a lui stava Tanja con un piatto. Col pretesto di voler imbandire un pranzo alle bambole, ella aveva chiesto all'inglese il permesso di portare una parte del suo dolce nella camera dei bambini, e invece l'aveva portato al fratellino. Griša, continuando a piangere sull'ingiustizia patita, mangiava la torta e fra i singhiozzi diceva: ``mangia anche tu, mangiamo insieme\ldots{} insieme''. 

Su Tanja aveva agito prima la pena per Griša, poi la coscienza della propria buona azione, sì che anche a lei venivano le lacrime agli occhi, ma non rifiutava e mangiava la propria parte. 

Scorta la madre, i piccoli si spaventarono, ma dal viso di lei capirono d'aver fatto bene, si misero a ridere con le bocche piene di torta, e cominciarono a pulire le labbra sorridenti con le mani, impiastricciando così di lacrime e marmellata i loro visi splendenti. 

- Mamma mia! Il vestito nuovo bianco! Tanja! Griša! - diceva la mamma, sforzandosi di salvare il vestito, ma sorridendo, con le lacrime agli occhi, d'un riso beato, entusiastico. 

I vestiti nuovi furono tolti, si fecero indossare alle bimbe dei camiciotti e ai bambini delle vecchie giacchette, e si fece attaccare alla carrozza lunga, di nuovo e con rincrescimento dell'amministratore, Buryj al timone per andare in cerca di funghi e al bagno. Un entusiastico grido si levò nella camera dei bambini e non si chetò fino alla partenza per il bagno. 

Di funghi se ne raccolse un cestino colmo; perfino Lily trovò un prugnolo. Era stata miss Hull a trovarne per prima e a mostrarglieli; ma ora aveva trovato da sola una grossa cappella di prugnolo, e questo fatto la rese oggetto di un'entusiastica ovazione generale: ``Lily ha trovato una cappella!''. 

Poi si andò verso il fiume; i cavalli furono lasciati all'ombra delle piccole betulle, e si andò al bagno. Il cocchiere Terentij, legati ad un albero i cavalli liberi dai freni, si sdraiò, schiacciando l'erba, all'ombra d'una betulla e si mise a fumare tabacco in foglie, mentre dal fiume gli giungeva l'allegro stridio infantile che non si chetava. 

Sebbene fosse faticoso badare a tutti i ragazzi e frenare le loro birichinate, sebbene fosse difficile ricordare e non confondere tutte quelle calzine, quei pantaloncini, quelle scarpette dei vari piedini, e inoltre snodare e sbottonare e riannodare fettuccine e bottoncini, Dar'ja Aleksandrovna, cui sempre era piaciuto fare i bagni, e che li riteneva utili per i ragazzi, di niente godeva tanto come di quel bagno fatto insieme con tutti i suoi bambini. Toccare quelle gambette paffute, stendendo su di esse le calzine, prendere nelle braccia e bagnare tutti quei corpicini nudi e sentir le strida ora gioiose ora spaventate; vedere quei volti ansanti, con gli occhi spalancati, impauriti e allegri, di quei suoi cherubini che si spruzzavano, tutto questo era un godimento grande per lei. 

Quando già una metà dei bambini fu rivestita, alcune donne parate a festa, che andavano a raccogliere erba egizia ed euforbia, si avvicinarono al bagno e si fermarono impacciate. Matrëna Filimonovna ne chiamò una, per darle a stendere un lenzuolo e una camicia caduti nell'acqua, mentre Dar'ja Aleksandrovna prese a discorrere con loro. Queste, che in principio ridevano nascondendosi il viso con la mano, e sembravano non capire le domande, si fecero presto coraggio e cominciarono a parlare, conquistando subito Dar'ja Aleksandrovna con la sincera ammirazione per i bambini che andavano via via indicando. 

- Guarda che bellezza, è bianca come lo zucchero - diceva una, ammirando Tanja e scotendo il capo. - Ma è magra\ldots{} 

- Sì, è stata malata. 

- Guarda, han fatto il bagno anche a lei - diceva un'altra, indicando la bambina lattante. 

- No, ha solo tre mesi - rispondeva con orgoglio Dar'ja Aleksandrovna. 

- Guarda! 

- E tu quanti ne hai? 

- Ne avevo quattro; ma ne sono restati due: un maschio e una femmina. Ecco, l'ho svezzata a carnevale. 

- E quanto ha? 

- Va per i due anni. 

- E perché l'hai allattata così a lungo? 

- È usanza nostra: le tre vigilie\ldots{} 

E la conversazione divenne quanto mai interessante per Dar'ja Aleksandrovna: come aveva partorito? di che cosa s'era ammalata? dov'era il marito? veniva spesso? 

Dar'ja Aleksandrovna non voleva staccarsi dalle donne, tanto l'interessava la conversazione avviata con loro, tanto identici in tutto erano i loro interessi. E la cosa più piacevole per Dar'ja Aleksandrovna era veder chiaramente come quelle donne l'ammirassero soprattutto perché aveva tanti figliuoli e tutti così belli. Le donne fecero anche ridere Dar'ja Aleksandrovna, ma l'inglese, che era stata la causa di quel riso per lei incomprensibile, si turbò. Una delle giovani donne, osservando l'inglese che si vestiva da ultima e che indossava una terza sottana, non poté trattenersi dall'osservare: ``Guarda, quella se ne mette e se ne mette e ancora ce ne ha!'' al che tutte erano scoppiate a ridere. 

\capitolo{IX}Circondata da tutti i bambini che avevano fatto il bagno e avevan le testine ancora umide, Dar'ja Aleksandrovna, con un fazzoletto in testa, si avvicinava già a casa, quando il cocchiere disse: 

- C'è un signore che arriva, mi pare, quello di Pokrovskoe. 

Dar'ja Aleksandrovna guardò innanzi a sé e si rallegrò vedendo la figura di Levin che le veniva incontro in cappello e cappotto grigio. Era sempre contenta di vederlo, ma in questo momento era particolarmente lieta ch'egli la vedesse in tutta la sua gioia. Nessuno più di Levin poteva apprezzarne il valore. 

Vistala, egli si trovò dinanzi uno di quei quadri di vita familiare che aveva sognato per sé per il futuro. 

- Sembrate una chioccia, Dar'ja Aleksandrovna. 

- Oh, come sono contenta! - diss'ella, tendendogli la mano. 

- Siete contenta, ma intanto non me lo avete fatto sapere. Da me c'è mio fratello. Soltanto ora ho ricevuto un biglietto di Stiva e ho saputo che eravate qua. 

- Di Stiva? - chiese con sorpresa Dar'ja Aleksandrovna. 

- Sì, scrive che avete cambiato residenza, e pensa che mi permetterete d'aiutarvi in qualche cosa - disse Levin; ma, appena detto questo, si confuse e, interrotto il discorso, seguitò a camminare in silenzio accanto alla carrozza, strappando i germogli dei tigli e spezzandoli coi denti. Si era confuso temendo che Dar'ja Aleksandrovna potesse non gradire l'aiuto di una persona estranea in cose che sarebbero dovute spettare al marito. A Dar'ja Aleksandrovna, invero, non garbava punto l'abitudine di Stepan Arkad'ic di affidare le faccende familiari a estranei. Ma capì subito che Levin aveva compreso. Anche per questa sua finezza d'intuito, per questa delicatezza, Dar'ja Aleksandrovna voleva bene a Levin. 

- Ma io ho capito, s'intende - disse Levin - questo vuol dire soltanto che volete vedermi, e io ne sono molto contento. Immagino che voi, padrona di casa cittadina, vi troviate a disagio in questo posto un po' selvaggio e, se vi occorre qualcosa, sono completamente ai vostri ordini. 

- Oh, no - disse Dolly. - Nei primi tempi sono stata a disagio; ma ora tutto si è aggiustato nel modo migliore, grazie alla mia vecchia njanja - disse, mostrando Matrëna Filimonovna, la quale, avendo compreso che si parlava di lei, sorrideva a Levin allegra e cordiale. Lo conosceva e giudicava che sarebbe stato un buon marito per la signorina e desiderava che la faccenda si concludesse. 

- Vogliate sedervi, ci stringeremo in qua - egli disse. 

- No, andrò a piedi. Ohé, ragazzi, chi di voi viene con me a fare a chi arriva prima coi cavalli? 

I bambini conoscevano molto poco Levin, non ricordavano neppure se e quando l'avessero visto, ma non mostrarono nei suoi riguardi quello strano senso di timidezza e repulsione che tanto spesso i bambini provano per le persone adulte che fingono, e per cui sono spesso così dolorosamente puniti. La finzione può ingannare, in ogni caso, la persona più intelligente e accorta, ma non il bambino, anche il più sciocco, ché, per quanto abilmente nascosta, la riconosce e se ne ritrae. Quali che fossero i difetti di Levin, di finzione non esisteva in lui neppure il più piccolo segno, e perciò i bambini gli mostrarono una simpatia pari a quella che scorsero sul viso della madre. All'invito di Levin, i due più grandi saltarono subito giù dalla carrozza, e corsero con lui così semplicemente come avrebbero corso con la njanja o con miss Hull o con la madre. Anche Lily volle andare da lui e la madre gliela consegnò; egli la mise a sedere su di una spalla e corse con lei. 

- Non abbiate paura, non abbiate paura, Dar'ja Aleksandrovna! - diceva sorridendo allegramente alla madre. - Non è possibile che le faccia del male e la faccia cadere. 

E guardando i suoi movimenti agili, forti, prudentemente accorti, e fin troppo tesi, la madre si tranquillizzò e sorrise allegra, guardandolo con approvazione. 

Ora, in campagna, con i bambini e Dar'ja Aleksandrovna che gli era simpatica, Levin si abbandonò a quella disposizione d'animo infantilmente gioiosa che Dar'ja Aleksandrovna amava tanto in lui. Correndo con i bambini, egli insegnava loro la ginnastica, faceva ridere miss Hull col suo pessimo inglese, e raccontava a Dar'ja Aleksandrovna le sue faccende campestri. 

Dopo pranzo Dar'ja Aleksandrovna, seduta sola con lui sul balcone, cominciò a parlare di Kitty. 

- Sapete? Kitty verrà qui e passerà l'estate con me. 

- Davvero? - egli disse, accendendosi, e subito, per cambiar discorso, disse: - Così vi devo mandare due mucche? Se volete regolare dal punto di vista economico la faccenda, allora vogliate pagarmi cinque rubli al mese, se non vi rincresce. 

- No, grazie, abbiamo già provveduto. 

- Su, allora andrò a dare un'occhiata alle vostre mucche, e se permettete darò delle istruzioni per nutrirle. Tutto sta nel foraggio. 

E Levin, solo per stornare il discorso, espose a Dar'ja Aleksandrovna la teoria sull'industria del latte che consisteva nel far conto che ogni vacca altro non fosse che una macchina per la trasformazione del foraggio in latte, e via di seguito. 

Egli parlava di queste cose, ma nello stesso tempo desiderava con tutta l'anima sentire particolari di Kitty pur temendo di averne. Lo terrorizzava il sospetto che potesse esserne sconvolta la propria calma conquistata. 

- Sì, è giusto, tutto questo va seguìto, ma chi lo farà? - rispondeva controvoglia Dar'ja Aleksandrovna. 

Ella aveva finalmente assestato le faccende domestiche per mezzo di Matrëna Filimonovna che non amava i cambiamenti; inoltre non credeva alla competenza di Levin in materia di economia rurale. Il ragionamento secondo cui la vacca è una macchina per fare il latte, la metteva in sospetto. Le pareva che ragionamenti simili potessero solo intralciare l'economia. Le pareva che molto più semplice fosse dare più foraggio e più beveraggio, come diceva Matrëna Filimonovna, a Petrucha e a Belopachaja e che il cuoco non prelevasse dalla cucina le risciacquature per farne usufruire la vacca della lavandaia. Questo sì che era chiaro. Invece i ragionamenti sul mangime farinoso erano vaghi e poco chiari. La verità però era ch'ella aveva voglia di parlare di Kitty. 
\enlargethispage*{1\baselineskip}

\capitolo{X}-Kitty mi scrive che desidera soltanto solitudine e calma - disse Dolly dopo il silenzio sopravvenuto. 

- E come va la sua salute, meglio? - domandò Levin agitato. 

- Grazie a Dio, è guarita del tutto. Io non ho mai creduto che avesse una malattia di petto. 

- Ah, son molto contento! - disse Levin, e a Dolly parve di scorgere una certa emozione, come una distensione, nel viso di lui mentre diceva questo e la guardava in silenzio. 

- Sentite, Konstantin Dmitric - disse Dar'ja Aleksandrovna, sorridendo del suo sorriso buono e un po' canzonatorio - perché siete arrabbiato con Kitty? 

- Io? Io non sono arrabbiato. 

- No, voi siete arrabbiato. Perché non siete passato né da noi né da loro quando siete stato a Mosca? 

- Dar'ja Aleksandrovna - disse egli, arrossendo fino alla radice dei capelli - mi meraviglio perfino che voi, tanto buona, non lo abbiate compreso. Come è che non avete almeno pietà di me, quando sapete\ldots{} 

- Cosa so? 

- Sapete che ho fatto una domanda di matrimonio e mi si è detto di no - pronunciò Levin e tutta la tenerezza che un momento prima lo aveva invaso per Kitty si tramutò in un senso di rancore per l'offesa ricevutane. 

- E perché credete che io lo sappia? 

- Perché tutti lo sanno. 

- Ecco, già in questo vi sbagliate; io non lo sapevo, pur immaginandolo. 

- Ebbene, lo sapete ora. 

- Sapevo soltanto che c'era stato qualcosa che l'aveva terribilmente tormentata, e che mi si pregava di non parlare mai di questo. E se non l'ha detto a me, non si è confidata con nessuno. Ma cosa mai vi è accaduto? Ditemelo. 

- Vi ho detto quello che è accaduto. 

- Quando? 

- Quando sono stato l'ultima volta da voi. 

- E sapete cosa vi dirò - disse Dar'ja Aleksandrovna: - ho un'enorme, enorme compassione di lei. Voi soffrite solo per orgoglio. 

- Può darsi - disse Levin - ma\ldots{} 

Ella lo interruppe. 

- Ma di lei, poveretta, ho un'enorme, enorme compassione. Adesso capisco tutto. 

- Eh, Dar'ja Aleksandrovna, scusatemi - diss'egli alzandosi. - Addio, Dar'ja Aleksandrovna, arrivederci. 

- No, aspettate - diss'ella agguantandolo per una manica. - Aspettate, sedetevi. 

- Vi prego, vi prego, non parliamo di questo - egli disse, sedendosi e sentendo nello stesso tempo che nel suo cuore si sollevava e s'agitava una speranza che gli era parsa sepolta. 

- Se non vi volessi bene - disse Dar'ja Aleksandrovna, e le vennero le lacrime agli occhi - se non vi conoscessi come vi conosco\ldots{} 

Il sentimento che era parso morto si ravvivava sempre di più, si sollevava e si impadroniva del cuore di Levin. 

- Sì, adesso ho capito tutto - proseguì Dar'ja Aleksandrovna. - Voi non potete capire questo; per voi, uomini, che siete liberi e potete scegliere, è sempre chiaro chi amate. Ma una ragazza nello stato d'attesa, col suo pudore femminile, virginale, una ragazza che vede voi, uomini, di lontano, prende tutto sulla parola; una ragazza ha e può avere un sentimento tale, da non saper cosa dire. 

- Sì, se il cuore non parla\ldots{} 

- No, il cuore parla, ma pensate: voi, uomini, avete delle intenzioni su una ragazza, andate in casa, fate amicizia, osservate, aspettate per vedere se troverete quel che vi piace, e poi, quando siete convinti di amare, fate la proposta di matrimonio\ldots{} 

- Via, non è affatto così. 

- È lo stesso, voi fate la proposta di matrimonio quando il vostro amore è venuto a maturità o quando fra due da scegliere s'è fatto il soprappeso. Ma una ragazza non la interrogano. Vogliono che ella scelga da sé, ma lei non può scegliere e risponde soltanto: ``sì'' e ``no''. 

``Sì, la scelta fra me e Vronskij'' pensò Levin e il cadavere che si ravvivava nell'anima sua morì di nuovo e premeva solo tormentosamente il suo cuore. 

- Dar'ja Aleksandrovna - diss'egli - così si sceglie un vestito, oppure non so che compera, ma non l'amore. La scelta è fatta, e tanto meglio\ldots{} E una ripetizione non può esserci. 

- Ah, quanto orgoglio, quanto orgoglio! - disse Dar'ja Aleksandrovna, quasi disprezzandolo per questo suo sentire che le appariva ben basso in confronto del sentimento che solo le donne conoscono. - Mentre voi facevate la proposta a Kitty, lei si trovava proprio nella situazione di non poter rispondere. C'era in lei il dilemma: o voi o Vronskij. Vedeva lui ogni giorno, e non vedeva voi da tempo. Supponiamo che fosse stata meno giovane: ad esempio, per me, al suo posto, non ci sarebbe stato dilemma. Mi era sempre stato antipatico quello lì, e i fatti\ldots{} 

Levin ricordò la risposta di Kitty. Ella aveva detto: ``No, questo non può essere\ldots{}''. 

- Dar'ja Aleksandrovna - egli disse asciutto - io apprezzo la vostra fiducia in me; penso che non sia giusta. Ma che io abbia ragione o torto\ldots{} questo orgoglio che voi disprezzate tanto, fa sì che ogni mio disegno su Katerina Aleksandrovna sia impossibile, voi comprendete, completamente impossibile. 

- Io dirò ancora una cosa sola: voi capite che parlo di una sorella che amo come i miei figli. Non dico che ella vi ami, ma volevo soltanto dire che il suo rifiuto in quel momento non significa nulla. 

- Non so - disse Levin, alzandosi. - Se sapeste quanto male mi fate! È come se vi fosse morto un bambino e vi si dicesse: ecco, era così e così, avrebbe potuto vivere e voi avreste potuto gioirne. Ed è morto, morto, morto\ldots{} 

- Come siete strano! - disse Dar'ja Aleksandrovna, osservando con un sorriso triste l'agitazione di Levin. - Sì, adesso capisco sempre di più - continuò pensosa. - Così voi non verrete da noi quando ci sarà Kitty. 

- No, non verrò. S'intende, io non sfuggirò Katerina Aleksandrovna, ma ogni volta che potrò, cercherò di liberarla del fastidio della mia presenza. 

- Siete molto, molto strano - ripeté Dar'ja Aleksandrovna, guardandolo con tenerezza in viso. - Su, va bene, non parliamone più. Perché sei venuta, Tanja? - disse Dar'ja Aleksandrovna in francese alla bambina che era entrata. 

- Dov'è la mia paletta, mamma? 

- Io ti sto parlando in francese, e tu rispondimi in francese. 

La bambina voleva, ma aveva dimenticato come si dice ``paletta'' in francese; la madre le suggerì la parola e poi, sempre in francese, disse dove poteva trovarla. Questo spiacque a Levin. 

Tutto ormai in casa di Dar'ja Aleksandrovna non gli piaceva più come poco prima, perfino i bambini. 

``E perché parla in francese coi bambini? - pensò. - Com'è poco naturale, com'è falso! E i bambini lo sentono. Si fa loro apprendere il francese e disimparare la sincerità'' e non sapeva che Dar'ja Aleksandrovna aveva già cambiato idea venti volte a questo proposito e tuttavia, pure a danno della sincerità, aveva trovato indispensabile educare in tal modo i figliuoli. 

- Ma dove mai dovete andare? Rimanete. 

Levin rimase lì fino al tè, ma la sua allegria era scomparsa e si sentiva ormai a disagio. 

Dopo il tè uscì in anticamera per ordinare di fare accostare al portone i cavalli; e quando rientrò trovò Dar'ja Aleksandrovna col viso sconvolto e le lacrime agli occhi. Durante la breve assenza di Levin era accaduto qualcosa che aveva distrutto in Dar'ja Aleksandrovna tutta la felicità di quel giorno e tutto il suo orgoglio per i suoi bambini. Griša e Tanja s'erano azzuffati per una palla. Dar'ja Aleksandrovna, sentendo gridare nella camera dei bambini, era accorsa e li aveva trovati avvinti in modo orribile: Tanja aveva afferrato Griša per i capelli e questi, col volto mostruoso di cattiveria, tirava pugni dove capitava. Nel vedere questa scena, qualcosa si era spezzato nel cuore di Dar'ja Aleksandrovna. Come se le tenebre si fossero addensate sulla sua vita, capiva che quei bambini, che erano il suo vanto, erano non solo bambini come tanti altri, ma tutt'altro che buoni, male educati, con tendenze perverse e volgari, bambini cattivi. 

Non poteva pensare ad altro, non poteva parlare d'altro e non seppe trattenersi dal raccontare a Levin la sua pena. 

Levin vedeva ch'ella soffriva, e cercava di consolarla, dicendo che non si trattava di cosa grave, che tutti i bambini vengono alle mani; eppure, mentre diceva questo, pensava dentro di sé: ``No, io non farò moine ai miei figli e non parlerò in francese con loro; ma non avrò bambini come questi. Non si devon guastare, non si devono deformare i bambini e allora saranno deliziosi. No, io non avrò bambini come questi''. 

Si accomiatò e andò via, né lei lo trattenne. 

\capitolo{XI}Verso la metà di luglio si presentò da Levin lo starosta del villaggio di sua sorella che era a venti verste da Pokrovskoe, a render conto dell'andamento degli affari e della fienagione. La rendita principale della tenuta della sorella di Levin proveniva dai prati fertilizzati dalle inondazioni. Negli anni precedenti, la fienagione era stata fatta dai contadini per venti rubli a desjatina. Quando Levin assunse l'amministrazione del fondo, esaminata la questione, aveva trovato che la fienagione poteva rendere di più e ne aveva fissato il prezzo a venticinque rubli per desjatina. 

I contadini non avevano accettato questo prezzo e, come aveva sospettato Levin, avevano allontanato gli altri compratori. Allora, recatosi sul posto, dispose che si raccogliesse il fieno, in parte per ingaggio e in parte trattenendo una quota. I contadini intralciarono con tutti i mezzi l'innovazione, ma l'affare andò bene e, fin dal primo anno, i prati resero quasi il doppio di prima. Per due anni ancora, fino all'anno precedente, era continuata la medesima reazione da parte dei contadini, ma il raccolto era andato bene lo stesso. Quell'anno poi i contadini si erano assunta tutta la fienagione, trattenendo per loro un terzo del prodotto e ora lo starosta era venuto a dire che il fieno era stato raccolto e che, temendo la pioggia, egli aveva fatto venire l'amministratore, aveva fatto la divisione e aveva ammucchiato, presente lui, undici covoni di fieno quale spettanza padronale. Dalle risposte evasive alla domanda sul quantitativo di fieno raccolto nel prato più esteso, dalla fretta con la quale aveva proceduto alla ripartizione del prodotto senza chiedere l'autorizzazione, da tutto il tono del discorso del contadino, Levin aveva capito che in quella divisione qualcosa di poco chiaro c'era stato e decise di andare lui stesso a sincerarsene. 

Giunto al villaggio all'ora di pranzo e lasciato il cavallo da un suo vecchio amico, marito della balia del fratello, Levin, desideroso di conoscere egli stesso i particolari della falciatura, si recò dal vecchio addetto alle api che si trovava nell'arniaio. Questi, il vecchio Parmenyc, loquace e bello a vedersi, accolse festoso Levin; gli mostrò tutta la sua azienda, si diffuse in mille particolari circa le api e la sciamatura di quell'anno; ma alle domande di Levin riguardanti la fienagione, rispose impreciso e svogliato. Questo confermò ancor più Levin nei suoi sospetti. Andò allora sui prati per vedere le biche del fieno. Da ciascuna bica non ne potevano venir fuori cinquanta carrettate e, per convincere di ciò i contadini, Levin ordinò di chiamare subito i carri da trasporto, di caricarvi su una bica e di trasportarla nella rimessa. Ne vennero fuori trentadue carrettate solamente. Malgrado lo starosta assicurasse che il fieno si rigonfia e poi nelle biche si abbassa e malgrado spergiurasse che tutto questo era andato secondo il volere di Dio, Levin rimase fermo nella sua determinazione: il fieno era stato diviso senza suo ordine e perciò egli non accettava quel fieno per cinquanta carrettate a bica. Dopo lunghe discussioni decisero la questione nel senso che i contadini avrebbero ritenute per sé quelle undici biche, calcolandole a cinquanta carrettate ciascuna, e che per determinare la parte spettante al padrone, si sarebbe fatta una nuova ripartizione. Tra trattative e decisioni si giunse all'ora del pranzo. Mentre si divideva l'ultimo fieno, Levin, affidato il rimanente controllo all'amministratore, andò a sedersi su una bica segnata da una canna e se ne stette a contemplare il prato che brulicava di gente. 

Dinanzi a lui, in un'ansa del fiume, al di là di un pantanello, si moveva allegramente, scoppiettando di voci sonore, una schiera variegata di donne, e su per il guaime verdechiaro si distendevano le onde grigie, sinuose, del fieno sparpagliato. Dietro le donne venivano i contadini coi forconi, e dalle onde del fieno si formavano alte, larghe e rigonfie le biche. A sinistra del prato già sgombro risonavano i carri e, tirate giù a grandi forcate, una dopo l'altra, le biche scomparivano e al loro posto venivano caricate pesanti carrettate del fieno profumato che sporgevano fin sulla groppa dei cavalli. 

- Bisogna raccoglierlo col tempo buono! Fieno ce ne sarà - disse il vecchio, sedendosi accanto a Levin. - Sembra tè, e non fieno! Guarda come lo tirano su! Sembra che abbian dato a beccare i semi agli anatroccoli! - aggiunse mostrando le biche che venivano caricate. - Dall'ora di pranzo ne han caricate una buona metà. - È l'ultimo carro, eh? - gridò rivolgendosi a un giovanotto che, in piedi sul fondo del carro e agitando le estremità delle redini di canapa, gli passava accanto. 

- L'ultimo, batjuška - gridò il giovane, trattenendo per un momento il cavallo; poi, sorridendo, si voltò a guardare la contadina, allegra, rubiconda e sorridente anch'essa, che sedeva nella parte posteriore del carro. 

- E questo chi è, un figliuolo? - chiese Levin. 

- L'ultimo - disse il vecchio con un sorriso carezzevole. 

- Che bel ragazzo! 

- Non c'è male. 

- È già sposato? 

- Sì, son tre anni, alla vigilia di san Filippo. 

- Be', e figliuoli niente? 

- Macché figliuoli! Per un anno intero si è comportato come uno stupido\ldots{} c'era da vergognarsi - rispose il vecchio. - Su, il fieno! È vero tè! - ripeté, desiderando cambiar discorso. 

Levin osservò attento Van'ka Parmenov e la moglie. Non lontano da lui la coppia caricava una bica. Ivan Parmenov stava diritto sul carro, riceveva il fieno, eguagliava e pestava le forcate colme, che gli tendeva, agile, la giovane bella moglie, prima a bracciate, poi con la forca. La donna lavorava con facilità, allegra e svelta. Il fieno grosso, riscaldato, non veniva subito alla forca. Ella dapprima l'assestava, ci conficcava dentro il bidente, poi, con un movimento elastico e veloce, vi si appoggiava con tutto il peso del corpo e subito, piegando la schiena stretta da una cintura rossa, si raddrizzava e, sporgendo il seno turgido sotto la pettina bianca, con una mossa rapida afferrava la forca e gettava la forcata su, in alto, sul carro. Ivan, svelto, cercando evidentemente di risparmiarle ogni più piccolo sforzo, afferrava, aprendo larghe le braccia, la forcata che gli veniva tesa e l'assestava sul carro. Tesogli l'ultimo fieno raccolto col rastrello, la donna scosse via le festuche che le si erano insinuate intorno al collo e, rimesso a posto il fazzoletto rosso sulla fronte bianca non ancora abbronzata, si ficcò sotto il carro per legare il carico. Ivan le diceva come dovesse agganciarlo alla freccia, e a una cosa detta da lei, scoppiò a ridere forte. Nell'espressione di tutti e due i volti c'era un amore forte, giovane, risvegliato da poco. 
\enlargethispage*{1\baselineskip}

\capitolo{XII}Il carico fu legato. Ivan saltò giù e condusse per la briglia il cavallo ben nutrito. La donna gettò il rastrello sul carro e con passo sicuro, agitando le braccia, andò verso le contadine che s'erano riunite formando quasi un cerchio. Ivan, raggiunta la strada, entrò a far parte del convoglio dei carri. Le donne, col rastrello sulle spalle, splendendo nei colori vivaci delle vesti e facendo strepito con le voci sonore, seguivano allegre i carri. Una voce rozza, selvatica di donna, intonò una canzone e la cantò fino al ritornello; poi, insieme, in coro, una cinquantina di voci varie, sane, robuste e acute, ripresero daccapo la stessa canzone. 

Le donne, così cantando, si avvicinarono a Levin, e a lui sembrò che una nuvola e un tuono gonfio di allegria si avanzasse venendogli incontro. La nuvola avanzò, lo afferrò, e la bica sulla quale era sdraiato e le altre biche e i carri e il bosco intero, il campo lontano, tutto cominciò a camminare e ad agitarsi al ritmo di quella selvaggia, ilare canzone tra gridi, sibili e sobbalzi. Levin ebbe invidia di quella sana allegria, e avrebbe voluto prender parte a quella gioiosa espressione di vita. Ma non poteva, e doveva tacere, guardare e ascoltare. Quando quella gente e il canto si sottrassero alla sua vista e al suo udito, un senso grave di tristezza per la propria solitudine, per il proprio ozio fisico, per la propria avversione verso quel mondo lo afferrò. 

Alcuni di quegli stessi contadini che più di tutti avevano discusso con lui per il fieno, quelli ch'egli aveva offeso o quelli che volevano ingannarlo, questi stessi contadini lo salutavano ora allegramente, e appariva chiaro che non avevano e che non avrebbero potuto avere verso di lui alcun'ombra di rancore, e non solo nessun pentimento, ma neppure più il ricordo di aver cercato di ingannarlo. Tutto era affondato nel gioioso mare del lavoro comune. Dio dà il giorno, Dio dà la forza. E il giorno e la forza sono consacrati al lavoro; e in se stesso il lavoro trova la sua ricompensa. Ma per chi il lavoro? Quali saranno i frutti del lavoro? Queste sono considerazioni secondarie e inconsistenti. Spesso Levin aveva ammirato questa vita, aveva spesso provato invidia per le persone che la vivevano; ma ora per la prima volta, sotto l'impressione dei rapporti che aveva notato tra Parmenov e la sua giovane moglie, per la prima volta si delineò chiara nella mente di Levin l'idea che dipendeva da lui sostituire alla vita che viveva, così greve e oziosa, così artificiale ed egoista, questa deliziosa, semplice vita di lavoro fatto in comune. 

Il vecchio, che era stato a sedere accanto a lui, già da tempo se n'era andato; la gente si era tutta dispersa. Quelli che abitavano vicini erano andati alle loro case e quelli che stavano lontano s'eran raccolti per cenare e dormire sul prato. Levin, inosservato, continuava a starsene sdraiato sulla bica e a guardare, ad ascoltare e riflettere. La gente che era rimasta a passar la notte nel prato, non dormì quasi, in quella breve notte estiva. Dapprima, durante la cena, si udirono chiacchiere e risa, poi di nuovo canti e risa. 

Tutta la lunga giornata di lavoro non aveva lasciato in quei contadini altra traccia che l'allegria. Prima dell'alba tutto si chetò. Si udivano solo l'incessante gracidare notturno delle ranocchie nella palude e lo sbuffare dei cavalli in giro per il prato nella nebbia levatasi all'alba. 

Rientrato in sé, Levin si alzò dalla bica e, guardate le stelle, capì che la notte era passata. 

``Be', e allora che farò? Come farò tutto questo?'' si chiese, cercando di esprimere a se stesso ciò che aveva pensato e sentito in quella breve nottata. Tutto quello che aveva pensato e sentito si divideva in tre distinti ordini di pensieri. Il primo comprendeva la rinuncia alla sua vecchia vita, alle cognizioni inutili, alla cultura che non serviva a nulla. Questa rinuncia l'avrebbe fatta con piacere, ed era per lui facile e semplice. Gli altri pensieri e le altre figurazioni riguardavano la vita che egli avrebbe voluto vivere ora. Sentiva chiaramente la purezza e la semplicità, la legittimità di questa vita, ed era convinto che vi avrebbe trovato quella soddisfazione tranquilla e dignitosa di cui avvertiva così morbosamente l'assenza. Ma l'ultimo ordine di idee si aggirava sul modo di compiere questo passaggio dalla vecchia vita alla nuova. E qui non vedeva più con chiarezza. ``Prender moglie? Avere un lavoro e la necessità di lavorare? Lasciare Pokrovskoe? Comprare della terra? Iscriversi in una società? Sposare una contadina? Come farò tutto questo? - si chiedeva di nuovo e non trovava risposta. - Ma io non ho dormito tutta la notte e non posso darmi una risposta chiara - si disse. - Chiarirò dopo. Una cosa è certa, che questa notte ho deciso il mio destino. Tutti i miei precedenti sogni sulla vita familiare non rispondono a questo. Questo sì, è molto più semplice ed è certamente migliore''. 

``Com'è bello! - pensò, guardando la strana conchiglia quasi madreperlacea di nuvole bianche a pecorelle fermatasi proprio sulla sua testa, a mezzo il cielo. - Come tutto è bello in questa notte luminosa! E quando ha avuto il tempo di formarsi questa conchiglia? Un momento fa ho guardato il cielo e non c'era nulla, solo due strisce bianche. Ecco, proprio così, inavvertitamente, sono cambiate le mie idee sulla vita!''. 

Uscì dal prato e andò verso il villaggio per la strada maestra. S'era levato un po' di vento e il tempo s'era fatto grigio, scuro. Era il momento della foschia che di solito precede l'alba, la vittoria completa della luce sulle tenebre. 

Rannicchiandosi in se stesso per il freddo, Levin camminava in fretta, guardando a terra. ``Cos'è questo? Viene qualcuno'' pensò, sentendo rumore di sonagli, e sollevò la testa. A quaranta passi da lui, venendogli incontro sulla strada maestra sulla quale egli camminava, avanzava un tiro a quattro con le valigie sull'imperiale. I cavalli del centro si stringevano all'asse per evitare le carreggiate, ma il cocchiere abile, che sedeva di sghembo a cassetta, teneva l'asse sopra una carreggiata sola, di modo che le ruote correvano sul liscio. 

Levin notò questo soltanto e, senza pensare a chi potesse trovarsi in viaggio, guardò distratto la vettura. 

Nella vettura sonnecchiava, in un angolo, una signora anziana e, al finestrino, evidentemente da poco risvegliata, sedeva una giovinetta che teneva stretti con le due mani i nastri di una cuffia bianca. Limpida e pensosa, tutta piena di una vita interiore bella e complessa, ignota a Levin, guardava, di là da lui, il sopravvenire dell'alba. 

Al momento in cui la visione stava per scomparire due occhi sinceri lo guardarono. Ella lo riconobbe, e uno stupore gioioso illuminò il suo viso. 

Egli non poteva sbagliarsi. Unici al mondo erano quegli occhi. Per lui un solo essere c'era al mondo capace di esprimere tutta la luce e tutto il senso della vita. Era lei. Kitty. Egli capì che veniva dalla stazione ferroviaria e andava a Ergušovo. E tutto quello che aveva agitato Levin in quella notte insonne, tutte quelle decisioni che egli aveva prese, tutto improvvisamente scomparve. Ricordò con ripugnanza la sua fantasticheria di sposare una contadina. Soltanto là, in quella vettura che si allontanava veloce e che era passata dall'altra parte della strada, soltanto là c'era la possibilità di risolvere l'enigma della sua vita che così tormentosamente l'opprimeva negli ultimi tempi. 

Ella non guardò più fuori. Il suono delle balestre cessò d'essere percettibile, si udì appena un bubbolio di campanelli. L'abbaiare dei cani significò che la vettura era passata attraverso il villaggio. Intorno rimasero i campi deserti, più avanti il villaggio e lui che, solitario ed estraneo a tutto, camminava solo sulla strada abbandonata. 

Guardò il cielo, sperando di trovarvi ancora quella conchiglia che aveva ammirato e che rappresentava per lui tutto il corso dei suoi pensieri e dei suoi sentimenti di quella notte. In cielo non c'era più nulla che somigliasse a una conchiglia. Ad altezza irraggiungibile, s'era compiuto già un misterioso cambiamento. Non c'era più traccia di conchiglia e c'era invece un tappeto eguale e disteso per l'intera metà del cielo di nuvole a pecorelle che rimpicciolivano sempre più. Il cielo s'inazzurrava e schiariva e con la stessa tenerezza, ma anche con la stessa irraggiungibilità, rispondeva al suo sguardo indagatore. 

``No - egli si disse - per quanto bella sia questa vita semplice e laboriosa, io non posso tornarvi. Io amo lei''. 

\capitolo{XIII}Nessuno, ad eccezione delle persone che vivevano proprio vicino ad Aleksej Aleksandrovic, sapeva che quest'uomo dall'aspetto impassibile, quest'uomo ragionatore avesse una debolezza contrastante con la linea fondamentale del suo carattere. Aleksej Aleksandrovic non poteva sentire e vedere con indifferenza le lacrime di un bambino o di una donna. La vista delle lacrime produceva in lui come un certo smarrimento, e gli faceva perdere del tutto la facoltà di riflettere. Il capo della sua cancelleria e il suo segretario, che sapevano ciò, avvertivano le postulanti di reprimere con ogni sforzo le lacrime, se non volevano guastare le loro cose. ``Si arrabbierà, smetterà di ascoltarvi'', dicevano. Infatti, in simili casi, il turbamento prodotto dalle lacrime in Aleksej Aleksandrovic si esprimeva con uno scatto d'ira: ``Non posso, non posso far nulla. Vi prego di andarvene'' gridava in simili casi. 

Quando, tornando dalle corse, Anna gli ebbe rivelato i suoi rapporti con Vronskij e, subito dopo, copertosi il viso con le mani, era scoppiata a piangere, Aleksej Aleksandrovic, malgrado il rancore che provava verso di lei, sentì nello stesso tempo un afflusso di quello sconvolgimento d'animo che sempre producevano in lui le lacrime. Sapendo ciò e sapendo che in quel momento la manifestazione dei suoi sentimenti non sarebbe stata adatta alla situazione, si era sforzato di contenere dentro di sé ogni moto di vita, e perciò era rimasto immobile senza guardarla. Proprio da questo gli era venuto nel volto quello strano livore di morte che aveva tanto colpito Anna. 

Quando furono giunti a casa, egli l'aiutò a scendere dalla vettura e, energicamente dominandosi, la salutò con l'abituale cortesia e disse quelle parole che non lo impegnavano in nulla: disse che l'indomani le avrebbe comunicata la sua decisione. 

Le parole della moglie, che erano venute a confermare i suoi sospetti peggiori, avevano prodotto un dolore atroce nel cuore di Aleksej Aleksandrovic. Questo dolore era reso ancora più aspro da quello strano senso di pena fisica verso di lei che gli avevano prodotto le lacrime. Ma, rimasto solo nella vettura, con gioiosa sorpresa, si sentì completamente liberato e da questa pena e dai dubbi e dai tormenti che negli ultimi tempi lo avevano fatto soffrire. 

Provava la sensazione di chi si fosse fatto cavare un dente che da tempo dolesse. Dopo un dolore tremendo e la sensazione che una cosa enorme, più grande della propria testa, sia stata tolta dalla mascella, il paziente, pur non credendo ancora al riacquistato benessere, sente però che non c'è più la cosa che per lungo tempo gli ha avvelenato l'esistenza, avvincendo tutta la sua attenzione; sente che può di nuovo vivere, pensare e interessarsi non più esclusivamente al proprio dente. Questa era la sensazione che provava Aleksej Aleksandrovic. Il suo dolore era stato non comune e terribile, ma era passato; ora egli sentiva che poteva di nuovo vivere e non pensare esclusivamente a sua moglie. 

``Senza onore e senza cuore e senza religione: una donna depravata. L'ho sempre pensato e sempre constatato, sebbene cercassi, per compassione, di ingannare me stesso'' diceva tra sé. E gli pareva davvero d'aver sempre constatato ciò; richiamò alla mente i particolari della vita di lei, nei quali prima non aveva notato mai nulla di riprovevole: adesso questi stessi particolari diventavano la prova evidente che ella era sempre stata una donna depravata. ``Ho sbagliato nel legare a lei la mia vita, ma nulla c'è di censurabile in questo mio errore, e non debbo per questo rendermi infelice. Non sono io il colpevole - diceva a se stesso - ma lei. Essa non esiste più per me''. 

Tutto quello che sarebbe poi accaduto di lei e del figlio, verso il quale, così come verso di lei, si eran mutati i suoi sentimenti, non lo interessava più. Ora occorreva una cosa sola: trovare il modo migliore, più decoroso e che importasse il minor sacrificio da parte sua, e fosse perciò il più giusto, per scuotersi quel fango ch'ella, cadendo, gli aveva schizzato addosso, e riprendere il cammino della sua vita attiva, onesta, utile. ``Non può rendere infelice la mia vita il fatto che una donna disprezzabile si sia resa colpevole; io debbo soltanto trovare una via d'uscita da questa situazione penosa nella quale ella mi ha posto. E la troverò - diceva a se stesso, accigliandosi sempre più. - Non sono il primo, e non sarò l'ultimo''. E per non parlare degli esempi storici, a cominciar da Menelao, tornato nella mente di tutti per La bella Elena, tutta una serie di esempi contemporanei di infedeltà di mogli verso mariti dell'alta società si presentò alla mente di Aleksej Aleksandrovic. ``Dar'jalov, Poltavskij, il principe Karibanov, il conte Paskudin; Dram\ldots{} sì\ldots{} anche Dram, uomo così degno e laborioso\ldots{} Semënov, cagin, Sigonin - ricordava. - Poniamo che un certo irragionevole ridicule cada su queste persone, ma io in questo non ho visto mai altro che una sventura e ne ho sempre avuto pietà - si andava dicendo Aleksej Aleksandrovic, e non era vero che egli avesse avuto compassione per sventure di tal genere; anzi, quanto più frequenti erano i casi di mogli che tradivano i mariti, tanto più in alto egli si considerava. - Questa è una sventura che può capitare a chiunque. E questa sventura ora ha colpito me. Tutto sta nel risolvere la situazione nel miglior modo possibile''. E cominciò a elencare ed esaminare i diversi modi con i quali avevano reagito le persone che si erano trovate in una posizione simile alla sua. 

``Dar'jalov si è battuto in duello\ldots{}''. 

Nella sua giovinezza, il duello, per Aleksej Aleksandrovic, era stato addirittura un incubo, perché egli era costituzionalmente un pavido, e lo sapeva bene. Aleksej Aleksandrovic non poteva pensare a una pistola puntata contro di lui senza inorridire, e in vita sua non aveva mai adoperato armi. Questo suo orrore lo aveva indotto, quando era giovane, a pensare spesso al duello e a immaginare se stesso in una situazione in cui fosse necessario esporre la propria vita al pericolo. Raggiunto poi il successo e una solida posizione sociale, aveva dimenticato da tempo questa sensazione; ma l'averci troppo pensato prevalse ora, e il terrore della propria vigliaccheria, anche adesso, gli parve così grande, che Aleksej Aleksandrovic indugiò a lungo e da ogni verso ad accarezzare col pensiero la soluzione del duello, pur sapendo in precedenza che mai, per nessuna ragione al mondo, si sarebbe battuto. 

``Senza dubbio la nostra società è tuttora tanto barbara (non così in Inghilterra), che moltissimi - e nel numero di questi `moltissimi' vi erano proprio le persone alla cui opinione Aleksej Aleksandrovic maggiormente teneva - sarebbero favorevoli al duello; ma quale risultato mai si raggiungerebbe? Mettiamo che io lo sfidi - continuò tra sé Aleksej Aleksandrovic mentre, rappresentatasi con chiarezza la notte che avrebbe passata dopo aver mandata la sfida, e la pistola puntata contro di lui, rabbrividiva e capiva che questo non sarebbe mai avvenuto - ammettiamo che io lo sfidi a duello. Ammettiamo che mi insegnino come fare - continuò a pensare - che mi mettano in posizione\ldots{} io premo il grilletto - si disse socchiudendo gli occhi - e ammettiamo che lo uccida - disse fra sé Aleksej Aleksandrovic e scosse il capo per scacciare questi sciocchi pensieri. - Quale senso può avere l'uccisione di un uomo al fine di regolare i propri rapporti con una moglie infedele e un figlio? Dovrei poi sempre necessariamente decidere come regolarmi con lei. Ma la cosa ancor più probabile, e che senza dubbio accadrebbe, è che sarei io a rimanere ucciso o ferito. Io, vittima innocente, ucciso o ferito. Cosa ancor più insensata. Ma questo è il meno: una sfida sarebbe, da parte mia, un'azione disonesta. Non so io, forse, che i miei amici eviterebbero a tutti i costi di farmi scendere sul terreno, che non lascerebbero mai esporre al pericolo la vita di un uomo di stato, necessario alla Russia? E che cosa accadrebbe allora? Che io, sicuro di non giungere al momento pericoloso, avrei cercato, con questa sfida, di acquistare soltanto un falso prestigio. E questo è disonesto, è falso, è un ingannare gli altri e se stesso. La soluzione del duello è, dunque, assurda, e nessuno la esige da me. Il mio scopo è quello di garantire la mia reputazione, quella reputazione che mi è necessaria per poter proseguire senza ostacoli la mia attività''. La sua attività al servizio dello stato che, anche prima, aveva un grande valore agli occhi di Aleksej Aleksandrovic, gli appariva ora particolarmente rilevante. 

Esaminata e respinta la soluzione del duello, passò a considerare quella offerta dal divorzio, altro espediente cui eran ricorsi alcuni di quei mariti ch'egli aveva ricordato. Passando in rassegna tutti i casi di divorzio a lui noti (molti ce ne erano stati nell'alta società), non ne trovò neppure uno in cui il divorzio avesse ottenuto lo scopo ch'egli perseguiva. In tutti quei casi il marito aveva o ceduto o venduto la moglie infedele, e quella stessa parte che, in conseguenza della propria colpa, non avrebbe più dovuto avere diritto a contrarre matrimonio, entrava in rapporti legalizzati, sia pure in modo fittizio e immaginario, con il nuovo coniuge. Inoltre, nel caso suo personale, non era possibile conseguire un divorzio legale, cioè un divorzio nel quale fosse proclamata solo la colpevolezza della moglie. Egli vedeva che le condizioni complesse di vita in cui egli era, non ammettevano la possibilità di quelle prove volgari che la legge pretendeva per il riconoscimento della colpevolezza della moglie; vedeva che quella certa raffinatezza di ambiente in cui viveva non permetteva neppure l'uso di queste prove, se pure ci fossero state, e che queste prove avrebbero fatto decadere più lui che lei nella pubblica opinione. 

Un tentativo di divorzio poteva portare soltanto a un processo scandaloso che avrebbe offerto ai nemici la gradita occasione per infamare e avvilire la sua alta posizione nel mondo. E quindi lo scopo principale: risolvere la situazione col minimo turbamento possibile, non si sarebbe ottenuto nemmeno col divorzio. Era inoltre evidente che col divorzio, ed anche con una semplice minaccia di divorzio, la moglie avrebbe rotto ogni rapporto col marito e si sarebbe unita con l'amante. E nell'animo di Aleksej Aleksandrovic, malgrado la sua attuale, completa, come gli pareva, sprezzante indifferenza verso la moglie, un sentimento permaneva nei riguardi di lei: il desiderio che ella non riuscisse a unirsi senza ostacoli a Vronskij, ch'ella non ottenesse così un vantaggio dalla propria colpa. Questo pensiero solo irritava a tal punto Aleksej Aleksandrovic che, rappresentatosi con chiarezza la cosa, emise un mugolio di intimo dolore e si sollevò e cambiò posizione nella carrozza; poi a lungo, dopo questo, avviluppò, accigliato, le gambe ossute e freddolose nello scialle di lana. 

Calmatosi, continuava a pensare: ``Oltre al divorzio formale, si potrebbe ricorrere alla separazione legale, come hanno fatto Karibanov, Paskudin e quel bravo Dram''. Ma anche questa soluzione presentava gli stessi inconvenienti umilianti del divorzio, gettava sua moglie nella braccia di Vronskij. ``No, non è possibile, non è possibile, non è possibile - disse ad alta voce rigirandosi di nuovo lo scialle. - Io non devo essere infelice, e lei e lui non devono poter essere felici''. 

La gelosia che lo aveva tormentato quando non era ancora a conoscenza di tutto era svanita nel momento in cui con le parole della moglie gli era stato strappato il dente. Ma questo sentimento si era trasformato in un altro: nel desiderio che non solo ella non avesse a trionfare, ma che della propria colpa dovesse sopportare la pena. Egli non lo confessava questo sentimento, ma in fondo all'anima voleva che ella soffrisse per aver turbato la sua pace, e per aver offeso il suo onore. E di nuovo, riesaminate le soluzioni del duello, del divorzio e della separazione, e nuovamente respintale, Aleksej Aleksandrovic si convinse che l'unica via d'uscita era questa: trattenere presso di sé la moglie, nascondendo al mondo quello che era accaduto e adoperando tutti i mezzi per spezzare quella relazione e punire in tal modo, ma questo non lo confessava a se stesso, la colpa di lei. ``Devo annunciarle la mia decisione, che cioè, avendo riflettuto sulla penosa situazione nella quale ella ha posto la famiglia, tutte le vie d'uscita sarebbero, per entrambe le parti, peggiori di uno statu quo apparente; e che a questo consento a patto che sia rigorosamente osservata da parte sua la inderogabile mia volontà, che ella cioè tronchi ogni suo rapporto con l'amante''. A confermarlo in questa decisione, quando però già era stata presa in maniera definitiva, balenò nella mente di Aleksej Aleksandrovic un'altra importante considerazione. ``Soltanto in questo modo agirò anche secondo i precetti della religione, soltanto con questa decisione non respingerò da me la moglie colpevole e le offrirò la possibilità di ravvedersi e consacrerò perfino, per quanto possa essermi penoso, una parte delle mie energie alla sua riabilitazione e alla sua salvezza''. Sebbene Aleksej Aleksandrovic sapesse di non avere alcun ascendente morale sulla moglie, e che da tutto quel tentativo di ravvedimento sarebbe venuta fuori soltanto menzogna, pur non avendo mai pensato, anche in momenti penosi, neppure una volta sola a cercare una guida nella religione, ora che la sua decisione sembrava coincidere con i precetti religiosi, questa solenne sanzione alla propria decisione lo soddisfaceva in pieno e lo tranquillizzava in parte. Gli era di sollievo pensare che nessuno potesse dire che, pur in momenti così gravi della sua vita, egli non avesse agito in conformità alle regole di quella religione, la cui bandiera aveva sempre tenuta alta tra la freddezza e la indifferenza generali. Pensando a particolari lontani nel futuro, Aleksej Aleksandrovic vedeva anche possibile che i suoi rapporti con la moglie potessero permanere quasi gli stessi di prima. Certamente egli non avrebbe mai più potuto ridarle la stima; ma non c'era alcuna ragione che la vita di lui fosse sconvolta, e venisse a soffrire lui del fatto che sua moglie era stata infedele e perversa. ``Sì, passerà del tempo, il tempo che accomoda tutto, e si ristabiliranno i rapporti di prima - diceva a se stesso Aleksej Aleksandrovic - si ristabiliranno cioè in modo che io non subirò più alcun squilibrio nel corso della mia vita. Deve essere infelice lei che è colpevole, non io che non lo sono, e che perciò non posso essere infelice''. 

\capitolo{XIV}Avvicinandosi a Pietroburgo, Aleksej Aleksandrovic non solo era risolutamente fermo in questa sua decisione, ma aveva già composta nella sua mente la lettera che avrebbe scritto alla moglie. Entrato in portineria, dette uno sguardo alle lettere e alle pratiche del ministero, e ordinò di portarle subito nel suo studio. 

- Rimandate chiunque: non ricevo nessuno - rispose alla domanda del portiere, accentuando le parole ``non ricevo nessuno'' con un certo compiacimento, il che indicava una buona disposizione di spirito. 

Giunto nello studio, Aleksej Aleksandrovic lo percorse due volte; si fermò presso l'enorme scrittoio, sul quale erano già state accese dal cameriere, che vi era entrato prima, sei candele; fece scricchiolare le dita e sedette, disponendo l'occorrente per scrivere. Poggiò i gomiti sul tavolo, inclinò la testa da un lato, pensò per circa un minuto, e cominciò a scrivere senza fermarsi un attimo. Scriveva a lei senza intestazione, e in francese, usando il pronome ``voi'' che in francese non ha quel tono di freddezza che ha in russo. 

\begin{quote}
``Nell'ultima nostra conversazione vi ho espresso il proposito di comunicarvi la mia decisione riguardo all'oggetto di essa. Dopo aver attentamente riflettuto a tutto, vi scrivo per adempiere quella promessa. La mia decisione è la seguente: quali che siano le vostre azioni, io reputo di non aver il diritto di spezzare quei vincoli con i quali siamo legati da un potere che viene dall'alto. La famiglia non può essere distrutta dal capriccio, dall'arbitrio o, peggio, dalla colpa di uno dei coniugi, e la nostra vita deve procedere come è proceduta sinora. Ciò è indispensabile per me, per voi, per nostro figlio. Sono convinto che siate pentita o vi pentiate di quanto ha dato occasione a questa lettera e che coadiuverete con me per estirpare dalla radice la causa del nostro contrasto e dimenticare il passato. In caso contrario, potete voi stessa prevedere quel che attende voi e vostro figlio. Di tutto questo spero parlarvi più dettagliatamente a voce. Giacché il tempo della villeggiatura sta per finire, vi prego di rientrare al più presto a Pietroburgo, non più tardi di martedì. Saranno date tutte le disposizioni necessarie per il vostro trasferimento. Vi prego notare che attribuisco una particolare importanza all'adempimento di questa mia richiesta.

A. Karenin

P. S. - In questa lettera è accluso il denaro che potrà essere necessario per le vostre spese''.
\end{quote} 

Lesse la lettera e ne rimase soddisfatto, particolarmente per il fatto che si era ricordato di accludervi il denaro; non vi era una sola parola dura, né una recriminazione, ma non vi era neppure indulgenza. C'era soprattutto un ponte d'oro per il ritorno. Piegata la lettera, spianatala col grosso tagliacarte di avorio massiccio e postala in una busta insieme col denaro, con la soddisfazione che sempre gli procurava l'uso dei suoi oggetti da scrittoio disposti in bell'ordine, sonò. 

- La consegnerai al corriere perché la faccia avere ad Anna Arkad'evna , domani in campagna - disse e si alzò. 

- Va bene, eccellenza. Volete il tè nello studio? 

Aleksej Aleksandrovic ordinò di servire il tè nello studio e, giocando col tagliacarte massiccio, andò verso la poltrona accanto alla quale erano preparati una lampada e un libro francese sulle Tavole eugubine del quale aveva iniziato la lettura. Sopra la poltrona era appeso un ritratto ovale di Anna, molto ben fatto, di un noto artista. Aleksej Aleksandrovic lo guardò. Gli occhi impenetrabili lo fissavano con ironia e impudenza, come in quell'ultima sera della loro spiegazione. Intollerabilmente impudente e provocante era per lui la vista del merletto nero, posato sulla testa, ed eseguito in modo perfetto dall'artista, dei capelli neri e della bellissima mano bianca dall'anulare coperto di anelli. Dopo aver guardato il ritratto per circa un minuto, si agitò tanto che le labbra gli tremarono ed emisero il suono ``brr'', ed egli si voltò dall'altra parte. Sedutosi in fretta nella poltrona, aprì il libro. Cominciò a leggere, ma non seppe ritrovare l'interesse, prima sempre vivo, per le Tavole eugubine. Scorreva il libro, ma pensava ad altro. Non alla moglie, ma a un complicato affare venuto fuori nell'ultimo periodo della sua attività di statista e che, in quel momento, costituiva la cosa più interessante del suo ufficio. Aveva esaminato profondamente quel problema, e ora nella sua mente nasceva un'idea geniale che, poteva dirlo senza vanteria, avrebbe risolto in pieno quell'affare, avrebbe fatto progredir lui nella carriera, debellato i suoi nemici ed arrecato grande utilità allo stato. Non appena il cameriere, portato il tè, uscì dalla stanza, Aleksej Aleksandrovic si alzò e andò allo scrittoio. Fatta avanzare nel centro la cartella degli affari in corso, ebbe un impercettibile sorriso di soddisfazione, tirò fuori dal portapenna una matita e si sprofondò nella lettura della pratica di quel complesso affare che s'era fatto portare e che riguardava la complicazione sopraggiunta. La complicazione era la seguente. La speciale qualità di Aleksej Aleksandrovic come uomo di governo, il tratto caratteristico tutto suo, e del resto di ogni impiegato che fa carriera, che insieme alla sua ostinata ambizione, al suo contegno, all'onestà e alla presunzione era stata la ragione della sua rapida carriera, consisteva nel disdegno verso il carattere burocratico degli affari, nella riduzione al minimo della corrispondenza, nel porsi a contatto diretto, per quanto possibile, col punto vivo delle questioni e nella economia di queste. Nella famosa commissione del 2 giugno, era stata esumata la questione della irrigazione dei campi del governatorato di Zarajsk, pratica che giaceva accantonata presso il ministero di Aleksej Aleksandrovic e che rappresentava un evidente esempio di spese infruttuose e di lungaggini burocratiche. Aleksej Aleksandrovic riteneva giusta la faccenda. Questa pratica era stata iniziata dal predecessore di Aleksej Aleksandrovic e infatti per tale affare si era speso e si spendeva molto denaro in modo del tutto improduttivo e non si veniva mai a capo di nulla. Aleksej Aleksandrovic, entrato in carica, s'era subito accorto di tali manchevolezze, e parve voler mettere mano alla cosa; ma nei primi tempi, quando non si sentiva ancora del tutto sicuro, sapeva che la faccenda ledeva troppi interessi e non era stata bene impostata; sopravvenute poi altre questioni, se ne era semplicemente dimenticato. Così questa, come tutte le altre questioni, andava avanti da sé, per forza di inerzia. (Molte persone ci mangiavano su, specialmente una famiglia molto per bene e amante della musica in cui tutte le figlie sonavano strumenti a corda. Aleksej Aleksandrovic conosceva questa famiglia, ed era padrino di una delle figlie maggiori). 

Secondo l'opinione di Aleksej Aleksandrovic, non era stato onesto che un ministero contrario avesse richiamato l'attenzione su quell'affare, perché in ogni ministero vi erano affari che, per certe convenienze di servizio, nessuno metteva in luce; ma poiché gli era stato lanciato quel guanto, egli coraggiosamente lo raccolse, e cominciò col richiedere una commissione speciale per lo studio e la verifica dei lavori affidati alla commissione per l'irrigazione dei campi del governatorato di Zarajsk; eliminando, però, ogni favoritismo anche nei rapporti di quei tali signori. 

Pretese la nomina di una commissione speciale per l'affare della sistemazione degli allogeni. Questo affare era stato messo in evidenza, per caso, nel comitato del 2 giugno ed era sostenuto con energia da Aleksej Aleksandrovic come cosa che non sopportava dilazione, dato lo stato deplorevole degli allogeni. Nel comitato questo affare servì di pretesto a dispute tra alcuni ministeri. Il ministero contrario ad Aleksej Aleksandrovic asseriva che la condizione degli allogeni era diventata sempre più florida, che la sistemazione della proposta poteva rovinare questa floridezza, e che, se qualcosa c'era da lamentare, questo qualcosa dipendeva solo dall'inadempimento, da parte del ministero di Aleksej Aleksandrovic, delle misure prescritte dalla legge. Ora Aleksej Aleksandrovic aveva intenzione di pretendere: in primo luogo, che venisse costituita una commissione nuova a cui affidare l'incarico di accertare sul posto quale fosse la condizione reale degli allogeni; in secondo luogo, se la condizione degli allogeni fosse risultata tale quale appariva dai dati ufficiali che erano nelle mani del comitato, si nominasse un'altra commissione, scientifica questa, per lo studio delle cause della desolante condizione degli allogeni dal punto di vista: a) politico, b) amministrativo, c) economico, d) etnografico, e) materiale, f) religioso; in terzo luogo, che si ordinasse al ministero contrario di fornire notizie sulle misure da esso attuate nell'ultimo decennio per eliminare le condizioni svantaggiose in cui versavano ora gli allogeni; infine, in quarto luogo, che si imponesse allo stesso ministero di dar conto delle ragioni per le quali esso, come risultava dalle notizie fornite al comitato sotto i nn. 17015 e 108308 del 5 dicembre 1863 e 7 giugno 1864, aveva agito proprio in modo opposto alle prescrizioni della legge fondamentale e organica, vol. \ldots{} art. 18 e nota dell'art. 36. Un colorito di animazione copriva a poco a poco il viso di Aleksej Aleksandrovic mentre egli fissava in breve lo schema di queste idee. Dopo aver riempito un foglio di carta, sonò e dette un biglietto per il direttore della cancelleria per avere dei dati che gli occorrevano. Alzatosi e fatto un giro per la stanza, guardò di nuovo il ritratto, aggrottò le sopracciglia e sorrise con disprezzo. Dopo aver letto ancora un po' il libro sulle Tavole eugubine e risvegliato in sé l'interesse verso di queste, Aleksej Aleksandrovic, alle undici, andò a dormire, e quando, supino nel letto, ricordò quello che era accaduto con la moglie, la cosa non gli apparve più sotto un aspetto così fosco. 

\capitolo{XV}Sebbene Anna avesse ribattuto Vronskij con tenacia e irritazione, quando egli le aveva detto che la posizione era insostenibile e l'aveva esortata a dire tutto al marito, ella intimamente sentiva falsa e disonesta la propria posizione e desiderava con tutta l'anima di cambiarla. Tornando col marito dalle corse, in un momento di impeto, gli aveva detto tutto; malgrado la pena provata, era contenta. Dopo che il marito l'aveva lasciata, ella si andava dicendo che ora ogni cosa si sarebbe definita e che, per lo meno, non ci sarebbero più stati la menzogna e l'inganno. Le sembrava fuor di dubbio che adesso la sua posizione si sarebbe definita per sempre. Poteva anche essere non buona questa nuova sua posizione, ma sarebbe sempre stata definita, e non più ambigua e mendace. Il dolore ch'ella aveva causato a se stessa e al marito nel pronunziare quelle parole, sarebbe stato compensato, così ella immaginava, dal fatto che tutto si sarebbe definito. La sera stessa, però, ella si era vista con Vronskij, e non gli aveva riferito nulla di quello che era accaduto tra lei e il marito, sebbene fosse evidente la necessità di parlargliene per chiarire la situazione. 

Quando l'indomani si svegliò, la prima cosa che le si presentò alla mente fu il colloquio col marito, e quelle parole pronunciate le risonarono così orribili che non riusciva più a capire come si fosse decisa a pronunciarle, quelle strane parole, e non riusciva a immaginare gli effetti che ne sarebbero derivati. Ma le parole erano state dette, e Aleksej Aleksandrovic era andato via senza dir nulla. ``Ho visto Vronskij e non gliene ho parlato. Mentre andava via, volevo richiamarlo per parlargli ma poi ho cambiato idea, perché mi pareva strano non avergli detto nulla al primo momento. Perché glielo volevo dire e non l'ho detto?''. E in risposta a questa domanda una vampa di rossore le si diffuse sul viso. Capiva ora perché aveva taciuto; capiva ora perché se ne vergognava. La posizione sua, che le era sembrata chiarita la sera innanzi, le si presentava ora, non solo poco chiara, ma senza via d'uscita. Aveva lo sgomento del disonore, al quale prima non aveva mai neppure pensato. Appena immaginava quello che il marito avrebbe fatto, le si affacciavano i pensieri più paurosi. Le venne in mente che sarebbe venuto subito l'intendente a cacciarla di casa, che il suo disonore sarebbe stato rivelato a tutto il mondo. Si chiedeva dove sarebbe andata a finire se fosse stata cacciata di casa, e non trovava risposta. 

Quando pensava a Vronskij le pareva che egli non l'amasse più, che già cominciasse ad essere stanco di lei, che lei non potesse offrirglisi più, e sentiva per questo un astio verso di lui. Le sembrava che le parole, che ella aveva dette al marito e che incessantemente ripeteva nella mente, le avesse dette a tutti e che tutti le avessero udite. Non poteva decidersi a guardare negli occhi le persone con le quali viveva. Non poteva decidersi a chiamare la cameriera e ancor meno a uscir fuori dove erano il figlio e la governante. 

La cameriera, che già da tempo era in ascolto presso la porta, entrò senza essere chiamata. Anna la guardò interrogativamente negli occhi e arrossì come spaurita. La donna si scusò di essere entrata, dicendo che le era parso di sentir sonare. Aveva portato un vestito e un biglietto. Il biglietto era di Betsy. Betsy le ricordava che quella mattina si riunivano da lei Liza Merkalova e la baronessa Stoltz coi loro corteggiatori Kaluzskij ed il vecchio Stremov per una partita di croquet. ``Venite anche solo per guardare, come studio di costumi. Vi aspetto'' terminava. 

Anna lesse il biglietto e sospirò profondamente. 

- Non mi occorre nulla, nulla - disse ad Annuška che le cambiava di posto le boccette e le spazzole sul tavolo da toeletta. - Va', mi vesto ed esco. Non ho bisogno di nulla, di nulla. 

Annuška uscì, ma Anna rimase a sedere nella medesima posizione in cui era, il capo e le braccia abbandonati. Di tanto in tanto un tremito le percorreva tutto il corpo: avrebbe voluto fare un movimento qualsiasi, dire qualche cosa e invece si immobilizzava di nuovo. Ripeteva continuamente: ``Dio mio, Dio mio''; ma né ``Dio'', né ``mio'' avevano senso alcuno. Il pensiero di cercare aiuto nella fede, malgrado non avesse mai avuto dubbi sulla religione nella quale era cresciuta, le era altrettanto impossibile quanto cercare aiuto presso lo stesso Aleksej Aleksandrovic. Sapeva bene, infatti, che l'aiuto della religione sarebbe stato possibile solo se avesse rinunciato a tutto quello che ora costituiva per lei la ragione stessa della vita. Non solo era in pena, ma cominciava ad avere spavento del suo nuovo stato d'animo, finora mai provato. Sentiva che dentro di sé tutto cominciava a sdoppiarsi, come talvolta si sdoppiano gli oggetti agli occhi stanchi. Non sapeva di che cosa avesse paura, che cosa volesse: se avesse paura o desiderasse quello che era stato o quello che sarebbe stato, e che cosa precisamente desiderasse, non sapeva. 

``Ahi, ma che cosa sto facendo!'' si disse, sentendo a un tratto dolore ai due lati del capo. Quando rientrò in sé, si accorse che stringeva con tutte e due le mani le ciocche dei capelli delle tempie e le tirava. Si levò di scatto e fece dei passi. 

- Il caffè è pronto, e mamzel' e Serëza aspettano - disse Annuška, che, entrata di nuovo, aveva trovato Anna nella stessa posizione di prima. 

- Serëza? Che fa Serëza? - chiese Anna, rianimandosi a un tratto e ricordandosi, per la prima volta in tutta la mattina, dell'esistenza del figlio. 

- A quanto pare, si è reso colpevole - rispose Annuška sorridendo. 

- Come s'è reso colpevole? 

- Nella stanza d'angolo c'erano delle pesche; sembra che lui, di nascosto, ne abbia mangiata una. 

Il ricordo del figlio strappò immediatamente Anna da quella situazione senza uscita nella quale si trovava. Ricordò l'atteggiamento, in parte sincero se pur molto esagerato, che ella aveva assunto negli ultimi anni, di madre che vive tutta pel suo figliuolo, e sentì con gioia che, pur nella condizione nella quale si trovava, era sempre in possesso di qualche cosa che stava a sé e per sé, indipendentemente da quelli che sarebbero stati i rapporti suoi col marito e con Vronskij. Questo suo possesso era il figlio. In qualsiasi condizione si fosse venuta a trovare non poteva abbandonare il figlio. La umiliasse pure e la scacciasse il marito, si raffreddasse pure Vronskij nei suoi riguardi e riprendesse a vivere la sua vita libera (pensò di nuovo a lui con rancore e rimprovero), ella non poteva abbandonare il figlio. Aveva uno scopo nella vita. E doveva agire, agire, per garantire questo suo rapporto col figlio, agire perché non glielo togliessero. Anzi presto, al più presto possibile doveva agire perché non avessero a sottrarglielo. Doveva prendere con sé il figlio e partire. Era l'unica cosa che doveva fare adesso. Doveva quindi calmarsi e uscire da quello stato di angoscia. Il pensiero di un'azione immediata collegata col figlio, il fatto di dover partire con lui per un luogo lontano, le dette questa tranquillità. 

Si vestì in fretta, scese a passi decisi, entrò nel salotto dove, di solito, era apparecchiato il caffè e dove l'attendevano Serëza e la governante. Serëza, tutto in bianco, stava in piedi presso la tavola sotto lo specchio e, la schiena e la testa chine, con una espressione di attenzione intensa che ben le era nota e che lo faceva rassomigliare al padre, giocava con dei fiori che aveva preso con sé. La governante aveva un'aria severa. Serëza proruppe in un grido acuto, come spesso gli accadeva: ``Oh mamma'' e stette incerto se correre ad abbracciarla e lasciare i fiori, o terminare la coroncina di fiori e andare poi da lei. 

La governante, dopo aver salutato, cominciò con lentezza e precisione a raccontare il misfatto commesso da Serëza; ma Anna non l'ascoltava: pensava se l'avrebbe portata o no con sé. ``No, non la condurrò - decise. - Partirò sola con mio figlio''. 

- Sì, va molto male - disse Anna, e preso il figlio per una spalla, lo guardò con occhio tutt'altro che severo, con uno sguardo timido che confuse e rallegrò il ragazzo, e lo baciò. - Lasciatelo con me - disse alla governante che la guardava sorpresa, e, senza lasciare la mano del figlio, sedette alla tavola dove era preparato il caffè. 

- Mamma, io\ldots{} io\ldots{} non\ldots{} - disse il ragazzo, cercando di capire dalla espressione del volto della madre che cosa dovesse venirgliene pel fatto della pesca. 

- Serëza - disse lei non appena la governante fu uscita dalla stanza - quel che hai fatto è male. Ma tu non lo farai più\ldots{} Mi vuoi bene? 

Sentiva le lacrime venirle agli occhi. ``Posso forse non amarlo? - si diceva, fissando lo sguardo del figlio spaventato e nello stesso tempo allegro. - È possibile mai che egli sia d'accordo col padre nel punirmi? È possibile che non abbia pietà di me?''. Già le lacrime le scorrevano pel viso, e per nasconderle, si alzò di scatto e si avviò quasi di corsa in terrazza. 

Dopo le piogge temporalesche degli ultimi giorni, era seguito un tempo freddo, ma limpido. Malgrado il sole vivido che passava attraverso le foglie lavate, l'aria era fredda. 

Al contatto dell'aria libera, rabbrividì, e per il freddo e per l'interno sgomento che con forza rinnovata l'assaliva. 

- Va', va' da Mariette - disse a Serëza che l'aveva seguita, e si mise a camminare sulla stuoia della terrazza. ``Possibile che non mi perdonino, che non capiscano che tutto quello che è accaduto non poteva non accadere?''. 

Indugiatasi a guardare le cime delle tremule che si dondolavano al vento con le foglie scintillanti e vivide nel sole freddo, capì che non le avrebbero perdonato, che tutto e tutti sarebbero stati spietati con lei, come quel cielo, come quel verde. E di nuovo sentì che nell'animo suo avveniva quel tale sdoppiamento. ``Non si deve pensare, non si deve - disse a se stessa - bisogna prepararsi. Per dove? Quando? Chi prendere con me? Sì\ldots{} a Mosca, col treno della sera. Annuška e Serëza, e soltanto le cose più indispensabili. Ma prima devo scrivere a tutti e due''. Entrò in fretta in casa, sedette al tavolo dello studio e scrisse al marito. 

``Dopo quello che è accaduto, non posso più rimanere nella vostra casa. Me ne vado e prendo con me mio figlio. Non conosco la legge e perciò non so con quale dei genitori debba stare il figlio; ma io lo prendo con me, perché senza di lui non posso vivere. Siate generoso, lasciatemelo''. 

Sino ad allora aveva scritto speditamente e con naturalezza, ma l'appello alla generosità che ella non gli riconosceva, e la opportunità di chiudere la lettera con qualche cosa di commovente, la fecero sostare. 

``Parlare della mia colpa e del mio pentimento non posso, perché\ldots{}''. 

Ristette di nuovo, non trovando un nesso tra i suoi pensieri. ``No, non ci vuol nulla più di quanto è strettamente necessario''. Ricopiò la lettera, eliminando l'accenno alla generosità, e la sigillò. 

L'altra lettera la doveva scrivere a Vronskij. ``Ho confessato a mio marito'' scrisse, ma poi non seppe andare avanti. Era così volgare, così poco attraente. ``E poi cosa posso scrivergli?''. Di nuovo il rossore della vergogna le coprì il volto. Pensò alla pace perduta, sentì rancore verso l'amante e con dispetto strappò in piccoli pezzi il foglietto con la frase scrittavi. ``Non occorre nulla'' si disse e, riposta la cartella, andò su, disse alla governante e alle persone di servizio che quel giorno sarebbe partita per Mosca, e subito si accinse a mettere insieme le sue robe. 

\capitolo{XVI}Per tutte le stanze della villa era un correre di portieri, giardinieri e servitori che portavano via la roba. Gli armadi e i cassettoni erano aperti; si era mandato già due volte di corsa alla bottega per lo spago; per terra si trascinava carta di giornali. Due bauli, le sacche e gli scialli da viaggio erano stati già portati in anticamera. Una carrozza padronale e due da nolo stavano ferme presso l'ingresso. Anna, che per il lavoro dei preparativi aveva abbandonato l'interna agitazione, approntava la sua sacca da viaggio stando in piedi accanto alla tavola nello studio, quando Annuška ne attirò l'attenzione verso un rumore di vettura che s'avvicinava. Anna guardò dalla finestra e vide presso la scala il fattorino di Aleksej Aleksandrovic che sonava alla porta d'ingresso. 

\begin{itemize} \itemsep1pt\parskip0pt\parsep0pt \item Va' a veder cos'è - disse e, in calma attesa, sedette in una poltrona, incrociando le mani sulle ginocchia. Il servitore portò un grosso plico con l'indirizzo di mano di Aleksej Aleksandrovic. \end{itemize} 

- Il fattorino ha l'incarico di portare la risposta - disse. 

- Va bene - rispose Anna e, appena l'uomo fu uscito, con le dita tremanti, strappò la busta. Ne venne fuori un plico di assegni non piegati, incollati in una fascetta. Ella liberò la lettera dalla busta e cominciò a scorrerla dalla fine. ``Ho disposto i preparativi per il trasferimento, do grande importanza all'adempimento della mia richiesta'' leggeva. Scorse più avanti, poi tornò indietro, lesse tutto e ancora una volta rilesse tutta la lettera daccapo. Quando ebbe finito sentì che aveva freddo e che su di lei era piombata una sventura così grande quale non poteva attendersi mai. 

La mattina s'era pentita d'aver parlato al marito e aveva desiderato una sola cosa, e cioè che quelle parole fossero come non dette. Ed ecco, la lettera riconosceva le parole come non dette, e le dava proprio quello che aveva desiderato. Ma ora questa lettera le appariva più terribile di tutto quello che avrebbe potuto immaginare. 

``Ha ragione, ha ragione! - si disse. - S'intende, egli ha sempre ragione; è cristiano, è magnanimo! Ma no! è vile, disgustoso! E questo, nessuno, all'infuori di me, lo capisce e nessuno lo capirà mai; e io non posso spiegarlo a nessuno. Gli altri dicono: è un uomo religioso, morale, onesto, intelligente; ma non sanno quello che so io. Non sanno ch'egli ha soffocato tutto quello che c'era di vivo in me; che neppure una volta gli è venuto in mente che io ero una donna viva che aveva bisogno d'amare. Non sanno che in ogni occasione mi ha umiliato, compiacendosene. Non ho forse cercato con tutte le mie forze di trovare uno scopo alla mia vita? Non ho forse provato ad amarlo, ad amare mio figlio quando già non potevo più amare lui? Ma è venuto poi il momento in cui ho compreso, in cui non mi è stato più possibile ingannare me stessa, in cui ho sentito che ero viva, che non avevo colpa se Dio mi aveva fatto così per l'amore e per la vita. E ora? Avesse ucciso me, avesse ucciso lui, avrei sopportato tutto, avrei perdonato tutto, ma no, egli\ldots{}''. 

``Com'è che non ho indovinato prima quello che avrebbe fatto? Che avrebbe fatto quello che è proprio conforme al suo carattere meschino? Egli avrà ragione e io sarò rovinata, io precipiterò ancora, ancora più in basso\ldots{}''. ``Voi stessa potrete supporre quello che attende voi e vostro figlio'' ricordava le parole della lettera. ``Questa è la minaccia di togliermi il figlio, e probabilmente, secondo la loro stupida legge, ciò è possibile. Ma forse non so perché dice così? Egli non crede neanche al mio amore per mio figlio e lo disprezza (così come l'ha sempre irriso), disprezza questo mio sentimento; non sa che io non abbandonerò mio figlio, che non posso abbandonarlo, che senza mio figlio non saprei vivere neppure con l'essere che amo, e sa pure che se abbandonassi mio figlio e fuggissi via da lui, agirei come la donna più abietta e svergognata, questo egli lo sa, e sa che io questo non avrò la forza di farlo''. 

``La nostra vita deve procedere come prima''; ella ricordò un'altra frase della lettera. ``Questa vita era tormentosa anche prima, è stata orribile negli ultimi tempi. Che sarà mai ora? Ed egli sa tutto questo, sa che io non potrò pentirmi di quello che ho fatto, che di questo io vivo, che amo; sa che oltre a menzogna e inganno non ne verrebbe fuori altro; ma egli sente il bisogno di continuare a tormentarmi. Io lo conosco, so che, come un pesce nell'acqua, egli nuota e gode nella menzogna. Ma no, io non glielo darò questo piacere, io spezzerò questa rete di menzogna nella quale egli vuole avvilupparmi; sarà quel che sarà. Tutto sarà preferibile alla menzogna e all'inganno!''. 

``Ma come, Dio mio! Dio mio! C'è forse al mondo una donna più infelice di me?''. 

- No, la spezzerò, la spezzerò! - gridò, scattando e trattenendo le lacrime. E si accostò allo scrittoio per scrivergli un'altra lettera, ma in fondo all'anima già sentiva che non avrebbe avuto la forza di uscire dalla situazione che era durata fino ad allora, per quanto falsa e disonorevole. Sedette allo scrittoio e, invece di scrivere, incrociò le braccia sul tavolo, poggiò la testa su di esse e pianse, così come piangono i bambini, singhiozzando e scotendo il petto. Piangeva perché quanto aveva sognato sulla chiarificazione e sistemazione del suo stato era distrutto per sempre. Tutto sarebbe rimasto così come prima, anzi molto peggio di prima. Sentiva che la posizione che occupava nella società alla quale apparteneva e che la mattina le era parsa cosa del tutto insignificante, quella posizione le era cara, sentiva che non avrebbe mai avuto l'ardire di cambiarla con l'altra ignominiosa della donna che lascia il marito e il figlio e si unisce all'amante; che per quanti sforzi facesse non sarebbe mai riuscita a far violenza a se stessa. Non avrebbe mai provato la libertà dell'amore, e sarebbe per sempre rimasta una moglie colpevole, sotto la minaccia continua d'essere accusata d'aver ingannato il marito per un legame infame con un altro uomo che era libero, ma col quale non poteva vivere una unica vita. Sapeva che così sarebbe stato, ed era tanto orribile tutto questo, che non poteva neppure immaginare come sarebbe andato a finire. E piangeva senza ritegno, così come piangono i bambini puniti. 

Il rumore dei passi del servitore la obbligò a ritornare in sé, e, nascondendo il viso, finse di scrivere. 

- Il fattorino vuole la risposta - riferì il servitore. 

- La risposta?\ldots{} sì - disse Anna - che aspetti, sonerò. 

``Che posso scrivere? - pensava. - Che posso decidere da sola? Che cosa so? Che cosa voglio? Che cosa desidero?''. Sentì che di nuovo nell'animo suo avveniva lo sdoppiamento. Ebbe di nuovo paura di questa sensazione e si aggrappò al primo pretesto di attività che le si parò innanzi e che potesse distrarla dal pensare a se stessa. ``Devo vedere Aleksej - così nel pensiero chiamava Vronskij - egli solo potrà dirmi cosa devo fare. Andrò da Betsy, forse lo vedrò là'' pensò, dimenticando completamente che proprio il giorno prima, quando gli aveva detto che non sarebbe andata dalla principessa Tverskaja, egli aveva soggiunto che perciò neanche lui ci sarebbe andato. Si accostò allo scrittoio, scrisse al marito: ``Ho ricevuto la vostra lettera. A.'' e, dopo aver sonato, la dette al servitore. 

- Non partiamo più - disse ad Annuška che era entrata. 

- Non partiamo proprio? 

- No, ma non disfate i bauli fino a domani e trattenete la carrozza. Io vado dalla principessa. 

- Quale abito devo preparare? 

\capitolo{XVII}Il gruppo della partita a croquet, alla quale la principessa Tverskaja aveva invitato Anna, doveva essere formato da due signore e dai loro rispettivi adoratori. Queste due donne erano le esponenti più in vista di un nuovo circolo scelto di Pietroburgo che si chiamava, a imitazione di qualche cosa già imitata, Les sept merveilles du monde. Queste signore appartenevano, è vero, a un ambiente elevato, ma questo era completamente ostile a quello frequentato da Anna. Inoltre il vecchio Stremov, una delle persone più influenti di Pietroburgo, l'adoratore di Liza Merkalova, era nemico, per ragioni di ufficio, di Aleksej Aleksandrovic. Per tutte queste considerazioni Anna non aveva voluto andare da Betsy, e a questo suo rifiuto si riferivano le allusioni della principessa Tverskaja nel biglietto che le aveva scritto. Ma ora Anna, nella speranza di incontravi Vronskij, volle andare. 

Anna giunse dalla principessa Tverskaja prima degli altri ospiti. 

Nel momento in cui entrava, il servitore di Vronskij, con le fedine ben lisce, simile a un gentiluomo di camera, entrava anch'esso. Si fermò sulla porta, e, toltosi il berretto, la lasciò passare. Anna lo conosceva, e solo in quel momento si ricordò che Vronskij il giorno innanzi le aveva detto che non sarebbe andato dalla principessa. Forse proprio per questo aveva mandato un biglietto. 

Ella aveva sentito, togliendosi il mantello nell'anticamera, come il servitore, che pronunciava perfino la lettera ``r'' come un gentiluomo di camera, aveva detto: ``da parte del conte alla principessa'' e aveva consegnato un biglietto. 

Avrebbe voluto chiedergli dove era il padrone. Avrebbe voluto tornare indietro e scrivere a Vronskij che venisse da lei, oppure andare lei stessa da lui. Ma né questa, né l'altra, né la terza cosa si potevano fare: si sentivano risonare già i campanelli che annunziavano nelle stanze attigue il suo arrivo e il servitore della principessa Tverskaja stava già di lato accanto alla porta aperta, aspettando che ella passasse nelle stanze interne. 

- La principessa è in giardino, sarà avvertita subito. Vuole avere la compiacenza di favorire in giardino? - disse un altro servo nella stanza accanto. 

La situazione era sempre la stessa, oscura, come a casa; anche peggiore, perché niente poteva fare, e non poteva vedere Vronskij e doveva restare in quell'ambiente così estraneo e così contrario alle sue condizioni di spirito. Ma Anna aveva un vestito che, lo sapeva, le stava bene; non era sola; intorno a lei c'era quell'abituale sfondo di ozio imperante e quindi stava meglio qui che a casa. Qui non doveva pensare a quel che avrebbe dovuto fare. Qui tutto andava da sé. A Betsy che le venne incontro in un abito bianco, la cui eleganza la colpì, Anna sorrise come sempre. La principessa Tverskaja stava con Tuškevic e una parente nubile che, con grande gioia dei genitori di provincia, passava l'estate presso la famosa principessa. 

Probabilmente in Anna c'era qualcosa d'insolito, perché Betsy lo notò subito. 

- Ho dormito male - rispose Anna, guardando il servitore che veniva loro incontro e che, secondo i suoi calcoli, portava il biglietto di Vronskij. 

- Come sono contenta che siate venuta! - disse Betsy. - Sono stanca, e proprio ora volevo prendere una tazza di tè, prima che gli altri arrivino. E voi - si rivolse a Tuškevic - potreste andare con Maša a provare il croquet-ground, là dove hanno tagliato l'erba. Io e voi avremo un po' di tempo per parlare un po' tra di noi prendendo il tè: will have a cosy chat, vero? - disse rivolta ad Anna con un sorriso, stringendole la mano che reggeva l'ombrellino. 

- Tanto più che non posso trattenermi a lungo da voi; devo andare dalla vecchia Vrede. Gliel'ho promesso da cento anni - disse Anna alla quale la bugia, estranea alla sua natura, non solo era divenuta facile e naturale, ma procurava perfino piacere. 

Perché avesse detto quello cui un minuto prima non pensava, non avrebbe potuto spiegarlo in nessun modo. Aveva detto ciò solo perché, non essendoci Vronskij, le era necessario esser sicura della propria libertà per cercare di vederlo in qualche modo. Ma perché proprio le fosse venuto sulle labbra il nome della vecchia damigella d'onore Vrede, dalla quale avrebbe dovuto andare come da tanti altri, non sapeva spiegarselo; e intanto, come apparve poi, nell'escogitare i mezzi accorti per incontrarsi con Vronskij, non riusciva a trovare niente di meglio. 

- No, non vi lascerò andare affatto! - rispose Betsy, guardando attenta Anna. - Invero mi offenderei, se non vi volessi bene. È come se temeste che la mia compagnia possa compromettervi. Per favore il tè per noi nel salottino - disse, socchiudendo come sempre gli occhi nel rivolgersi al servitore. Preso da lui il biglietto, lo lesse. - Aleksej ci ha fatto un brutto tiro - disse in francese - scrive che non verrà - aggiunse con un tono così naturale e semplice, come se mai le fosse passato per la mente che Vronskij potesse interessare Anna altrimenti che come giocatore di croquet. 

Anna sapeva che Betsy era al corrente di tutto, ma quando la sentiva parlare in sua presenza di Vronskij, per un momento si persuadeva ch'ella non sapesse nulla. 

- Ah - disse con indifferenza, come se poco le interessasse la cosa, e continuò sorridendo: - Come può compromettere qualcuno la vostra compagnia? - Questi giuochi di parole, questo voler celare il segreto avevano, come del resto per tutte le donne, un grande fascino per Anna. E non solo la necessità di nascondere, o lo scopo per cui si nasconde, ma lo stesso procedimento del nascondere la seduceva. - Io non posso essere più cattolica del papa - ella disse. - Stremov e Liza Merkalova sono il fior fiore della società. Poi sono ricevuti dovunque e io - accentuò in particolare quell'io - non sono mai stata severa ed intollerante. Non ne ho il tempo, ecco perché. 

- No, voi forse non volete incontrarvi con Stremov? Lasciate pure che lui e Aleksej Aleksandrovic spezzino delle lance al comitato; questo non ci riguarda. Ma in società egli è l'uomo più amabile che io conosca, ed è un appassionato giocatore di croquet. Ecco, vedrete. Malgrado la sua posizione ridicola di vecchio amatore di Liza, bisogna vedere come se la cava bene. È molto simpatico. Safo Stoltz, non la conoscete. È un tipo originale, proprio originale. 

Mentre Betsy parlava, Anna nello stesso tempo capiva, dallo sguardo vivace e intelligente di lei, ch'ella aveva intuito in parte la situazione sua, e stava ideando qualcosa. Esse erano in un piccolo studio. 

- Però bisogna scrivere ad Aleksej - e Betsy sedette al tavolo, scrisse alcune righe e mise in busta. - Scrivere che venga a pranzo. A pranzo mi rimane una signora senza cavaliere. Guardate, è efficace? Scusatemi, vi lascio un momento. Vi prego, sigillate e mandate. Devo dare un ordine. 

Senza esitare un attimo, Anna sedette al tavolo e, senza leggere la lettera di Betsy, vi scrisse in fondo: ``Mi è indispensabile vedervi. Venite nei pressi del giardino di Vrede. Vi sarò alle sei''. Sigillò, e Betsy, rientrata, consegnò, lei presente, la lettera. 

Proprio come aveva detto la principessa Tverskaja, durante il tè, che fu portato su di un tavolino-vassoio, in un piccolo salotto fresco, s'avviò fra le due donne a cosy chat, fino all'arrivo degli ospiti. Esse malignarono sulle signore che aspettavano e la conversazione indugiò su Liza Merkalova. 

- È molto carina, e mi è sempre stata simpatica - disse Anna. 

- Voi dovete volerle bene. Va pazza per voi. Ieri si è avvicinata a me dopo le corse ed era desolata di non avervi trovata. Dice che voi siete una vera eroina da romanzo e che se fosse un uomo farebbe mille sciocchezze per voi. Stremov dice che le fa lo stesso. 

- Ma ditemi, vi prego, io non ho mai potuto capire - disse Anna, dopo aver taciuto un po', e con un tono tale che mostrava chiaramente che la sua non era una domanda oziosa, ma era per lei più importante di quanto sarebbe dovuto apparire. - Ditemi, per favore, che c'è fra lei e il principe Kaluzskij, il cosiddetto Miška. Li ho visti poco. Che c'è? 

Betsy sorrise con gli occhi e guardò attentamente Anna. 

- C'è una maniera nuova - disse. - Hanno scelto questa maniera qua. Non badano alle convenienze. Ma c'è modo e modo di non curarsene. 

- Già, ma quali sono i suoi rapporti con Kaluzskij ? 

Betsy d'improvviso cominciò a ridere allegramente, cosa che accadeva di rado. 

- Voi invadete il campo della principessa Mjagkaja. Questa è una domanda da enfant terrible - e Betsy, evidentemente, voleva contenersi, ma non ci riusciva, e scoppiò in quel riso comunicativo delle persone che ridono di rado. - Bisogna chiederlo a loro - disse ridendo fino alle lacrime. 

- No, voi ridete - disse Anna, involontariamente contagiata dal riso - ma io non ho mai potuto capire. Non capisco, in questo, la parte del marito. 

- Il marito? Il marito di Liza Merkalova le porta gli scialli ed è sempre pronto a servirla. E più in fondo, in queste faccende, nessuno vuol ficcarci il naso. Vedete, nella buona società non si parla, e neppure si pensa, a certi particolari della toletta intima. Così anche per queste cose. 

- Sarete alla festa dei Rolandaki? - disse Anna per cambiar discorso. 

- Non credo - rispose Betsy e, senza guardare l'amica, riempì di tè le piccole tazze trasparenti. Avvicinata la tazza ad Anna, tirò fuori una sigaretta e, introdottala in un bocchino d'argento, si mise a fumare. 

- Ecco, vedete, io sono in una condizione felice - cominciò a dire ormai seria, dopo aver preso in mano la tazza. - Capisco voi e capisco Liza. Liza è una di quelle nature ingenue che, come i bambini, non capiscono che cosa sia bene e che cosa male. Almeno, non lo capiva quando era molto giovane. Ora sa che questa mancanza di discernimento le si addice: può darsi pure che non voglia capire di proposito - diceva Betsy con un sorriso sottile. - Tuttavia questo le si addice. Vedete, la stessa cosa può essere vista tragicamente, e divenire un tormento, mentre può essere considerata come di lieve importanza e divenire perfino piacevole. Voi forse sareste incline a considerare le cose troppo tragicamente. 

- Come vorrei conoscer gli altri, come conosco me stessa! - disse Anna seria e pensosa. - Sono peggiore o migliore degli altri? Peggiore, credo. 

- Siete un enfant terrible, un enfant terrible - ripeté Betsy. - Ma ecco che arrivano. 

\capitolo{XVIII}Si sentirono dei passi, una voce maschile, poi una voce femminile e delle risa; ed entrarono gli ospiti attesi: Safo Stoltz e un giovanotto sprizzante salute, il cosiddetto Vas'ka. Si vedeva che gli aveva giovato nutrirsi di carne sanguinolenta, di tartufi e di vino di Borgogna. Vas'ka s'inchinò alle signore e le guardò, ma solo per un attimo. Era entrato dopo Safo e l'aveva seguita nel salotto come se le fosse stato legato, senza staccar da lei gli occhi sfavillanti, che sembrava volessero mangiarsela. Safo Stoltz era una bionda dagli occhi neri. Entrò a piccoli passi svelti, sui tacchi alti delle scarpette, e strinse forte, da uomo, le mani alle signore. 

Anna non aveva finora incontrato mai, neanche una volta, questa nuova celebrità, e fu sorpresa della sua bellezza, dell'eccentricità del suo abbigliamento e dell'arditezza dei suoi modi. Sulla testa di capelli suoi e non suoi, d'un tenero color d'oro, era innalzata una tale impalcatura che la testa sembrava eguagliare in altezza il busto armoniosamente sporgente e molto scollato sul davanti. Lo slancio in avanti era tale che, ad ogni movimento, si disegnavano sotto al vestito le forme delle ginocchia e della parte superiore delle gambe, e involontariamente ci si chiedeva dove in realtà finisse, sotto quella costruzione ondeggiante, il suo vero corpo, piccolo e snello, tanto scoperto di sopra e tanto nascosto nelle sue parti inferiori. 

Betsy si affrettò a presentarla ad Anna. 

- Figuratevi, stavamo quasi per schiacciare due soldati - ella cominciò subito a raccontare, ammiccando, sorridendo e tirando indietro lo strascico che aveva al primo momento gettato troppo da un lato. - Andavo con Vas'ka\ldots{} Ah, sì, non vi conoscete. - E, pronunciando il nome di lui, presentò il giovanotto e, arrossendo, rise sonoramente del proprio errore, di averlo presentato cioè come Vas'ka a chi non lo conosceva. 

Vas'ka si inchinò ancora una volta ad Anna, ma non le disse nulla. Si rivolse a Safo: 

- La scommessa è perduta. Siamo arrivati prima, pagate - egli disse, sorridendo. 

Safo rise ancor più allegramente. 

- Non ora però - ella disse. 

- È lo stesso, incasserò dopo. 

- Va bene, va bene. Ah, sì - si rivolse improvvisamente alla padrona di casa. - Sono brava io\ldots{} Vi ho condotto un ospite. Ecco anche lui. 

Il giovane ospite inatteso che Safo aveva condotto, e di cui si era dimenticata, era però un ospite così importante che, malgrado la sua giovinezza, tutte e due le signore si alzarono ad accoglierlo. 

Era costui un nuovo adoratore di Safo. Anch'egli, come Vas'ka, la seguiva dappertutto. 

Ben presto giunsero il principe Kaluzskij e Liza Merkalova con Stremov. Liza Merkalova era una bruna magra con un viso sonnolento di tipo orientale e con degli occhi deliziosi, indefinibili, come dicevano tutti. Il genere del suo abbigliamento (Anna lo notò subito e lo apprezzò) era pienamente rispondente alla sua bellezza. Quanto Safo era brusca e sostenuta, tanto Liza era morbida e abbandonata. 

Ma Liza, secondo il gusto di Anna, era molto più attraente. Betsy aveva detto di lei ad Anna che aveva preso il tono della bambina incosciente; ma quando Anna la vide, sentì che non era vero. Era proprio la donna incosciente, corrotta, ma simpatica e docile. È vero che il suo tono era lo stesso di quello di Safo; così come per Safo, due adoratori la seguivano, uno giovane e l'altro vecchio, quasi fossero cuciti alle sue vesti, e la divoravano con gli occhi; ma in lei c'era qualcosa che era al di sopra di quanto la circondava, c'era in lei lo splendore schietto dell'acqua di un brillante fra i vetri. Questo splendore illuminava i suoi occhi deliziosi, davvero indefinibili. Lo sguardo stanco e nello stesso tempo appassionato di quegli occhi circondati da un cerchio scuro, stupiva per la sua completa sincerità. Dopo aver guardato in quegli occhi, sembrava a ognuno di conoscerla tutta e, conosciutala, di non poterla non amare. Alla vista di Anna, il suo viso si illuminò improvvisamente di un sorriso gioioso. 

- Ah, come son contenta di vedervi! - ella disse, avvicinandosi. - Ieri alle corse stavo per raggiungervi proprio nel momento in cui andavate via. Volevo tanto vedervi proprio ieri. Non è vero che è stato orribile? - ella disse, guardando Anna col suo sguardo che sembrava scoprire tutta l'anima. 

- Già, non m'aspettavo proprio che potesse impressionare tanto - disse Anna, arrossendo. 

Il gruppo si alzò in quel momento per andare in giardino. 

- Io non vengo - disse Liza, sorridendo e sedendosi accanto ad Anna. - Voi neppure andate? Non so che gusto ci sia a giocare a croquet! 

- No, mi piace - disse Anna. 

- Ecco, ecco, come fate voi a non annoiarvi? Si guarda voi e ci si rallegra. Voi vivete, e io mi annoio. 

- Come, vi annoiate? Fate parte del gruppo più allegro di Pietroburgo! - disse Anna. 

- Forse quelli che non sono della nostra compagnia si annoiano ancora di più; ma noi, noi non siamo allegri, io sicuramente no, e ci annoiamo terribilmente, terribilmente. 

Safo, accesa una sigaretta, uscì in giardino con i due giovanotti. Betsy e Stremov rimasero a prendere il tè. 

- Come, vi annoiate? - disse Betsy. - Safo ha detto che ieri si sono tanto divertiti a casa vostra. 

- Oh, una tale malinconia! - disse Liza Merkalova. - Sono venuti tutti da me dopo le corse. E sempre gli stessi, sempre gli stessi! E sempre la stessa cosa. Tutta la sera ci siamo trascinati per i divani. Che c'è di allegro? No, come fate voi a non annoiarvi? - si rivolse di nuovo ad Anna. - Basta guardarvi per dire: ecco una donna che può essere felice o infelice, ma che non si annoia. Insegnatemi, come fate? 

- Non faccio in nessun modo - rispose Anna, arrossendo per quelle domande insistenti. 

- Ecco il modo migliore - disse Stremov, entrando nella conversazione. 

Stremov era un uomo sui cinquant'anni, dai capelli grigi, ma ancora fresco, molto brutto, ma con un viso espressivo e intelligente. Liza Merkalova era nipote di sua moglie ed egli passava con lei le sue ore libere. Incontrata Anna Karenina, egli, nemico per ragioni di ufficio di Aleksej Aleksandrovic, come uomo di mondo e intelligente, aveva cercato di essere particolarmente gentile con lei, moglie del suo nemico. 

- In nessun modo - replicò, sorridendo con finezza - è il mezzo migliore. Da tempo dico - proseguì, rivolgendosi a Liza Merkalova - che, per non annoiarsi, bisogna non pensare che ci si annoia. Così come non si deve temere di non dormire se si ha paura dell'insonnia. Lo stesso vi ha detto Anna Arkad'evna. 

- Sarei molto contenta d'aver detto questo, perché non solo è intelligente, ma è la verità - disse Anna, sorridendo. 

- No, ditemi perché non si può dormire e non ci si può non annoiare? 

- Per dormire bisogna lavorare, ed anche per divertirsi, bisogna lavorare. 

- Perché dovrei lavorare, quando il mio lavoro non serve a nessuno? E fingere io non so e non voglio. 

- Siete incorreggibile - disse Stremov senza guardarla, e si rivolse di nuovo ad Anna. 

Incontrando di rado Anna, egli non poteva dirle che delle cose banali, e di queste cose le parlava: di quando sarebbe andata a Pietroburgo, del bene che le voleva la contessa Lidija Ivanovna, ma con una espressione tale che mostrava come egli desiderasse con tutta l'anima di riuscirle simpatico e mostrarle la sua considerazione e anche più. 

Entrò Tuškevic, annunziando che tutta la compagnia aspettava i giocatori di croquet. 

- No, non andate, vi prego - pregava Liza Merkalova avendo sentito che Anna andava via. Stremov si unì a lei. 

- È un troppo grande contrasto - egli diceva - andare, dopo di qua, dalla vecchia Vrede. Dopo tutto per lei sarete un'occasione per fare un po' di maldicenza, mentre qui voi potete suscitare soltanto i migliori sentimenti, i più lontani e opposti alla maldicenza - egli diceva. 

Anna rimase un attimo pensosa, per la indecisione. I discorsi lusinghieri di quell'uomo intelligente, la simpatia ingenua, infantile che le mostrava Liza Merkalova, e tutto quell'abituale apparato mondano, tutto ciò era così facile, mentre l'attendeva una cosa tanto difficile, che per un attimo fu incerta se rimanere e allontanare ancora il momento penoso della spiegazione. Ma, prospettandosi quello che l'avrebbe attesa poi nella solitudine della casa se non avesse preso alcuna decisione, ricordatasi di quel gesto terribile per lei, anche nella memoria, dei capelli tirati con tutte e due le mani, si scusò e andò via. 

\capitolo{XIX}Vronskij, malgrado la vita mondana apparentemente leggera, era un uomo che detestava il disordine. Ancora giovane, al corpo dei paggi aveva provato l'umiliazione di un rifiuto quando, trovandosi in cattive condizioni finanziarie, aveva chiesto del denaro in prestito, e da quella volta non si era messo mai più in una condizione simile. 

Per tenere sempre in ordine le sue cose, più o meno spesso, a seconda delle circostanze, si appartava un cinque volte all'anno e metteva in chiaro i suoi affari. Chiamava questo la resa dei conti, ovvero faire la lessive. 

Il giorno dopo le corse, svegliatosi tardi, senza radersi né fare il bagno, Vronskij indossò l'uniforme e, distribuiti sulla tavola il denaro, i conti e le lettere, si mise al lavoro. Petrickij, svegliatosi e visto il compagno alla scrivania, sapendo che in un momento simile era solito arrabbiarsi, si vestì piano e uscì senza dargli noia. 

Ogni uomo, conoscendo fin nei più piccoli particolari la complessità della propria situazione, presuppone involontariamente che tale complessità e la difficoltà di scioglierla siano cose esclusivamente attinenti alla propria persona, e non pensa in nessun modo che altri si trovino assediati da affari altrettanto complessi quanto i propri. Così pure sembrava a Vronskij. Ed egli, non senza un intimo compiacimento, e non senza ragione, pensava che chiunque altro, trovatosi in così difficili condizioni, si sarebbe da tempo messo negli impicci, e sarebbe stato costretto ad agire male. Ma Vronskij sentiva che proprio ora gli era indispensabile fare i conti e chiarire la sua situazione per non trovarsi negli impicci. 

La prima cosa a cui Vronskij si accinse, come alla più facile, furon gli affari di denaro. Copiato con la sua scrittura minuta sulla busta d'una lettera tutto quello che egli doveva, tirò la somma e trovò che doveva 17.000 rubli e alcune centinaia, che accantonò per sistemare tutto. Contato il denaro e aggiuntovi quello risultante dal libretto di banca, trovò che gli restavano 1.800 rubli, mentre incassi fino all'anno nuovo non se ne prevedevano. Rifatto il conto dei debiti, lo ricopiò dopo averlo diviso in tre gruppi. Nel primo gruppo trovavano posto i debiti che dovevano essere pagati subito, o per i quali, in ogni caso, bisognava tener pronto il denaro, in modo che alla richiesta seguisse il pagamento senza neppure un attimo di indugio. Questi debiti ammontavano a circa 4.000 rubli: 1.500 per il cavallo e 2.500 per la garanzia prestata al giovane compagno Veneskij che, in presenza di Vronskij, aveva perduto questo denaro, giocando con un baro. Vronskij voleva dare subito la somma (la possedeva), ma Veneskij e Jašvin avevano insistito per pagare loro e non Vronskij che non aveva neppure giocato. Tutto questo era una bellissima cosa, ma Vronskij sapeva che, pur avendo preso parte in questo sordido affare solo coll'assumere sulla parola la garanzia per Veneskij, gli era indispensabile aver pronti quei 2.500 rubli da buttare all'imbroglione per non aver più nulla a che fare con lui. Così, per questo primo importantissimo gruppo di debiti occorreva avere sotto mano 4.000 rubli. Nel secondo gruppo, di ottomila rubli, erano compresi debiti meno importanti. Erano in prevalenza debiti di scuderia da corsa, con l'inglese, col sellaio e via di seguito. Per tali debiti occorreva tenere da parte circa 2.000 rubli per essere completamente tranquillo. L'ultimo gruppo di debiti, verso fornitori, verso alberghi e verso il sarto, poteva essere trascurato. Ci volevano dunque almeno seimila rubli per le spese correnti e ce n'erano solo 1.800. Per un uomo con 100.000 rubli di rendita, a tanto si riteneva ammontasse il patrimonio di Vronskij, questi debiti potevano non sembrare troppo gravosi; ma erano ben lontani da lui quei 100.000 rubli! L'enorme patrimonio paterno, che rendeva da solo 200.000 rubli all'anno, era indiviso fra i fratelli. Al tempo in cui il fratello maggiore, pieno di debiti, s'era ammogliato con la principessina Varja cirkova, figlia del decabrista, senza un soldo, Aleksej aveva ceduto al fratello maggiore tutta la rendita del patrimonio paterno, riservando per sé solo 25.000 rubli all'anno. Aleksej aveva detto allora al fratello che questa somma gli sarebbe stata sufficiente fino al giorno in cui non si sarebbe ammogliato, il che probabilmente non si sarebbe verificato mai. E il fratello, comandante di uno dei reggimenti più fastosi e da poco sposato, fu ben lieto di accettare un simile dono. La madre, che aveva un patrimonio a sé, oltre ai 25.000 rubli fissi, dava ad Aleksej 20.000 rubli all'anno, e Aleksej li spendeva presto. Nell'ultimo tempo la madre, indispettita con lui per la sua relazione e per la sua partenza da Mosca, non gli aveva più mandato quel denaro. E perciò Vronskij, abituato a vivere con 45.000 rubli e ricevutine all'anno solo 25.000, si trovava in difficoltà. Per uscirne non poteva chiedere il denaro alla madre. L'ultima sua lettera, ricevuta il giorno prima, l'aveva particolarmente irritato, perché ella faceva intendere d'esser pronta ad aiutarlo perché avesse successo in società e in carriera, ma non per condurre una vita che scandalizzava tutta la buona società. L'intento della madre di ricattarlo l'aveva offeso nel profondo dell'anima ed aveva aumentato la sua freddezza verso di lei. D'altra parte, egli non poteva ritrattare la sua generosa rinunzia a favore del fratello, pur sentendo confusamente, in previsione di alcune eventualità derivanti dalla sua relazione con la Karenina, che quella generosa rinunzia era stata fatta con leggerezza, e che a lui, pur non sposato, potevano far comodo tutti i 100.000 rubli di rendita. Ma ritrattarsi non si poteva. Gli bastava solo pensare alla moglie del fratello, ricordare come quella gentile e simpatica Varja in ogni occasione opportuna gli ripetesse ch'ella ricordava la sua generosità e che l'apprezzava tanto, per capire l'impossibilità di togliere quello che era stato dato. Era impossibile quanto percuotere una donna, quanto rubare o mentire. Una sola cosa era possibile e si doveva fare, e ad essa Vronskij si decise senza un attimo di esitazione: prendere in prestito da un usuraio diecimila rubli, e questo non sarebbe stato difficile, ridurre in genere le proprie spese e vendere i cavalli da corsa. Deciso ciò, egli scrisse subito un biglietto a Rolandaki che più di una volta gli aveva proposto di comprargli i cavalli. Dopo mandò a chiamare l'inglese e l'usuraio, e distribuì secondo i conti i denari che aveva. Terminati questi affari, scrisse una fredda e tagliente risposta alla lettera della madre. Dopo, tirati fuori dal portafoglio tre biglietti di Anna, li rilesse, li bruciò e, riandando con la mente alla conversazione del giorno innanzi, si fece pensoso. 

\capitolo{XX}La vita di Vronskij era così particolarmente serena perché egli si era fatto un codice di regole che definiva in modo sicuro quello che si doveva e quello che non si doveva fare. Questo codice abbracciava una cerchia di casi molto limitata, ma in compenso queste norme erano sicure e Vronskij, non uscendo mai da quella cerchia, non aveva mai tentennamenti nelle sue azioni. Queste norme stabilivano in modo non dubbio che un baro lo si dovesse pagare, ma che non era necessario pagare il sarto; che non si dovesse mentire agli uomini, ma alle donne sì; che non si dovesse ingannare nessuno, ma che un marito lo si poteva ingannare senz'altro; che si dovessero perdonare le offese, ma che si poteva offendere, e via di seguito. Regole, queste, che potevano essere assurde, cattive, ma che erano sicure; adempiendole, Vronskij si sentiva tranquillo e poteva andare a testa alta. Negli ultimi tempi, però, in seguito alla sua relazione con Anna, Vronskij aveva cominciato a rendersi conto che il codice delle sue norme non contemplava proprio tutti i casi e che in futuro si sarebbero presentati dubbi e difficoltà nei quali egli già non trovava il filo conduttore. 

Gli attuali rapporti suoi con Anna e il marito erano per lui semplici e chiari. Essi erano chiaramente ed esattamente definiti nel codice di regole dalle quali egli si faceva guidare. 

Ella era una donna per bene che gli aveva dato il proprio amore, ed egli l'amava, perciò ella era per lui una donna degna dello stesso, e anche maggiore, rispetto che una moglie legittima. Si sarebbe fatto tagliare una mano prima di offenderla con una parola, con un'allusione, o di non mostrarle tutta la considerazione sulla quale può contare una donna. 

I rapporti con la società erano chiari anch'essi. Tutti potevano sapere, sospettare, ma nessuno doveva osare di parlare della sua relazione. In caso contrario era pronto a far tacere quelli che avrebbero parlato e a far rispettare l'onore, non più esistente, della donna che egli amava. 

I rapporti col marito erano i più chiari di tutti. Sin dal momento in cui Anna si era innamorata di lui, egli riteneva di avere su di lei, egli solo, un suo proprio diritto indiscutibile. Il marito era solo un personaggio superfluo e fastidioso. Senza dubbio ci faceva una figura pietosa, ma che farci? Un solo diritto aveva il marito: quello di pretendere soddisfazione con l'arma alla mano, e a questa eventualità Vronskij era stato pronto fin dal primo momento. 

Ma recentemente erano apparsi dei rapporti nuovi, intimi tra lui e lei, che avevano sconvolto Vronskij per la loro indeterminatezza. Appena ieri, ella gli aveva detto di essere incinta. Ed egli aveva sentito che questa notizia e la risposta ch'ella si aspettava da lui esigevano qualcosa che non rientrava nel codice delle norme che dirigevano la sua vita. Infatti era stato preso alla sprovvista, e al primo momento, quando ella gli aveva detto la cosa, il cuore gli aveva suggerito di pretendere che lasciasse il marito. Lo aveva subito detto, ma ora, riflettendo, vedeva chiaro che sarebbe stato meglio farne a meno; e intanto, dicendosi questo, temeva che ciò fosse riprovevole. 

``Se ho detto di lasciare il marito, questo significa unirsi con me. Sono io pronto a questo? Come la porterò via adesso, se non ho denari? Ammettiamo che a questo potrei provvedere\ldots{} ma come portarla via se sono tuttora in servizio? Ma se l'ho detto, è necessario che io sia pronto a farlo, debbo cioè avere del denaro e debbo dare le dimissioni''. 

E rifletteva. La questione di dare o no le dimissioni lo aveva portato a meditare su un altro suo intimo interesse, noto a lui solo, ma essenziale, anche se nascosto, per la sua vita. 

Il successo era una vecchia ambizione della sua infanzia e della sua giovinezza; sogno ch'egli non confessava neppure a se stesso, ma che era così forte che anche ora questa passione lottava col suo amore. I suoi primi passi nella società e nella carriera erano stati fortunati, ma due anni addietro aveva commesso un grosso errore. Per dar prova della propria indipendenza e di voler progredire, aveva rifiutato una posizione offertagli, sperando che questo rifiuto potesse conferirgli maggior prestigio; accadde invece che fu giudicato troppo temerario, e fu lasciato stare; e ora, volente o nolente, acquistatasi questa fama di uomo libero, cercava di sostenerla, comportandosi con finezza e intelligenza, in modo da parere che non avesse rancore contro nessuno, che non si considerasse offeso da nessuno, e che volesse solo starsene in pace, perché contento di sé. Ma, in fondo, fin dall'anno scorso, quando era andato a Mosca, aveva cessato di esserlo. Sentiva che questa condizione di uomo indipendente, che può tutto e non vuole nulla, cominciava a diventar piatta; già molti cominciavano a pensare ch'egli non avrebbe potuto nulla, fuorché essere un onesto e bravo ragazzo. La sua relazione con la Karenina, che aveva fatto tanto scalpore, e che aveva attirato l'attenzione generale, dandogli nuovo prestigio, aveva calmato per un certo tempo in lui il tarlo dell'ambizione; ma da una settimana in qua questo tarlo s'era ridestato con rinnovata energia. Un amico d'infanzia, della stessa cerchia, dello stesso ambiente, e suo compagno al corpo dei paggi, Serpuchovskoj, licenziatosi con lui e suo rivale in classe e in ginnastica, in birbonate e in sogni ambiziosi, era tornato in quei giorni dall'Asia centrale, dopo aver ricevuto due promozioni e una ricompensa che era data di rado a generali così giovani. 

Appena giunto a Pietroburgo, si era parlato di lui come di un astro di prima grandezza che sorgeva. Coetaneo di Vronskij e compagno suo di collegio, egli era generale e aspettava una nomina che poteva avere influenza sul corso degli affari di stato, mentre lui, Vronskij, sebbene indipendente e brillante e amato da una donna deliziosa, era un semplice capitano al quale si lasciava la libertà di essere indipendente quanto e come voleva. ``S'intende, io non invidio e non posso invidiare Serpuchovskoj, ma il suo successo mi dimostra che basta aspettare il momento buono, e la carriera di un uomo come me può essere fatta ben presto. Tre anni fa egli era nella stessa condizione nella quale mi trovo io ora. Dando le dimissioni, brucerei le mie navi. Rimanendo in servizio non perdo nulla. Ella stessa ha detto che non vuole cambiare lo stato delle cose. E io che posseggo il suo amore, non posso invidiare Serpuchovskoj''. E, arricciandosi con un movimento lento i baffi, si alzò dalla tavola e fece un giro per la stanza. I suoi occhi splendevano in modo particolarmente chiaro ed egli sentiva quella disposizione d'animo ferma, tranquilla e gioiosa che lo prendeva sempre quando aveva chiarito la propria posizione. Tutto era così netto e preciso come dopo i conti che aveva sistemato poco prima. Si rase la barba, s'immerse in un bagno freddo e uscì. 

\capitolo{XXI}-E io ti vengo dietro. Il bucato è durato un pezzo, oggi - disse Petrickij. - Be', è finito? \\
- È finito - rispose Vronskij, sorridendo soltanto con gli occhi e arricciando la punta dei baffi così cautamente come se, dopo l'ordine in cui erano stati messi i suoi affari, ogni movimento troppo ardito e lesto potesse distruggerlo. 

- Fatto questo sembra proprio che tu esca da un bagno - disse Petrickij. - Io vengo da Griška - così chiamavano il comandante del reggimento - ti aspettano. 

Vronskij, senza rispondere, guardò il compagno, pensando ad altro. 

- Sì, c'è musica da lui? - disse, prestando orecchio alle note emesse dalla cornetta a tempo di polca e di valzer che giungevano fino a lui. - Cos'è, c'è festa? 

- È arrivato Serpuchovskoj. 

- Ah - disse Vronskij - nemmeno lo sapevo. 

Il sorriso dei suoi occhi brillò ancor più chiaramente. 

Una volta che aveva stabilito con se stesso d'esser felice del suo amore e di aver sacrificato ad esso la propria ambizione, assunta, almeno, questa parte, Vronskij non poteva sentire né invidia per Serpuchovskoj, né irritazione verso di lui perché, arrivato al reggimento, non era venuto da lui per primo. Serpuchovskoj era un buon amico, ed egli era felice di rivederlo. 

- Ah, ne sono lieto. 

Il comandante del reggimento, Demin, occupava una grande casa di possidenti. Tutta la compagnia era sul vasto terrazzo di sotto. Nel cortile, la prima cosa che saltò agli occhi furono i cantanti in uniforme estiva, in piedi, accanto a una piccola botte di vodka, e la sana, allegra figura del comandante circondato dagli ufficiali. Venendo fuori sul primo gradino del terrazzo, costui, gridando più forte della musica che sonava una quadriglia di Offenbach, ordinò qualcosa e fece alcuni cenni ai soldati che stavano da un lato. Il gruppo di soldati, di marescialli e di sottufficiali si accostò al terrazzo insieme a Vronskij. Tornato presso al tavolo, il comandante del reggimento venne fuori sulla scala con una coppa in mano e pronunciò il brindisi: ``Alla salute del nostro antico compagno e valoroso generale, principe Serpuchovskoj. Urrà!''. 

Dietro il comandante uscì anche Serpuchovskoj con una coppa in mano. 

- Tu diventi sempre più giovane, Bondarenko - disse rivolto a un ben fatto, rubicondo maresciallo che era stato richiamato in servizio per la seconda volta, e che stava diritto davanti a lui. 

Vronskij non vedeva Serpuchovskoj da tre anni. Questi aveva preso un aspetto più maschio con le fedine più folte, ma era rimasto snello quale era e sorprendeva, non tanto per la bellezza, quanto per la delicatezza e nobiltà del viso e della figura. Il solo mutamento che Vronskij notò in lui, fu quel calmo continuo splendore che si fissa sul volto delle persone che hanno successo e che sono sicure del riconoscimento di questo successo da parte di tutti. Vronskij conosceva questo splendore e subito lo notò in Serpuchovskoj. 

Scendendo la scala, Serpuchovskoj scorse Vronskij. Un sorriso di gioia gli illuminò il volto. Fece un cenno con la testa, sollevò la coppa, salutando Vronskij e mostrando con questo gesto che voleva avvicinarsi prima al maresciallo che, inchinatosi, piegava già le labbra al bacio. 

- Su, ecco anche lui! - gridò il comandante del reggimento. - E Jašvin mi ha detto che eri di umore nero! 

Serpuchovskoj dette un bacio sulle umide e fresche labbra del bel giovane maresciallo e, asciugandosi la bocca col fazzoletto, si accostò a Vronskij. 

- Eh, come son contento! - disse stringendogli la mano e appartandosi con lui. 

- Occupatevi di lui! - gridò a Jašvin il comandante del reggimento, indicando Vronskij, e scese giù dai soldati. 

- Perché ieri non eri alle corse? Pensavo di vederti là - disse Vronskij esaminando Serpuchovskoj. 

- Sono venuto, ma tardi. Perdona - soggiunse, e si rivolse all'aiutante di campo. - Per favore ordinate di distribuire da parte mia a ognuno il suo. 

Ed in fretta, tirò fuori dal portafogli tre biglietti da cento rubli e arrossì. 

- Vronskij! Qualcosa da mangiare, o da bere? - chiese Jašvin. - Ehi, da' da mangiare qui al conte. Ed ecco, bevi. 

La baldoria dal comandante si protrasse a lungo. 

Si bevve molto. Dondolarono e gettarono in aria Serpuchovskoj. Dopo si fece dondolare il comandante del reggimento. Poi, davanti ai cantanti, ballò lo stesso comandante con Petrickij. Dopo, il comandante del reggimento, già infiacchito, sedette su di una panca nel cortile e cominciò a dimostrare a Jašvin la superiorità della Russia sulla Prussia, specie nell'attacco di cavalleria, e per un momento la baldoria si chetò. Serpuchovskoj entrò in casa, nella stanza da toletta, per lavarsi le mani, e ci trovò Vronskij che si versava addosso dell'acqua. Toltasi la divisa estiva e messo il collo rosso, coperto di peli, sotto il getto d'acqua del lavabo, frizionava il corpo con le mani. Finita l'abluzione, Vronskij sedette accanto a Serpuchovskoj. Tutti e due s'erano seduti su di un divanetto e tra loro cominciò una conversazione che interessava molto entrambi. 

- Io di te ho saputo tutto attraverso mia moglie - disse Serpuchovskoj. - Sono contento che tu la veda spesso. 

- È amica di Varja, e queste sono le uniche donne di Pietroburgo con le quali mi vedo volentieri - rispose, sorridendo Vronskij. Sorrideva perché prevedeva il tema su cui si sarebbe svolta la conversazione e gli faceva piacere. 

- Le uniche? - chiese di rimando, sorridendo, Serpuchovskoj. 

- Sì, e anch'io sapevo di te, ma non solo attraverso tua moglie - disse Vronskij, respingendo quella vaga allusione con un'espressione severa del volto. - Sono stato molto contento del tuo successo, ma per nulla affatto sorpreso. Mi aspettavo ancora di più. 

Serpuchovskoj sorrise. Gli faceva piacere, era evidente, l'opinione che si aveva di lui e non cercava di nasconderlo. 

- Io, al contrario, lo confesso sinceramente, m'aspettavo di meno. Ma sono contento, molto contento. Sono ambizioso, è questa la mia debolezza, lo confesso. 

- Forse non lo confesseresti, se non avessi successo - disse Vronskij. 

- Non credo - disse Serpuchovskoj, sorridendo di nuovo. - Non dico che non potrei vivere senza di questo, ma mi annoierei. S'intende, forse sbaglio, ma mi sembra di avere delle possibilità in quella sfera di azione che ho scelto, e mi pare che nelle mie mani il potere, quale che sia, se ci sarà, starà meglio che nelle mani di molti a me noti - disse Serpuchovskoj con la raggiante consapevolezza del successo. - E perciò quanto più sono vicino alla mèta, tanto più sono contento. 

- Forse questo va così per te, ma non per tutti. Io pensavo lo stesso, ma ecco che vivo e trovo che non vale la pena vivere solo per questo - disse Vronskij. 

- Eccolo, eccolo! - disse ridendo Serpuchovskoj. - Io avevo già cominciato a dire che avevo sentito parlare di te, del rifiuto\ldots{} S'intende, io ti ho approvato. Ma in ogni cosa ci vuole la misura. E io penso che il gesto in sé è stato buono, ma tu non hai agito così come si sarebbe dovuto. 

- Quel ch'è fatto è fatto; e tu sai, io non rimpiango mai. E poi sto benissimo. 

- Benissimo\ldots{} per un po' di tempo. Ma poi questo non ti basterà. Non direi così a tuo fratello. È un caro ragazzo, come questo nostro padrone di casa. Vedi - aggiunse, prestando orecchio al grido di ``urrà'' - anche lui si diverte, ma questo non può accontentare te. 

- Io non dico d'esser soddisfatto. 

- Già, ma non è solo questo. Uomini come te sono necessari. 

- A chi? 

- A chi? Alla società. La Russia ha bisogno di uomini, ha bisogno di un partito, altrimenti tutto va alla deriva. 

- Che cosa allora? Il partito di Bertenev contro i comunisti russi? 

- No - disse Serpuchovskoj, accigliandosi per la stizza di vedersi sospettato di una simile sciocchezza. - Tout ça est une blague. Questo è sempre stato e sarà. Non c'è nessun comunista. Ma le persone intriganti hanno sempre sentito la necessità di inventare un partito nocivo, pericoloso. Questo è un vecchio sistema. No, c'è bisogno di un partito di governo, di persone indipendenti come te e come me. 

- Ma perché mai? - e Vronskij nominò alcune persone che erano al potere. - Ma perché dici che non vi sono uomini indipendenti? 

- Solo perché non hanno o non hanno avuto dalla nascita una posizione indipendente, non hanno avuto un nome, né quella vicinanza al sole così come abbiamo avuto noi sin dalla nascita. Costoro si possono comprare col denaro o con la protezione. E lasciano passare delle idee e certe tendenze in cui essi non credono affatto, che danneggiano, al solo fine di avere una casa dal governo e tanto di stipendio. Cela n'est pas plus fin que ça, quando guardi nelle loro carte. Forse io sarò peggiore o più sciocco di loro. Ma ho certamente un vantaggio rilevante: che è più difficile comprarmi. E uomini cosiffatti sono più che mai necessari. 

Vronskij ascoltava attentamente, ma lo interessava non tanto il contenuto delle parole, quanto il modo col quale considerava le cose Serpuchovskoj, che pensava già di lottare per il potere e in quel mondo aveva già le sue simpatie e antipatie; mentre per lui nella carriera rientravano soltanto gli interessi dello squadrone. Vronskij intendeva quanto potesse essere forte Serpuchovskoj con la sua indubbia capacità a comprendere le cose, con la sua intelligenza e con il dono della parola così raro nella sfera in cui viveva. E per quanto se ne vergognasse, provava invidia. 

- Tuttavia per questo mi manca la dote principale - rispose - il desiderio del potere. L'ho avuto ma è passato. 

- Perdonami, non è vero - disse, sorridendo, Serpuchovskoj. 

- No, è vero, è vero, ora, ad essere sincero - aggiunse Vronskij. 

- Se è vero ora è un'altra cosa; ma questa ora non ci sarà sempre. 

- Può darsi - rispose Vronskij. 

- Tu dici, può darsi - continuò Serpuchovskoj, come indovinando il suo pensiero - e io ti dico certamente. E per questo volevo vederti. Tu hai agito così come dovevi. Questo lo capisco, ma perseverare non devi. Io ti chiedo solo carte blanche. Io non ti proteggo\ldots{} Benché, poi, perché non dovrei proteggerti? Tu hai protetto me tante volte! Spero che la nostra amicizia sia al di sopra di questo. Sì - egli disse, sorridendo teneramente come una donna. - Dammi carte blanche, esci dal reggimento e io ti rimetterò dentro inavvertitamente. 

- Ma capisci, non ho bisogno di nulla - disse Vronskij - se non di questo, che tutto continui ad essere così com'è stato. 

Serpuchovskoj si alzò e gli si mise di fronte. 

- Tu hai detto: che tutto continui ad essere così com'è stato. Io capisco perché dici così. Ma ascolta: noi siamo coetanei, può darsi che tu abbia conosciuto donne in maggior numero di me. - Il sorriso e i gesti di Serpuchovskoj dicevano che Vronskij non doveva temere, ch'egli avrebbe sfiorato con delicatezza, con riguardo il punto dolente. - Ma io sono ammogliato e, credimi, che pur conoscendo soltanto la propria moglie (come ha scritto qualcuno), se la ami, conosci tutte le donne meglio che se ne avessi conosciute mille. 

- Veniamo subito - gridò Vronskij all'ufficiale che era entrato un momento nella stanza e li invitava ad andare dal comandante del reggimento. 

Vronskij voleva ora ascoltare e sapere che cosa l'amico gli avrebbe detto. 

- Ed eccoti la mia opinione. Le donne sono la principale pietra di inciampo nell'attività di un uomo. È difficile amare una donna e fare qualcosa. C'è un solo mezzo per amare comodamente e scansare gli ostacoli, e questo mezzo è il matrimonio. Come, come dirti quello che penso - disse Serpuchovskoj, cui piacevano i paragoni. - Aspetta, aspetta! Sì, è come portare un fardeau e fare qualcosa con le mani; si può solo quando il fardeau è legato alla schiena, e questo è il matrimonio. E questo io l'ho sentito dopo essermi sposato. Mi si sono liberate a un tratto le mani. Ma senza il matrimonio, a trascinarsi dietro questo fardeau, le mani sono così impegnate, che non si può far nulla. Guarda Mazankov, Krupov. Si son giocata la carriera per le donne. 

- Quali donne! - disse Vronskij, pensando alla francese e all'attrice con cui erano in relazione le due persone nominate. 

- Tanto peggio se è più alta la posizione della donna in società: tanto peggio. È come se, invece di trascinare il fardeau con le mani, lo si strappasse a un altro. 

- Tu non hai mai amato - disse piano Vronskij, guardando avanti a sé e pensando ad Anna. 

- Forse. Ma ricordati quel che ti ho detto. E ancora. Le donne hanno tutte più senso pratico che non gli uomini. Noi facciamo dell'amore qualcosa d'immenso, ma esse sono sempre terre-à-terre. 

- Subito, subito! - disse rivolto al servo che era entrato. Ma il servo non era venuto per chiamarli, come egli pensava. Il servo portava un biglietto a Vronskij. 

- Un uomo ha portato questo da parte della principessa Tverskaja. 

Vronskij dissuggellò la lettera e diventò rosso. 

- M'è venuto mal di testa, vado a casa - disse a Serpuchovskoj. 

- Allora, addio. Mi dai carte blanche? 

- Ne riparleremo dopo, ti troverò a Pietroburgo. 

\capitolo{XXII}Erano già le sei e perciò, per giungere in tempo e non andare con i propri cavalli che tutti conoscevano, Vronskij prese posto nella vettura di Jašvin e ordinò di andare il più presto possibile. La vecchia carrozza di piazza a quattro posti era ampia. Sedette in un angolo, distese le gambe sul sedile davanti e si fece pensieroso. 

La coscienza confusa di quella sistemazione che aveva dato ai suoi affari, il ricordo vago dell'amicizia di Serpuchovskoj che lo riteneva un essere necessario e, soprattutto, l'attesa dell'incontro, tutto si fondeva in un unico gioioso senso di vita. Questa sensazione era così forte che egli involontariamente sorrise. Tirò giù le gambe, mise l'una sul ginocchio dell'altra, e, presala in mano, tastò il polpaccio elastico della gamba ferita il giorno prima nella caduta, e, riversatosi all'indietro, respirò varie volte a pieni polmoni. 

``Bene, molto bene!'' si disse. Anche altre volte aveva provato la gioiosa sensazione del proprio corpo come ora. Gli piaceva sentire quel leggero dolore nella gamba solida, gli piaceva la sensazione muscolare del movimento del proprio petto nel respirare. Quella stessa chiara e fresca giornata d'agosto, che così disperatamente aveva agito su Anna, pareva a lui eccitante e vivificante e gli rinfrescava il viso e il collo accaldati dall'abluzione. L'odore della brillantina dei suoi baffi gli pareva particolarmente piacevole in quell'aria fresca. Tutto ciò che vedeva dal finestrino della carrozza, in quell'aria fredda e tersa, nella luce pallida del tramonto era egualmente fresco, allegro e forte come lui; così i tetti delle case, che rilucevano ai raggi del sole calante, e i contorni netti dei recinti e degli angoli delle costruzioni, così le sagome dei pedoni e delle vetture che si incontravano di rado, così il verde immobile degli alberi e delle erbe, e il campo con i solchi regolari delle patate, così le ombre contorte, cadenti dalle case e dagli alberi, dai cespugli, e dagli stessi solchi delle patate. Tutto era bello come un grazioso paesaggio allora allora finito e ricoperto di lacca. 

- Va', va' - disse, sporgendosi dal finestrino, e, tirato fuori dalla tasca un biglietto da tre rubli, lo ficcò in mano al vetturino che s'era voltato verso di lui. La mano del vetturino tastò qualcosa vicino al fanale, si sentì il fischio della frusta e la vettura rotolò in fretta sul lastrico levigato. 

``Non ho bisogno di nulla, oltre questa felicità - pensava, guardando il bottoncino d'osso del campanello tra gli spazi dei finestrini e immaginandosi Anna così come l'aveva vista l'ultima volta. - E più passa il tempo e più l'amo. Ecco anche il giardino della villa governativa della Vrede. Dov'è mai? Dove? Come? Perché ha fissato qui l'appuntamento e ha scritto in una lettera di Betsy?'' pensava soltanto ora; ma non aveva già più tempo di pensare. Fece fermare i cavalli prima di arrivare al viale e, aperto lo sportello, saltò giù dalla carrozza in corsa e andò per il viale che porta alla casa. Nel viale non c'era nessuno ma, guardando a destra, scorse lei. Aveva il viso nascosto da un velo, ma egli, in uno sguardo gioioso, avvolse il movimento particolare, tutto suo, dell'andatura, dell'abbandono delle spalle, e della posizione del capo, e immediatamente qualcosa di simile a una corrente elettrica percorse il suo corpo. Sentì con rinnovata forza se stesso, dai movimenti elastici delle gambe, fino al moto dei polmoni in respirazione, e qualcosa gli vellicò le labbra. 

Incontratisi ella gli strinse forte la mano. 

- Non ti dispiace se ti ho fatto venire? Mi era indispensabile vederti - ella disse, e la piega seria e severa delle labbra ch'egli scorse di sotto al velo mutò di colpo la sua disposizione d'animo. 

- Io spiacente! Ma come sei venuta, da dove? 

- Non mette conto - ella disse, poggiando il braccio su quello di lui - andiamo, devo parlarti. 

Egli capì che qualcosa era accaduto e che quell'incontro non sarebbe stato lieto. Quando era con lei non aveva una volontà propria: non sapeva le ragioni dell'agitazione di lei e sapeva già che quella stessa agitazione gli si sarebbe comunicata. 

- Che c'è, che c'è? - chiedeva stringendo il braccio di lei col gomito e cercando di leggerle i pensieri nel viso. 

Ella fece qualche passo in silenzio e, facendosi coraggio, improvvisamente si fermò. 

- Non ti ho raccontato ieri - cominciò, respirando in fretta e con pena - che tornando a casa con Aleksej Aleksandrovic, io gli ho detto che non potevo più essere sua moglie, che\ldots{} tutto gli ho detto. 

Egli l'ascoltava, chinandosi involontariamente e con tutto il corpo, desiderando con questo di alleviare a lei il peso della sua situazione. Ma dopo quelle parole si drizzò improvvisamente e il suo viso prese un'espressione orgogliosa e severa. 

- Sì, sì, è meglio, mille volte meglio! Capisco come sia stato penoso - disse. 

Ma lei non ascoltava le sue parole, gli leggeva i pensieri nell'espressione del viso. Ella non poteva sapere che quell'espressione del viso si collegava alla prima idea che era venuta in mente a Vronskij: all'inevitabilità, adesso, del duello. A lei non era neppure venuta in mente l'idea del duello, e perciò dette una diversa spiegazione a questa fugace espressione di severità. 

Ricevuta la lettera del marito, ella sapeva già in fondo all'anima che tutto sarebbe rimasto come prima e ch'ella non avrebbe avuto la forza di buttar via la sua posizione sociale, di abbandonare il figlio e di unirsi all'amante. La mattinata trascorsa dalla principessa Tverskaja l'aveva rafforzata ancor più in questo: tuttavia questo incontro era straordinariamente importante per lei. Ella sperava che l'incontro avrebbe cambiato la loro situazione, che l'avrebbe salvata. Se egli a quella notizia, risolutamente, appassionatamente, senza un attimo di esitazione le avesse detto: ``lascia tutto e fuggi con me'' ella avrebbe abbandonato il figlio e sarebbe andata con lui. Ma la notizia datagli non produsse in lui l'effetto ch'ella s'attendeva: egli stava lì come offeso di qualcosa. 

- Non mi è stato per nulla penoso. È avvenuto da sé - ella disse con irritazione - ed ecco\ldots{} - ella trasse fuori dal guanto la lettera del marito. 

- Capisco, capisco - egli la interruppe, dopo aver preso la lettera e cercando, senza leggerla, di calmarla; - io desideravo una cosa sola, chiedevo una cosa sola, uscir fuori da questa situazione per dedicare la mia vita alla tua felicità. 

- Perché mi dici questo? - ella disse. - Posso forse dubitarne? Se dubitassi\ldots{} 

- Chi è che viene? - disse a un tratto Vronskij, indicando due signori che venivano alla loro volta. - Può darsi che ci conoscano - e in fretta si diresse in un viottolo laterale, tirandosela appresso. 

- Ah, per me è lo stesso! - ella disse. Le sue labbra tremavano. E a lui pareva che gli occhi di lei lo guardassero di sotto il velo con una strana cattiveria. - Così io dico che non è questo che importa, ora: di questo io non posso dubitare; ma ecco, cosa egli mi scrive. Leggi. - Si fermò di nuovo. 

Di nuovo come nel primo momento della notizia della rottura di lei col marito, Vronskij, nel leggere la lettera, si lasciò andare a quella sensazione istintiva che destavano in lui i rapporti col marito offeso. Ora, mentre teneva la lettera in mano, involontariamente si raffigurava la sfida che forse quel giorno stesso o l'indomani avrebbe trovato a casa, e persino il duello durante il quale, con quella stessa fredda e orgogliosa espressione che aveva in quel momento, avrebbe sparato in aria, e sarebbe rimasto sotto la mira del marito offeso. E a questo punto gli era balenato in mente quello che poco prima gli aveva detto Serpuchovskoj e che egli stesso aveva pensato la mattina, che sarebbe stato meglio non legarsi, e sentiva che questo suo pensiero non poteva certo comunicarlo a lei. 

Leggendo la lettera, egli alzò gli occhi su di lei, ma nel suo sguardo non c'era decisione alcuna. Ella capì subito che egli aveva già prima pensato qualcosa su questo dentro di sé. Ella sapeva che ora, qualunque cosa dicesse, non le avrebbe detto tutto quello che pensava. L'ultima speranza era delusa. E questo non se lo aspettava. 

- Tu vedi che uomo è - ella disse con voce tremante; - egli\ldots{} 

- Perdonami, ma io sono contento di questo - aggiunse Vronskij. - Grazie a Dio, lasciami finire di parlare - soggiunse, supplicandola con uno sguardo di dargli il tempo di spiegare le sue parole. - Sono contento perché questa faccenda non può, non può assolutamente rimanere così come egli suppone. 

- Perché non può? - prese a dire Anna, trattenendo le lacrime, evidentemente non dando ormai alcun valore a quello che egli avrebbe detto. Ella sentiva che il suo destino era deciso. 

Vronskij voleva dire che dopo il duello, inevitabile secondo lui, quello stato di cose non sarebbe potuto continuare, ma disse altro. 

- Non può continuare. Spero che adesso lo lascerai. Io spero - si confuse e arrossì - che mi permetterai di dare ordine e provvedere alla nostra vita. Domani\ldots{} - e voleva continuare. 

Ella non lo lasciò finire. 

\begin{itemize} \itemsep1pt\parskip0pt\parsep0pt \item E mio figlio? - gridò. - Vedi cosa scrive? Dovrei lasciarlo, ma io non voglio e non posso fare questo. \end{itemize} 

- Ma, in nome di Dio, cosa è meglio? Lasciare il figlio o continuare a vivere in questa situazione umiliante? 

- Umiliante per chi? 

- Per tutti, e più di tutti per te. 

- Tu dici, umiliante\ldots{} non lo dire. Queste parole non hanno senso per me - ella disse con voce che le tremava. Non voleva, ora, che egli le dicesse ciò che non sentiva. Le rimaneva solo l'amore di lui e voleva amarlo. - Tu capisci che dal giorno che ho cominciato ad amarti, tutto per me è cambiato. Per me non c'è che una sola cosa, il tuo amore. Se questo è mio, allora mi sento così in alto, così forte che nulla per me può essere umiliante. Sono orgogliosa del mio stato perché\ldots{} orgogliosa che\ldots{} orgogliosa\ldots{} - Non finì di pronunciare di che cosa fosse orgogliosa. Lacrime di vergogna e di disperazione soffocarono la sua voce. Tacque e scoppiò in singhiozzi. 

Anch'egli sentiva qualcosa venirgli su verso la gola e vellicargli il naso, e per la prima volta nella sua vita sentì che stava per piangere. Non avrebbe potuto dire che cosa proprio l'avesse commosso tanto; aveva pena di lei e sentiva che non poteva aiutarla, mentre egli era colpevole dell'infelicità sua, egli le aveva fatto del male. 

- Non è forse possibile il divorzio? - disse piano. Ella scosse il capo senza rispondere. - Non si può forse pretendere tuo figlio e lasciare lui? 

- Sì, ma tutto dipende da lui. Ora è da lui che devo andare - ella disse seccamente. Il suo presentimento che tutto sarebbe rimasto come prima non l'aveva ingannata. - Martedì sarò a Pietroburgo e si deciderà. 

- Sì - disse. - Ma non parliamo più di questo. 

La vettura che Anna aveva mandato via e che aveva fatto poi venire al cancello del giardino delle Vrede, si accostò. Ella salutò Vronskij e andò a casa. 

\capitolo{XXIII}Il lunedì c'era la solita seduta della commissione del 2 giugno. Aleksej Aleksandrovic entrò nell'aula della riunione, salutò, come al solito, i membri e il presidente, e sedette al suo posto, poggiando le mani sulle carte preparate davanti a lui. Fra queste carte c'erano anche le notizie necessarie e lo schema della proposta che aveva deciso di fare. Del resto non gli occorrevano neppure gli appunti. Ricordava tutto e non aveva bisogno di ripetersi mentalmente quello che avrebbe detto. Sapeva che, al momento opportuno, visto davanti a sé il viso dell'avversario che invano avrebbe cercato di darsi un'aria indifferente, il discorso sarebbe venuto fuori da sé, molto meglio che se lo avesse preparato adesso. Prevedeva che il contenuto del discorso sarebbe stato elevato, che ogni parola avrebbe avuto un significato. Frattanto, ascoltando la solita relazione, aveva l'aspetto più innocente, più inoffensivo. Nessuno avrebbe potuto sospettare, guardando le mani bianche dalle vene gonfie che palpavano così delicatamente con le dita lunghe le due estremità dei fogli di carta bianca posti davanti a lui, e quel capo chino da un lato con una impronta di stanchezza, che da un momento all'altro sarebbero usciti dalle sue labbra discorsi tali da scatenare una tempesta, da suscitare tra i commissari grida e reciproche interruzioni, tanto da costringere il presidente a richiamare all'ordine. Quando la relazione fu terminata, Aleksej Aleksandrovic con la sua voce calma, stridula, dichiarò ch'egli aveva da comunicare alcune sue considerazioni sulla questione della sistemazione degli allogeni. L'attenzione si rivolse a lui. Aleksej Aleksandrovic tossì e, senza guardare l'avversario, ma scelto, come sempre faceva nel pronunciare i suoi discorsi, il primo individuo che stava seduto dinanzi a lui - questa volta un vecchietto piccolo, tranquillo, che non aveva mai nessuna opinione - cominciò ad esporre le sue considerazioni. Quando si arrivò alla legge fondamentale e organica, l'avversario saltò su e cominciò a ribattere. Stremov, anche lui membro della commissione e anche lui colto nel vivo, cominciò a giustificarsi, e nell'insieme ne venne fuori una seduta tempestosa; ma Aleksej Aleksandrovic trionfò, e la sua proposta fu accolta; furono nominate tre nuove commissioni e il giorno dopo, in un certo ambiente di Pietroburgo, non si fece altro che parlare di questa seduta. Il successo di Aleksej Aleksandrovic fu persino maggiore di quello che egli si aspettava. 

La mattina dopo, martedì, Aleksej Aleksandrovic, svegliatosi, ricordò con soddisfazione la vittoria del giorno innanzi e non poté non sorridere, pur tentando di mostrarsi indifferente, quando il direttore della cancelleria, adulandolo, lo informò delle voci giunte fino a lui su quello ch'era accaduto in seno alla commissione. 

Intrattenendosi con il capo della cancelleria, Aleksej Aleksandrovic dimenticò completamente che quel giorno era martedì, giorno da lui fissato per l'arrivo di Anna Arkad'evna, e si meravigliò e dispiacque quando il servitore venne ad annunziarne l'arrivo. 

Anna era giunta a Pietroburgo la mattina presto, era stata mandata per lei la carrozza in seguito a un suo telegramma, perciò Aleksej Aleksandrovic doveva pur sapere del suo arrivo. Ma quando giunse, egli non le andò incontro. Le dissero che non era uscito dalla sua camera e che era occupato con il capo della cancelleria. Ella fece sapere al marito che era arrivata, andò nel proprio studiolo e si occupò di disfare le valigie, in attesa ch'egli venisse da lei. Ma passò un'ora, ed egli non si fece vivo. Ella uscì in sala da pranzo col pretesto di dare un ordine e parlò a voce alta proprio perché egli udisse e la raggiungesse là; ma non comparve sebbene ella si fosse accorta che era venuto fin sulla porta dello studio ad accompagnare il capo della cancelleria. Ella sapeva che, come al solito, sarebbe andato via presto per recarsi in ufficio, e desiderava vederlo prima per definire i loro rapporti. 

Fece un giro per la sala e si diresse decisamente verso lo studio. Quando vi entrò, egli, in uniforme d'ufficio, evidentemente pronto per andar via, era seduto accanto a un tavolino sul quale aveva poggiato i gomiti, e guardava tristemente davanti a sé. Ella lo vide prima che lui la scorgesse e capì che pensava a lei. 

Vistala, egli volle alzarsi, cambiò idea, ma il suo viso s'infiammò, cosa del tutto nuova per Anna, e in fretta s'alzò dirigendosi verso di lei e guardandola non negli occhi ma più in alto, sulla fronte e sull'acconciatura. Le si avvicinò, le prese la mano e la pregò di sedersi. 

- Sono molto contento che siate venuta - disse, sedendosi accanto a lei e, desiderando evidentemente di dire qualcosa, esitò. Parecchie volte egli fece per parlare, ma tacque. Sebbene nel prepararsi a quell'incontro ella avesse imparato a disprezzarlo e ad accusarlo, non sapeva cosa dirgli e aveva pietà di lui. E così il silenzio durò abbastanza a lungo. 

- Serëza sta bene? - egli disse e, senza aspettar risposta, soggiunse: - oggi non pranzerò a casa e ora devo andar via. 

- Io volevo andare a Mosca - ella disse. 

- No, avete fatto molto, molto bene a venire - egli disse e di nuovo tacque. 

Vedendo che egli non aveva la forza di cominciare a parlare, cominciò lei stessa. 

- Aleksej Aleksandrovic - disse guardandolo e senza abbassare gli occhi sotto lo sguardo di lui fisso sulla pettinatura - io sono una donna colpevole, sono una donna cattiva, ma sono la stessa che allora vi ha parlato, e sono venuta a dirvi che non posso cambiare in nulla. 

- Non vi ho detto questo - egli disse deciso e guardandola con odio diritto negli occhi - e questo proprio mi aspettavo. - Nell'impeto d'ira, era ritornato di nuovo padrone di tutte le sue facoltà. - Ma come vi ho detto allora e come vi ho scritto - prese a dire con voce tagliente, stridula - ora vi ripeto che io non sono obbligato a sapere questo. Io lo ignoro. Non tutte le mogli sono come voi generose tanto da affrettarsi a comunicare una notizia così piacevole ai mariti. - S'indugiò in modo particolare sulla parola ``piacevole''. - Io ignoro tutto ciò finché il mondo lo ignora, finché il mio nome non è svergognato. E perciò vi dico soltanto che i nostri rapporti devono essere quali sono sempre stati e che solo in caso che vi compromettiate, io sarò costretto a prendere delle misure per difendere il mio onore. 

- Ma i nostri rapporti non possono essere quelli di prima - disse Anna con voce timida, guardandolo con spavento. 

Nel vedere di nuovo quei gesti calmi, nel sentire quella voce penetrante, infantile e canzonatoria, la repulsione ch'ella sentiva per lui fece svanire quel sentimento di pietà che poco prima aveva sentito, e ora aveva soltanto paura; ma voleva, a ogni costo, chiarire la sua situazione. 

- Io non posso essere vostra moglie, quando\ldots{} - stava per cominciare. 

Egli si mise a ridere d'un riso cattivo. 

- Si vede che il genere di vita che avete scelto, si è riflesso sulle vostre idee. Per quel tanto che io rispetto e disprezzo e questo e quello\ldots{} rispetto il vostro passato, ma disprezzo il presente\ldots{} ero ben lontano dalla interpretazione che voi avete dato alle mie parole. 

Anna sospirò e chinò il capo. 

- D'altra parte non capisco come, avendo voi tanta spregiudicatezza - continuò, riscaldandosi - da annunziare a vostro marito la vostra infedeltà, senza trovare, a quanto sembra, nulla di biasimevole in questo, stimiate ora riprovevole l'adempimento dei doveri di moglie nei riguardi del marito. 

- Aleksej Aleksandrovic che cosa mai vi occorre da me? 

- Mi occorre non incontrare qui quell'uomo e che vi comportiate in modo che né il mondo né la servitù possano accusarvi\ldots{} che non lo vediate. Mi pare che non sia molto. E in compenso di questo godrete dei diritti di una moglie onesta, senza adempierne i doveri. Ecco tutto quello che ho da dirvi. Ora devo andar via. Non pranzo a casa. 

Si alzò e si diresse verso la porta. Anche Anna si alzò. Egli, inchinandosi senza proferire parola, si fece precedere da lei. 

\capitolo{XXIV}La notte che Levin trascorse sulla bica di fieno non passò invano per lui. L'azienda agricola che conduceva gli era divenuta d'un tratto odiosa, ed era divenuta priva di qualsiasi interesse per lui. Malgrado l'ottimo raccolto, non vi erano mai stati, o almeno mai gli era parso che ci fossero stati, tanto insuccesso e tanta ostilità tra lui e i contadini, come in quell'anno, e la causa di questo insuccesso e di questa ostilità gli si rivelava ora, in piena luce. Il fascino che aveva esercitato per lui lo stesso lavoro dei contadini, il contatto più intimo che, per questo, aveva avuto con loro, il senso di invidia che aveva provato per loro, per la loro vita, il desiderio di viverla, quella stessa loro vita, che, in quella notte, non era stato più un sogno, ma un proposito, di cui aveva riflettuto i particolari, tutto questo aveva mutato talmente la sua opinione circa l'azienda da lui condotta, che non poteva in alcun modo ritrovare, ora, in essa l'interesse di prima, e non poteva non vedere chiara la ragione dei suoi rapporti spiacevoli con i contadini, rapporti che costituivano la base della questione. Armenti di vacche bellissime, belle come la Pava, tutta la terra concimata, e rivoltata con gli aratri, nove campi eguali circondati da giunchi, novanta desjatiny di concio rivoltato in profondità, i seminativi in fila e via di seguito, tutto questo sarebbe stato bellissimo se fosse stato fatto da lui stesso e dai collaboratori, da uomini che simpatizzassero con lui. Ma egli ora vedeva chiaramente (il suo studio per un volume di economia rurale nel quale era proclamato come elemento essenziale in tale economia l'elemento lavoratore, lo aveva molto aiutato in questo) che l'azienda che egli conduceva rappresentava soltanto una crudele e ostinata lotta tra lui e i lavoratori, nella quale da una parte, la sua, c'era un'incessante, intensa aspirazione a rifare tutto su di un modello ritenuto il migliore, dall'altra, invece, c'era l'ordine naturale delle cose. E scorgeva che in questa lotta, la massima tensione di forze da parte sua e la mancanza di ogni sforzo e perfino di ogni proponimento dall'altra, pervenivano soltanto alla conseguenza che l'azienda non andava avanti e che si sciupavano attrezzi bellissimi, bestiame superbo, e terra. La cosa preminente era che non solo andava perduta del tutto la propria energia, ma che egli non poteva non sentire, ora che il senso della sua azienda gli si era rivelato, che lo scopo di questa energia fosse il meno degno. In sostanza, in che consisteva la lotta? Egli teneva dietro a ogni suo soldo (e non poteva non tenerci dietro perché gli bastava allentare per poco la sorveglianza per non avere denaro sufficiente per pagare i lavoratori); essi invece pensavano solo a lavorare tranquillamente e piacevolmente secondo la loro abitudine. Egli aveva interesse a che ogni lavoratore rendesse quanto più possibile, che non si distraesse, che badasse a non rompere i vagli, che riflettesse a quello che faceva; il lavoratore, invece, aveva interesse a lavorare nel modo più piacevole possibile, con respiro, e soprattutto senza preoccupazione, lasciandosi andare, senza pensare. E proprio in quell'estate Levin aveva constatato ciò ad ogni passo. Aveva mandato a falciare del trifoglio per seminarvi il fieno, scegliendo le desjatiny di terra meno buona, dove erano cresciute le erbe e l'artemisia, che non servivano per sementa, e gli avevano falciato le migliori desjatiny da semi, asserendo per giustificarsi che così aveva detto l'amministratore, e lo consolavano dicendo che il fieno sarebbe stato ottimo; ma egli sapeva che avevan fatto così perché quelle desjatiny di terra si falciavano con minor fatica. Aveva mandato un'essiccatrice a ventilare il fieno e l'avevano rotta ai primi giri, perché il contadino s'era annoiato di starci su seduto a cassetta sotto le ali che si agitavano. E gli dicevano: ``Degnatevi di non inquietarvi; le donne sparnazzeranno alla svelta''. Gli aratri s'erano mostrati inadatti, perché al lavoratore non entrava in mente di dover abbassare il dentale, e, girando con forza, l'aratro tormentava i cavalli e sciupava il terreno: e lo pregavano di non inquietarsene. I cavalli li avevan lasciati pascolare nel frumento perché non uno solo dei lavoratori voleva fare da guardiano notturno e, dato l'ordine di farlo senz'altro, avevan fatto a turno la guardia di notte e Van'ka, dopo aver lavorato tutto il giorno, si era addormentato, e aveva confessato il suo peccato, dicendo: ``Come volete voi''. Avevano fatto crepare le tre migliori vacche perché le avevano lasciate andare a pascolare là dove non c'era abbeveratoio su per il guaime del trifoglio, e non avevano voluto credere in nessun modo che si erano gonfiate col trifoglio, e raccontavano, per consolarsi, che al vicino erano morti centoventi capi di bestiame. Tutto questo lo facevano, non perché qualcuno di loro volesse male a Levin o alla sua azienda, al contrario, egli sapeva che gli volevano bene, lo consideravano un signore alla mano (che è la lode più alta); ma lo facevano solo perché volevano lavorare allegramente e senza affanno; e gl'interessi suoi erano non solo estranei e incomprensibili a loro, ma fatalmente opposti ai loro più legittimi interessi. Già da tempo Levin si sentiva insoddisfatto del suo modo di condurre l'azienda. Vedeva che la barca faceva acqua, ma non trovava e non cercava neppure la falla, ingannando di proposito se stesso. Ma ormai non poteva ingannarsi più. Quell'azienda che egli conduceva gli era divenuta non solo priva di interesse, ma odiosa, e non poteva occuparsene più. A questo si aggiungeva anche la presenza a trenta verste da lui di Kitty Šcerbackaja che egli voleva e non poteva vedere. Dar'ja Aleksandrovna Oblonskaja, quando egli era stato da lei, l'aveva invitato a tornare: andare per rinnovare la proposta di matrimonio a sua sorella che ora, da quanto gli si faceva capire, l'avrebbe accolta? Levin, rivedendo Kitty Šcerbackaja, aveva capito che non aveva cessato di amarla; ma egli non poteva andare dagli Oblonskij, sapendo che era là. Il fatto che la sua domanda di matrimonio era stata respinta, poneva fra lui e lei una barriera insormontabile. ``Io non posso chiederle di essere mia moglie solo perché ella non può essere la moglie di colui che desiderava'' diceva fra sé. Il pensiero di questo lo rendeva freddo e ostile. ``Non avrò la forza di parlare con lei senza doverle rimproverare qualche cosa, di guardarla senza rancore: ed ella mi odierà ancora di più, come del resto è prevedibile. E poi, come posso io, ora, dopo tutto quello che mi ha detto Dar'ja Aleksandrovna, andare da loro? Posso forse fingere di non sapere quello che mi ha detto? E andrei io, pieno di generosità, a perdonarla, a farle grazia? Io dinanzi a lei nella parte di chi la perdona e la degna del proprio amore! Perché mai Dar'ja Aleksandrovna mi ha parlato di questo? Avrei potuto vederla per caso, e allora tutto si sarebbe svolto da sé, ma ora è impossibile!''. 

Dar'ja Aleksandrovna gli mandò un biglietto, chiedendogli una sella da signora per Kitty. ``Mi hanno detto che avete una sella - gli scriveva. - Spero che la porterete voi stesso''. 

Questa cosa non la poteva proprio sopportare. Come mai una donna intelligente e delicata poteva umiliare a tal punto la sorella? Scrisse dieci biglietti e li strappò uno dopo l'altro, e mandò la sella senza rispondere. Scrivere che sarebbe andato, non poteva, perché non poteva andare; scrivere che non poteva andare perché qualcosa glielo impediva o perché partiva, era ancora peggio. Mandò la sella senza la risposta, e il giorno dopo, con la coscienza di aver compiuto qualcosa di vergognoso, affidata l'azienda divenutagli odiosa all'amministratore, partì per un lontano distretto dove c'erano bellissime paludi da beccacce, ospite del suo amico Svijazskij che da poco gli aveva scritto, pregandolo di attuare l'antico progetto di recarsi un po' da lui. Le paludi da beccacce nel distretto di Surov tentavano già da tempo Levin, ma per gli affari dell'azienda aveva sempre rinviato questo viaggio. Ora invece era contento di allontanarsi dagli Šcerbackij e, soprattutto, dall'azienda, e di andare a caccia, cosa che, in tutte le sue amarezze, era sempre stata per lui la migliore delle consolazioni. 

\capitolo{XXV}Per il distretto di Surov non c'era strada ferrata né diligenza, e Levin andò coi cavalli suoi, in un tarantas. 

A mezza strada si fermò a mangiare da un ricco contadino. Il vecchio calvo, arzillo, con una barba rossiccia, canuta sulle guance, aprì il portone, serrandosi contro lo stipite, per lasciar passare la trojka. Mostrato al cocchiere il posto sotto la tettoia nel cortile vasto, nuovo, pulito e ben curato, dove c'eran degli aratri bruciacchiati, il vecchio invitò Levin a entrare nella stanza. Una giovane donna pulitamente vestita, con gli zoccoli ai piedi scalzi, strofinava curva il pavimento di un ingresso nuovo. Ella si spaventò del cane che era corso dietro a Levin e dette un grido, ma subito rise del proprio spavento, accortasi che il cane non l'avrebbe toccata. Mostrata a Levin col braccio dalla manica rimboccata la porta della stanza, nascose di nuovo, curvandosi, il suo bel viso e continuò a lavorare. 

- Il samovar, eh? - chiese. 

- Sì, per favore. 

La stanza era grande, con una stufa olandese e un'intelaiatura. Sotto le icone c'erano una tavola pitturata a disegni, una panca e due sedie. All'entrata un armadietto con le stoviglie. Le imposte erano chiuse, c'erano poche mosche e tutto era così pulito che Levin si preoccupò che Laska, avendo corso per via ed essendosi bagnata nelle pozzanghere, non avesse a sporcare il pavimento, e le indicò un posto in un angolo accanto alla porta. Dopo aver guardato la stanza, Levin uscì nel cortile dietro la casa. La giovane donna, bella a vedersi, con gli zoccoli e i secchi vuoti che faceva oscillare sulla stanga, corse davanti a lui a prendere acqua dal pozzo. 

- Fa' presto - gridò allegramente dietro di lei il vecchio, e si accostò a Levin. - Ebbene, signore, andate da Nikolaj Ivanovic Svijazskij? Anche lui si ferma da noi - cominciò ciarliero, appoggiandosi coi gomiti alla balaustra della scala. 

Mentre il vecchio raccontava della sua conoscenza con Svijazskij, il portone cigolò ed entrarono i lavoratori che tornavan dal campo con gli aratri e gli erpici. I cavalli attaccati agli aratri ed agli erpici erano ben pasciuti e grandi. I lavoratori evidentemente erano gente di casa: due erano giovani e avevano le camicie d'indiana e i berretti; gli altri due, uno vecchio e l'altro giovane, erano a opra, e avevano le camicie di canapa. Allontanatosi dall'ingresso, il vecchio si avvicinò ai cavalli e prese a staccarli. 

- Che cosa hanno arato? - chiese Levin. 

- Hanno arato in giro in giro per le patate. Anche noi teniamo un pezzetto di terra. Tu, Fedot, non lasciare andare il castrato, mettilo invece vicino al trogolo, ne attaccheremo un altro. 

- Ohi, babbo, quei vomeri che avevo ordinato di prendere, li ha portati sì o no? - chiese il giovane, robusto di statura, evidentemente figlio del vecchio. 

- Nel\ldots{} nella slitta - rispose il vecchio, avvolgendo ad anello le redini abbandonate e gettate a terra. - Metti in ordine intanto che mangiano. 

La giovane donna, bella a vedersi, con le brocche piene che le facevano tendere le spalle, attraversò l'ingresso. Apparvero da qualche parte altre donne, alcune giovani, belle, di mezza età ed altre vecchie, brutte, con bambini e senza bambini. 

Il samovar cominciò a brontolare nel tubo; gli operai e quelli di casa, posti in stalla i cavalli, andarono a mangiare. Levin tirò fuori dalla carrozza le sue provviste e invitò il vecchio a bere il tè. 

- L'ho già bevuto oggi - disse il vecchio, accettando con evidente piacere la proposta. - Ma se è per farvi compagnia\ldots{} 

Prendendo il tè, Levin seppe tutta la storia dell'azienda del vecchio. Il vecchio aveva affittato dieci anni addietro centoventi desjatiny da una proprietaria, e l'anno precedente se le era comprate e ne aveva prese in affitto altre trecento da un proprietario vicino. Una piccola parte del terreno, la peggiore, l'aveva data in affitto, ma quaranta desjatiny nel campo le arava lui con la famiglia e con due opre a giornata. Il vecchio si lamentava che gli affari andavano male. Ma Levin capiva che egli si lamentava solo per abitudine, e che l'azienda andava bene. Se le cose fossero andate male, egli non avrebbe comprato a centocinque rubli, non avrebbe dato moglie a tre suoi figliuoli e a un nipote, non avrebbe ricostruito due volte dopo gli incendi e tutto non sarebbe andato sempre di bene in meglio. Malgrado le lamentele del vecchio si vedeva che era giustamente orgoglioso del proprio benessere, orgoglioso dei figli, dei nipoti, delle nuore, dei cavalli, delle mucche e, in particolare, del fatto che egli mandava avanti tutta quella azienda. Dalla conversazione col vecchio, Levin capì ch'egli non era alieno dalle innovazioni. Seminava molte patate, e le patate, che Levin aveva visto mentre si avvicinava alla fattoria, sfiorivano già e cominciavano a germogliare, quando da lui cominciavano appena a spuntare. Egli arava sotto le patate con ``l'aratra'', come egli chiamava l'aratro preso in prestito dal proprietario. Seminava il frumento. Il particolare che, sarchiando la segala, il vecchio dava da mangiare ai cavalli la segala sarchiata, colpì molto Levin. Quante volte, vedendo questo ottimo mangime andar perduto, voleva che si raccogliesse!, ma ciò risultava sempre impossibile. Il contadino invece lo utilizzava e non si stancava di far le lodi di un mangime simile. 

- E le donnette allora che devono fare? Portano i mucchi sulla strada e il carro si avvicina. 

- Ed ecco, invece, da noi proprietari tutto va male coi lavoratori - disse Levin, dandogli un bicchiere col tè. 

- Grazie - rispose il vecchio; prese il bicchiere, ma rifiutò lo zucchero, indicando la pallottolina ch'era restata, rosicchiata da lui. - E come condurre l'azienda con gli operai? - egli disse. - Una rovina. Ecco, prendiamo, sia pure Svijazskij. Noi sappiamo che la terra è la sua: una bellezza; pure anche lui non ha a lodarsi del raccolto. Tutta incuria! 

- Ma, dimmi, tu fai andare avanti l'azienda con gli operai? 

- Ma questo è affar nostro di contadini. Possiamo arrivare a far tutto da soli. Se un operaio non rende, va via: ce la facciamo anche da soli. 

- Babbo, Finogen ha detto di procurargli del catrame - disse, entrando, la donna con gli zoccoli. 

- Proprio così, signore! - disse il vecchio, alzandosi, si fece il segno della croce lentamente, ringraziò Levin e uscì. 

Quando Levin entrò nella capanna da lavoro per chiamare il cocchiere, vide tutti gli uomini della famiglia a tavola. Le donne in piedi servivano. Un giovane e robusto figliuolo, con la bocca piena di zuppa, raccontava qualcosa di buffo e tutti ridevano, e in particolare la donna con gli zoccoli che scodellava la minestra di cavolo nella tazza. 

Può darsi benissimo che il bel viso della donna con gli zoccoli avesse contribuito molto a dare a Levin l'impressione di buona amministrazione in quella casa di contadini; ma quest'impressione fu così profonda che Levin non poté in nessun modo distoglierne il pensiero. E per tutta la strada, dalla casa del vecchio a quella di Svijazskij, vi pensò continuamente, come se l'impressione che quell'azienda gli aveva fatto esigesse da lui una particolare attenzione. 

\capitolo{XXVI}Svijazskij era maresciallo della nobiltà nel suo distretto. Aveva cinque anni più di Levin ed era ammogliato da molto tempo. Nella sua casa viveva una giovane cognata, che era molto simpatica a Levin. E Levin sapeva che Svijazskij e sua moglie desideravano molto dargli in moglie questa ragazza. Lo sapeva con certezza, come lo sanno sempre i cosiddetti pretendenti, tuttavia non lo avrebbe mai detto a nessuno, e sapeva pure che, nonostante volesse ammogliarsi, nonostante che quella ragazza molto attraente sarebbe dovuta essere, secondo le informazioni, un'ottima moglie, tuttavia gli sembrava tanto impossibile, anche se non fosse stato innamorato di Kitty, di sposare lei, quanto volare in cielo. E questa consapevolezza gli avvelenava il piacere che sperava di ricavare dalla visita a Svijazskij. 

Ricevuta la lettera di Svijazskij con l'invito per la caccia, Levin aveva pensato a questa circostanza, e aveva finito per giudicare un'infondata supposizione le mire che attribuiva a Svijazskij e così, malgrado tutto, decise di andare. Inoltre, in fondo all'animo, voleva provarsi, voleva misurarsi ancora una volta con questa ragazza. La vita domestica degli Svijazskij era straordinariamente piacevole, e lo stesso Svijazskij era il miglior tipo di amministratore pubblico della provincia e, appena l'aveva conosciuto, aveva suscitato in Levin un grande interesse. 

Svijazskij era una di quelle persone che sempre s'imponevano all'attenzione di Levin: persone, il cui modo di ragionare molto coerente, anche se non originale, fila diritto per conto proprio, e la cui vita precisamente definita e salda nelle direttive, scorre poi da sé, in modo del tutto autonomo e quasi sempre in senso opposto al ragionamento. Svijazskij era un uomo di idee più che mai liberali e disprezzava la nobiltà perché riteneva che in maggioranza i nobili fossero segreti fautori della servitù della gleba e restassero in ombra per vigliaccheria. Considerava la Russia un paese rovinato, tipo Turchia, e stimava il governo russo spregevole tanto da non degnarsi, lui, neppure di criticarne seriamente gli atti, e intanto prestava servizio per quel governo, ed era un perfetto maresciallo della nobiltà e per istrada portava sempre il berretto con la coccarda e l'orlo rosso. Riteneva che una vita degna di un uomo si potesse condurla soltanto all'estero, dove andava non appena se ne presentasse l'occasione, e intanto in Russia conduceva con vivissimo interesse un'azienda molto vasta e perfezionata, e seguiva e conosceva tutto quello che avveniva in Russia. Considerava il contadino russo qualcosa di intermedio fra la scimmia e l'uomo, e nello stesso tempo, alle elezioni provinciali, era lui che più volentieri di tutti stringeva la mano ai contadini e ascoltava le loro opinioni. Non credeva a nulla, né al diavolo, né all'acqua santa, ma si preoccupava molto del problema del miglioramento delle condizioni del clero e della riduzione delle parrocchie, e si affannava inoltre perché la chiesa del suo villaggio non fosse soppressa. 

Nella questione femminile era dalla parte degli estremi fautori della completa libertà della donna e particolarmente del suo diritto al lavoro, ma viveva con la moglie in modo che tutti ammiravano la loro affettuosa vita familiare priva di figli, e organizzava la vita di sua moglie in modo ch'ella non dovesse e non potesse far nulla, oltre che occuparsi, in compagnia del marito, del come passar meglio e il più allegramente possibile il tempo. 

Se Levin non avesse posseduto la speciale facoltà di dar peso alla parte migliore delle persone, il carattere di Svijazskij non avrebbe presentato per lui nessuna difficoltà e nessun problema; si sarebbe detto: ``è uno sciocco o un poco di buono'' e tutto sarebbe stato chiaro. Ma egli non poteva dire che fosse stupido perché Svijazskij era senza dubbio non solo un uomo molto intelligente, ma anche molto colto e portava il suo bagaglio culturale con non comune semplicità. Non c'era materia ch'egli non conoscesse; ma delle sue cognizioni dava prova solo quando vi era costretto. Ancora meno Levin poteva dire che fosse un poco di buono, perché indubbiamente Svijazskij era un uomo onesto, buono, intelligente, che, allegramente, con vivacità e senza posa, faceva un lavoro molto apprezzato da tutti quelli che lo circondavano, e certo poi non faceva e non avrebbe saputo fare nulla di male con intenzione. 

Levin cercava di capire, e non capiva: e guardava sempre a lui e alla sua vita come a un enigma vivente. 

Erano amici e perciò Levin si permetteva di insistere con Svijazskij pel desiderio di attingere il fondo della sua visione della vita; ma non gli riusciva mai. Ogni volta che Levin tentava di penetrare più in là delle stanze di ricevimento della mente di Svijazskij, aperte a tutti, notava che Svijazskij si turbava un poco; un'ansietà appena percettibile appariva nel suo sguardo, come se temesse che Levin potesse capirlo, e gli opponeva perciò una benevola e allegra resistenza. 

Ora, dopo la sua delusione per l'azienda, era per Levin oltremodo piacevole starsene un po' da Svijazskij. A parte la vista di quei colombi felici, contenti di loro stessi e di tutti, del loro nido ben ordinato, piaceva ora a Levin, sentendosi così scontento della vita, veder raggiunto in Svijazskij quel segreto che gli dava tanta chiarezza, precisione e allegria nella vita. Oltre a ciò, Levin sapeva che avrebbe incontrato dagli Svijazskij alcuni proprietari vicini e in questo momento lo interessava ascoltar quei tali discorsi sull'azienda, sul raccolto, sull'assunzione degli operai e via di seguito; discorsi che Levin era solito considerare come qualcosa di molto utile, ma che adesso gli sembravano i soli importanti. ``Questo, forse, non era importante ai tempi della servitù della gleba, o non è più importante in Inghilterra. In tutti e due i casi c'erano e ci sono delle condizioni quanto mai definite; ma da noi, poiché ora tutto ciò è stato messo sottosopra e si va appena appena assestando, la questione della sistemazione delle nuove condizioni è l'unica per la Russia'' pensava Levin. 

La caccia fu molto meno fortunata di quello che s'aspettava Levin. La palude era senz'acqua, e non c'erano beccacce. Camminò una giornata intera e ne ammazzò appena tre, ma in compenso riportò, come sempre dalla caccia, un appetito eccellente, un'eccellente disposizione d'animo e quel certo risveglio intellettuale che in lui si accompagnava sempre al moto fisico. E a caccia, quando gli pareva di non pensare a nulla, che è che non è, di nuovo gli tornava in mente il vecchio con la sua famiglia, e la viva impressione che gli era rimasta sembrava esigesse attenzione non solo per se stessa, ma anche perché gli pareva si collegasse alla soluzione di un qualche cosa. 
\enlargethispage{\baselineskip}

La sera, al tè, la conversazione con i due proprietari che erano giunti per certi affari di tutele, si svolse interessante così come Levin si era ripromesso. 

Levin sedeva accanto alla padrona di casa presso la tavola da tè e doveva tener viva la conversazione con lei e con la cognata che era seduta di fronte a lui. La padrona era una donna dal viso tondo, bionda e non alta, tutta splendente di fossette e sorrisi. Levin cercava attraverso lei di giungere alla soluzione dell'enigma per lui interessante del marito; ma non aveva piena libertà di pensiero, perché si sentiva tormentosamente a disagio. Si sentiva tormentosamente a disagio, perché davanti a lui sedeva la cognata con un vestito che gli pareva indossato apposta per lui, con una scollatura speciale a trapezio sul petto bianco; questo scollo quadrangolare, malgrado il petto fosse molto bianco, o proprio perché questo era molto bianco, toglieva a Levin la libertà di pensiero. Egli immaginava, probabilmente ingannandosi, che quello scollo era stato scelto proprio per lui, e non si riteneva in diritto di guardarlo e cercava di non guardarlo; ma sentiva di essere colpevole solo per il fatto che lo scollo era stato fatto. A Levin pareva di ingannare qualcuno, di dover chiarire qualche cosa, che però non si poteva in nessun modo chiarire, perciò arrossiva continuamente, era inquieto e a disagio. Il suo disagio si comunicava anche alla graziosa cognata. Ma la padrona pareva non avvedersene e faceva in modo che la ragazza partecipasse alla conversazione. 

- Voi dite - continuava la padrona - che tutto quello che è russo non può interessare mio marito. Al contrario, egli vive, sì, felice all'estero, ma non mai come qua. Qua egli si sente nel suo ambiente. Ha tanto da fare, e ha il dono di interessarsi a tutti. Ah, non siete stato nella nostra scuola? 

- Ho visto\ldots{} È una casetta circondata di edera? 

- Sì, è l'occupazione di Nast'ja - disse, indicando la sorella. 

- Insegnate proprio voi? - chiese Levin, cercando di guardare al di là dello scollo, sentendo però che, guardando da quella parte, dovunque fissasse gli occhi, avrebbe sempre visto quello scollo. 

- Sì, proprio io vi ho insegnato, e vi insegno, ma abbiamo un'ottima maestra. Anche la ginnastica abbiamo introdotto. 

- No, vi ringrazio, non voglio più tè - disse Levin e, pur sapendo di commettere una scortesia, non avendo più la forza di continuare la conversazione, si alzò, arrossendo. - Sento che là si parla di un argomento molto interessante - soggiunse e si avvicinò all'altro estremo della tavola dove sedeva il padrone di casa con i due proprietari. Svijazskij sedeva di fianco alla tavola, girando una tazza nella mano poggiata sul gomito, e con l'altra raccogliendo nel pugno la barba ch'egli avvicinava al naso e lasciava poi cader giù come se l'annusasse. Con gli occhi neri lucenti guardava fisso il proprietario dai baffi grigi che si animava, ed evidentemente quei discorsi lo divertivano. Il proprietario si lamentava dei contadini. Per Levin era ovvio che Svijazskij fosse in grado di dare una risposta alle lamentele del proprietario sì da annientare d'un tratto la sostanza di quei discorsi, ma che per la sua posizione non potesse darla, questa risposta, e che perciò ascoltasse, non senza compiacimento, il comico discorso del proprietario. 

Il proprietario dai baffi grigi, evidentemente, era un fautore convinto della servitù della gleba, un vecchio abitante di campagna, appassionato proprietario di terre. Questi segni Levin li scorgeva nell'abito, un soprabito fuori moda, frusto e decisamente non adatto per un proprietario, nei suoi occhi intelligenti, accigliati, nella parlata armoniosa, nel tono di comando acquisito ormai per lunga abitudine, e nei gesti risoluti delle mani grandi, belle, abbronzate, adorne solo del vecchio anello nuziale all'anulare. 

\capitolo{XXVII}-Se non ci dispiacesse lasciare quel che s'è avviato\ldots{} fatica se n'è fatta tanta\ldots{} butterei tutto all'aria, venderei, me ne andrei, come Nikolaj Ivanyc, a sentire La bella Elena - disse il proprietario con un sorriso cordiale che illuminò il suo vecchio viso intelligente. 

- Già, ecco, eppure non lo lasciate - disse Nikolaj Ivanovic Svijazskij - dunque c'è il tornaconto. 

- L'unico vantaggio è che vivo a casa mia, non è roba comprata, né presa in affitto. Già: e poi speri sempre che il popolo rinsavisca! Ma lo credereste? questa è un'ubriacatura, uno sbandamento. Si son divisi tutto, non c'è più una cavalla, né una vaccarella. Crepan di fame, ma intanto, prendete un operaio a giornata! Cercherà l'occasione per rovinarvi e ancora di mandarvi innanzi al giudice di pace. 

- In compenso anche voi lo denuncerete al giudice di pace - disse Svijazskij. 

- Lo denuncerò? Ma per nulla al mondo! Cominceranno tali discussioni che non ci sarà da stare allegri con la denuncia! Ecco: alla fabbrica hanno preso la caparra e sono andati via. Che ha fatto il giudice di pace? Li ha assolti. Tutto dovrebbe essere tenuto su dal tribunale del distretto e dall'anziano. Lui sì che li bastona all'uso antico! E se non ci fosse questo\ldots{} lascia via tutto! Fuggi in capo al mondo! 

Evidentemente il proprietario stuzzicava Svijazskij, ma Svijazskij non solo non si arrabbiava, ma, si vedeva, ci si divertiva. 

- Sì, ecco, noi conduciamo la nostra azienda senza ricorrere a codeste misure - egli disse sorridendo - io, Levin, il signore. 

E indicò l'altro proprietario. 

- Già, va' da Michail Petrovic, a chiedergli come fa. È forse razionale la sua azienda domestica? - disse il proprietario, facendo sfoggio evidente della parola ``razionale''. 

- Io ho un'azienda modesta - disse Michail Petrovic. - Ringrazio Iddio. Il mio modo di condurla consiste tutto nel far che siano pronti i soldi per le tasse d'autunno. Vengono i contadini: ``padrone, salvaci tu''. Ebbene, tutti i miei vicini son contadini, ti fan pena. E via, dài per il primo trimestre e di' soltanto: ``ragazzi, ricordatevi, vi ho aiutato, date anche voi una mano quando ci sarà bisogno: la semina dell'avena, la raccolta del fieno, la mietitura'': e così si stabilisce a tanto per tassa. Ci son quelli senza coscienza anche fra loro, è vero. 

Levin, che conosceva da tempo questi sistemi patriarcali, scambiò uno sguardo con Svijazskij e interruppe Michail Petrovic, rivolgendosi di nuovo al proprietario dai baffi grigi. 
\enlargethispage*{1\baselineskip}

- Allora, voi come la pensate? - chiese. - Come bisogna condurre, ora, un'azienda? 

- Sì, condurla al metodo di Michail Petrovic o darla a mezzadria o in fitto ai contadini si può; ma è proprio così che si distrugge la ricchezza generale della nazione. Da me, dove la terra rendeva nove col lavoro dei servi della gleba e una buona amministrazione, a mezzadria rende tre. L'emancipazione ha rovinato la Russia. 

Svijazskij guardò Levin con occhi ridenti e gli fece persino un segno canzonatorio appena percettibile, ma Levin non trovava ridicole le parole del proprietario; egli ne capiva il valore più di quanto non lo capisse Svijazskij. E molto di quello che disse poi il proprietario, per dimostrare perché la Russia era rovinata dall'emancipazione, gli parve perfino molto vero per lui, nuovo e incontestabile. Il proprietario, evidentemente, esponeva un'idea sua propria, il che accade di rado, e un'idea che era sorta in lui non come effetto del desiderio di occupare con qualche cosa il cervello ozioso, ma un'idea che era venuta fuori dalle contingenze stesse della vita, che egli aveva elaborato nella solitudine della campagna e che aveva esaminato da ogni lato. 

- La questione, permettetemi di osservare, è che ogni progresso lo si attua solo d'autorità - egli disse evidentemente per mostrare che anche lui non era estraneo alla cultura. - Prendete le riforme di Pietro, di Caterina, di Alessandro. Prendete la storia d'Europa. Tanto più per quanto riguarda il progresso della vita rurale. Anche la patata, quella pure è stata introdotta da noi d'autorità. Certamente c'è stato un tempo in cui non si conosceva nemmeno l'aratro primitivo. Anche questo l'hanno introdotto forse solo al tempo degli appannaggi, e certamente d'autorità. Al tempo nostro, quando c'era la servitù della gleba, noi proprietari conducevamo l'azienda attuando dei perfezionamenti; gli essiccatoi, i vagli, le concimaie, e tutti gli altri strumenti, tutto introducevamo d'autorità; e i contadini dapprincipio si opponevano, dopo ci imitavano. Ora, con l'abolizione della servitù, ci hanno tolto l'autorità, e le nostre aziende anche se già portate a un livello più alto, devono discendere a un livello più barbaro e primitivo. È così che io la intendo. 

- Ma perché mai? Se l'azienda è razionale la potete condurre con l'affitto - disse Svijazskij. 

- Ma se non c'è autorità! Con che mai la posso condurre? permettetemi di chiedere. 

``Eccola, la forza lavoratrice, l'elemento principale dell'economia'' pensò Levin. 

- Con i lavoratori. 

- Già, ma i lavoratori non vogliono lavorare bene e con strumenti buoni. Il nostro lavoratore fa solo una cosa: s'ubriaca come un porco, e guasta tutto quello che gli date. Abbevera i cavalli così da farli scoppiare, una bardatura buona la rompe, una ruota cerchiata ve la cambia e se la beve; nella macchina per la battitura ci getta un perno, per spezzarla. Tutto quello che non è fatto da lui lo disgusta. Proprio per questo si è abbassato tutto il livello dell'azienda rurale. Le terre sono abbandonate, sono coperte di assenzio o sono distribuite ai contadini; e là, dove ne producevano un milione, producono qualche centinaia di migliaia di stai di grano; in genere la ricchezza è diminuita. Se avessero fatto lo stesso, ma con misura! 

E cominciò a svolgere il suo piano di emancipazione secondo il quale questi inconvenienti sarebbero stati eliminati. 

A Levin non interessava questo piano; ma quando egli finì, Levin tornò alla sua prima tesi e disse, rivolto a Svijazskij e cercando di indurlo ad esprimere la sua ponderata opinione: 

- Il fatto che il livello dell'azienda si sia abbassato e che, dati i nostri rapporti con i lavoratori, non sia possibile condurre in maniera vantaggiosa un'azienda razionale, è del tutto vero - egli disse. 
\enlargethispage*{1\baselineskip}

- Non sono d'accordo - ribatté ormai con serietà Svijazskij. - Io vedo solo che noi non sappiamo condurre l'azienda e che, d'altra parte, quest'azienda che noi abbiamo condotto durante la servitù della gleba, certamente non era troppo alta, ma invece troppo bassa di livello. Ma noi non abbiamo macchine né buon bestiame da lavoro, non abbiamo una buona amministrazione, e non sappiamo fare i conti. Chiedete a un proprietario; egli non sa quello che gli conviene e quello che non gli conviene. 

- Contabilità all'italiana - disse ironico il proprietario. - In qualunque modo fai i conti, quando ti sciupano tutto, non c'è guadagno. 

- Perché ti sciupano? Una cattiva macchina per battere, il vostro topcak russo li spezzeranno; ma la mia macchina a vapore non la spezzeranno. Un ronzino russo, di razza da tiro, uno di quelli da trascinar per la coda, ve lo sciuperanno, ma mettete su dei percesi o almeno dei buoni cavalli da tiro, non ve li sciuperanno, e così per tutto. Occorre portare a un livello più alto l'azienda. 

- Ma ci fossero i mezzi Nikolaj Ivanovic! Voi state bene, ma io che debbo mantenere un figlio all'università, mandare i piccoli al ginnasio, io i percesi non me li posso comprare. 

- E per questo ci sono le banche. 

- Per costringermi a vendere all'asta le ultime cose? No, grazie. 

- Io non sono d'accordo che si debba o si possa sollevare lo stato dell'azienda domestica - disse Levin. - È di questo che mi occupo io e ne ho i mezzi, e non posso fare niente. Le banche non so a chi siano utili. Quanto a me, per qualunque cosa abbia speso del denaro nell'azienda, ho sempre perduto tutto: bestiame\ldots{} perdita, macchine\ldots{} perdita. 

- Ecco, la precisa verità - affermò, persino ridendo dalla soddisfazione, il proprietario dai baffi grigi. 

- E non sono il solo - continuò Levin - io mi appello a tutti i proprietari che conducono razionalmente una azienda; tutti, salvo rare eccezioni, lavorano in perdita. Su, dite voi, è forse attiva la vostra azienda? - disse Levin, e subito nello sguardo di Svijazskij notò quella fugace espressione di spavento che egli vi scorgeva ogni volta che voleva andare oltre le stanze da ricevimento della mente di Svijazskij. 

Inoltre questa domanda, da parte di Levin, non era del tutto onesta. La padrona di casa, durante il tè, gli aveva detto proprio allora che quell'estate avevano fatto venire da Mosca un tedesco esperto di computisteria che per cinquecento rubli aveva verificato i conti della loro azienda e aveva trovato tremila rubli e più di deficit. Non ricordava con precisione quanto, ma sembrava che il tedesco avesse spaccato il millesimo. 

Il proprietario, a sentir l'allusione ai profitti dell'azienda di Svijazskij, sorrise, conoscendo, evidentemente, quale potesse essere il profitto del vicino maresciallo della nobiltà. 

- Può darsi che sia infruttuosa - rispose Svijazskij. - Questo dimostra soltanto o che sono un cattivo padrone o che spendo il capitale per aumentare la rendita. 

- Ah, la rendita! - esclamò Levin con orrore. - Può darsi che in Europa si percepisca una rendita là dove la terra è stata migliorata dal lavoro, ma da noi tutta la terra è stata peggiorata, o rovinata dal lavoro; dunque niente rendita. 

- Come, non c'è rendita? Ma questa è di regola. 

- Allora siamo fuori regola. La rendita per noi non spiega nulla, al contrario, confonde. No, ditemi, come la teoria della rendita può essere\ldots{} 
\enlargethispage*{1\baselineskip}

- Volete del latte cagliato? Maša, portaci qua del latte cagliato o dei lamponi - disse rivolto alla moglie. - Quest'anno i lamponi si mantengono straordinariamente a lungo. 

E nel più piacevole stato d'animo Svijazskij si alzò e si allontanò, supponendo, evidentemente, che la conversazione fosse finita proprio nel punto in cui a Levin sembrava che fosse appena cominciata. 

Rimasto privo di un interlocutore, Levin continuò la conversazione col proprietario, cercando di dimostrargli che tutte le difficoltà dipendevano dal fatto che noi non vogliamo conoscere la peculiarità e le abitudini del lavoratore; ma il proprietario era, come tutti gli uomini che pensano col proprio cervello e in solitudine, tetragono alla comprensione d'un pensiero altrui e particolarmente appassionato al proprio. Egli insisteva sempre nel dire che il contadino russo è un porco, che ama la sporcizia e che per farlo uscire dalla sporcizia ci vuole autorità e questa non c'è, che occorre il bastone, e che invece si è diventati così liberali da cambiar d'un tratto il millenario bastone in certi avvocati e in certi incarceramenti, grazie ai quali si dava da mangiare una buona zuppa a questi cialtroni di contadini puzzolenti, preoccupandosi di calcolare il loro spazio vitale. 

- Perché non pensate - disse Levin, cercando di tornare alla questione - che si possa trovare con la forza lavoratrice un rapporto tale che il lavoro risulti produttivo? 

- Ciò non avverrà mai col popolo russo! Non c'è autorità - rispose il proprietario. 

- E come si possono trovare rapporti nuovi? - disse Svijazskij che, dopo aver mangiato del latte cagliato e fumato una sigaretta, si era avvicinato di nuovo a quelli che discutevano. - Tutti i rapporti possibili con la forza lavoratrice sono definiti e studiati - egli disse. - La comunità primordiale, un resto di barbarie, con la mutua garanzia, va in rovina da sé; il diritto di servitù è stato annientato, rimane solo il lavoro libero, e le sue forme sono definite e già bell'e pronte; non si può che prendere queste: l'operaio a giornata, il bracciante, il fittavolo, e di qua non si esce. 

- Ma l'Europa non è contenta di queste forme. 

- Non è contenta e ne cerca di nuove. Ne troverà, probabilmente. 

- Io dico solo questo - rispose Levin. - Perché non dobbiamo cercare anche noi da parte nostra? 

- Perché è lo stesso che inventare dei nuovi metodi per la costruzione delle strade ferrate. Sono pronti, sono già inventati. 

- Ma se non ci convengono, se sono idioti? - disse Levin. 

E di nuovo notò l'espressione di spavento negli occhi di Svijazskij. 

- Sì, questo: canteremo vittoria, abbiamo trovato quello che l'Europa cerca! Io so tutto questo, ma, perdonatemi, voi conoscete quel che è stato fatto in Europa a proposito della questione dell'organizzazione dei lavoratori? 

- No, non abbastanza. 

- Questa questione occupa ora le menti migliori in Europa. La tendenza di Schulze-Delitzsch\ldots{} Poi questa enorme letteratura sulla questione operaia, la tendenza più liberale di Lassalle\ldots{} L'organizzazione di Mühlhausen è già un fatto, forse lo sapete. 

- Ne ho un'idea, ma molto confusa. 

- No, lo dite voi; probabilmente saprete tutto questo non peggio di me. Io, s'intende, non sono un professore di sociologia, ma ciò mi interessa e, davvero, se v'interessa, occupatevene. 

- Ma a che mai sono pervenuti? 

- Perdonate\ldots{} 
\enlargethispage*{1\baselineskip}

I proprietari di terre s'erano alzati, e Svijazskij, avendo di nuovo fermato Levin nella sua antipatica abitudine di guardare quello che si trovava al di là delle stanze di ricevimento del proprio cervello, andò ad accompagnare gli ospiti. 
\enlargethispage*{1\baselineskip}

\capitolo{XXVIII}Levin s'annoiava terribilmente quella sera, con le signore: lo agitava, come non mai prima, il pensiero che quell'insoddisfazione che egli provava circa la sua azienda, fosse, non una sua personale situazione, ma una condizione generale delle cose in Russia; che la sistemazione dei rapporti con i lavoratori in modo che producessero così come da quel contadino presso il quale s'era fermato a mezza strada, non fosse un sogno, ma un problema che urgesse risolvere. E gli pareva che si dovesse tentare di risolverlo, questo problema. 

Dopo essersi congedato dalle signore e dopo aver promesso di rimanere ancora l'indomani per tutto il giorno, per andare insieme a cavallo a visitare una frana interessante nel bosco demaniale, Levin, prima di andare a dormire, passò un momento per lo studio del padrone di casa a prendere dei libri sulla questione operaia che Svijazskij gli aveva offerto. 

Lo studio di Svijazskij era una enorme stanza mobiliata con armadi pieni di libri e due tavole: uno scrittoio massiccio posto al centro della stanza, l'altra, una tavola rotonda, coperta degli ultimi numeri di giornali e di riviste in varie lingue disposti a raggiera intorno alla lampada. Presso lo scrittoio c'era un banco con delle cassette piene di pratiche di vario genere contrassegnate da etichette dorate. 

Svijazskij tirò fuori i libri e sedette su di una poltrona a dondolo. 

- Cos'è che guardate? - disse a Levin che, fermo vicino alla tavola, dava una scorsa alle riviste. 

- Ah, sì, qui c'è un articolo molto interessante - disse Svijazskij, indicando la rivista che Levin teneva tra le mani. - Vi è dimostrato - aggiunse con allegra animazione - che il responsabile principale della spartizione della Polonia non è stato affatto Federico. Risulta\ldots{} 

E con quella chiarezza che gli era propria, espose in breve quelle nuove, molto importanti e interessanti scoperte. Malgrado la mente di Levin fosse ora tutta occupata dal pensiero della azienda, egli ascoltava il padrone di casa domandandosi: ``Che c'è in lui? E perché, perché gli interessa la spartizione della Polonia?''. Quando Svijazskij finì, Levin involontariamente chiese: ``E allora?''. Ma non c'era nulla di interessante, c'era solo che ``risultava''. Ma Svijazskij non spiegò né trovò necessario spiegare perché questo era interessante. 

- Sì, ma mi ha molto interessato il proprietario rabbioso - disse Levin dopo aver sospirato. - È intelligente e ha detto molte cose esatte. 

- Oh, via! Un inveterato sostenitore segreto della servitù della gleba, come tutti loro - disse Svijazskij. 

- Di cui voi siete il capo. 

- Già, solo che io li capeggio dall'altra parte - disse, ridendo, Svijazskij. 

- Ecco cosa mi interessa molto - disse Levin. - Egli ha ragione di dire che gli affari nostri, cioè l'azienda razionale, non va; va solo l'azienda usuraia, come quella di quel bonaccione, o la più semplice\ldots{} Di chi la colpa? 

- S'intende, nostra. Ma già\ldots{} non è poi vero che non vada. Da Vasil'cikov va. 

- Una fabbrica\ldots{} 

- Tuttavia non so che cosa vi sorprenda. Il popolo è in uno stato così basso di sviluppo materiale e morale che, evidentemente, si oppone a tutto quello di cui ha invece bisogno. In Europa l'azienda razionale va perché il popolo è istruito; dunque da noi bisogna educare il popolo, ecco tutto. 

- Ma come istruire il popolo? 

- Per istruire il popolo ci vogliono tre cose: le scuole, le scuole, le scuole. 

- Ma voi avete detto che il popolo si trova a un grado di sviluppo materiale bassissimo. In che modo, allora, possono aiutare le scuole? 

- Sentite, voi mi ricordate l'aneddoto sui consigli a un malato: ``Dovreste provare un purgante''. ``Me l'hanno dato\ldots{} peggio\ldots{}''. ``Dovreste provare le sanguisughe''. ``Me le hanno applicate\ldots{} peggio''. E così siamo anche noi, io e voi. Io dico: economia politica, voi dite\ldots{} peggio. Io dico: socialismo, voi\ldots{} peggio, l'istruzione, voi\ldots{} peggio. 

- Ma in che modo mai le scuole potranno essere di aiuto? 

- Daranno al popolo altre esigenze. 

- Ecco, io questo non l'ho mai capito - ribatté Levin con calore. - In che modo le scuole possano aiutare il popolo a migliorare la sua condizione materiale. Voi dite, la scuola, l'istruzione, gli daranno nuovi bisogni. Tanto peggio, perché ancor meno avrà la capacità di soddisfarli. E in che modo la conoscenza dell'addizione e della sottrazione e del catechismo lo aiuteranno a migliorare il proprio stato materiale, io non l'ho mai potuto capire. L'altro ieri sera ho incontrato una donna con un bambino poppante e le ho chiesto dove fosse andata. Mi ha detto: ``Sono andata dalla mamma. Al bambino gli era venuto il frigno, così gliel'ho portato per farlo curare''. Le ho chiesto come la donnetta curasse quel piangere continuo. ``Mette a sedere il bambino sulla gruccia delle galline e poi dice qualche cosa''. 

- Ebbene, ecco, lo dite voi stesso. Per questo c'è bisogno di istruire il popolo, perché non vada a far curare il frigno sulla gruccia - disse Svijazskij, sorridendo allegramente. 

- Ah no - disse Levin con stizza. - Per me questo metodo di cura è simile a quello che vuol curare il popolo con le scuole. Il popolo è povero e ignorante; questo noi lo vediamo con la stessa chiarezza con la quale la fattucchiera vede il frigno, che è reso evidente dal fatto che il ragazzo frigna. Ma come e perché contro questo malanno della povertà e dell'ignoranza, debbano giovare le scuole è così incomprensibile come è incomprensibile che contro il frigno giovino le grucce delle galline. Bisogna aiutare il popolo in quanto è povero. 

- Bene, almeno in questo andate d'accordo con Spencer che amate così poco; anche lui dice che l'educazione può essere la conseguenza di un grande benessere e di un'agiatezza, di frequenti lavaggi, come egli dice, ma non del saper leggere e far di conto. 

- Ebbene, ecco, io sono molto contento, o, al contrario, molto poco contento di andar d'accordo con lo Spencer; io questo l'ho sempre pensato. Non sono le scuole che possono aiutare, ma gioverà un'organizzazione economica in virtù della quale il popolo sarà più ricco, e avrà più tempo libero; allora ci saranno anche le scuole. 

- Ma in tutta Europa attualmente le scuole sono obbligatorie. 

- E come mai voi stesso concordate in questo con lo Spencer? - domandò Levin. 

Ma negli occhi di Svijazskij balenò quella tale espressione di spavento, ed egli disse sorridendo: 

- No, questa faccenda del frigno è magnifica! Ma è proprio capitata a voi? 

Levin si accorse che non avrebbe mai trovato il nesso tra la vita di quell'uomo e le sue idee. Evidentemente a lui era del tutto indifferente la conseguenza a cui l'avrebbe portato il ragionamento. E si rammaricava quando il filo del ragionamento lo conduceva a volte in un vicolo cieco. Da questo rifuggiva, portando il discorso su qualcosa di piacevolmente allegro. 

Tutte le impressioni del giorno, a cominciare da quella del contadino presso il quale s'era fermato a mezza strada e che era servita come base fondamentale alle altre impressioni e a tutti i pensieri della giornata, avevano fortemente scosso Levin. Quel simpatico Svijazskij, che sfruttava le sue idee soltanto per uso pubblico ed evidentemente aveva altre basi di vita, misteriose per Levin, e che pertanto guidava l'opinione pubblica, una folla il cui nome è legione, con idee che non applicava alla vita propria; quel proprietario rabbioso, che diceva cose giustissime nei suoi ragionamenti, basati su esperienze concrete, ma che era ingiusto nella sua avversione contro tutta una classe, e la migliore classe della Russia; la sua stessa insoddisfazione della propria attività e la confusa speranza di trovar rimedio, tutto ciò confluiva in un senso di intima inquietudine e di attesa in una soluzione prossima. 

Rimasto nella camera assegnatagli, coricato su di un saccone a molle che lo faceva sobbalzare a ogni movimento del braccio o della gamba, Levin non dormì a lungo. Neppure la conversazione con Svijazskij, sebbene molte cose intelligenti fossero state dette fra di loro, interessava Levin; ma gli argomenti di quel proprietario esigevano un esame. Gli tornarono in mente tutte le sue parole e corresse dentro di sé la risposta che gli aveva dato. 

``Già avrei dovuto dirgli: voi dite che la nostra azienda non va perché il contadino odia tutti i perfezionamenti, e che occorre introdurli di autorità; ma se, senza questi perfezionamenti, l'azienda agraria non andasse del tutto, allora, sì, voi avreste ragione; ma essa invece va, ma soltanto dove il contadino opera in conformità con le sue abitudini, come lì dal vecchio che ho incontrato a mezza strada. Il fatto che non siamo soddisfatti della mia e della vostra azienda dimostra che colpevoli siamo o noi o i lavoratori. Noi già da tempo andiamo innanzi a modo nostro, all'europea, senza tener conto delle peculiarità dell'elemento lavoratore. Proviamo a considerare la forza lavoratrice non come una forza lavoratrice astratta, ma come contadino russo, con i suoi istinti, e costruiamo conformemente ad esso l'azienda. Immaginatevi, avrei dovuto dirgli, che da voi l'azienda vada come dal vecchio; che voi abbiate trovato il modo di interessare i lavoratori al buon successo del lavoro e che abbiate trovato nei perfezionamenti quello stesso punto di mezzo che essi apprezzano, e che infine, senza isterilire il suolo, ricaviate il doppio o il triplo in confronto di prima. Dividete a metà, date la metà alla forza lavoratrice; la differenza che vi rimane sarà più remunerativa per voi, e alla forza lavoratrice spetterà di più. Ma per fare ciò, occorre abbassare il livello dell'azienda e interessare i lavoratori alla prosperità di questa. Come riuscire, è questione di particolari, ma indubbiamente ciò è possibile''. 

Questa idea gettò Levin in una forte agitazione. Non dormì per metà della notte, pensando ai particolari dell'esecuzione della sua idea. Egli non aveva intenzione di partire il giorno dopo, ma ora decise di partire la mattina presto per tornare a casa. Inoltre quella cognata con la scollatura suscitava in lui un sentimento simile alla vergogna e al pentimento di una cattiva azione. La ragione precipua dell'immediata partenza era poi questa: occorreva proporre presto ai contadini il nuovo programma, prima della semina d'autunno, in modo che si potesse seminare già secondo i suoi nuovi criteri. Aveva deciso di dare un aspetto del tutto diverso all'azienda di prima. 

\capitolo{XXIX}La realizzazione del piano di Levin presentava molte difficoltà; ma egli si batté con quanta forza aveva, e ne ottenne, sia pure non proprio quello che voleva, ma la convinzione almeno di poter credere, senza ingannarsi, che l'impresa valesse il lavoro. Una delle difficoltà principali consisteva nella circostanza che l'azienda era già avviata, che non si poteva fermare e riordinare daccapo, e che bisognava rinnovare la macchina mentre era in moto. 

Quando, la sera stessa in cui arrivò a casa, comunicò all'amministratore i suoi piani, l'amministratore con evidente soddisfazione, concordò con lui in quella parte del discorso in cui si dimostrava che quanto si era fatto finora era stato assurdo e poco conveniente. L'amministratore disse che egli l'aveva detto da tempo, e che non si era voluto ascoltarlo. Per quello poi che si riferiva alla proposta fattagli da Levin: prender parte come consocio insieme ai lavoratori in tutta l'azienda agricola, l'amministratore dette a vedere solo un grande scoraggiamento e nessuna opinione definita, e subito portò il discorso sulla necessità di trasportare l'indomani gli ultimi covoni di segala e di mandare a fare la seconda aratura; di modo che Levin capì che si era ancora lontani dal suo programma. 

Quando, infatti, lo prospettava ai contadini e faceva ad essi la proposta dell'affitto di terre a nuove condizioni, s'imbatteva pure nella difficoltà principale che essi, cioè, erano così occupati dalla fatica giornaliera da non avere il tempo di riflettere ai vantaggi e agli svantaggi dell'impresa. 

Un contadino semplicione, Ivan il pecoraio, parve capire in pieno la proposta di Levin, di prendere parte cioè, con la sua famiglia, agli utili della stalla, e aderì in pieno a questa impresa. Ma appena Levin gli prospettava i vantaggi futuri, sulla faccia di Ivan si esprimevano agitazione e rammarico per non poter rimanere ad ascoltare tutto fino all'ultimo, ed egli si dava in fretta a una qualche faccenda che non ammetteva dilazione: o afferrava la forca per finir di cacciare il fieno dal recinto, o cominciava a versare l'acqua, o a riassestare il letame. 

Un'altra difficoltà consisteva nell'invincibile sospetto dei contadini che lo scopo del proprietario non potesse consistere in altro che non fosse il desiderio di spogliarli il più possibile. Essi erano fermamente convinti che qualunque cosa avesse detto loro, il vero scopo sarebbe stato quello che egli non avrebbe detto. Ed essi stessi, nell'esporre il proprio modo di vedere, parlavano molto, ma non dicevano mai in che cosa consistesse il loro vero scopo. Inoltre (Levin sentiva che il proprietario bilioso aveva ragione), i contadini, come prima ed immutabile condizione per un qualsiasi accordo, ponevano quella di non essere costretti a introdurre un qualsiasi nuovo procedimento nell'azienda e l'uso di attrezzi nuovi. Essi convenivano che l'aratro arava meglio, che l'estirpatrice lavorava con maggiore rendimento, ma trovavano mille ragioni per non dover usare né l'uno né l'altra, e, sebbene egli fosse convinto che occorresse abbassare il livello dell'azienda, gli spiaceva rinunziare a perfezionamenti il cui vantaggio era tanto evidente. Malgrado tutte queste difficoltà, egli raggiunse il suo scopo, e verso l'autunno la cosa cominciò ad andare, così almeno gli pareva. 

Dapprima Levin pensò di dare tutta l'azienda, nello stato in cui era, in affitto ai contadini, ai lavoratori e all'amministratore, a nuove condizioni di compartecipazione; ma ben presto si convinse che ciò non era possibile, e si decise a suddividere l'azienda. La stalla, il giardino, l'orto, i prati, i campi, tutti divisi in settori, dovevano costituire parti separate. Quel semplicione d'Ivan il pecoraio, che pareva a Levin avesse capito la cosa meglio di tutti quanti, messa insieme un'artel' nella sua massima parte formata da persone di famiglia, diventò consocio della stalla. I campi lontani, che erano rimasti otto anni incolti sotto gli sterpi, furono assunti, con l'aiuto di un legnaiolo intelligente, Fëdor Rezunov, da sei famiglie di contadini alle nuove condizioni di compartecipazione, e il contadino Šuraev prese in affitto, alle stesse condizioni, tutti gli orti. Il resto rimase ancora come prima, ma queste tre parti erano già il principio della nuova sistemazione, e tenevano occupato completamente Levin. 

Era vero che nella stalla le cose finora non andavano meglio di prima, e Ivan si era vivamente opposto al locale caldo per le mucche e alla produzione del burro, asserendo che, se si tenevano le mucche in ambiente freddo, ci sarebbe voluta minor quantità di foraggio e che il burro di crema acida era più redditizio; ed era vero che pretendeva una paga come per l'innanzi, e non si dava per inteso che il denaro così pagatogli non era una paga, ma un'anticipazione sulla sua quota di partecipazione al guadagno. 

Era vero che i contadini dell'artel' di Fëdor Rezunov non avevano arato per la seconda volta con gli aratri prima della semina, così come era stato convenuto, giustificandosi che il tempo era stato breve. Era vero che i contadini di quest'artel', pur avendo pattuito di condurre la cosa su nuove basi, non dicevano che quella terra era condotta in comune, ma a mezzadria; e più di una volta i contadini di quell'artel' e lo stesso Rezunov avevano detto a Levin: ``se prendeste l'affitto per la terra, sareste più tranquillo voi e per noi sarebbe una liberazione''. Inoltre questi contadini rimandavano sempre, con vari pretesti, la convenuta costruzione di una stalla e di un granaio e avevan tirato in lungo fino all'inverno. 

Era vero che Šuraev avrebbe voluto distribuire gli orti da lui assunti in compartecipazione in piccoli lotti ai contadini. Evidentemente aveva capito tutto a rovescio, e pareva che, coscientemente, avesse capito a rovescio le condizioni alle quali era stata data la terra. Era vero che spesso, parlando con i contadini e spiegando loro tutti i vantaggi dell'impresa, Levin sentiva ad ogni momento che i contadini udivano solo il suono della sua voce e sapevano fermamente che qualunque cosa dicesse, essi non si sarebbero fatti ingannare da lui. In particolare egli sentiva questo, quando parlava con il più intelligente dei contadini, con Rezunov, e notava una certa scintilla negli occhi di lui che, mentre era irrisoria verso Levin, esprimeva anche la ferma convinzione che, se qualcuno dovesse essere ingannato, questo qualcuno non sarebbe certamente stato lui, Rezunov. 

Malgrado tutto questo Levin pensava che la faccenda andava e che, tenendo strettamente i conti e insistendo nelle sue idee, egli avrebbe mostrato loro, nel futuro, i vantaggi di una simile sistemazione, e la cosa allora sarebbe andata da sé. 

Tutte queste faccende, insieme al resto dell'azienda che era rimasta nelle sue mai, insieme allo studio per il suo libro, occuparono tanto Levin per tutta l'estate, che egli non andò quasi mai a caccia. Alla fine di agosto seppe, dalla persona venuta a riportare la sella, che gli Oblonskij erano andati a Mosca. Sentiva che non avendo risposto alla lettera di Dar'ja Aleksandrovna, aveva bruciato le sue navi con questa scortesia, che non poteva ricordare senza arrossire; e ormai non sarebbe più andato da loro. Proprio allo stesso modo aveva agito con Svijazskij, essendo partito senza salutare. Ma neanche da loro sarebbe più andato. Tutto questo gli era però indifferente. L'affare della nuova sistemazione dell'azienda lo occupava come mai finora nessun'altra cosa in vita sua. Lesse i libri datigli da Svijazskij, e, ordinati quelli che non aveva, lesse anche i libri di economia politica e di socialismo che riguardavano questo argomento, e, come si aspettava, non trovò nulla che si riferisse all'opera da lui intrapresa. Nei libri di economia politica, nello Stuart Mill, per esempio, che studiò per primo con grande entusiasmo, sperando di trovare ad ogni passo la soluzione delle questioni che lo interessavano, trovò delle leggi tratte dalla situazione dell'economia europea, ma non poté in nessun modo rendersi conto perché queste leggi, inapplicabili in Russia, dovessero essere dichiarate generali. Lo stesso vide anche nei libri sul socialismo; o erano bellissime fantasie, ma inattuabili, alle quali si era appassionato quando era ancora studente, o erano correzioni, revisioni di uno stato di cose in cui si era venuta a trovare l'Europa e con il quale l'agricoltura non aveva nulla in comune. L'economia politica diceva che le leggi, secondo le quali si era sviluppata e si sviluppava la ricchezza dell'Europa, erano leggi universali e indubitabili; e la dottrina socialista diceva invece che andare innanzi secondo queste leggi portava alla rovina. Né l'una né l'altra davano non già una soluzione, ma neppure il più piccolo accenno a quello che lui, Levin, e tutti i contadini e i proprietari terrieri russi dovessero fare con i loro milioni di braccia e di desjatiny di terra, perché fossero le une e le altre il più possibile produttive per il benessere generale. 

Datosi a questi studi, leggeva coscienziosamente tutto quello che riguardava la materia, e aveva intenzione di andare in ottobre all'estero per studiare la questione sul posto, affinché non gli capitasse più, in questa faccenda, quello che spesso gli era capitato in varie altre. Gli era accaduto che, appena aveva cominciato a capire il pensiero dell'interlocutore e a esprimere il suo, gli era stato detto a un tratto: ``E Kauffmann, e Johns, e Dubois, e Miceli? Non li avete letti? Leggeteli, hanno sviscerato tutta la questione''. 

Egli vedeva, ora, chiaramente, che Kauffmann e Miceli non avevano nulla da dirgli. Sapeva lui quello che ci voleva. Vedeva che la Russia aveva terre bellissime, ottimi lavoratori e che in alcuni casi, come dal contadino a mezza strada, i lavoratori e le terre rendevano molto; nella maggioranza dei casi, invece, quando i capitali venivano impiegati all'europea, producevano poco, e ciò dipendeva unicamente dal fatto che i lavoratori volevano lavorare e lavoravano bene solo a quel modo che era loro proprio, e che la loro opposizione alle novità non era casuale ma costante, avendo radici nello spirito stesso del popolo. Pensava che il popolo russo, che aveva la vocazione di dissodare con alacrità e di popolare enormi estensioni disabitate, finché le terre non fossero tutte occupate, si atteneva ai metodi necessari per questo lavoro, e che questi metodi non erano poi così cattivi come di solito si pensava. E voleva dimostrare ciò in teoria, nel libro, e, nella pratica, con la sua azienda agraria. 

\capitolo{XXX}Alla fine di settembre fu trasportato il legname per la costruzione della stalla sul terreno concesso all'artel' e fu venduto il burro e se ne divise l'utile. Nell'azienda, in pratica, le cose andavano ottimamente, o, per lo meno, così sembrava a Levin. Per chiarire però teoricamente tutta la faccenda e terminare l'opera che, secondo i suoi sogni, avrebbe dovuto non solo portare una rivoluzione nell'economia politica, ma annientare completamente questa scienza, e instaurarne un'altra, basata sui rapporti del contadino con la terra, occorreva non solo andare all'estero e studiare sul posto tutto quello che era stato fatto con questo indirizzo, ma trovare argomenti convincenti per dimostrare che tutto quello che era stato fatto là, non era quello che occorreva fare. Levin attendeva solo la consegna del frumento per ritirare il denaro e andare all'estero. Ma cominciarono le piogge, che non permisero di raccogliere le patate e il grano rimasti nel campo, e fermarono tutti i lavori e perfino la consegna del frumento. Per le strade c'era un fango insormontabile; due mulini furono divelti dalla piena, e il tempo divenne sempre peggiore. Il 30 settembre, fin dal mattino, apparve il sole e, sperando nel sereno, Levin si decise a fare i preparativi per il viaggio. Ordinò d'insaccare il frumento, mandò l'amministratore dal compratore a ritirare il denaro, ed egli stesso andò in giro per la fattoria a dare gli ultimi ordini prima della partenza. 

Dopo aver fatto tutto questo, madido d'acqua che, a rivoli, gli scendeva per la giacca di pelle giù per il collo e nei gambali, ma nella migliore disposizione d'animo, sveglio ed alacre, Levin prese la via del ritorno, a sera. Ma il maltempo, verso sera, si scatenò ancor peggio; la grandine sferzava il cavallo che, tutto bagnato, scoteva le orecchie e la testa, sì che andava di traverso; ma Levin, sotto il cappuccio, si sentiva a suo agio e guardava allegramente intorno a sé, ora i ruscelli torbidi che correvano per le carreggiate, ora le gocce d'acqua che gocciavano giù da ogni ramo nudo, ora il bianco delle piazzuole di grandine minuta che non si era ancora sciolta sulle tavole del ponte, ora il fogliame dell'olmo, carnoso e ancora denso di umori, che s'era ammassato per terra in uno strato spesso, intorno all'albero spoglio. Malgrado la tetraggine della natura circostante, egli si sentiva fortemente eccitato. I discorsi fatti con i contadini in un villaggio lontano gli avevano dimostrato che cominciavano ad abituarsi ai nuovi rapporti. Il vecchio portiere, dal quale s'era fermato per asciugarsi, evidentemente approvava il piano di Levin se, di sua iniziativa, proponeva di entrare in società per l'allevamento del bestiame. 

``Occorre solo andare tenacemente verso il proprio scopo, ed io riuscirò ad attuare la mia idea - pensava Levin. - E c'è una ragione al mio lavoro e alla mia fatica. Questa faccenda non riguarda solo la mia persona, ma qui si tratta del bene generale. Tutta l'economia domestica, la cosa principale, la situazione di tutto il popolo deve cambiare completamente. Invece della povertà, una ricchezza generale, un'agiatezza; invece dell'odio, la concordia e il legame degli interessi. In una parola, una rivoluzione incruenta, ma una profondissima rivoluzione; inizialmente nella piccola cerchia del nostro distretto, poi nel governatorato, poi nella Russia, nel mondo intero. Perché un'idea giusta non può non essere feconda. Sì, questo è uno scopo per cui vale la pena di lavorare. E il fatto che ad aver avuto questa idea sia stato io, Kostja Levin, quello stesso che giunse al ballo in cravatta nera e al quale la Šcerbackaja disse di no, e che, quanto alla propria persona, è così pietoso e insignificante, questo non significa nulla. Io sono sicuro che Franklin si sentiva insignificante allo stesso mio modo, ed aveva, così come faccio io, poca stima di sé, quando considerava se stesso. Questo non significa nulla. E anche lui, probabilmente, avrà avuto la sua Agaf'ja Michajlovna alla quale confidare i suoi piani''. 

In tali pensieri Levin giunse che era già buio a casa. 

L'amministratore, che era andato dal compratore, venne e portò una parte del denaro del frumento. Il contratto col portiere era stato concluso e per istrada l'amministratore aveva saputo che il grano altrove era rimasto nei campi, così che i loro centosessanta covoni non raccolti erano poca cosa in confronto di quello che era rimasto agli altri. 

Dopo aver pranzato, Levin sedette, come al solito, con un libro sulla poltrona, e, leggendo, continuava a pensare al suo viaggio imminente collegato alla sua opera. Ora gli si presentava in modo particolarmente chiaro tutta la sostanza della questione e nella sua mente si formavano interi periodi che esprimevano di per sé l'essenza del suo pensiero. ``Questo bisogna scriverlo - pensò. - Devo scrivere una breve introduzione che prima non ritenevo necessaria''. Si alzò per andare allo scrittoio, e Laska, che era sdraiata ai suoi piedi, stirandosi, si alzò essa pure e lo guardò come a chiedergli dove andasse; ma non c'era tempo per scrivere, perché erano venuti i capoccia a prendere gli ordini, e Levin andò da loro in anticamera. 

Dopo aver impartito gli ordini, cioè le disposizioni per i lavori dell'indomani e dopo aver ricevuto tutti i contadini che avevano da parlargli, Levin andò nello studio e tornò al suo lavoro. Laska si coricò sotto la tavola, Agaf'ja Michajlovna con la calza in mano si mise a sedere al proprio posto. 

- Ma non c'è ragione che vi annoiate - gli disse Agaf'ja Michajlovna. - Perché restate a casa? Se andaste alle acque termali, dal momento che vi siete già preparato? 

- Parto proprio domani, Agaf'ja Michajlovna. Bisogna risolvere la faccenda. 

- Ma che lavoro il vostro! Come se non aveste già ricompensato abbastanza i contadini! E dicono: il vostro signore riceverà per questo una grazia dallo zar. Ed è strano: perché dovete voi angustiarvi tanto per i contadini? 

- Io non mi angustio per loro; lo faccio per me. 

Agaf'ja Michajlovna conosceva tutti i particolari dei piani economici di Levin. Spesso Levin le esponeva in tutti i particolari i suoi pensieri e non di rado discuteva con lei che a volte non conveniva con le sue spiegazioni. Ma ora aveva capito in modo affatto diverso quello che egli le aveva detto. 

- All'anima propria, è certo, bisogna pensarci più di tutto - ella disse con un sospiro. - Ecco, Parfën Denisyc, un analfabeta, è morto così come Iddio lo conceda a ognuno - ella disse, alludendo a un domestico morto da poco. - L'hanno comunicato, gli hanno dato l'olio santo. 

- Non è questo che dico - egli rispose. - Io dico che lo faccio per mio vantaggio. È maggiore il mio vantaggio, se i contadini lavorano meglio. 

- Ma per quanto voi facciate, se uno è pigro cadrà sempre come un sacco. Se la coscienza c'è lavorerà, altrimenti non c'è niente da fare. 

- Su, via, voi stessa dite che Ivan s'è messo d'impegno a custodire il bestiame. 

- Io dico solo - rispose Agaf'ja Michajlovna, evidentemente non a caso, ma con severa coerenza di pensiero - che voi avete bisogno di prender moglie, ecco! 

Il riferimento di Agaf'ja Michajlovna alla stessa cosa cui egli aveva proprio allora pensato, lo amareggiò e lo offese. Levin si accigliò e, senza risponderle, si accinse di nuovo al suo lavoro, ripetendosi tutto quello che aveva pensato sul significato di questo lavoro. Solo ogni tanto tendeva l'occhio allo sferruzzare di Agaf'ja Michajlovna, e, ricordando quello che non voleva ricordare, si accigliava di nuovo. 

Alle nove si sentì un campanellino e il sordo traballare di una carrozza nel fango. 

- Su, ecco che sono arrivati anche gli ospiti; la noia passerà - disse Agaf'ja Michajlovna, alzandosi e dirigendosi verso la porta. Ma Levin la sorpassò. In quel momento il suo lavoro non andava, ed egli era felice e contento di un ospite qualsiasi. 

\capitolo{XXXI}Dopo essere sceso giù fino a mezza scala, Levin sentì nell'ingresso un tossicchiare a lui noto; ma lo sentì poco chiaro a causa del rumore dei propri passi e sperò di essersi sbagliato; poi scorse una figura alta, ossuta, nota, e gli parve, non si poteva ormai più sbagliare, eppure ancora lo sperava, che quell'uomo lungo che si toglieva la pelliccia e tossiva fosse suo fratello Nikolaj. 

Levin voleva bene a suo fratello, ma stare con lui a lungo era un tormento. Ora poi che, per effetto del pensiero che gli era venuto in mente e dell'accenno di Agaf'ja Michajlovna, si trovava in uno stato d'animo oscuro e confuso, l'incontro imminente col fratello gli parve proprio penoso. Invece di un ospite allegro, sano, estraneo, ch'egli sperava lo distraesse dalla sua nebulosità spirituale, doveva rivedere il fratello che gli leggeva dentro da parte a parte, che avrebbe suscitato in lui i pensieri più intimi, che lo avrebbe obbligato a confidarsi in pieno. E questo non voleva. 

Irritatosi contro se stesso per questo senso di repulsione, Levin corse nell'ingresso. Appena vide il fratello, quel senso di intima delusione scomparve immediatamente e si mutò in pena. Per quanto spaventoso fosse anche prima suo fratello Nikolaj per la magrezza e l'aria malandata, in quel momento era ancora più smagrito, ancora più debole. Era uno scheletro ricoperto di pelle. 

Stava dritto nell'ingresso, storcendo il collo lungo e magro e strappandone la sciarpa, e sorrideva in modo strano e penoso. Visto questo sorriso, umile e sottomesso, Levin sentì che un convulso gli afferrava la gola. 

- Ecco, son venuto da te - disse Nikolaj con voce cupa, senza staccare un attimo gli occhi dal viso del fratello. - Da tempo ne avevo voglia, ma non stavo mai bene. Adesso invece mi sono rimesso - diceva, asciugandosi la barba con le grandi palme magre. 

- Sì, sì - rispose Levin. E provò ancor più terrore quando nell'abbraccio sentì con le labbra l'aridità della carne del fratello e vide da vicino i suoi grandi occhi stranamente luccicanti. 

Qualche settimana prima, Levin aveva scritto al fratello che, per la vendita della piccola parte di patrimonio che era rimasta indivisa fra i germani, spettavano a lui, per la sua quota, duemila rubli. 

Nikolaj disse che era venuto per ritirare questo denaro, per stare un po' nel suo nido, per toccare un po' la terra e riprendere forza, come i giganti, per la sua prossima attività. Malgrado l'accentuato incurvamento, malgrado la magrezza che sorprendeva in rapporto alla statura, i suoi movimenti, come al solito, erano agili e a scatti. Levin lo introdusse nello studio. 

Nikolaj cambiò d'abito con particolare cura, cosa che prima non faceva, pettinò i capelli radi e dritti ed entrò, sorridendo, di sopra. 

Era di umore carezzevole e allegro come quando era fanciullo e come spesso lo ricordava Levin. Ricordò perfino Sergej Ivanovic senza rancore. Vista Agaf'ja Michajlovna, scherzò con lei e le chiese notizie dei vecchi servi. La notizia della morte di Parfën Denisyc agì in modo spiacevole su di lui. Sul suo viso si espresse lo spavento, ma si riprese subito. 

- Già, era vecchio ormai - egli disse e cambiò discorso. - Eh, sì, starò un mese, due da te, e poi a Mosca; tu sai, Mjagkov mi ha promesso un posto ed entrerò in servizio. Ora ordinerò la mia vita in maniera del tutto diversa. Lo sai, ho allontanato quella donna. 

- Mar'ja Nikolaevna? come mai, perché? 

- Ah, una donna disgustosa! M'ha dato un mucchio di dispiaceri. - Ma egli non raccontò quali erano stati questi dispiaceri. Non poteva dire che aveva cacciato Mar'ja Nikolaevna perché il tè che faceva era debole e, principalmente, perché ella aveva cura di lui come di una persona malata. - Poi, in genere, adesso voglio cambiar vita del tutto. Io, s'intende, come tutti del resto, ho fatto delle sciocchezze, ma il patrimonio è l'ultima cosa, non lo rimpiango. Purché ci sia la salute, e la salute, grazie a Dio, s'è rimessa. 

Levin ascoltava e cercava e non sapeva trovare cosa dire. Forse Nikolaj s'accorgeva di questo; cominciò a interrogare il fratello sulle sue cose; e Levin era contento di parlare di sé perché poteva parlare senza fingere. Raccontò al fratello i suoi piani e la sua attività. 

Il fratello ascoltava; ma evidentemente non si interessava. 

Questi due uomini erano così consanguinei e così vicini l'uno all'altro che il minimo movimento, il minimo cambiamento di voce diceva per entrambi più di tutto quello che si può dire a parole. 

Ora per entrambi vi era un solo pensiero: la malattia e la prossimità della morte di Nikolaj, e questo pensiero soffocava tutto il resto. Ma né l'uno né l'altro aveva il coraggio di parlarne e perciò, qualunque cosa dicessero, senza esprimere quello che unicamente interessava, aveva un tono falso. Levin non s'era mai sentito così contento come adesso che, finita la serata, si doveva andare a letto. Mai con nessun estraneo, né in nessuna visita ufficiale, era stato così poco naturale, così falso come ora. E la coscienza e il pentimento di questa falsità lo rendevano ancora più insincero. Avrebbe voluto piangere sul suo caro fratello morente, e doveva ascoltare e parlare su come avrebbe vissuto. 

Poiché in casa c'era umido e una sola camera era riscaldata Levin mise il fratello a dormire nella propria camera, di là da un tramezzo. 

Il fratello s'era coricato, dormiva e non dormiva: ma, come un ammalato, si rivoltava, tossiva, e, quando non poteva espettorare, brontolava qualcosa. A volte, mentre respirava con difficoltà, diceva: ``Ah, Dio mio!''. A volte, quando lo spurgo lo soffocava, esclamava con stizza: ``Ah, diavolo!''. Levin a lungo non dormì ascoltandolo. I pensieri di Levin erano i più difformi, ma tutti finivano in una sola cosa: la morte. 

La morte, l'inevitabile fine di tutto, per la prima volta gli si presentava con una violenza ineluttabile, e questa morte che era là in quel fratello caro che gemeva nel sonno e che per abitudine invocava indifferentemente ora Dio ora il diavolo, non era così lontana come gli era sempre parsa. Era anche in lui: lo sentiva. Se non ora domani, se non domani fra trenta anni, non era forse lo stesso? E cosa fosse questa morte inevitabile, egli non solo non lo sapeva, né mai ci aveva neppure pensato, ma non sapeva e non osava pensarci. 

``Io lavoro, voglio fare qualche cosa, ma ho dimenticato che tutto finisce, che c'è la morte''. Stava seduto sul letto nel buio, rannicchiato, stringendo fra le braccia le ginocchia e, trattenendo il respiro per la tensione del pensiero, meditava. Ma quanto più tendeva il pensiero tanto più chiaramente gli appariva che era senza dubbio così, che egli aveva realmente dimenticato, tralasciato una piccola circostanza della vita, che sarebbe cioè venuta la morte e che tutto sarebbe finito, che non valeva la pena d'intraprendere cosa alcuna e che rimediare a questo non si poteva in nessun modo. Sì, era terribile, ma era così. 

``Eppure io sono vivo ancora. E adesso che farò mai, che farò?'' diceva con disperazione. Accese una candela e cautamente si alzò e andò verso lo specchio e cominciò a guardarsi il viso e i capelli. Sì, sulle tempie v'erano dei capelli bianchi. Aprì la bocca. I molari cominciavano a guastarsi. Scoprì le braccia muscolose. Sì, c'era ancora vigore. Ma anche Nikolen'ka, che là respirava coi resti dei suoi polmoni, aveva avuto un corpo sano. E a un tratto ricordò che, bambini, si coricavano insieme e aspettavano solo che Fëdor Bogdanyc uscisse dalla porta, per gettarsi addosso l'un l'altro i guanciali e ridere, ridere irresistibilmente, così che neanche il terrore di Fëdor Bogdanyc riusciva a frenare quella consapevolezza di gioia di vivere che scaturiva e spumeggiava oltre i limiti. ``E ora questo petto vuoto e incurvato\ldots{} e io che non so perché e che cosa mi accadrà\ldots{}''. 

- Ach, ach! Ah, diavolo! Che fai in giro? non dormi? - lo chiamò la voce del fratello. 

- Ma non so, l'insonnia. 

- E io dormivo bene, adesso non sono più sudato. Guarda, tocca la camicia. Niente sudore? 

Levin palpò, andò di là dal tramezzo, spense la candela, ma ancora per molto tempo non dormì. Gli si era appena chiarita la questione di come vivere, che gli era apparsa un'altra insolubile questione: la morte. 

``Eh, sì, egli muore, sì, morirà in primavera. Come venirgli in aiuto? Che posso dirgli? Che cosa ne so io? Avevo perfino dimenticato che questa cosa esistesse''. 

\capitolo{XXXII}Levin aveva già da tempo osservato che, quando tra persone ci si sente a disagio per eccessiva cedevolezza e sottomissione, allora ben presto nasce qualcosa d'insopportabile per eccessiva pretesa e cavillosità. Sentiva che questo sarebbe accaduto anche col fratello. E invero, la mansuetudine del fratello Nikolaj non durò a lungo. Fin dalla mattina dopo divenne irritabile, e molestò con insistenza il fratello, toccandolo nei punti più dolorosi per lui. 

Levin si sentiva colpevole e non poteva rimediarvi. Sentiva che se tutti e due non avessero finto, e avessero invece parlato come si dice, a cuore aperto, dicendo solo quello che sentivano, allora si sarebbero solo guardati negli occhi l'un l'altro e Konstantin avrebbe detto soltanto: ``morirai, morirai, morirai'' e Nikolaj avrebbe risposto: ``lo so che morirò, e ho paura, paura, paura!''. E niente più avrebbero detto se avessero parlato a cuore aperto. Ma così non si poteva vivere, perché Konstantin si provava a fare quello che per tutta la vita aveva tentato e non aveva saputo mai fare, quello cioè che, secondo lui, molti sapevano fare tanto bene e di cui non si può fare a meno nella vita: si provava a dire quello che non pensava; e continuamente sentiva che ciò gli riusciva falso, che il fratello se ne accorgeva e se ne irritava. 

Due giorni dopo Nikolaj invitò il fratello a esporgli di nuovo il suo piano e prese non solo a giudicarlo, ma si mise di proposito a confonderlo col comunismo. 

- Tu hai preso soltanto un'idea non tua, poi l'hai fatta diventar mostruosa e vuoi applicarla all'inapplicabile. 

- Ma io ti dico che questo non ha nulla di comune. Loro, negano il diritto di proprietà privata, il capitale, l'ereditarietà, mentre io, senza negare questo stimolo principale - Levin era contrariato con se stesso di usare questi termini, ma da che s'era appassionato al suo lavoro, aveva cominciato involontariamente a usare sempre più spesso termini non russi - voglio solo regolare il lavoro. 

- È proprio così, hai preso un pensiero non tuo, ne hai tolto via tutto quello che ne costituisce la forza e vuoi far credere che è qualcosa di nuovo - disse Nikolaj, torcendo irritato il collo dentro la cravatta. 

- Ma la mia idea non ha nulla in comune\ldots{} 

- Là - diceva Nikolaj con gli occhi scintillanti di cattiveria e sorridendo ironicamente - là almeno vi è il fascino, per così dire, geometrico della chiarezza, della certezza. Può darsi che sia un'utopia. Ma ammettiamo che di tutto il passato si possa fare tabula rasa: non c'è proprietà, non c'è famiglia, e allora anche il lavoro si organizza. Ma da te non c'è nulla\ldots{} 

- Perché confondi? Io non sono mai stato comunista. 

- E io lo sono stato, e trovo che sia una cosa prematura, ma ragionevole e che avrà un avvenire come il cristianesimo dei primi secoli. 

- Io ritengo solo che la forza lavoratrice debba essere considerata da un punto di vista naturale, debba cioè essere studiata, riconosciuta nelle sue peculiarità e\ldots{} 

- Ma questo è completamente inutile. Questa forza trova da sé, secondo il suo grado di sviluppo, un certo modo di estrinsecarsi. Dappertutto ci sono stati gli schiavi, poi i metayers; e da noi c'è il lavoro a mezzadria, c'è il fitto, il lavoro del bracciante, che cosa cerchi di più? 

Levin d'un tratto s'accalorò a queste parole, perché in fondo all'anima temeva che ciò fosse vero, vero il fatto ch'egli in fondo volesse tenersi in equilibrio tra il comunismo e le forme passate, e che questo fosse difficilmente possibile. 

- Io cerco mezzi per lavorare produttivamente e per me e per i lavoratori. Voglio organizzare\ldots{} - rispose con calore. 

- Tu non vuoi organizzare niente; come hai sempre fatto in tutta la tua vita, hai soltanto voglia di fare l'originale, di mostrare che non sfrutti così semplicemente i contadini, ma che lo fai per un'idea. 

- Già, tu così pensi\ldots{} e basta! - rispose Levin, sentendo che il muscolo della sua guancia sinistra guizzava irresistibilmente. 

- Tu non avevi e non hai convinzioni, ma vuoi solo soddisfare il tuo amor proprio. 

- Sta benissimo; ma lasciami stare! 

- Ti lascerò! ed è un pezzo che ne era ora, e va' al diavolo! E mi spiace molto d'esser venuto. 

Per quanto poi Levin cercasse di calmare il fratello, Nikolaj non volle sentir nulla; diceva che era molto meglio separarsi, e Konstantin vedeva che al fratello era semplicemente venuta a noia la vita. 

Nikolaj era già del tutto pronto a partire, quando Konstantin di nuovo andò da lui e gli chiese, senza spontaneità, di scusarlo se in qualche modo l'aveva offeso. 

- Ah, che magnanimità! - disse Nikolaj, e sorrise. - Se vuoi aver ragione, posso farti questo favore. Hai ragione, ma tuttavia io parto! 

Proprio solo al momento di partire, Nikolaj scambiò l'abbraccio e disse a un tratto, guardando stranamente serio il fratello: 

- Tuttavia non serbarmi rancore, Kostja! - e la voce gli tremò. 

Furono queste le uniche parole che furono dette sinceramente. Levin capì che sotto queste parole si sottointendeva: ``tu vedi e sai che sto male e che forse non ci vedremo più''. Levin intese ciò, e le lacrime gli sgorgarono dagli occhi. Baciò ancora una volta il fratello, ma non poté e non seppe dirgli nulla. 

Due giorni dopo la partenza del fratello, anche Levin andò all'estero. Incontratosi alla stazione ferroviaria con Šcerbackij, il cugino di Kitty, questi fu colpito dalla tristezza di Levin. 

- Cosa c'è? - gli chiese Šcerbackij . 

- Ma, niente; al mondo c'è poco da stare allegri. 

- Come poco? ecco, venite con me a Parigi, invece di andare a quel non so che Mühlhausen. Vedrete come si sta allegri. 

- No, per me è finita. È tempo di morire. 

- Ecco un bello scherzo! - disse ridendo Šcerbackij. - Io mi sono appena preparato a cominciare. 

- Già, anch'io pensavo così fino a poco fa, ma ora so che morirò presto. 

Levin diceva quello che realmente pensava negli ultimi tempi. In tutto vedeva soltanto la morte o l'avvicinarsi di essa. Ma l'opera da lui intrapresa l'interessava ancora di più. Era pur necessario concludere in qualche modo la vita, finché non fosse giunta la morte. L'oscurità per lui copriva tutto; ma proprio a causa di questa oscurità sentiva che l'unico filo conduttore era la sua opera, e con le ultime forze si aggrappava ad essa e vi si teneva stretto. 