\parte{PARTE QUINTA}\label{parte-quinta} 
\pagestyle{pagina}

\capitolo{I}\label{i-4} 

La principessa Šcerbackaja trovava che celebrare il matrimonio prima della quaresima, alla quale mancavano cinque settimane, non era possibile, poiché metà del corredo non poteva essere pronto per quella data; ma non poteva non convenire con Levin che, dopo la quaresima, sarebbe stato già troppo tardi, poiché la vecchia zia paterna del principe Šcerbackij era molto ammalata e poteva morire da un momento all'altro, e allora il lutto avrebbe ostacolato il matrimonio. E perciò, dopo aver deciso di dividere il corredo in due parti, corredo di casa e corredo personale, la principessa acconsentì a celebrare il matrimonio prima della quaresima. Aveva deciso di preparare subito il corredo personale e di spedir dopo quello di casa, ma si irritava con Levin che non riusciva in nessun modo a risponderle seriamente se acconsentiva oppure no a questo. Una decisione simile era quanto mai opportuna, perché la giovane coppia, subito dopo il matrimonio, si sarebbe recata in campagna, dove il corredo di casa non sarebbe stato necessario. 

Levin continuava a essere sempre in quello stato di esaltazione nel quale gli era dato ritenere che lui e la sua felicità formassero il principale ed essenziale scopo di tutto quello che esisteva e che per adesso non gli occorresse pensare a preoccuparsi di nulla, che tutto per lui farebbero e avrebbero fatto gli altri. Non aveva neppure progetti e scopi per la vita futura; ne lasciava la decisione agli altri, sapendo che tutto sarebbe stato bellissimo. Suo fratello Sergej Ivanovic, Stepan Arkad'ic e la principessa lo guidavano in quello che doveva fare. Era completamente d'accordo su quello che gli proponevano. Il fratello aveva preso del denaro in prestito per lui, la principessa consigliava di partire per Mosca dopo il matrimonio, Stepan Arkad'ic di andare all'estero. Egli consentiva a tutto. ``Fate quel che volete, se ciò vi rallegra. Io sono felice e la felicità mia non può essere maggiore né minore, qualunque cosa facciate'' pensava. Quando riferì a Kitty il consiglio di Stepan Arkad'ic di andare all'estero, si sorprese molto ch'ella non acconsentisse e che avesse certe pretese ben definite sulla loro vita avvenire. Ella sapeva che Levin in campagna aveva un lavoro che amava. E non solo non intendeva quel lavoro, ma egli lo vedeva, non voleva intenderlo. Questo, però, non le impediva di ritenerlo molto importante. E poiché sapeva che la loro dimora sarebbe stata in campagna, non desiderava di andare all'estero dove non avrebbe vissuto, ma là dove sarebbe stata la loro dimora. Questa intenzione, espressa in modo definito, sorprese Levin. Ma poiché per lui ciò era indifferente, pregò subito Stepan Arkad'ic, come se questo fosse stato un suo dovere, di andare in campagna e di preparare là tutto quello ch'egli sapeva fare, con quel gusto di cui tanto disponeva. 
\enlargethispage*{1\baselineskip}

- Però senti - disse una volta Stepan Arkad'ic a Levin, tornando dalla campagna dove aveva preparato tutto per l'arrivo degli sposi - ce l'hai un certificato che attesti che ti sei confessato? 

- No. E perché? 

- Senza di questo non ci si può sposare. 

- Ahi, ahi, ahi! - gridò Levin. - Io, vedi, mi pare che sono nove anni che non mi son più comunicato. Non ci pensavo. 

- Bene! - disse ridendo Stepan Arkad'ic - e chiami nichilista me! Non se ne può fare a meno, tuttavia. Ti devi confessare. 

- E quando? Non ci sono che quattro giorni. 

Stepan Arkad'ic accomodò anche questo. Levin cominciò a prepararsi alla comunione. Per Levin, come per qualsiasi essere che non crede, ma che nello stesso tempo rispetta la fede degli altri, la presenza e la partecipazione a qualsiasi rito della Chiesa erano molto incresciose. Adesso, in quello stato in cui era di sensibilità e di intenerimento verso tutto, la necessità di fingere non solo gli era penosa, ma gli sembrava del tutto impossibile. Adesso, in quel suo stato di esaltazione ed effusione avrebbe dovuto o mentire o compiere un sacrilegio. Non si sentiva in grado di fare né l'una né l'altra cosa. Ma per quanto interrogasse Stepan Arkad'ic se fosse possibile ricevere il certificato senza confessarsi, Stepan Arkad'ic dichiarava che non era possibile. 

- Ma, del resto, che cosa ti costa? due giorni. Ed è un vecchietto simpaticissimo, intelligente. Ti caverà questo dente senza fartene accorgere. 

Nell'assistere alla prima messa, Levin si sforzò di ravvivare in sé le reminiscenze giovanili di quel forte sentimento religioso che aveva provato fra i sedici e i diciassette anni. Ma subito si convinse che era assolutamente impossibile. Cercò allora di considerare la cosa come una consuetudine senza senso, vuota, simile a quella di far le visite; ma sentì che neanche così poteva compierla in nessun modo. Levin si trovava, nei riguardi della religione, nella situazione più indefinita, come, del resto, la maggior parte dei suoi contemporanei. Credere non poteva, ma nello stesso tempo non era fermamente convinto che tutto questo fosse falso. E perciò, non essendo in grado di credere al significato di quello che compiva, né di considerarlo con indifferenza come vuota formalità, durante tutto il tempo della preparazione alla comunione, provò un senso di disagio e di vergogna, compiendo cose che lui stesso non intendeva e, perciò, come gli diceva una voce interna, qualcosa da falso e di poco buono. 

Durante il tempo delle funzioni sacre, ora ascoltava le preghiere, cercando di dar loro un significato che non si allontanasse dalle sue opinioni, ora, sentendo di non poterle intendere e di doverle criticare, si sforzava di non ascoltarle, e si occupava dei suoi pensieri, delle osservazioni e dei ricordi che con straordinaria vivezza gli erravano per la testa durante quell'ozioso stare in piedi in chiesa. 

Rimase durante tutta la messa, fino alle preghiere della sera e ai vespri, e l'indomani, alzatosi prima del solito, senza bere il tè, giunse alle otto in chiesa per ascoltare il mattutino e confessarsi. 

In chiesa non c'era nessuno all'infuori di un povero soldato, due vecchiette e i sacrestani. 
\enlargethispage*{1\baselineskip}

Un giovane diacono, con le due metà della schiena marcatamente delineate sotto la tunica sottile, gli venne incontro, e subito avvicinatosi a un tavolino presso il muro, cominciò a leggere le preghiere. A misura che la lettura andava avanti, in particolare alla frequente e veloce ripetizione delle parole: ``Signore, abbi pietà'' che risonavano come ``pietasign'', Levin sentiva che il suo pensiero era chiuso e sigillato e che in quel momento scuoterlo e commuoverlo non conveniva, ché ne sarebbe venuta fuori una confusione; e perciò, rimanendo in piedi dietro al diacono, continuava a pensare a se stesso, senza ascoltare né intendere. ``È sorprendente quanta espressione ci sia nella sua mano'' pensava, ricordando che il giorno prima erano rimasti a sedere accanto al tavolo d'angolo. Non avevano nulla da dirsi, come sempre capitava in quel tempo, e lei, posta una mano sul tavolo, l'aveva aperta e richiusa, e poi s'era messa a ridere da sola, guardandone il movimento. Egli ricordò come avesse baciato la mano e come poi si fosse messo a osservarne le linee convergenti sulla palma rosea: ``Di nuovo pietasign - pensò Levin, segnandosi, inchinandosi e guardando l'agile movimento della schiena del diacono che s'inginocchiava. - Lei poi ha preso la mia mano e ne ha esaminate le linee. Hai una bella mano, ha detto. - Ed egli guardò la propria mano e la mano tozza del diacono. - Ecco, ora finirà presto - pensava. - No, pare che cominci daccapo - pensò, prestando orecchio alle preghiere. - No, finisce; ecco che si è inchinato fino a terra. Questo avviene sempre prima della fine''. 

Il diacono, dopo aver preso, senza farsene accorgere, un biglietto da tre rubli nel manichino di velluto, disse che lo avrebbe iscritto e, facendo risonare sveltamente gli stivali nuovi sulle lastre di pietra della chiesa vuota, si avviò verso l'altare. Dopo un minuto ne uscì fuori e fece cenno a Levin. Il pensiero, fino a quel momento chiuso nella mente di Levin, si agitò, ma egli si affrettò a respingerlo. ``In qualche modo si farà'' pensò e si avviò verso l'ambone. Salì i gradini e, voltando a destra, vide il sacerdote. Un piccolo prete vecchio, con una barba rada, bianca a metà, con gli occhi stanchi, buoni, stava in piedi dinanzi a un leggio e volgeva i fogli del messale. Dopo essersi lievemente inchinato a Levin, cominciò subito a leggere le preghiere con la voce di chi ne ha l'abitudine. Quando finì, s'inchinò fino a terra e rivolse il viso verso Levin. 

- Qui Cristo è invisibilmente presente nell'accogliere la vostra confessione - disse, indicando un Crocifisso. - Credete voi tutto quello che ci insegna la Santa Chiesa Apostolica? - continuò il prete, distogliendo gli occhi dal viso di Levin e incrociando le mani sotto la stola. 

- Io ho dubitato, io dubito di tutto - disse Levin con voce sgradevole a se stesso, e tacque. 

Il sacerdote attese qualche secondo, se mai egli dicesse ancora qualcosa; poi, chiusi gli occhi, nella parlata veloce, in ``o'', di Vladimir, disse: 

- I dubbi sono propri della debolezza umana, ma noi dobbiamo pregare, affinché il Signore misericordioso ci renda forti. Quali peccati particolari avete? - aggiunse senza il più piccolo intervallo, quasi cercando di non perdere tempo. 

- Il mio peccato principale è il dubbio. Io dubito di tutto e mi trovo sempre nel dubbio. 

- Il dubbio è proprio della debolezza umana - ripeté il sacerdote con le stesse parole. - In che cosa, soprattutto, avete dei dubbi? 

- Io dubito di tutto. Dubito, a volte, perfino dell'esistenza di Dio - disse involontariamente Levin, e inorridì della sconvenienza di quel che aveva detto. Ma le parole di Levin non avevano prodotto impressione sul sacerdote, a quanto parve. 

- E quali dubbi vi possono essere sull'esistenza di Dio? - egli disse in fretta, con un sorriso appena percettibile. Levin taceva. 
\enlargethispage*{1\baselineskip}

- E che dubbio potete avere sul Creatore, quando guardate le Sue creazioni? - continuò il sacerdote, con la solita voce affrettata. - E chi ha abbellito di astri la volta celeste? Chi ha rivestito la terra della sua bellezza? Come sarebbe avvenuto tutto ciò senza un Creatore? - disse, guardando interrogativamente Levin. 

Levin sentiva che sarebbe stato sconveniente entrare in una discussione filosofica col sacerdote, e perciò dette in risposta solo quello che si riferiva all'interrogazione. 

- Non so - disse. 

- Non sapete? Ma come potete dubitare che Dio abbia creato tutto? - disse il sacerdote con sorpresa ilare. 

- Io non capisco nulla - disse Levin, diventando rosso e sentendo che le sue parole erano sciocche e non potevano non essere sciocche in una situazione come la sua. 

- Pregate Iddio e chiedete a Lui. Perfino i santi padri hanno avuto dubbi e hanno chiesto a Dio la conferma della loro fede. Il demonio ha un grande potere, e noi non dobbiamo sottometterci a lui. Pregate Iddio, chiedete a Lui. Pregate Iddio - ripeté in fretta. 

Il sacerdote tacque un poco, come se si fosse messo a riflettere. 

- Voi, come ho sentito, vi preparate a contrarre matrimonio con la figlia del mio fedele e figlio spirituale principe Šcerbackij? - aggiunse con un sorriso. - Una bellissima fanciulla! 

- Sì - rispose Levin, arrossendo per il sacerdote. ``Perché gli occorre di domandar questo in confessione?'' pensò. E, quasi in risposta al pensiero di lui, il prete disse: 

- Voi vi preparate a contrarre matrimonio, e Dio, forse, vi concederà una prole, non è vero? Ebbene, quale educazione potete dare ai vostri figli se non vincerete in voi stesso la tentazione del demonio, che vi trascina verso l'incredulità? - disse con rimprovero mite. - Se amate la vostra prole, voi, come buon padre di famiglia, augurerete al vostro figliuolo non solo la ricchezza, il fasto, gli onori; gli augurerete anche la salvezza, l'illuminazione spirituale attraverso la luce della verità. Non è forse così? E che cosa risponderete quando il fanciullo innocente vi chiederà: ``Padre, chi ha fatto tutto quello che mi rallegra in questo mondo, la terra, le acque, il sole, i fiori, le erbe?''. Possibile che gli rispondiate:``Non lo so''? Voi non potete non saperlo, quando, per Sua grande misericordia, il Signore Iddio ve lo ha rivelato. Oppure il figlio vostro vi chiederà: ``Cosa mi aspetta nella vita d'oltretomba?''. Cosa gli direte se non saprete nulla? E come gli risponderete? Lo abbandonerete alle tentazioni del diavolo e del mondo? Questo non è bene - disse, e si fermò, chinando la testa da un lato e guardando Levin con gli occhi miti, benevoli. 

Levin adesso non rispondeva nulla, non perché non volesse entrare in discussione con il sacerdote, ma perché nessuno mai gli aveva fatto tali domande, e prima che i suoi figli gli avessero poste queste domande, ci sarebbe stato ancora tempo per pensare e rispondere. 

- Voi entrate in un periodo della vita - proseguì il sacerdote - in cui bisogna scegliere una via e attenervisi. Pregate Dio, affinché per Sua bontà vi aiuti e abbia pietà di voi - concluse. - Il Signore e Iddio nostro Gesù Cristo, con la grazia divina e la liberalità del Suo amore per gli uomini, ti perdoni, o figlio\ldots{} - e, finita la preghiera di assoluzione, il sacerdote lo benedisse e lo lasciò andare. 

Tornato a casa, quel giorno, Levin provava la sensazione gioiosa che il disagio fosse finito e fosse finito in modo tale da non aver dovuto mentire. Inoltre gli era rimasto il ricordo confuso che ciò che aveva detto quel bravo e caro vecchietto non fosse così sciocco come gli era parso in principio, ma che in esso ci fosse qualcosa che bisognava chiarire. 
\enlargethispage*{1\baselineskip}

``S'intende, non adesso - pensava Levin - ma prima o poi sì''. Levin adesso sentiva più di prima di aver nell'animo qualcosa di confuso e di poco chiaro e che nei rapporti con la religione egli era in quella stessa situazione che così chiaramente scopriva negli altri, che non gli piaceva affatto e che riprovava nell'amico Svijazskij. 

Levin, trascorrendo quella sera con la fidanzata in casa di Dolly, fu particolarmente allegro e, spiegando a Stepan Arkad'ic quel suo stato di eccitamento, disse che si sentiva allegro come un cane al quale abbiano insegnato a saltare attraverso un cerchio e che, avendo capito alla fine e compiuto quel che si pretende da lui, si mette a guaire e, agitando la coda, salta per l'entusiasmo sui tavoli e sulle finestre. 

\capitolo{II}\label{ii-4} 

Il giorno del matrimonio, secondo l'usanza (la principessa e Dar'ja Aleksandrovna insistevano che ci si attenesse alle usanze), Levin non vide la sposa e pranzò nel suo albergo con tre scapoli venuti da lui per caso: Sergej Ivanovic, Katavasov, un compagno di università ora professore di scienze naturali, che Levin, incontrato per strada, si era trascinato a casa, e cirikov, il compare d'anello, giudice di pace a Mosca, compagno di Levin nella caccia all'orso. Il pranzo fu molto allegro. Sergej Ivanovic era di ottimo umore ed era divertito dalla originalità di Katavasov. Katavasov, sentendo che la sua originalità era apprezzata e capita, ne faceva sfoggio. cirikov sosteneva con bonarietà e allegria qualsiasi conversazione. 

- Ecco dunque - diceva Katavasov, strascinando le parole per un'abitudine dovuta all'insegnamento cattedratico - quale ragazzo pieno di possibilità era il nostro amico Konstantin Dmitric! Parlo di assenti, perché lui ormai non c'è più. E amava la scienza allora, finita l'università, e aveva interessi umani; mentre adesso, una parte delle sue abitudini è diretta a ingannare se stesso, e l'altra metà a giustificare questo inganno. 

- Un nemico più deciso di voi del matrimonio non l'ho mai visto - disse Sergej Ivanovic. 

- No, non sono nemico. Sono un amico della suddivisione del lavoro. Le persone che non possono far nulla devono creare gli uomini, ma le altre devono cooperare alla loro educazione e felicità. Così l'intendo io. Un'infinità di persone ama confondere queste due funzioni, io non sono fra queste. 

- Come sarò felice, quando verrò a sapere che vi siete ingannato! - disse Levin. - Per favore invitatemi alle vostre nozze. 

- Io sono già innamorato. 

- Già, della seppia. Sai - disse Levin al fratello - Michail Semënyc scrive un'opera sulla nutrizione e\ldots{} 

- Via, non fate confusioni! Non ha importanza su che cosa. Fatto sta che io amo proprio la seppia. 

- Ma essa non vi impedirà di amare vostra moglie. 

- Lei, no, ma la moglie, sì che me lo impedirà. 

- E perché? 

- Ma ecco, vedrete. Ecco, a voi piace l'azienda domestica, la caccia, ebbene, vedrete! 

- E oggi c'è stato Archip e ha detto che nel Prudnoj c'è un branco di cervi e ci sono due orsi - disse cirikov. 

- Be', li prenderete senza di me. 

- Anche questo è vero - disse Sergej Ivanovic. - D'ora in poi da' un addio alla caccia all'orso. Tua moglie non ti lascerà andare. 

Levin sorrise. L'immagine della moglie, che non gli permetteva di andare, gli era così cara che era pronto a rinunciare per sempre al piacere di vedere gli orsi. 

- Ma è pur peccato che quei due orsi li prendano senza di voi. E vi ricordate a Chapilovo l'ultima volta? Sarà una caccia meravigliosa - disse cirikov. 

Levin non voleva togliersi l'illusione che, senza di lei, ci potesse essere, in una qualche parte, qualcosa di buono, e perciò non disse nulla. 

- Non per nulla si è stabilito l'uso di dire addio alla vita da scapolo - disse Sergej Ivanovic. - Per quanto si possa essere felici si rimpiange sempre la libertà. 

- Ma, dite la verità, non avete la sensazione, come lo sposo di Gogol' d'aver voglia di saltar via dalla finestra? 

- Sicuro che ce l'ha, ma non lo confessa! - disse Katavasov e prese a ridere forte. 

- Ebbene, la finestra è aperta\ldots{} Andiamo subito a Tver'! Uno dei due orsi è una femmina e si può prendere nella tana. Davvero, andiamo col treno delle cinque. E qui, facciano quello che vogliono - disse, sorridendo, cirikov. 

- Ma ecco, in verità di Dio - disse, sorridendo, Levin - non posso trovare nell'anima mia questo senso di rimpianto per la libertà. 

- Ma voi adesso avete nell'animo un tale caos che non ci trovate nulla - disse Katavasov. - Aspettate a raccapezzarvici un po', e poi troverete! 

- No, sentirei, sia pure poco, a parte il mio sentimento - non voleva dire dinanzi a lui ``amore'' - e la felicità mia, che tuttavia mi spiacerebbe perdere la libertà\ldots{} E invece è proprio di questa sottrazione di libertà che sono contento. 

- Male! soggetto senza speranza! - disse Katavasov. - Via, beviamo alla sua guarigione, o auguriamogli soltanto che una centesima parte dei suoi sogni si avveri. E questa sarà una felicità quale non c'è mai stata sulla terra! 

Poco dopo il pranzo gli ospiti se ne andarono per fare in tempo a mutar d'abito per la cerimonia. 

Rimasto solo e riandando ai discorsi di quegli scapoli, Levin si chiese ancora una volta se avesse nell'animo quel senso di rimpianto per la libertà di cui essi avevano parlato. Sorrise a un simile quesito. ``La libertà? e perché la libertà? La felicità è solo nell'amore e nel desiderare, nel pensare con i suoi pensieri e con i suoi desideri, cioè nessuna libertà, questa è la felicità''. 

``Ma conosco io forse i suoi pensieri, i suoi desideri, i suoi sentimenti?'' gli mormorò all'improvviso chi sa quale voce. Il sorriso gli sparì dal volto, ed egli si fece pensieroso. E a un tratto lo afferrò una strana sensazione. Lo afferrò il terrore e il dubbio, il dubbio di tutto. 

``E se non mi ama? E se mi sposa solo per prender marito? Se lei stessa non sa quello che fa? - si chiedeva. - Potrebbe ravvedersi e, appena sposata, capire che non mi ama e che non può amarmi''. E cominciarono a venirgli i pensieri più strani su di lei, i pensieri peggiori. Era geloso di Vronskij come un anno prima, come se quella sera in cui l'aveva vista con Vronskij fosse stata la sera precedente. Sospettava ch'ella non gli avesse detto tutto. 

Saltò su in fretta. ``No, così non si può! - disse a se stesso disperato. - Andrò da lei le chiederò, le dirò per l'ultima volta: `noi siamo liberi, non è meglio fermarsi? Tutto è preferibile a una infelicità continua, all'infamia, all'infedeltà!'\,''. Con la disperazione nell'animo e con un senso di rancore verso tutti, verso se stesso, verso di lei, uscì dall'albergo e si diresse verso casa Šcerbackij. 

La trovò nelle stanze interne. Sedeva su di un baule e dava disposizioni a una donna, scegliendo fra mucchi di abiti di vario colore distribuiti sulle spalliere delle seggiole e sul pavimento. 

- Ah! - gridò nel vederlo e si illuminò di gioia. - Come tu, come voi? - fino a quell'ultimo giorno gli parlava ora col tu ora col voi. - Ecco, non me l'aspettavo! E io sto scegliendo i miei vestiti da ragazza, a chi questo\ldots{} 

- Ah, questo è molto bello! - disse lui guardando torvo la donna. 

- Va' via, Dunjaša, sonerò poi - disse Kitty. - Che hai? - domandò, dandogli decisamente del tu, appena la donna fu uscita. Ella aveva notato il viso di lui, agitato e cupo, e il terrore l'aveva afferrata. 

- Kitty, mi tormento. Non posso tormentarmi solo - disse con la disperazione nella voce, fermandosi dinanzi a lei e guardandola supplichevole negli occhi. Egli vedeva già dal viso di lei pieno di amore, di sincerità, che nulla poteva venir fuori da quel che s'era proposto di dire, tuttavia gli era necessario ch'ella stessa lo dissuadesse. - Sono venuto a dirti che siamo ancora in tempo. Tutto questo si può annientare e riparare. 

- Che cosa? Non capisco nulla. Che cosa ti è successo? 

- Quello che ho detto mille volte e che non posso non pensare\ldots{} che io non merito te. Tu non puoi acconsentire a sposarmi. Pensaci. Ti sei ingannata. Pensaci proprio bene. Tu non puoi amarmi\ldots{} Se\ldots{} dimmelo piuttosto - disse senza guardarla. - Sarò infelice. Lascia che tutti dicano quello che vogliono; tutto è preferibile alla infelicità\ldots{} Tutto è meglio ora, finché siamo in tempo\ldots{} 

- Non capisco - rispondeva lei, spaventata - sarebbe allora che tu vuoi rinunciare\ldots{} che non si deve? 

- E già, se tu non mi ami. 

- Ma sei impazzito! - gridò lei, avvampando di stizza. Ma il viso di lui era così pietoso ch'ella trattenne la stizza e, gettati via gli abiti dalla poltrona, cambiò di posto, sedendosi accanto a lui. - Cosa pensi? dimmi tutto. 

- Io penso che tu non puoi amarmi. Perché dovresti amarmi? 

- Dio mio! cosa posso mai\ldots{} - disse lei, e cominciò a piangere. 

- Ah, che ho fatto! - egli gridò e, postosi in ginocchio dinanzi a lei, cominciò a baciarle le mani. 

Quando la principessa, cinque minuti dopo, entrò nella stanza, li trovò completamente rappacificati. Kitty non solo lo aveva rassicurato di amarlo, ma gli aveva persino spiegato perché lo amava, rispondendo alla sua domanda. Gli aveva detto che lo amava perché lo capiva interamente, perché sapeva che cosa gli doveva piacere, e che tutto quello che piaceva a lui, era bene. E questo a lui parve del tutto chiaro. Quando la principessa entrò, sedevano l'uno accanto all'altra sul baule, scegliendo gli abiti e discutendo sul fatto che Kitty voleva dare a Dunjaša quel vestito marrone che aveva indosso quando Levin le aveva fatto la sua proposta di matrimonio, e lui insisteva perché non fosse dato ad alcuno e diceva che a Dunjaša si poteva dare quello azzurro. 

- Ma come, non capisci? Lei è bruna e non le starà bene\ldots{} Io ho tutto calcolato\ldots{} 

Avendo saputo perché era venuto, la principessa, un po' per scherzo e un po' sul serio, si arrabbiò e lo mandò a casa a vestirsi, invece di impedire a Kitty di pettinarsi, ora che Charles stava per arrivare. 

- Anche così non mangia niente, in questi giorni, e si è fatta brutta, e tu vieni ancora a sconvolgerla con le tue sciocchezze - gli disse. - Vattene via, caro. 

Levin, colpevole e vergognoso, ma rasserenato, tornò in albergo. Suo fratello, Dar'ja Aleksandrovna e Stepan Arkad'ic, tutti in gran gala, lo aspettavano già per benedirlo con l'icona. Non c'era tempo da perdere. Dar'ja Aleksandrovna doveva ancora passare da casa per prendere un figliuolo tutto impomatato e arricciato, che doveva portare l'icona insieme con la fidanzata. Poi una carrozza bisognava mandarla a rilevare il compare d'anello, e l'altra, che avrebbe portato Sergej Ivanovic, bisognava mandarla indietro\ldots{} Insomma, di considerazioni molto complicate ce n'erano tante. Una cosa era fuor di dubbio, che non si poteva indugiare, perché erano già le sei. 

La benedizione con l'icona non riuscì per niente. Stepan Arkad'ic si mise in una posa comicamente solenne accanto alla moglie, prese l'icona e, ordinato a Levin di inchinarsi fino a terra, lo benedisse con il suo sorriso buono e canzonatorio e lo baciò tre volte; lo stesso fece Dar'ja Aleksandrovna e subito si affrettò ad andar via e si confuse nel dare le disposizioni per le carrozze. 

- Su, allora, ecco cosa faremo: tu vai a prender lui con la nostra carrozza, e se poi Sergej Ivanovic fosse così buono di passarlo a prendere e rimandare la vettura indietro\ldots{} 

- Ma certo, sono molto contento. 

- E noi verremo subito con lui. La roba è stata spedita? - chiese Stepan Arkad'ic. 

- È stata spedita - rispose Levin, e ordinò a Kuz'ma di aiutarlo a vestirsi. 

\capitolo{III}\label{iii-4} 

Una folla di gente, in gran parte femminile, inondava la chiesa illuminata per il matrimonio. Le signore che non avevano fatto in tempo a ficcarsi dentro, si affollavano intorno alle finestre, urtandosi, discutendo e guardando attraverso le grate. 

Più di venti carrozze erano già state disposte lungo la via dai gendarmi. Un ufficiale di polizia, senza curarsi del gelo, stava dinanzi all'ingresso, splendente nella sua uniforme. Le carrozze si avvicinavano ininterrottamente, e ora signore con fiori e con gli strascichi sollevati, ora uomini che si toglievano il chepì o il cappello nero, entravano in chiesa. Nell'interno erano già accesi i due candelabri e tutte le candele dinanzi alle immagini. Lo splendore dell'oro sul fondo rosso dell'iconostasi, l'intaglio dorato delle icone e l'argento dei candelabri e dei candelieri, le lastre del pavimento, gli arazzi e gli stendardi al di sopra e accanto ai cori, gli scalini dell'ambone, e i vecchi libri anneriti, le stole e le cotte, tutto era inondato di luce. Nella parte destra della chiesa riscaldata, in un mare di frac, cravatte bianche e divise, di seta, velluto, raso, acconciature e fiori, di spalle, braccia nude e guanti lunghi, si svolgeva una conversazione animata che risonava stranamente nell'alto della cupola. Ogni volta che strideva la porta nell'aprirsi, la conversazione si quietava, e tutti guardavano, aspettandosi di vedere entrare lo sposo e la sposa. Ma la porta era stata aperta più di dieci volte, e ogni volta era un invitato o un'invitata in ritardo che si univa alla cerchia degli altri a destra, o una spettatrice che, ingannato o commosso l'ufficiale di polizia, si univa alla folla estranea a sinistra. E i parenti e gli estranei erano passati attraverso tutte le fasi dell'attesa. 

In principio avevano creduto che lo sposo e la sposa sarebbero venuti subito, e non avevano dato alcun significato a questo ritardo. Poi avevano cominciato a guardare la porta sempre più spesso, domandandosi di tanto in tanto se non fosse accaduto qualcosa. Infine, questo ritardo divenne increscioso, e i parenti e gli invitati cercarono di non apparire preoccupati degli sposi e di essere presi dalla loro conversazione. 

Il protodiacono, come per ricordare che il suo tempo era prezioso, tossicchiava con impazienza, facendo tremare i vetri delle finestre. Sul coro si sentivano le voci che provavano e i cantori che, presi dalla noia, si soffiavano il naso. Il sacerdote mandava ogni momento il sacrestano o il diacono a vedere se arrivava lo sposo, e lui stesso, con la tunica viola e la cintura ricamata, si affacciava sempre più spesso alla porta laterale in attesa dello sposo. Finalmente una delle signore, guardando l'ora, disse: ``ma è proprio strano!'' e tutti gli invitati si misero in agitazione e cominciarono a esprimere ad alta voce la propria sorpresa e il proprio scontento. Uno dei testimoni andò ad informarsi che cosa era successo. Kitty, intanto, da tempo già completamente pronta, con l'abito bianco, il velo lungo e la corona di fiori di arancio, stava dritta nella sala di casa Šcerbackij, con la madrina e la sorella L'vova, e guardava dalla finestra, aspettando invano, già da più di mezz'ora, notizie dal suo testimone sull'arrivo dello sposo in chiesa. 

E Levin intanto, coi pantaloni, ma senza il panciotto e il frac, camminava su e giù per la camera d'albergo, affacciandosi continuamente alla porta del corridoio e guardando. Ma nel corridoio non compariva la persona ch'egli aspettava e, tornando indietro disperato e agitando le braccia, si rivolgeva a Stepan Arkad'ic che fumava tranquillamente. 

- C'è mai stato un uomo in una posizione così terribilmente stupida? 

- Sì, è sciocco - confermò Stepan Arkad'ic, sorridendo teneramente. - Ma calmati, porteranno subito. 

- No, ma come! - diceva con rabbia trattenuta Levin. - E questi stupidi panciotti aperti! È impossibile! - diceva, guardando lo sparato sgualcito della propria camicia. - E che succederà se hanno già portato la roba alla ferrovia! - gridò con disperazione. 

- Allora ne metterai una mia. 

- E bisognava far così da un pezzo. 

- Ma non sta bene esser ridicoli\ldots{} Aspetta, ``s'appianerà''. 

Il fatto era che, quando Levin aveva chiesto di vestirsi, Kuz'ma, il vecchio servo di Levin, aveva portato il frac, il panciotto e tutto quello che occorreva. 

- E la camicia? - aveva detto Levin. 

- La camicia l'avete indosso - aveva risposto Kuz'ma con un sorriso tranquillo. 

Una camicia pulita Kuz'ma non aveva pensato a lasciarla e, avuto l'ordine di prendere e portare tutto in casa Šcerbackij, dalla quale la sera sarebbero partiti gli sposi, aveva messo tutto dentro, tranne il frac. La camicia, indossata fin dalla mattina, era sgualcita e impossibile a mettersi con la moda dei panciotti aperti. Mandar dagli Šcerbackij era lontano. Avevano mandato a comprare una camicia. Il servitore era tornato indietro: era tutto chiuso, perché era domenica. Avevano mandato da Stepan Arkad'ic e avevano portato una camicia; ma era stretta e corta in modo impossibile. Avevano, infine, mandato dagli Šcerbackij a disfare i bagagli. Si aspettava lo sposo in chiesa e lui camminava per la stanza come una belva in gabbia, guardando fuori nel corridoio e ricordando con terrore e disperazione quel che aveva detto a Kitty e quello che ella poteva pensare, adesso, di lui. 

Alla fine Kuz'ma, il colpevole, respirando a stento, irruppe nella stanza con la camicia. 

- Li ho appena trovati. Caricavano già tutto sul carro - disse Kuz'ma. 

Dopo tre minuti, senza guardare l'ora, per non inasprire le ferite, Levin correva per il corridoio. 

- Tanto non ci fai nulla - diceva Stepan Arkad'ic con un sorriso, tenendogli dietro senza furia. - ``S'appianerà, s'appianerà''\ldots{} ti dico. 

\capitolo{IV}\label{iv-4} 

- Sono arrivati! Eccolo! Ma qual'è? Quello più giovane, eh? E lei, matuška, più morta che viva! - si cominciò a dire tra la folla quando Levin, incontrata la sposa all'ingresso, entrò in chiesa insieme con lei. 

Stepan Arkad'ic raccontò alla moglie la causa del ritardo e gli invitati parlottarono fra di loro, sorridendo. Levin non notava nulla e nessuno; senza abbassar gli occhi, guardava la sposa. 

Tutti dicevano che, in quegli ultimi giorni, lei s'era sciupata e che con l'acconciatura da sposa era molto meno carina del solito; ma Levin non trovava questo. Guardava quella sua pettinatura alta con il lungo velo bianco e i fiori bianchi, il colletto alto a pieghe, che in modo così verginale chiudeva di lato e scopriva davanti il collo lungo e la vita meravigliosamente sottile, e a lui sembrava ch'ella stesse come non mai, non perché quei fiori, quel velo, quel vestito ordinato a Parigi aggiungessero qualcosa alla sua bellezza, ma perché, malgrado il fasto dell'acconciatura, l'espressione del viso gentile, dello sguardo, delle labbra, era sempre quella stessa espressione di innocente sincerità, tutta sua. 

- Credevo che già te ne volessi fuggire - disse lei, sorridendogli. 

- È così sciocco quello che m'è successo che fa vergogna a dirlo - rispose Levin arrossendo, e fu costretto a voltarsi verso Sergej Ivanovic che gli si era avvicinato. 

- Carina la tua storia della camicia! - disse Sergej Ivanovic, scotendo la testa e sorridendo. 

- Già - rispose Levin senza capire di che cosa gli parlassero. 

- Eh, via, Kostja, ora bisogna decidere - disse Stepan Arkad'ic con un'aria di finto spavento - è una questione seria. Tu ora sei proprio in grado di capirne tutta l'importanza. Mi hanno chiesto: si devono accendere le candele bruciate o quelle non bruciate? La differenza è di dieci rubli - aggiunse, atteggiando le labbra a un sorriso. - Ho deciso io, ma temo che tu non me ne dia l'approvazione. 

Levin capì che si trattava di uno scherzo, ma non riuscì a sorridere. 

- E allora, come? quelle bruciate o quelle non bruciate? qui sta la questione. 

- Sì, sì, quelle non bruciate. 

- Ah, sono proprio contento! La questione è decisa! - disse Stepan Arkad'ic, sorridendo. - Ma come ci si istupidisce in questa situazione! - disse rivolto a cirikov, mentre Levin, dopo averlo guardato con aria smarrita, si era avvicinato alla sposa. 

- Attenta, Kitty, metti per prima il piede sul tappeto - disse la contessa Nordston, avvicinandosi. - Come state bene! - disse rivolta a Levin. 

- Be', niente paura? - chiese Mar'ja Dmitrievna, una vecchia zia. 

- Non hai freddo? Sei pallida. Aspetta, fermati! - disse la sorella di Kitty, la L'vova, e, disponendo a cerchio le braccia piene, bellissime, con un sorriso le acconciò i fiori sul capo. 

Dolly si avvicinò, voleva dire qualcosa ma non riuscì a parlare, si mise a piangere e poi a ridere forzatamente. 

Kitty guardava tutti con i suoi occhi assenti, come Levin. A tutti i discorsi rivolti a lei poteva rispondere solo con un sorriso di gioia, che in lei, in quel momento, era del tutto naturale. 

Intanto i sacerdoti avevano di nuovo indossato i paramenti e il prete e il diacono si erano diretti verso il leggio posto nel vestibolo della chiesa. Il sacerdote si rivolse a Levin, dicendogli qualcosa. Levin non sentì quel che il sacerdote gli aveva detto. 

- Prendete per mano la sposa e conducetela - disse il compare d'anello a Levin. 

Per un pezzo Levin non capì che cosa si volesse da lui. Per un pezzo lo corressero e stavano già per desistere - infatti o non la prendeva con la mano giusta o le prendeva quella che non doveva - quando finalmente capì che la doveva prendere con la sua destra, senza cambiar posto, proprio per la destra. Quando finalmente ebbe presa la sposa con la mano destra, così come si doveva, il sacerdote fece alcuni passi in avanti e si fermò dinanzi al leggio. La folla dei parenti e degli amici, con un fruscio e ronzio di strascichi e discorsi, fece ressa dietro di loro. Qualcuno, chinatosi, acconciò il velo della sposa. Nella chiesa s'era fatto un silenzio tale che si sentiva gocciolar la cera. 

Il sacerdote, un vecchietto con la cotta, le ciocche di capelli lucide d'argento divise in due parti dietro le orecchie, liberate le piccole mani di vecchio sotto la pesante pianeta d'argento dalla croce dorata sulla schiena, sfogliava qualcosa sul leggio. 

Stepan Arkad'ic si avvicinò cauto, sussurrò qualcosa e, dopo aver ammiccato a Levin, si fece di nuovo indietro. 

Il sacerdote accese due ceri ornati di fiori e, tenendoli inclinati nella mano sinistra così che la cera ne gocciolava, si voltò con il viso verso gli sposi. Era lo stesso prete che aveva confessato Levin. Guardò con uno sguardo stanco e triste lo sposo e la sposa e, liberata di sotto la pianeta la mano destra, benedisse lo sposo e nello stesso modo, ma con una sfumatura di cauta tenerezza, impose le dita ripiegate sul capo chino di Kitty. Poi dette loro i ceri e, preso il turibolo, si allontanò. 

``Possibile che sia vero?'' pensò Levin e si voltò a guardare la sposa. Vedeva, un po' dall'alto, il profilo di lei e, dal moto appena percettibile delle labbra e delle ciglia, sapeva ch'ella sentiva il suo sguardo. Ella non si voltò, ma l'alto colletto a pieghe si mosse, sollevandosi, verso il piccolo orecchio rosa. Egli vedeva che il respiro si era fermato nel petto di lei e che, nel guanto lungo, la piccola mano che reggeva il cero aveva cominciato a tremare. 

Tutta la sua agitazione per la camicia e per il ritardo, la conversazione con gli amici e i parenti, il loro disappunto, la sua situazione ridicola, tutto scomparve a un tratto, ed egli provò gioia e sgomento. 

Un bel protodiacono alto, in dalmatica d'argento, con i ricci ondulati spartiti al centro, si avanzò con sicurezza e, sollevato su due dita, con gesto consueto, il manipolo, si fermò di fronte al prete. 

``Be-ne-di-ci Si-gno-re!'' echeggiarono lente le note solenni, l'una dietro l'altra, facendo oscillare delle onde d'aria. 

``Sia benedetto Iddio nostro sempre, adesso e ognora e nei secoli dei secoli'' rispose umilmente il vecchietto seguitando a sfogliare qualcosa sul leggio. E, effondendosi per tutta la chiesa, dalle finestre fino alla volta, l'accordo pieno del coro invisibile dei chierici si levò ampio e armonioso, si rafforzò, si fermò per un attimo e si spense piano. 

Pregavano, come del resto sempre, per la pace suprema e per la salvezza, per il Sinodo e per lo zar; pregavano in questo momento anche per il servo di Dio Konstantin e per Ekaterina che si univano in matrimonio. 

``Perché sia mandato loro l'amore perfetto, la pace e l'aiuto, preghiamo il Signore'' pareva respirare tutta la chiesa nella voce del protodiacono. 

Levin ascoltava le parole e queste lo stupivano. 

``Come hanno indovinato; che cos'è mai l'aiuto, l'aiuto? - pensava ricordando tutti i suoi recenti timori e dubbi. - Che cosa so io? che cosa posso in questa terribile cosa - pensava - senza aiuto? Proprio di aiuto ho bisogno ora''. 

Quando il diacono ebbe finito la preghiera, il sacerdote si rivolse agli sposi con il libro. ``Iddio eterno, Tu che hai congiunto quelli che erano lontani - leggeva con voce mite, intonata - e che hai stabilito un'unione d'amore indistruttibile; Tu che hai benedetto Isacco e Rebecca, che hai mostrato ai loro discendenti la Tua promessa: benedici Tu stesso i Tuoi servi Konstantin ed Ekaterina, indirizzandoli verso ogni opera di bene. Poiché misericordioso e pieno d'amore sei, o Dio, a Te la gloria innalziamo, al Padre, al Figlio e allo Spirito Santo, ora e sempre, e nei secoli''. ``Amen'' risonò di nuovo nell'aria l'invisibile coro. 

``\,`Che hai congiunto quelli che erano lontani e hai stabilito un'unione d'amore': come sono profonde queste parole e come corrispondono a quello che io sento in questo momento! - pensava Levin. - Sente anche lei come me?''. E voltatosi a guardarla, incontrò lo sguardo di lei. 

E da questo sguardo, egli concluse ch'ella sentiva così come lui. Ma non era vero; ella non intendeva quasi nulla delle parole del servizio divino e non le ascoltava neppure, durante la funzione. Non voleva sentirle, né intenderle; tanto forte era quell'unico sentimento che le invadeva l'anima e diveniva sempre più forte. Questo sentimento era la gioia del pieno compiersi di ciò che da un mese e mezzo si era compiuto nell'animo suo e che durante quelle sei settimane l'aveva rallegrata e tormentata. Nell'animo suo, in quel giorno in cui, in abito marrone, nella sala della casa sull'Arbat, si era avvicinata a lui in silenzio e gli si era data, nell'animo suo, in quel giorno e in quell'ora si era compiuto un completo distacco da tutta la sua vita di prima ed era cominciata, pur continuando in realtà l'antica, un'altra vita, completamente nuova, completamente sconosciuta. Queste sei settimane erano state il periodo più beato e più tormentato per lei. Tutta la sua vita, i suoi desideri, le sue speranze si erano concentrati in quel solo uomo, per lei ancora incomprensibile, al quale la legava un sentimento ancora più incomprensibile dell'uomo stesso, che ora avvicinava, che ora respingeva; e nello tesso tempo ella aveva continuato a vivere nelle stesse condizioni di vita di prima. Vivendo la sua vita di prima, aveva orrore di sé, della sua completa, insormontabile indifferenza verso tutto il suo passato: verso le cose e le abitudini, verso le persone che le avevano voluto bene e gliene volevano, verso la madre rammaricata di quell'indifferenza, verso il padre tenero e caro, fino allora amato più di tutti al mondo. Un momento aveva orrore di quell'indifferenza, un momento si rallegrava di quello che ve l'aveva condotta. Non poteva più pensare, né desiderare altro che la vita con quell'uomo; ma questa nuova vita non c'era ancora, ed ella non riusciva neppure a immaginarsela con chiarezza. Non c'era che attesa: lo sgomento e la gioia del nuovo e dell'ignoto. Ed ecco, da un momento all'altro, l'attesa e l'ignoto, e il rimorso di aver rinunciato alla vita di prima: tutto sarebbe finito e sarebbe cominciato qualcosa di nuovo. Questo qualcosa di nuovo non poteva non essere terribile per la sua incertezza; ma, per quanto pauroso fosse, s'era già compiuto sei settimane prima nell'animo suo, e adesso si santificava soltanto quello che già da tempo era avvenuto nell'animo suo. 

Voltosi di nuovo verso il leggio, il sacerdote afferrò con difficoltà il piccolo anello di Kitty e, chiesta la mano di Levin, glielo infilò nella prima falange del dito. ``Si sposa il servo di Dio Konstantin con la serva di Dio Ekaterina''. E, infilato l'anello grande nel dito piccolo e roseo, commovente di fragilità, di Kitty, il sacerdote pronunciò le stesse parole. 

Varie volte gli sposi cercarono di indovinare che cosa si dovesse fare, e ogni volta sbagliarono, e il prete li corresse sottovoce. Finalmente, fatto quello che occorreva, dopo averli segnati d'un segno di croce con gli anelli, il sacerdote consegnò di nuovo a Kitty quello grande e a Levin quello piccolo; di nuovo essi si confusero e due volte fecero passare l'anello da una mano all'altra senza che ne venisse fuori quello che si richiedeva. 

Dolly, cirikov e Stepan Arkad'ic si fecero avanti per correggerli. Si produssero confusione, bisbigli e sorrisi, ma l'espressione solennemente commossa degli sposi non mutò; al contrario, confondendo le mani, essi apparvero ancor più seri e solenni, e il sorriso col quale Stepan Arkad'ic sussurrò che ognuno infilasse il proprio anello, si spense involontariamente sulle sue labbra. Sentiva che qualsiasi sorriso li avrebbe offesi. 

``Poiché Tu dall'origine hai creato l'uomo e la donna - leggeva il sacerdote dopo lo scambio degli anelli - e da Te è congiunta la moglie al marito per la provvidenza e la procreazione del genere umano. Così Tu stesso, o Signore Iddio nostro, che hai inviato la verità alla Tua discendenza e la Tua promessa ai servi Tuoi, padri nostri, per generazioni e generazioni Tuoi eletti, guarda il servo Tuo Konstantin e la serva Tua Ekaterina, e conferma le nozze loro nella fede e nella concordia, nella verità e nell'amore\ldots{}''. 

Levin sentiva sempre più che tutte le sue idee sul matrimonio, i suoi sogni su come avrebbe costruito la sua vita, erano fanciullaggini, e che questo era qualcosa che lui non aveva finora inteso e che in quel momento ancor meno intendeva, sebbene si compisse in lui; un fremito sempre più alto gli sollevava il petto e le lacrime indocili gli venivano agli occhi. 

\capitolo{V}\label{v-4} 

Nella chiesa c'era tutta Mosca, tra parenti e amici. E durante il rito nuziale, nella chiesa illuminata a giorno, fra le donne adornate, le fanciulle e gli uomini in frac e cravatta bianca e in uniforme, non veniva mai meno un discorrere convenientemente sommesso, tenuto vivo soprattutto dagli uomini, mentre le signore erano prese dallo studio di tutti i particolari della cerimonia che sempre le commuove tanto. 

Nel circolo più vicino alla sposa c'erano le due sorelle: Dolly e la maggiore, la L'vova, donna calma e bella, giunta dall'estero. 

- Come mai Marie è in quel viola quasi nero, a un matrimonio? - chiedeva la Korsunskaja. 

- Con quel suo colorito è l'unica salvezza\ldots{} - rispondeva la Drubeckaja. - Mi sorprendo come mai abbiano fatto il matrimonio di pomeriggio. È da mercanti\ldots{} 

- È più bello. Anch'io mi sono sposata di sera - rispondeva la Korsunskaja e sospirò, ricordando come fosse graziosa quel giorno, come risibilmente innamorato di lei fosse suo marito e come ora tutto fosse diverso. 

- Dicono che non si sposa chi fa da compare d'anello più di dieci volte, e io volevo farlo per la decima volta per acquietarmi, ma il posto era occupato - diceva il conte Sinjavin alla graziosa principessa carskaja, che aveva delle mire su di lui. 

La carskaja gli rispondeva soltanto con un sorriso. Guardava Kitty e pensava a come e quando si sarebbe trovata lei al posto di Kitty, in piedi con il conte Sinjavin, e come allora gli avrebbe ricordato lo scherzo di oggi. 

Šcerbackij diceva alla vecchia damigella d'onore Nikolaevna che aveva intenzione di porre la corona nuziale sullo chignon di Kitty perché fosse felice. 

- Non occorreva mettersi lo chignon - rispondeva la Nikolaevna, la quale aveva da lungo tempo deciso che, se il vecchio vedovo che stava pescando l'avesse sposata, il matrimonio si sarebbe svolto nel modo più semplice. - Io non amo questo fasto. 

Sergej Ivanovic parlava con Dar'ja Dmitrievna, sostenendo per scherzo che l'usanza di partire dopo il matrimonio è diffusa perché gli sposi novelli si vergognano sempre un poco. 

- Vostro fratello può essere orgoglioso. È un miracolo, tanto è carina. Lo invidiate, penso. 

- Sono già passato attraverso questo, Dar'ja Dmitrievna - egli rispose, e il suo viso improvvisamente prese un'espressione seria e triste. 

Stepan Arkad'ic raccontava alla cognata il suo giuoco di parole sul divorzio. 

- Bisogna accomodare la corona - rispondeva lei, senza ascoltarlo. 

- Che peccato che sia così sciupata! - diceva la contessa Nordston alla L'vova. - Eppure lui non vale neanche un suo dito, non è vero? 

- No, mi piace tanto. Non perché sia il mio futuro beau-frère - rispondeva la L'vova - ma come si comporta bene! Ed è così difficile comportarsi bene in questa circostanza, non essere ridicoli. E lui non è ridicolo, non è impacciato, si vede che è commosso. 

- Mi pare che la cosa fosse attesa. 

- Eh, sì. Lei l'ha sempre amato. 

- Via, guardiamo chi dei due mette prima il piede sul tappeto. Io l'ho detto a Kitty. 

- Fa lo stesso - rispondeva la L'vova. - Noi siamo tutte mogli docili, l'abbiamo nel sangue. 

- E io invece mi ci son messa per prima con Vasilij. E voi, Dolly? 

Dolly stava in piedi accanto a loro, le ascoltava, ma non rispondeva. Era commossa. Aveva le lacrime agli occhi e non avrebbe potuto dir nulla senza mettersi a piangere. Era felice per Kitty e per Levin; ritornando col pensiero al suo matrimonio, guardò Stepan Arkad'ic, sempre raggiante, dimenticò tutto il presente e ricordò solo il suo primo amore innocente. Rammentò non solo se stessa, ma tutte le donne a lei vicine e note; le ricordò in quell'unico momento solenne per loro, quando così come ora Kitty, stavano sotto la corona nuziale con l'amore, la speranza e l'ansia nel cuore, rinunciando al passato ed entrando in un futuro misterioso. Fra tutte le spose che le tornarono in mente, ricordò anche Anna, a lei cara, e a proposito della quale, da non molto, aveva sentito parlare di divorzio. Anche lei, egualmente pura, era stata lì, in piedi, con i fiori d'arancio e il velo. E ora? ``Terribilmente strano'' si disse. 

Non solo le sorelle, le amiche e i parenti seguivano tutti i particolari della funzione; ma le donne estranee, le spettatrici, con un'emozione che spezzava loro il respiro, temendo di perdere ogni movimento, seguivano l'espressione del viso dello sposo e della sposa, e con irritazione non rispondevano, e spesso non ascoltavano neppure, i discorsi indifferenti degli uomini, che facevano osservazioni scherzose o estranee. 

- Come mai ha pianto tanto? Si sposa forse controvoglia? 

- E come controvoglia, con un giovane così bello! Un principe, vero? 

- È la sorella questa in raso bianco? Su, senti come urla il diacono: ``che tema suo marito''. 

- Sono di cudovo? 

- Sono del Sinodo. 

- Ho fatto parlare il servitore. Dice che lui la porta subito nelle sue proprietà. È tanto ricco, dicono. Perciò gliel'hanno data. 

- No, la coppia è bella. 

- Ed ecco voi, Mar'ja Vlas'evna, dicevate che le carnaline non si portano staccate. Guarda un po' quella vestita color pulce, dicono che sia un'ambasciatrice, con quel risvolto\ldots{} Così, di nuovo così. 

- E come è carina la sposa, guarnita come un'agnellina! E qualunque cosa diciate, fa sempre pena. 

Così si parlottava fra la folla delle spettatrici che erano riuscite a varcare la porta della chiesa. 

\capitolo{VI}\label{vi-4} 

Quando la funzione terminò, un chierico distese dinanzi al leggio, nel centro della chiesa, un pezzo di seta rosa, il coro si mise a cantare un salmo difficile e complesso, nel quale il basso e il tenore si rispondevano tra di loro, e il sacerdote, voltatosi, indicò agli sposi il pezzo di seta rosa disteso. Per quanto tutti e due avessero sentito parlare spesso della superstizione che, chi per primo mette il piede sul tappeto, quegli diviene il capo della famiglia, né Levin né Kitty riuscirono a ricordarsene, quando fecero quei pochi passi. Non sentirono neppure le osservazioni fatte ad alta voce, né le discussioni sul fatto che, secondo quanto avevano osservato alcuni, lui ci si era messo per primo, e quanto ad altri invece, tutti e due insieme. 

Dopo le solite domande sul desiderio di contrarre matrimonio e di non essere promessi ad altri, e dopo le risposte che risonarono strane a loro stessi, cominciò una nuova funzione. Kitty ascoltava le parole della preghiera, desiderando intenderne il senso, ma non poteva. Un sentimento di festosità e di gioia luminosa, a misura che il rito si compiva, invadeva sempre più l'animo suo e le toglieva la possibilità di raccogliersi. 

Si pregava ``perché fosse loro donata la purezza, e il frutto delle viscere per il loro bene, perché si rallegrassero della vista dei figli e delle figlie''. Si ricordava anche che Iddio aveva creato la donna dalla costola di Adamo, e che ``per questo l'uomo lascerà il padre e la madre e si unirà con la moglie, ed essi saranno due in una sola carne'' e ``che questo è un grande mistero''; si chiedeva ``che Dio concedesse loro fecondità e benedizione, come a Isacco e a Rebecca, a Giuseppe, a Mosè e a Sefora, e che essi giungessero a vedere i figli dei loro figli''. ``Tutto questo è molto bello - pensava Kitty ascoltando le parole - tutto questo non potrebbe essere diversamente'' e un sorriso di gioia, che si comunicava involontariamente a tutti quelli che la guardavano, le splendeva sul viso luminoso. 

- Mettetela per bene! - risonarono i consigli nel momento in cui il sacerdote impose loro le corone, e Šcerbackij, con la mano tremante nel guanto a tre bottoni, tenne la corona in alto, sulla testa di lei. 

- Mettetemela! - mormorò lei, sorridendo. 

Levin si voltò a guardarla, e fu sorpreso dello splendore gioioso del viso di lei; e questo sentimento gli si comunicò. Divenne, come lei, luminoso e allegro. 

Si rallegrarono di ascoltare la lettura dell'epistola dell'apostolo e l'eco della voce del protodiacono all'ultimo versetto, atteso con tanta impazienza dal pubblico estraneo. Si rallegrarono di bere nella tazza dalla forma schiacciata il vino rosso, tiepido, unito all'acqua, ed ancor più si rallegrarono quando il sacerdote, toltasi la pianeta e prese nella sua le loro mani, li condusse, fra gli slanci del basso che emetteva l'``Isaia giubila!'', accanto al leggio. Šcerbackij e cirikov, che sostenevano le corone, impigliandosi nello strascico della sposa, anch'essi sorridenti e come rallegrati da qualcosa, ora si fermavano, ora urtavano contro gli sposi, alle soste del sacerdote. La scintilla di gioia che si era accesa in Kitty sembrava essersi comunicata a tutti quelli che erano in chiesa. A Levin sembrava che pure il sacerdote e il diacono avessero voglia di sorridere, così come lui. 

Dopo aver tolto le corone dalle teste, il sacerdote finì di leggere l'ultima preghiera e si rallegrò con gli sposi. Levin guardò Kitty e fino a quel momento non l'aveva mai vista così. Ella era deliziosa per quella nuova luce di felicità che era nel suo viso. Levin avrebbe voluto dire qualcosa, ma non sapeva se tutto era finito, oppure no. Il sacerdote lo tolse d'impaccio. Sorrise con la sua bocca da buono e disse piano: ``Baciate vostra moglie, e voi baciate vostro marito'' e tolse loro di mano i ceri. 

Levin baciò lievemente le labbra sorridenti di lei, le offrì il braccio e, provando la sensazione di una nuova strana vicinanza, uscì dalla chiesa. Non credeva, non poteva credere che fosse vero. Soltanto quando i loro sguardi timidi e attoniti si incontrarono, credette, perché sentì che ormai erano una cosa sola. 

Dopo il pranzo, quella notte stessa, i giovani sposi partirono per la campagna. 

\capitolo{VII}\label{vii-4} 

Vronskij e Anna viaggiavano già da tre mesi insieme per l'Europa. Avevano visitato Venezia, Roma, Napoli ed erano appena arrivati in una piccola città italiana, dove volevano stabilirsi per un certo tempo. 

Il capocameriere, un bell'uomo, con una scriminatura che incominciava dal collo nei capelli folti impomatati, in frac e grande sparato bianco di batista alla camicia, con una filza di ciondoli sulla pancetta arrotondata, rispondeva, le mani in tasca e gli occhi socchiusi e disdegnosi, qualcosa di arcigno a un signore che s'era fermato. Avendo sentito dall'altra parte dell'ingresso dei passi che salivano la scala, il capocameriere si voltò e, visto il conte russo che da loro occupava le stanze migliori, tolse rispettosamente le mani di tasca e, inchinandosi, riferì che l'inserviente era andato e che la faccenda dell'affitto del palazzo si era conclusa. L'amministratore era pronto a firmare il contratto. 

- Ah! Sono molto contento - disse Vronskij. - E la signora è in casa o no? 

- La signora è uscita a passeggiare, ma è rientrata or ora - rispose il cameriere. 

Vronskij si tolse il cappello floscio dalle falde larghe e asciugò col fazzoletto la fronte sudata e i capelli lunghi fino a metà orecchie e pettinati all'indietro, in modo da nascondere la calvizie. Guardato distrattamente il signore che stava ancora là e che lo contemplava, fece per passare. 

- Questo signore è un russo e ha chiesto di voi - disse il capocameriere. 

Con un senso di irritazione, perché non riusciva a sfuggire in nessun posto ai conoscenti, misto al desiderio di trovare una qualche distrazione alla monotonia della propria vita, Vronskij guardò ancora una volta il signore che si era allontanato e fermato e, nello stesso istante, a tutt'e due si illuminarono gli occhi. 

- Golenišcev! 

- Vronskij! 

Golenišcev era stato, infatti, compagno di Vronskij al corpo dei paggi. Golenišcev, allora, apparteneva al partito liberale; era uscito dal corpo con un grado civile, ma non era mai stato impiegato in alcun posto. Finito il corso, i due compagni si erano completamente perduti di vista e in seguito s'erano incontrati una volta soltanto. 

In quell'incontro Vronskij aveva capito che Golenišcev aveva scelto un'attività del tutto libera e intellettuale e che, perciò, voleva spregiare l'attività e la condizione sociale di Vronskij. Per questo Vronskij, in quell'incontro con Golenišcev, gli aveva opposto quella fredda e orgogliosa resistenza che egli sapeva opporre alla gente, e il cui senso era questo: ``Vi può piacere o non piacere il mio modo di vivere, ma questo per me è assolutamente indifferente: mi dovete stimare se volete conoscermi''. Golenišcev era stato sprezzantemente indifferente al tono di Vronskij. Quell'incontro sembrava avesse dovuto separarli ancor più. Adesso, invece, si erano illuminati e avevano dato un grido di gioia nel riconoscersi. Vronskij non si aspettava in nessun modo di rallegrarsi tanto per Golenišcev, ma probabilmente non sapeva neanche lui quanto si annoiasse. Dimenticò l'impressione spiacevole dell'ultimo incontro e con un viso aperto, gioioso, tese la mano al compagno di un tempo. Un'eguale espressione di gioia tramutò la prima espressione di titubanza del viso di Golenišcev. 

- Come sono contento di incontrarti! - disse Vronskij, mostrando, in un sorriso cordiale, i suoi forti denti bianchi. 

- E io sento dire ``Vronskij'', ma quale non sapevo. Molto, molto contento. 

- Entriamo, allora. Ebbene, cosa fai? 

- Vivo qua, già da due anni. Lavoro. 

- Ah! - disse Vronskij con interesse. - Entriamo allora. 

E per la solita abitudine dei russi, invece di dire proprio in russo quello che voleva nascondere alla servitù, cominciò a parlare in francese. 

- Conosci la Karenina? Viaggiamo insieme. Vado da lei - disse in francese, guardando attento il viso di Golenišcev. 

- Ah! non sapevo - rispose con interesse Golenišcev sebbene lo sapesse. - Sei arrivato da parecchio tempo, qua? - soggiunse. 

- Io, da tre giorni - rispose Vronskij, esaminando ancora una volta con intenzione il viso del compagno. 

``Già, è un uomo per bene e considera la cosa così come va considerata - si disse Vronskij dopo aver capito l'espressione del viso di Golenišcev e la ragione del suo mutamento di discorso. - Gli si può far conoscere Anna; egli considera la cosa così come va considerata''. 

Vronskij, in quei tre mesi che aveva passato con Anna all'estero, nel fare amicizia con gente nuova, si era sempre posta la domanda come ogni nuova persona avrebbe considerato i suoi rapporti con Anna, e, nella maggioranza dei casi, aveva incontrato nella gente una certa comprensione, così ``come si deve''. Ma se avessero chiesto a lui e a quelle persone che intendevano la cosa così ``come si deve'', in che cosa consistesse questo loro intendere, e lui e queste persone si sarebbero trovate in difficoltà. 

In fondo, quelli che, secondo Vronskij, capivano la cosa ``come si deve'', non la intendevano affatto, ma si comportavano, in genere, come si comportano le persone beneducate riguardo a tutte le questioni complesse e insolubili che d'ogni parte circondano la vita; si comportavano secondo le convenienze, evitando allusioni e domande spiacevoli. Facevano finta di capire il senso della situazione, di riconoscerla, perfino di approvarla, ma di considerare fuori posto e superfluo spiegare tutto ciò. 

Vronskij indovinò subito che Golenišcev era uno di questi e perciò fu doppiamente contento di vederlo. Difatti Golenišcev si comportò con la Karenina, quando fu introdotto da lei, così come Vronskij poteva desiderare. Evitava, naturalmente, senza il più piccolo sforzo, tutti i discorsi che potevano portar disagio. 

Egli prima non conosceva Anna e fu stupito della sua bellezza e ancor più della semplicità con cui accettava la sua situazione. Ella arrossì quando Vronskij introdusse Golenišcev, e questo rossore infantile, che si diffuse sul viso di lei, aperto e bello, gli piacque in modo straordinario. E gli piacque in modo particolare, perché subito, come apposta, per non far sorgere equivoci davanti a una persona estranea, ella chiamò Vronskij semplicemente Aleksej e disse che andavano a stare in una casa allora presa in affitto, che la gente del luogo chiamava ``palazzo''. Questo atteggiamento leale e semplice dinanzi alla propria situazione, piacque a Golenišcev. Osservando la maniera cordialmente allegra, decisa, di Anna, conoscendo Aleksej Aleksandrovic e Vronskij, a Golenišcev sembrava di comprenderla in pieno. Gli sembrava di capire quello che lei non riusciva a capire in nessun modo: il fatto che lei dopo aver fatto l'infelicità del marito, dopo aver abbandonato lui e il figlio e dopo aver perduto la propria reputazione, potesse, tuttavia, sentirsi decisamente gaia e felice. 

- C'è nella guida - disse Golenišcev a proposito di quel palazzo che Vronskij aveva preso in affitto. - Là c'è un bellissimo Tintoretto. Dell'ultimo periodo. 

- Sapete cosa? Il tempo è bellissimo, andiamo là, diamoci un'occhiata ancora una volta - disse Vronskij ad Anna. 

- Sono molto contenta, vado subito a mettermi il cappello. Cosa dite, fa caldo? - disse, fermandosi sulla porta e guardando interrogativamente Vronskij. E di nuovo un vivace rossore le coprì il viso. 

Vronskij capì dal suo sguardo ch'ella non sapeva in quali rapporti egli volesse essere con Golenišcev, e che temeva di non essersi comportata come avrebbe voluto lui. Egli la guardò con uno sguardo tenero, prolungato. 

- No, non molto - disse. 

E a lei parve d'aver capito tutto, principalmente ch'egli era contento di lei; e sorridendo uscì con passo veloce dalla porta. 

Gli amici si guardarono l'un l'altro, e sul viso di tutte e due passò un'ombra di disagio, come se Golenišcev, che evidentemente l'aveva ammirata, volesse dire qualcosa di lei e non trovasse le parole, mentre Vronskij desiderava e temeva la stessa cosa. 

- Allora, ecco come va - cominciò Vronskij per cominciare un discorso. - Allora ti sei stabilito qua? - continuò ricordando che gli avevano detto che Golenišcev scriveva qualcosa. 

- Già, scrivo la seconda parte dei Due princìpi - disse Golenišcev, accendendosi di soddisfazione a questa domanda - cioè, per essere precisi, non scrivo ancora, ma vado preparando e raccogliendo il materiale. Sarà molto più ampia e comprenderà quasi tutte le questioni. Da noi, in Russia, non si vuole capire che siamo gli eredi di Bisanzio - e qui cominciò una lunga, calorosa spiegazione. 

Vronskij in principio si sentì a disagio, perché non conosceva neppure il primo capitolo dei Due princìpi, di cui l'autore gli parlava come di cosa nota. Ma poi, quando Golenišcev cominciò a esporre le sue idee e Vronskij poté seguirlo, allora, anche senza conoscere i Due princìpi, l'ascoltò con interesse, perché Golenišcev parlava bene. Ma l'agitata irritazione con la quale Golenišcev parlava dell'argomento che lo interessava, colpì e amareggiò Vronskij. Quanto più s'ingolfava nel discorso, tanto più gli si accendevano gli occhi, tanto più concitatamente ribatteva i suoi presunti avversari e tanto più agitata e offesa diveniva l'espressione del suo viso. Ricordando Golenišcev come un ragazzo magro, vivace, cordiale e nobile, sempre il primo della classe al corpo dei paggi, Vronskij non riusciva a capire in nessun modo le ragioni di quell'irritazione, e non l'approvava. In particolare, non gli piaceva che Golenišcev, uomo della buona società, si mettesse allo stesso livello di quegli scribacchini che lo irritavano, e si arrabbiasse con loro. Ne valeva la pena? Questo non piaceva a Vronskij; malgrado ciò, egli sentiva che Golenišcev non era felice e ne aveva pena. Un'infelicità, quasi un'alienazione mentale, si scopriva in quel viso mobile, abbastanza bello, mentre egli, senza notare neppure l'apparizione di Anna, continuava a esprimere in fretta e con calore le proprie idee. 

Quando Anna apparve in cappello e pellegrina e, giocherellando con l'ombrello con un movimento rapido della mano, si fermò vicino a lui, Vronskij con un senso di sollievo si distolse dagli occhi dolenti di Golenišcev fissi su di lui e guardò con rinnovato amore la sua deliziosa amica, piena di vita e di gioia. Golenišcev tornò in sé con difficoltà e in un primo momento fu triste e cupo; ma Anna, disposta affabilmente verso tutti (così era in quel periodo), lo rianimò presto col suo modo di fare semplice e gaio. Dopo aver tentato vari argomenti di conversazione, l'indusse a parlare di pittura di cui egli parlava molto bene, e prese ad ascoltarlo con attenzione. Giunsero a piedi fino alla casa presa in affitto e la visitarono. 

- Sono molto contenta di una cosa - diceva Anna a Golenišcev sulla via del ritorno. - Aleksej avrà un buon atelier. Prendi assolutamente tu quella stanza - diceva a Vronskij in russo e dandogli del tu, poiché aveva già capito che, nella loro solitudine, Golenišcev sarebbe divenuta una persona intima, e che dinanzi a lui non bisognava fingere. 

- Dipingi, forse? - disse Golenišcev, volgendosi in fretta a Vronskij. 

- Sì, molto tempo fa me ne sono interessato ed ora ho ripreso un poco - disse Vronskij, arrossendo. 

- Ha un gran talento - disse Anna con un sorriso gioioso. - Io, s'intende, non posso giudicare. Ma alcuni competenti hanno detto la stessa cosa. 

\capitolo{VIII}\label{viii-4} 

Anna, in quel periodo di libertà e rapida guarigione, si sentiva imperdonabilmente felice e piena di gioia di vivere. Il ricordo dell'infelicità del marito non avvelenava più la sua felicità. Questo ricordo, da una parte, era troppo terribile per poterci pensare; dall'altra aveva dato a lei una felicità troppo grande per pentirsene. Il ricordo di tutto quello che le era accaduto dopo la malattia, la riconciliazione col marito, la rottura, la notizia della ferita di Vronskij, la sua apparizione, i preparativi per il divorzio, l'abbandono del tetto maritale, l'addio al figlio, tutto questo le sembrava un sogno febbrile dal quale si era svegliata all'estero, sola, con Vronskij. Il ricordo del male causato al marito destava in lei una sensazione simile alla ripugnanza e vicina a quella che proverebbe un uomo che, nell'annegare, abbia strappato via da sé un essere che gli si era aggrappato. Quest'essere era annegato. Era stato male, s'intende, ma era stata l'unica salvezza, ed era meglio non ricordare particolari così paurosi. 

Nel primo momento del distacco le era venuto in mente un solo ragionamento, che la tranquillizzava su quel che aveva fatto; e ora che ricordava tutto il passato, ricordava questo solo ragionamento. ``Io ho fatto inevitabilmente l'infelicità di questo uomo - pensava - ma non voglio profittare di questa infelicità; anch'io soffro e soffrirò: sono privata di quello che prima mi era più caro, sono privata dell'onestà del mio nome e di mio figlio. Ho agito male e perciò non voglio la felicità, non voglio il divorzio e soffrirò la vergogna e il distacco da mio figlio''. Ma, per quanto sinceramente volesse soffrire, Anna non soffriva. Vergogna non ce n'era. Con quel tatto che in così grande misura avevano entrambi, all'estero, evitando le signore russe, non si mettevano mai in una posizione falsa e incontravano ovunque persone che fingevano di capire completamente la loro posizione molto meglio di loro stessi. Il distacco dal figlio che amava, neanche questo la tormentava, nei primi tempi. La bambina, la figlia avuta da lui, era così graziosa che Anna di rado ricordava il figlio. 

Il bisogno di vivere, reso più forte dalla guarigione, era così prepotente, e le condizioni di vita così nuove e piacevoli, che Anna si sentiva imperdonabilmente felice. Quanto più conosceva Vronskij, tanto più l'amava. Lo amava per lui stesso e per il suo amore per lei. Il completo possesso di lui la rendeva continuamente felice. La vicinanza di lui le era sempre piacevole. Tutti i tratti del suo carattere, che veniva a conoscere, le erano sempre più indicibilmente cari. Il suo aspetto, diverso negli abiti borghesi, era affascinante per lei come per una ragazza innamorata. In tutto quello ch'egli diceva, pensava e operava, ella vedeva qualcosa di particolarmente nobile ed elevato. Il proprio entusiasmo dinanzi a lui spesso la sgomentava: cercava, e non le riusciva, di trovar qualcosa in lui che non fosse bello. Non osava mostrargli la consapevolezza della propria nullità di fronte a lui. Le sembrava che, sapendo questo, egli potesse disincantarsi più presto di lei; e in questo momento, pur non avendone alcun motivo, nulla temeva tanto quanto perdere il suo amore. Ma non poteva non essergli riconoscente per il suo comportamento verso di lei, e non poteva non dimostrargli di apprezzarlo. Lui, pur avendo, così le pareva, una spiccata inclinazione per l'attività politica, nella quale doveva sostenere una parte eminente, aveva sacrificato la sua ambizione per lei, senza mai mostrare il più piccolo rimpianto. Era più di prima amorevolmente rispettoso verso di lei, e il pensiero ch'ella non sentisse mai il disagio della propria posizione, non lo abbandonava neppure un attimo. Lui, così virile, nei rapporti con lei non solo non la contrariava mai, ma non aveva una propria volontà e sembrava preoccupato solo dal pensiero di prevenire i desideri di lei. E lei non poteva non apprezzare ciò, sebbene l'intensità delle attenzioni verso di lei, l'atmosfera di premura di cui egli la circondava, a volte le pesassero. 

Vronskij intanto, malgrado il completo appagamento di quello ch'egli aveva così a lungo desiderato, non era pienamente felice. Ben presto sentì che l'appagamento del desiderio gli aveva dato solo un granello di sabbia di quella montagna di felicità che si attendeva. Questo appagamento gli aveva mostrato l'eterno errore che commettono gli uomini che si figurano la felicità nell'appagamento di un desiderio. Nel primo periodo in cui era unito a lei e aveva indossato gli abiti borghesi, aveva sentito tutto l'incanto della libertà che prima non conosceva, e della libertà nell'amore; e ne fu contento, ma non a lungo. Ben presto sentì che nell'animo suo s'era destato il desiderio dei desideri: la malinconia. Indipendentemente dalla propria volontà cominciò ad aggrapparsi ad ogni capriccio passeggero, scambiandolo per un'aspirazione e uno scopo. Sedici ore della giornata bisognava pure occuparle con qualcosa, giacché all'estero vivevano in piena libertà, al di fuori di quella cerchia di condizioni di vita sociale che, a Pietroburgo, assorbiva loro il tempo. Ai piaceri della vita da scapolo, che nei precedenti viaggi all'estero avevano occupato Vronskij, non si poteva neppure pensare, giacché un esperimento di tal genere aveva prodotto in Anna un abbattimento inaspettato e inadeguato in seguito a una cena fatta a tarda ora con amici. Relazioni con la società locale e con quella russa, data l'indeterminatezza della loro posizione, non si potevano avere. La visita ai monumenti più importanti, oltre al fatto che tutto era stato visitato, non aveva per lui, russo e uomo d'ingegno, quell'inspiegabile importanza che le attribuiscono gli inglesi. 

E, come un animale affamato afferra qualsiasi cosa gli capiti, sperando di trovarvi cibo, così pure Vronskij, del tutto inconsapevole, s'aggrappava ora alla politica, ora ai libri nuovi, ora ai quadri. 

In gioventù aveva avuto disposizione alla pittura e, non sapendo come spendere il denaro, aveva cominciato a raccogliere incisioni; si fermò, quindi, sulla pittura, prese ad occuparsene e ripose in essa quella insoddisfatta riserva di desideri che reclamava d'essere appagata. 

Aveva attitudine a intendere l'arte e ad imitare con fedeltà, con gusto, l'opera d'arte; credette così d'avere ciò che occorre all'artista. Dopo un certo tempo d'incertezza sul genere di pittura da scegliere, religioso, storico, di genere o realistico, si mise a dipingere. Intendeva qualsiasi genere, e poteva ispirarsi a questo e a quello; non immaginava che si potesse del tutto ignorare quali generi di pittura esistessero e che ci si potesse ispirare direttamente a quello che c'è nell'anima, senza preoccuparsi se quello che si è dipinto appartiene a un certo determinato genere. Poiché non sapeva questo e non traeva ispirazione direttamente dalla vita, ma mediamente, dalla vita già incarnata nell'arte, egli si ispirava molto alla svelta, e con facilità otteneva che quanto dipingeva fosse molto simile a quel tal genere che voleva imitare. 

Più di tutti gli altri gli piaceva il francese, grazioso e d'effetto, e in questo genere cominciò a fare il ritratto di Anna in costume italiano: questo ritratto, a lui e a tutti quelli che lo vedevano, sembrava molto ben riuscito. 

\capitolo{IX}\label{ix-4} 

Il vecchio palazzo abbandonato, dai soffitti alti, modellati e gli affreschi sui muri, dai pavimenti a mosaico, le pesanti tende di damasco giallo alle finestre alte, e i vasi sulle mensole e sui camini, dalle porte intagliate e le sale oscure con i quadri appesi, questo palazzo, dopo che vi presero alloggio, con lo stesso suo aspetto esteriore, manteneva Vronskij nel piacevole errore ch'egli non fosse tanto il proprietario russo, il gran cacciatore a riposo, quanto un illuminato amatore e protettore di arti, e lui stesso un modesto artista che avesse rinunciato al mondo, alle relazioni, all'ambiente, per la donna amata. 

La parte assunta da Vronskij, col passaggio nel palazzo, riuscì perfettamente, e, fatta la conoscenza di alcune persone interessanti per mezzo di Golenišcev, in un primo tempo egli fu tranquillo. Dipingeva, sotto la guida di un maestro italiano, degli studi dal vero, e si occupava di vita medioevale italiana. La vita medioevale italiana, negli ultimi tempi, aveva tanto affascinato Vronskij che perfino il cappello e lo scialle di lana sulla spalla cominciò a portare alla foggia medioevale, cosa che gli donava molto. 

- E noi viviamo e non sappiamo nulla - disse una volta Vronskij a Golenišcev che era venuto da lui di buon'ora. - Hai veduto il quadro di Michajlov? - disse, tendendogli un giornale russo ricevuto appena quella mattina e mostrandogli un articolo sull'artista russo che viveva nella stessa città e che aveva ultimato un quadro del quale, da lungo tempo, si parlava e che era stato acquistato in anticipo. Nell'articolo c'erano rimproveri al governo e all'accademia perché un artista così notevole era lasciato privo di incoraggiamento e d'aiuti. 

- Ho visto - rispose Golenišcev. - S'intende, egli non è privo di talento, ma è su di una via completamente falsa. Sempre la stessa maniera di trattare il Cristo e la pittura religiosa alla Ivanov-Strauss-Renan. 

- Cosa rappresenta il quadro? - chiese Anna. 

- Cristo dinanzi a Pilato. Cristo è rappresentato come un ebreo, con tutto il realismo della nuova scuola. 

E, portato dalla domanda sul contenuto del quadro a uno dei suoi temi preferiti, Golenišcev cominciò a parlare: 

- Io non capisco come possano sbagliarsi così grossolanamente. Cristo ha già la sua incarnazione definita nell'arte dei grandi\ldots{} Dunque, se non vogliono rappresentare Iddio, ma un rivoluzionario o un saggio, che prendano pure dalla storia Socrate, Franklin, Carlotta Corday, ma Cristo, no. Essi prendono proprio quel personaggio che non si può prendere per l'arte, ma dopo\ldots{} 

- Ebbene, è vero che questo Michajlov si trova in tanta miseria? - chiese Vronskij, pensando che lui, come mecenate russo, avrebbe dovuto aiutare l'artista, bello o brutto che fosse il quadro. 

- È difficile. È un ritrattista famoso. Avete visto il ritratto della Vasil'cikova? Ma sembra che ora non voglia far più ritratti, e perciò è probabile che sia in ristrettezze. Io dico che\ldots{} 

- Non si potrebbe pregarlo di fare il ritratto ad Anna Arkad'evna ? - disse Vronskij. 

- Perché a me? - chiese Anna. - Dopo il tuo, io non voglio altro ritratto. Piuttosto ad Annie - così ella chiamava la bambina. - Eccola - aggiunse, dopo aver dato un'occhiata dalla finestra alla bella balia italiana che aveva portato fuori la bambina in giardino, e voltandosi subito a guardare Vronskij. La bella balia, che serviva da modella a Vronskij per una testa di un suo quadro, era l'unico dolore segreto di Anna. Vronskij, ritraendola, ne ammirava la bellezza e il tipo medioevale, e Anna non aveva il coraggio di confessarsi di temere d'essere gelosa di questa balia, e perciò blandiva e viziava particolarmente lei e il suo bambino. 

Vronskij guardò anche lui dalla finestra e guardò Anna negli occhi, ma poi, voltosi subito a Golenišcev, disse: 

- E tu, lo conosci questo Michajlov? 

- L'ho incontrato. Ma è un originale e non ha nessuna cultura. Sapete, uno di quegli uomini nuovi selvaggi che adesso s'incontrano di frequente; sapete, uno di quei liberi pensatori che sono educati d'embléè nelle idee del materialismo, della negazione, dell'ateismo. Prima succedeva - diceva Golenišcev senza notare o senza voler notare che sia Anna che Vronskij desideravano interloquire - prima succedeva che il libero pensatore fosse un uomo educato nelle idee della religione, della legge, della morale, e che da solo fosse giunto con lotte e stenti al libero pensiero; ma adesso è comparso un nuovo tipo di libero pensatore istintivo, il quale cresce senza neppure sentir dire che ci sono leggi morali, religiose, che ci sono delle autorità; cresce, senz'altro, nell'idea di negar tutto, cioè come un selvaggio. Ecco, lui è così. È figlio, mi pare, di un capocameriere moscovita e non ha ricevuto alcuna istruzione. Quando entrò in accademia e acquistò fama, da un uomo non sciocco qual'era, desiderò di istruirsi. E si rivolse a quella che gli sembrava la fonte della cultura: alle riviste. E voi capite, nei tempi passati, un uomo che avesse voluto istruirsi, mettiamo un francese, avrebbe cominciato con lo studiare tutti i classici, i tragici, gli storici, i filosofi; e voi intendete tutto il lavoro intellettuale che avrebbe avuto dinanzi a sé. Ma da noi, adesso, egli si è imbattuto proprio nella letteratura nichilista: ha fatto sua, molto presto, tutta l'essenza della scienza che nega, ed è bell'e pronto. E ancora questo sarebbe poco: venti anni fa, avrebbe trovato in questa letteratura i segni della lotta con le autorità, con le opinioni secolari, e da questa lotta avrebbe capito che c'era stato qualcosa di diverso; ma adesso si imbatte in una letteratura tale, che, in essa, non vengono degnate neppure di una discussione le opinioni di un tempo, ma si dice francamente: non c'è nulla, évolution, selezione, lotta per l'esistenza, ed è tutto. Io, nel mio articolo\ldots{} 

- Sapete cosa? - disse Anna che già da tempo scambiava, cauta, delle occhiate con Vronskij e sapeva che non la cultura di quell'artista lo interessava, ma solo l'idea di aiutarlo e di ordinargli il ritratto. - Sapete cosa? - disse interrompendo decisa Golenišcev che non la smetteva di parlare. - Andiamo da lui! 

Golenišcev ritornò in sé e acconsentì volentieri. Ma poiché l'artista abitava lontano, stabilirono di prendere una vettura. 

Dopo un'ora Anna, a fianco di Golenišcev, e Vronskij invece seduto nel sedile anteriore della vettura, si avvicinavano a una brutta costruzione moderna di un quartiere periferico. Avendo saputo dalla moglie del portiere, che venne loro incontro, che Michajlov riceveva nel suo studio, ma che in quel momento era a casa, in un quartiere a due passi da lì, la mandarono da lui con i loro biglietti da visita, per chiedere il permesso di vedere i quadri. 

\capitolo{X}\label{x-4} 

Il pittore Michajlov, come sempre del resto, era al lavoro quando gli portarono i biglietti da visita di Vronskij e di Golenišcev. La mattina aveva lavorato nello studio al quadro grande. Venuto a casa, si era arrabbiato con la moglie perché non aveva saputo destreggiarsi con la padrona di casa che pretendeva denaro. 

- Te l'ho detto venti volte, non metterti a dare spiegazioni. Anche così sei sciocca, ma se cominci a spiegarti in italiano, allora diventi tre volte sciocca - le disse dopo una lunga discussione. 

- Allora tu non trascurare, la colpa non è mia. Se avessi denaro\ldots{} 

- Lasciami in pace, per l'amor di Dio! - gridò Michajlov con le lacrime nella voce e, tappatesi le orecchie, andò nella sua stanza di lavoro, di là da un tramezzo, chiudendo la porta dietro di sé. ``Insensata!'' borbottava, sedendo al tavolo e, distesovi un cartone, cominciò subito a lavorare con particolare ardore a un disegno incominciato. 

Non lavorava mai con tanto ardore e successo come quando la vita gli andava male e, soprattutto, quando litigava con la moglie. ``Ah! se potesse sprofondare in qualche parte!'' pensava e continuava a lavorare. Stava facendo il disegno per una figura d'uomo in stato d'ira. Il disegno era già fatto; ma egli non ne era contento. ``No, quell'altro era migliore\ldots{} Dov'è?''. Andò dalla moglie e, accigliato, senza guardarla, domandò alla bambina più grande dov'era quel cartone che aveva dato loro. Il cartone col disegno abbandonato si trovò, ma era sporco e macchiato di stearina. Egli prese, tuttavia, il disegno, lo stese dinanzi a sé sul tavolo e, allontanandolo e socchiudendo gli occhi, cominciò a guardarlo. Improvvisamente sorrise e agitò con gioia le mani. 

- Sì, così - pronunciò, e subito, afferrata una matita, cominciò a disegnare alla svelta. La macchia di stearina aveva dato un atteggiamento nuovo alla figura. 

Disegnava questo nuovo atteggiamento e, a un tratto, gli venne in mente il viso energico, dal mento prominente, del venditore di sigari; e quello stesso viso, quello stesso mento egli dette alla sua figura. Rise di gioia. La figura a un tratto da morta che era, inventata, divenne viva e tale che non si poteva mutarla. Questa figura viveva, ed era definita con chiarezza e precisione. Si poteva correggere il disegno secondo le esigenze della figura, si poteva e si doveva perfino divaricare le gambe in altro modo, si poteva cambiare del tutto la posizione della mano sinistra, buttare all'indietro i capelli. Eppure, facendo queste correzioni, egli non alterava la figura, ma toglieva solo quello che la velava. Era come se togliesse via quei veli che non la rendevano perfettamente visibile; ogni nuovo tratto tendeva solo a maggiormente esprimere tutta la figura nella sua forza, così come gli era emersa d'un tratto per effetto della macchia prodotta dalla stearina. Stava finendo con cautela la figura quando gli portarono i biglietti. 

- Subito, subito! 

E andò dalla moglie. 

- Su, via, Saša, non ti arrabbiare! - disse, sorridendo timido e affettuoso. - Tu non ne hai colpa. La colpa è mia. Accomoderò tutto. - E rappacificatosi con la moglie, si mise il cappotto olivastro dal collo di velluto e il cappello e andò nello studio. Aveva già dimenticato la figura che gli era riuscita. Ora lo rallegrava e lo agitava la visita del suo studio da parte di quei personaggi russi importanti, arrivati in vettura. 

Del proprio quadro, quello che stava attualmente sul cavalletto, nel profondo dell'animo suo, aveva una sola idea, che un quadro simile nessuno mai l'avesse dipinto. Egli non pensava che fosse migliore di tutti quelli di Raffaello, ma sapeva che quanto aveva voluto esprimervi, nessuno mai l'aveva espresso. Questo lo sapeva fermamente e lo sapeva da gran tempo, da quando aveva cominciato a dipingerlo; ma i giudizi degli altri, quali che fossero, avevano tuttavia per lui un'importanza enorme e lo agitavano fino in fondo all'anima. Ogni osservazione, anche la più inconsistente, che mostrasse che i giudici vedevano una sia pur piccola parte di quello che vedeva lui in quel quadro, lo impressionava profondamente. Ai suoi giudici attribuiva sempre una profondità di comprensione maggiore di quella che lui stesso non avesse, e da loro aspettava sempre qualcosa che lui stesso non aveva scorto nel quadro. E spesso, nei giudizi degli osservatori, gli sembrava di trovarlo questo qualcosa. 

Egli si avvicinava con passo svelto alla porta dello studio e, malgrado l'agitazione, la figura di Anna tenuemente illuminata che, dritta nell'ombra del portone, ascoltava Golenišcev che le parlava calorosamente di qualcosa, e che nel medesimo tempo mostrava di voler esaminare l'artista che si avvicinava, lo stupì. Ma non notò neppure che, nell'avvicinarsi a loro, egli aveva afferrato e assorbito questa impressione, così come aveva fatto col mento del venditore di sigari, per nasconderla chi sa dove, e trarla fuori di là quando ce ne sarebbe stato bisogno. I visitatori, delusi già dalle precedenti descrizioni di Golenišcev del pittore, furono ancor più delusi dal suo aspetto esteriore. Di media statura, tarchiato, con un'andatura inquieta, Michajlov col cappello marrone, il cappotto olivastro e i pantaloni stretti, quando già da tempo si portavano larghi, con quel suo viso piatto, ordinario, e con quella espressione di timidezza, mista al desiderio di mantenere la propria dignità, produsse proprio un'impressione sgradevole. 

- Vi prego di voler passare - egli disse, cercando di prendere un'aria indifferente e, entrato nell'ingresso, tirò fuori di tasca la chiave, e aprì la porta. 

\capitolo{XI}\label{xi-4} 

Entrando nello studio, il pittore Michajlov guardò ancora una volta gli ospiti e annotò ancora, nella sua fantasia, l'espressione del viso di Vronskij, in particolare i suoi zigomi. Sebbene il suo senso artistico lavorasse continuamente per raccogliere materiale, sebbene egli sentisse una agitazione sempre maggiore all'avvicinarsi del giudizio sul suo lavoro, tuttavia con rapidità e finezza d'intuito, attraverso segni impercettibili, si andò formando una idea su quelle tre persone. Quello (Golenišcev), era un russo di qua. Michajlov non ne ricordava il nome, né dove l'avesse incontrato, né di che cosa avessero parlato insieme. Ricordava solo il suo viso, come ricordava tutti i visi che vedeva una sola volta o due; ma ricordava anche che era uno di quei visi messi da parte, nella sua immaginazione, nel gran reparto di quelli dall'espressione mutevole e povera. I capelli lunghi e la fronte molto aperta davano un'importanza esteriore a un viso nel quale non c'era che una piccola espressione infantile, concentrata sopra la radice stretta del naso. Vronskij e la Karenina, secondo le considerazioni di Michajlov, dovevano essere russi di gran famiglia e ricchi, che non capivano nulla di arte, come tutti i russi ricchi, del resto, ma che se ne mostravano intenditori ed amatori. ``Probabilmente, ormai, avranno visitato le antichità e ora fanno il giro degli studi moderni, di un ciarlatano tedesco o di uno stupido inglese preraffaellita, e da me sono venuti solo per completare la visita'' pensava. Sapeva molto bene il modo di fare dei dilettanti (e quanto più intelligenti sono, tanto è peggio), di visitare, cioè, gli studi degli artisti contemporanei col solo scopo di avere il diritto di dire che l'arte è in decadenza, e che quanto più si guardano i nuovi, tanto più si vede come siano rimasti inimitabili i grandi maestri d'un tempo. Tutto questo se l'aspettava, tutto questo lo scorgeva nei loro visi, lo vedeva nell'indifferente negligenza con cui parlavano fra di loro, con cui guardavano i manichini e i busti e passeggiavano liberamente, aspettando ch'egli scoprisse il quadro. Malgrado ciò, mentre voltava i suoi studi, sollevava le tendine e toglieva il lenzuolo, egli sentiva una grande agitazione, tanto più che sebbene, secondo lui, tutti i russi di gran famiglia e ricchi dovessero essere bestie e stupidi, Vronskij e in modo particolare Anna gli piacevano. 

- Ecco, volete favorire? - disse, allontanandosi da un lato con la sua andatura irrequieta e indicando il quadro. - È l'esortazione di Pilato. Capitolo XXVII del vangelo di Matteo - disse, sentendo che le labbra cominciavano a tremargli per l'agitazione. Si allontanò e si mise dietro di loro. 

In quei pochi secondi in cui i visitatori guardarono il quadro in silenzio, anche Michajlov lo guardò con occhio indifferente, distaccato. In quei pochi secondi egli credette, in principio, che il giudizio più alto e più giusto sarebbe stato pronunciato da loro, proprio da quei visitatori che aveva tanto disprezzato un momento prima. Aveva dimenticato tutto quello che pensava del suo quadro prima, in quei tre anni in cui l'aveva dipinto, aveva dimenticato tutti i pregi di cui non aveva dubitato; guardava ora il quadro con occhio indifferente, distaccato, nuovo, e non ci vedeva nulla di bello. Vedeva, in primo piano, il viso irritato di Pilato e quello calmo del Cristo e, in secondo piano, le figure dei servi di Pilato e il viso di Giovanni che osservava quanto accadeva. Ogni viso che era sorto in lui dopo tanta ricerca, dopo tanti errori e correzioni, col proprio carattere a sé stante, ogni viso che gli aveva dato tanto tormento e tanta gioia, e tutti quei visi tante volte cambiati di posto per l'insieme, tutte le sfumature di colori e di toni, ottenute con tanto sforzo, adesso tutto quell'insieme, visto con gli occhi loro, gli sembrava una cosa volgare, mille volte ripetuta. Il viso che più gli era caro, quello del Cristo, nel centro del quadro, che tanto entusiasmo gli aveva dato quando lo aveva scoperto, si disperse tutto per lui, ora che guardava il quadro con gli occhi loro. Vedeva una copia ben fatta (anzi neanche ben fatta, adesso ci vedeva un cumulo di difetti), di quegli infiniti Cristi del Tiziano, di Raffaello, del Rubens e di quegli stessi soldati e Pilati. Tutto questo era volgare, povero e risaputo e perfino dipinto male, con troppi colori e con fiacchezza. Essi avrebbero avuto ragione di pronunciare frasi di cortese ipocrisia dinanzi all'artefice per poi compiangerlo e irriderlo quando fossero rimasti soli. 

Quel silenzio gli divenne troppo penoso (sebbene non durasse più di un minuto). Per spezzarlo e per far vedere che non era agitato, fatto uno sforzo su di sé, si rivolse a Golenišcev. 

- Mi pare di aver avuto il piacere di incontrarvi - disse, volgendosi a guardare con inquietudine ora Anna, ora Vronskij per non perdere neppure un tratto dell'espressione dei loro visi. 

- E come! ci siamo visti in casa Rossi, ricordate, quella sera in cui quella signorina italiana, una nuova Rachel, declamava - cominciò a dire con disinvoltura Golenišcev, distaccando senza il minimo rimpianto gli occhi dal quadro e rivolgendosi all'artista. 

Avendo però notato che Michajlov si aspettava un giudizio sul quadro, aggiunse: 

- Il vostro quadro è andato molto avanti da quando l'ho visto l'ultima volta. E, come allora, anche adesso mi colpisce straordinariamente la figura di Pilato. Si capisce così bene che quest'essere, questo buono e bravo giovane, ma burocrate fino in fondo all'anima, non sa quello che fa. Ma mi pare\ldots{} 

Tutto il viso mobile di Michajlov a un tratto si illuminò: gli occhi si accesero. Voleva dire qualcosa, ma non poté pronunciare nulla per l'agitazione, e finse di schiarirsi la voce. Per quanto mediocre egli stimasse la possibilità di intendere l'arte di Golenišcev, per quanto inconsistente fosse quella giusta osservazione sulla vera espressione di Pilato come burocrate, per quanto increscioso dovesse riuscirgli il vedere espressa per prima un'osservazione così inconsistente, mentre non si diceva nulla delle cose importanti, Michajlov fu entusiasta di quella osservazione. Anch'egli pensava della figura di Pilato quello che aveva detto Golenišcev. Che questa osservazione fosse una delle infinite che, Michajlov fermamente sapeva, sarebbero state tutte giuste, non diminuì per lui il significato dell'osservazione di Golenišcev. Prese ad amare Golenišcev per quella osservazione, e da uno stato di abbattimento passò a un tratto all'entusiasmo. Improvvisamente il quadro tornò a vivere dinanzi a lui con tutta l'indicibile complessità di quello che vive. Michajlov di nuovo tentò di esprimere, che egli così intendeva Pilato; ma le labbra gli tremarono indocili, ed egli non poté pronunciarlo. Vronskij ed Anna pure dicevano qualcosa, con quella voce bassa con cui di solito si parla alle mostre dei quadri e per non offendere l'artista e per non dire ad alta voce una sciocchezza, così facile a dirsi in tema d'arte. A Michajlov sembrava che il quadro avesse fatto impressione anche su di loro. Si avvicinò. 

- Com'è sorprendente l'espressione del Cristo! - disse Anna. Di tutto quello che aveva visto questa espressione le era piaciuta maggiormente, e sentiva che questa era il nucleo del quadro, e che perciò la lode avrebbe fatto piacere all'artista. - Si vede che ha pena di Pilato. 

Era di nuovo una di quelle infinite considerazioni giuste che si potevano fare sul quadro e sulla figura di Cristo. Ella aveva detto ch'egli aveva pena di Pilato. Nell'espressione del Cristo ci doveva essere anche un'espressione di pena perché in Lui c'era l'espressione dell'amore, della calma ultraterrena, della preparazione alla morte e della consapevolezza della vanità delle parole. Naturalmente, in Pilato, c'era l'espressione del burocrate, e nel Cristo la pietà, giacché l'uno è la personificazione della vita del corpo, l'altro della vita dello spirito. Tutto ciò e molte altre cose balenarono nella mente di Michajlov. E di nuovo il suo viso si illuminò di entusiasmo. 

- Sì, e come è fatta questa figura, quanto respiro! Si può girarle intorno - disse Golenišcev, mostrando evidentemente, con questa osservazione, che non approvava il contenuto e il pensiero della figura. 

- Sì, d'una maestria eccezionale! - disse Vronskij. - Come risaltano queste figure sullo sfondo! Ecco la tecnica - egli disse rivolto a Golenišcev, alludendo a una conversazione che era corsa fra di loro sul fatto che Vronskij disperava di acquistare questa tecnica. 

- Sì, sì, sorprendente! - confermarono Golenišcev e Anna. 

Malgrado lo stato di eccitamento in cui era, l'osservazione sulla tecnica provocò una fitta dolorosa al cuore di Michajlov, ed egli, dopo aver guardato irritato Vronskij, si accigliò a un tratto. Aveva sentito spesso la parola tecnica, e decisamente non capiva che cosa si intendesse con questa parola. Egli sapeva che con questa parola si intendeva la facoltà meccanica di dipingere e di disegnare, del tutto indipendente dal contenuto. Spesso aveva notato come anche, in un giudizio veritiero, si contrapponesse la tecnica al valore intimo del lavoro, come se fosse possibile dipingere bene quello che era informe. Sapeva che ci voleva molta attenzione e cautela per non danneggiare l'opera stessa nel toglierle un velo o tutti i veli; ma l'arte non aveva nulla a che fare con la tecnica. Se a un fanciullo o alla propria cuoca si fosse dischiuso tutto quello ch'egli aveva visto, allora anche costei avrebbe potuto cavar fuori quello che vedeva. Invece, il più esperto e abile maestro della tecnica, con la sola facoltà meccanica, non può dipingere nulla se non gli si dischiudono prima le possibilità del contenuto. Inoltre egli vedeva che, a parlar di tecnica, non gli si potevano certo fare degli elogi. In tutto quello che dipingeva e aveva dipinto vedeva difetti che gli ferivano gli occhi dovuti all'audacia con la quale toglieva i veli, e che ormai non poteva correggere senza sciupare l'intera opera. E su quasi tutte le figure e i volti egli vedeva ancora i resti dei veli, non completamente tolti, che sciupavano il quadro. 

- Una cosa si potrebbe dire, se mi permettete di fare quest'osservazione\ldots{} - notò Golenišcev. 

- Ah, sono molto contento, e ve ne prego - disse Michajlov, sorridendo con finzione. 

- È che voi avete fatto di lui un uomo-Dio e non un Dio-uomo. Del resto so che volevate proprio questo. 

- Non potevo dipingere quel Cristo che non ho nell'anima - disse Michajlov torvo. 

- Sì, ma in tal caso, se mi permettete di dire la mia idea\ldots{} il vostro quadro, del resto, è così bello che la mia osservazione non lo guasta, e poi è una mia opinione personale. In voi è un'altra cosa. Anche il motivo è un altro. Ma prendiamo, magari, Ivanov. Io credo che, se il Cristo è abbassato al grado di personaggio storico, sarebbe stato meglio per Ivanov scegliere un altro tema storico, fresco, non sfruttato. 

- Ma se questo è il tema più alto che si presenti all'arte? 

- A cercare se ne troveranno altri. Ma l'arte non ammette discussioni e ragionamenti. E davanti al quadro di Ivanov per il credente e per il miscredente si presenta la questione: ``È Dio o non è Dio?'' e ciò distrugge l'unità dell'impressione. 

- E perché? Mi pare che per le persone colte - disse Michajlov - ormai non possa esistere discussione. 

Golenišcev non acconsentì a questo e, attenendosi alla sua prima idea sull'unità dell'impressione, necessaria all'arte, sgominò Michajlov. 

Michajlov si agitava, ma non sapeva dire nulla in difesa della propria idea. 

\capitolo{XII}\label{xii-4} 

Anna e Vronskij già da tempo si scambiavano occhiate, deplorando la concettosa verbosità del loro amico; finalmente Vronskij, senza aspettare il padrone di casa, passò a un altro piccolo quadro. 

- Ah, che incanto! Una meraviglia! Che incanto! - dissero a una voce. 

``Che cosa mai è loro piaciuto tanto?'' pensò Michajlov. Si era perfino dimenticato di quel quadro da lui dipinto tre anni prima. Aveva dimenticato tutte le pene e gli entusiasmi che aveva vissuto per quel quadro, che per vari mesi lo aveva avvinto incessantemente di notte e di giorno; lo aveva dimenticato, come sempre dimenticava i quadri compiuti. Non gli piaceva neppure più guardarlo, e lo aveva esposto solo perché aspettava un inglese che desiderava comprarlo. 

- Già, ecco, un vecchio studio - disse. 

- Com'è bello! - disse Golenišcev, anche lui, evidentemente, preso dalla grazia del quadro. 

Due ragazzi, all'ombra di un canneto, pescavano con la lenza. Uno di loro, il più grande, aveva appena gettato la lenza e faceva uscire con cura il galleggiante di là dal cespuglio, tutto assorto in questa faccenda; l'altro, il più piccolo, stava sdraiato sull'erba, poggiando la testa bionda e scarmigliata sulle braccia, e guardava con gli occhi azzurri e pensosi l'acqua. A cosa pensava? 

L'entusiasmo dinanzi a questo suo quadro provocò in Michajlov la stessa agitazione di prima, ma egli temeva e non amava quell'ozioso sentimento verso ciò che era compiuto, e perciò, pur rallegrato dalle lodi, volle attrarre l'attenzione dei visitatori verso un terzo quadro. 

Ma Vronskij chiese se il quadro era in vendita. In quel momento per Michajlov, agitato dai visitatori, un discorso su di una questione di denaro era molto spiacevole. 

- È esposto per la vendita - rispose, accigliandosi cupo. 

Quando i visitatori se ne furono andati, Michajlov sedette di fronte al quadro di Pilato e di Cristo, e nella sua mente riandò a tutto quello che era stato detto e, anche se non detto, sottinteso dai visitatori. E, cosa strana, quello che aveva avuto tanta importanza per lui mentre essi erano là, quando egli si era trasferito col pensiero nel loro modo di vedere, a un tratto, perse ogni significato. Cominciò a guardare il quadro con uno sguardo pienamente d'artista e giunse ad essere sicuro della sua bellezza e, perciò, della sua importanza, cosa di cui aveva bisogno per giungere a quella tensione che escludeva ogni altro interesse e che sola gli era necessaria per lavorare. 

La gamba del Cristo, di scorcio, non era, tuttavia, come doveva. Prese la tavolozza e si mise a lavorare. Correggendo la gamba, egli osservava continuamente la figura di Giovanni in secondo piano, che i visitatori non avevano neppure notato, ma che era, egli lo sapeva, la cosa più compiuta. Ritoccata la gamba, voleva mettersi a lavorare intorno a quella figura, ma si sentiva troppo agitato per far questo. Non riusciva a lavorare né quand'era troppo calmo né quand'era troppo commosso e vedeva tutto con chiarezza eccessiva. Vi era solo una gradazione, nel passaggio tra l'aridità e l'ispirazione, in cui era possibile lavorare. Ma ora egli era troppo agitato. Voleva coprire il quadro, ma si fermò, e, trattenuto con la mano il lenzuolo, sorridendo beato, si soffermò a lungo a guardare la figura di Giovanni. Infine, quasi distaccandosene con rimpianto, abbassò il lenzuolo e stanco, ma felice, andò a casa. 

Vronskij, Anna e Golenišcev, nel tornare a casa, erano particolarmente animati e allegri. Parlavano di Michajlov e dei suoi quadri. La parola ``talento'', con cui intendevano una qualità innata, quasi fisica, indipendente dalla mente e dal cuore, e con la quale volevano definire tutto quello che era stato vissuto dall'artista, ricorreva particolarmente spesso nella loro conversazione, poiché era loro indispensabile fissare in termini quello di cui non avevano nessuna idea, ma di cui volevano parlare. Dicevano che il talento non gli si poteva negare, ma che il suo talento non si era sviluppato per mancanza di cultura, calamità comune a tutti gli artisti russi. Ma il quadro dei ragazzi era loro rimasto nella memoria e ogni tanto vi tornavano su. 

- Che incanto! Come gli è riuscito bene e come è semplice! E lui non capisce neppure come sia bello. Non bisogna lasciarselo sfuggire e comprarlo - diceva Vronskij. 

\capitolo{XIII}\label{xiii-4} 

Michajlov vendette il quadro a Vronskij e acconsentì a fare il ritratto ad Anna. Nel giorno stabilito venne e cominciò il lavoro. 

Il ritratto, dopo cinque sedute, colpì tutti, e in particolare Vronskij, non soltanto per la somiglianza, ma anche per la sua particolare bellezza. Era strano come Michajlov avesse potuto cogliere la bellezza particolare di lei. ``Bisognava amarla e conoscerla, come l'ho amata io, per cogliere proprio quella sua cara espressione spirituale'' pensava Vronskij, pur avendo, solo da questo ritratto, imparato a conoscere quella sua cara espressione spirituale. Ma l'espressione era così vera, che a lui e agli altri pareva di conoscerla da tempo. 

- Io mi affatico da tempo e non ho concluso nulla - diceva del proprio ritratto - e lui ha guardato e ha dipinto. Ecco, cosa vuol dire la tecnica. 

- Questa verrà - lo consolava Golenišcev, nella cui opinione Vronskij aveva un certo talento e, soprattutto, quella cultura che dà una visione superiore dell'arte. La fiducia di Golenišcev nel talento di Vronskij era sostenuta anche dal fatto che egli aveva bisogno della simpatia e delle lodi di Vronskij per i suoi articoli e per le sue idee, e sentiva che le lodi e l'appoggio dovevano essere scambievoli. 

In casa altrui, e in particolare nel palazzo di Vronskij, Michajlov era tutt'altro uomo che non nel suo studio. Era ostilmente rispettoso, quasi temesse l'amicizia con persone che non stimava. Chiamava Vronskij ``vostra eccellenza'', non rimaneva mai a pranzo, malgrado gli inviti di Anna e di Vronskij, e non veniva che per le sedute. Anna era verso di lui più cordiale che non gli altri e gli era riconoscente per il ritratto. Vronskij era con lui più che cortese e, evidentemente, si interessava del giudizio dell'artista sul proprio quadro. Golenišcev non lasciava sfuggir l'occasione per ispirare a Michajlov le vere idee sull'arte. Ma Michajlov rimaneva egualmente freddo verso tutti. Anna sentiva dal suo sguardo ch'egli si compiaceva di guardarla, ma che sfuggiva le conversazioni con lei. Nei discorsi di Vronskij sulla sua pittura taceva ostinatamente e così pure ostinatamente taceva quando gli facevano vedere il quadro di Vronskij; era evidente che i discorsi di Golenišcev gli pesavano e non li ribatteva. 

In generale Michajlov, col suo modo di trattare sostenuto e freddo, quasi ostile, non piacque loro per nulla, quando lo conobbero più da vicino. E furono contenti allorché, finite le sedute, rimase nelle loro mani un magnifico ritratto ed egli cessò di venire. 

Golenišcev, per primo, espresse l'idea che tutti avevano avuto, che cioè Michajlov fosse semplicemente invidioso di Vronskij. 

- Ammettiamo che non provi invidia, perché ha talento; ma lo irrita il fatto che un uomo di corte e ricco, conte per giunta (perché loro odiano tutto ciò), senza particolare fatica, faccia la stessa cosa, se non pure meglio, di lui che vi ha dedicato tutta la vita. Perché ciò che più conta è la cultura che lui non ha. 

Vronskij difendeva Michajlov, ma in fondo all'animo credeva in questo, perché, secondo lui, un uomo d'un altro mondo inferiore doveva provare invidia. 

Il ritratto di Anna, la stessa cosa dipinta dal vero da lui e da Michajlov, avrebbe dovuto mostrare a Vronskij la differenza che esisteva fra lui e Michajlov; ma egli non la vedeva. Decise soltanto di non lavorare più al ritratto di Anna, ritenendolo ormai superfluo, dopo quello di Michajlov. Continuò, invece, il quadro di ambiente medioevale. E lui stesso e Golenišcev, e in particolare Anna, lo ritenevano molto bello, perché molto più somigliante ai quadri famosi che non il quadro di Michajlov. 

Michajlov intanto, malgrado il ritratto di Anna lo avesse molto appassionato, fu ancora più contento di loro quando le sedute terminarono ed egli non fu più costretto a sentire il vaniloquio di Golenišcev sull'arte, e poté cancellare dalla memoria la pittura di Vronskij. Egli sapeva che non si poteva proibire a Vronskij di divertirsi con la pittura; sapeva che lui e tutti i dilettanti avevano il pieno diritto di dipingere quello che pareva loro, ma questo gli spiaceva. Non si può proibire a un uomo di farsi una gran bambola di cera e di baciarla. Ma se quest'uomo con la bambola venisse a sedere dinanzi a un innamorato e cominciasse ad accarezzare la bambola così come l'innamorato può accarezzare colei che ama, all'innamorato questo spiacerebbe. Un sentimento simile di sgradevolezza provava Michajlov alla vista della pittura di Vronskij; provava scherno e stizza, pena e offesa. 

La passione di Vronskij per la pittura e il medioevo non durò a lungo. Aveva tanto gusto in pittura che non poteva terminare il proprio quadro. Il quadro si fermò. Egli sentiva confusamente che i suoi difetti, poco avvertiti nell'abbozzo, sarebbero apparsi rilevanti, se avesse continuato. Gli accadeva la stessa cosa che accadeva a Golenišcev il quale sentiva di non aver nulla da dire, e ingannava continuamente se stesso col dire che il suo pensiero non era ancora maturo, che lo avrebbe compiuto e che preparava materiali. Ma Golenišcev era irritato e tormentato da tutto ciò, mentre Vronskij non poteva ingannarsi e tormentarsi e, soprattutto non poteva irritarsi. Con la decisione propria del suo carattere, senza spiegar nulla e senza giustificarsi, smise di occuparsi di pittura. 

Ma, senza questa occupazione, la sua vita e quella di Anna, sorpresa della delusione di lui, sembrarono così noiose nella cittadina italiana, così evidentemente vecchio e sudicio parve, a un tratto, il palazzo, così spiacevoli parvero le macchie sulle tende, le crepe sui pavimenti, lo stucco spaccato sui cornicioni e così tedioso parve il fatto di veder sempre lo stesso Golenišcev, il professore italiano e il viaggiatore tedesco, che bisognò cambiar vita. Decisero di andare in Russia, in campagna, a Pietroburgo. Vronskij aveva in mente di fare la spartizione dei beni col fratello e Anna di vedere il figlio. D'estate pensavano poi di andare nella grande tenuta di Vronskij. 

\capitolo{XIV}\label{xiv-4} 

Levin era ammogliato da tre mesi. Era felice, ma in modo del tutto diverso da come si aspettava. A ogni passo, egli provava una delusione per quello che aveva sognato e un incanto nuovo, inaspettato. Levin era felice, ma, entrato nella vita di famiglia, vedeva a ogni passo che era completamente diversa da quella che aveva immaginato. A ogni passo provava quello che prova l'uomo che, dopo aver ammirato il facile, sereno incedere di una barchetta su di un lago, segga egli stesso in quella barchetta. Vede che non basta star seduti senza ondeggiare, ma che bisogna riflettere, senza dimenticare neppure per un attimo la direzione, che sotto i piedi c'è l'acqua e che bisogna remare, e alle braccia non abituate ciò fa male, che soltanto stare a guardare è facile, ma farlo, anche se piacevole, è molto difficile. 

Da scapolo gli era accaduto, nell'osservare la vita matrimoniale altrui, di sorridere con sprezzo nell'animo suo delle preoccupazioni meschine, dei litigi, delle gelosie. Secondo lui, nella sua futura vita coniugale non solo non poteva esserci nulla di simile, ma anche tutte le forme esteriori dovevano essere completamente diverse da quelle degli altri, così almeno gli pareva. E a un tratto, invece, la sua vita coniugale non solo non si svolgeva in modo particolare, ma, al contrario, risultava tutta fatta di quelle stesse insignificanti meschinità, così disprezzate prima, e che adesso, contro il suo volere, acquistavano una straordinaria e incontestabile importanza. E Levin vedeva che la organizzazione di tutte queste piccole cose era tutt'altro che facile come gli sembrava prima. Pur credendo di avere le idee più chiare sulla vita familiare, Levin, come tutti gli uomini, immaginava istintivamente la vita di famiglia come una gioia di amare che nulla deve impedire, e dalla quale le piccole preoccupazioni non devono distogliere. Secondo lui, egli doveva attendere al proprio lavoro e prender riposo da questo nella gioia d'amare. Lei doveva essere amata e basta. Egli dimenticava, infatti, come tutti gli uomini, che anche lei doveva attendere a un suo lavoro. E si sorprendeva come lei, quella poetica e deliziosa Kitty, potesse non solo nelle prime settimane, ma nei primi giorni di vita in comune, pensare, affannarsi e ricordarsi delle tovaglie, dei mobili, delle materasse per gli ospiti, di un vassoio, del cuoco, del pranzo e via di seguito. Già quand'era stato fidanzato, si era sorpreso della sicurezza con cui ella aveva rinunciato al viaggio all'estero e aveva deciso di andare in campagna, quasi avesse avuto in mente qualcosa che si doveva fare, e quasi ella potesse pensare, oltre al suo amore, a un qualcosa che ne fosse al di fuori. Questo lo aveva offeso allora, e anche adesso le cure e le preoccupazioni meschine di lei, lo offendevano. Ma vedeva che questo le era indispensabile. E amandola, pur irridendo a queste preoccupazioni, senza saperne il perché, non poteva non compiacersene. Egli scherzava sul modo di disporre i mobili portati da Mosca, di attaccare le tendine, di adornare modernamente la sua stanza, di predisporre la distribuzione degli ospiti, di Dolly, di sistemare la sua nuova cameriera, di ordinare il pranzo al vecchio cuoco, di entrare in discussione con Agaf'ja Michajlovna, allontanandola dalla dispensa. Egli vedeva che il vecchio cuoco sorrideva, compiacendosi, nell'ascoltare gli ordini di lei, inesperti, assurdi; vedeva che Agaf'ja Michajlovna scoteva il capo pensosa e tenera alla nuove disposizioni della giovane signora nella dispensa; vedeva che Kitty era straordinariamente graziosa, quando, ridendo e piangendo, veniva da lui a dirgli che Maša, la cameriera, era abituata a considerarla signorina e che, perciò, nessuno le obbediva. Tutto questo gli sembrava grazioso, ma strano, e pensava che, senza tutto questo, sarebbero stati meglio. 

Non intendeva quale senso di mutamento ella provasse ora che, dopo aver desiderato, talvolta, a casa i cavoli col kvas o i confetti e non aver avuto né gli uni né gli altri, poteva ordinare tutto quello che voleva, comprare mucchi di confetti, spendere quanto denaro voleva e ordinare tutti i pasticcini che desiderava. 

Adesso sognava con gioia l'arrivo di Dolly con i bambini, proprio perché avrebbe potuto ordinare per i bambini i dolci preferiti da ognuno, e perché Dolly avrebbe apprezzato tutta la sua nuova organizzazione. Lei stessa non sapeva perché, ma le faccende domestiche l'attiravano irresistibilmente. Sentendo per istinto l'avvicinarsi della primavera e sapendo che ci sarebbero state anche le giornate grigie, intesseva, così come poteva, il suo nido, e si affrettava a intesserlo e a imparare a intesserlo nello stesso tempo. 

Quell'affannarsi di Kitty, fatto di piccole cose, così contrario all'ideale di una felicità più alta che Levin aveva nei primi tempi, era una delle delusioni, mentre questo stesso grazioso affannarsi di cui non capiva il senso, ma che non poteva non aver caro, era uno dei nuovi incanti. 

Un'altra delusione e un altro incanto erano i litigi. Levin non avrebbe mai potuto immaginare che tra lui e sua moglie potessero esservi altri rapporti oltre quelli teneri, comprensivi, pieni d'amore; e a un tratto, invece, fin dai primi giorni litigarono, in modo tale ch'ella gli rimproverò di non amarla, di amare solo se stesso e si mise a piangere e ad agitar le mani. 

Questo loro primo litigio avvenne perché Levin era andato alla nuova fattoria, vi si era trattenuto una mezz'ora di più, e, volendo passare per la strada più breve, vi si era smarrito. Andava a casa, pensando solo a lei, al suo amore, alla sua felicità, e quanto più si avvicinava, tanto più si accendeva in lui la tenerezza per lei. Corse in camera con quello stesso sentimento, e ancor più forte, con il quale era andato a casa Šcerbackij a far la sua domanda di matrimonio. E a un tratto, invece, lo accolse un'espressione torva, mai vista in lei. Voleva baciarla, ella lo respinse. 

- Cos'è successo? 

- Tu sei allegro\ldots{} - cominciò lei, cercando d'essere velenosamente calma. 

Ma aveva appena aperto la bocca, che parole di rimprovero, d'insensata gelosia e tutto quello che l'aveva tormentata nella mezz'ora trascorsa immobile accanto alla finestra, le sfuggirono. Soltanto ora, per la prima volta, egli capì con chiarezza quello che non aveva capito quando, dopo il rito, l'aveva condotta fuori della chiesa. Capì che, non solo ella gli era vicina, ma che ora non sapeva più dove finiva lei e dove cominciava lui. Capì ora questo, attraverso il tormentoso senso di sdoppiamento che provava. Si sentì offeso dapprima, ma nello stesso momento sentì che non poteva essere offeso da lei che era lui stesso. Provò in un primo momento, una sensazione simile a quella che prova un uomo che, ricevuto a un tratto un forte colpo alle spalle si volti con rabbia e con desiderio di vendetta per trovare il colpevole, e si convinca che è stato lui stesso a colpirsi involontariamente e non c'è contro chi arrabbiarsi e bisogna sopportare e placare il dolore. 

In seguito non provò mai più con tanta forza una sensazione simile, ma quella prima volta, a lungo, non riuscì a riaversi. Un sentimento istintivo pretendeva la giustificazione e la dimostrazione della colpa di lei; ma mostrare la colpa di lei significava irritarla maggiormente e aumentare quel distacco che era la causa di tutta la pena. Un sentimento consueto lo spingeva a scrollare da sé la colpa e a rigettarla su di lei; un altro sentimento più forte lo spingeva a eliminare presto, il più presto possibile, il distacco avvenuto, senza consentirgli di aumentare. Rimanere sotto un'accusa così ingiusta era tormentoso, ma, dopo essersi giustificato, farle del male, era ancora peggio. Come un uomo affannato dal dolore nel dormiveglia, egli voleva strappare, gettar via da sé il punto dolente e, tornato in sé, sentiva che il punto dolente era lui stesso. Bisognava soltanto aiutare a sopportare il punto dolente, ed egli si sforzò di far questo. 

Si rappacificarono. Lei, riconosciuta la propria colpa, anche senza confessarlo, divenne più tenera verso di lui, ed essi provarono una nuova raddoppiata gioia d'amore. Ma ciò non impedì che quegli urti si ripetessero, e anche con particolare frequenza, per i motivi più inaspettati e inconsistenti. Questi urti provenivano spesso anche dal fatto che essi non sapevano ancora che cosa fosse importante per loro e perché in tutto quel primo tempo fossero spesso di cattivo umore. Quando l'uno era di buon umore e l'altro di cattivo, la pace non veniva turbata ma quando tutti e due erano di cattivo umore, allora gli urti venivano fuori da così incomprensibili cause e così inconsistenti, che dopo non riuscivano in nessun modo a ricordare per quale motivo avessero litigato. È vero che, quando erano tutti e due di buon umore, la gioia della loro vita si raddoppiava; tuttavia questo primo periodo fu difficile per loro. 

Per tutto il primo periodo si sentì, in modo particolarmente vivo, una certa tensione, come il tendersi da una parte e dall'altra di quella catena che li aveva avvinti. In generale, quel mese di luna di miele, il primo mese di matrimonio, dal quale, secondo la tradizione, Levin si aspettava tanto, non solo fu senza miele, ma rimase nel ricordo di entrambi come il più difficile e umiliante periodo della loro vita. Tutti e due, in seguito, cercarono di cancellare dalla memoria le circostanze assurde e umilianti di quel periodo insano, in cui di rado erano stati tutti e due di umore normale, in cui di rado erano stati loro stessi. 

Soltanto nel terzo mese di matrimonio, dopo il ritorno da Mosca, dove avevano passato un mese, la loro vita divenne più piana. 

\capitolo{XV}\label{xv-4} 

Erano da poco arrivati da Mosca, ed erano contenti della loro solitudine. Lui nello studio, accanto al tavolo, scriveva. Lei, in quel vestito lilla scuro che aveva portato nei primi giorni del matrimonio e ora aveva indosso di nuovo perché così particolarmente caro e impresso nella memoria di lui, sedeva sul divano, su quello stesso vecchio divano di pelle, che era sempre stato nello studio, al tempo del nonno e del padre di Levin; ricamava una broderie anglaise. Egli pensava e scriveva, senza cessare di sentire con gioia la presenza di lei. Le sue occupazioni, che riguardavano l'azienda domestica e il libro, nel quale dovevano essere esposte le basi di una nuova economia, non erano state da lui abbandonate; ma come, tempo addietro, queste occupazioni e questi pensieri gli erano sembrati piccoli e insignificanti rispetto alle tenebre che ricoprivano tutta la sua vita, così ora, proprio allo stesso modo, piccoli e insignificanti gli sembravano rispetto alla vita futura inondata di una luce chiara di felicità. Continuava le sue occupazioni, ma sentiva che il centro di gravità della propria attenzione si era spostato su di un'altra cosa e che, in seguito a ciò, egli considerava il lavoro in maniera del tutto diversa e con maggiore chiarezza. Prima, per lui, quest'attività era un mezzo per salvarsi dalla vita. Prima sentiva che, senza quest'attività, la sua vita sarebbe stata troppo buia. Ora, invece, queste occupazioni gli erano indispensabili perché la vita non fosse troppo uniformemente luminosa. Avendo ripreso in mano le proprie carte, rileggendo quello che aveva scritto, constatò con soddisfazione che valeva la pena di occuparsene. Il lavoro era nuovo e utile. Molte idee di un tempo gli parvero superflue ed estremiste, ma molti problemi gli si chiarirono quando ravvivò nella memoria tutta l'opera. In questo momento scriveva un capitolo sulle cause della situazione svantaggiosa dell'agricoltura in Russia. Dimostrava che la povertà della Russia derivava non solo dalla ingiusta distribuzione della proprietà terriera e da una falsa tendenza, ma a ciò avevano cooperato negli ultimi tempi la civilizzazione straniera, introdotta in Russia anormalmente, e soprattutto le vie di comunicazione, le ferrovie, che avevano portato all'accentramento nelle città, al diffondersi del lusso e in seguito a questo, a tutto danno dell'agricoltura, allo sviluppo dell'industria manifatturiera, del credito e del suo satellite, il giuoco di borsa. Gli sembrava che, con uno sviluppo normale della ricchezza dello stato, tutti questi fenomeni si sarebbero fatti avanti solo quando nell'agricoltura si fosse già impiegato un lavoro considerevole, quando questa si fosse posta in condizioni regolari o almeno definite; che la ricchezza del paese dovesse crescere in modo uniforme e tale che gli altri settori della ricchezza non superassero l'agricoltura; che in conformità di una data situazione agricola dovessero essere anche le vie di comunicazione ad essa corrispondenti, e che, con la errata utilizzazione della terra, le ferrovie, volute non da una necessità economica, ma politica, fossero premature e, invece di incrementare l'agricoltura come ci si aspettava, l'avessero arrestata, superando l'agricoltura stessa e incrementando l'industria e il credito; che, perciò, come in un animale lo sviluppo unilaterale e prematuro d'un solo organo ostacolerebbe lo sviluppo generale, così, per lo sviluppo generale della ricchezza in Russia, il credito, le vie di comunicazione, lo sviluppo dell'attività manifatturiera, indubbiamente indispensabili in Europa dove erano state tempestive, avrebbero prodotto solamente danno, allontanando la questione principale, urgente, dell'organizzazione agricola. 

Mentre egli scriveva, ella pensava come suo marito fosse stato poco spontaneamente premuroso verso il giovane principe carskij, il quale, alla vigilia della partenza, con molto poco tatto, era stato galante con lei. ``Perché è geloso - ella pensava. - Dio mio! com'è simpatico e sciocco! È geloso di me! Se sapesse che tutti loro sono per me come Pëtr il cuoco! - ella pensava, guardando con un senso, per lei strano, di proprietà la nuca e il collo rosso di lui. - È peccato staccarlo dalle sue occupazioni, ma ne avrà del tempo, bisogna guardargli il viso; sentirà che lo guardo? Voglio che si volti\ldots{} lo voglio!'' ed ella aprì ancor più gli occhi, desiderando così rafforzare l'effetto del proprio sguardo. 

- Sì, attirano a sé tutte le linfe e dànno un falso splendore - mormorò, fermandosi nello scrivere, e, sentendo ch'ella lo guardava e sorrideva, si voltò. 

- Che c'è? - domandò, sorridendo e alzandosi. 

``S'è voltato'' ella pensò. 

- Niente, volevo che ti voltassi - disse lei, guardandolo e desiderando di indovinare se era irritato o no d'essere stato distratto. 

- Be', come stiamo bene noi due! Io, davvero - disse, accostandosi a lei e splendendo d'un sorriso di felicità. 

- Sto tanto bene, non andrò in nessun posto, specialmente a Mosca no. 

- E a che cosa pensavi? 

- Io? pensavo\ldots{} No, no, va', scrivi, non ti distrarre - disse lei increspando le labbra - anch'io adesso devo tagliare questi buchini, non vedi? 

Prese le forbici e cominciò a tagliare. 

- No, di', allora cosa? - disse, sedendosi accanto a lei e seguendo il movimento circolare delle forbicine. 

- A che cosa pensavo? Pensavo a Mosca, alla tua nuca. 

- Perché proprio a me una simile felicità? Non è naturale. È troppo bello - disse lui, baciandole la mano. 

- Per me, invece, meglio si sta, e più è naturale. 

- E tu hai una trecciolina - disse lui, volgendole il capo con delicatezza. - Una trecciolina\ldots{} Vedi, ecco qua. Sì, sì, noi lavoriamo proprio. 

Il lavoro non continuò, ed essi si scostarono, d'un tratto, come colpevoli, quando Kuz'ma entrò ad annunciare che il tè era pronto. 

- E dalla città sono venuti? - chiese Levin a Kuz'ma. 

- Sono venuti or ora; stanno dividendo le lettere. 

- Vieni presto, allora - ella gli disse, andando via dallo studio - altrimenti leggerò senza di te. E soniamo a quattro mani. 

Rimasto solo e riuniti i suoi quaderni in una cartella nuova, comperata da lei, cominciò a lavarsi le mani in un lavabo nuovo, elegante, con tutto l'occorrente, anche questo apparso con lei. Levin sorrideva ai propri pensieri e scoteva il capo con disapprovazione; un senso simile al rimorso lo tormentava. Qualcosa di vergognoso, di molle, di capuano, com'egli lo definiva, era nella sua vita di ora. ``Vivere così non è bene - pensava. - Ecco, tra poco sono tre mesi, e io non faccio quasi nulla. Quest'oggi, forse per la prima volta, mi son messo al lavoro, ebbene? Ho appena cominciato che ho lasciato andare. Perfino le mie occupazioni solite, anche quelle, ho lasciato andare. Non vado quasi neppure più, né a piedi né a cavallo, in giro per l'azienda. Ora mi fa pena lasciarla, ora vedo che lei si annoia. E io invece, prima del matrimonio, pensavo che la mia vita fosse mediocre, che non valesse e che solo dopo il matrimonio sarebbe cominciata la vera vita. Ma ecco che son quasi tre mesi, e io non ho mai passato il tempo così oziosamente e inutilmente. No, così non può andare, bisogna cominciare. S'intende, la colpa non è sua. A lei non si può rimproverare nulla. Devo essere io più deciso, devo difendere la mia indipendenza d'uomo. Altrimenti potrei io stesso abituarmi e fare abituare lei\ldots{} S'intende, la colpa non è sua'' egli si diceva. 

Ma era difficile per un uomo scontento non dar la colpa ad altri, e proprio alla persona che più di tutti gli era vicina, per quello di cui era scontento. E a Levin veniva confusamente in testa non che ella fosse colpevole (lei non poteva essere colpevole di nulla), ma che colpevole fosse la sua educazione troppo superficiale e frivola (``quello sciocco di carskij; lei, lo so, voleva, ma non sapeva fermarlo''). ``Sì, oltre l'interesse per la casa (questo ce l'ha), oltre il proprio abbigliamento e la broderie anglaise, non ha altri interessi seri. Né interesse per il mio lavoro, né per l'azienda, né per i contadini, né per la musica, in cui è pur esperta, né per la lettura. Non fa nulla ed è pienamente soddisfatta''. Levin fra di sé biasimava ciò, e non capiva ancora che ella si preparava a quel periodo di attività che stava per giungere per lei, quando sarebbe stata nello stesso tempo la moglie, la padrona di casa e avrebbe portati in sé, avrebbe allevato e educato i propri figli. Non capiva ch'ella lo sapeva per istinto e che, preparandosi a questo lavoro pieno di ansie, non si rimproverava i momenti di spensieratezza e di felicità d'amore che aveva adesso, intessendo lieta il proprio nido di domani. 

\capitolo{XVI}\label{xvi-4} 

Quando Levin andò di sopra, sua moglie era seduta vicino a un nuovo samovar d'argento, davanti a un nuovo servizio da tè e, posta a sedere accanto a un tavolino la vecchia Agaf'ja Michajlovna con la tazza di tè versatole, leggeva una lettera di Dolly, con la quale era in continua e frequente corrispondenza. 

- Ecco, la vostra signora m'ha messa a sedere, mi ha ordinato di sedere con lei - disse Agaf'ja Michajlovna, sorridendo benevola verso Kitty. 

In queste parole di Agaf'ja Michajlovna, Levin intuì lo scioglimento di un dramma accaduto in quegli ultimi tempi fra Agaf'ja Michajlovna e Kitty. Egli vedeva che, malgrado tutta l'amarezza provocata ad Agaf'ja Michajlovna dalla nuova padrona, che le aveva tolto le redini della direzione, Kitty l'aveva tuttavia conquistata e l'aveva costretta a volerle bene. 

- Ecco, ho letto anche una tua lettera - disse Kitty, porgendogli una lettera sgrammaticata. - È di quella donna, mi pare, di tuo fratello\ldots{} - ella disse. - Non ho finito di leggere. E questa è dai miei e da Dolly. Figurati! Dolly ha portato Griša e Tanja dai Sarmatskij a un ballo di bambini; Tanja era vestita da marchesa. 

Ma Levin non l'ascoltava; fattosi rosso, aveva preso la lettera di Mar'ja Nikolaevna, l'amante di un tempo di Nikolaj e aveva cominciato a leggerla. Era già la seconda lettera di Mar'ja Nikolaevna. Nella prima scriveva che il fratello l'aveva scacciata senza sua colpa, e con commovente ingenuità aggiungeva che, sebbene ella fosse di nuovo nella miseria, non chiedeva nulla, non desiderava nulla, ma che il pensiero che Nikolaj Dmitrievic si rovinasse senza di lei, la struggeva, e chiedeva al fratello di sorvegliarlo. Adesso scriveva un'altra cosa. Aveva ritrovato Nikolaj Dmitrievic, si era di nuovo unita a lui a Mosca ed era andata con lui in una città capoluogo di governatorato, dove egli aveva ricevuto un posto al servizio dello stato. Ma là egli aveva litigato con il capo ed era tornato a Mosca, e in viaggio si era tanto ammalato che difficilmente si sarebbe riavuto, ella scriveva. ``Ha sempre ricordato voi e anche di denari non ce n'è più''. 

- Leggi, Dolly scrive di te - stava per cominciare Kitty sorridendo, ma si fermò a un tratto, notando l'espressione mutata del viso del marito. 

- Che hai? Cos'è successo? 

- Mi scrive che Nikolaj sta per morire. Io vado. 

Il viso di Kitty si mutò a un tratto. I suoi pensieri su Tanja che faceva da marchesa, su Dolly, tutto scomparve. 

- Quando parti? 

- Domani. 

- E io vengo con te, posso? - ella disse. 

- Kitty, via, cos'è questo? - disse lui con rimprovero. 

- Come cosa? - domandò Kitty, offesa dal fatto ch'egli accogliesse controvoglia e con dispetto la sua proposta. - Perché non posso venire? Non ti darò fastidio. Io\ldots{} 

- Io vado perché mio fratello muore - disse Levin. - Perché tu\ldots{} 

- Perché io? per lo stesso motivo che ci vai tu. 

``E in un momento così grave per me, pensa solo che si annoierà a star sola'' pensò Levin. E questo pretesto, in un fatto così grave, lo irritò. 

- Non è possibile - disse severo. 

Agaf'ja Michajlovna, vedendo che la cosa andava a finire in un litigio, posò silenziosa la tazza e uscì. Kitty non l'aveva neppur notata. Il tono con cui suo marito aveva detto le ultime parole, la offese in modo particolare perché, evidentemente, egli non credeva a quello ch'ella diceva. 

- E io ti dico che se vai tu, verrò con te, verrò assolutamente - cominciò a dire in fretta e con rabbia. - Perché non è possibile? Perché dici che non è possibile? 

- Perché andare Dio sa dove, chi sa per quali strade, in quali alberghi. Tu mi sarai d'intralcio - rispose Levin, cercando di mantenere il suo sangue freddo. 

- Niente affatto. Io non ho bisogno di nulla. Dove puoi star tu, là anch'io\ldots{} 

- Ma via, non fosse altro che per il fatto che là c'è quella donna di cui tu non puoi fare la conoscenza. 

- Io non so nulla e non voglio saper nulla, chi ci sia e che cosa ci sia. Io so che il fratello di mio marito sta per morire e mio marito va da lui e io vado con mio marito per\ldots{} 

- Kitty, non ti arrabbiare. Ma pensa, questa è una cosa grave, mi spiace pensare che tu vi mescoli una sensazione di debolezza, il disappunto di rimanere sola. Via, se proprio ti è così noioso restar sola, vieni allora a Mosca. 

- Ecco, tu sempre mi attribuisci pensieri cattivi e volgari - cominciò a dire lei con lacrime d'offesa e di rabbia. - Nient'affatto, io non per debolezza, niente affatto\ldots{} Sento che il mio dovere è di stare con mio marito, quando egli soffre, ma tu vuoi farmi del male apposta, vuoi non capire apposta\ldots{} 

- No, questo è terribile. Questo significa essere schiavo! - proruppe Levin alzandosi e senza aver più la forza di contenere la propria stizza. Ma proprio in questo momento sentì che colpiva se stesso. 

- Allora perché ti sei sposato? Saresti libero. Perché, se te ne penti? - ella cominciò a dire, si alzò di scatto e corse in salotto. 

Quando egli andò a cercarla, piangeva singhiozzando. 

Egli cominciò a parlare, cercando quelle parole che potessero non dissuaderla, ma calmarla. Ma lei non l'ascoltava e non consentiva in nulla. Egli si chinò verso di lei e le prese la mano che gli faceva resistenza. Le baciò la mano, le baciò i capelli, le baciò ancora la mano; lei taceva sempre. Ma quando egli le prese il viso fra tutte e due le mani e disse: ``Kitty!'' ella a un tratto tornò in sé, pianse ancora e poi fece la pace. 

Fu deciso di partire l'indomani insieme. 

Levin disse alla moglie ch'egli aveva creduto al suo desiderio di andare solo per rendersi utile, acconsentì con lei che la presenza di Mar'ja Nikolaevna accanto al fratello non presentava nulla di sconveniente; ma in fondo all'animo partiva scontento di lei e di se stesso. Era scontento di lei perché non gli aveva permesso di allontanarsi quando era necessario (e come era strano che lui, che, fino a poco tempo addietro, non aveva coraggio di credere ch'ella lo amasse, ora si sentisse infelice perché lo amava troppo!), ed era scontento di sé perché non aveva mostrato carattere. Ancor meno era d'accordo, in fondo all'animo, che a lei non dovesse interessare la donna che era col fratello, e con terrore pensava a tutti gli urti che ne sarebbero potuti derivare. Già il solo fatto che sua moglie, la sua Kitty, sarebbe stata nella stessa stanza con una donna perduta, lo faceva rabbrividire di ribrezzo e di orrore. 

\capitolo{XVII}\label{xvii-4} 

L'albergo della città di provincia, nel quale giaceva a letto Nikolaj Levin, era uno di quegli alberghi di capoluogo che vengono costruiti secondo modelli perfezionati, con le migliori intenzioni di pulizia, comodità e perfino eleganza, ma che per il pubblico che li frequenta si trasformano con straordinaria celerità in locande sudice con pretese di modernità, e diventano, per queste stesse pretese, ancora peggiori dei vecchi alberghi semplicemente sudici. Questo albergo era già in questo stato; e un soldato con una divisa sporca e una sigaretta in bocca, che doveva far da portiere, e la scala di passaggio in ghisa, tetra e sgradevole, il cameriere disinvolto con un frac unto, e la sala centrale con un mazzo di fiori di cera impolverato che adornava la tavola, il sudiciume, la polvere, il disordine disseminati ovunque e, nello stesso tempo, quel certo nuovo presuntuoso affannarsi, modernamente collegato con il movimento ferroviario, dell'albergo, produssero sui Levin, dopo la loro recente vita di sposi, una penosa sensazione, specialmente perché l'impressione equivoca, prodotta dall'albergo, non si confaceva in nessun modo con quello che li aspettava. 

Come sempre, dopo la domanda sul prezzo della camera desiderata, risultò che nessuna camera buona era libera: una camera buona era occupata da un ispettore delle ferrovie, un'altra da un avvocato di Mosca, una terza dalla principessa Astaf'eva venuta dalla campagna. Ne rimaneva una sporca, accanto alla quale promisero di liberarne un'altra per la sera. Irritato con la moglie perché si avverava quello ch'egli aveva immaginato, che cioè al momento dell'arrivo, mentre egli aveva il cuore in ansia al pensiero del fratello, invece di correre subito da lui, avrebbe dovuto preoccuparsi di lei, Levin introdusse la moglie nella camera loro assegnata. 

- Va', va' - gli disse lei, guardandolo con uno sguardo timido, colpevole. 

Egli uscì in silenzio, e proprio là s'imbatté in Mar'ja Nikolaevna che aveva saputo del suo arrivo e non aveva avuto il coraggio di entrare da lui. Era proprio come l'aveva vista a Mosca: lo stesso vestito di lana, le stesse braccia e il collo scoperti e lo stesso viso butterato, benevolmente ottuso, un po' ingrassato. 

- Ebbene, che c'è? Come sta? che c'è? 

- Molto male. Non sta in piedi. Non faceva che aspettar voi. Lei\ldots{} Voi\ldots{} siete con vostra moglie\ldots{} 

Levin nel primo momento non capì quello che l'intimidiva, ma lei glielo spiegò subito. 

- Io andrò via, andrò in cucina - mormorò. - Sarà contento. Ha sentito, la conosce e la ricorda all'estero. 

Levin capì ch'ella intendeva sua moglie, e non sapeva cosa rispondere. 

- Andiamo, andiamo! - disse. 

Ma s'era appena mosso, che la porta della sua camera si aprì, e Kitty si sporse fuori. Levin arrossì di vergogna e di rabbia contro sua moglie, che metteva se stessa e lui in quella situazione penosa, ma Mar'ja Nikolaevna arrossì ancora di più. S'era tutta rattrappita e s'era fatta rossa fino alle lacrime e, afferrate con tutte e due le mani le punte del fazzoletto, le ravvolgeva con le dita rosse, non sapendo che dire e che fare. 

Nel primo attimo Levin vide un'espressione di avida curiosità nello sguardo con cui Kitty osservava quella donna per lei incomprensibile e paurosa; ma questo durò solo un attimo. 

- E dunque come sta? - disse rivolta al marito e subito dopo a lei. 

- Ma non si può discorrere nel corridoio! - disse Levin, guardando con stizza un signore che, tentennando sulle gambe, attraversava in quel momento il corridoio andandosene per i fatti suoi. 

- Su, allora, entrate - disse Kitty, rivolgendosi a Mar'ja Nikolaevna ch'era tornata in sé; ma, avendo notato il viso spaventato del marito: - Oppure andate, andate, andate e mandatemi a chiamare - disse, e rientrò in camera. Levin andò dal fratello. 

Egli non si aspettava per nulla quello che vide e sentì dal fratello. Si aspettava di trovare quello stato di autoinganno che, aveva sentito dire, i tisici hanno spesso, e che così fortemente lo aveva colpito durante il soggiorno autunnale del fratello. Si aspettava di trovare i segni fisici di una morte prossima più definiti, una maggiore debolezza, una maggiore magrezza, ma sempre quella stessa situazione. Si aspettava di provar lui stesso quel sentimento di pietà per la perdita del fratello prediletto e di orrore dinanzi alla morte che aveva provato allora, ma solo in grado più alto. E si preparava a questo; trovò, invece, una cosa affatto diversa. 

In una camera piccola, sudicia, coperta di sputi sui riquadri dipinti dei muri, di là dalla sottile intelaiatura dove si sentiva parlare, in un'aria impura di un lezzo soffocante, su di un letto scostato dal muro, giaceva un corpo sotto una coperta. Un braccio di questo corpo era al di sopra della coperta e la mano enorme, come un rastrello, di questo braccio era incomprensibilmente attaccata a un fuso sottile ed eguale dall'estremità al centro. La testa era adagiata di lato su di un guanciale. Levin poteva vedere i capelli sudati, radi sulle tempie, e la fronte tesa, quasi trasparente. 

``Non può essere che questo corpo orribile sia di mio fratello Nikolaj'' pensò Levin. Ma quando egli si fece dappresso e vide il viso, il dubbio non fu più possibile. Malgrado il pauroso mutamento del viso, a Levin bastò guardare quegli occhi vividi che si erano levati su di lui che entrava, bastò notare il leggero movimento della bocca sotto i baffi sottili, per capire la verità paurosa, che quel corpo morto era suo fratello vivo. 

Gli occhi scintillanti guardavano severi e accusatori il fratello che entrava. E subito, con questo sguardo, si stabilì un rapporto vivo tra vivi. Levin sentì una riprovazione nello sguardo fisso su di lui e provò rimorso per la propria felicità. 

Quando Konstantin lo prese per una mano, Nikolaj sorrise. Il sorriso era debole, appena percettibile, e nonostante il sorriso, l'espressione severa degli occhi non mutò. 

- Non ti aspettavi di trovarmi così - pronunciò a stento. 

- Sì\ldots{} no - diceva Levin confondendosi nelle parole\ldots{} - Come mai non mi hai fatto sapere prima, cioè anche durante il periodo del mio matrimonio? Ho chiesto notizie dappertutto. 

Bisognava parlare per non tacere, ma egli non sapeva che cosa dire, tanto più che il fratello non rispondeva e guardava soltanto, senza abbassare lo sguardo, penetrando, evidentemente, il senso di ogni parola. Levin comunicò al fratello che sua moglie era venuta con lui. Nikolaj mostrò piacere, ma disse che temeva di spaventarla col suo stato. Seguì un silenzio. Improvvisamente Nikolaj si mosse e cominciò a dire qualcosa. Levin si aspettava qualcosa di particolarmente importante e significativo dall'espressione del viso, ma Nikolaj cominciò a parlare della sua salute. Incolpava il dottore, rimpiangeva che non ci fosse il medico famoso di Mosca, e Levin capì ch'egli sperava ancora. 

Al primo momento di silenzio Levin si alzò, desiderando liberarsi, sia pure per un attimo, da quella sensazione tormentosa, e disse che andava a chiamare sua moglie. 

- Sì, sì, va bene, e io intanto dirò di pulire un poco qua. È tutto sporco, e ci deve puzzare, credo. Maša! Metti in ordine - disse a stento il malato. - E quando avrai messo in ordine, vattene - aggiunse, guardando interrogativamente il fratello. 

Levin non rispose nulla. Uscito nel corridoio, si fermò. Aveva detto che avrebbe condotto la moglie, ma ora, rendendosi conto del sentimento che provava, decise che, al contrario, avrebbe cercato di persuaderla a non andare dal malato. ``Perché deve tormentarsi come me?'' pensò. 

- Ebbene? che c'è? come? - chiese Kitty con viso spaventato. 

- Ah, è orribile, orribile! Perché sei venuta? - disse Levin. 

Kitty tacque per qualche secondo, guardando timida e implorante il marito; poi si avvicinò e con tutte e due le mani si afferrò al suo gomito. 

- Kostja! portami da lui, staremo meglio tutti e due. Purché tu mi porti, portami per favore, e poi te ne vai - ella cominciò a dire. - Devi capire che veder te e non veder lui mi è molto più penoso. Là forse io posso essere utile a te e a lui. Ti prego, promettimi! - ella supplicava il marito come se la felicità della sua vita dipendesse da questo. 

Levin dovette acconsentire e, riavutosi, dimentico del tutto di Mar'ja Nikolaevna, andò di nuovo dal fratello con Kitty. 

Con passo leggero, guardando sempre il marito e mostrandogli un viso coraggioso e compassionevole, ella entrò nella stanza del malato e, voltatasi con calma, chiuse l'uscio senza far rumore. A passi leggeri si avvicinò svelta al lettuccio del malato e, accostandosi in modo che egli non avesse da voltare il capo, prese subito nella sua mano fresca, giovane, lo scheletro enorme della mano di lui, la strinse e, con quella sommessa animazione compassionevole, ma non offensiva, propria solo delle donne, cominciò a parlare con lui. 

- Ci siamo incontrati, ma non ci conoscevamo, a Soden - ella disse. - Voi non pensavate che sarei diventata vostra sorella. 

- Non mi avreste riconosciuto? - egli disse con un sorriso che si era illuminato quando ella era entrata. 

- No, vi avrei riconosciuto. Come avete fatto bene a farcelo sapere! Non c'era giorno che Kostja non si ricordasse di voi e non ne fosse inquieto. 

Ma l'animazione del malato non durò a lungo. Ella non aveva finito di parlare che sul viso di lui si formò di nuovo l'espressione severa di rimprovero e di invidia di colui che muore verso chi vive 

- Temo che qui non stiate del tutto bene - ella disse, sottraendosi al suo sguardo fisso ed esaminando la stanza. - Occorre chiedere un'altra stanza al padrone - ella disse al marito - anche per essere più vicini. 

\capitolo{XVIII}\label{xviii-4} 

Levin non poteva guardare tranquillamente il fratello, non poteva essere naturale e calmo in sua presenza. Quando entrava dal malato i suoi occhi e la sua attenzione si velavano incoscientemente, ed egli non vedeva e non distingueva i particolari dello stato del fratello. Sentiva un lezzo tremendo, vedeva la sporcizia, il disordine, la situazione penosa, udiva i lamenti, e aveva la sensazione che non si potesse porre rimedio a questo. Non gli veniva neppure in mente che, esaminando i particolari della condizione del malato, pensando come giacesse, là sotto la coperta, quel corpo, come si disponessero, contraendosi, quelle gambe smagrite, i femori, la schiena, non fosse possibile disporre meglio, fare qualcosa che, sia pure non meglio, almeno fosse meno peggio. Il gelo gli penetrava nella schiena, quando cominciava a pensare a questi particolari. Era assolutamente convinto che non si potesse far nulla, né per prolungargli la vita, né per alleviargli le sofferenze. Ma la consapevolezza del fatto che egli riconosceva impossibile qualsiasi rimedio, era sentita dal malato e lo irritava. E perciò Levin si sentiva ancora più tormentato. Stare nella camera del malato gli era penoso, non starci, più penoso ancora. E continuamente, con vari pretesti, ne usciva e di nuovo vi entrava, senza aver la forza di rimaner solo. 

Kitty, invece, pensava, sentiva e agiva in modo del tutto diverso. Alla vista del malato ne aveva provato pietà. E la pietà, nell'animo suo di donna, aveva prodotto, invece della sensazione di orrore e di disgusto che aveva prodotto nel marito, la necessità di agire, di rendersi conto di tutti i particolari dello stato del malato e di aiutarlo. E poiché in lei non esisteva il più piccolo dubbio ch'ella dovesse portargli aiuto, non dubitava neppure del fatto che ciò fosse possibile, e si era messa subito all'opera. Quegli stessi particolari, il cui pensiero aveva prodotto orrore nel marito, richiamarono subito la sua attenzione. Mandò a chiamare un medico, mandò in farmacia, fece spazzare, spolverare, lavare dalla donna arrivata con lei e da Mar'ja Nikolaevna; lei stessa lavò, bagnò, pose qualcosa sotto la coperta. Per ordine suo portarono dentro e tolsero via qualcosa dalla camera del malato. Lei stessa andò varie volte nella propria camera senza far caso a chi incontrava; tirò fuori e portò lenzuola, federe, asciugamani, camicie. 

Il cameriere, che nella sala comune serviva il pranzo ad alcuni ingegneri, era venuto varie volte al suo richiamo con il viso irritato, ma non aveva potuto non eseguire gli ordini, poiché ella li dava in maniera così insistentemente dolce, che non ci si poteva in nessun modo sottrarre. Levin non approvava tutto questo; non credeva che ne venisse fuori qualcosa di utile per l'ammalato. Più di tutto, poi, temeva che l'ammalato si irritasse. Ma l'ammalato, pur mostrandosi indifferente, non si irritava, ma si vergognava soltanto, e in genere sembrava interessarsi a quello che ella gli andava facendo. Tornato dal dottore, dal quale Kitty l'aveva mandato, Levin, aperta la porta, trovò il malato nel momento in cui gli cambiavano la biancheria per ordine di Kitty. Lo scheletro lungo e bianco della schiena dalle scapole enormi, sporgenti e dalle costole e le vertebre in fuori, era scoperto, e Mar'ja Nikolaevna e il cameriere s'erano confusi in una manica della camicia e non riuscivano a dirigervi il braccio lungo, penzoloni. Kitty, che aveva chiuso in fretta la porta dietro a Levin, non guardava da quella parte; ma il malato si lamentò ed ella si diresse in fretta verso di lui. 

- Presto, via - ella disse. 

- Ma non venite - pronunciò il malato con rabbia - faccio da me\ldots{} 

- Che avete detto? - chiese Mar'ja Nikolaevna. 

Ma Kitty sentì e capì ch'egli si vergognava e che gli spiaceva di mostrarsi nudo davanti a lei. 

- Io non guardo, non guardo! - disse lei, dirigendo il braccio. - Mar'ja Nikolaevna, andate dall'altra parte, mettete a posto - soggiunse. 

- Dammi, per favore, la boccetta che ho nel sacchetto piccolo - ella disse rivolta al marito - sai, nel taschino di lato; portamela, per favore, e intanto qui metteranno in ordine. 

Tornando con la boccetta, Levin trovò il malato già disteso nel letto e tutto intorno completamente diverso. Il lezzo greve si era cambiato in un profumo d'aceto che Kitty, sporgendo le labbra e gonfiando le guance arrossate, andava spruzzando da un tubicino. Polvere non se ne vedeva in nessun posto, ai piedi del letto c'era un tappeto. Sulla tavola erano disposte le boccette, una caraffa, ed erano piegate la biancheria necessaria e il lavoro di Kitty di broderie anglaise. Su di un altro tavolo, accanto al letto del malato, c'erano una bevanda, una candela e delle polverine. Lo stesso infermo, lavato e pettinato, giaceva fra le lenzuola pulite, sui guanciali sollevati, con una camicia pulita dal colletto bianco intorno al collo estremamente esile e, con una nuova espressione di speranza, guardava Kitty senza staccarne gli occhi. Il dottore, trovato al club e condotto da Levin, non era quello che curava Nikolaj Levin e del quale egli era scontento. Il nuovo dottore tirò fuori l'astuccio e ascoltò l'ammalato, scosse il capo, prescrisse una medicina e spiegò con particolare minuzia come prendere la medicina, e quale dieta osservare. Egli consigliava uova crude o appena cotte e acqua di selz con latte fresco a una certa temperatura. Quando il dottore se ne andò, il malato disse qualcosa al fratello, ma Levin sentì solo le ultime parole: ``la tua Katja'' e, dallo sguardo con cui egli la guardò, Levin capì che ne faceva le lodi. Egli volle vicino a sé anche Katja, come la chiamava lui. 

- Sto già molto meglio - disse. - Ecco, con voi sarei guarito da tempo. Come sto bene! - Le prese la mano e l'accostò alle labbra, ma, quasi temendo che questo potesse spiacerle, cambiò idea, la lasciò andare e l'accarezzò soltanto. Kitty prese quella mano con tutte e due le sue e la strinse. 

- Adesso mettetemi sul lato sinistro e andate a dormire - egli pronunciò. 

Nessuno capì quello ch'egli aveva detto, Kitty sola capì. Ella capiva perché non desisteva dal seguire col pensiero quello che gli era necessario. 

- Dall'altro lato - ella disse al marito; - dorme sempre da quella parte. Fagli cambiar posizione, non sta bene chiamare la servitù. Io non posso. E voi non potete? - si rivolse a Mar'ja Nikolaevna. 

- Io ho paura - rispose Mar'ja Nikolaevna. 

Per quanto terribile fosse per Levin circondare con le braccia quel corpo terrificante, afferrare sotto la coperta quelle membra che voleva ignorare, tuttavia, sottomettendosi all'influenza della moglie, Levin fece il viso risoluto che sua moglie conosceva e, ficcate le mani, lo afferrò; tuttavia, malgrado la sua forza, fu sorpreso dallo strano peso di quelle membra sfinite. Mentre lo voltava e sentiva il proprio collo stretto dal braccio enorme, smagrito, Kitty, in fretta, senza far rumore, capovolse il guanciale, lo sprimacciò e vi accomodò la testa del malato e i suoi capelli radi, di nuovo appiccicati alle tempie. 

Il malato trattenne nella propria mano la mano del fratello. Levin sentiva ch'egli voleva fare qualcosa con la sua mano e che la tirava chi sa dove. Levin lasciava fare, sentendosi una stretta al cuore. Sì, egli la tirò verso la propria bocca e la baciò. Levin sussultò per i singhiozzi e, senza aver la forza di dir nulla, uscì dalla camera. 

\capitolo{XIX}\label{xix-4} 

``L'ha nascosto ai saggi e l'ha rivelato ai fanciulli e ai semplici'' così pensava Levin di sua moglie, discorrendo con lei quella sera. Levin pensava al detto del Vangelo non perché si considerasse un saggio. Non si considerava un saggio, ma non poteva non sapere d'essere più intelligente di Agaf'ja Michajlovna e non poteva non sapere che, pensando alla morte, ci pensava con tutte le forze dell'anima. Sapeva pure che molte grandi intelligenze maschili, di cui aveva letto le considerazioni sulla morte, ci pensavano, ma non sapevano neppure la centesima parte di quello che sapevano sua moglie e Agaf'ja Michajlovna. Per quanto diverse fossero queste due donne, Agaf'ja Michajlovna e Katja, come la chiamava suo fratello Nikolaj, e come adesso era particolarmente caro per Levin chiamarla, in questo erano perfettamente simili. Tutte e due sapevano con certezza che cosa fosse la vita e cosa fosse la morte, e pur senza saper rispondere e neppure capire le questioni che si presentavano a Levin, tutte e due non avevano dubbi sull'importanza di questo fenomeno e lo consideravano proprio allo stesso modo, d'accordo non solo fra di loro, ma dividendo questa loro concezione con milioni di uomini. La prova che esse sapessero con certezza cosa fosse la morte consisteva nel fatto che, senza un attimo di esitazione, sapevano come regolarsi con i moribondi, e non ne avevano paura. Levin e gli altri, invece, pur dissertando a lungo sulla morte, evidentemente non sapevano che cosa fare quando la gente muore. Se Levin in quel momento fosse stato solo con il fratello Nikolaj, l'avrebbe guardato con orrore, e con orrore ancora più grande avrebbe atteso, e nulla di più avrebbe saputo fare. 

Ed era ancora poco: egli non sapeva che cosa dire, come guardare, come camminare. Parlare di cose estranee gli sembrava offensivo, impossibile; parlare della morte, di cose tetre, non si poteva. Tacere neppure si poteva. ``Se lo guardo, penserà che lo osservo, che ho paura; se non lo guardo crederà che penso ad altro. Se cammino in punta di piedi, sarà scontento; ma poggiare tutto il piede, c'è da vergognarsi''. Kitty, invece, si vedeva, non pensava e non aveva il tempo di pensare a sé; pensava a lui, perché sapeva quella tale cosa, e tutto andava bene. Raccontava anche qualcosa di sé e del suo matrimonio, e sorrideva e lo compativa e lo carezzava, e parlava di casi di guarigione, e tutto andava bene; dunque ella sapeva. La prova che l'attività sua, come quella di Agaf'ja Michajlovna, non fosse istintiva, animale, irrazionale, consisteva nel fatto che oltre la cura fisica, oltre l'alleviamento delle sofferenze, sia Agaf'ja Michajlovna che Kitty esigevano per il moribondo qualcosa di ancor più importante della cura fisica, qualcosa che non aveva nulla di comune con le condizioni fisiche. Agaf'ja Michajlovna, parlando del vecchio che era morto, aveva detto: ``Ebbene, sia lodato Iddio! l'hanno comunicato, gli hanno dato l'estrema unzione, conceda Iddio a ognuno di morire così''. Katja, proprio alla stessa maniera, oltre tutte le preoccupazioni per la biancheria, per le piaghe, per le bevande, aveva fin dal primo giorno convinto il malato della necessità di comunicarsi e di ricevere l'estrema unzione. 

Rientrato in camera sua, al numero due, Levin sedette, col capo chino, non sapendo che fare. Senza parlare di cena, di sonno, senza riflettere a quello che avrebbe fatto, egli non poteva neanche parlare con sua moglie: si vergognava. Kitty, al contrario, era più attiva del solito. Era perfino più animata del solito. Ordinò di portare la cena, disfece lei stessa le valigie, lei stessa aiutò a fare i letti e non dimenticò di cospargerli di polvere persica. C'erano in lei l'eccitamento e l'intuito che appaiono negli uomini prima di un combattimento, di una lotta, nei momenti decisivi e pericolosi della vita, nei momenti in cui l'uomo mostra una volta per sempre il proprio valore e in cui tutto il suo passato non sembra vano, ma come una preparazione a questi momenti. 

Tutto il lavoro le riusciva, e non erano ancora le dodici, che già tutte le cose erano assestate con pulizia, con cura, come se proprio la camera d'albergo fosse simile alla casa, alla propria camera: i letti fatti, le spazzole, i pettini, gli specchietti messi fuori, i tovagliolini distesi. 

Levin trovava imperdonabile in questo momento il fatto di mangiare, dormire, di parlare perfino, e sentiva che ogni suo movimento era poco adatto. Lei, invece, metteva in ordine le spazzole e faceva tutto ciò senza che vi fosse nulla di offensivo. 

Però non poterono mangiar nulla e per lungo tempo non poterono prender sonno; anzi per un pezzo non riuscirono neppure a sdraiarsi per dormire. 

- Sono molto contenta di averlo convinto a ricevere domani l'estrema unzione - ella diceva, sedendo in vestaglia dinanzi allo specchio pieghevole e pettinando con un pettine fitto i suoi capelli morbidi, profumati. - Io non ho mai visto questa funzione; ma so, mamma me lo diceva, che ci sono delle preghiere per la guarigione. 

- Possibile che tu pensi ch'egli possa guarire? - disse Levin, guardando la sottile scriminatura che si chiudeva continuamente dietro alla piccola testa rotonda, appena ella faceva passare avanti il pettine. 

- L'ho domandato al dottore: ha detto che non potrà vivere più di tre giorni. Ma loro possono mai sapere ciò? Io intanto sono contenta d'averlo convinto - ella disse, guardando di lato il marito di là dai capelli. - Tutto può essere - ella soggiunse con quella espressione particolare, un po' accorta, che aveva sempre in viso quando parlava di religione. 

Dopo il loro colloquio sulla religione, quando erano ancora fidanzati, né lui né lei ne avevano mai più parlato; ma lei osservava le sue pratiche, frequentava la chiesa e pregava sempre con la tranquilla costante consapevolezza che così bisognasse fare. Malgrado le assicurazioni di lui sul contrario, ella era fermamente convinta che egli fosse cristiano come lei e ancora di più, e che tutto quello ch'egli diceva non fosse che una delle sue risibili uscite maschili, simili a quella che diceva a proposito della broderie anglaise, che, cioè, la povera gente rammendava i buchi e lei, invece, li tagliava a bella posta, e via di seguito. 

- Sì, ecco, quella donna, Mar'ja Nikolaevna, non sapeva organizzare tutto questo - disse Levin. - E\ldots{} devo riconoscere che sono molto, molto contento che tu sia venuta. Tu sei una tale purezza\ldots{} - Egli le prese la mano e non la baciò (baciarle la mano in quella prossimità di morte gli sembrava sconveniente), ma la strinse soltanto, guardando con un'espressione colpevole gli occhi di lei che s'erano illuminati. 

- Ti saresti talmente tormentato da solo - ella disse e, sollevando in alto le mani che coprivano le guance arrossite di soddisfazione, avvolse le trecce sulla nuca e le appuntò con le forcine. - No - ella continuò - lei non sapeva\ldots{} Io, per fortuna, ho imparato molte cose a Soden. 

- Possibile che là ci fossero malati simili? 

- C'era di peggio. 

- Per me è orribile non poterlo ritrovare così come era da giovane\ldots{} non puoi credere che giovane delizioso fosse, ma io allora non lo capivo. 

- Ci credo, ci credo. Come sento che saremmo stati amici, io e lui! - ella disse e, spaventata di quello che aveva detto, guardò il marito, e le lacrime le spuntarono negli occhi. 

- Sì, sareste stati - disse lui triste. - Ecco uno di quegli uomini di cui si dice che non sono fatti per questo mondo. 

- Ma abbiamo molti giorni dinanzi a noi, bisogna coricarsi - disse Kitty, dopo aver guardato il suo minuscolo orologio 

\capitolo{XX}\label{xx-4} 

\emph{LA MORTE} 

Il giorno dopo, il malato ricevette la comunione e l'estrema unzione. Durante il rito Nikolaj Levin pregava con ardore. Nei suoi grandi occhi, fissi sull'icona, posta su di un tavolo da giuoco ricoperto di un tovagliolino colorato, c'erano una preghiera e una speranza così appassionate, che Levin provava raccapriccio a guardare. Levin sapeva che questa preghiera appassionata e questa speranza avrebbero reso solo più penoso per lui il distacco dalla vita che amava tanto. Levin conosceva il fratello e il corso dei suoi pensieri; sapeva che la sua mancanza di fede era sorta non perché gli fosse più facile vivere senza una fede, ma perché di volta in volta le spiegazioni modernamente scientifiche dei fenomeni del mondo avevano soppiantato la fede, e perciò sapeva che questo suo ritorno alla fede non era legittimo, non era compiuto attraverso lo stesso pensiero, ma era soltanto momentaneo, interessato, per una folle speranza di guarigione. Levin sapeva pure che Kitty aveva rafforzato questa speranza con il racconto delle guarigioni da lei sentite. Tutto questo Levin lo sapeva, e gli era tormentoso vedere quello sguardo supplichevole, pieno di speranza, e quella mano scheletrica che si sollevava con stento a segnarsi sulla fronte stirata, sulle spalle sporgenti e sul petto vuoto rantolante, incapace di trattenere in sé quella vita che l'ammalato chiedeva. Durante la funzione Levin pregava e faceva proprio quello che lui, miscredente, aveva fatto mille volte. Diceva, rivolgendosi a Dio: ``Fa', se esisti, fa' che quest'uomo guarisca (questo si è pur verificato molte volte), e Tu salverai lui e me''. 

Dopo l'unzione il malato, a un tratto, migliorò molto. Per un'ora intera non tossì neppure una volta, sorrise, baciò la mano a Kitty, ringraziandola fra le lacrime, e disse di star bene, che non aveva dolore in nessun posto e che aveva appetito e si sentiva in forze. Si sollevò perfino da sé, quando gli portarono la minestra, e chiese anche una costoletta. Per quanto le sue condizioni fossero disperate, per quanto fosse evidente, guardandolo, che non poteva guarire, Levin e Kitty in quell'ora furono nella medesima eccitazione felice, ma timida, per la paura di sbagliare. 

``Meglio?''. ``Sì, molto''. ``Sorprendente''. ``Non c'è nulla di straordinario''. ``Tuttavia sta meglio'' essi dicevano sottovoce, sorridendosi l'un l'altro. 

Questa illusione non durò a lungo. Il malato si addormentò tranquillo, ma, dopo mezz'ora la tosse lo svegliò. E a un tratto scomparvero tutte le speranze e in coloro che lo circondavano e in lui stesso. La realtà della sofferenza distrusse, indubitatamente, le speranze di prima, senza lasciarne neppure più il ricordo, in Levin e in Kitty e nel malato stesso 

Senza ricordare neppure quello cui aveva creduto mezz'ora prima, quasi il ricordo fosse vergognoso, egli pretese che gli dessero dello iodio per inalazioni, in una fiala ricoperta da un pezzetto di carta con dei forellini. Levin gli porse il vasetto, e lo stesso sguardo di speranza appassionata, con cui il malato aveva ricevuto l'estrema unzione, si diresse adesso sul fratello, pretendendo da lui la conferma delle parole del dottore sul fatto che le inalazioni di iodio producono miracoli. 

- Katja non c'è? - rantolò lui, guardandosi in giro, quando Levin gli ebbe ripetuto stentatamente le parole del dottore. - No, allora si può dire\ldots{} Per lei ho fatto questa commedia. È così cara, ma noi due ormai non ci possiamo ingannare. Ecco, a questo io credo - disse e, afferrata la fiala con la mano ossuta, cominciò a respirarvi sopra. 

Dopo le sette di sera, Levin e la moglie prendevano il tè nella loro camera, quando Mar'ja Nikolaevna corse da loro trafelata. Era pallida e le labbra le tremavano. 

- Muore! - mormorò. - Ho paura che muoia subito. 

Tutti e due corsero da lui. Sollevatosi, egli stava seduto sul letto, i gomiti appoggiati, con la lunga schiena ricurva e la testa china. 

- Cosa ti senti? - chiese sottovoce Levin dopo un silenzio. 

- Sento che me ne vado - mormorò Nikolaj, articolando lentamente le parole, con fatica, ma con straordinaria precisione. Non sollevava il capo, ma volgeva in su soltanto gli occhi senza raggiungere con lo sguardo il viso del fratello. - Katja, va' via! - sussurrò ancora. 

Levin saltò su e con un mormorio imperioso la obbligò ad uscire. 

- Me ne vado - egli disse di nuovo. 

- Perché pensi questo? - disse Levin, tanto per dire qualcosa. 

- Perché me ne vado - egli ripeté come se avesse preso ad amare questa espressione. - È la fine. 

Mar'ja Nikolaevna si accostò. 

- Se vi sdraiaste, stareste meglio - ella disse. 

- Presto sarò disteso, quieto, morto - disse ironico, con rabbia. - Su, sdraiatemi, come volete. 

Levin adagiò il fratello sulla schiena, sedette accanto a lui e, senza respirare, gli osservava il viso. Il moribondo giaceva, con gli occhi chiusi, ma sulla fronte, sia pure di rado, si movevano dei muscoli, come in un uomo che pensi profondamente, con tensione. Levin istintivamente pensava con lui quello che in lui si compiva, eppure malgrado lo sforzo del pensiero per procedere insieme, vedeva dall'espressione di quel viso severo e calmo e dal movimento del muscolo al di sopra del sopracciglio, che il moribondo sempre più si chiariva quello che rimaneva sempre egualmente oscuro per Levin. 

- Sì, sì, così - pronunciò a intervalli, lentamente, il moribondo. - Aspettate. - Poi tacque di nuovo. - Così! - strascicò a un tratto tranquillamente, come se tutto fosse risolto per lui. - O Signore! - disse, e sospirò pesantemente. 

Mar'ja Nikolaevna gli tastò i piedi. 

- Si freddano - mormorò. 

A lungo, molto a lungo, come parve a Levin, il malato rimase immobile, disteso. Ma era ancora sempre vivo, e di quando in quando sospirava. Levin era già stanco per la tensione del pensiero. Sentiva, nonostante tutta la tensione del pensiero, che non poteva capire quello che era così. Sentiva di essersi già da tempo distaccato dal morente. Egli non poteva già più pensare alla stessa questione della morte, ma istintivamente gli sorgeva il pensiero di quello che avrebbe dovuto fare subito: chiudergli gli occhi, vestirlo, ordinare la bara. E, cosa strana, si sentiva completamente impassibile, e non sentiva né dolore, né distacco, e tanto meno pietà del fratello. Se adesso aveva un sentimento verso il fratello, era piuttosto un senso di invidia per quella conoscenza ch'egli, morente, aveva e che lui non poteva avere. 

Ancora a lungo rimase seduto così, curvo su di lui, aspettando sempre la fine. Ma la fine non veniva. La porta si aprì e apparve Kitty. Levin si alzò per fermarla. Ma mentre si alzava, sentì un movimento del morente. 

- Non te ne andare - disse Nikolaj e tese una mano. Levin gli diede la sua e fece un gesto irato verso la moglie, perché andasse via. 

Con la mano del morente nella propria, rimase lì seduto per mezz'ora, per un'ora, per un'altra ora. Ormai non pensava già più alla morte. Pensava a cosa stesse facendo Kitty, a chi si trovasse nella camera accanto, se il medico avesse, oppure no, una casa di sua proprietà. Aveva voglia di mangiare e di dormire. Liberò la mano con precauzione e tastò i piedi. I piedi erano freddi, ma il malato respirava. Levin voleva di nuovo uscire in punta di piedi, ma il malato si mosse di nuovo e disse: 

- Non te ne andare. 

\begin{center}\rule{3in}{0.4pt}\end{center} 

Si fece giorno; le condizioni del malato erano sempre le stesse. Levin, liberata pian piano la mano e senza guardare il morente, andò in camera sua e si addormentò. Quando si svegliò, invece della notizia della morte del fratello che si aspettava, seppe che il malato era nella condizione di prima. Aveva ripreso a sedersi, a tossire, aveva ricominciato a mangiare, s'era messo a parlare, e aveva di nuovo smesso di parlare di morte, aveva di nuovo cominciato a esprimere la speranza di guarire, e s'era fatto ancora più irritabile e tetro di prima. Nessuno, né il fratello, né Kitty potevano calmarlo. Si irritava con tutti e diceva a tutti cose spiacevoli, incolpava tutti delle proprie sofferenze e pretendeva che gli portassero il medico famoso da Mosca. A tutte le domande che gli facevano su come si sentiva, rispondeva nello stesso modo, con un'espressione di rancore e di rimprovero. 

- Soffro terribilmente, insopportabilmente! 

Il malato soffriva sempre di più e in modo particolare per le piaghe che ormai non riuscivano più a cicatrizzarsi, e si irritava sempre più contro quelli che gli stavano intorno, soprattutto perché non gli portavano il celebre medico di Mosca. Kitty cercava in tutti i modi di venirgli in aiuto, di calmarlo; ma tutto era inutile, e Levin vedeva che lei stessa era sfinita fisicamente e moralmente, sebbene non volesse riconoscerlo. Quel senso di morte che era stato suscitato in tutti dal suo addio alla vita, nella notte in cui aveva chiamato il fratello, si era disperso. Tutti sapevano che inevitabilmente e presto sarebbe morto, che era già morto a metà. Tutti desideravano soltanto una cosa, ch'egli morisse al più presto possibile, e tutti, nascondendolo, gli davano le medicine dalle fiale, cercavano rimedi e dottori, ingannando lui e loro stessi e ingannandosi l'un l'altro. Tutto questo era una menzogna, una disgustosa, offensiva e sacrilega menzogna. E questa menzogna, e per il carattere che gli era proprio, e perché era lui più di tutti che amava quegli che moriva, Levin la sentiva in modo particolarmente doloroso. 

Levin, che da tempo era preoccupato dal pensiero di rappacificare i fratelli almeno dinanzi alla morte, aveva scritto a Sergej Ivanovic e, ricevutane risposta, lesse la lettera al malato. Sergej Ivanovic scriveva che non poteva venire di persona, ma, con espressioni commoventi, chiedeva perdono al fratello. 

Il malato non disse nulla. 

- E cosa gli devo scrivere? - domandò Levin. - Spero che tu non sia arrabbiato contro di lui. 

- No, per nulla! - rispose Nikolaj stizzito da questa domanda. - Scrivigli che mi mandi il dottore. 

Passarono ancora tre giorni tormentosi; il malato era sempre nelle stesse condizioni. Il senso di desiderio della sua morte lo provavano adesso indistintamente tutti quelli che lo vedevano: i camerieri dell'albergo e il padrone, e i clienti, il medico e Mar'ja Nikolaevna, e Levin e Kitty. Soltanto il malato non esprimeva questo sentimento, al contrario, si irritava perché non gli portavano il dottore, e continuava a prendere la medicina e parlava di vivere. Soltanto in rari momenti, quando l'oppio lo costringeva per un attimo a smemorarsi delle ininterrotte sofferenze, egli diceva a volte, nel dormiveglia, quello che nell'anima sua era più forte che in tutti gli altri: ``ah, e fosse almeno la fine!'' oppure: ``quando finirà?''. 

Le sofferenze, crescendo uniformi, compivano l'opera loro e lo preparavano alla morte. Non c'era posizione in cui non soffrisse, non vi era un attimo in cui egli si assopisse, non vi era membro del corpo che non gli dolesse, che non lo tormentasse. Perfino i ricordi, le impressioni, i pensieri di quel corpo eccitavano ora in lui una repulsione. La vista di altre persone, i loro discorsi, i suoi ricordi personali, tutto questo era per lui solo tormento. Quelli che lo circondavano sentivano ciò e inconsciamente non si permettevano dinanzi a lui né movimenti liberi, né conversazioni, né la manifestazione dei loro desideri. Tutta la sua vita si fondeva in un solo senso di pena e nel desiderio di liberarsene. 

Evidentemente si compiva in lui quel rivolgimento che doveva portarlo a guardare alla morte come alla fine dei suoi desideri, come alla felicità. Prima, ogni singolo desiderio, provocato da una sofferenza o da una privazione, come la fame, la stanchezza, la sete, veniva soddisfatto con una funzione del corpo che dava piacere; ma adesso la privazione e la sofferenza non ricevevano soddisfazioni, anzi il tentativo di soddisfazione provocava una nuova sofferenza. E perciò tutti i desideri si fondevano in un unico desiderio: nel desiderio di liberarsi di tutte le sofferenze e della loro fonte, del corpo. Ma per esprimere questo desiderio di liberazione egli non aveva parole, e perciò non parlava, e per abitudine chiedeva il soddisfacimento di quei desideri che non potevano più essere soddisfatti. ``Mettetemi dall'altro lato'' diceva, e subito dopo voleva che lo mettessero così come prima. ``Datemi del brodo. Portate via il brodo. Raccontate qualcosa, perché state zitti?''. E così, appena cominciavano a parlare, chiudeva gli occhi ed esprimeva stanchezza, indifferenza e disgusto. 

Dieci giorni dopo il suo arrivo nella cittadina, Kitty si ammalò. Le venne mal di capo, vomito e per tutta la mattina non poté alzarsi dal letto. 

Il medico spiegò che il malanno derivava dalla stanchezza, dall'agitazione, e prescrisse serenità di spirito. 

Dopo pranzo però Kitty si alzò e andò come sempre, col lavoro, dal malato. Egli la guardò arcigno quand'ella entrò, e sorrise con sprezzo quando ella disse che era stata male. Quel giorno egli si soffiava il naso di continuo e si lamentava penosamente. 

- Come vi sentite? - ella chiese. 

- Peggio - rispose a stento. - Male! 

- Dove vi duole? 

- Dappertutto. 

- Quest'oggi finirà, guardate - disse Mar'ja Nikolaevna, sia pur sottovoce, ma in modo che il malato, che sentiva benissimo, come aveva notato Levin, doveva averla udita. Levin le fece segno di tacere e si voltò a guardare il malato. Nikolaj aveva sentito; ma queste parole non produssero nessuna impressione su di lui. Il suo sguardo era sempre teso e accusatore. 

- Perché pensate questo? - le domandò Levin quando ella uscì dietro di lui nel corridoio. 

- Ha cominciato a spogliarsi - disse Mar'ja Nikolaevna. 

- Come spogliarsi? 

- Ecco, così - ella disse, tirando le pieghe del suo vestito di lana. In realtà, egli aveva notato che tutto quel giorno il malato aveva afferrato quello che aveva addosso e pareva che volesse strappar via qualcosa. 

La previsione di Mar'ja Nikolaevna era giusta. Verso sera il malato non aveva più la forza di sollevare le braccia e guardava soltanto davanti a sé senza mutare l'espressione dello sguardo, attenta e concentrata. Perfino quando il fratello o Kitty si chinavano su di lui in modo ch'egli potesse vederli, egli guardava in quello stesso modo. Kitty mandò a chiamare il prete, per legger la preghiera degli agonizzanti. 

Mentre il prete leggeva la preghiera, il morente non dava alcun segno di vita; gli occhi erano chiusi. Levin, Kitty e Mar'ja Nikolaevna stavano in piedi accanto al letto. Il prete non aveva ancora finito di leggere, che il morente si stirò, sospirò e aprì gli occhi. Il prete, finita la preghiera, appoggiò sulla fronte fredda la croce, poi la ravvolse lentamente nella stola e, dopo aver sostato ancora due minuti in silenzio, toccò la mano enorme, divenuta fredda ed esangue. 

- È finito - disse il prete e voleva andar via; ma improvvisamente i baffi sottili del morente si mossero, e con chiarezza nel silenzio, emessi dal profondo del petto, si sentirono i suoni netti e precisi: 

- Non del tutto\ldots{} Presto. 

Dopo un istante il viso si rischiarò, sotto i baffi apparve un sorriso, e le donne, raccoltesi, si diedero a vestire il morto, affaccendandosi. 

La vista del fratello e la presenza della morte rinnovarono nell'animo di Levin quel senso di paura dinanzi all'inesplicabile inevitabilità della morte, che lo aveva sconvolto quella sera d'autunno, quando il fratello era giunto da lui. Questo senso, adesso, era ancora più forte; ancora meno di prima egli si sentiva in grado di capire il senso della morte, e ancora più terribile gliene appariva l'inevitabilità; ma ora, grazie alla vicinanza della moglie, questo senso non lo gettava nella disperazione: malgrado la morte, egli sentiva la necessità di vivere e di amare. Sentiva che l'amore lo salvava dalla disperazione e che l'amore, sotto la minaccia della disperazione, diveniva ancora più forte e puro. 

Dinanzi ai suoi occhi si era appena compiuto un mistero di morte, rimasto sempre inesplicabile, che ne sorgeva un altro, altrettanto inesplicabile, che richiamava all'amore e alla vita. 

Il medico confermò le sue supposizioni riguardo a Kitty. Il suo malessere era dovuto alla gravidanza. 

\capitolo{XXI}\label{xxi-4} 

Dal momento in cui Aleksej Aleksandrovic capì dalle spiegazioni di Betsy e di Stepan Arkad'ic che da lui si pretendeva non solo ch'egli lasciasse in pace sua moglie, senza affaticarla con la sua presenza, ma che sua moglie stessa desiderava ciò, egli si sentì così smarrito da non poter decidere nulla da solo, non sapendo lui stesso cosa desiderare in quel momento; così, abbandonandosi nelle mani di coloro che con tanto compiacimento si occupavano delle sue faccende, rispondeva acconsentendo a tutto. Solo quando Anna abbandonò la casa e la signorina inglese mandò a chiedere se dovesse pranzare con lui o a parte, egli capì, per la prima volta, con chiarezza, la propria posizione, e n'ebbe orrore. 

La cosa più difficile, in tutto questo, era ch'egli non riusciva in nessun modo a legare e a fondere il suo passato con quello che era avvenuto ora. Non quel passato in cui egli aveva vissuto felice con la moglie lo sconvolgeva. Il passaggio da quello alla conoscenza dell'infedeltà della moglie egli lo aveva già vissuto dolorosamente; e la condizione gli era stata penosa ma comprensibile. Se la moglie, allora, rivelata la propria infedeltà, fosse fuggita, egli si sarebbe addolorato, sarebbe stato infelice, ma non si sarebbe trovato in quella condizione senza via d'uscita, assurda, in cui si trovava ora. Non poteva in nessun modo legare insieme il suo recente perdono, la sua commozione, il suo amore per la moglie malata e per la creatura dell'altro, con quello che accadeva adesso, con il fatto, cioè, che, come a ricompensa di tutto, egli si trovava solo, ricoperto di vergogna, deriso, non utile a nessuno e disprezzato da tutti. 

I primi due giorni dopo la partenza della moglie, Aleksej Aleksandrovic ricevette i clienti, il capo di gabinetto, andò al comitato e pranzò in sala, come al solito. Senza rendersi conto perché facesse questo, tese tutte le forze dell'animo suo per avere un aspetto calmo e perfino indifferente. Rispondendo alle domande su come disporre le cose e le stanze di Anna Arkad'evna, faceva il più grande sforzo su di sé per avere l'aspetto di un uomo per il quale l'avvenimento accaduto non fosse imprevisto e non avesse in sé nulla che uscisse fuori dalla serie degli avvenimenti soliti, e otteneva lo scopo: nessuno poteva scorgere in lui i segni della disperazione. Ma due giorni dopo la partenza, quando Kornej gli presentò il conto di un magazzino di mode, che Anna aveva dimenticato di pagare, e riferì che il commesso stava lì, Aleksej Aleksandrovic fece chiamare il commesso. 

- Scusate, Eccellenza, se oso disturbarvi. Ma se Sua Eccellenza ordina di rivolgersi alla Signora, voglia compiacersi di comunicarne l'indirizzo. 

Aleksej Aleksandrovic si fece pensieroso, come parve al commesso, e a un tratto, voltatosi, sedette al tavolo. Chinata la testa fra le mani, rimase a lungo in questa posizione, più di una volta provò a parlare e si fermò. 

Kornej, che aveva capito i sentimenti del padrone, pregò il commesso di venire un'altra volta. Rimasto di nuovo solo, Aleksej Aleksandrovic capì di non avere più la forza di sostenere la parte dell'uomo fermo e calmo. Ordinò di staccare i cavalli dalla carrozza che aspettava, di non ricevere più nessuno e non uscì in sala a pranzare. 

Sentiva di non poter sostenere quell'attacco generale di sprezzo e di crudeltà che aveva scorto apertamente sul viso di quel commesso, di Kornej e, senza eccezione, di tutti quelli che aveva incontrato in quei due giorni. Sentiva di non poter allontanare da sé l'odio degli uomini, perché quest'odio non derivava dal fatto ch'egli fosse cattivo (in tal caso non avrebbe cercato di essere migliore), ma dal fatto ch'era infelice in maniera vergognosa e ripugnante. Sapeva che proprio perché il suo cuore era lacerato, gli uomini sarebbero stati senza pietà verso di lui. Sentiva che gli uomini l'avrebbero distrutto, così come i cani strozzavano un cane dilaniato che guaisce dal dolore. Sapeva che l'unico modo di salvarsi dagli uomini era nascondere loro le proprie ferite, e questo aveva cercato inconsciamente di fare per due giorni, ma adesso sentiva che già non aveva più la forza di continuare questa lotta impari. 

La sua disperazione aumentava perché sapeva di essere assolutamente solo con il suo dolore. Non soltanto a Pietroburgo non aveva neppure una persona alla quale confidare tutto quello che provava, che lo compatisse non come alto funzionario, non come membro della società, ma semplicemente come uomo che soffre, ma in nessun altro posto aveva una persona simile. 

Aleksej Aleksandrovic era cresciuto orfano. Erano due fratelli. Il padre non se lo ricordavano, la madre era morta quando Aleksej Aleksandrovic aveva dieci anni. Il loro patrimonio non era grande. Uno zio Karenin, funzionario importante e, un tempo, favorito del defunto zar, li aveva educati. 

Terminati i corsi al ginnasio e all'università con medaglie d'onore, Aleksej Aleksandrovic, con l'aiuto dello zio, intraprese subito una carriera burocratica di primo piano e da quel momento si dette tutto all'ambizione del funzionario. Né al ginnasio, né all'università, né più tardi in servizio, Aleksej Aleksandrovic aveva mai stretto relazioni di amicizia con alcuno. Il fratello era la persona spiritualmente più vicina a lui, ma, funzionario del ministero degli esteri, aveva vissuto sempre all'estero, dove poi era morto presto, subito dopo il matrimonio di Aleksej Aleksandrovic. 

Durante il suo governatorato, una zia di Anna, ricca signora di provincia, aveva presentato l'uomo non più giovane, ma giovane governatore, alla nipote e l'aveva messo in una situazione tale per cui egli doveva o dichiararsi o andar via dalla città. Aleksej Aleksandrovic esitò a lungo. In quel momento gli argomenti a favore di questo passo erano tanti quanti quelli contro, e non c'era un motivo determinante che lo costringesse a venir meno alla sua regola di astenersi, nel dubbio. Ma la zia di Anna gli fece giungere all'orecchio, per mezzo di un conoscente, ch'egli si era già compromesso con la ragazza e che un dovere di onore lo obbligava a far la sua domanda di matrimonio. Egli fece la sua richiesta, e dette alla fidanzata e alla moglie tutto quel sentimento di cui era capace. 

L'affetto che provava per Anna aveva escluso dall'animo suo le ultime esigenze di rapporti cordiali con le persone. E ora, fra tutti i suoi conoscenti, non ne aveva alcuno intimo. C'erano molte di quelle che si chiamano relazioni, ma rapporti di amicizia non ce n'erano. Aleksej Aleksandrovic aveva molte persone da poter invitare a pranzo a casa sua, alle quali poteva chiedere la partecipazione in un affare che lo interessava, una raccomandazione per qualche protetto, persone con le quali poteva giudicare le azioni delle alte personalità del governo, ma i rapporti con queste persone erano chiusi in un campo solo, fermamente circoscritto dall'uso e dall'abitudine. C'era un compagno d'università con il quale era stato in dimestichezza più tardi e con il quale avrebbe potuto parlare di un dolore suo personale; ma questo compagno era provveditore in una circoscrizione scolastica lontana. Delle persone poi che erano a Pietroburgo, le più vicine e le più adatte erano il direttore della cancelleria e il medico. 

Michail Vasil'evic Šljudin, capo di gabinetto, era un uomo semplice, intelligente, buono e morale, e in lui Aleksej Aleksandrovic sentiva una certa simpatia verso la propria persona; ma la loro attività di servizio, che durava da cinque anni, aveva frapposto una barriera alle spiegazioni intime. 

Aleksej Aleksandrovic, finita la firma delle carte, tacque a lungo, guardando di tanto in tanto Michail Vasil'evic e varie volte tentò, ma non gli riuscì di parlare. Aveva già preparato la frase: ``avete sentito del mio dolore''; ma finì col dire, come al solito: ``allora mi preparerete questo'' e così lo congedò. 

L'altra persona era il medico, che era anche lui ben disposto nei suoi riguardi; ma tra di loro si era già da tempo tacitamente convenuto che bisognava far presto perché erano entrambi carichi di affari. 

Alle sue amicizie femminili, alla prima fra queste, la contessa Lidija Ivanovna, Aleksej Aleksandrovic non pensava. Tutte le donne, così come donne, erano per lui terribili e insopportabili 

\capitolo{XXII}\label{xxii-4} 

Aleksej Aleksandrovic aveva dimenticato la contessa Lidija Ivanovna, ma lei non lo aveva dimenticato. Proprio in quel momento penoso di disperata solitudine, ella venne da lui e, senza farsi annunciare, entrò nel suo studio. Lo trovò nella stessa posizione nella quale si era seduto, con il capo appoggiato su tutte e due le mani 

- J'ai forcé la consigne - disse, entrando a passi svelti e ansante per l'agitazione e il movimento rapido. - Ho sentito tutto! Aleksej Aleksandrovic! - ella proseguì, stringendo forte, con tutte e due le mani, la mano di lui e guardandolo negli occhi con i suoi bellissimi occhi pensosi. 

Aleksej Aleksandrovic, aggrottando le sopracciglia, si alzò e, liberata la mano da lei, le accostò una sedia. 

- Volete favorire, contessa? Io non ricevo perché sono ammalato - disse e le labbra gli tremarono. 

- Amico mio! - ripeté la contessa Lidija Ivanovna, senza staccare gli occhi da lui, e improvvisamente le sopracciglia le si sollevarono dalla parte interna, formando un triangolo sulla fronte; il suo viso brutto e giallognolo divenne ancora più brutto; ma Aleksej Aleksandrovic sentiva che aveva pena di lui e che era pronta a piangere. E la commozione lo prese: afferrò la mano morbida di lei e cominciò a baciarla. 

- Amico mio! - ella disse con voce rotta dall'agitazione. - Voi non dovete abbandonarvi al dolore. Il vostro dolore è grande, ma dovete pur trovare consolazione. 

- Sono distrutto, ucciso, non sono più un uomo! - disse Aleksej Aleksandrovic, lasciando andare la mano di lei, ma continuando a guardarle gli occhi pieni di lacrime. - La mia situazione è orribile perché non trovo in nessun posto, non trovo in me stesso nessun punto d'appoggio. 

- Voi troverete appoggio, ma non lo cercate in me, sebbene io vi preghi di credere alla mia amicizia - ella disse con un sospiro. - L'appoggio nostro è l'amore, quell'amore che Egli ci ha lasciato. Il Suo peso è leggero - ella disse con quello sguardo entusiastico che Aleksej Aleksandrovic conosceva così bene. - Lui vi sosterrà e vi aiuterà. 

Sebbene in queste parole ci fosse la commozione di fronte ai propri elevati sentimenti e ci fosse quella nuova entusiastica disposizione mistica, recentemente diffusasi a Pietroburgo, che ad Aleksej Aleksandrovic era parsa oziosa, tuttavia in questo momento Aleksej Aleksandrovic provò piacere nel sentirle. 

- Sono debole. Sono annientato. Io non ho previsto nulla e ora non capisco nulla. 

- Amico mio! - ripeteva Lidija Ivanovna. 

- Non già la perdita di quello che adesso non è più, non è questo - continuava Aleksej Aleksandrovic. - Io non ho rimpianti. Ma non posso non vergognarmi davanti agli uomini per questa situazione in cui mi trovo. È male, ma non posso, non posso. 

- Non siete voi che avete compiuto l'alto gesto del perdono, del quale io sono entusiasta e con me tutti, ma Lui, che abitava nel vostro cuore - disse la contessa Lidija Ivanovna, sollevando entusiasticamente gli occhi - e perciò voi non potete vergognarvi della vostra azione. 

Aleksej Aleksandrovic si accigliò e, piegate le mani, cominciò a far scricchiolare le dita. 

- Bisogna conoscere tutti i particolari - egli disse con voce stridula. - Le forze umane hanno dei limiti, contessa, e io ho raggiunto i miei limiti. Oggi, per tutto il giorno, ho dovuto dare ordini, ordini che riguardavano la casa, derivanti - e sottolineò la parola ``derivanti'' - dalla mia nuova situazione di uomo solo. La servitù, la governante, i conti\ldots{} Questo fuoco sottile mi ha bruciato, non ho avuto la forza di sopportare. A pranzo\ldots{} ieri sono quasi scappato via da tavola. Non potevo sopportare come mio figlio mi guardava. Non mi chiedeva il senso di tutto questo, ma voleva chiederlo, e io non potevo sopportare quello sguardo. Lui aveva paura di guardarmi, ma questo è poco\ldots{} - Aleksej Aleksandrovic voleva dire del conto che gli avevano portato, ma la voce cominciò a tremare ed egli si fermò. Quel conto su carta azzurra per un cappello e dei nastri non poteva ricordarlo senza provar pena di se stesso. 

- Capisco, amico mio! - disse la contessa Lidija Ivanovna. - Io capisco tutto. L'aiuto e la consolazione voi dovete trovarli non in me, anche se non sono venuta che per aiutarvi, se posso. Se potessi alleviarvi tutte queste piccole preoccupazioni umilianti\ldots{} Io capisco che c'è bisogno della parola di una donna, di un ordine di donna. Date l'incarico a me? 

Aleksej Aleksandrovic le strinse la mano in silenzio, con gratitudine. 

- Ci occuperemo insieme di Serëza. Io non sono forte nelle cose pratiche. Ma mi ci proverò, sarò la vostra governante. Non mi ringraziate. Io non sono sola a farlo\ldots{} 

- Non posso non esservi grato. 

- Ma, amico mio, non vi abbandonate al sentimento cui avete accennato, vergognarsi di quanto c'è di più alto per il cristiano: ``chi si umilia sarà innalzato''. E ringraziare me non potete. Bisogna ringraziare Lui e chiedere aiuto a Lui. In Lui solo troveremo calma, consolazione, salvezza e amore - disse lei e, sollevati gli occhi al cielo, cominciò a pregare, come parve ad Aleksej Aleksandrovic dal suo silenzio. 

Aleksej Aleksandrovic adesso l'ascoltava, e quelle espressioni che prima, anche senza riuscirgli sgradite, gli parevano oziose, adesso gli parvero naturali e consolanti. Ad Aleksej Aleksandrovic non piaceva quel nuovo spirito entusiastico. Era un credente che si interessava alla religione soprattutto in senso politico, e la nuova dottrina che permetteva alcune interpretazioni, proprio perché apriva le porte alla discussione e all'analisi, gli spiaceva come principio. Egli prima si era mostrato freddo e persino ostile a questa nuova dottrina, e con la contessa Lidija Ivanovna, che ne era appassionata, non ne aveva discusso mai, e aveva evitato con cura, tacendo, le sue sfide. Ora invece, per la prima volta, egli ascoltava con piacere le sue parole, e intimamente non le avversava. 

- Vi sono molto, molto grato per quanto avete fatto e per quanto mi avete detto - disse, quando ella ebbe finito di pregare. 

La contessa Lidija Ivanovna strinse ancora una volta le mani del suo amico. 

- Ora mi accingo all'opera - ella disse con un sorriso, dopo aver taciuto un po' e asciugando sul viso i resti delle lacrime. - Vado da Serëza. Solo in casi estremi mi rivolgerò a voi. - Si alzò e uscì. 

La contessa Lidija Ivanovna stette una mezz'ora da Serëza e là, bagnando di lacrime le guance del ragazzo spaurito, gli disse che suo padre era un santo e che sua madre era morta. 

La contessa Lidija Ivanovna mantenne la promessa. Ella prese su di sé tutte le cure riguardanti l'organizzazione e la direzione della casa di Aleksej Aleksandrovic, ma non aveva esagerato nel dire che non era forte nelle cose pratiche. Tutto quello che veniva ordinato da lei doveva esser cambiato, perché non era eseguibile, e lo cambiava Kornej, il maggiordomo di Aleksej Aleksandrovic, che, senza farsi notare da nessuno, conduceva tutta la casa di Karenin con calma e prudenza, riferendo l'indispensabile mentre il padrone si vestiva. L'aiuto di Lidija Ivanovna fu, tuttavia, molto efficace: ella dette un appoggio morale ad Aleksej Aleksandrovic con la coscienza del suo amore e della sua considerazione per lui, e soprattutto col riportarlo al cristianesimo, pensiero per lei consolante; lo trasformò cioè da indifferente e pigro credente in un appassionato e convinto seguace di quella nuova interpretazione della dottrina cristiana che negli ultimi tempi si era diffusa a Pietroburgo. Aleksej Aleksandrovic poteva facilmente persuadersi di questa interpretazione. Aleksej Aleksandrovic, così come Lidija Ivanovna e le altre persone che condividevano le loro opinioni, era privo di una certa profondità di immaginazione, di quella facoltà, cioè, dell'anima, grazie alla quale le immagini suscitate dalla fantasia divengono così reali da pretendere la rispondenza con altre immagini e con la realtà. Egli non vedeva nulla di impossibile e di poco coerente nell'immaginare che la morte, esistente per coloro che non credono, per lui non esistesse, e che, possedendo egli la fede più piena, della cui intensità era giudice lui stesso, non ci fosse neanche più colpa nell'anima sua, e che qui sulla terra egli provasse già la salvezza completa. 

È vero che la vanità e l'errore di questa rappresentazione della propria fede erano confusamente avvertite da Aleksej Aleksandrovic, ed egli sapeva pure che, quando si era abbandonato a un sentimento immediato, senza pensare affatto che il proprio perdono fosse l'effetto di una forza superiore, aveva provato una felicità ben più grande di quella che provava ora pensando ogni momento che nell'anima sua c'era il Cristo e che egli, firmando le carte, eseguiva la Sua volontà; tuttavia, per Aleksej Aleksandrovic era indispensabile pensare in tal modo, gli era così indispensabile, nella sua umiliazione, avere quella superiorità spirituale, sia pure immaginaria, dalla quale lui, disprezzato da tutti, poteva disprezzare gli altri, che vi si teneva aggrappato come a una vera salvezza, a una sua salvezza immaginaria. 

\capitolo{XXIII}\label{xxiii-4} 

La contessa Lidija Ivanovna, giovanissima, piena di entusiasmi, era stata maritata a un ricco, nobile, cordiale e dissoluto gaudente. Dopo un mese, il marito l'aveva abbandonata, e alle entusiastiche assicurazioni di tenerezza di lei aveva opposto un'irrisione e perfino un'ostilità che le persone, cui era noto il buon cuore del conte e che non vedevano nessun difetto nell'entusiastica Lidija, non riuscivano a spiegarsi in nessun modo. Da quel momento, anche se non divorziati, vivevano divisi e quando il marito si incontrava con la moglie la trattava sempre con quella immutata, velenosa irrisione, di cui non si riusciva a capire la causa. 
\enlargethispage*{1\baselineskip}

La contessa Lidija Ivanovna aveva cessato da tempo di essere innamorata del marito, ma da allora non aveva mai smesso di essere innamorata di qualcuno. Le accadeva di innamorarsi di varie persone nello stesso tempo, di uomini e donne; le accadeva di innamorarsi di quasi tutte le persone che eccellevano in un qualche modo. Era stata innamorata di tutte le nuove principesse e dei principi che entravano a far parte della famiglia imperiale, di un metropolita, di un vicario e di un prete. Era stata innamorata di un giornalista, di tre slavi e di Komisarov, d'un ministro, d'un dottore, d'un missionario inglese e di Karenin. Tutti questi amori, ora più deboli, ora più forti, non le avevano impedito di estendere i più larghi e complicati rapporti nella corte e nel gran mondo. Ma dal momento in cui la sventura aveva colpito Karenin, ed ella l'aveva preso sotto la sua personale protezione, dal momento in cui s'era affaccendata in casa Karenin prendendosi cura del suo buon andamento, ella aveva sentito che tutti i precedenti amori non erano mai stati veri, e che adesso era veramente innamorata del solo Karenin. Il sentimento ch'ella provava in questo momento verso di lui, le sembrava più forte di tutti i precedenti amori. Analizzando il proprio sentimento e paragonandolo con i precedenti, vedeva con chiarezza che non si sarebbe mai innamorata di Komisarov se costui non avesse salvato la vita all'imperatore, che non si sarebbe innamorata di Ristic-Kudzickij se non ci fosse stata la questione slava, ma che Karenin ella lo amava per lui stesso, per la sua alta anima incompresa, per il tono sottile a lei caro della voce, dalle intonazioni strascicate, per il suo sguardo stanco, per il suo carattere, per le sue mani morbide dalle vene gonfie. Ella non solo gioiva di un incontro con lui, ma cercava sul viso di lui i segni dell'impressione che lei stessa produceva. Voleva piacergli non solo con i discorsi, ma con tutta la persona. Adesso per lui si occupava del proprio abbigliamento molto più di prima. Si sorprendeva a far sogni su quello che sarebbe stato se lei non fosse stata maritata e lui libero. Arrossiva di agitazione quando egli entrava nella stanza; non poteva trattenere un sorriso d'entusiasmo quando egli le diceva qualcosa di piacevole. 

Già da qualche giorno la contessa Lidija Ivanovna si trovava in grande agitazione. Aveva saputo che Anna e Vronskij erano a Pietroburgo. Bisognava salvare Aleksej Aleksandrovic da un incontro con lei, bisognava salvarlo perfino dalla tormentosa notizia che quella donna terribile era nella stessa città in cui era lui, e che ad ogni momento egli poteva incontrarla. 

Lidija Ivanovna, per mezzo dei suoi conoscenti, si informava di quello che avevano intenzione di fare quelle ``persone ripugnanti'', come ella chiamava Anna e Vronskij, e cercava di guidare, in quei giorni, tutti i movimenti del suo amico, affinché non avesse a incontrarli. Il giovane aiutante di campo, amico di Vronskij, attraverso il quale ella riceveva le notizie e che dalla contessa Lidija Ivanovna sperava di ricevere una concessione, le disse che avevano ultimato i loro affari e che l'indomani sarebbero partiti. Lidija Ivanovna aveva già cominciato a rasserenarsi quando, proprio la mattina dopo, le portarono un biglietto di cui ella riconobbe con orrore la scrittura. Era la scrittura di Anna Karenina. La busta era di carta doppia come scorza d'albero; sul foglio giallo, oblungo, c'era un enorme monogramma, e la lettera emanava un ottimo profumo. 

- Chi l'ha portato? 

- Un inserviente d'albergo. 

La contessa Lidija Ivanovna per un pezzo non riuscì a mettersi a sedere per leggere la lettera. L'agitazione le provocò un attacco d'asma, cui andava soggetta. Quando si fu calmata, lesse la seguente lettera in francese. 

\begin{quote}
``Madame la Comtesse, 

i sentimenti cristiani che riempiono il vostro cuore mi dànno, lo sento, l'imperdonabile ardire di scrivervi. Io sono infelice per il distacco da mio figlio. Supplico voi per ottenere il permesso di vederlo una volta prima della mia partenza. Perdonatemi se mi rivolgo a voi. Mi rivolgo a voi e non ad Aleksej Aleksandrovic, perché non voglio far soffrire quest'uomo generoso col ricordarmi a lui. Conoscendo la vostra amicizia per lui, mi comprenderete. Manderete Serëza da me, o devo venire io in casa a una determinata ora, o mi farete sapere quando e dove posso vederlo fuori? Non suppongo un rifiuto, conoscendo la generosità di colui dal quale ciò dipende. Voi non potete immaginare come sia ansiosa di vedere mio figlio, e quanto vi sia grata per il vostro aiuto. 

Anna
\end{quote}
\enlargethispage*{1\baselineskip}

Tutto in questa lettera irritò la contessa Lidija Ivanovna: e il contenuto, e l'accenno alla generosità e, soprattutto, il tono, come le sembrava, disinvolto. 

- Di' che non ci sarà risposta - disse la contessa Lidija Ivanovna e subito, aperta una cartella, scrisse ad Aleksej Aleksandrovic che sperava vederlo dopo mezzogiorno, durante gli auguri a corte. 

``Ho bisogno di parlare con voi per una faccenda triste e importante. Là stabiliremo dove. Meglio di tutto da me, dove farò preparare il vostro tè. È indispensabile. Egli ci impone la croce, ma Egli ci dà la forza'' aggiunse per prepararlo. 

La contessa Lidija Ivanovna scriveva, di solito, due o tre biglietti al giorno ad Aleksej Aleksandrovic. Per comunicare con lui amava questo sistema, che aveva quella certa raffinata misteriosità che non avevano i suoi rapporti personali. 

\capitolo{XXIV}\label{xxiv-3} 

Gli auguri erano finiti. Quelli che andavano via, incontrandosi, parlavano dell'ultima novità del giorno, delle ricompense appena ricevute e del cambiamento degli alti funzionari. 

- Se dessero il ministero della guerra alla contessa Mar'ja Borisova e facessero capo di Stato Maggiore la principessa Vatkovskaja - diceva rivolto a una damigella d'onore, bella e slanciata, che gli aveva chiesto del movimento, un vecchietto canuto con l'uniforme ricamata in oro. 

- E me aiutante di campo - rispondeva la damigella, sorridendo. 

- Voi l'avete già la nomina. A voi per l'amministrazione ecclesiastica e, per aiuto, Karenin. 

- Buon giorno, principe! - disse il vecchietto, stringendo la mano a colui che si era avvicinato. 

- Cosa dicevate di Karenin? - disse il principe. 

- Che lui e Putjatov hanno ricevuto l'Aleksandr Nevskij. 

- Pensavo che l'avesse già. 

- No, guardatelo - disse il vecchietto, indicando con il cappello ricamato Karenin in uniforme di corte e con la nuova fascia rossa a tracolla, fermo sulla porta della sala con uno dei membri influenti del consiglio di stato. - È felice e contento, come un soldone di rame - aggiunse, fermandosi per stringere la mano a un bel ciambellano dalla statura atletica. 

- No, è invecchiato - disse il ciambellano. 

- Per le preoccupazioni. Adesso non fa che scrivere progetti. Non lascerà il disgraziato finché non gli avrà esposto tutto. 

- Come invecchiato? Il fait des passions. Penso che ora la contessa Lidija Ivanovna sarà gelosa della moglie per lui. 

- Ma cosa! Per favore, non parlate della contessa Lidija Ivanovna. 

- Ma è forse male che sia innamorata di Karenin? 

- Ma è vero che la Karenina è qui? 

- Cioè, non qui a corte, ma a Pietroburgo. L'ho incontrata ieri, con Aleksej Vronskij, bras dessus bras dessous, per la Morskaja. 

- C'est un homme qui n'a pas\ldots{} - cominciò il ciambellano, ma si fermò cedendo il passo e salutando un personaggio della famiglia imperiale. 

Così si parlava, senza posa, di Aleksej Aleksandrovic, giudicandolo e irridendolo, mentre lui, sbarrata la strada al membro del consiglio di stato che aveva afferrato, e senza smettere neppure per un attimo la propria esposizione per non lasciarselo sfuggire, gli esponeva punto per punto un suo progetto finanziario. 

Quasi nello stesso tempo in cui la moglie aveva abbandonato Aleksej Aleksandrovic, gli era accaduto l'avvenimento più doloroso per un funzionario: la fine del movimento di ascesa nella carriera. Questa fine si era verificata e tutti vedevano ciò chiaramente, ma lui, Aleksej Aleksandrovic, non riconosceva ancora che la sua carriera era finita. Fosse l'urto con Stremov, fosse la disgrazia con la moglie, o fosse che Aleksej Aleksandrovic era giunto al limite che gli era stato assegnato, certo è che quell'anno era evidente che la sua carriera di funzionario era finita. Occupava ancora un posto importante, era membro di molte commissioni e comitati; ma era ormai l'uomo che s'era esaurito tutto e dal quale non ci si attendeva più nulla. Qualunque cosa egli dicesse, qualunque cosa proponesse, lo si ascoltava come se quello che proponeva fosse da gran tempo noto e fosse proprio quello che non era necessario. 

Ma Aleksej Aleksandrovic non lo sentiva; al contrario, allontanatosi dalla partecipazione diretta all'attività governativa, vedeva ora più chiaramente di prima i difetti e gli errori dell'attività degli altri, e stimava suo dovere indicare i mezzi per correggerli. Ben presto, dopo la sua separazione dalla moglie, cominciò a scrivere i suoi primi appunti sul nuovo tribunale, primi dell'innumere serie di appunti inutili su tutti i rami dell'amministrazione che era destinato a scrivere. 

Aleksej Aleksandrovic non solo non avvertiva la sua posizione senza speranza nel mondo burocratico e non solo non se ne amareggiava, ma era più che mai contento della propria attività. 

``Chi ha moglie si cura delle cose mondane, si preoccupa di compiacere la moglie; chi non è ammogliato si preoccupa delle cose del Signore, si preoccupa di compiacere il Signore'' dice l'apostolo Paolo, e Aleksej Aleksandrovic, che adesso si lasciava guidare in tutte le sue cose dalla Sacra Scrittura, ricordava spesso questo testo. Gli pareva, dal momento in cui era rimasto senza moglie, di servire, con quei progetti, il Signore più di prima. 

L'evidente impazienza del membro del consiglio di stato, che voleva sfuggirgli, non scomponeva Aleksej Aleksandrovic; egli smise di esporre soltanto quando il membro, profittando del passaggio di un personaggio di casa imperiale, scivolò via. 

Rimasto solo, Aleksej Aleksandrovic abbassò la testa, raccogliendo le idee, poi si voltò distrattamente e si diresse verso una porta presso la quale sperava di incontrare la contessa Lidija Ivanovna. 

``E come sono tutti fisicamente forti e sani! - pensò Aleksej Aleksandrovic, guardando le fedine profumate, ben in ordine, di un aitante ciambellano e il collo rosso di un principe stretto nell'uniforme, accanto al quale doveva passare. - È giusto dire che nel mondo tutto è male'' pensò guardando ancora una volta di traverso i polpacci del ciambellano. 

Movendo le gambe lentamente, Aleksej Aleksandrovic, con la solita aria di stanchezza e di dignità, si inchinò a quei signori che parlavano di lui, e, guardando la porta, cercò con gli occhi la contessa Lidija Ivanovna. 

- Ah! Aleksej Aleksandrovic! - disse il vecchietto, con gli occhi che scintillavano maligni, nel momento in cui Karenin lo raggiungeva e chinava il capo con gesto freddo. - Non mi sono ancora congratulato con voi - disse, indicando la fascia appena ricevuta. 

- Vi ringrazio - rispose Aleksej Aleksandrovic. - Che bella giornata! - soggiunse, secondo la sua abitudine, sottolineando in modo particolare la parola ``bella''. Che lo irridessero lo sapeva, ma da loro non si aspettava null'altro che ostilità; vi si era già abituato. 

Viste le spalle giallognole, sporgenti dal busto, della contessa Lidija Ivanovna e gli occhi pensosi di lei, belli e invitanti, Aleksej Aleksandrovic sorrise, scoprendo i suoi denti bianchi, inalterabili, e le si avvicinò. 

La toletta di Lidija Ivanovna le era costata molta fatica, come, del resto, sempre il suo abbigliamento in questi ultimi tempi. Adesso, lo scopo del suo abbigliamento era del tutto opposto a quello che ricercava trent'anni prima. Allora le piaceva abbellirsi per qualche cosa, e quanto più tanto meglio. Adesso, invece, era immancabilmente adornata in modo così poco conveniente alla sua età e alla sua figura, che si preoccupava solo di non rendere troppo stridente il contrasto fra gli ornamenti e il suo aspetto esteriore. E nei riguardi di Aleksej Aleksandrovic aveva raggiunto lo scopo, ché a lui ella sembrava attraente. Per lui ella era l'unica isola non solo di simpatia, ma di amore, in mezzo al mare di ostilità e di irrisione che lo circondava. 

Passando attraverso la fila degli sguardi di scherno, egli tendeva istintivamente verso lo sguardo innamorato di lei, come una pianta verso la luce. 

- Mi rallegro - gli disse, indicando la fascia con gli occhi. 

Trattenendo un sorriso di compiacimento, egli alzò le spalle e socchiuse gli occhi, come a dire che questo non lo poteva rallegrare. La contessa Lidija Ivanovna sapeva bene che questa era una delle sue più grandi gioie, anche se egli non l'avrebbe mai confessato. 

- Come va il nostro angelo? - disse la contessa Lidija Ivanovna, intendendo Serëza. 

- Non posso dire d'essere pienamente contento di lui - disse Aleksej Aleksandrovic sollevando le sopracciglia e aprendo gli occhi - e anche Sitnikov non è contento. - Sitnikov era il precettore al quale era stata affidata l'educazione mondana di Serëza. - Come vi dicevo, c'è in lui una certa freddezza per le questioni importanti che devono commuovere l'animo di ogni uomo e di ogni fanciullo - e Aleksej Aleksandrovic cominciò a esporre le proprie idee sull'unica questione che lo interessava oltre l'ufficio, l'educazione del figlio. 

Quando Aleksej Aleksandrovic, con l'aiuto di Lidija Ivanovna, era tornato alla vita e all'attività, aveva sentito il dovere di occuparsi dell'educazione del figlio che era rimasto con lui. Non essendosi mai occupato prima di questioni di educazione, Aleksej Aleksandrovic aveva dedicato un po' di tempo allo studio teorico della materia. Letti alcuni libri di antropologia, di pedagogia e di didattica, Aleksej Aleksandrovic si era formato un piano di educazione e, fatto venire il miglior pedagogo di Pietroburgo per la direzione, si era accinto all'opera. E questa opera lo teneva continuamente occupato. 

- Sì, ma il cuore? Io vedo in lui il cuore del padre e, con un cuore simile, un bambino non può essere cattivo - disse Lidija Ivanovna con entusiasmo. 

- Sì, può darsi\ldots{} Per quel che riguarda me, io compio il mio dovere. È tutto quello che posso fare. 

- Venite da me - disse la contessa Lidija Ivanovna, dopo un certo silenzio; - dobbiamo parlare di una cosa triste per voi. Io darei tutto per liberarvi di alcuni ricordi, ma gli altri non la pensano così. Ho ricevuto una lettera da lei. Lei è qui, a Pietroburgo. 

Aleksej Aleksandrovic rabbrividì al ricordo della moglie, e immediatamente sul suo viso apparve quella fissità di morte che esprimeva un completo abbandono in questa faccenda. 

- Me l'aspettavo - disse. 

La contessa Lidija Ivanovna lo guardò con entusiasmo, e lacrime di estasi le vennero agli occhi dinanzi alla grandezza dell'anima di lui. 

\capitolo{XXV}\label{xxv-3} 

Quando Aleksej Aleksandrovic entrò nello studio piccolo, accogliente, pieno di porcellane antiche e di ritratti, della contessa Lidija Ivanovna, la padrona non c'era ancora. Si stava cambiando. 

Sulla tavola rotonda era distesa una tovaglia e c'era un servizio cinese e una teiera d'argento a spirito. Aleksej Aleksandrovic guardò distrattamente, in giro, gli innumerevoli ritratti noti che adornavano lo studio e, sedutosi presso la tavola, aprì il Vangelo che vi stava sopra. Il fruscio del vestito di seta della contessa lo distrasse. 

- Su, ecco, adesso ci metteremo a sedere tranquillamente - disse la contessa Lidija Ivanovna, con un sorriso agitato, insinuandosi in fretta fra il divano e la tavola - e parleremo prendendo il tè. 

Dopo alcune parole di preparazione, la contessa Lidija Ivanovna, respirando pesantemente e diventando rossa, consegnò nelle mani di Aleksej Aleksandrovic la lettera da lei ricevuta. 

Finita di leggere la lettera, egli tacque a lungo. 

- Io non credo di avere il diritto di rifiutare ciò - egli disse timido, sollevando gli occhi. 

- Amico mio, voi non vedete in nessuno il male! 

- Io, al contrario, vedo che tutto è male. Ma è giusto questo? 

Sul suo viso c'erano l'incertezza e la ricerca di consiglio, di appoggio e di guida in una faccenda per lui incomprensibile. 

- No - lo interruppe la contessa Lidija Ivanovna. - C'è un limite a tutto. Io capisco l'immoralità - disse, non del tutto sinceramente, poiché non aveva mai potuto capire che cosa conducesse le donne all'immoralità; - ma non capisco la crudeltà, e con chi? con voi! Come soggiornare nella stessa città in cui siete voi? No, davvero, più si vive, più si impara. E io imparo a capire la vostra elevatezza e la sua meschinità. 

- E chi getterà la prima pietra? - disse Aleksej Aleksandrovic, evidentemente soddisfatto della propria parte. - Io ho perdonato tutto e perciò non posso privarla di quello che è un'esigenza d'amore per lei, d'amore per il figlio. 

- Ma è amore questo, amico mio? È sincero? Ammettiamolo, voi avete perdonato, voi perdonate\ldots{} ma abbiamo noi il diritto di agire sull'anima di quell'angelo? Egli prega per lei e chiede a Dio di perdonarle i peccati\ldots{} E così è meglio. Ma ora, cosa penserà? 

- Non avevo pensato a questo - disse Aleksej Aleksandrovic, evidentemente consentendo. 

La contessa Lidija Ivanovna si coprì il viso con le mani e tacque un po'. Pregava. 

- Se chiedete il mio consiglio - ella disse, dopo aver pregato e scoprendo il viso - io non vi consiglio di far questo. Non vedo forse come soffrite, come ciò ha riaperto tutte le vostre ferite? Ma del resto voi, come sempre, dimenticate voi stesso. Ma a che cosa mai può condurre ciò? A nuove pene da parte vostra, a tormenti per il bambino. Se in lei è rimasto qualcosa di umano, ella stessa non deve desiderare ciò. No, io non lo consiglio, non ho dubbi, e, se me ne autorizzate, le scriverò. 

Aleksej Aleksandrovic acconsentì, e la contessa Lidija Ivanovna scrisse la seguente lettera in francese. 

\begin{quote}
``Gentile Signora,

il ricordarvi a vostro figlio può produrre da parte sua domande alle quali non è possibile rispondere senza introdurre, nell'animo del bambino, lo spirito del giudizio su quello che deve essere sacro per lui, e perciò vi prego di comprendere il rifiuto di vostro marito nello spirito dell'amore cristiano. Chiedo per voi misericordia dall'Altissimo.

Contessa Lidija''.
\end{quote} 

Questa lettera raggiunse lo scopo segreto che la contessa Lidija Ivanovna nascondeva a se stessa. Offese Anna nel profondo dell'anima. 

Da parte sua Aleksej Aleksandrovic, tornando da Lidija Ivanovna a casa sua, non poté, quel giorno, darsi alle sue solite occupazioni, né trovare quella calma spirituale di uomo credente e salvato che aveva provato prima. 

Il ricordo della moglie, che era tanto colpevole dinanzi a lui e di fronte alla quale egli era così santo, come gli diceva giustamente la contessa Lidija Ivanovna, non avrebbe dovuto turbarlo, ma egli non era tranquillo: non riusciva a capire il libro che leggeva, non riusciva a scacciare i ricordi tormentosi dei suoi rapporti con lei, gli errori che gli sembrava aver commesso verso di lei. Il ricordo di come aveva accolto, ritornando dalle corse, la sua confessione di infedeltà (in particolare il fatto che aveva preteso da lei solo le convenzioni esteriori, senza sfidare lui a duello) lo tormentava come un rimorso. Lo tormentava pure il ricordo della lettera che aveva scritto a lei; in particolare il proprio perdono, non necessario a nessuno, e le sue preoccupazioni per una creatura non sua gli bruciavano il cuore di vergogna e di rimorso. 

E proprio un sentimento di vergogna e di rimorso provava adesso, esaminando tutto il suo passato con lei e ricordando le parole impacciate con le quali egli, dopo lunghe esitazioni, le aveva fatto la sua domanda di matrimonio. 

``Ma di che cosa sono colpevole io?'' diceva a se stesso. E questa domanda suscitava sempre in lui un'altra domanda: ``Sentivano forse diversamente, amavano forse diversamente, si sposavano forse diversamente quelle altre persone, quei Vronskij, quegli Oblonskij\ldots{} quei ciambellani dai polpacci grassi?''. E gli si presentava tutta una serie di persone dense di umori, forti, che non avevano dubbi, le quali involontariamente attiravano, sempre e dovunque, la sua attenzione curiosa. Scacciava da sé questi pensieri, cercava di convincersi che non viveva per la vita di questo mondo, ma per quella eterna; che nell'animo suo c'erano pace e amore. Ma l'aver commesso in questa vita terrena, per lui insignificante, degli errori insignificanti lo tormentava come se non ci fosse neppure mai stata quella salvezza eterna in cui credeva. Ma questa tentazione non durò a lungo, e ben presto nell'animo di Aleksej Aleksandrovic si ristabilirono la calma e la elevatezza, grazie alle quali egli poteva dimenticare quello che non voleva ricordare. 

\capitolo{XXVI}\label{xxvi-3} 

- E allora, Kapitonyc? - disse Serëza, rosso in viso e allegro, tornando dalla passeggiata, alla vigilia del suo compleanno e dando il cappotto a pieghe al vecchio portiere che sorrideva al piccolo uomo dall'alto della sua statura. - Be', è venuto oggi l'impiegato col braccio al collo? L'ha ricevuto papà? 

- L'ha ricevuto. Era appena uscito il capo gabinetto che io l'ho annunciato - disse il portiere, ammiccando. - Vi prego, lo levo io. 

- Serëza! - disse lo slavo istitutore fermandosi sulla porta che dava nelle stanze interne - levatelo da solo. 

Ma Serëza, pur avendo sentito la voce fiacca dell'istitutore, non vi fece attenzione. Stava fermo, tenendosi con la mano alla cintura del portiere, e lo guardava in viso. 

- Be', e papà ha fatto per lui quello che occorreva? 

Il portiere fece un cenno affermativo col capo. 

L'impiegato col braccio al collo, che era già andato sette volte a chiedere qualcosa ad Aleksej Aleksandrovic, interessava Serëza e il portiere. Serëza l'aveva trovato una volta nell'ingresso e aveva sentito come chiedeva pietosamente al portiere di annunciarlo, dicendo che gli toccava di morire insieme ai figliuoli. 

Da quel momento Serëza, dopo averlo incontrato ancora una volta nell'ingresso, se ne interessò. 

- Ebbene, era contento? - chiese. 

- E come non esser contento! È scappato via di qua, saltando quasi. 

- E hanno portato qualcosa? - chiese Serëza, dopo un attimo di silenzio. 

- Eh, signorino - disse in un bisbiglio il portiere, scotendo il capo - da parte della contessa. 

Serëza capì subito che quello di cui parlava il portiere, era il regalo della contessa Lidija per il suo compleanno. 

- Cosa dici? Dove? 

- Kornej l'ha portato dentro da papà. Deve essere una bella cosa! 

- Com'è grande? Così? 

- Più piccola, ma bella. 

- Un libro? 

- No, una cosa. Andate, andate, Vasilij Lukic chiama - disse il portiere, che aveva sentito i passi dell'istitutore che si avvicinavano e, raddrizzando accortamente la manina nel guanto tolto a mezzo che lo teneva per la cintura, accennò col capo verso Lukic. 

- Vasilij Lukic, un minutino! - disse Serëza con quel sorriso allegro e amorevole, che conquistava sempre l'imperioso Vasilij Lukic. 

Serëza era troppo allegro, era troppo felice perché potesse non far partecipe anche il suo amico portiere della gioia di famiglia appresa alla passeggiata al Giardino d'Estate dalla nipote della contessa Lidija Ivanovna. Questa gioia gli sembrava particolarmente importante perché coincideva con la gioia dell'impiegato e con la propria, per i giocattoli che gli avevano portato. A Serëza sembrava che quello fosse un giorno in cui tutti dovessero essere contenti e allegri. 

- Lo sai? Papà ha ricevuto l'Aleksandr Nevskij? 

- E come se non lo so! Sono già venuti a rallegrarsi. 

- Be', è contento? 

- E come non esser contento del favore dello zar? Vuol dire che l'ha meritato - disse, severo e compreso, il portiere. 

Serëza si fece pensieroso, guardando il viso del portiere, studiato fin nei minimi particolari, soprattutto il mento che pendeva fra le fedine canute, e che nessuno vedeva, tranne Serëza che non lo guardava altrimenti che di sotto in su. 

- Be', e tua figlia è un pezzo che è stata da te? 

La figlia del portiere era una ballerina del balletto. 

- E quando si può venire nei giorni di lavoro? Anche loro devono studiare. Anche per voi c'è lo studio; andate signorino. 

Giunto nella stanza, Serëza, invece di mettersi a sedere per fare le lezioni, confidò al maestro la sua supposizione che quello che avevano portato doveva essere una macchina. 

- Che ne pensate? - chiese. 

Ma Vasilij Lukic pensava solo che bisognava studiare la lezione di grammatica per il maestro che sarebbe venuto alle due. 

- No, ditemi solo, Vasilij Lukic - domandò a un tratto, già seduto dietro al tavolo di studio e trattenendo nelle mani il libro - che cosa c'è di più dell'Aleksandr Nevskij? Lo sapete? papà ha ricevuto l'Aleksandr Nevskij. 

Vasilij Lukic rispose che al di sopra dell'Aleksandr Nevskij c'era l'ordine di Vladimir. 

- E più su? 

- Più su di tutto l'Andrej Pervozvannyj. 

- E più su dell'Andrej? 

- Non lo so. 

- E come, neanche voi lo sapete? - Serëza, appoggiatosi sui gomiti, si sprofondò in meditazioni. 

Le sue meditazioni erano più complesse e varie. Egli si figurava come suo padre avrebbe ricevuto a un tratto l'ordine di Vladimir e quello di Andrej, e come, in seguito a ciò, quel giorno alla lezione sarebbe stato molto più indulgente, e come lui stesso, divenuto grande, avrebbe ricevuto tutti gli ordini, anche quello che avrebbero creato più su di quello di Andrej. Appena creato, egli lo avrebbe meritato. Ne avrebbero creato uno ancora più in alto e lui, subito, ecco, se lo merita. 

In queste meditazioni il tempo passò e quando venne il maestro, la lezione sui complementi di tempo, di luogo e di modo non era preparata, e il maestro non solo non fu contento, ma si dispiacque. Il dolore del maestro commosse Serëza. Si sentiva colpevole perché non aveva studiato la lezione; per quanto si sforzasse non ci riusciva in nessun modo. Finché il maestro spiegava, ci credeva e sembrava capire, ma non appena rimaneva solo, non poteva assolutamente capire e ricordarsi come un'espressione così corta e facile a intendersi quale ``a un tratto'' fosse un complemento di modo; gli dispiacque di avere addolorato il maestro, e voleva consolarlo. 

Scelse un momento in cui il maestro taceva, guardando il libro. 

- Michail Ivanyc, quando è il vostro onomastico? - chiese a un tratto. 

- Sarebbe meglio che pensaste al vostro lavoro, ché l'onomastico non ha nessuna importanza per un essere ragionevole. È un giorno come un altro, nel quale bisogna lavorare. 

Serëza guardò il maestro, la sua barbetta rada, gli occhiali che erano scesi più giù del taglio ch'era sul naso, e si fece così pensieroso che non sentì più nulla di quello che gli spiegava il maestro. Egli capiva che il maestro non pensava a quello che diceva; lo sentiva dal tono con cui le parole erano dette. ``Ma perché si sono messi tutti d'accordo nel dire queste cose sempre alla stessa maniera, tutte le più noiose e inutili cose? Perché mi allontana da sé, perché non mi vuol bene?'' si chiedeva con tristezza e non riusciva a immaginare una risposta. 

\capitolo{XXVII}\label{xxvii-3} 

Dopo il maestro, c'era la lezione del padre. Finché non giunse il padre, Serëza sedette al tavolo, gingillandosi con un coltellino, e cominciò a pensare. Nel numero delle occupazioni preferite da Serëza c'era la ricerca della madre durante la passeggiata. Egli non credeva alla morte in genere, e in particolare alla morte di lei, sebbene Lidija Ivanovna glielo avesse detto e il padre riconfermato, e perciò anche dopo che gli avevano detto che era morta, egli, durante la passeggiata, continuava a cercarla. Ogni donna bella e piacente, con i capelli scuri era sua madre. Quando vedeva una figura di donna simile, nell'animo suo si sollevava un tale sentimento di tenerezza, ch'egli si sentiva soffocare e gli venivano le lacrime agli occhi. E aspettava da un momento all'altro ch'ella si avvicinasse a lui e sollevasse il velo. Si sarebbe, allora, visto tutto il viso, ella avrebbe sorriso, l'avrebbe abbracciato, lui avrebbe sentito il profumo di lei, avrebbe sentito la tenerezza della sua mano, e si sarebbe messo a piangere felice come una volta, di sera, che le si era coricato ai piedi e lei gli aveva fatto il solletico, ed egli aveva riso e morso la mano bianca con gli anelli. Quando poi aveva saputo, per caso, dalla njanja che sua madre non era morta, e quando il padre e Lidija Ivanovna gli spiegarono che era morta per lui perché era cattiva (alla quale cosa non credeva in nessun modo perché l'amava), egli la cercò e l'aspettò sempre allo stesso modo. Quel giorno, al Giardino d'Estate, c'era una signora col velo lilla, ch'egli aveva seguito con un gran batticuore credendo che fosse lei, mentre si avvicinava a loro per un sentiero. Ma quella signora non era giunta fino a loro e si era nascosta chi sa dove. Quel giorno Serëza aveva sentito più forte che mai uno slancio di amore per lei e adesso, mentre aspettava il padre, aveva tagliuzzato l'orlo del tavolo con il temperino, guardando dinanzi a sé con gli occhi scintillanti, smarrito nel pensiero di lei. 

- Viene papà - lo richiamò Vasilij Lukic. 

Serëza saltò su, si accostò al padre e, baciatagli la mano, lo guardò attento, cercando dei segni di gioia per aver ricevuto l'Aleksandr Nevskij. 

- Hai passeggiato bene? - chiese Aleksej Aleksandrovic, sedendosi nella sua poltrona, attirando a sé il libro dell'Antico Testamento e aprendolo. Sebbene Aleksej Aleksandrovic avesse detto più di una volta a Serëza che ogni buon cristiano deve conoscere con sicurezza la storia sacra, egli stesso controllava spesso l'Antico Testamento, e Serëza l'aveva notato. 

- Sì, è stato molto divertente, papà - disse Serëza, sedendosi di sbieco sulla sedia e dondolandosi, il che era proibito. - Ho visto Naden'ka - Naden'ka era la nipote di Lidija Ivanovna che veniva educata presso di lei. - Mi ha detto che vi hanno dato una croce nuova. Siete contento, papà? 

- In primo luogo, non dondolarti, per piacere - disse Aleksej Aleksandrovic. - E in secondo luogo, non è cara la ricompensa, ma il lavoro. E vorrei che tu lo capissi. Ecco, se tu lavorerai, se studierai per avere una ricompensa, il lavoro ti sembrerà faticoso; ma se lavorerai - Aleksej Aleksandrovic parlava, ricordando come si era sostenuto con la coscienza del dovere durante il noioso lavoro della mattinata, consistente nella firma di centodiciotto carte - amando il lavoro, vi troverai una ricompensa per te. 

Gli occhi scintillanti di tenerezza e di felicità di Serëza si spensero e si abbassarono sotto lo sguardo del padre. Era quello stesso tono che da tempo il padre usava con lui e che da tempo Serëza aveva preso a secondare. Il padre gli parlava sempre come se si rivolgesse a un ragazzo immaginato da lui, uno di quelli che esistono nei libri, per nulla affatto simile a Serëza. E Serëza, col padre, si sforzava sempre di fingersi proprio questo tale ragazzo da libro. 

- Tu capisci questo, spero? - disse il padre. 

- Sì, papà - rispose Serëza, fingendosi il ragazzo immaginario. 

La lezione consisteva nell'imparare a memoria alcuni versetti del Vangelo e nel ripetere il principio dell'Antico Testamento. I versetti del Vangelo Serëza li sapeva discretamente, ma nel momento in cui prese a recitarli si mise a fissare l'osso della fronte del padre che si piegava così rapidamente verso la tempia, si confuse, e traspose la fine di un versetto con una parola eguale al principio di un altro. Per Aleksej Aleksandrovic era evidente ch'egli non capiva quello che diceva, e si irritò. 

Aggrottò le sopracciglia e cominciò a spiegare quello che Serëza aveva sentito già molte volte e che non ricordava mai perché lo capiva troppo chiaramente, così come, press'a poco, il fatto che ``a un tratto'' era complemento di modo. Serëza guardava il padre con uno sguardo spaventato e pensava solo a un'unica cosa: l'avrebbe costretto o no a ripetere quello che aveva detto, come a volte accadeva? E questo pensiero spaventò tanto Serëza che non capì più nulla. Ma il padre non lo costrinse a ripetere e passò alla lezione sull'Antico Testamento. Serëza raccontò bene gli avvenimenti per se stessi, ma quando bisognò rispondere alle domande su quello che significavano allegoricamente alcuni di essi, non seppe niente, pur essendo stato punito per questa lezione. Il passo, poi, dove non sapeva dire proprio nulla ed esitava e tagliuzzava il tavolo e si dondolava sulla seggiola era quello in cui bisognava parlare dei patriarchi antidiluviani. Fra questi non ne conosceva nessuno all'infuori di Enoch, assunto vivo in cielo. Prima ricordava dei nomi, ma adesso li aveva dimenticati completamente, proprio perché Enoch era il suo personaggio preferito di tutto l'Antico Testamento, e all'assunzione al cielo di Enoch vivo si collegava tutto un lungo ragionamento, a cui egli, adesso, si abbandonò guardando con gli occhi fissi la catena dell'orologio del padre e un bottone del panciotto abbottonato a metà. 

Alla morte, di cui così spesso gli parlavano, Serëza non credeva per nulla. Non credeva che le persone da lui amate potessero morire, e in particolare ch'egli sarebbe morto. Questo era per lui assolutamente impossibile e incomprensibile. Ma gli dicevano che tutti sarebbero morti; egli interrogava perfino persone alle quali credeva e quelle lo confermavano: anche la njanja lo diceva, suo malgrado. Ma Enoch non era morto, perciò non tutti morivano. ``E perché mai, non può ognuno essere così degno dinanzi a Dio da essere assunto vivo in cielo?'' pensava Serëza. I cattivi, cioè quelli a cui Serëza non voleva bene, quelli potevano morire, ma i buoni potevano essere tutti come Enoch. 

- Su, allora, quali sono i patriarchi? 

- Enoch, Enos. 

- Ma questi li hai già detti. Male, Serëza, molto male. Se non cerchi di conoscere quello che è più necessario di tutto per un cristiano - disse il padre, alzandosi - allora cosa mai può interessarti? Io sono scontento di te, anche Pëtr Ignat'ic - era costui l'istitutore più importante - è scontento di te\ldots{} Io devo punirti. 

Il padre e l'istitutore erano scontenti di Serëza, e realmente egli studiava molto male. Ma non si poteva dire in nessun modo che fosse un ragazzo senza capacità. Al contrario era molto più versato di quei ragazzi che l'istitutore portava come esempio a Serëza. Secondo il padre, egli non voleva apprendere quello che gli insegnavano. In realtà invece non poteva studiarlo. Non poteva perché nell'animo suo c'erano esigenze più imperiose di quelle che presentavano il padre e l'istitutore. Queste esigenze erano in contrasto, ed egli lottava proprio con i suoi educatori. 

Aveva nove anni, era un bambino; ma l'anima sua la conosceva, gli era cara, la proteggeva come la palpebra protegge l'occhio, e senza la chiave dell'amore non lasciava entrare nessuno nella sua anima. I suoi educatori si lamentavano che non voleva studiare, e la sua anima era piena di sete di sapere. E apprendeva da Kapitonyc, dalla njanja, da Naden'ka, da Vasilij Lukic e non dai maestri. Quell'acqua che i padre e l'istitutore attendevano per le loro ruote, correva già da tempo e lavorava in altro luogo. 

Il padre punì Serëza col proibirgli di andare da Naden'ka, la nipote di Lidija Ivanovna; ma questa punizione si dimostrò favorevole per Serëza. Vasilij Lukic era di buon umore e gli mostrò come si fanno i mulini a vento. Passò tutta la sera lavorando e vagheggiando un mulino a vento fatto in modo che si potesse girarci sopra; afferrarsi con le mani alle ali o legarsi e girare. Alla madre Serëza non pensò per tutta la sera; ma, messosi a letto, se ne ricordò, a un tratto, e pregò con parole sue perché la mamma l'indomani, per il suo compleanno, smettesse di nascondersi e venisse da lui. 

- Vasilij Lukic, sapete per che cosa ho pregato in più, fuori del conto? 

- Per andare meglio nello studio? 

- No. 

- Per i giocattoli? 

- No. Non indovinerete. Per una cosa bellissima, ma è un segreto! Quando accadrà ve lo dirò. Non l'avete indovinato? 

- No, non l'indovinerò. Ditemelo voi - disse Vasilij Lukic sorridendo, il che accadeva di rado. - Via, coricatevi, spengo la candela. 

- Ma senza la candela per me è più chiaro quello che vedo e quello per cui ho pregato. Ecco che stavo quasi per dirvi il segreto - disse Serëza, ridendo allegramente. 

E come ebbero portato via la candela, Serëza udì e sentì sua madre. Era china su di lui e lo carezzava con uno sguardo pieno d'amore. Ma poi apparvero i mulini a vento, il coltellino, tutto si confuse ed egli si addormentò. 

\capitolo{XXVIII}\label{xxviii-3} 

Arrivati a Pietroburgo, Vronskij e Anna si erano fermati in uno dei migliori alberghi: Vronskij, a parte, nel piano di sotto, Anna sopra con la bambina, la balia e la cameriera, in un grande appartamento composto di quattro stanze. 

Fin dal primo giorno dell'arrivo Vronskij andò dal fratello. Là trovò sua madre che era venuta da Mosca per affari. La madre e la cognata lo accolsero come se nulla fosse; lo interrogarono sul suo viaggio all'estero, parlarono dei conoscenti comuni, ma neppure con una parola accennarono alla sua relazione con Anna. Il fratello poi, venuto il giorno dopo da Vronskij, di mattina, domandò di lei, e Aleksej Vronskij gli disse francamente che considerava la propria relazione con la Karenina come un matrimonio, che sperava di render possibile il divorzio e di sposarla, e che fino a quel momento la considerava come sua moglie, e lo pregava di riferire ciò alla madre e a sua moglie. 

- Se il mondo non approva ciò, a me è indifferente - disse Vronskij - ma se i miei parenti vogliono rimanere in rapporti di parentela con me, devono essere in questi stessi rapporti con mia moglie! 

Il fratello maggiore, che rispettava sempre le idee del minore, non sapeva bene se egli avesse ragione o no, finché il mondo non avesse risolto tale questione; lui stesso poi, da parte sua, non aveva nulla in contrario, e insieme con Aleksej andò da Anna. 

Vronskij, in presenza del fratello, come del resto in presenza di tutti, dava del voi ad Anna e la trattava come una conoscente intima, ma era sottinteso che il fratello conoscesse i loro rapporti, e si parlava del fatto che Anna andava nella proprietà di Vronskij. 

Malgrado tutta la sua esperienza mondana, Vronskij, in seguito alla nuova situazione in cui si trovava, era in uno strano errore. Sembrava che egli avrebbe dovuto capire come la società fosse preclusa a lui e ad Anna; adesso invece nella sua testa erano sorte certe nuove confuse considerazioni, che, cioè, così era soltanto nei tempi passati, ma che ora, col rapido progresso (senza accorgersene era divenuto seguace d'ogni progresso), il giudizio della società era mutato e che la questione, se essi sarebbero stati ricevuti in società, non era ancora risolta. ``Si intende - pensava - il mondo della corte non la riceverà, ma le persone intime possono e devono intendere ciò così come va inteso''. 

Si può restar seduti parecchie ore, incrociando le gambe, nella stessa posizione, quando si sa che nulla impedisce di cambiar posizione; ma se una persona sa che deve rimaner seduto così, con le gambe incrociate, gli verranno i crampi, le gambe si stireranno e si stringeranno in quel punto dove egli vorrebbe allungarle. Vronskij sperimentava questa stessa cosa rispetto alla società. Pur sapendo, in fondo all'animo, che la società era preclusa, voleva provare se questa si fosse mutata oppure no e se li avrebbe ricevuti. Si accorse, invece, subito, che la società era aperta a lui personalmente ed era preclusa ad Anna. Come nel giuoco del gatto e del topo, le braccia sollevate per lui, si abbassavano immediatamente dinanzi ad Anna. 

Una delle prime signore del gran mondo di Pietroburgo che Vronskij vide fu sua cugina Betsy. 

- Finalmente! - ella lo accolse con gioia. - E Anna? Come sono contenta! Dove siete? Immagino come vi sembri orribile la nostra Pietroburgo dopo il vostro delizioso viaggio; immagino la vostra luna di miele a Roma. E il divorzio? È stato fatto tutto? 

Vronskij notò che l'entusiasmo di Betsy diminuì quando seppe che il divorzio non c'era ancora. 

- Mi scaglieranno la pietra addosso, lo so - ella disse - ma verrò da Anna; sì, verrò assolutamente. Non vi fermerete molto qui? 

E realmente quello stesso giorno ella andò da Anna; ma il suo tono era ormai ben diverso da quello di prima. Era palesemente orgogliosa del proprio coraggio, e desiderava che Anna apprezzasse la fedeltà della sua amicizia. Rimase non più di dieci minuti parlando delle novità mondane, e nell'andarsene disse: 

- Non m'avete detto quand'è il divorzio. Del resto, io non mi faccio scrupoli, ma gli altri, quelli che alzano il bavero, quelli vi colpiranno con il loro gelo, finché non sarete sposati. E questo è così semplice adesso. Ça se fait. Così, allora, partite venerdì? Peccato che non ci vedremo più. 

Dal tono di Betsy, Vronskij avrebbe dovuto capire quello che doveva aspettarsi dalla società, ma fece ancora un tentativo nella propria famiglia. In sua madre non sperava. Sapeva che sua madre, entusiasta di Anna quando l'aveva conosciuta la prima volta, adesso era implacabile contro di lei perché era stata causa del dissesto nella carriera del figlio. Ma egli riponeva grandi speranza in Varja, la moglie del fratello. Gli sembrava che non avrebbe scagliato lei la prima pietra, e con semplicità e franchezza sarebbe andata da Anna e l'avrebbe ricevuta. 

Fin dal giorno dopo il suo arrivo, Vronskij andò da lei e, trovatala sola, espresse francamente il suo desiderio. 

- Tu sai, Aleksej - ella disse, dopo averlo ascoltato - come io ti voglia bene e come sia pronta a fare tutto per te; ma ho taciuto, perché sapevo che non posso essere utile a te e ad Anna Arkad'evna - ella disse, pronunciando con particolare sforzo ``Anna Arkad'evna''. - Ti prego di non credere che io voglia biasimare. Mai; forse io, al posto suo, avrei fatto lo stesso. Non entro e non voglio entrare in particolari - ella diceva, guardando timidamente il viso di lui. - Ma bisogna chiamar le cose col loro nome. Tu vuoi che io vada da lei, che la riceva e, con questo, la riabiliti in società; ma tu devi capire che io non posso far questo. Le figlie mi crescono, io devo vivere nella società per mio marito. Ammettiamo che io venga da Anna Arkad'evna; lei capirà che non posso invitarla a casa mia o che devo farlo in modo ch'ella non incontri le persone che pensano diversamente; questo offenderà lei stessa. Io non posso risollevarla\ldots{} 

- Ma io non considero ch'ella sia caduta più in basso di quanto non lo siano centinaia di donne che voi ricevete! - la interruppe ancora più cupo Vronskij, e si alzò in silenzio, avendo capito che la decisione della cognata era irremovibile. 

- Aleksej! non arrabbiarti con me. Ti prego, comprendimi, io non ne ho colpa - prese a dire Varja, guardandolo con un sorriso timido. 

- Io non sono arrabbiato con te - egli disse sempre cupo - ma ciò mi fa doppiamente male. Mi fa male anche perché rompe la nostra amicizia. Ammettiamo che non la rompa, ma la indebolisce. Tu pure comprenderai come, per me, questo non possa essere diversamente. - E con questo se ne andò via. 

Vronskij capì che ulteriori tentativi erano inutili e che bisognava passare quei pochi giorni a Pietroburgo come in un paese straniero, sfuggendo ogni rapporto col mondo di prima per non sottomettersi a dispiaceri e affronti, così tormentosi per lui. Una delle cose più spiacevoli della situazione di Pietroburgo era che Aleksej Aleksandrovic e il suo nome sembrava fossero dappertutto. Non si poteva cominciare a parlare di nulla, senza che il discorso non si aggirasse intorno ad Aleksej Aleksandrovic; non si poteva andare in nessun posto senza incontrar lui. Così almeno pareva a Vronskij, così come a un uomo che abbia un dito malato pare di urtar sempre, contro tutto, proprio con quel dito. 

La permanenza a Pietroburgo sembrava ancor più incresciosa a Vronskij perché in tutto quel tempo egli vedeva in Anna un umore nuovo, per lui incomprensibile. Ella, un momento, era innamorata di lui, un momento, diventava fredda, irritabile e impenetrabile. Qualcosa la tormentava, e qualcosa ella gli nascondeva e sembrava non notare neppure quelle offese che a lui avvelenavano l'esistenza e che per lei, per la sua finezza d'intuito, dovevano essere più tormentose ancora. 

\capitolo{XXIX}\label{xxix-3} 

Uno degli scopi del viaggio in Russia era per Anna l'incontro con il figlio. Dal giorno in cui era partita dall'Italia, questo pensiero dell'incontro non aveva cessato d'agitarla. E quanto più si avvicinava a Pietroburgo, tanto più grande le appariva la gioia e l'importanza di quest'incontro. Non si domandava neppure come sarebbe avvenuto l'incontro. Le sembrava naturale e semplice vedere il figlio, quando sarebbe stata nella stessa città dove era lui; ma giunta a Pietroburgo, a un tratto le si presentò chiara la sua posizione nella società, e capì che combinare l'incontro era difficile. 

Era a Pietroburgo già da due giorni. Il pensiero del figlio non l'aveva abbandonata neanche un attimo, eppure ancora non l'aveva visto. Andare direttamente in casa, dove poteva incontrarsi con Aleksej Aleksandrovic, sentiva di non averne il diritto. Avrebbero potuto impedirle di entrare e offenderla. Scrivere ed entrare in rapporti col marito, le era tormentoso anche il solo pensarlo: le riusciva d'essere calma solo quando non pensava al marito. Vedere il figlio alla passeggiata, dopo aver saputo dove e quando andava, le sembrava ben poca cosa: si era tanto preparata a quest'incontro, doveva dirgli tante cose, voleva tanto abbracciarlo e baciarlo. La vecchia njanja di Serëza poteva aiutarla e istruirla, ma non era più in casa di Aleksej Aleksandrovic. In queste esitazioni e nella ricerca della njanja erano passati questi due giorni. 

Avendo saputo degli intimi rapporti di Aleksej Aleksandrovic con la contessa Lidija Ivanovna, Anna, il terzo giorno, si decise a scriverle la lettera che le era costata tanta fatica, nella quale diceva, intenzionalmente, che il permesso di vedere il figlio doveva dipendere dalla generosità del marito. Ella sapeva che, se avessero mostrato la lettera al marito, egli, per sostenere la parte di uomo generoso, non glielo avrebbe rifiutato. 

L'inserviente che aveva portato la lettera, consegnò la risposta più crudele e inaspettata per lei, che, cioè, non ci sarebbe stata risposta. Non s'era mai sentita così umiliata come nel momento in cui, chiamato il fattorino, aveva sentito da lui il racconto particolareggiato di come egli avesse aspettato a lungo e di come poi gli avessero detto: ``Non c'è risposta''. Anna si sentiva umiliata, offesa, ma vedeva che, dal suo punto di vista, la contessa Lidija Ivanovna aveva ragione. Il suo dolore era tanto più forte in quanto era solo. Non poteva e non voleva farne partecipe Vronskij. Sapeva che per lui, anche se questa era la causa principale dell'infelicità di lei, la questione dell'incontro col figlio sarebbe apparsa come la cosa più insignificante. Sapeva che mai egli avrebbe avuto la forza di capire tutta la profondità della sua pena; sapeva che per quel tono freddo, col quale egli accennava alla cosa, avrebbe preso a odiarlo. E aveva paura di questo, più di qualunque altra cosa al mondo, e perciò gli nascondeva tutto quello che riguardava il figlio. 

Rimasta in casa tutto il giorno, aveva escogitato vari mezzi per vedersi col figlio e si era soffermata sulla decisione di rivolgersi al marito. Stava già scrivendo, quando le portarono la lettera di Lidija Ivanovna. Il silenzio della contessa l'aveva domata e sottomessa, ma la lettera, tutto quello ch'ella scorse fra le righe, la irritò a tal punto e così disgustoso le parve quel rancore di fronte alla sua appassionata tenerezza verso il figlio, che si rivoltò contro gli altri e cessò d'incolpare se stessa. 

``Questa freddezza è la finzione di un sentimento! - diceva a se stessa. - Hanno solo bisogno di offendere me e di tormentare il bambino, e io dovrei sottostare a loro! Per nulla al mondo! Lei è peggiore di me. Io almeno non mento''. E subito decise che l'indomani, giorno del compleanno di Serëza, sarebbe andata direttamente in casa del marito, avrebbe corrotto i servitori, avrebbe ingannato, ma a qualunque costo avrebbe visto il figlio, avrebbe spezzato l'inganno crudele nel quale avevano ravvolto il disgraziato fanciullo. 

Andò in un negozio di giocattoli, comprò tanti giocattoli e preparò il suo piano d'azione. Sarebbe andata la mattina presto, alle otto, quando Aleksej Aleksandrovic probabilmente ancora non s'era alzato. Avrebbe avuto in mano del denaro per il portiere e per il servo che l'avrebbero lasciata passare, e, senza sollevare il velo, avrebbe detto che veniva da parte del padrino di Serëza e che aveva l'incarico di lasciare i giocattoli accanto al letto del bambino. Solo le parole da dire al figlio non aveva preparato. Per quanto ci pensasse non riusciva a immaginare nulla. 

Il giorno dopo, alle otto del mattino, Anna discendeva sola da una carrozza da nolo e sonava all'ingresso principale della sua casa di un tempo. 

- Va' a vedere cosa vogliono. C'è una signora - disse Kapitonyc, non ancora vestito, in cappotto e soprascarpe, dopo aver guardato dalla finestra la signora, avvolta in un velo, che stava dritta proprio accanto alla porta. 

L'aiutante del portiere, un ragazzo che Anna non conosceva, le aveva appena aperto la porta che lei vi si era già infilata e, cacciato fuori dal manicotto un biglietto da tre rubli, glielo ficcò frettolosamente in mano. 

- Serëza\ldots{} Sergej Alekseic - ella pronunciò e voleva andare avanti. Dopo aver guardato il biglietto, l'aiutante del portiere la fermò all'altra porta di vetro. 

- Ma chi volete? - chiese. 

Ella non sentiva le sue parole e non rispondeva nulla. 

Avendo notato la confusione della sconosciuta, lo stesso Kapitonyc le venne incontro, la lasciò passare dalla porta e domandò cosa desiderasse. 

- Da parte del principe Skorodumov a Sergej Alekseic - ella disse. 

- Non è ancora alzato - disse il portiere, osservandola attentamente. 

Anna non si aspettava che l'atmosfera immutata dell'anticamera di quella casa, dove aveva vissuto nove anni, l'avrebbe impressionata con tanta violenza. Uno dietro l'altro i ricordi, felici e tormentosi, si sollevarono nell'animo suo, e per un attimo ella dimenticò perché si trovava là. 

- Volete aspettare? - disse Kapitonyc, togliendole la pelliccia. 

Tolta la pelliccia, Kapitonyc la guardò in viso, la riconobbe e in silenzio si inchinò profondamente. 

- Prego, eccellenza - disse. 

Ella voleva dire qualcosa, ma la sua voce si rifiutò di emettere un suono qualsiasi; dopo aver guardato il vecchio con l'implorazione di chi è colpevole, si avviò su per la scala a passi leggeri e svelti. 

Kapitonyc, tutto piegato in avanti e inciampando con le soprascarpe negli scalini, le correva dietro, cercando di oltrepassarla. 

- C'è il maestro là, forse non è vestito. Vado ad annunciare. 

Anna continuava ad andare per la scala nota, senza capire quello che il vecchio diceva. 

- Di qua, favorite a sinistra. Perdonate se non è pulito. Adesso è nella stanza dei divani di prima - diceva il portiere, riprendendo fiato. - Permettete, eccellenza, abbiate pazienza, do un'occhiata - disse e, sorpassatala, socchiuse una porta grande e scomparve dietro di essa. Anna rimase in attesa. - S'è appena svegliato - disse il portiere, uscendo di nuovo dalla porta. 

E in quel momento, mentre il portiere diceva questo, Anna sentì il suono di uno sbadiglio infantile. Dal solo suono di questo sbadiglio, Anna riconobbe Serëza e lo vide vivo dinanzi a sé. 

- Lascia, lascia, va' - cominciò a dire, e varcò la grande porta. A destra della porta c'era un letto e sul letto era seduto, sollevato, il fanciullo con la sola camicina sbottonata, e, col corpicino piegato, si stiracchiava, mentre finiva uno sbadiglio. Nel momento in cui le labbra si unirono, si disposero a un sorriso beato e sonnolento, e con questo sorriso egli di nuovo ricadde all'indietro, lentamente e con dolcezza. 

- Serëza! - ella mormorò, avvicinandosi a lui senza farsi sentire. 

Durante il distacco da lui e in quell'ondata d'amore che aveva provato in tutto quell'ultimo tempo, ella se lo era sempre figurato bambino, di quattro anni, così come più di tutto l'aveva amato. Ora egli non era neppure più come lo aveva lasciato; era ancor più lontano da quello di quattro anni, era cresciuto ancora e smagrito. Il viso era magro, i capelli erano corti. E come erano lunghe le braccia! Come era cambiato da quando l'aveva lasciato! Ma era lui, con la forma della testa, con le sue labbra, con il piccolo collo morbido e le spallucce larghe. 

- Serëza!- ella ripeté proprio sull'orecchio del bambino. 

Egli si sollevò sul gomito, girò la testa arruffata da tutte e due le parti, come cercando qualcosa e aprì gli occhi. Per alcuni secondi guardò in silenzio e interrogativamente la madre immobile su di lui; poi, d'un tratto, sorrise beato e, chiusi di nuovo gli occhi che non volevano stare aperti, si gettò giù, riverso, e non all'indietro, ma verso di lei, verso le sue braccia. 

- Serëza! bambino caro! - ella esclamò, soffocando e abbracciandogli il corpo morbido. 

- Mamma! - egli disse, movendosi sotto le sue braccia per sentirne il contatto con le varie parti del corpo. 

Sorridendo assonnato, sempre con gli occhi chiusi, l'afferrò, attraverso la spalliera del letto, per le spalle, con le manine paffute, si appoggiò a lei, inondandola di quel caro odore di sonno e tepore che hanno solo i bambini, e cominciò a fregarsi col viso al collo e alle spalle di lei. 

- Lo sapevo - disse, aprendo gli occhi. - Oggi è il mio compleanno. Lo sapevo che saresti venuta. Mi alzo subito. 

E dicendo questo, riprendeva sonno. 

Anna lo guardava avidamente, vedeva come era cresciuto e come si era cambiato nella sua assenza. 

Riconosceva e non riconosceva le sue gambe nude, così grandi ora che si erano liberate della coperta, riconosceva quelle guance smagrite, quei ricci tagliati, corti sulla nuca, dove ella lo baciava tanto spesso. Lo palpava tutto e non poteva dir nulla: le lacrime la soffocavano. 

- E perché piangi, mamma? - disse egli, svegliandosi completamente. - Mamma, perché piangi? - gridò con voce piagnucolosa. 

- Io? non piangerò più\ldots{} Piango di gioia. Non ti vedevo da tanto. Non lo farò più, non lo farò più - disse lei, inghiottendo le lacrime e voltandosi dall'altra parte. - Su via, è ora di vestirsi - aggiunse, rimessasi, e dopo un po' di silenzio e senza lasciare la mano di lui, sedette accanto al letto sulla sedia dove erano disposti i vestiti. 

- Come ti vesti senza di me? come\ldots{} - voleva cominciare a parlare semplicemente e allegramente, ma non poté, e di nuovo si voltò dall'altra parte. 

- Non mi lavo più con l'acqua fredda: papà non ha voluto. E Vasilij Lukic l'hai visto? verrà. Ma tu ti siedi sui miei vestiti! 

E Serëza scoppiò a ridere. Ella lo guardò e sorrise. 

- Mamma, mammina mia bella! - egli gridò, gettandosi di nuovo verso di lei e abbracciandola. Pareva che soltanto ora, dopo aver visto il sorriso di lei, avesse capito chiaramente quello che era successo. - Questo non ci vuole - disse, togliendole il cappello. E come se l'avesse scoperta di nuovo, senza cappello, si slanciò di nuovo ad abbracciarla. 

- Ma cosa pensavi mai di me? Pensavi che ero morta? 

- Non ci ho mai creduto. 

- Non ci credevi, piccolo mio? 

- Lo sapevo, lo sapevo! - ripeteva lui la sua frase preferita e, presale la mano che gli carezzava i capelli, cominciò a premerne la palma sulla bocca e a baciarla. 

\capitolo{XXX}\label{xxx-3} 

Vasilij Lukic, intanto, senza capire in principio chi fosse quella signora e capito poi, dai discorsi, che era la madre che aveva abbandonato il marito e che lui non conosceva perché assunto quando lei era andata via, era nel dubbio se entrare o no, o se comunicare la cosa a Aleksej Aleksandrovic. Considerato infine che il suo dovere consisteva nel fare alzare a una determinata ora Serëza, e che perciò non aveva ragione di distinguere chi fosse seduto là, la madre o altri, ma che pur bisognava compiere il proprio dovere, si vestì, si accostò alla porta e l'aprì. 

Ma le carezze della madre e del figlio, il suono delle loro voci e quello che dicevano, tutto questo gli fece cambiar idea. Scosse il capo e, dopo aver sospirato, chiuse la porta. ``Aspetterò ancora dieci minuti'' si disse, tossendo e asciugandosi le lacrime. 

Nello stesso tempo, fra la servitù di casa, era sorta grande agitazione. Tutti avevano saputo che era venuta la signora e che Kapitonyc l'aveva lasciata entrare, e che adesso era nella camera del bambino, e intanto, dopo le otto, il signore entrava lui nella camera del bambino, e tutti capivano che l'incontro dei coniugi era impossibile e che bisognava impedirlo. Kornej, il maggiordomo, sceso in portineria, aveva domandato chi l'avesse lasciata passare e come, e saputo che Kapitonyc l'aveva ricevuta e accompagnata, rimproverava il vecchio. Il portiere taceva ostinato, ma quando Kornej gli disse che per questo sarebbe stato scacciato di casa, Kapitonyc gli balzò addosso e, agitando le mani sotto al viso di Kornej, cominciò a dire: 

- E già, ecco, tu non l'avresti lasciata entrare! In dieci anni di servizio non hai visto altro che bontà e ora andresti a dirle: ``Prego, vada fuori''. Tu, solo la politica conosci bene! Proprio così! Pensa bene a te, che rubi al padrone e porti via le pellicce di vaio. 

- Soldataccio! - disse con sprezzo Kornej e si voltò verso la njanja che entrava. - Ecco, dite voi, Mar'ja Efimovna: l'ha lasciata entrare, non l'ha detto a nessuno - disse Kornej rivolto a lei. - Aleksej Aleksandrovic verrà fuori subito, andrà nella camera del bambino. 

- Dio mio, Dio mio! - diceva la njanja. - Voi, Kornej Vasil'evic, dovreste trattenerlo, il padrone, e io corro su; in qualche modo la porterò via. Dio mio, Dio mio! 

Quando la njanja entrò nella camera del bambino, Serëza stava raccontando alla madre come fosse caduto insieme a Naden'ka, rotolando dall'alto, e come avesse fatto tre capriole. Ella ascoltava il suono della sua voce, vedeva il viso e il giuoco d'espressione, sentiva la sua mano, ma non capiva quello che egli diceva. Doveva andar via, doveva lasciarlo, pensava e sentiva solo questo. Aveva sentito i passi di Vasilij Lukic che si era avvicinato alla porta e aveva tossito; aveva sentito i passi della njanja che s'era accostata, ma lei era rimasta a sedere, come impietrita, senza aver la forza di parlare, né di alzarsi. 

- Signora cara! - cominciò a dire la njanja, avvicinandosi ad Anna e baciandole le mani e le spalle. - Ecco che Dio ha portato una gioia al nostro festeggiato. Non siete cambiata per nulla. 

- Ah, njanja mia cara, non sapevo che foste in casa - disse Anna, riavutasi per un momento. 

- Non ci sto io. Sto con una mia figliuola, ma sono venuta a fare gli auguri. Anna Arkad'evna, amore mio! 

La njanja a un tratto si mise a piangere e cominciò di nuovo a baciarle la mano. 

Serëza, splendente negli occhi e nel sorriso e aggrappandosi con una mano alla madre e con l'altra alla njanja, pestava il tappeto con le gambette tornite. La tenerezza della njanja, ch'egli amava, verso la madre, lo mandava in visibilio. 

- Mamma, lei viene spesso da me, e quando verrà\ldots{} - cominciò a dire, ma si fermò avendo notato che la njanja aveva sussurrato qualcosa alla madre e che sul viso della madre s'erano espressi lo spavento e qualcosa di simile alla vergogna che le stava così male. 

Ella gli si avvicinò. 

- Mio caro! - disse. 

Non poteva dire ``addio'', ma l'espressione del suo viso lo disse ed egli capì. 

- Caro, caro Kutik - ella disse il nomignolo con cui lo chiamava da piccolo - non ti scorderai di me? Tu\ldots{} - non poté dire altro. 

Quante parole trovò poi, che avrebbe potuto dirgli! E adesso non sapeva e non poteva dire nulla. Egli capì ch'ella era infelice e che lo amava. Capì perfino quello che diceva la njanja sottovoce. Sentì le parole: ``dopo le otto'', e capì che questo era detto del padre, e che la madre non poteva incontrarsi con lui. Questo lo capiva, ma una cosa non poteva capire: perché sul suo viso apparivano lo spavento e la vergogna?\ldots{} Lei non era colpevole, ma aveva paura di lui e si vergognava di qualcosa. Egli voleva fare una domanda, che gli avrebbe chiarito questo dubbio, ma non osava farla: vedeva ch'ella soffriva e aveva pena di lei. Le si strinse in silenzio e disse sottovoce: 

- Non te ne andare ancora. Non verrà presto. 

La madre lo allontanò da sé, per capire s'egli pensava quello che diceva, e nell'espressione spaventata del viso di lui, lesse che non solo parlava del padre, ma le chiedeva cosa avrebbe dovuto pensare di lui. 

- Serëza, piccolo mio - ella disse - amalo, è più buono di me, e io sono colpevole davanti a lui. Quando sarai grande, giudicherai. 

- Più buono di te non c'è nessuno!\ldots{} - egli gridò disperato attraverso le lacrime e, presala per le spalle, cominciò a stringerla a sé con tutta la forza, tremando nelle braccia per lo sforzo. 

- Amore mio, piccolo mio! - sussurrò Anna e si mise a piangere anche lei, debolmente, infantilmente, come piangeva lui. 

Intanto una porta si aprì, entrò Vasilij Lukic. All'altra porta si sentirono dei passi e la njanja, con un sussulto spaventato, disse:``Viene'' e porse il cappello ad Anna. 

Serëza si lasciò andare giù sul letto e si mise a singhiozzare, coprendosi il viso con le mani. Anna tolse via quelle mani, baciò ancora una volta il viso bagnato e a passi rapidi si avviò verso la porta. Aleksej Aleksandrovic le veniva incontro. Vistala, egli si fermò e abbassò il capo. 

Malgrado ella avesse allora allora detto che era migliore di lei, nel rapido sguardo che gli lanciò, cogliendone la figura in tutti i particolari, un senso di repulsione, di rancore verso di lui e di invidia per il figlio l'afferrò. Con un movimento rapido ella abbassò il velo e, affrettato il passo, uscì quasi correndo dalla stanza. 

Non aveva fatto neanche in tempo a tirar fuori i giocattoli, e li riportò a casa così com'erano, quei giocattoli che il giorno prima aveva scelto nel negozio con tanto amore e tanta tristezza. 

\capitolo{XXXI}\label{xxxi-3} 

Per quanto Anna avesse fortemente desiderato l'incontro col figlio, per quanto da tempo ci avesse pensato e ci si fosse preparata non si aspettava in nessun modo che quest'incontro potesse sconvolgerla tanto. Rientrata nel suo solitario appartamento d'albergo, a lungo non riuscì a capire perché si trovasse là. ``Tutto è finito, sono di nuovo sola'' si disse e, senza togliersi il cappello, sedette in una poltrona accanto al camino. Posando gli occhi immobili sull'orologio di bronzo che stava sulla tavola in mezzo alle finestre, cominciò a pensare. 

La cameriera francese, assunta all'estero, entrò per proporle di vestirsi. Ella la guardò con sorpresa e disse: 

- Dopo. 

Un cameriere le offrì il caffè. 

- Dopo - ella disse. 

La nutrice italiana, vestita la bambina, entrò con lei e la porse ad Anna. La bambina paffuta, ben nutrita, come sempre, vista la madre, girò le manine nude protese, piccole, con le palme all'ingiù e, sorridendo con la piccola bocca sdentata, cominciò ad annaspare con le manine, come un pesce con le pinne, frusciando sulle pieghe inamidate della sottanina ricamata. Non si poteva non sorridere, non baciare la piccola; non si poteva non tenderle un dito al quale ella si aggrappò stringendo e sussultando in tutto il corpo; non si poteva non tenderle il labbro ch'ella afferrò nella piccola bocca a mo' di bacio. E tutto questo Anna lo fece: e la prese in braccio e la fece saltare e le baciò la gota fresca e i piccoli gomiti scoperti; ma, nel vedere quella bambina, le si chiariva ancor più che il sentimento che provava per lei non era neppure amore in confronto di quello che provava per Serëza. Tutto in quella bambina era grazioso, eppure, chi sa perché, non le prendeva il cuore. Nel primo bambino, sia pure avuto da un uomo non amato, erano state riposte tutte le forze dell'amore insoddisfatto; la bambina era stata messa al mondo nelle condizioni più difficili, e in lei non era stata riposta neppure la centesima parte delle cure che si erano prodigate al primo. Inoltre, nella bambina tutto era ancora attesa, mentre Serëza era già una personcina, una persona amata; in lui lottavano già il pensiero, il sentimento, egli capiva, amava, la giudicava - ella pensava - ricordando le parole e lo sguardo di lui. Ed era divisa per sempre non solo fisicamente, ma spiritualmente da lui, e a questo non si poteva porre rimedio. 

Ella diede la bambina alla nutrice, la congedò e aprì un medaglione in cui era il ritratto di Serëza quando aveva la stessa età della bambina. Si alzò e, toltasi il cappello, prese da sopra un tavolo un album in cui c'erano le fotografie del figlio in altre età. Voleva fare il confronto e cominciò a tirar fuori dall'album le fotografie. Le tirò fuori tutte. Ne rimase una, l'ultima, la migliore. Egli era a cavallo di una seggiola con una camicia bianca, aveva gli occhi accigliati e con la bocca sorrideva. Era un'espressione tutta sua, la migliore. Con le mani piccole e agili, che, proprio quel giorno si movevano con particolare tensione nelle dita bianche, sottili, ella afferrò varie volte l'angolo della fotografia, ma la fotografia si lacerava ed ella non riusciva a staccarla. Il tagliacarte non era sul tavolo, ed ella, con una fotografia che prese lì accanto (una fotografia di Vronskij in cappello tondo e con i capelli lunghi fatta a Roma), tirò fuori la fotografia del figlio. ``Sì, eccolo!'' disse, guardando la fotografia di Vronskij, e a un tratto ricordò chi era la causa del suo presente dolore. Non si era ricordata di lui neppure una volta in tutta quella mattinata. Ma adesso, a un tratto, visto quel volto virile, nobile, a lei noto e caro, sentì un inatteso afflusso d'amore per lui. 

``Ma dov'è mai? Come mai mi lascia sola con le mie pene?'' pensò a un tratto con un sentimento di rimprovero, dimenticando che lei stessa gli nascondeva tutto quello che riguardava il figlio. Mandò da lui a chiedergli di venire subito; lo aspettava col cuore che veniva meno, cercando le parole con cui avrebbe detto tutto, e le espressioni d'amore di lui che l'avrebbero consolata. La persona mandata tornò con la risposta che egli aveva un ospite, ma che sarebbe andato subito da lei; aveva ordinato di chiederle se poteva riceverlo col principe Jašvin che era arrivato a Pietroburgo. ``Non verrà solo e da ieri a pranzo non mi ha visto; non verrà in modo che io possa dirgli tutto, ma verrà con Jašvin''. E a un tratto le venne uno strano pensiero: e s'egli si fosse disincantato di lei? 

E, riandando agli avvenimenti di quegli ultimi giorni, le sembrava di vedere in tutto una conferma di questo pensiero spaventoso: nel fatto ch'egli il giorno prima avesse pranzato fuori casa, che avesse insistito perché a Pietroburgo prendessero dimora separatamente, e che perfino adesso non venisse da lei solo, quasi evitando un incontro a quattr'occhi. 

``Ma deve dirmelo. Mi occorre saperlo. Se lo saprò, allora saprò cosa fare'' diceva a se stessa senza aver la forza di figurarsi la situazione in cui si sarebbe trovata, dopo essersi convinta della indifferenza di lui. Pensava ch'egli si fosse disincantato di lei, si sentiva vicina alla disperazione, ma gliene veniva una particolare forma di eccitamento. Sonò per la cameriera e andò nel bagno. Nel vestirsi si occupò più che in tutti quei giorni del suo abbigliamento, come se egli, dopo essersi disincantato di lei, potesse di nuovo amarla per quel vestito o per quella pettinatura che più le si addicevano. 

Sentì suonare il campanello prima di essere pronta. 

Quando uscì in salotto, non lui, ma Jašvin le venne incontro con lo sguardo. Quanto a lui, stava osservando le fotografie del figlio ch'ella aveva dimenticato sul tavolo e non aveva premura di guardarla. 

- Ci conosciamo - ella disse ponendo la sua piccola mano nella mano enorme di Jašvin che si era turbato (cosa strana data la sua natura gigantesca e il suo viso rude). - Ci conosciamo dall'anno passato, alle corse. Date qua - ella disse prendendo a Vronskij, con un movimento rapido, le fotografie del figlio, ch'egli stava guardando e fissandolo significativamente con gli occhi scintillanti. - E quest'anno sono state belle le corse? Invece di queste, io ho visto le gare sul Corso a Roma. Ma a voi non piace la vita all'estero - ella disse, sorridendo affabile. - Io vi conosco e conosco tutti i vostri gusti, pur avendovi visto poco. 

- Questo mi spiace, perché i miei gusti sono per lo più cattivi - disse Jašvin, mordicchiandosi il baffo sinistro. 

Dopo aver parlato un po' e notato che Vronskij guardava l'orologio, Jašvin le chiese se sarebbe rimasta ancora a lungo a Pietroburgo e, raddrizzata la figura enorme, prese il berretto. 

- Non credo a lungo - ella disse con impaccio, dopo aver guardato Vronskij. 

- Così non ci vedremo più? - disse Jašvin, alzandosi e rivolgendosi a Vronskij. - Dove pranzi? 

- Venite da me - disse Anna decisa, quasi irritata con se stessa della propria agitazione, e arrossendo come sempre quando mostrava dinanzi a una persona nuova la propria posizione. - Qui non si mangia bene, ma almeno starete insieme. Fra tutti i compagni di reggimento, a nessuno Aleksej vuol tanto bene quanto a voi. 

- Ne sono molto contento - disse Jašvin con un sorriso, dal quale Vronskij capì che Anna gli era piaciuta molto. 

Jašvin salutò e uscì, Vronskij rimase indietro. 

- Vai anche tu? - ella disse. 

- Sono già in ritardo - egli rispose. - Avviati! Ti raggiungo subito! - gridò a Jašvin. 

Ella lo prese per mano e, senza abbassare gli occhi, lo guardò, cercando cosa dire per trattenerlo. 

- Aspetta, devo dirti qualcosa - e, presa la mano corta di lui, se la premette contro il collo. - Non fa nulla che l'ho invitato a pranzo? 

- Hai fatto benissimo - egli disse con un sorriso calmo, scoprendo i denti serrati e baciandole la mano. 

- Aleksej, non sei cambiato verso di me? - ella disse, stringendogli la mano con tutte e due le sue. - Aleksej, io mi tormento qui. Quando andiamo via? 

- Presto, presto. Non puoi credere come sia penosa anche per me la nostra vita qui - egli disse, e ritrasse la mano. 

- Sì, va', va' - ella disse offesa, e s'allontanò rapida da lui. 

\capitolo{XXXII}\label{xxxii-3} 

Quando Vronskij tornò a casa, Anna non c'era ancora. Gli dissero che, subito dopo di lui, era venuta una signora ed ella era uscita con lei. Il fatto che fosse uscita senza lasciar detto dove fosse andata e che finora non fosse rientrata, che la mattina fosse andata ancora in un altro posto senza dirgli dove, tutto questo, insieme all'espressione stranamente eccitata del viso e il ricordo del tono ostile con cui, dinanzi a Jašvin, gli aveva quasi strappato di mano le fotografie del figlio, lo costrinse a riflettere. Decise che era indispensabile spiegarsi con lei. E l'aspettò in salotto. Anna però non rientrò sola, ma portò con sé una sua zia, una vecchia zitella, la principessa Oblonskaja. Era lei che era venuta la mattina e con lei Anna era andata a far delle spese. Anna pareva non notare l'espressione del viso di Vronskij, preoccupato e interrogativo, e gli raccontava allegramente quello che aveva comprato la mattina. Egli vedeva che in lei avveniva qualcosa di particolare: negli occhi scintillanti, quando, come un lampo, si fermavano su di lui, c'era un'attenzione tesa, e nel discorso e nei movimenti vi erano quella velocità e grazia nervosa che, nel primo tempo della loro unione, lo avevano affascinato tanto e che ora lo rendevano inquieto e timoroso. 

Il pranzo era apparecchiato per quattro. Tutti si erano riuniti per passare nella piccola sala da pranzo quando giunse Tuškevic con un'ambasciata per Anna da parte della principessa Betsy. La principessa Betsy pregava di scusarla se non era venuta a salutare; non stava bene, e pregava Anna di andare da lei fra le sei e mezzo e le nove. Vronskij guardò Anna nel sentire questa delimitazione di tempo che mostrava ch'erano state prese le misure necessarie perché ella non incontrasse nessuno; ma Anna sembrava non averlo notato. 

- È un gran peccato che io non possa proprio fra le sei e mezzo e le nove - disse, sorridendo appena. 

- La principessa se ne rammaricherà molto. 

- Anche io. 

- Forse andate a sentire la Patti? - disse Tuškevic. 

- La Patti? Mi date un'idea. Ci andrei se potessi avere un palco. 

- Posso pensarci io - si offrì Tuškevic. 

- Vi sarei molto, molto grata - disse Anna. - Ma non volete pranzare con noi? 

Vronskij alzò le spalle in modo appena percettibile. Non capiva in nessun modo quel che stesse facendo Anna. Perché s'era portata dietro quella vecchia principessa, perché faceva rimanere a pranzo Tuškevic e, cosa più stupefacente di tutto, perché mandava costui a prendere un palco? Era mai possibile pensare di andare, nella sua posizione, ad una serata d'abbonamento in cui cantava la Patti, in cui ci sarebbe stata tutta la società che lei conosceva? La guardò con uno sguardo serio, ma lei gli rispose con quello stesso sguardo provocante, fra l'allegro e il disperato, di cui egli non riusciva ad afferrare il senso. A pranzo, Anna fu animata e gaia; pareva che civettasse con Tuškevic e con Jašvin. Quando si alzarono da tavola e Tuškevic andò a prendere il palco e Jašvin a fumare, Vronskij scese con lui nelle proprie stanze. Dopo essere rimasto un po', corse di sopra. Anna era già vestita d'un abito chiaro di seta con velluto, fatto fare a Parigi, col petto scoperto e un prezioso merletto bianco in testa che le inquadrava il viso e faceva risaltare in modo particolare la sua bellezza luminosa. 

- Andate proprio a teatro? - disse lui, cercando di non guardarla. 

- Perché me lo domandate con tanto sgomento? - ella disse, ancor più offesa ch'egli non la guardasse. - Perché non dovrei andarci? 

Pareva non capire il significato delle parole di lui. 

- S'intende, non c'è nessuna ragione - egli disse, accigliandosi. 

- Ecco, questo proprio dico anch'io - confermò lei, non raccogliendo, con intenzione, l'ironia del tono di lui e rimboccando tranquillamente il lungo guanto profumato. 

- Anna, in nome di Dio! cosa v'è successo? - chiese lui, cercando di scuoterla, proprio come un tempo aveva fatto il marito. 

- Non capisco cosa mi domandiate. 

- Lo sapete che non si può andare. 

- Perché? Non andrò sola. La principessa Varvara è andata a vestirsi, verrà con me. 

Egli si strinse nelle spalle con aria perplessa e desolata. 

- Ma non sapete forse\ldots{} - prese a dire. 

- Ma io non voglio saperlo! - ella gridò quasi. - Non voglio. Mi pento forse di quello che ho fatto? No, no e no. E se si ricominciasse daccapo, sarebbe lo stesso. Per noi, per me e per voi, una cosa sola è importante: se ci amiamo l'un l'altro. E altre considerazioni non ci sono. Perché viviamo qui separatamente e non ci vediamo? perché non posso andare? Io ti amo e per me tutto il resto è indifferente - ella disse in russo, guardandolo con uno scintillio particolare, per lui incomprensibile - se tu non sei cambiato. Perché non mi guardi? 

Egli la guardò. Vedeva tutta la bellezza del suo viso e dell'abbigliamento che sempre le si adattava così bene. Ma proprio quella sua bellezza ed eleganza lo irritavano, adesso. 

- Il mio sentimento non può cambiare, lo sapete, ma vi prego di non andare, vi supplico - egli disse di nuovo in francese con tenera preghiera nella voce, ma con freddezza nello sguardo. 

Ella non sentiva le parole, ma vedeva la freddezza dello sguardo, e rispose con irritazione: 

- E io vi prego di dirmi perché non devo andare. 

- Perché questo vi può arrecare quel\ldots{} - si confuse. 

- Non capisco niente. Jašvin n'est pas compromettant e la principessa Varvara non è per nulla peggiore delle altre. Ma ecco anche lei. 

\capitolo{XXXIII}\label{xxxiii-2} 

Vronskij provava, per la prima volta, contro Anna, un sentimento di collera, quasi di rancore per la sua meditata incomprensione della propria situazione. Questo sentimento era rafforzato dal fatto ch'egli non poteva esprimergliene la causa. S'egli le avesse detto francamente quello che pensava, le avrebbe detto: ``Apparire in teatro in questo abbigliamento con la principessa che tutti conoscono, significa non solo riconoscere la propria situazione di donna perduta, ma anche gettare una sfida al mondo, cioè rinunciarvi per sempre''. 

Egli non poteva dirle questo. ``Ma come può non capire ciò? ma cosa accade in lei?'' si diceva. Sentiva come, in uno stesso tempo, diminuisse la propria stima verso di lei e aumentasse la coscienza della sua bellezza. 

Tornò accigliato in camera sua e, sedutosi vicino a Jašvin, che aveva disteso le sue lunghe gambe su di una sedia e beveva cognac con acqua di seltz, si fece portare la stessa cosa. 

- Tu dici Mogucij di Lankovskij. È un buon cavallo e ti consiglio di comprarlo - disse Jašvin dopo aver guardato il viso scuro del compagno. - Ha il posteriore che pende, ma per le gambe e la testa non si può desiderare nulla di meglio. 

- Credo che lo comprerò - rispose Vronskij. 

Il discorso sui cavalli lo interessava, ma non dimenticava neppure per un attimo Anna, e prestava involontariamente orecchio al suono dei passi per il corridoio, guardando di tanto in tanto l'orologio sul camino. 

- Anna Arkad'evna ha ordinato di riferire che è andata a teatro. 

Jašvin, versato un altro bicchierino di cognac nell'acqua di seltz, bevve e si alzò, abbottonandosi. 

- Ebbene, andiamo? - egli disse, sorridendo appena sotto i baffi e facendo vedere con questo sorriso che capiva la ragione dell'umor nero di Vronskij, ma che non ci dava importanza. 

- Io non vado - rispose, torvo, Vronskij. 

- E io devo, ho promesso. Su, arrivederci. Ma tu vieni in poltrona, prendi la poltrona di Krasinskij - aggiunse Jašvin uscendo. 

- No, ho da fare. 

``Con la moglie preoccupazioni, con chi non è moglie, peggio ancora'' pensò Jašvin, uscendo dall'albergo. 

Vronskij, rimasto solo, si alzò dalla sedia e si mise a camminare per la stanza. 

``Ma oggi cos'è? La quarta in abbonamento\ldots{} C'è Egor con la moglie e la madre, probabilmente. Significa che tutta Pietroburgo è là. Adesso sarà entrata, si sarà tolta la pelliccia, sarà apparsa alla luce. Tuškevic, Jašvin, la principessa Varvara\ldots{} - egli si immaginava. - E io che faccio mai? Ho paura forse e ho affidato a Tuškevic di proteggerla? Comunque si consideri la cosa, ciò è sciocco, è sciocco\ldots{} Ma perché mi mette in questa posizione?'' disse facendo un gesto con la mano. 

Con questo movimento si impigliò nel tavolino su cui erano l'acqua di seltz e la caraffa con il cognac e per poco non lo buttò giù. Volle afferrarlo, lo fece cadere e, per la stizza, diede un calcio al tavolo e sonò. 

- Se vuoi rimanere al mio servizio - disse al cameriere che era entrato - ricordati il tuo dovere. Che questo non accada più. Porta via. 

Il cameriere, sentendosi innocente, voleva giustificarsi, ma, guardato il signore, capì dal viso che bisognava tacere e, scusandosi in fretta, si chinò sul tappeto e cominciò a separare le bottiglie e i bicchieri interi da quelli rotti. 

- Questo non è affar tuo, manda il servo e preparami il frac. 

Vronskij entrò in teatro alle otto e mezzo. Lo spettacolo era in pieno fervore. La maschera, un vecchietto, tolse la pelliccia a Vronskij e, riconosciutolo, lo chiamò ``eccellenza'' e gli propose di non prendere il biglietto, ma di chiamare semplicemente Fëdor. Nel corridoio illuminato non c'era nessuno, all'infuori della maschera e di due inservienti con le pellicce sul braccio, che ascoltavano vicino alla porta. Di là dalla porta socchiusa si sentivano i suoni precisi d'un accompagnamento staccato dell'orchestra e una voce femminile che pronunciava chiaramente una frase musicale. La porta si aprì, lasciando passare una maschera che scivolò via, e la frase, che si avviava alla fine, colpì chiaramente l'udito di Vronskij. Ma la porta si chiuse subito e Vronskij non sentì la fine della frase della cavatina, ma capì, dal tuono di applausi di là dalla porta, che la cavatina era finita. Quando egli entrò nella sala tutta illuminata dai lampadari e dai becchi a gas di bronzo, il fragore continuava ancora. Sul palcoscenico la cantante, con le spalle nude, splendenti di brillanti, chinandosi e sorridendo, raccoglieva, con l'aiuto del tenore che la teneva per mano, i mazzi di fiori che volavano goffamente di là dalla ribalta, e si avvicinava a un signore con la scriminatura in mezzo ai capelli lucidi di pomata che protendeva, con le braccia lunghe attraverso la ribalta, un qualche cosa. Tutto il pubblico in platea, come anche nei palchi, si agitava, si spingeva in avanti, gridava e applaudiva. Il direttore d'orchestra, sul podio, aiutava e faceva da tramite e si riaccomodava la cravatta bianca. Vronskij entrò nella corsia della platea e, fermatosi, cominciò a guardare in giro. Quel giorno meno che mai fece attenzione all'ambiente abituale, al brusio, a tutto quel noto, poco interessante, variopinto gregge di spettatori nel teatro pieno zeppo. 

Nei palchi c'erano determinate signore, sempre le stesse, con determinati ufficiali in fondo al palco; le stesse, Dio sa quali, variopinte donne, e poi divise, soprabiti e la stessa folla sporca in loggione; e in tutta quella folla, tra palchi e prime file, c'erano una quarantina di uomini e di donne veri. E su questa oasi Vronskij rivolse subito la sua attenzione e prese contatto con essa. 

L'atto finiva quand'egli era entrato, e perciò, senza passare nel palco del fratello, andò fino alla prima fila di poltrone e si fermò accanto al proscenio con Serpuchovskoj il quale, piegando un ginocchio e battendo col tacco sulla ribalta, l'aveva visto da lontano, e se l'era chiamato a sé con un sorriso. 

Vronskij non aveva visto ancora Anna, e deliberatamente non guardava dalla parte sua. Ma sapeva, dalla direzione degli sguardi, dove si trovava. Senza farsi notare, guardava in giro, ma non la cercava; aspettandosi il peggio, cercava con gli occhi Aleksej Aleksandrovic. Per sua fortuna questa volta Aleksej Aleksandrovic non era in teatro. 

- Quanto poco di militare è rimasto in te! - disse Serpuchovskoj. - Un diplomatico, un artista, ecco, qualcosa di simile. 

- Già, appena sono tornato a casa, ho messo il frac - rispose Vronskij, sorridendo e tirando fuori il binocolo con lentezza. 

- Ecco, in questo, lo riconosco, ti invidio. Quando torno dall'estero e metto queste - e toccò le cordelline dell'uniforme - rimpiango la libertà. 

Serpuchovskoj aveva rinunciato già da tempo a far rientrare in servizio Vronskij, ma gli voleva bene come prima, e adesso era particolarmente gentile con lui. 

- Peccato, sei arrivato tardi per il primo atto. 

Vronskij, ascoltando con un orecchio solo, portava il binocolo dal primo ordine di palchi al secondo. Accanto a una signora in turbante e a un vecchietto calvo, che ammiccava rabbiosamente nella lente del binocolo, Vronskij, a un tratto, vide la testa di Anna, superba, meravigliosamente bella e sorridente nella cornice dei merletti. Era nel quinto palco della prima fila a venti passi da lui. Era seduta e, voltandosi leggermente, diceva qualcosa a Jašvin. L'attaccatura della testa sulle spalle belle e larghe e lo splendore contenuto ed eccitato dei suoi occhi e di tutto il viso, gli ricordarono proprio l'Anna ch'egli aveva visto al ballo di Mosca. Ma, adesso, egli sentiva in tutt'altro modo quella bellezza. Nel suo sentimento per lei non c'era adesso più nulla di misterioso, e perciò la bellezza di lei, pur attraendolo più fortemente di prima, lo offendeva a un tempo. Ella non guardava dalla sua parte, ma Vronskij sentiva che lo aveva già visto. 

Quando Vronskij diresse di nuovo il binocolo verso il palco, notò che la principessa Varvara era particolarmente arrossata, rideva con affettazione e guardava di continuo nel palco vicino; Anna, invece, chiuso il ventaglio e battendo con esso sul velluto rosso, guardava chi sa dove, ma non vedeva, ed evidentemente non voleva vedere quello che avveniva nel palco accanto. Sul viso di Jašvin c'era l'espressione che soleva avere quando perdeva al giuoco. Accigliatosi, ficcava sempre più profondamente in bocca il baffo sinistro e guardava di sbieco il palco vicino. 

In quel palco, a sinistra, c'erano i Kartasov. Vronskij li conosceva e sapeva che Anna era in rapporti amichevoli con loro. La Kartasova, una donna magra, piccola, stava ritta nel palco e, voltando la schiena ad Anna, si metteva la mantella che il marito le tendeva. Aveva il viso pallido e irritato, e diceva qualcosa con agitazione. Kartasov, un uomo grasso, calvo, volgendosi continuamente a guardare Anna, cercava di calmare la moglie. Quando la moglie uscì, il marito si attardò a lungo, cercando con gli occhi lo sguardo di Anna, con l'evidente desiderio di salutarla. Ma Anna, mostrando ostentatamente di non accorgersi di lui, voltandosi indietro, diceva qualcosa a Jašvin, che si chinava su di lei con la testa dai capelli corti. Kartasov uscì senz'aver potuto salutare, e il palco rimase vuoto. 

Vronskij non capì cosa fosse precisamente accaduto fra i Kartasov e Anna, ma capì che era accaduto qualcosa di umiliante per Anna. Lo capì e da quello che aveva visto e più ancora dal viso di Anna la quale, egli lo sapeva, aveva raccolto le sue ultime forze per sostenere la parte intrapresa. E questa parte di calma esteriore le riusciva perfettamente. Quelli che non conoscevano lei e il suo ambiente, e che non avevano sentito tutte le espressioni di compassione, di indignazione e di sorpresa da parte delle signore, perché ella si era permessa di farsi vedere in società, e in modo così appariscente, in quella sua acconciatura di merletto e in tutta la sua bellezza, quelli ne ammiravano la calma e la bellezza, e non sospettavano ch'ella provasse la sensazione d'una persona esposta al palo dell'infamia. 

Sapendo che qualcosa era accaduto, ma non sapendo precisamente cosa, Vronskij provava un'ansia assillante e, sperando di riuscire a informarsi, andò nel palco del fratello. Scelto il passaggio proprio di fronte al palco di Anna, nell'uscire si scontrò con l'antico comandante di reggimento che parlava con due amici. Vronskij sentì che era stato pronunciato forte il nome di Karenin, e notò che il comandante del reggimento si era affrettato a chiamare ad alta voce Vronskij, dopo aver guardato significativamente quelli che parlavano. 

- Ah, Vronskij! E quando al reggimento? Noi non ti lasciamo andare senza un banchetto. Tu hai radici più antiche di tutti noi - disse il comandante. 

- Non farò in tempo, mi spiace, sarà per un'altra volta - disse Vronskij, e corse su per la scala, nel palco del fratello. La vecchia contessa, madre di Vronskij, era nel palco del fratello, coi suoi riccioli d'acciaio. Varja, con la principessina Sorokina, gli vennero incontro nel corridoio del primo ordine. 

Accompagnata la principessina Sorokina dalla madre, Varja diede il braccio al cognato e subito cominciò a parlare con lui di quello che lo interessava. Era così agitata come egli non l'aveva vista mai. 

- Io ritengo ciò basso e disgustoso, e m.me Kartasova non aveva alcun diritto. M.me Karenina\ldots{} - cominciò. 

- Ma cosa? Io non so. 

- Come, non hai sentito? 

- Capisci, io sarò l'ultimo a sentirlo. 

- C'è un essere più perfido di questa Kartasova? 

- Ma che cosa ha fatto? 

- Me l'ha raccontato mio marito\ldots{} Ha offeso la Karenina. Suo marito aveva cominciato a parlare con lei attraverso il palco, e la Kartasova gli ha fatto una scenata. Dicono che abbia detto qualcosa di offensivo e che sia uscita. 

- Conte, vostra maman vi chiama - disse la principessina Sorokina affacciandosi alla porta del palco. 

- E io non faccio che aspettarti - gli disse la madre, sorridendo con irrisione. - Non ti si vede mai. 

Il figlio vedeva ch'ella non riusciva a trattenere un sorriso di compiacimento. 

- Buona sera, maman. Venivo da voi - disse freddo. 

- Come mai non vai faire la cour à madame Karenine? - soggiunse, quando la principessina Sorokina si fu allontanata. - Elle fait sensation. On oublie la Patti pour elle. 

- Maman, vi prego di non parlarmi di questo - egli rispose accigliato. 

- Io dico quello che dicono tutti. 

Vronskij non rispose nulla e, dette poche parole alla principessina Sorokina, uscì. Sulla porta incontrò il fratello. 

- Ah, Aleksej! - disse il fratello. - Che bassezza! Una stupida e niente più\ldots{} Volevo andar da lei adesso. Andiamo insieme. 

Vronskij non lo ascoltava. Andò giù a passi svelti: sentiva di dover agire, ma non sapeva come. La stizza contro di lei, perché poneva se stessa e lui in una situazione così falsa, mista alla pietà per le sue pene, lo agitava. Scese in platea e si avviò verso il palco di prim'ordine di Anna. Vicino al palco stava in piedi Stremov e parlava con lei. 

- Tenori non ce ne sono più. Le moule en est brisé. 

Vronskij le fece un inchino e si fermò a salutare Stremov. 

- Mi sembra che siate venuto in ritardo e che non abbiate ascoltato l'aria più bella - disse Anna a Vronskij guardandolo ironicamente, così almeno gli parve. 

- Sono un cattivo intenditore - disse lui, guardandola severo. 

- Come il principe Jašvin - disse lei, sorridendo - il quale trova che la Patti canta troppo forte. Vi ringrazio - ella disse, dopo aver preso nella piccola mano dal guanto lungo il programma sollevato da Vronskij, e in quell'attimo, il suo bel viso ebbe un brivido. Ella si alzò e andò in fondo al palco. 

Avendo notato che per l'atto seguente il palco di lei era rimasto vuoto, Vronskij, suscitando gli zittii del teatro che aveva fatto silenzio ai suoni di una cavatina, uscì dalla platea e se ne andò a casa. 

Anna era già a casa. Quando Vronskij entrò da lei, ella era sola e nella stessa acconciatura che aveva a teatro. Sedeva sulla prima poltrona vicino al muro e guardava davanti a sé. Lo guardò e immediatamente riprese la posizione di prima. 

- Anna - egli disse. 

- Tu, tu hai colpa di tutto! - ella gridò con lacrime di disperazione e rancore nella voce, alzandosi. 

- Ti ho pregato, ti ho supplicato di non andare, lo sapevo che sarebbe stato spiacevole per te. 

- Spiacevole! - gridò lei. - Terribile! Per quanto vivrò non dimenticherò. Ha detto che è vergognoso sedere accanto a me. 

- Parole di una donna sciocca! - egli disse - ma perché rischiare, provocare?\ldots{} 

- Io odio la tua calma. Tu non avresti dovuto ridurmi a questo. Se tu mi amassi\ldots{} 

- Anna! che c'entra la questione del mio amore?\ldots{} 

- Sì, se tu mi amassi come ti amo io, se tu ti tormentassi come me\ldots{} - disse lei, guardandolo con un'espressione di spavento. 

Egli sentiva pietà per lei, e tuttavia stizza. La rassicurava del proprio amore perché vedeva che adesso soltanto quest'unica cosa poteva calmarla, e non la rimproverava a parole, ma nell'animo suo la rimproverava. 

E quelle assicurazioni d'amore, che a lui sembravano così volgari tanto da provar vergogna nel pronunciarle, lei le sorbiva e a poco a poco si calmava. Il giorno dopo, completamente rappacificati, partirono per la campagna. 
