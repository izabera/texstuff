\parte{PARTE QUARTA}\label{parte-quarta} 
\pagestyle{pagina}

\capitolo{I}I Karenin, marito e moglie, continuavano a vivere nella stessa casa, s'incontravano ogni giorno, ma erano completamente estranei l'uno all'altra. Aleksej Aleksandrovic si era imposto la regola di vedere ogni giorno sua moglie, perché la servitù non avesse il diritto di sospettare, ma evitava di pranzare a casa. Vronskij non andava mai in casa di Aleksej Aleksandrovic, ma Anna lo vedeva fuori e il marito lo sapeva. 

La situazione era tormentosa per tutti e tre, e nessuno di loro sarebbe stato in grado di protrarla neppure di un giorno se non ne avesse atteso il mutamento e se non avesse avuto la convinzione che si trattava di una dolorosa difficoltà del momento che sarebbe passata. Aleksej Aleksandrovic aspettava che questo amore passasse così come passavano tutte le cose, così che tutti avrebbero poi dimenticato e il suo nome non sarebbe rimasto disonorato. Anna, dalla quale dipendeva la situazione, e che più di tutti ne era tormentata, la sopportava, perché non soltanto sperava, ma era fermamente convinta che tutto questo si sarebbe presto risolto e chiarito. Davvero non sapeva che cosa avrebbe potuto risolvere la situazione, ma era fermamente convinta che sarebbe avvenuto ora, molto presto. Vronskij, che senza volere si sottometteva a lei, aspettava anche lui qualche cosa di indipendente dal proprio volere che chiarisse tutte queste difficoltà. 

A metà dell'inverno Vronskij trascorse una settimana molto noiosa. Era stato addetto alla persona di un principe straniero venuto a Pietroburgo, per fargli conoscere le cose più notevoli della città. Vronskij era indubbiamente rappresentativo; inoltre aveva l'arte di essere deferente con dignità e aveva l'abitudine di trattare con persone di tale rango; perciò era stato addetto al principe. Ma l'incarico gli parve molto pesante. Il principe non voleva tralasciare di veder nulla di quanto, tornato a casa, gli avrebbero chiesto se avesse visto in Russia; e per quel che riguardava se stesso, voleva godersi quanto più possibile gli svaghi russi. Vronskij era obbligato a condurlo di qua e di là. La mattina andavano a visitare le cose notevoli, la sera partecipavano ai divertimenti nazionali. Il principe godeva di una salute non comune anche fra i principi, e con la ginnastica e le cure continue del corpo aveva acquistato un tale vigore che, malgrado gli eccessi a cui si abbandonava nei bagordi, era sempre fresco come un grosso cetriolo verde di qualità olandese. Aveva viaggiato molto e trovava che uno dei maggiori vantaggi dell'attuale facilità di comunicazioni consisteva nella possibilità di accedere agli svaghi di ogni paese. Era stato in Spagna, e là aveva fatto serenate e stretto amicizia con una spagnola che sonava il mandolino. In Svizzera aveva ucciso una Gemse. In Inghilterra aveva saltato a cavallo gli ostacoli in frac rosso, e aveva ucciso, per scommessa, duecento fagiani. in Turchia era stato in un harem, in India aveva viaggiato su di un elefante, e ora in Russia voleva assaggiare tutti gli svaghi peculiarmente russi. 

A Vronskij, che gli era accanto come un maestro di cerimonie, costava fatica distribuire tutti i divertimenti offerti al principe da varie personalità; c'erano, infatti, i cavalli trottatori, i bliny, la caccia all'orso, le trojke e gli zigani e le baldorie russe con la rottura delle stoviglie. E il principe con straordinaria facilità aveva fatto sua l'anima russa, fracassava vassoi e stoviglie, faceva seder sulle ginocchia una zigana e pareva chiedesse: ``Dunque, che altro c'è? possibile che solo in questo consista l'anima russa?''. 

In sostanza, a tutti gli svaghi russi il principe preferiva le attrici francesi, una ballerina dei balletti e lo champagne di marca bianca. Vronskij aveva le stesse abitudini del principe; ma o perché egli stesso era cambiato negli ultimi tempi, o per la troppo grande dimestichezza acquistata col principe, quella settimana gli parve terribilmente faticosa. Durante tutto il tempo, provò continuamente una sensazione simile a quella che prova un uomo addetto a un pazzo pericoloso che abbia paura del pazzo e che tema, nello stesso tempo, stando vicino a lui, di uscir di cervello. Vronskij sentiva ogni momento la necessità di non attenuare neppure per un attimo il tono di severa deferenza ufficiale verso il principe, per non essere da lui offeso. La maniera del principe di trattare con quelle stesse persone che, con stupore di Vronskij, non stavan più nella pelle dalla voglia di offrirgli divertimenti russi, era una maniera sprezzante. I suoi giudizi sulle donne russe che aveva desiderato conoscere avevano più di una volta costretto Vronskij ad arrossire d'indignazione. Ma la ragione principale per la quale il principe riusciva fastidioso a Vronskij, era che in lui vedeva riprodotto se stesso. E il fatto di guardare in quello specchio non lusingava il suo amor proprio. Era infatti un uomo molto fatuo e molto presuntuoso, molto sano e molto pulito, e null'altro. Era un gentleman, questo sì, era vero, e Vronskij non poteva negarlo. Era costante e dignitoso verso i superiori, schietto e semplice con i suoi pari e sprezzantemente buono con gli inferiori. Anche Vronskij si comportava così e riteneva tale comportamento una qualità; ma nei rapporti col principe egli era un inferiore e l'atteggiamento sprezzante e benevolo verso di lui lo indignava. 

``Che stupido bue! Possibile che anch'io sia fatto così?'' pensava. 

Così, quando dopo sei giorni si congedò da lui, prima che egli partisse per Mosca, e ne ricevette i ringraziamenti, fu felice d'essersi liberato da quella posizione di disagio e da quello specchio sgradevole. Lo salutò alla stazione, di ritorno dalla caccia all'orso dove, per tutta la notte, avevano assistito all'esibizione della bravura russa. 

\capitolo{II}Tornato a casa Vronskij trovò un biglietto di Anna. Diceva: ``Sono malata ed infelice. Non posso uscire in vettura, ma non posso resistere ancora senza vedervi. Venite di sera. Alle sette Aleksej Aleksandrovic va al consiglio e vi rimane fino alle dieci''. Dopo aver riflettuto un attimo sul fatto strano ch'ella, malgrado l'ingiunzione del marito di non riceverlo, lo invitasse a casa sua, decise di andare. 

In quell'inverno Vronskij era stato promosso colonnello, era uscito dal reggimento e viveva solo. Dopo aver fatto colazione, si sdraiò subito su di un divano, e in pochi minuti i ricordi delle disgustose scene della sua vita di quegli ultimi giorni, si confusero e si collegarono con l'immagine di Anna e di un certo contadino, esattore delle imposte, che aveva avuto una parte importante nella caccia all'orso, ed egli si addormentò. Si svegliò che era buio, tremando di terrore, e accese subito una candela: ``Che cos'è, che cosa ho visto di terribile nel sogno? Sì, sì, quel contadino esattore delle imposte, mi pare, piccolo, sudicio, con la barba arruffata, curvo, faceva qualcosa, e, a un tratto, s'è messo a dire certe strane parole in francese. Già, non c'era nient'altro che questo, nel sogno - si disse. - Ma perché era così spaventoso?''. Ricordò di nuovo con chiarezza il contadino e le incomprensibili parole francesi che quello pronunciava, e il terrore gli corse come un brivido per la schiena. 

``Che sciocchezza!'' pensò Vronskij, e guardò l'ora. Erano già le otto e mezzo. Chiamò il servo, si vestì in fretta e uscì sulla scala, del tutto dimentico del sogno e tormentato solo dal fatto di essere in ritardo. Avvicinandosi all'ingresso dei Karenin, guardò l'orologio e vide che erano le nove meno dieci. Una carrozza alta, stretta, alla quale era attaccata una coppia di cavalli grigi, stava ferma davanti all'ingresso. Riconobbe la carrozza di Anna. ``Viene lei da me - pensò Vronskij - e sarebbe meglio. Mi spiace entrare in questa casa. Ma è lo stesso, ormai non posso nascondermi'' si disse e, con quel modo di fare, sin dall'infanzia tutto suo, di chi sa di non aver da vergognarsi, uscì dalla slitta e si avvicinò alla porta. La porta si aprì e il portiere con uno scialle da viaggio in mano chiamò la carrozza. Vronskij, pur non abituato a notare i particolari, in quel momento osservò l'espressione di sorpresa con cui il portiere lo guardò. Proprio sulla porta, Vronskij quasi si scontrò con Aleksej Aleksandrovic. Un lume a gas illuminava in pieno il suo viso esangue, smagrito sotto il cappello nero e la cravatta bianca che spiccava fra il castoro del cappotto. Gli occhi immobili, appannati di Karenin guardarono in faccia Vronskij. Vronskij s'inchinò e Aleksej Aleksandrovic, dopo aver masticato un po' fra i denti, portò la mano al cappello e passò. Vronskij vide ch'egli, senza più guardare, sedeva nella vettura, prendeva attraverso il finestrino lo scialle e il binocolo e si rannicchiava dentro. Vronskij entrò in anticamera. Aveva le sopracciglia aggrottate e negli occhi brillava una scintilla cattiva e sprezzante. 

``Ecco che situazione! - pensò. - Se lottasse, se difendesse il suo onore, io potrei agire, dare sfogo ai miei sentimenti, ma questa debolezza o vigliaccheria\ldots{} mi fa apparire un traditore, e io non volevo e non voglio esserlo''. 

Dal tempo del suo colloquio con Anna nel giardino della Vrede, le idee di Vronskij erano molto cambiate. Egli, sottomettendosi involontariamente alla debolezza di Anna che gli si dava tutta e aspettava da lui la decisione del proprio destino, sottomettendosi fin dal principio a tutto, non pensava più da tempo che quel legame potesse finire, come aveva pensato in un primo momento. I suoi piani di ambizione si erano di nuovo ritirati in buon ordine e, sentendo di essere uscito ormai da quel tipo di attività in cui ogni cosa è ben definita, si era abbandonato al sentimento, e questo sentimento lo avvinceva sempre e sempre più forte a lei. 

Mentre era ancora nell'ingresso udì i passi di lei che si allontanavano. Capì che lo aspettava, che tendeva l'orecchio e che proprio allora era tornata nel salotto. 

- No - gridò vedendolo, e al primo suono della propria voce le vennero le lacrime agli occhi - no, se continuerò così, questo accadrà ancora molto, molto prima! 

- Che cosa, amica mia? 

- Che cosa? Io ti aspetto, mi tormento, un'ora, due. No, basta! Io non posso arrabbiarmi con te. Forse non potevi. No, non lo farò! 

Poggiò tutte e due le mani sulle spalle di lui e guardò a lungo, con uno sguardo profondo, entusiastico e nello stesso tempo indagatore, il viso di lui. Lo andava indagando come se volesse guardarlo per tutto il tempo che non lo aveva visto. Come sempre ad ogni incontro, fondava l'immaginaria figurazione di lui (incomparabilmente migliore, impossibile nella realtà), con lui, così com'era. 

\capitolo{III}-L'hai incontrato? - chiese quando furono seduti presso la tavola sotto la lampada. - Ecco la punizione per essere venuto in ritardo! 

- Già, ma come mai? Non doveva essere al consiglio? 

- C'è stato, ed è tornato per andare non so dove. Ma questo è nulla. Non ne parlare. Dove sei stato? Sempre col principe? 

Ella conosceva tutti i particolari della vita di lui. Egli voleva dire che non aveva dormito la notte e che perciò aveva preso sonno, ma guardando il viso agitato e felice di lei, ebbe rimorso. E disse che era dovuto andare a riferire sulla partenza del principe. 

- Ma ora è finito. Tu non crederai come ciò mi sia stato insopportabile. 

- Perché poi? Dopo tutto, è questa la vita che fate sempre voi uomini scapoli - ella disse, aggrottando le sopracciglia e, messasi al suo lavoro a maglia che era sulla tavola, cominciò a liberarne l'uncinetto, senza guardare Vronskij. 

- Io già da tempo l'ho abbandonata questa vita - egli disse, meravigliandosi del cambiamento di espressione del viso di lei, e cercando di penetrarne il senso. - E confesso - disse, mostrando con un sorriso i suoi denti bianchi e regolari - che, osservando questa vita in questa settimana, mi sono visto come in uno specchio, e me ne è venuta una sensazione sgradevole. 

Ella teneva in mano il lavoro a maglia, non lavorava e guardava lui con uno sguardo strano, luminoso e ostile. 

- Stamane Liza è venuta da me - saltò su a dire - quelli là ancora non hanno paura di venire da me, malgrado la contessa Lidija Ivanovna: e mi ha raccontato della vostra serata ateniese. Che orrore! 

- Io volevo dire solo che\ldots{} 

Ella lo interruppe. 

- Era quella Thérèse che conoscevi prima? 

- Volevo dire\ldots{} 

- Come siete disgustosi, voi uomini! Come mai non riuscite a immaginare che una donna questo non lo può dimenticare? - disse, riscaldandosi sempre più e rivelando così la ragione della sua irritazione. - Specialmente una donna che non conosce la tua vita. Che ne so io? che cosa potevo saperne? - ella diceva - quello che tu mi avresti detto. E come posso sapere che hai detto la verità? 

- Anna! Tu mi offendi. Non mi credi forse? Forse non ti ho detto che non c'è un pensiero in me che io non ti abbia rivelato? 

- Sì, sì - ella disse, cercando evidentemente di scacciare i pensieri di gelosia. - Ma se sapessi come è penoso per me! Io credo, sì, ti credo\ldots{} Così, che dicevi? 

Ma egli non poté ricordare quello che voleva dire. Questi accessi di gelosia, che negli ultimi tempi la prendevano sempre più spesso, lo atterrivano; e benché cercasse di nasconderlo, lo raffreddavano verso di lei, anche se riconosceva che la causa della gelosia era il suo amore per lui. Quante volte s'era detto che l'amore di lei era la felicità! ed ecco, ella lo amava come può amare una donna che per l'amore abbia distrutto tutti gli altri beni di questa vita, ma egli era molto più lontano dalla felicità ora, che quando era partito da Mosca per seguire lei. Allora egli si considerava infelice, ma la felicità era una cosa da raggiungere; ora, invece, sentiva che la felicità più piena era già nel passato. Ella già non era più quale egli l'aveva veduta nei primi tempi. E moralmente e fisicamente era peggiorata. Si era dilatata, e, nel momento in cui aveva accennato all'attrice, un'espressione cattiva era passata sul suo viso, alterandolo. Egli la guardava ora come un uomo guarda un fiore da lui colto e già appassito, nel quale a stento riconosce la bellezza che lo ha spinto a coglierlo e a distruggerlo. Malgrado questo, sentiva che allora, quando l'amore era più violento, egli avrebbe potuto, se l'avesse voluto fermamente, strapparlo quell'amore dal proprio cuore, ma adesso, quando, come in questo momento, gli pareva di non sentire più amore per lei, egli avvertiva che questo legame non poteva più essere spezzato. 

- Su, via, cosa mi volevi dire del principe? L'ho scacciato, l'ho scacciato ``il diavolo'' - aggiunse. Chiamavano fra di loro ``il diavolo'' la gelosia. - Già, cosa avevi cominciato a dire del principe? Perché per te è stato così faticoso? 

- Oh, insopportabile - egli disse, cercando di afferrare il filo del pensiero smarrito. - Non ci guadagna, a conoscerlo da vicino. Per definirlo, è un animale ottimamente nutrito, di quelli che alle esposizioni vincono i primi premi e niente più - disse con un dispetto che interessò lei. 

- No, come mai? - ribatté. - Però ha visto molte cose, è una persona colta. 

- È tutta un'altra cultura la loro. Serve solo per poterla disprezzare la cultura, così come disprezzano tutto, tranne i piaceri animaleschi. 

- Sì, però voi tutti li amate questi piaceri! - ella disse, e di nuovo egli notò quello sguardo torvo che lo sfuggiva. 

- Come mai lo difendi tanto? - egli disse, sorridendo. 

- Non lo difendo, mi è del tutto indifferente; ma penso che se a te stesso non fossero più piaciuti questi svaghi, avresti rifiutato di fargli compagnia. Ma a te piace guardare Teresa in costume d'Eva\ldots{} 

- Daccapo, daccapo ``il diavolo'' - disse Vronskij, prendendole la mano ch'ella aveva poggiato sulla tavola e baciandola. 

- Già, ma io non posso. Tu non sai come io mi sia tormentata, aspettandoti! Io penso di non essere gelosa. Non sono gelosa. Ti credo quando sei qui con me; ma quando tu sei chi sa dove, solo, e conduci quella tua vita per me incomprensibile\ldots{} 

Ella si scostò da lui, liberò finalmente l'uncinetto dal lavoro, e in fretta, con l'aiuto dell'indice, cominciarono a sovrapporsi una dopo l'altra le maglie di lana bianca che risplendevano sotto la luce della lampada; e, in fretta, il polso sottile nel polsino ricamato prese a girare nervosamente. 

- Dunque, dove l'hai incontrato Aleksej Aleksandrovic? - risonò a un tratto, senza naturalezza, la voce di lei. 

- Ci siamo scontrati sulla porta. 

- E lui ti ha salutato così? 

Ella allungò il viso e, socchiusi gli occhi, cambiò espressione, piegò le braccia, e sul suo bel volto Vronskij vide ad un tratto la stessa espressione con la quale Aleksej Aleksandrovic l'aveva salutato. Sorrise, ed ella rispose gaia con quel suo simpatico riso che era uno dei suoi incanti maggiori. 

- Non ci capisco nulla - disse Vronskij. - Se, dopo la tua rivelazione in campagna, l'avesse rotta con te, se mi avesse sfidato a duello\ldots{} ma questo non capisco: come può sopportare una simile posizione? Soffre, evidentemente. 

- Lui? - disse lei con un sorriso. - È completamente soddisfatto. 

- Perché ci tormentiamo, quando tutto potrebbe andar tanto bene? 

- Solo lui no. Che forse non lo conosco io, non so che è tutto impastato di menzogna? Si può forse, sentendo qualcosa, vivere come vive lui con me? Non capisce, non sente niente. Può forse un uomo che sente qualcosa, vivere nella stessa casa con la moglie colpevole? Può forse parlarle? Darle del tu? 

E di nuovo involontariamente gli rifaceva il verso: ``Tu, ma chère, tu, Anna!''. 

- Non è un uomo, non è un uomo, è un fantoccio. Nessuno sa quello che so io. Oh, al suo posto, da molto tempo avrei ucciso, avrei fatto a pezzi questa moglie come me, e non le direi, ``tu ma chère, Anna''. Non è un uomo, è una macchina ministeriale. Non capisce che io sono tua moglie, che lui è un estraneo, una cosa superflua. Non ne parliamo, non ne parliamo più\ldots{} 

- Tu hai torto, hai torto, amica mia - disse Vronskij, cercando di calmarla. - Ma fa lo stesso, non parliamo più di lui. Dimmi, cosa hai fatto? Come stai? Cos'è questo malessere e cosa ha detto il dottore? 

Ella lo guardava con una gioia irridente. Aveva trovato, si vedeva, altri lati ridicoli e deformi nel marito e aspettava il momento per rifarli. Ma egli continuò: 

- Io credo che non sia un malessere, ma che tutto dipende dal tuo stato. A quando? 

Lo scintillio irridente si spense negli occhi di lei, ma un altro sorriso, in cui c'era qualcosa di sconosciuto per lui e una calma tristezza, tramutò l'espressione di prima. 

- Preso, presto. Tu dicevi che la nostra posizione è tormentosa, che bisogna romperla. Se sapessi come mi è penosa, che cosa non darei per amarti liberamente e coraggiosamente! Non mi tormenterei e non ti tormenterei con la mia gelosia\ldots{} Sarà presto, ma non tanto come pensiamo. 

E al pensiero di come si sarebbe svolta la cosa, ella parve avere tanta pietà di se stessa che le lacrime le vennero agli occhi e non poté continuare. Posò la mano, che brillava sotto la lampada per gli anelli e la bianchezza, sulla manica della sua giacca. 

- Non sarà così come noi pensiamo. Io non volevo dirtelo, ma tu mi ci hai costretta. Presto, presto, tutto si risolverà, e noi tutti ci calmeremo e non ci tormenteremo più. 

- Non capisco - disse lui, pur comprendendola. 

- Tu mi domandavi quando? Presto. E io non sopravviverò. Non m'interrompere! - e si affrettò a parlare. - Lo so, lo so con certezza. Morirò e sono contenta di morire e di liberare me stessa e voi. 

Le lacrime le sgorgarono dagli occhi; egli si chinò sulla mano di lei e cominciò a baciarla, cercando di nascondere la propria agitazione che, egli sapeva, non aveva nessun fondamento, ma che non riusciva a dominare. 

- Ecco, così va meglio - ella diceva, stringendogli forte la mano. - Ecco la sola, la sola cosa che ci è restata. 

Egli tornò in sé e chinò il capo. 

- Che sciocchezza! Che assurda sciocchezza stai dicendo! 

- No, è vero. 

- Che cosa è vero? 

- Che morirò. Ho fatto un sogno. 

- Un sogno? - ripeté Vronskij, e in un baleno ricordò il contadino esattore del proprio sogno. 

- Già, un sogno - diceva lei. - Da tempo ho spesso questo sogno. Mi vedo correre in camera; devo prendere qualcosa, devo sapere qualcosa; sai, come succede in sogno - diceva lei, dilatando gli occhi pieni di terrore - e nella camera in un angolo c'è qualcosa. 

- Che sciocchezza, come puoi credere\ldots{} 

Ma lei non si lasciò interrompere. Quello che diceva era troppo importante per lei. 

- Ecco, questo qualcosa si volta e io vedo che è un contadino con la barba arruffata, piccolo che fa paura. Voglio fuggire, ma lui, ecco, si china sopra un sacco e con le mani ci rimesta dentro. - E rifaceva il gesto di quest'uomo che rimestava nel sacco. Il terrore era sul suo viso. E Vronskij, ricordando il proprio sogno, sentiva lo stesso terrore che riempiva l'animo di lei. - Rimesta e aggiunge in francese presto presto e, sai, biascica: ``Il faut battre le fer, le broyer, le pétrir\ldots{}''. E io dalla paura volevo svegliarmi, mi sono svegliata\ldots{} ma mi sono svegliata nel sogno. E ho cominciato a chiedermi che cosa significasse tutto questo. E Kornej mi dice: ``di parto, di parto morirete, di parto, padrona mia\ldots{}''. E mi sono svegliata. 

- Che sciocchezza, che sciocchezza! - diceva Vronskij, ma egli stesso sentiva che non c'era convinzione nella propria voce. 

- Non ne parliamo più. Suona, ordinerò di portare il tè. Ma aspetta, ora non c'è molto, io\ldots{} 

Ma improvvisamente tacque. L'espressione del suo viso si era istantaneamente cambiata. Il terrore e l'agitazione si erano mutati in una espressione di calma, pacata e felice attenzione. Egli non poté capire il segreto di questo mutamento. Ella aveva sentito agitarlesi in seno la nuova vita. 

\capitolo{IV}Aleksej Aleksandrovic, dopo l'incontro con Vronskij sulla scala di casa sua, era andato, come aveva stabilito, all'opera italiana. Ci era rimasto per due atti e aveva visto tutti quelli che aveva interesse di vedere. Tornato a casa, esaminò attentamente l'attaccapanni, e, notato che il cappotto militare non c'era più, passò, secondo il solito, in camera sua. Ma, contrariamente al solito, non si mise a letto, passeggiò avanti e indietro per lo studio fino alle tre di notte. Un senso di rancore verso la moglie, che non aveva voluto rispettare le convenienze e adempiere l'unica condizione impostale: non ricevere in casa sua l'amante, non gli dava pace. Ella non aveva adempiuto la sua richiesta, ed egli doveva punirla e dar corso alla sua minaccia: chiedere il divorzio e toglierle il figlio. Egli conosceva tutte le difficoltà collegate a questa faccenda, ma aveva detto che lo avrebbe fatto ed ora doveva mettere in atto la minaccia. La contessa Lidija Ivanovna aveva accennato che questa era la migliore delle soluzioni per il suo caso e, negli ultimi tempi, la pratica dei divorzi era giunta a un punto di perfezione che Aleksej Aleksandrovic vedeva la possibilità di superare le difficoltà formali. Inoltre, una disgrazia non viene mai sola, e l'affare della sistemazione degli allogeni e quello dell'irrigazione dei campi del governatorato di Zarajsk avevano procurato ad Aleksej Aleksandrovic tali dispiaceri burocratici che in tutto quest'ultimo tempo egli era stato in uno stato di estrema agitazione. Non aveva dormito tutta la notte e la sua irritazione, con un enorme crescendo, era giunta al mattino agli estremi limiti. Si vestì in fretta e, come se portasse in mano una coppa piena di fiele e temesse di versarla e di perdere insieme col fiele l'energia necessaria per la spiegazione con la moglie, andò da lei non appena seppe che si era alzata. 

Anna, che credeva di conoscere bene suo marito, fu sorpresa dal suo aspetto quando egli entrò. La fronte era corrugata, gli occhi guardavano torvi in avanti, evitando lo sguardo di lei; la bocca era come sigillata dalla durezza e dallo sprezzo. Nell'andatura, nei gesti, nel tono della voce, c'erano una decisione e una fermezza, quali la moglie non aveva mai visto in lui. Entrò nella stanza e, senza salutarla, si diresse diritto verso lo scrittoio e, afferrate le chiavi, aprì il cassetto. 

- Che vi occorre? - gridò lei. 

- Le lettere del vostro amante - egli disse. 

- Non sono qui - ella disse, chiudendo il cassetto, ma da questo gesto egli capì che aveva indovinato giusto e, respinta rudemente la mano di lei, afferrò rapido il portafogli nel quale sapeva ch'ella metteva le carte più interessanti. Ella fece per strapparglielo di mano, ma egli la respinse. 

- Sedete, devo parlarvi - disse, mettendosi il portafogli sotto il braccio e stringendolo così forte col gomito da sollevar la spalla. 

Ella lo guardava sorpresa e spaventata. 

- Vi ho detto che non vi avrei mai permesso di ricevere il vostro amante in casa mia. 

- Avevo bisogno di vederlo per\ldots{} 

Si fermò non trovando nessuna giustificazione. 

- Non entro nei particolari del perché una donna debba veder l'amante. 

- Io volevo, io soltanto\ldots{} - disse, avvampando. La sua villania l'aveva irritata e le aveva dato coraggio. - Possibile che non sentiate come vi sia facile offendermi? - ella disse. 

- Si può offendere un uomo onesto e una donna onesta, ma dire a un ladro che è un ladro è solo la constatation d'un fait. 

- Questo nuovo tratto di crudeltà non mi era ancora noto in voi. 

- Voi chiamate crudeltà il fatto che un marito lasci la libertà alla propria moglie, dandole l'onesto asilo del suo nome sotto la sola condizione di salvare le convenienze? Questa è crudeltà? 

- È peggiore della crudeltà, questa è vigliaccheria, se volete saperlo - gridò Anna in uno scoppio di rabbia e, alzatasi, fece per andar via. 

- No - gridò lui con la sua voce stridula di un tono più alto del normale e, afferratala con le sue dita grosse per un braccio in maniera così forte da farle restare impressi i segni rossi del bracciale che vi aveva premuto, la costrinse a sedere al suo posto. - Vigliaccheria? Se la volete usare questa parola, vigliaccheria è questo: abbandonare il marito e il figlio per un amante, e mangiare il pane del marito! 

Ella chinò il capo. Non solo non disse quello che aveva detto all'amante, che era lui suo marito e che il marito era un intruso: non lo pensò neppure. Sentiva tutta la verità che era nelle parole di lui e disse solo, sottovoce: 

- Voi non potevate definire la mia situazione in modo peggiore di quello che io non la senta; ma perché dite tutto ciò? 

- Perché lo dico? perché? - continuò, sempre irritato, lui. - Perché lo sappiate: poiché non avete adempiuto la mia volontà nell'osservare le convenienze, io agirò in modo che questa situazione finisca. 

- Presto, presto finirà anche se dura così - ella disse e di nuovo le lacrime, al pensiero della morte vicina, ora desiderata, le vennero agli occhi. 

- Finirà più presto di quanto non immaginiate, con il vostro amante! Voi avete bisogno di soddisfare una passione carnale\ldots{} 

- Aleksej Aleksandrovic! Io non dico che questo sia poco generoso, ma è disonesto percuotere chi è a terra. 

- Già! è solo a voi che pensate! Ma le sofferenze di un uomo che è stato vostro marito non vi interessano. Per voi è indifferente che tutta la sua vita sia crollata, che egli abbia sop\ldots{} sop\ldots{} sopperto. 

Aleksej Aleksandrovic parlava così in fretta che s'era impappinato e non riusciva in alcun modo a pronunciare la parola ``sofferto''. Aveva finito col pronunciare ``sopperto''. A lei venne da ridere, ma immediatamente ebbe vergogna che qualcosa potesse eccitarle il riso in un momento simile. E per la prima volta, per un attimo, si trasferì in lui, soffrì il suo dolore, e ne ebbe pena. Ma cosa mai poteva dire o fare? Abbassò la testa e tacque. Egli pure tacque per un po' di tempo e poi cominciò a parlare con una voce già meno stridula, con una voce fredda, sottolineando le parole scelte a caso, che non avevano nessuna particolare importanza. 

- Sono venuto a dirvi\ldots{} - egli disse. 

Ella lo guardò. ``No, mi è parso - pensò, ricordando la espressione del suo viso quando s'era impappinato sulla parola''sofferto``; - no, può forse un uomo con quegli occhi appannati, con quella presuntuosa calma sentire qualche cosa?''. 

- Io non posso cambiare nulla - ella mormorò. 

- Son venuto a dirvi che domani parto per Mosca, e non tornerò più in questa casa, e voi avrete notizia delle mie decisioni attraverso l'avvocato, al quale affiderò la pratica del divorzio. Mio figlio andrà da mia sorella - disse Aleksej Aleksandrovic, ricordando con uno sforzo quel che voleva dire del figlio. 

- Vi occorre Serëza per farmi del male - ella disse guardandolo di sotto in su. - Voi non lo amate\ldots{} Lasciatemi Serëza! 

- E sì, ho perso anche l'amore per mio figlio, perché la mia avversione per voi ha trascinato insieme anche lui. Tuttavia lo prenderò. Addio! 

E voleva andar via, ma ora fu lei a trattenerlo. 

- Aleksej Aleksandrovic, lasciatemi Serëza! - mormorò lei ancora una volta. - Io non ho altro da dire. Lasciatemi Serëza fino al mio\ldots{} Partorirò presto, lasciatemelo! 

Aleksej Aleksandrovic s'infiammò, e, svincolato bruscamente il braccio, uscì dalla stanza in silenzio. 

\capitolo{V}La sala d'aspetto del noto avvocato di Pietroburgo era piena di gente quando Aleksej Aleksandrovic vi entrò. Tre signore, una anziana, una giovane e la moglie di un mercante; tre signori: il primo, un banchiere tedesco con un anello al dito, il secondo, un commerciante con la barba, il terzo, un impiegato rabbioso in piccola tenuta con una decorazione al collo, aspettavano già da tempo. Due procuratori scrivevano sui tavoli, facendo stridere le penne. Il materiale per scrivere che Aleksej Aleksandrovic amava in maniera particolare, era veramente buono. Aleksej Aleksandrovic non poté non notarlo. Uno dei procuratori, senza alzarsi, accigliatosi, si voltò irritato verso Aleksej Aleksandrovic. 

- Che vi occorre? 

- Ho un affare con l'avvocato. 

- L'avvocato è occupato - rispose severo il procuratore, indicando con la penna quelli che aspettavano, e continuò a scrivere. 

- Non può trovare un po' di tempo? - disse Aleksej Aleksandrovic. 

- Non ha tempo libero, è sempre occupato. Favorite aspettare. 

- Volete allora compiacervi di dargli il mio biglietto da visita? - disse dignitosamente Aleksej Aleksandrovic, vedendo l'impossibilità di mantenere l'incognito. 

Il procuratore prese il biglietto e, disapprovandone evidentemente il contenuto, sparì dietro una porta. 

Aleksej Aleksandrovic era propenso in teoria a un'azione giudiziaria; ma alcuni particolari di procedura non li approvava del tutto per certe altre ragioni di ufficio a lui note, e li criticava per quanto poteva criticare una qualunque cosa che avesse avuto la sanzione sovrana. Tutta la sua vita era trascorsa in attività burocratiche, perciò quando non concordava con qualche cosa, questo disaccordo veniva attenuato dalla consapevolezza dell'inevitabilità degli errori e della possibilità della loro correzione in un qualsiasi settore. Nelle nuove istituzioni giudiziarie non approvava le condizioni in cui era stata posta l'avvocatura. Ma finora non aveva mai avuto a che fare con essa, perciò la disapprovava solo in teoria; ora invece la sua disapprovazione s'era rafforzata per l'impressione sfavorevole prodottagli dalla sala d'aspetto dell'avvocato. 

- Viene subito - disse il procuratore, e, dopo due minuti, apparvero sulla soglia la figura lunga di un vecchio giurista che parlava con l'avvocato e l'avvocato stesso. 

L'avvocato era piccolo, tarchiato, calvo, con una barba rossoscura, le sopracciglia chiare, lunghe e la fronte rialzata. Era parato come uno sposo, dalla cravatta e dalla catena doppia fino alle scarpe di coppale. Il viso era intelligente, maschio; gli ornamenti da zerbinotto e di pessimo gusto. 

- Accomodatevi - disse l'avvocato, rivolgendosi ad Aleksej Aleksandrovic e, fatto passare l'accigliato Karenin davanti a sé, chiuse la porta. 

- Non vi dispiace? - e indicò la poltrona presso la scrivania carica di carte, mentre egli stesso sedeva al posto presidenziale, fregandosi le piccole mani dalle dita corte, coperte di peli bianchi, e piegando la testa da un lato. Ma s'era appena assestato nella sua posa, che sulla tavola volò una tignola. L'avvocato con una velocità quale non ci si poteva aspettare da lui, aprì le mani, afferrò la tignola e riprese la posa di prima. 

- Prima di parlare del mio affare - disse Aleksej Aleksandrovic seguendo sorpreso con gli occhi i movimenti dell'avvocato - devo farvi osservare che la questione della quale vengo a parlarvi, deve rimanere segreta. 

Un sorriso appena percettibile sollevò i baffi rossicci e prominenti dell'avvocato. 

- Non sarei un avvocato se non sapessi custodire i segreti affidatimi. Ma se desiderate una conferma\ldots{} 

Aleksej Aleksandrovic lo guardò in faccia e vide che gli occhi grigi, intelligenti, ridevano e pareva che sapessero già tutto. 

- Voi sapete il mio nome? - proseguì Aleksej Aleksandrovic. 

- Conosco voi e la vostra utile\ldots{} - e qui di nuovo afferrò una tignola - attività, come ogni russo, del resto - disse l'avvocato, inchinandosi. 

Aleksej Aleksandrovic sospirò, raccogliendo le proprie forze. Ma una volta deciso, continuò con la sua voce stridula, senza timori, senza esitazioni e sottolineando alcune parole. 

- Io ho la sventura - cominciò Aleksej Aleksandrovic - di essere un marito tradito e desidero rompere legalmente i rapporti con mia moglie, cioè divorziare, ma in modo che mio figlio non rimanga con la madre. 

Gli occhi grigi dell'avvocato cercavano di non ridere, ma saltellavano qua e là per una gioia irrefrenabile, e Aleksej Aleksandrovic vedeva che in essi c'era non la sola gioia di un uomo che riceveva un incarico vantaggioso, ma come una certa entusiastica festosità, c'era uno scintillio simile a quello cattivo che aveva sorpreso negli occhi della moglie. 

- Voi desiderate la mia collaborazione per ottenere il divorzio? 

- Sì, proprio, vi devo però dire che rischio di abusare della vostra cortesia. Sono venuto solo a consigliarmi preventivamente con voi. Desidero il divorzio, ma per me è importante conoscere le forme attraverso le quali è possibile ottenerlo. Molto probabilmente, dunque, se le forme non coincideranno con le mie aspirazioni, rinuncerò all'azione legale. 

- Oh, è sempre così - disse l'avvocato - e questo dipende del tutto dalla vostra volontà. 

L'avvocato abbassò gli occhi verso i piedi di Aleksej Aleksandrovic, sentendo di poter offendere il cliente con la propria incontenibile ilarità. Guardò una tignola che volava davanti al naso di lui e tese la mano, ma non l'afferrò per un riguardo alla situazione di Aleksej Aleksandrovic. 

- Benché nelle linee generali le nostre disposizioni legislative in materia mi siano note - continuò Aleksej Aleksandrovic - desidererei conoscere le forme con cui in pratica si compiono gli affari di questo genere. 

- Voi desiderate - rispose l'avvocato senza alzare gli occhi, entrando non senza soddisfazione nel tono del discorso del suo cliente - che io vi esponga le vie attraverso le quali è possibile attuare il vostro proposito. 

E, a un cenno affermativo del capo di Aleksej Aleksandrovic, continuò, guardando solo di rado, di sfuggita, il viso chiazzato di Aleksej Aleksandrovic. 

- Il divorzio, secondo le nostre leggi - disse con una leggera disapprovazione verso le leggi - è possibile, come vi è noto, nei seguenti casi\ldots{} Aspettate! - disse rivolto al procuratore che s'era infilato attraverso la porta, tuttavia si alzò, gli disse alcune parole e sedette di nuovo. - Nei seguenti casi: incapacità fisica dei coniugi, assenza di notizie per cinque anni - disse, piegando il pollice ricoperto di peli - poi adulterio - pronunciò questa parola con evidente soddisfazione. - Le suddivisioni sono le seguenti - egli continuava a piegare le dita grasse, sebbene i casi e le loro suddivisioni non potessero essere elencati insieme: - difetti fisici del marito o della moglie, quindi adulterio del marito o della moglie. - Avendo piegato tutte le dita, le raddrizzò e continuò: - Questo è il punto di vista teorico, ma io suppongo che voi mi abbiate fatto l'onore di rivolgervi a me per conoscere l'applicazione pratica. E perciò, facendomi guidare dagli antecedenti, devo dirvi che i casi di divorzio si riducono tutti ai seguenti: difetti fisici, niente, a quanto posso capire, e così pure assenza di notizie?\ldots{} 

Aleksej Aleksandrovic chinò il capo affermativamente. 

- Si riducono ai seguenti: adulterio di uno dei coniugi e prova della colpevolezza di una delle parti risultante da reciproco accordo, oppure, fuori di un tale accordo, prova legale. Devo far presente che nella pratica l'ultimo caso s'incontra di rado - disse l'avvocato e, guardando di sfuggita Aleksej Aleksandrovic, tacque come un venditore di pistole che, descritti i vantaggi di questa o di quell'arma, attenda la scelta del compratore. Ma Aleksej Aleksandrovic taceva e perciò l'avvocato proseguì: - La cosa più usuale e semplice e la più ragionevole a mio modo di vedere, è l'adulterio ammesso da accordo reciproco. Non mi sarei permesso di esprimermi così, parlando con una persona di scarsa cultura - disse l'avvocato - ma suppongo che per voi sia comprensibile. 

Aleksej Aleksandrovic era così sconvolto che non capì subito la ragionevolezza dell'adulterio ammesso per reciproco accordo e si leggeva la perplessità nel suo sguardo; ma l'avvocato gli venne subito in aiuto. 

- Due persone non possono più vivere insieme, ecco il fatto. E se tutti e due sono in questo d'accordo, allora i particolari e i modi diventano indifferenti. E nello stesso tempo questo è il mezzo più semplice e sicuro. 

Aleksej Aleksandrovic ora capiva in pieno. Ma aveva le sue esigenze religiose che gli impedivano l'accettazione di questa procedura. 

- Questo è fuori questione nel caso presente - egli disse. - Qui una sola cosa è possibile: la prova del fatto è confermata dalle lettere che sono in mio possesso. 

L'avvocato a sentir nominare le lettere, strinse le labbra ed emise un suono flebile tra il compassionevole e lo sprezzante. 

- Abbiate la bontà di considerare - egli riprese - che gli affari di questo genere sono decisi, come vi è noto, dalla giurisdizione ecclesiastica; e i preti, in affari di questo genere, sono ghiotti di particolari minutissimi - egli disse con un sorriso che mostrava simpatia verso il gusto dei preti. - Le lettere, senza dubbio, possono confermare altri elementi, ma le prove devono essere ottenute per via diretta, cioè per testimoni. D'altra parte, poi, se mi fate l'onore di degnarmi della vostra fiducia, riservatemi la scelta dei mezzi da adoperare. Chi vuole il resultato, accetti anche i mezzi. 

- Se è così\ldots{} - cominciò, improvvisamente impallidendo Aleksej Aleksandrovic, ma, in quel momento, l'avvocato si alzò e andò di nuovo verso la porta dove era il procuratore che l'aveva interrotto. 

- Ditele che non siamo al mercato! - disse e ritornò da Aleksej Aleksandrovic. 

Tornando al posto, afferrò, inosservato, un'altra tignola. ``Sarà bello il mio reps in estate\ldots{}'' pensò, accigliandosi. 

- Così voi state dicendo\ldots{} - disse. 

- Vi comunicherò la mia decisione per iscritto - disse Aleksej Aleksandrovic alzandosi, e si appoggiò alla tavola. Dopo essere rimasto in piedi un po' in silenzio, disse: - Dalle vostre parole posso concludere che il conseguimento del divorzio è possibile. Vi pregherei di comunicarmi pure quali sono le vostre condizioni\ldots{} 

- Tutto è possibile se mi concederete piena libertà di azione - disse l'avvocato senza rispondere in tutto alla domanda. - Quando posso contare di ricevere vostre comunicazioni? - chiese, dirigendosi verso la porta e brillando con gli occhi e gli stivaletti di coppale. 

- Fra una settimana. E, se assumete il patrocinio di questo affare, sarete gentile di comunicarmi a quali condizioni. 

- Molto bene. 

L'avvocato si inchinò ossequioso, fece uscire dalla porta il cliente, e, rimasto solo, si abbandonò alla propria ilarità. Divenne così allegro che, contrariamente alle sue abitudini, fece uno sconto alla signora che contrattava con lui, cessò di afferrar tignole e decise che per l'inverno prossimo occorreva ricoprire il mobilio con del velluto, come da Sigonin. 

\capitolo{VI}Aleksej Aleksandrovic aveva ottenuto una brillante vittoria nella seduta della commissione del 17 agosto, ma le conseguenze di questa vittoria gli spezzarono le ali. La nuova commissione per investigare sotto tutti i rapporti la vita degli allogeni, era stata sostituita e mandata sul posto con inconsueta sollecitudine ed energia suscitate da Aleksej Aleksandrovic. Dopo tre mesi era stata presentata la relazione. La vita degli allogeni era stata osservata sotto l'aspetto politico, amministrativo, economico, etnografico, materiale e religioso. E a tutti questi problemi erano state date delle risposte ottimamente redatte ed inequivocabili, che non ammettevano dubbi, poiché non erano il prodotto del pensiero di un uomo soggetto a errori, ma erano tutte frutto di un'attività burocratica. Le risposte erano il risultato dei dati ufficiali, dei rapporti dei governatori e degli arcivescovi, basati sui rapporti dei capi dei distretti e dei sovrintendenti ecclesiastici, basati, a loro volta, sui rapporti delle amministrazioni comunali e dei prelati e perciò tutte queste risposte erano indubitabili. Tutte le questioni a proposito del perché, per esempio, i raccolti fossero scarsi, del perché gli abitanti si attenessero alle loro credenze e così via, questioni che, senza la comodità della macchina burocratica, non si risolvono e non possono essere risolte per secoli, ebbero una chiara indubitabile soluzione. E questa soluzione era in favore dell'idea di Aleksej Aleksandrovic. Ma Stremov, sentitosi toccato nel vivo nell'ultima seduta, all'arrivo delle relazioni della commissione usò una tattica che Aleksej Aleksandrovic non sospettava, e, non solo difese con molto calore l'attuazione delle misure proposte da Karenin, ma ne propose altre che di queste erano le estreme conseguenze. Queste misure, eccessive rispetto a quello che era il pensiero fondamentale di Aleksej Aleksandrovic, furono accolte, e allora la tattica di Stremov si scoprì. Queste misure, portate all'estremo, apparvero d'un tratto così assurde che sia gli uomini di stato che l'opinione pubblica e le signore intellettuali e i giornali, tutti, nello stesso tempo, vi si scagliarono contro, esprimendo la propria indignazione, e si scagliarono contro le misure stesse e contro il loro fautore, Aleksej Aleksandrovic. Stremov allora si fece da parte, fingendo d'aver solo voluto ciecamente seguire il piano di Karenin, mentre egli stesso si mostrava sorpreso e confuso di quello che era stato fatto. Questo spezzò le ali ad Aleksej Aleksandrovic. Ma nonostante la salute che deperiva e i dispiaceri familiari, Aleksej Aleksandrovic non si arrese. In seno alla commissione si era prodotta una scissione. Alcuni membri con Stremov a capo, giustificavano il proprio errore dicendo d'aver avuto fiducia nella commissione di revisione guidata da Aleksej Aleksandrovic, la quale aveva presentato un rapporto che altro non era risultato che un'assurdità e della carta scritta. Aleksej Aleksandrovic, insieme col partito delle persone che vedevano il pericolo di considerar le pratiche in modo così rivoluzionario, seguitava a sostenere i dati elaborati dalla commissione ispettiva. In seguito a ciò, nelle alte sfere e persino nei salotti si confuse tutto e, malgrado questo interessasse estremamente tutti, nessuno riusciva a capire se gli allogeni fossero realmente nella miseria e perissero o se prosperassero. La posizione di Aleksej Aleksandrovic, in conseguenza di ciò e in parte in conseguenza del disonore caduto su di lui per l'infedeltà della moglie, si fece molto vacillante. Ma pure in questo stato di cose, Aleksej Aleksandrovic prese una decisione importante. Con stupore della commissione, dichiarò che avrebbe chiesto l'autorizzazione ad andare sul posto per ispezionare di persona. E, sollecitatane l'autorizzazione, Aleksej Aleksandrovic si mise in viaggio per governatorati lontani. 

Il viaggio di Aleksej Aleksandrovic suscitò grande scalpore, tanto più che, proprio all'atto di partire, egli restituì ufficialmente, con documento, il denaro assegnatogli per le spese dei dodici cavalli necessari per raggiungere il luogo della missione. 

- Giudico questo molto nobile - diceva a questo proposito Betsy con la principessa Mjagkaja. - Perché dare il denaro per i cavalli da posta, quando tutti sanno che adesso ci sono dovunque le strade ferrate? 

Ma la Mjagkaja non era d'accordo e l'opinione della Tverskaja la irritava persino. 

- Voi parlate bene - ella disse - voi che avete non so quanti milioni, ma a me piace che mio marito vada in missione nell'estate. Gli fa bene alla salute, gli fa piacere fare un viaggio e io ormai ho stabilito che con quel denaro, in casa mia, si pagano carrozza e cocchiere. 

Di passaggio per governatorati lontani, Aleksej Aleksandrovic si fermò tre giorni a Mosca. 

Il giorno dopo il suo arrivo, andò in carrozza a far visita al governatore generale. All'incrocio del vicolo Gazetnyj, dove si affollano sempre carrozze e vetture, Aleksej Aleksandrovic sentì ad un tratto il proprio nome gridato da una voce così forte e vivace che non poté non rimanere colpito. All'angolo del marciapiedi, in cappotto corto alla moda, con un cappello a falde strette messo di lato e un sorriso splendente fra le labbra rosse e i denti bianchi, allegro, giovane, raggiante, stava Stepan Arkad'ic che, energicamente e imperiosamente gridava e pretendeva che egli si fermasse. Si teneva con una mano al finestrino di una carrozza che s'era fermata all'angolo, dalla quale si sporgevano una testa di donna con un cappello di velluto e due testoline di bimbi, e sorrideva e faceva segno con l'altra mano al cognato. La signora sorrideva con un sorriso buono, e faceva anche lei dei gesti al Aleksej Aleksandrovic. 

Era Dolly con i bambini. 

Aleksej Aleksandrovic aveva deciso di non vedere nessuno a Mosca e meno di tutti il fratello di sua moglie. Sollevò il cappello, e voleva passar via, ma Stepan Arkad'ic ordinò al cocchiere di fermare e corse verso di lui sulla neve. 

- Ma come non far sapere niente! È molto che sei qui? E io ieri sono stato da Dussau e ho visto sulla tabella ``Karenin'' e non m'è venuto in mente che fossi tu! - diceva Stepan Arkad'ic , ficcandosi con la testa nel finestrino della carrozza. - E sarei venuto io. Come sono contento di vederti! - diceva, battendo un piede contro l'altro per scuoter via la neve. - Ma come non senti di essere colpevole a non farti vedere! - ripeteva. 

- Non ne ho avuto il tempo, sono molto occupato - rispose secco Aleksej Aleksandrovic. 

- Andiamo da mia moglie, vuole tanto vederti. 

Aleksej Aleksandrovic si sbarazzò dello scialle nel quale erano avviluppate le sue gambe freddolose e, uscito dalla carrozza, si fece strada fra la neve verso Dar'ja Aleksandrovna. 

- Che c'è mai, Aleksej Aleksandrovic, perché ci evitate così? - disse Dolly sorridendo. 

- Sono stato molto occupato. Sono molto contento di vedervi - disse con un tono che chiaramente diceva che ne era invece contrariato. - Come va la vostra salute? 

- Ebbene, che ne è della mia cara Anna? 

Aleksej Aleksandrovic mugolò qualcosa e voleva andarsene, ma Stepan Arkad'ic lo trattenne. 

- Ma, ecco cosa facciamo domani. Dolly, invitalo a pranzo! Inviteremo anche Koznyšev e Pescov per offrirgli dell'intelligencija moscovita. 

- Così, vi prego, venite - disse Dolly - noi vi aspettiamo alle cinque, alle sei se volete. E la mia cara Anna? Come da tempo\ldots{} 

- Sta bene - mugolò Aleksej Aleksandrovic accigliandosi. - Molto lieto! - e si diresse verso la carrozza. 

- Verrete? - gridò Dolly. 

Aleksej Aleksandrovic pronunciò qualcosa che Dolly non poté sentire fra il rumore delle vetture che si movevano. 

- Passerò domani - gli gridò Stepan Arkad'ic . 

Aleksej Aleksandrovic sedette nella vettura e vi si sprofondò in modo da non vedere e da non essere visto. 

- Che originale! - disse Stepan Arkad'ic alla moglie e, guardata l'ora, fece un movimento con la mano che voleva essere una carezza sul viso della moglie e dei bambini e s'incamminò spavaldo per il marciapiedi. 

- Stiva! Stiva! - disse Dolly, arrossendo. 

Egli si voltò. 

- Ho bisogno di denaro, sai, per comprare un cappotto a Griša e un altro a Tanja. Dammi dunque il denaro. 

- Non fa nulla; di' che pagherò poi - e scomparve dopo aver salutato allegramente, con un cenno del capo, un conoscente che passava. 

\capitolo{VII}Il giorno dopo era domenica. Stepan Arkad'ic andò al Bol'šoj Teatr per assistere alle prove di un balletto e per consegnare a Maša cibisova, una graziosa ballerina che di recente ne faceva parte perché da lui protetta, i coralli promessile il giorno innanzi, e, nell'oscurità diurna del teatro, dietro una quinta, riuscì a baciarne il visetto simpatico, splendente di gioia per il regalo. Oltre al regalo di coralli doveva prendere accordi con lei per un appuntamento dopo il balletto. Spiegatole perché non poteva trovarsi all'inizio del ballo, promise di venire all'ultimo atto e di condurla a cena. Dal teatro Stepan Arkad'ic andò all'Ochotnyj Rjad, scelse egli stesso il pesce e gli asparagi per il pranzo e alle dodici era già da Dussau per recarsi dai tre personaggi che, per sua fortuna, alloggiavano nello stesso albergo; da Levin che s'era fermato lì ed era tornato da poco dall'estero, dal suo nuovo capo che era allora allora entrato in carica, e che ispezionava Mosca, e dal cognato Karenin per averlo a ogni costo a pranzo. 

A Stepan Arkad'ic piaceva mangiar bene, ma ancor più dare un pranzo, non grandioso, ma raffinato e per cibi e per la scelta dei commensali. La lista del pranzo odierno era proprio di suo gusto: ci sarebbero stati i persici vivi, gli asparagi e, come pièce de résistance, un meraviglioso, ma semplice roastbeef, e i vini adatti: questo per il cibo e le bevande. Come invitati ci sarebbero stati Kitty e Levin e, perché questo non desse nell'occhio, anche una cugina e il giovane Šcerbackij, e, come pièce de résistance, Sergej Koznyšev e Aleksej Aleksandrovic: Sergej Ivanovic moscovita e filosofo, Aleksej Aleksandrovic pietroburghese e uomo d'azione; inoltre avrebbe invitato quell'originale di Pescov, liberale, chiacchierone, musicista, storico e simpaticissimo scapolo cinquantenne che avrebbe costituito la salsa o il contorno a Koznyšev e Karenin. Egli li avrebbe eccitati e aizzati l'un contro l'altro. 

Il denaro del compratore del legname era stato incassato alla seconda scadenza e non era ancora speso; Dolly era molto carina e buona in questi ultimi tempi, e l'idea del pranzo, sotto tutti gli aspetti, rallegrava Stepan Arkad'ic . Egli, si trovava nella più felice disposizione d'animo. Aveva solo due ragioni di malcontento; ma entrambe si sperdevano nel mare di benevola allegria che fluttuava nell'animo suo. Esse erano: la prima, che il giorno avanti, incontrato per via Aleksej Aleksandrovic, aveva notato ch'egli era stato asciutto e brusco con lui e, associando questa espressione del viso di Aleksej Aleksandrovic e il fatto che non era venuto da loro e non aveva fatto sapere nulla di sé con le voci che circolavano sul conto di Anna e Vronskij, Stepan Arkad'ic indovinò che c'era qualcosa che non andava tra marito e moglie. 

Questa era una delle cose spiacevoli. L'altra alquanto spiacevole era che il nuovo capo, come tutti i nuovi capi, aveva fama di uomo terribile, che si alzava la mattina alle sei, lavorava come un bue e pretendeva un egual lavoro dai dipendenti. Inoltre questo nuovo capo aveva anche fama di orso pel suo modo di trattare, ed era, secondo le voci, una persona di tendenze completamente contrarie a quelle del capo precedente e seguite da Stepan Arkad'ic. Il giorno prima egli si era presentato in ufficio in divisa, e il nuovo capo era stato molto amabile e s'era messo a parlare con Oblonskij come con un amico; perciò Stepan Arkad'ic reputava suo dovere fargli visita in finanziera. Il pensiero che il nuovo capo potesse riceverlo male, rappresentava la seconda circostanza spiacevole. Ma Stepan Arkad'ic sentiva istintivamente che tutto si sarebbe ``appianato'' nel modo migliore. ``Tutti sono esseri umani, uomini, come noi, peccatori; perché arrabbiarsi e litigare?'' pensava, entrando nell'albergo. 

- Salve, Vasilij - disse, passando col cappello di sghembo per un corridoio e rivolgendosi a un cameriere di sua conoscenza - ti sei fatto crescere le fedine? Levin è al n. 7, eh? Accompagnami, per favore. E informati se il conte Anickin - era il suo nuovo capo - riceve o no. 

- Sissignore - rispose Vasilij, sorridendo. - È un pezzo che attendiamo i vostri ordini. 

- Sono stato qui, ieri, ma dall'altro ingresso. È il n. 7? 

Levin stava in piedi in mezzo alla stanza con un contadino di Tver' e misurava con un aršin una pelle d'orso fresca, quando entrò Stepan Arkad'ic. 

- Ehi! L'avete ucciso voi? - gridò Stepan Arkad'ic. - Bel giocattolo! Un'orsa? Buongiorno, Archip! 

Strinse la mano al contadino e sedette su di una sedia senza togliersi cappello e cappotto. 

- Ma metti via questa roba, siedi un po' - disse Levin, togliendogli il cappello. 

- Non ho tempo, son venuto solo per un attimo - rispose Stepan Arkad'ic. Sbottonò il cappotto e poi se lo tolse e rimase un'ora intera, discorrendo con Levin di caccia e di cose intime. 

- Dimmi, ti prego, cosa hai fatto all'estero? dove sei stato? - disse Stepan Arkad'ic quando il contadino fu uscito. 

- Sì, sono stato in Germania, in Prussia, in Francia e in Inghilterra; ma non nelle capitali, nelle città industriali, e ho visto molte cose nuove. E sono contento d'esserci stato. 

- Già, io so la tua idea di organizzare il lavoratore. 

- Per nulla affatto: in Russia non può esistere una questione operaia. In Russia c'è la questione dei rapporti del popolo lavoratore con la terra; c'è anche là, ma là si tratta di riparare quel che s'è guastato, invece da noi\ldots{} 

Stepan Arkad'ic ascoltava attentamente Levin. 

- Già, già - diceva. - È molto probabile che tu abbia ragione - disse. - Ma io sono felice che tu sia di buon umore; vai a caccia di orsi e lavori, e sei tutto preso dal lavoro. E invece Šcerbackij mi diceva d'averti incontrato, che eri in non so quale stato d'abbattimento, che parlavi sempre di morte. 

- Già, ma non smetto di pensare alla morte - disse Levin. - È vero che è ora di morire. E che tutto questo è vanità. Io ti dirò il vero: ho straordinariamente cara la mia idea e il lavoro, ma in sostanza, pensaci: tutto questo mondo è in fin dei conti una piccola muffa che è spuntata su di un minuscolo pianeta. E noi pensiamo di essere in possesso di qualcosa di grande\ldots{} pensieri, affari! Granelli di sabbia tutti questi! 

- Ma questo, amico mio, è vecchio come il mondo! 

- È vecchio; ma quando lo capisci chiaramente, allora tutto, in un certo modo si riduce a niente. Quando capisci che oggi o domani morirai, e non resterà più nulla, che tutto sarà annientato! Ecco, io considero molto importante la mia idea, eppure questa anche a pensarla attuata appare così insignificante come fare il giro della pelle di quest'orso. Allora passi la vita distraendoti con la caccia, col lavoro, solo per non pensare alla morte. 

Stepan Arkad'ic sorrideva finemente e benevolmente ascoltando Levin. 

- Su, s'intende! Ecco che tu sei venuto dalla parte mia. Ricordi che m'investivi perché cercavo i piaceri nella vita? 

- Non esser, moralista, così duro! 

- No, tuttavia nella vita vi è tanto di buono che\ldots{} - Levin si confuse. - Ma non so. So soltanto che morirò presto. 

- E perché presto? 

- E sai, c'è meno gioia nella vita quando pensi alla morte, ma c'è più calma. 

- Al contrario, quando si è alla fine si sta più allegri. Su, però per me è ora - disse Stepan Arkad'ic, alzandosi per la decima volta. 

- Ma no, siedi! - diceva Levin, trattenendolo. - Quando ci vedremo ora? Io vado via domani. 

- E io, che bravo! son venuto apposta\ldots{} Vieni assolutamente oggi a pranzo da me. Ci sarà tuo fratello, mio cognato Karenin. 

``Forse lei è qui'' pensò Levin, e voleva chiedere di Kitty. Aveva sentito che al principio dell'inverno era stata a Pietroburgo dalla sorella sposata a un diplomatico e non sapeva se ne era tornata o no; ma rinunciò a chiedere. ``Che ci sia o non ci sia, per me è lo stesso''. 

- Allora, verrai? 

- Eh, s'intende. 

- Alle cinque e in finanziera. 

E Stepan Arkad'ic si alzò e andò giù dal nuovo capo. L'istinto non l'aveva ingannato. Il nuovo terribile capo si mostrò un uomo molto affabile, e Stepan Arkad'ic fece colazione con lui e si trattenne così a lungo che solo verso le quattro si recò da Aleksej Aleksandrovic. 

\capitolo{VIII}Aleksej Aleksandrovic, tornato dalla messa, passò tutta la mattina in casa. In quella mattinata aveva da fare due cose: in primo luogo ricevere e dare le direttive a una deputazione di allogeni che andava a Pietroburgo e che attualmente si trovava a Mosca; in secondo luogo scrivere la lettera promessa all'avvocato. La deputazione fatta venire per iniziativa di Aleksej Aleksandrovic presentava molti svantaggi e persino dei pericoli, e Aleksej Aleksandrovic era molto contento d'averla incontrata a Mosca. Infatti i membri di questa deputazione non avevano la minima idea della parte che rappresentavano e del compito loro affidato. Erano ingenuamente convinti che tutto consistesse nell'esporre la necessità e la vera situazione delle cose, chiedendo l'aiuto del governo, ma non pensavano menomamente che alcune loro dichiarazioni e pretese potessero essere sostenute dal partito opposto ad Aleksej Aleksandrovic e rovinare perciò tutto l'affare. Aleksej Aleksandrovic li catechizzò a lungo, stese per loro un programma dal quale non dovevano deviare e, congedatili, scrisse delle lettere a Pietroburgo circa l'indirizzo che la deputazione avrebbe dovuto seguire. Il più valido aiuto in questa faccenda doveva venire dalla contessa Lidija Ivanovna. Era specialista in fatto di deputazioni, e nessuno come lei sapeva farle valere e dare loro un preciso indirizzo. Sistemate queste faccende, Aleksej Aleksandrovic scrisse all'avvocato. Senza la minima esitazione gli diede l'autorizzazione a procedere come meglio riteneva. Nella lettera incluse i tre biglietti di Vronskij ad Anna, trovati nel portafogli sottratto. 

Dal momento in cui Aleksej Aleksandrovic era andato via di casa, con l'intenzione di non tornare più in famiglia, dal momento in cui era stato dall'avvocato, e aveva detto, sia pure a una persona, la sua intenzione, e proprio dal momento in cui aveva tradotto questa faccenda della sua vita in un affare legale, si era abituato sempre più alla decisione, e adesso vedeva chiara la possibilità di attuarla. Stava sigillando la busta per l'avvocato, quando sentì il suono forte della voce di Stepan Arkad'ic che discuteva col cameriere e insisteva perché lo si annunciasse. 

``È lo stesso - pensò Aleksej Aleksandrovic - meglio: dichiarerò subito la mia posizione nei confronti di sua sorella e spiegherò perché non posso pranzare da lui''. 

- Fa' passare - disse a voce alta, raccogliendo le carte e mettendole nella cartella. 

- Ecco, vedi che menti, è in casa! - rispondeva la voce di Stepan Arkad'ic al cameriere che non lo lasciava passare e, togliendosi il cappotto nel camminare, Oblonskij entrò nella stanza. - Sì, son molto contento d'averti trovato! Così io spero\ldots{} - cominciò allegramente. 

- Non posso venire - disse Aleksej Aleksandrovic freddo, in piedi e senza far sedere l'ospite. 

Aleksej Aleksandrovic pensava di stabilire subito quei rapporti di fredda cortesia, che gli pareva dovessero ora correre tra lui e il fratello della moglie, contro la quale andava intentando un giudizio di divorzio; ma non aveva fatto i conti con quel mare di bonarietà che straripava dall'animo di Stepan Arkad'ic . 

Stepan Arkad'ic aprì i suoi occhi scintillanti, chiari. 

- Perché non puoi? Cosa vuoi dire? - disse con perplessità, in francese. - No, è già promesso. E noi tutti contiamo su di te! 

- Io voglio dire che non posso venire, perché quei rapporti di parentela che esistevano tra noi, devono cessare. 

- Come? cioè, come mai? perché? - pronunciò con un sorriso Stepan Arkad'ic . 

- Perché do inizio a una causa di divorzio contro vostra sorella, contro mia moglie. Ho dovuto\ldots{} 

Ma Aleksej Aleksandrovic non aveva ancora fatto in tempo a finire il suo discorso che Stepan Arkad'ic aveva già agito in modo del tutto diverso da quello ch'egli si aspettava. Stepan Arkad'ic mise un gemito e si sedette in una poltrona. 

- No, Aleksej Aleksandrovic, che dici mai? - gridò Oblonskij e la sofferenza si espresse sul suo viso. 

- È così. 

- Perdonami, io non posso e non voglio crederci. 

Aleksej Aleksandrovic sedette, vedendo che le sue parole non avevano avuto quell'effetto ch'egli si aspettava e che ormai inevitabilmente avrebbe dovuto spiegarsi e che i suoi rapporti col cognato, quali che fossero state le sue spiegazioni, sarebbero rimasti inalterati. 

- Sì, sono stato messo nella penosa necessità di esigere il divorzio - egli disse. 

- Io una cosa sola ti dico, Aleksej Aleksandrovic. Io ti conosco per un uomo eccellente, giusto; conosco Anna, scusami, non posso cambiare l'opinione che ho di lei, di bravissima, ottima donna, e perciò, perdonami, non posso credere a questo. Qui c'è un equivoco - egli disse. 

- Già, se si trattasse solo di un equivoco\ldots{} 

- Permetti, io capisco - interruppe Stepan Arkad'ic . - Ma s'intende\ldots{} Una cosa sola: non bisogna precipitare. Non si deve, non si deve avere fretta. 

- Io non ho avuto fretta - disse freddo Aleksej Aleksandrovic - e in simili casi nessuno può dar consigli. Io ho fermamente deciso. 

- È terribile! - disse Stepan Arkad'ic sospirando penosamente. - Avrei fatto una sola cosa, Aleksej Aleksandrovic. Ti supplico, fa' ciò che ti dico - disse. - La causa non è ancora cominciata, a quanto ho capito. Prima di iniziare il giudizio, vedi mia moglie, parla con lei. Ella vuol bene ad Anna come a una sorella, vuol bene a te, ed è una donna sorprendente. Per amor di Dio, parla con lei. Dammi questa prova di amicizia, te ne supplico! 

Aleksej Aleksandrovic si fece pensieroso e Stepan Arkad'ic lo guardava con simpatia, senza interrompere il suo silenzio. 

- Ci andrai da lei? 

- Ma, non so. È per questo che non son venuto da voi. Suppongo che i nostri rapporti debbano cambiare. 

- E perché mai? Non ne vedo la necessità. Permettimi di tener presente che, oltre ai nostri rapporti di parentela, tu hai avuto per me, almeno in parte, quei sentimenti di amicizia che io ho sempre avuto per te. È stima sincera - disse Stepan Arkad'ic stringendogli la mano. - Se anche le tue peggiori supposizioni fossero vere, io non prendo e non prenderò mai su di me la responsabilità di giudicare in favore dell'una o dell'altra parte, e non vedo la ragione per la quale i nostri rapporti debbano cambiare. Ma ora, fa' una cosa, vieni da mia moglie. 

- Già, noi consideriamo in modo diverso questa faccenda - disse freddamente Aleksej Aleksandrovic. - Del resto, non ne parliamo più. 

- No, perché non venire? Magari oggi a pranzo. Mia moglie ti aspetta. Ti prego, vieni. E soprattutto, parla con lei. È una donna sorprendente. Per amor di Dio, ti supplico in ginocchio. 

- Se lo volete tanto, verrò - disse, sospirando, Aleksej Aleksandrovic. 

E desiderando cambiar discorso, domandò di quello che interessava entrambi: del nuovo capo di Stepan Arkad'ic, uomo non ancora vecchio, che aveva improvvisamente ricevuto una così alta promozione. 

- Ebbene, l'hai visto? - disse Aleksej Aleksandrovic, con un sorriso velenoso. 

- Come no, ieri è stato da noi in tribunale. Sembra che sappia molto bene il fatto suo e sia molto attivo. 

- Sì, ma a che è diretta la sua attività? - disse Aleksej Aleksandrovic. - Ad agire o a ricalcare quello che è stato fatto? La disgrazia del nostro stato è quest'amministrazione burocratica di cui egli è un degno rappresentante. 

- Davvero non so, ma so una cosa sola: è un ottimo giovane - rispose Stepan Arkad'ic . - Sono stato subito da lui, e davvero è un ottimo giovane. Abbiamo fatto colazione insieme, e io gli ho insegnato a fare quella bevanda, sai, il vino con le arance. Rinfresca molto. È strano che non la conoscesse. Gli è piaciuta molto. Sì, davvero è una simpatica persona. 

Stepan Arkad'ic guardò l'orologio. 

- Ah, Dio mio, sono già più delle quattro, e io devo ancora andare da Dolgovušin! Allora, ti prego, vieni a pranzo! Non puoi credere come addoloreresti mia moglie! 

Aleksej Aleksandrovic lo accompagnò in tutt'altro modo da come l'aveva ricevuto. 

- Ho promesso e verrò - rispose con tristezza. 

- Credimi, apprezzo ciò e spero che non te ne pentirai - rispose sorridendo Stepan Arkad'ic. E, infilatosi il cappotto camminando, urtò col braccio la testa del cameriere, rise e uscì. - Alle cinque e in finanziera, per favore! - gridò ancora, tornando verso la porta. 

\capitolo{IX}Erano le cinque passate e già alcuni ospiti cominciavano ad arrivare, quando giunse anche il padrone di casa. Entrò insieme con Sergej Ivanovic Koznyšev e Pescov che si erano trovati nello stesso momento sul pianerottolo. Erano costoro i due principali esponenti dell'intelligencija moscovita così come li aveva definiti Oblonskij. Erano persone stimabili per carattere e per ingegno. Si stimavano reciprocamente, ma erano quasi in tutto completamente e irrimediabilmente discordi fra loro, non perché avessero tendenze opposte, ma proprio perché erano d'uno stesso partito (i nemici li fondevano in uno), e in questo partito ciascuno aveva la propria sfumatura. E poiché è molto difficile metter d'accordo dissensi lievi, indefiniti, essi non solo non concordavano mai nelle opinioni, ma erano da tempo abituati, senza irritarsi, a irridere l'uno l'incorreggibile aberrazione dell'altro. 

Stavano per entrare, conversando del tempo, quando li raggiunse Stepan Arkad'ic. Nel salotto sedevano già il principe Aleksandr Dmitrievic, suocero di Oblonskij, il giovane Šcerbackij, Turovcyn, Kitty e Karenin. 
\enlargethispage*{1\baselineskip}

Stepan Arkad'ic si accorse subito che nel salotto senza di lui, le cose non andavano bene. Dar'ja Aleksandrovna, in abito di gala di seta grigia, evidentemente preoccupata dei bambini che dovevano pranzare in camera loro, e del marito che non arrivava ancora, non aveva saputo fondere bene quella riunione. Stavano tutti seduti come figli di pope in visita (così diceva il vecchio principe), visibilmente perplessi come mai fossero capitati là, spremendo le parole per non star zitti. Il buon Turovcyn, si vedeva, non si sentiva a suo agio, e il sorriso delle grosse labbra col quale accolse Stepan Arkad'ic, diceva chiaramente: ``Ehi, amico, mi hai piantato qui con gli intelligentoni! Ecco, andare a bere sia pure allo Château des fleurs, questo sì che è affar mio!''. Il vecchio principe sedeva in silenzio, guardando di sbieco con i suoi occhi scintillanti Karenin, e Stepan Arkad'ic capì che egli aveva già pensato una qualche battuta di spirito per bollare quest'uomo di stato per il quale si imbandiva un banchetto come se fosse uno storione. Kitty guardava la porta, raccogliendo le proprie forze per non arrossire all'entrata di Konstantin Levin. Il giovane Šcerbackij, che non era stato presentato a Karenin cercava di mostrare che ciò non lo imbarazzava affatto. Karenin, secondo l'uso di Pietroburgo, a quel pranzo con signore, era in frac e cravatta bianca, e Stepan Arkad'ic capì dalla faccia di lui che era venuto soltanto perché aveva promesso, ma che, facendo atto di presenza in quella compagnia, compiva un dovere increscioso. Era proprio lui il principale responsabile di quella freddezza che aveva gelato tutti gli ospiti prima dell'arrivo di Stepan Arkad'ic. 

Entrando in salotto, Stepan Arkad'ic si scusò, spiegò che era stato trattenuto da quel tale principe che era l'eterno capro espiatorio di tutti i suoi ritardi e di tutte le sue assenze e, in un attimo, presentò tutti e, messi insieme Aleksej Aleksandrovic con Sergej Koznyšev, lanciò loro il tema della russificazione della Polonia al quale essi si aggrapparono subito insieme a Pescov. Battendo sulla spalla di Turovcyn gli sussurrò qualcosa di ameno e lo mise a sedere accanto a sua moglie e al principe. Poi disse a Kitty che quel giorno era molto carina, e presentò Šcerbackij a Karenin. In un momento maneggiò tutta quella pasta sociale in modo tale che il salotto cominciò ad andare a gonfie vele, e le voci cominciarono a risonare animate. Mancava soltanto Konstantin Levin. Ma fu per il meglio, perché, andando in sala da pranzo Stepan Arkad'ic si accorse con terrore che il Porto e lo Xeres erano stati presi da Deprès e non da Levais, e, dato ordine di mandare al più presto il cocchiere da Levais, si diresse di nuovo nel salotto. 

In sala da pranzo si incontrò con Konstantin Levin. 

- Non sono in ritardo? 

- Puoi forse arrivare in ritardo tu? - disse Stepan Arkad'ic, prendendolo sotto il braccio. 

- Hai molta gente? E chi mai? - chiese Levin, involontariamente arrossendo mentre faceva cadere col guanto la neve dal berretto. 

- Tutti dei nostri. Kitty è qui. Andiamo dunque: ti farò fare la conoscenza di Karenin. 

Stepan Arkad'ic, malgrado il suo liberalismo sapeva che la conoscenza di Karenin non poteva non essere lusinghiera e perciò la offriva ai suoi amici migliori. Ma in quel momento Konstantin Levin non era in grado di provare tutto il piacere di questa conoscenza. Egli non aveva più visto Kitty da quella serata per lui memorabile in cui aveva incontrato Vronskij, a non voler tener conto dell'attimo in cui l'aveva intraveduta sulla strada maestra. In fondo all'anima sapeva che l'avrebbe vista quel giorno. Ma, per voler tener la mente libera si sforzava di convincersi che non lo sapeva. Adesso invece, nel sentire ch'ella era là, provò, a un tratto, una tale gioia e nello stesso tempo una tale agitazione, che gli si mozzò in gola il respiro e non poté dire quello che voleva. 

``Come, come sarà? Così com'era prima, o com'era nella carrozza? Che succederà se Dar'ja Aleksandrovna ha detto il vero? E perché poi non potrebbe esser vero?'' pensava. 

- Ah, ti prego, fammi fare la conoscenza di Karenin - rispose a stento e, con passo disperatamente deciso, entrò nel salotto e la vide. 

Ella non era più quella di prima, né com'era nella carrozza; era del tutto un'altra. 

Era spaurita, timida, confusa, e, per questo, ancora più incantevole. Lo scorse nell'attimo stesso in cui egli entrò nella stanza. Lo aspettava. Si rallegrò e si confuse della sua gioia a un punto tale che vi fu un momento, proprio quando egli si avvicinò alla padrona di casa e la guardò di nuovo, che a lei e a lui e a Dolly, che vedeva tutto, parve ch'ella non avrebbe resistito e che sarebbe scoppiata a piangere. Arrossì, impallidì, arrossì di nuovo e restò sospesa, con le labbra che le tremavano appena, aspettandolo. Egli le si avvicinò, s'inchinò e tese la mano in silenzio. Se non fosse stato il lieve tremito delle labbra e l'umore che le velava gli occhi e ne aumentava lo splendore, il suo sorriso sarebbe stato quasi calmo, quando disse: 
\enlargethispage*{1\baselineskip}

- Da quanto tempo non ci vediamo! - e con disperata risolutezza strinse con la sua mano fredda la mano di lui. 

- Voi non mi avete visto, ma io sì che vi ho visto - disse Levin, splendido d'un sorriso di gioia. - Vi ho veduto quando dalla ferrovia andavate a Ergušovo. 

- Quando? - ella chiese con sorpresa. 

- Andavate a Ergušovo - diceva Levin, sentendo di soffocar per la gioia che gli invadeva l'anima. ``Come ho osato di associare il pensiero di qualcosa d'impuro con quest'essere commovente? E già, pare che sia vero quello che diceva Dar'ja Aleksandrovna'' pensava. 

Stepan Arkad'ic lo prese per un braccio e lo portò verso Karenin. 

- Permettete che vi presenti. - E disse i loro nomi. 

- Molto piacere di incontrarvi di nuovo - disse freddo Aleksej Aleksandrovic, stringendo la mano di Levin. 

- Vi conoscete? - chiese Stepan Arkad'ic con stupore. 

- Abbiamo passato tre ore insieme in treno - disse Levin, sorridendo - ma ne siamo usciti, come da un ballo in maschera, con la curiosità, io almeno. 

- Ecco, signori, accomodatevi - disse Stepan Arkad'ic, facendo segno verso la sala da pranzo. 

Gli uomini andarono in sala da pranzo e si accostarono alla tavola con gli antipasti, coperta di sei qualità di vodka e di altrettante specie di formaggi con le palettine d'argento o senza, i caviali, le aringhe, le conserve di varie specie e i piatti con le fettine di pane francese. 

Gli uomini stavano in piedi presso le vodke profumate e gli antipasti, e la conversazione sulla russificazione della Polonia fra Sergej Ivanovic Koznyšev, Karenin e Pescov tacque in attesa del pranzo. 

Sergej Ivanovic che, come nessun altro, sapeva chiudere la discussione più astratta e seria spargendo inaspettatamente un po' di sale attico e cambiando così l'umore degli interlocutori, lo fece anche adesso. 

Aleksej Aleksandrovic aveva dimostrato che la russificazione della Polonia poteva compiersi solo in virtù degli alti principi che dovevano essere introdotti dall'amministrazione russa. 

Pescov aveva insistito sul fatto che un popolo assimila un altro popolo solo quando ha una popolazione più densa. Koznyšev aveva annuito all'una ed all'altra idea ma con riserve. E quando uscirono dal salotto, per concludere la conversazione, Koznyšev disse sorridendo: 

- Perciò per la russificazione degli allogeni vi è un mezzo solo: far nascere quanti più bambini è possibile. Ecco, io e mio fratello agiamo nel modo peggiore. Ma voi, signori uomini ammogliati, e in particolare voi Stepan Arkad'ic, agite del tutto patriotticamente. Quanti ne avete? - e si voltò affettuosamente sorridendo al padrone di casa, tendendogli un minuscolo bicchierino. 

Tutti risero, e in modo particolarmente allegro Stepan Arkad'ic. 

- Sì, ecco il mezzo migliore! - egli disse, masticando bene il formaggio e versando una certa vodka di una speciale qualità nel bicchierino teso. La conversazione era realmente finita nello scherzo. 

- Questo formaggio non è cattivo. Ne volete? - diceva il padrone di casa. - Possibile che tu ti sia dato di nuovo alla ginnastica? - disse rivolto a Levin, tastando con la sinistra i suoi muscoli. Levin sorrise, tese il braccio e, sotto le dita di Stepan Arkad'ic, come un rotondo formaggio, si sollevò, al di sotto del panno sottile della finanziera, una massa di acciaio. 
\enlargethispage*{1\baselineskip}

- Che bicipite! Un vero Sansone! 

- Credo che occorra molta forza per la caccia all'orso - disse Aleksej Aleksandrovic che aveva le idee più nebulose sulla caccia, mentre stendeva il formaggio e ne copriva, come una ragnatela, la midolla del pane. 

Levin sorrise. 

- Nessuna. Al contrario, anche un bambino può uccidere un orso - disse, facendosi da parte con un lieve inchino alle signore che, insieme con la padrona di casa, si avvicinavano al tavolo degli antipasti. 

- E voi avete ucciso un orso, mi han detto - disse Kitty, cercando invano di afferrare con la forchetta un fungo bizzarro che scivolava via, e scotendo le trine attraverso le quali biancheggiava il suo braccio. - Ci sono forse degli orsi da voi? - aggiunse, volgendo verso di lui a metà la deliziosa testolina e sorridendo. 

Sembrava che non ci fosse nulla di straordinario in quello ch'ella diceva, ma quale significato impossibile a dirsi c'era per lui in ogni suono, in ogni muover delle labbra, degli occhi, delle mani, mentre ella diceva queste parole! C'era l'implorazione del perdono e la fiducia in lui, e una carezza, una lieve, timida carezza, e una promessa e una speranza, e l'amore per lui al quale egli non poteva non credere e che lo soffocava di gioia. 

- Siamo andati nel governatorato di Tver'. Ritornando di là mi sono incontrato col vostro beau-frère, ossia col cognato del vostro beau-frère - diss'egli con un sorriso. - È stato un incontro buffo. 

E allegramente e con brio raccontò come, non avendo dormito tutta la notte, avesse fatto irruzione nello scompartimento di Aleksej Aleksandrovic con indosso un pellicciotto. 

- Il conduttore, giudicandomi, contrariamente al proverbio, dall'abito, mi voleva cacciar fuori; ma allora io ho cominciato ad esprimermi in uno stile elevato, e\ldots{} voi pure - disse rivolto a Karenin, avendone dimenticato il nome - in principio, a causa di quel pellicciotto, volevate cacciarmi via, ma poi avete preso le mie difese, del che vi sono molto grato! 

- In genere sono assai poco definiti i diritti dei passeggeri sulla scelta del posto - disse Aleksej Aleksandrovic, asciugando col fazzoletto la punta delle dita. 

- Vedevo che eravate dubbioso sul mio conto - disse Levin, sorridendo cordialmente - ma mi sono affrettato a cominciare un discorso intelligente per riparare al mio pellicciotto. 

Sergej Ivanovic, continuando una conversazione con la padrona di casa e ascoltando con un orecchio il fratello, lo guardò di sbieco. ``Che gli succede oggi? Sembra un trionfatore!'' pensò. Non sapeva che Levin sentiva che gli eran spuntate le ali. Levin sapeva che ella ascoltava le sue parole e che le piaceva udirle. E quest'unica cosa l'interessava. Non solo in quella stanza, ma in tutto il mondo, esistevano ormai soltanto lui, che aveva acquistato di fronte a se stesso e a lei un enorme significato e importanza, e lei. Si sentiva trasportato ad un'altezza che gli faceva girar la testa, e là, chi sa dove, ma in basso e lontano, c'erano tutti quei buoni e bravi Karenin, Oblonskij e il mondo intero. 

Senza che nessuno se n'avvedesse, senza guardarli e come se ormai non ci fosse più posto per metterli a sedere, Stepan Arkad'ic fece sedere vicini Levin e Kitty. 

- Via, tu siediti magari qui - disse a Levin. 
\enlargethispage*{1\baselineskip}

Il pranzo fu altrettanto raffinato quanto il vasellame di cui Stepan Arkad'ic era un appassionato. La minestra à la Marie Louise era riuscita ottima; i minuscoli sfogliantini che si scioglievano in bocca erano irreprensibili. Due camerieri e Matvej, in cravatta bianca, facevano l'ufficio loro con le pietanze e coi vini senza farsi notare, accorti e abili. Il pranzo dal lato materiale riuscì benissimo; non meno bene riuscì dal lato spirituale. La conversazione, ora generale, ora a gruppi, non languì mai e, verso la fine del pranzo, si avvivò tanto che gli uomini si alzarono senza cessare di parlare, e persino Aleksej Aleksandrovic si era animato. 

\capitolo{X}A Pescov piaceva toccare il fondo delle questioni e non si acquietò alle parole di Sergej Ivanovic, tanto più che egli sentiva l'infondatezza dell'opinione di lui. 

- Io non ho mai inteso parlare - egli disse, mentre si consumava la minestra, rivolto ad Aleksej Aleksandrovic - della sola densità di popolazione, bensì di questa in concomitanza con la struttura sociale e non con i sistemi. 

- Mi sembra - rispondeva senza fretta e senza voglia Aleksej Aleksandrovic - che sia la stessa cosa. Secondo me, può influire su di un altro popolo solo quello che ha un più alto grado di progresso, che\ldots{} 

- Ma proprio qui sta la questione - interruppe con la sua voce di basso Pescov che si affrettava sempre a interloquire e sembrava metter tutta l'anima in quello che diceva - che significa un più alto grado di progresso? Gli inglesi, i francesi, i tedeschi, chi di questi è al più alto grado di sviluppo? Chi assimilerà l'altro? Noi vediamo che il Reno è francesizzato, eppure i tedeschi non sono da meno dei francesi - egli gridò. - Qui ci deve essere un'altra legge. 

- Mi sembra che il predominio stia sempre dalla parte della vera cultura - disse Aleksej Aleksandrovic, alzando lievemente le sopracciglia. 

- Ma in che cosa consisteranno i segni di una vera cultura? - disse Pescov. 

- Io suppongo che questi segni siano noti - disse Aleksej Aleksandrovic. 

- Pienamente noti? - s'introdusse con un sottile sorriso Sergej Ivanovic. - Ora io ritengo che la cultura vera può essere soltanto quella classica; ma vediamo quanto sono aspre le dispute dall'una e dall'altra parte, e non possiamo negare che anche il campo avversario ha forti argomenti a proprio favore. 

- Voi siete un classico, Sergej Ivanovic. Bevete del rosso? - disse Stepan Arkad'ic. 

- Io non esprimo la mia opinione su questa o quella cultura - disse Sergej Ivanovic, tendendo il proprio bicchiere, con un sorriso di condiscendenza, come verso un bambino - io dico solo che entrambe le parti dispongono di validi argomenti - continuò, rivolgendosi ad Aleksej Aleksandrovic. - Io, per cultura, sono un classico, ma in questa questione non posso avere un'opinione certa. Io non vedo argomenti chiari per cui alla cultura classica si debba dare la preferenza di fronte alle scienze positive. 

- Quelle naturali hanno un'influenza pedagogica-formativa - replicò Pescov. - Prendete l'astronomia, prendete la botanica, la zoologia con il suo sistema di leggi. 

- Non posso concordare in pieno su ciò - rispose Aleksej Aleksandrovic; - mi sembra che non si possa non riconoscere che lo stesso processo delle forme filologiche agisca in modo particolarmente benefico sullo sviluppo spirituale. Oltre a ciò non si può negare che l'influenza degli scrittori classici sia in sommo grado morale, mentre, per disgrazia, all'insegnamento delle scienze naturali sono collegate quelle false, nocive dottrine che formano la piaga del nostro tempo. 

Sergej Ivanovic voleva dire qualcosa, ma Pescov, con la sua grossa voce di basso, lo interruppe. Egli cominciò a dimostrare l'infondatezza di quella opinione. Sergej Ivanovic tranquillamente aspettava il proprio turno e aveva evidentemente pronta una vittoriosa obiezione. 

- Ma - disse Sergej Ivanovic, sorridendo con finezza e rivolto a Karenin - non si può disconoscere che ponderare esattamente tutti i vantaggi e gli svantaggi delle une e delle altre scienze sia cosa difficile, e che la questione su quali debbano preferirsi non andrebbe decisa così alla svelta e in maniera definitiva, se dalla parte della cultura classica non ci fosse quel vantaggio che avete enunciato poc'anzi; l'influenza morale, disons le mot, anti-nichilista. 

- Senza dubbio. 

- Se non ci fosse questo vantaggio dell'influenza anti-nichilista da parte della cultura classica, noi avremmo pensato ancora, avremmo ponderato le argomentazioni di tutte e due le correnti - diceva Sergej Ivanovic con un sorriso sottile - avremmo fatto largo all'una e all'altra tendenza. Ma adesso sappiamo che in queste pillole di cultura classica sta la salutare forza dell'anti-nichilismo, e noi coraggiosamente le propiniamo ai nostri pazienti\ldots{} E che fare quando non ci sarà più neppure questo sale benefico? - concluse, con un pizzico di sale attico. 

Le pillole di Sergej Ivanovic provocarono il riso di tutti, e specialmente di Tuškevic che, ascoltando la conversazione, aveva atteso che arrivasse il momento spiritoso. 

Stepan Arkad'ic non s'era sbagliato a invitare Pescov. Con Pescov una conversazione intelligente non poteva venire meno neppure un istante. Sergej Ivanovic aveva appena chiuso con lo scherzo quella discussione che Pescov subito ne sollevò un'altra. 

- Nemmeno si può dire - disse - che il governo abbia questo scopo. Il governo sembra lasciarsi guidare da considerazioni di ordine generale, rimanendo poi indifferente agli effetti che possono avere le misure adottate. Per esempio, la questione dell'istruzione femminile dovrebbe essere considerata dannosa, ma il governo apre corsi e università femminili. 

E la conversazione subito si diresse verso il nuovo tema dell'istruzione femminile. 

Aleksej Aleksandrovic espresse il pensiero che di solito l'istruzione delle donne si confonde con la questione della libertà delle donne, e soltanto per questo può essere considerata dannosa. 

- Io, al contrario, considero le due questioni indissolubilmente legate - disse Pescov. - È un circolo vizioso. La donna è priva di diritti per insufficienza di istruzione, e l'insufficienza di istruzione deriva dalla mancanza di diritti. Non bisogna dimenticare che l'asservimento delle donne è così grande e inveterato che noi spesso non vogliamo renderci conto dell'abisso che le divide da noi - egli disse. 

- Voi avete detto ``diritti'' - disse Sergej Ivanovic, dopo aver atteso una pausa di Pescov - diritto di occupare le cariche di giurato, di consigliere, di presidente del tribunale; diritto d'impiegato, di membro del parlamento\ldots{} 

- Senza dubbio. 

- Ma se le donne, sia pure come rara eccezione, possono occupare questi posti, mi pare che abbiate usato in modo non proprio l'espressione ``diritti''. Meglio avreste detto: doveri. Ognuno sarà d'accordo che, coprendo una qualche carica di giurato, di consigliere, d'impiegato telegrafico, sentiamo di adempiere un dovere. E perciò è più giusto dire che le donne cercano dei doveri, e del tutto legittimamente. E non si può non simpatizzare verso questo loro desiderio di aiutare il comune lavoro maschile. 

- Perfettamente giusto - affermò Aleksej Aleksandrovic. - La questione, io credo, consiste solo nel vedere se esse sono adatte a compiere quei doveri. 

- Probabilmente ve ne saranno molte adatte - intervenne Stepan Arkad'ic quando l'istruzione sarà diffusa fra di loro. Lo vediamo\ldots{} 

- E il proverbio? - disse il principe che, da tempo, andava prestando orecchio alla conversazione facendo brillare i suoi piccoli occhi canzonatori - anche in presenza delle figliuole si può dire: ``Capelli lunghi\ldots{}''. 

- Pensavamo proprio così dei negri fino alla loro liberazione - disse Pescov urtato. 

- Io trovo strano soltanto questo: che le donne vadano in cerca di nuovi doveri - disse Sergej Ivanovic - quando, per disgrazia nostra, vediamo che gli uomini fanno di tutto per eluderli. 

- I doveri sono congiunti ai diritti; il potere, il denaro, gli onori; è questo che cercano le donne - disse Pescov. 

- Sarebbe lo stesso se io pretendessi il diritto di fare la balia, e mi rammaricassi che le donne fossero pagate e che a me non volessero dar la paga - disse il vecchio principe. 

Tuškevic scoppiò a ridere forte, e a Sergej Ivanovic spiacque di non aver detto lui quella battuta. Perfino Aleksej Aleksandrovic sorrise. 

- Già, ma l'uomo non può allattare - disse Pescov - mentre la donna\ldots{} 

- No, un inglese ha allattato, su di una nave il suo bambino - disse il vecchio principe, permettendosi tanta libertà di linguaggio alla presenza delle figliuole. 

- Quanti inglesi ci sono di questa specie, tante donne funzionario ci saranno - disse Sergej Ivanovic. 

- Sì, ma che deve fare una ragazza che non abbia famiglia? - s'intromise Stepan Arkad'ic, pensando alla cibisova che aveva sempre davanti agli occhi, simpatizzando con Pescov e sostenendolo. 

- Se esaminate bene la storia di una tale ragazza, troverete che essa ha abbandonato la famiglia propria o quella della sorella dove avrebbe potuto avere un lavoro femminile - disse con irritazione Dar'ja Aleksandrovna, entrando improvvisamente nella conversazione e indovinando probabilmente quale ragazza avesse presente Stepan Arkad'ic . 

- Ma noi combattiamo per un principio, per un'idea! - ribatteva Pescov con la sua voce sonora di basso. - La donna può avere il diritto di essere indipendente, colta. Ella è impacciata, oppressa dalla consapevolezza dell'impossibilità di esserlo. 

- E io sono impacciato e oppresso dal fatto che non mi prenderanno come balia nell'ospizio dei trovatelli! - disse di nuovo il vecchio principe con grande gioia di Tuškevic che per ridere lasciò cadere nella salsa, dalla parte grossa, un asparagio. 

\capitolo{XI}Tutti prendevano parte alla conversazione generale, tranne Kitty e Levin. \\
In principio, quando si parlava dell'influenza che un popolo può avere su di un altro, a Levin veniva involontariamente in mente quello che avrebbe potuto dire in proposito, ma questi pensieri, prima molto importanti per lui, non avevano ora il più piccolo interesse. Gli pareva persino strano come mai cercassero di parlare tanto di cose di cui nessuno aveva bisogno. Proprio allo stesso modo, a Kitty sembrava che dovesse essere interessante quello che dicevano a proposito dei diritti e dell'istruzione delle donne. Quante volte ella aveva pensato a questo, a proposito della sua amica Varen'ka, del suo penoso stato di dipendenza, quante volte aveva pensato fra di sé cosa sarebbe avvenuto di lei stessa, se non si fosse maritata, e quante volte aveva discusso di questo con la sorella! Ma ora tutto ciò non la interessava per nulla. Fra lei e Levin si era avviata una conversazione a parte, anzi non una conversazione, ma una certa misteriosa comunione che a ogni minuto li legava più da vicino e suscitava in entrambi un sentimento di gioioso spavento innanzi all'ignoto nel quale entravano. 

Alla domanda di Kitty come avesse potuto scorgerla in carrozza l'anno prima, Levin aveva raccontato che tornava dalla falciatura per la strada maestra e che l'aveva incontrata. 

- Era la prima mattina. Voi probabilmente vi eravate appena svegliata. Vostra maman sonnecchiava in un angolino. Era un mattino meraviglioso. Io cammino e penso: chi sarà mai nella carrozza dal tiro a quattro? Una bella quadriglia coi bubboli, ed ecco per un attimo mi siete balenata voi, e attraverso il finestrino vedo che siete seduta così, e con tutte e due le mani tenete i nastri della cuffietta e pensate intensamente a qualcosa - diceva sorridendo. - Come vorrei sapere a che pensavate allora! Era una cosa importante? 

``Non ero forse spettinata?'' ella pensò; ma visto il sorriso entusiastico che suscitavano nel ricordo di lui questi particolari, sentì che, al contrario, l'impressione da lei prodotta era stata ottima. Arrossì e rise felice. 

- Davvero, non ricordo. 

- Come ride contento Tuškevic! - disse Levin, compiacendosi del fatto che avesse le lacrime agli occhi e il corpo sobbalzante dal ridere. 

- È molto che lo conoscete? - chiese Kitty. 

- Chi non lo conosce! 

- Ma mi pare che lo riteniate un uomo cattivo. 

- Cattivo no, insignificante. 

- E non è vero! E smettete subito di giudicarlo così! - disse Kitty. - Anch'io avevo una cattiva opinione di lui; è invece gentilissimo, straordinariamente buono. Ha un cuore d'oro. 

- Com'è che avete potuto conoscere il suo cuore? 

- Noi siamo stati grandi amici. Lo conosco molto bene. L'inverno scorso, subito dopo\ldots{} che eravate stato da noi - ella disse con un sorriso colpevole e fiducioso insieme - Dolly aveva tutti i bambini con la scarlattina e lui passò da lei per caso. E figuratevi - ella diceva sottovoce - gliene venne tanta pena che si fermò, e cominciò ad aiutarla a curar bambini come una balia. Sai, racconto a Konstantin Dmitric quel che ha fatto Tuškevic durante la scarlattina - disse chinandosi verso la sorella. 

- Sì, meraviglioso, un tesoro! - disse Dolly guardando Tuškevic, il quale, accortosi che si parlava di lui, sorrise affabile. Levin guardò ancora una volta Tuškevic e si meravigliò di non aver capito prima tutto il fascino di quell'uomo. 

- Sono colpevole, colpevole e non penserò mai più male della gente! - disse allegro, esprimendo con sincerità ciò che in quel momento sentiva. 

\capitolo{XII}Nella conversazione avviatasi sui diritti delle donne vi erano argomenti scabrosi a trattarsi in presenza delle signore, sulla ineguaglianza dei diritti nel matrimonio. Pescov, durante il pranzo, era incappato diverse volte in tali argomenti; ma Sergej Ivanovic e Stepan Arkad'ic lo avevano prudentemente distolto. 

Quando poi si furono alzati da tavola e le signore uscirono, Pescov, senza seguirle, si rivolse ad Aleksej Aleksandrovic e si mise ad esporre la ragione principale di tale ineguaglianza. L'ineguaglianza dei coniugi, secondo lui, consisteva nel fatto che l'infedeltà della moglie e l'infedeltà del marito erano punite in modo ineguale e dalla legge e dall'opinione pubblica. 

Stepan Arkad'ic si accostò in fretta ad Aleksej Aleksandrovic, offrendogli da fumare. 

- No, non fumo - rispose con calma Aleksej Aleksandrovic e, quasi volesse dimostrare di proposito che non temeva un tale discorso, si voltò con un sorriso freddo a Pescov. 

- Suppongo che le basi di questo modo di vedere siano nell'essenza delle cose - egli disse cercando di passare in salotto; ma qui a un tratto cominciò inaspettatamente a parlare Tuškevic, rivolto ad Aleksej Aleksandrovic. 

- E avete sentito di Priacnikov? - disse Tuškevic, eccitato dallo champagne bevuto e ansioso di rompere da tempo il silenzio che gli pesava. - Vasja Priacnikov - disse con quel buon sorriso tra le labbra umide e rosse, rivolgendosi di preferenza all'ospite più importante, Aleksej Aleksandrovic - m'hanno raccontato oggi s'è battuto in duello a Tver' con Kvytskij e l'ha ucciso. 

Come sempre pare che la lingua batte dove il dente duole, anche ora Stepan Arkad'ic sentiva che, disgraziatamente, quel giorno il discorso cadeva ogni momento sul punto dolente di Aleksej Aleksandrovic. Voleva di nuovo allontanare il cognato, ma lo stesso Aleksej Aleksandrovic chiese con curiosità: 

- Perché si é battuto Priacnikov? 

- Per sua moglie. È stato bravo. L'ha sfidato e l'ha ucciso. 

- Ah! - disse con indifferenza Aleksej Aleksandrovic e, sollevate le sopracciglia, passò in salotto. 

- Come son contenta che siate venuto! - gli disse Dolly con un sorriso apprensivo, venendogli incontro nel salotto di passaggio. - Devo parlare un po' con voi. Sediamoci qui. 

Sempre con quella espressione di indifferenza che gli davano le sopracciglia sollevate, Aleksej Aleksandrovic sedette accanto a Dar'ja Aleksandrovna, fingendo di sorridere. 

- Tanto più - egli disse - che volevo chiedere il vostro permesso per congedarmi subito. Devo partire domani. 

Dar'ja Aleksandrovna era pienamente convinta dell'innocenza di Anna e sentiva che impallidiva e che le labbra le tremavano di rabbia verso quell'uomo freddo e insensibile che così pacatamente si accingeva a distruggere la sua innocente amica. 

- Aleksej Aleksandrovic - ella disse con una decisione disperata, guardandolo negli occhi. - Vi ho chiesto di Anna, voi non mi avete risposto. Cosa ne è di lei? 

- Sta bene, mi pare, Dar'ja Aleksandrovna - rispose Aleksej Aleksandrovic senza guardarla. 

- Aleksej Aleksandrovic, perdonatemi\ldots{} io non ho il diritto\ldots{} ma amo e stimo Anna come una sorella, vi chiedo, vi supplico di dirmi: che c'è fra voi? di che cosa l'accusate? 

Aleksej Aleksandrovic si accigliò e, quasi chiudendo gli occhi, abbassò la testa. 

- Suppongo che vostro marito vi abbia riferito le ragioni per le quali considero necessario cambiare i miei precedenti rapporti con Anna Arkad'evna - egli disse, senza guardarla negli occhi, e guardando contrariato Šcerbackij che passava per il salotto. 

- Io non credo, non credo, non posso credere a questo! - esclamò Dolly con un gesto energico, stringendo davanti a sé le mani ossute. Si alzò in fretta e poggiò la mano sulla manica di Aleksej Aleksandrovic. - Qui ci disturbano, passiamo di qua, vi prego. 

L'agitazione di Dolly agiva su Aleksej Aleksandrovic. Egli si alzò e la seguì umilmente nella stanza di studio dei ragazzi. Sedettero a una tavola ricoperta di tela cerata tagliuzzata dai temperini. 

- Io non credo, non credo a questo! - esclamò Dolly cercando di afferrare lo sguardo di lui che la sfuggiva. 

- Non si può non credere ai fatti, Dar'ja Aleksandrovna - egli disse, accentuando la parola ``fatti''. 

- Ma che ha fatto mai? - disse Dar'ja Aleksandrovna. - Che cosa ha fatto precisamente? 

- Ha violato i suoi doveri e ha tradito suo marito. Ecco quello che ha fatto - egli disse. 

- No, no, non può essere! No, per amor di Dio, voi vi sbagliate! - diceva Dolly, toccandosi le tempie con le mani e chiudendo gli occhi. 

Aleksej Aleksandrovic sorrise freddamente, desiderando mostrare a lei e a se stesso la fermezza della propria convinzione; ma quella calorosa difesa, pur senza farlo tentennare, gli inaspriva la ferita. Egli prese a parlare con grande animazione. 

- È assai difficile sbagliarsi quando la moglie stessa confessa tutto al marito e gli dice che otto anni di vita in comune e un figlio non contano, che tutto questo è uno sbaglio e che vuol vivere daccapo - disse lui furioso, aspirando forte col naso. 

- Anna e la colpa\ldots{} non riesco ad associarli, non ci posso credere. 

- Dar'ja Aleksandrovna!- egli disse guardando ora diritto nel buon viso agitato di Dolly e sentendo che la lingua involontariamente gli si scioglieva. - Io darei molto perché un dubbio fosse ancora possibile. Quando dubitavo, c'era ancora la speranza; ma ora non c'è speranza, e tuttavia io dubito di tutto. Dubito a tal punto di tutto, che odio mio figlio e a volte non credo che sia figlio mio. Sono molto infelice. 

Non gli era necessario dir questo: Dar'ja Aleksandrovna lo aveva capito subito, appena egli l'aveva guardata in viso; ed ebbe pietà di lui e la fede nell'innocenza della sua amica vacillò. 

- Ah, è terribile, terribile! Ma è possibile che sia vero, che vi siate deciso a divorziare? 

- Son ricorso all'ultima misura. Non ho più nulla da fare. 

- Niente da fare, niente da fare\ldots{} - ripeteva lei con le lacrime agli occhi. - No, non niente da fare! - disse. 

- Appunto questo è orribile in questa specie di dolore, che non si può, come in un qualsiasi altro, in una perdita, in una morte, portare la croce; qui bisogna agire - egli disse quasi indovinando il pensiero di lei. - Bisogna uscire dalla situazione umiliante in cui siete stato posto: non si può vivere in tre. 

- Capisco, capisco bene - disse Dolly, e abbassò il capo. Taceva pensando a se stessa e al proprio dolore familiare e improvvisamente, con un gesto energico, alzò il capo e con fare supplichevole giunse le mani: - Ma, aspettate! Voi siete cristiano. Pensate a lei! Che ne sarà di lei, se l'abbandonate? 

- Ci ho pensato, Dar'ja Aleksandrovna, e ci ho pensato molto - disse Aleksej Aleksandrovic. Il suo viso era arrossato a chiazze e gli occhi appannati guardavano dritto in lei. Dar'ja Aleksandrovna ora lo compativa con tutta l'anima. - Ho fatto questo, dopo che mi fu annunciato da lei stessa il mio disonore: ho lasciato tutto come prima. Ho dato la possibilità di una resipiscenza, ho cercato di salvarla. Ebbene? Lei non ha adempiuto la mia più piccola pretesa, il rispetto delle convenienze - egli disse, accalorandosi. - Si può salvare un essere che non si vuole perdere; ma se una natura è tutta così corrotta, pervertita, che la stessa rovina le sembra una salvezza, che fare mai? 

- Tutto, tranne il divorzio! - rispose Dar'ja Aleksandrovna. 

- Ma cosa tutto? 

- No, è orribile. Ella non sarà la moglie di nessuno, si perderà. 

- Che posso farci? - disse Aleksej Aleksandrovic, dopo aver alzato le spalle e le sopracciglia. Il ricordo dell'ultima azione della moglie lo irritò talmente che tornò freddo come al principio della conversazione. - Vi ringrazio molto della vostra simpatia, ma devo andare - disse, alzandosi. 

- No, aspettate. Voi non dovete rovinarla. Aspettate, vi dirò di me. Mi sono sposata e mio marito mi ha ingannato, nell'ira della gelosia volevo abbandonare tutto, volevo io stessa\ldots{} Ma poi sono tornata in me. Chi mai? Anna mi ha salvato. Ed ecco, io vivo. I bambini crescono, mio marito ritorna in famiglia e sente il torto suo e diventa sempre migliore, e io vivo\ldots{} Io ho perdonato, anche voi dovete perdonare! 

Aleksej Aleksandrovic ascoltava, ma le parole di lei non agivano più. Nella sua anima erompeva di nuovo tutto il rancore di quel giorno in cui aveva deciso il divorzio. Si scosse e cominciò a parlare con voce penetrante, forte: 

- Perdonare non posso e non voglio, e lo considero ingiusto. Per questa donna ho fatto tutto, e lei ha calpestato tutto nel fango che le è proprio. Io non sono un uomo cattivo, io non ho mai odiato nessuno, ma odio lei con tutte le forze dell'anima mia e non posso perdonare perché troppo la odio per tutto il male che mi ha fatto! - disse con lacrime di rancore nella voce. 

- Amate chi vi odia - mormorò timidamente Dar'ja Aleksandrovna. 

Aleksej Aleksandrovic rise sprezzante. Questo lo sapeva da tempo, ma questo non poteva essere applicato al suo caso. 

- Amate chi vi odia, ma amare chi si odia non è possibile. Perdonatemi d'avervi sconvolta. A ciascuno il suo dolore! - E, ritornato padrone di sé, Aleksej Aleksandrovic salutò tranquillo e andò via. 

\capitolo{XIII}Quando si erano alzati da tavola, Levin avrebbe voluto seguire Kitty; ma temeva di spiacerle con una corte troppo evidente. Rimase nella cerchia degli uomini, prendendo parte alla conversazione generale e, senza guardare Kitty, ne percepiva i movimenti, gli sguardi e il posto dove era in salotto. 

Subito, e senza il più piccolo sforzo, eseguiva la promessa che le aveva fatto, di pensar sempre bene di tutti e di voler bene a tutti. La conversazione volgeva sulla ``comunità'' nella quale Pescov vedeva un certo principio particolare, da lui detto principio corale. Levin non era d'accordo né con Pescov né col fratellastro che, in un certo modo, a modo uso, riconosceva e non riconosceva il valore della comunità russa. Ma parlava con loro cercando solo di metterli d'accordo e di smussare le loro obiezioni. Non si interessava affatto a quello che egli stesso diceva, ancora meno a quello che dicevano loro, e desiderava una cosa sola, che tutti si trovassero bene e a loro agio. Adesso per lui una cosa sola era importante. E questa cosa sola dapprima era là nel salotto, ma poi si era mossa e si era fermata presso la porta. Senza voltarsi, egli ne sentiva lo sguardo fisso su di lui e il sorriso, e non poté non voltarsi. Ella stava dritta sulla porta con Šcerbackij e lo guardava. 

- Pensavo che andaste al piano - disse, avvicinandosi a lei. - Ecco quello che mi manca in campagna: la musica. 

- Noi siamo venuti solo per tirarvi fuori di qua - disse lei, ricompensandolo di un sorriso come di un dono - e vi ringrazio che siate venuto via. Che gusto c'è a discutere? Tanto nessuno mai convincerà l'altro. 

- Già, è vero - disse Levin - succede il più delle volte che si discute calorosamente perché non si riesce in alcun modo a capire che cosa voglia precisamente dimostrare l'avversario. 

Levin spesso aveva notato che, anche nelle discussioni fra le persone più intelligenti, dopo sforzi enormi, dopo un'enorme quantità di sottigliezze e di parole, coloro che discutono giungono alla fine a riconoscere che quello per cui a lungo si sono battuti per dimostrare l'uno all'altro, era loro noto da gran tempo, fin dall'inizio della conversazione, ma che piacevano loro cose diverse e non volevano nominare quello che piaceva per non essere contraddetti. Spesso aveva sperimentato che durante una discussione si capisce quello che piace al contraddittore, e improvvisamente quella stessa cosa comincia a piacere e allora tutti gli argomenti cadono come cosa inutile; talvolta aveva sperimentato il contrario: si espone finalmente quello che piace per cui si sono escogitati tutti gli argomenti, e se accade che si espongano bene e con sincerità, a un tratto l'avversario è d'accordo e smette di discutere. Proprio questo egli voleva dire. 

Ella corrugò la fronte, cercando di capire. Ma Levin aveva appena cominciato a spiegare che ella aveva già capito. 

- Capisco: bisogna sapere per che cosa egli discute, cosa gli piace, allora si può\ldots{} 

Ella aveva indovinato in pieno e aveva esposto il pensiero espresso male da lui. Levin sorrise con gioia; tanto era stato rapido il passaggio dall'oscura verbosa discussione di Pescov con suo fratello a questa laconica e chiara, quasi tacita comunicazione dei pensieri più complicati. 

Šcerbackij si era allontanato, e Kitty, avvicinandosi a un tavolo da giuoco che era lì aperto, sedette e, preso in mano un pezzo di gesso, cominciò a disegnare sul panno verde fiammante. 

Ripresero la conversazione che s'era tenuta a tavola, sulla libertà e sulle occupazioni delle donne. Levin era d'accordo con Dar'ja Aleksandrovna che una ragazza non sposata può trovare un lavoro femminile nella propria famiglia. Egli lo confermava col fatto che nessuna famiglia può fare a meno di un aiuto, che in ogni famiglia, ricca o povera, ci sono e ci devono essere della bambinaie salariate o delle parenti. 

- No - disse Kitty arrossendo, ma guardando tanto più coraggiosamente con i suoi occhi sinceri - una ragazza può essere posta in una condizione tale che non può, senza sentirsene umiliata, occuparsi in famiglia, e lei stessa\ldots{} 

Egli capì l'allusione. 

- Oh, sì - disse - sì, sì, avete ragione! 

E comprese il valore di tutto quello che aveva dimostrato a pranzo Pescov sulla libertà delle donne, per il solo fatto che vedeva nel cuore di Kitty il terrore dello stato nubile e della umiliazione e, poiché l'amava, aveva sentito quel terrore e quell'umiliazione in se stesso, e aveva rinunciato, d'un tratto, alle proprie argomentazioni. 

Sopravvenne un silenzio. Ella disegnava col gesso sul tappeto del tavolo. I suoi occhi brillavano d'una luce calma. Immedesimandosi nello stato d'animo di lei, egli sentiva in tutto il suo essere una tensione di felicità che diventava sempre più intensa. 

- Ah, ho disegnato tutto il tavolo! - disse lei e, deposto il gesso, fece per alzarsi. 

``E come farò a rimanere solo senza di lei?'' pensò Levin con terrore e prese il gesso. 

- Aspettate - disse, sedendo al tavolo. - Da tempo volevo chiedervi una cosa. 

Egli la guardava dritto negli occhi carezzevoli, sebbene spaventati. 

- Domandate, vi prego. 

- Ecco - egli disse, e scrisse le lettere iniziali: q, m, a, r: q, n, p, e, s, m, o, a? Quelle lettere volevano significare: ``quando mi avete risposto: questo non può essere, significava mai o allora?''. Non c'era nessuna probabilità che ella potesse decifrare questa frase complicata; ma egli la guardò con tanta ansia come se la sua vita dipendesse dall'aver ella capito o no quelle parole. 

Kitty lo guardò seria, poi poggiò la fronte corrugata sulla mano e cominciò a leggere. Di tanto in tanto dava un'occhiata a lui, domandandogli con lo sguardo: ``È quello che penso?''. 

- Ho capito - disse, arrossendo. 

- Che parola è questa? - egli disse, indicando l'm con cui era significata la parola ``mai''. 

- Questa parola significa ``mai'' - ella disse - ma non è vero! 

Egli cancellò in fretta quel che era scritto, le dette il gesso e si alzò. Ella scrisse: a, i, n, p, r, d. 

Dolly si consolò completamente del dolore arrecatogli dalla conversazione con Aleksej Aleksandrovic, quando sorprese queste due figure: Kitty col gesso in mano e con un sorriso timido che guardava di sotto in su Levin, e la bella figura di lui curva sul tavolo, con gli occhi ardenti, fissi ora sul tavolo ora su di lei. A un tratto egli s'illuminò tutto: aveva capito. La scritta significava: ``allora io non potevo rispondere diversamente''. 

Egli la guardò interrogativamente con timore. 

- Soltanto allora? 

- Sì - rispose il sorriso di lei. 

- E o\ldots{} ora? - egli domandò. 

- Ebbene, ecco leggete. Dirò quello che desidererei. Quello che desidererei tanto. - Ella scrisse le iniziali: ``c, p, d, e, p, q, c, e, s''. Questo significava: ``Che possiate dimenticare e perdonare quello che è stato''. 

Egli afferrò il gesso con le dita tese, tremanti, e, spezzatolo, scrisse le iniziali di quel che segue: ``Non ho nulla da dimenticare e perdonare, non ho mai cessato di amarvi''. 

Ella lo guardò con un sorriso che s'era fermato sul suo volto. 

- Ho capito - disse piano. 

Egli sedette e scrisse una lunga frase. Ella capì tutto e senza chiedere: ``È così?'' prese il gesso e rispose immediatamente. 

Egli per parecchio tempo non riuscì a capire quello ch'ella aveva scritto e la guardava spesso negli occhi. La mente gli si annebbiò di gioia. Non riusciva in nessun modo a sostituire alle lettere le parole ch'ella intendeva; ma negli occhi di lei, raggianti di felicità, capì tutto quello che doveva sapere. E scrisse tre lettere sole. Ma non aveva ancora finito di scriverle e lei già leggeva dietro il suo braccio e terminava lei stessa e scriveva la risposta: ``Sì''. 

- Giocate al secrétaire? - disse il vecchio principe accostandosi. - Su, però, andiamo, se vuoi arrivare in tempo a teatro. 

Levin si alzò e accompagnò Kitty alla porta. 

Nella loro conversazione era stato detto tutto; ch'ella lo amava e che avrebbe detto al padre e alla madre ch'egli sarebbe andato da loro l'indomani mattina. 

\capitolo{XIV}Quando Kitty se ne fu andata e Levin rimase solo, egli sentì una tale inquietudine e un tale impaziente desiderio di giungere presto all'indomani mattina, al momento in cui l'avrebbe vista di nuovo e si sarebbe unito a lei per sempre, che ebbe paura, come della morte, di quelle quattordici ore da passare ancora lontano da lei. Gli era indispensabile rimanere a parlare con qualcuno, per non restare solo, per ingannare il tempo. Stepan Arkad'ic sarebbe stato per lui l'interlocutore preferito, ma se ne andava via, come diceva, a una serata, e in realtà, al balletto. Levin fece solo in tempo a dirgli che era felice, che gli voleva bene e che mai, mai avrebbe dimenticato quello ch'egli aveva fatto per lui. Lo sguardo e il sorriso di Stepan Arkad'ic dimostrarono a Levin ch'egli capiva in pieno questo sentimento. 

- Be', non è più ora di morire? - disse Stepan Arkad'ic, stringendo la mano di Levin con emozione. 

- N-n-no - disse Levin. 

Dar'ja Aleksandrovna, salutandolo, si congratulò in un certo modo, dicendo: 

- Sono proprio contenta che vi siate di nuovo incontrato con Kitty; si debbono tener care le vecchie amicizie. 

Ma a Levin non erano piaciute le parole di Dar'ja Aleksandrovna. Ella non poteva capire come tutto questo fosse elevato e inaccessibile a lei, e non doveva osare accennarvi. Levin si congedò da loro, ma, per non restar solo, si attaccò a suo fratello. 

- Dove vai? 

- A una seduta. 

- Su, vengo con te. Posso? 

- Perché no, andiamo - disse, sorridendo, Sergej Ivanovic. - Ma che hai oggi? 

- Che ho? La felicità! - disse Levin, abbassando il finestrino della carrozza in cui avevano preso posto. - Non ti fa niente? Altrimenti si soffoca. Ho la felicità. Perché non ti sei sposato? 

Sergej Ivanovic sorrise. 

- Sono molto contento, pare che sia una brava rag\ldots{} -cominciò Sergej Ivanovic. 

- Non parlare, non parlare, non parlare! - gridò Levin, afferrandolo con tutte e due le mani per il bavero della pelliccia e avviluppandovelo. ``È una brava ragazza'' erano parole così semplici, così comuni, così inadatte al suo sentimento! 

Sergej Ivanovic rise di un riso allegro come di rado gli accadeva. 

- Be', però, si può dire che ne sono molto contento? 

- Questo si potrà domani, domani e basta! Niente, niente, silenzio! - disse Levin, e, avvoltolo ancora una volta nella pelliccia, aggiunse: - Ti voglio tanto bene. Dunque, posso rimanere alla seduta? 

- Ma s'intende, si può. 

- Di che si parla oggi, da voi? - chiese Levin, senza cessare di sorridere. 

Giunsero alla seduta. Levin ascoltava il segretario, che, balbettando, leggeva il verbale, evidentemente a lui stesso incomprensibile; ma Levin scorgeva dalla faccia di questo segretario che era un simpatico, buono e brav'uomo; lo vedeva da come si confondeva e si turbava quando leggeva il verbale. Dopo cominciarono i discorsi. Si discuteva di dover defalcare alcune somme e situare alcuni tubi, e Sergej Ivanovic disse qualcosa di spiacevole a due membri del consiglio e parlò trionfante a lungo di qualcosa; ma l'altro membro, segnato qualcosa su di un foglietto di carta, in principio fu timido, ma poi gli rispose molto velenosamente e con garbo. E poi anche Svijazskij (c'era anche lui) disse qualcosa molto bene e con nobiltà. Levin li ascoltava e s'accorgeva che né quelle somme defalcate, né quei tubi esistevano, che non c'era nulla di tutto questo, e che essi non si arrabbiavano affatto e che erano buone e brave persone e che tutto fra di loro andava bene, in modo tanto cordiale. Non davano noia a nessuno e tutti si trovavano bene. Notava Levin che quel giorno tutti erano per lui trasparenti, e che da piccoli segni prima inosservati, egli riusciva a conoscere l'animo di ciascuno e vedeva chiaramente che tutti erano brave persone. Di più, quel giorno, tutti gli volevano straordinariamente bene. Lo si vedeva dal modo col quale gli rivolgevano la parola, e nel modo affabile e affettuoso col quale lo guardavano anche quelli che non lo conoscevano. 

- Su, allora, sei contento? - gli chiedeva Sergej Ivanovic. 

- Molto. Non avrei mai pensato che la cosa fosse così interessante. Bello, bellissimo. 

Svijazskij si avvicinò a Levin e lo invitò a prendere il tè a casa sua. Levin non poteva in nessun modo rendersi conto e ricordare perché mai Svijazskij gli fosse stato antipatico, e che cosa avrebbe voluto trovare in lui. Era un uomo intelligente e buono in modo straordinario. 

- Molto contento - disse e chiese della moglie e della cognata. E per una strana connessione di idee (nella sua immaginazione il pensiero della cognata di Svijazskij si collegava al matrimonio) gli parve che a nessuno meglio che alla moglie e alla cognata di Svijazskij si potesse parlare della propria felicità, e fu molto contento di andare da loro. 

Svijazskij lo interrogò sul suo lavoro in campagna, sostenendo, come al solito, che non c'era nessuna possibilità di creare qualcosa che non fosse stata già trovata in Europa; ma adesso questo non urtava più Levin. Al contrario, sentiva che Svijazskij aveva ragione, che tutto il lavoro che si faceva era insignificante e scorgeva la sorprendente mitezza e delicatezza con cui Svijazskij evitava di far valere le proprie ragioni. Le signore Svijazskij erano particolarmente gentili. A Levin pareva che esse sapessero già tutto e simpatizzassero con lui, ma che non parlassero della cosa solo per delicatezza. Rimase da loro un'ora, due, tre, parlando di argomenti vari, e celava solo quello che gli riempiva l'anima, e non si accorgeva che ormai era diventato oltremodo noioso, e che da tempo le signore avrebbero preferito andare a dormire. Svijazskij lo accompagnò in anticamera, sbadigliando e meravigliandosi dello strano stato in cui era l'amico. Era l'una passata. Levin tornò in albergo, e il pensiero di restare solo con la propria impazienza a passare le dieci ore che ancora gli restavano, lo spaventò. Il cameriere di turno gli accese le candele e voleva andar via, ma Levin lo trattenne. Questo cameriere, Egor, che prima Levin non aveva mai notato, gli si rivelò, d'un tratto, come un uomo molto intelligente e bravo, e soprattutto buono. 

- Be', Egor, è difficile riuscire a non dormire? 

- Che fare? È il nostro mestiere che è fatto così. Nelle case dei signori si sta tranquilli, ma, in compenso, qui c'è più guadagno. 

Venne in chiaro che Egor aveva famiglia, tre maschi e una figlia sarta che egli voleva maritare al commesso di una valigeria. 

Levin, a questo proposito, comunicò a Egor la sua idea che nel matrimonio la cosa principale è l'amore, e che con l'amore si sarebbe stati sempre felici, perché la felicità è solo dentro di noi. 

Egor ascoltava con attenzione ed evidentemente aveva capito in pieno il pensiero di Levin, ma per confermarlo uscì nell'osservazione, inattesa per Levin, che, quando viveva dai suoi buoni signori, era stato sempre contento di loro, e ora era pienamente contento del suo padrone sebbene fosse un francese. 

``È un uomo straordinariamente buono'' pensava Levin. 

- Be', e tu, Egor, quando hai preso moglie, l'amavi tua moglie? 

- E come se l'amavo! - rispose Egor. 

E Levin credette che anche Egor si trovasse in uno stato di euforia e avesse l'intenzione di manifestare tutti i suoi sentimenti più intimi. 

- Anche la mia vita è straordinaria. Io sin da piccolo\ldots{} - cominciò con gli occhi che gli luccicavano, evidentemente contagiato dall'entusiasmo di Levin, così come si è contagiati dallo sbadiglio. 

Ma in quel momento si udì una scampanellata; Egor andò via e Levin rimase solo. Non aveva mangiato a pranzo, aveva rifiutato il tè e la cena da Svijazskij, ma non poteva neppure pensare a dormire. Nella camera c'era fresco, eppure il caldo lo soffocava. Aprì ambedue i finestrini e sedette sulla tavola proprio di fronte ad essi. Di là, oltre un tetto coperto di neve, si vedevano una croce lavorata con delle catene e, sopra a questa, il triangolo della costellazione sorgente dell'Auriga col giallo chiaro della Capra. Egli guardava ora la croce, ora la costellazione, aspirando l'aria gelata che entrava con uniformità in camera, e seguiva, come in sogno, le immagini e i ricordi che gli sorgevano nella mente. Dopo le tre, notò dei passi nel corridoio e guardò dalla porta. Era Mjaskin, il giocatore a lui noto che rientrava dal club. Camminava con aria cupa, aggrottando le sopracciglia e spurgando. ``Povero disgraziato!'' pensò Levin, e lacrime di amore e di compassione per quell'uomo gli vennero agli occhi. Voleva parlare con lui, consolarlo; ma, accortosi di aver indosso la sola camicia, cambiò idea, e sedette di fronte al finestrino per fare un bagno nell'aria fredda e guardare quella croce silenziosa, ma per lui piena di significato, dalla forma sorprendente, e la stella giallo-chiara che si levava. Dopo le sei cominciarono a far rumore i lucidatori dei pavimenti, qualcuno sonò per qualche servizio, e Levin sentì che cominciava a gelare. Chiuse il finestrino, si lavò, si vestì e uscì in strada. 

\capitolo{XV}Le strade erano ancora deserte. Levin si diresse verso casa Šcerbackij. L'ingresso principale era chiuso e tutto dormiva. Tornò indietro, rientrò in casa e ordinò un caffè. Il cameriere di turno di giorno, già non più Egor, glielo portò. Levin voleva attaccar discorso con lui, ma questi fu chiamato da una scampanellata e andò via. Levin provò a bere un po' di caffè e mise in bocca una ciambellina, ma la bocca non sapeva proprio che farsene delle ciambelle. Sputò la ciambella, infilò il cappotto e uscì di nuovo. Erano le nove passate, quando, per la seconda volta, si accostò alla scala degli Šcerbackij. In casa s'erano appena alzati, e il cuoco andava a far la spesa. Dovevano passare almeno altre due ore. 

Tutta quella notte e la mattina seguente Levin aveva vissuto inconsciamente e si era sentito del tutto fuori della vita materiale. Non aveva mangiato durante il giorno, non aveva dormito per due notti, aveva passato alcune ore, svestito, al gelo, e si sentiva non solo fresco e sano come non mai, ma come staccato completamente dal corpo; si moveva senza alcuno sforzo di muscoli e sentiva di poter fare qualsiasi cosa. Era sicuro che, se fosse stato necessario, sarebbe volato in cielo o avrebbe smosso l'angolo di una casa. Passò il resto del tempo in istrada, guardando continuamente l'ora e voltandosi di qua e di là. 

E ciò che vide in quel momento non lo vide mai più. Bambini che andavano a scuola, colombi grigio-azzurri che volavano dal tetto sul marciapiede e ciambelle cosparse di farina, esposte da mani invisibili, lo commossero in modo particolare. Quelle ciambelle, quei colombi, i due bambini non erano di questa terra. In un solo attimo, uno dei bambini corse verso il colombo e guardò Levin sorridendo; il piccione starnazzò e volò via luccicando al sole fra i granelli di neve che tremavano nell'aria, e da una vetrina esalò odor di pane cotto al forno e le ciambelle furono esposte. Tutto questo insieme era straordinariamente bello, e Levin rideva e piangeva di gioia. Fatto un giro per il vicolo Gazetnyj e la Kislovka, tornò di nuovo in albergo, e dopo aver posto dinanzi a sé l'orologio, sedette, aspettando le dodici. Nella camera accanto parlavano di macchine e di una certa truffa e tossivano della tosse di prima mattina. Costoro non capivano che la lancetta dell'orologio si avvicinava alle dodici. La lancetta si avvicinò. Levin uscì sulla scala. I vetturini, evidentemente, già sapevano tutto. Con le facce gioconde circondarono Levin, discutendo fra di loro e offrendo i loro servigi. Senza offendere gli altri e promettendo di andare in seguito anche con loro, Levin ne scelse uno e ordinò di andare dagli Šcerbackij. Il cocchiere era proprio bello con il colletto bianco della camicia tirato fuori dal gabbano e teso sul collo pieno, rosso e forte. Questo cocchiere aveva una slitta alta e comoda, quale Levin non ne trovò mai più. Il cavallo poi era un buon cavallo e cercava di correre, ma non riusciva a muoversi dal posto. Il cocchiere conosceva gli Šcerbackij e, dopo aver fatto cerchio delle braccia e detto ``ih!'', in modo particolarmente deferente verso il cliente, si fermò all'ingresso del palazzo. Il portiere degli Šcerbackij sapeva tutto certamente. Si vedeva dal sorriso degli occhi e da come disse: 

- Ehi, da un pezzo non vi si vede Konstantin Dmitric ! 

Non solo sapeva tutto ma evidentemente giubilava di gioia e si sforzava di nasconderla. Guardando i suoi cari occhi di vecchio, Levin intravide perfino qualcosa di nuovo nella propria felicità. 

- Sono alzati? 

- Favorite! E quello lasciatelo qua - disse, sorridendo quando Levin tornò indietro a prendere il cappello. Questo voleva significare qualcosa. 

- A chi devo annunciare? - chiese il cameriere. 

Il cameriere, sebbene giovane e di quelli nuovi, un bell'imbusto, era tuttavia buono e per bene e anche lui capiva tutto. 

- Alla principessa\ldots{} al principe\ldots{} alla principessina - disse Levin. 

Il primo personaggio che vide fu m.lle Linon. Attraversava la sala e i suoi riccioli e il suo viso splendevano. Egli aveva appena cominciato a parlare con lei che improvvisamente, di là dalla porta, si udì un fruscio di vesti: m.lle Linon scomparve agli occhi di Levin, e gli si comunicò un gioioso terrore della propria imminente felicità. M.lle Linon si affrettò a lasciarlo e andò verso un'altra porta. Appena fu uscita, dei passi leggeri, svelti svelti risonarono sul pavimento di legno e la sua felicità, la sua vita, egli stesso, quello che aveva cercato e desiderato tanto a lungo, si avvicinò veloce a lui. Ella non aveva camminato, ma era stata portata verso di lui come da una forza invisibile. 

Egli vedeva soltanto i chiari occhi di lei, sinceri, spaventati dalla stessa gioia di amore che riempiva il cuore di lui. Brillavano questi occhi sempre più a misura che si avvicinavano e lo accecavano con la loro luce d'amore. Ella si fermò dinanzi a lui, fino a toccarlo. Le mani si sollevarono e gli si abbandonarono sulle spalle. 

Ella aveva fatto tutto quello che poteva, era corsa a lui e gli si era data tutta, con timore e con gioia. Egli l'abbracciò, e premette le labbra su quella bocca che cercava il suo bacio. 

Anche lei non aveva dormito per tutta la notte e per tutta la mattina l'aveva atteso. La madre e il padre acconsentivano in pieno ed erano felici della sua felicità. Ella lo aspettava. Voleva dirgli la sua felicità e quella di lui. Si era preparata ad andargli incontro da sola, e si era rallegrata a questo pensiero, ma era timida e vergognosa ed ella stessa non sapeva quello che avrebbe fatto. Aveva udito i passi e la voce di lui e aveva atteso di là dalla porta che m.lle Linon fosse andata via. M.lle Linon era andata via. Senza pensare, senza chiedersi come e perché, era corsa a lui e aveva fatto tutto quello che aveva fatto. 

- Andiamo dalla mamma - disse prendendolo per mano. Egli non poté dire nulla per molto tempo, non tanto perché temesse di sciupare con le parole la elevatezza del proprio sentimento, quanto perché ogni volta che voleva dire qualcosa, invece delle parole, sentiva che gli sarebbero sfuggite lacrime di gioia. Le prese la mano e la baciò. 

- Possibile che sia vero? - disse alla fine con voce sorda. - Non posso credere che tu mi ami! 

Ella sorrise di questo ``tu'' e della timidezza con la quale egli la guardava. 

- Sì - ella disse, significativamente, lentamente. - Sono così felice! 

Senza lasciare la mano di lui, entrò nel salotto. La principessa, dopo che li ebbe visti, cominciò a respirare forte, e subito si mise a piangere, e poi a ridere, e con un passo così energico quale Levin non si aspettava da lei, corse verso di loro e, abbracciato il capo di Levin, gli baciò e bagnò di lacrime le guance. 

- Allora tutto è concluso! Sono contenta. Amala. Sono felice\ldots{} Kitty. 

- Avete fatto presto - disse il vecchio principe, cercando di fare l'indifferente: ma Levin notò che i suoi occhi erano umidi quando si voltò verso di lui. - Da tanto, sempre, ho desiderato questo! - egli disse, prendendo la mano di Levin e tirandolo a sé. - E ancora, quando questa sventatella, s'era messa in testa\ldots{} 

- Papà - gridò Kitty e gli chiuse la bocca con le mani. 

- Su, basta - disse. - Sono molto\ldots{} Ah! Come sono sciocco\ldots{} 

Abbracciò Kitty, le baciò il viso, la mano, di nuovo il viso, e le fece un segno di croce. E Levin, nel vedere Kitty che a lungo e teneramente baciava la mano carnosa di lui, fu preso da un impeto d'amore per il vecchio principe, per quest'uomo che prima gli era estraneo. 

\capitolo{XVI}La principessa sedeva in poltrona, tacendo e sorridendo; il principe le stava accanto; Kitty stava presso la poltrona del padre senza lasciargli la mano. Tutti tacevano. 

La principessa per prima, con poche parole, riportò i pensieri e i sentimenti di tutti loro alle questioni pratiche della vita. E questo parve a tutti strano e persino penoso al primo momento. 

- Quando allora? Bisogna dare la benedizione e annunciarlo. E a quando le nozze? Che ti pare, Aleksandr? 

- Eccolo - disse il vecchio principe, indicando Levin - è lui il personaggio di centro. 

- Quando? - disse Levin, arrossendo. - Domani. Se domandate a me, per me, oggi la benedizione e domani le nozze. 

- Su, basta, mon cher, non dire sciocchezze! 

- Allora, fra una settimana. 

- È quasi pazzo. 

- No, perché mai? 

- Ma pensa! - disse la madre, sorridendo, felice di questa fretta. - E il corredo? 

``Possibile che ci sia di mezzo il corredo e tutto il resto? - pensò Levin con terrore. - E forse, il corredo e la benedizione e tutto il resto possono sciupare la mia felicità? Niente può guastarla! - Guardò Kitty e notò che non era per nulla contrariata dal fatto che si pensasse al corredo. - Si vede che è necessario'' pensò Levin. 

- Io di questo non capisco nulla, ho espresso soltanto il desiderio mio - disse, scusandosi. 

- Così, decideremo noi. Ora dobbiamo dare la benedizione e partecipare il fidanzamento. Si fa così. 

La principessa si accostò al marito, lo baciò e voleva andar via, ma egli la trattenne, l'abbracciò e, teneramente, come un giovane innamorato, la baciò più volte sorridendo. I vecchi, evidentemente, s'erano confusi per un attimo e non sapevano bene se erano loro a essere innamorati, o soltanto la figliuola. Quando il principe e la principessa uscirono, Levin si accostò alla fidanzata e le prese la mano. Egli era ora padrone di sé e poteva parlare e doveva dirle tante cose. Ma disse tutt'altro. 

- Come sapevo che sarebbe stato così! Non ho sperato mai, ma dentro di me sono sempre stato sicuro - disse. - Credo che fosse predestinato. 

- E io - ella disse - perfino allora\ldots{} - Si fermò e poi continuò, guardandolo decisa coi suoi occhi sinceri. - Perfino allora, quando ho respinto da me la mia felicità, ho amato sempre voi solo, ma mi ero esaltata. Devo dire\ldots{} Potete dimenticare? 

- Forse è stato per il meglio. Voi dovete perdonare molte cose. Io vi devo dire\ldots{} 

Era una di quelle cose che aveva deciso di dirle. Aveva deciso di dirle, fin dai primi giorni, due cose: una, che non era puro come lei, e l'altra che non aveva fede. Era tormentoso, ma riteneva di dover dire e l'una e l'altra cosa. 

- No, non ora, dopo! - egli disse. 

- Bene, dopo, ma me lo direte: assolutamente. Io non ho paura di nulla. Ho bisogno di sapere tutto. Ormai è concluso\ldots{} 

Egli terminò: 

- È concluso nel senso che mi prenderete così come sono, non rinuncerete a me? Sì? 

- Sì, sì. 

La loro conversazione fu interrotta da m.lle Linon che sorridendo con tenerezza, anche se con affettazione, venne a congratularsi con la sua allieva prediletta. Non era ancora andata via che vennero a congratularsi i domestici. Poi vennero i parenti, e cominciò quel beato stordimento dal quale Levin non uscì se non il giorno dopo le nozze. Levin si sentiva continuamente a disagio, si annoiava ma la tensione della sua felicità continuava sempre, aumentava. Sentiva ogni momento che si pretendevano da lui molte cose che egli ignorava, ma faceva tutto quello che gli dicevano e tutto questo gli procurava piacere. Pensava che il suo fidanzamento non dovesse aver nulla in comune con quello degli altri, che le solite abitudini avrebbero sciupato la sua particolare felicità, ma finì col fare tutto quello che fanno gli altri e malgrado ciò la sua felicità divenne sempre più grande e sempre più fuori del comune, una felicità che non aveva nulla di paragonabile. 

- Ora mangeremo i confetti - diceva m.lle Linon, e Levin andava a comprare i confetti. 

- Via, sono molto contento- diceva Svijazskij. - Vi consiglio di prendere i mazzi di fiori di Fomin. 

- Ci vogliono?- ed egli andava da Fomin. 

Il fratello gli diceva che occorreva prendere in prestito del denaro, perché ci sarebbero state molte spese, regali\ldots{} 

- Ci vogliono dei regali? - ed egli correva da Ful'de. 

E dal pasticciere e da Fomin e da Ful'de gli sembrava che lo aspettassero, che fossero contenti di lui e solennizzassero la sua felicità così come tutti quelli che avevano a che fare con lui in quei giorni. Era straordinario non solo che tutti lo amassero, ma che anche le persone antipatiche, fredde, indifferenti, ammirandolo, gli si sottomettessero in tutto e avessero pel suo sentimento tatto e tenerezza, condividendo la persuasione sua d'essere l'uomo più felice del mondo perché la sua fidanzata era il vertice di ogni perfezione. Lo stesso provava anche Kitty. Quando la contessa Nordston si permise di accennare al fatto che avrebbe desiderato per lei qualcosa di meglio, Kitty si accalorò tanto e dimostrò con tanta convinzione che non poteva esservi al mondo alcuno migliore di Levin, che la contessa Nordston dovette riconoscerlo e da allora in poi, in presenza di Kitty, accolse Levin con un sorriso di ammirazione. 

La spiegazione da lui promessa fu l'unico avvenimento penoso di quel periodo. Si consigliò col vecchio principe, e, ottenutone il permesso, consegnò a Kitty il suo diario nel quale era scritto quello che lo tormentava. Aveva scritto questo diario proprio pensando alla sua futura sposa. Lo tormentavano due cose: la sua impurità e la sua mancanza di fede. Quest'ultima confessione passò inosservata. Ella era religiosa, non aveva mai avuto dubbi sulla verità della religione; ma l'ateismo esteriore di lui non la turbava quasi. Ella conosceva per mezzo dell'amore tutta l'anima sua e nell'anima sua vedeva tutto quello che voleva, e che un tale stato d'animo significasse non aver fede, le era assolutamente indifferente. L'altra confessione invece la fece piangere amaramente. 

Levin le aveva dato il suo diario non senza lotta interiore. Sapeva che fra di loro non potevano e non dovevano esserci segreti e perciò aveva deciso di far così, ma non si era reso conto degli effetti che ne sarebbero potuti derivare, non si era trasferito in lei. Appena giunse da lei quella sera, e, prima del teatro, entrò nella stanza di lei e vide il caro e pietoso viso piangente, infelice per un dolore irrimediabilmente prodottole da lui, solo allora comprese l'abisso che separava il proprio passato dalla innocenza di colomba di lei ed ebbe orrore di quello che aveva fatto. 

- Prendete, prendete con voi questi orribili scritti - ella disse respingendo i quaderni che le stavano davanti sul tavolo. - Perché me li avete dati? No, ma forse è meglio - aggiunse, avendo pena del viso disperato di lui. - Ma è orribile, orribile. 

Egli abbassò il capo e tacque. Non poteva dir nulla. 

- Voi non mi perdonerete - mormorò. 

- No, io ho già perdonato, ma è orribile! 

Ma la felicità di lui era così grande che questa confessione non la turbò, le diede solo una sfumatura nuova. Gli aveva perdonato; ma da quel momento in poi egli si considerò ancora più indegno di lei, ancora più si inchinò moralmente dinanzi a lei e ancor più apprezzò la propria immeritata felicità. 

\capitolo{XVII}Rivolgendo involontariamente nella mente le impressioni dei discorsi fatti durante e dopo il pranzo, Aleksej Aleksandrovic tornava nella sua solitaria stanza d'albergo. Le parole di Dar'ja Aleksandrovna circa il perdono lo avevano soltanto irritato. L'adempiere o meno la regola cristiana era questione difficile a risolversi nel suo caso, non se ne poteva parlare alla leggera ed era stata già da tempo decisa da Aleksej Aleksandrovic negativamente. Di quanto si era detto gli erano impresse maggiormente le parole dello sciocco e buon Tuškevic: È stato bravo, lo ha sfidato e lo ha ucciso. Tutti, evidentemente, avevano approvato, sebbene, per cortesia, non l'avessero detto. 

``Del resto, questa faccenda è conclusa, è inutile pensarci su'' si disse Aleksej Aleksandrovic. E, pensando solo al viaggio imminente e all'ispezione da compiere, entrò in camera e chiese al portiere che l'accompagnava, dove fosse il suo servitore. Il portiere disse che era uscito proprio in quel momento. Aleksej Aleksandrovic ordinò di portargli il tè, sedette alla tavola e, preso il Frum, cominciò a combinare l'itinerario del viaggio. 

- Due telegrammi - disse il servitore di ritorno, entrando in camera. - Scusate, eccellenza, ero appena uscito. 

Aleksej Aleksandrovic prese i telegrammi e li dissuggellò. Il primo telegramma portava la notizia della nomina di Stremov allo stesso posto cui aspirava Karenin. Aleksej Aleksandrovic gettò via il dispaccio, e, fattosi rosso in viso, si alzò e prese a camminare per la stanza. ``Quos vult perdere dementat'' disse, riferendo quel quos alle persone che avevano contribuito a quella nomina. Non lo irritava il fatto di non aver conseguito lui quel posto, né d'essere stato, evidentemente, messo da parte; ma per lui era inconcepibile, sorprendente come costoro non si accorgessero che quel ciarlatano, quel parolaio di Stremov fosse meno di ogni altro adatto. Come non si accorgevano che rovinavano se stessi, il proprio prestige con quella nomina! 

``Ancora qualcosa di questo genere'' disse con bile, aprendo il secondo dispaccio. Il telegramma era della moglie. La firma a matita turchina ``Anna'' gli saltò per prima agli occhi. ``Muoio, prego, supplico venire. Morirò più tranquilla col perdono'' egli lesse. Sorrise sprezzante e gettò via il telegramma. Che questo fosse inganno e astuzia era fuor di dubbio, così gli parve al primo momento. 

``Non c'è inganno dinanzi al quale ella si arresti. Deve partorire. Forse una malattia di parto. Ma quale ne è lo scopo? Legittimare il neonato, compromettere me e ostacolare il divorzio - pensava. - Ma lì è detto qualcosa: muoio\ldots{}''. Rilesse il telegramma; e improvvisamente il senso letterale di quello che aveva letto lo colpì. ``E se fosse vero? - si disse. - Se è vero che in un momento di sofferenza e di prossimità alla morte, ella si penta sinceramente e io, sospettando in questo un inganno, mi rifiuti di andare? Non solo sarebbe crudele e tutti mi giudicherebbero male, ma sarebbe stolto da parte mia''. 

- Pëtr, trattieni la carrozza. Vado a Pietroburgo - disse al servitore. 

Aleksej Aleksandrovic decise di andare a Pietroburgo e di vedere la moglie. Se la malattia era un inganno, avrebbe taciuto e sarebbe ripartito. Se realmente era malata, in punto di morte, egli le avrebbe perdonato se la trovava ancora in vita, e le avrebbe reso gli estremi onori se fosse arrivato troppo tardi. 

Per tutto il viaggio non pensò più a quello che doveva fare. 

Con un senso di stanchezza e di trascuratezza, derivato dalla notte in treno, Aleksej Aleksandrovic andava nella nebbia mattutina di Pietroburgo per il Nevskij Prospekt deserto e guardava dinanzi a sé senza pensare a ciò che lo aspettava. Non poteva pensare perché, immaginando quello che sarebbe accaduto, non poteva scacciare l'idea che la morte di lei avrebbe risolto d'un tratto la difficoltà della propria situazione. I fornai, le botteghe chiuse, i vetturini notturni, i portieri che spazzavano i marciapiedi baluginavano dinanzi ai suoi occhi ed egli osservava tutto ciò cercando di soffocare il pensiero di quello che lo aspettava e che tuttavia desiderava. Si accostò alla scala. Una vettura con un cocchiere addormentato stava all'ingresso. Entrando nel vestibolo Aleksej Aleksandrovic trasse come da un angolo remoto del cervello la propria decisione e si consultò; in quell'angolo era scritto: ``Se c'è inganno, allora calmo disprezzo, e via. Se è vero, allora salvare le convenienze''. 

Il portiere aprì la porta ancora prima che Aleksej Aleksandrovic sonasse. Il portiere Petrov, chiamato Kapitonyc, aveva un aspetto strano, così com'era in quella vecchia finanziera senza cravatta e in pantofole. 

- Come sta la signora? 

- Ha partorito ieri felicemente. 

Aleksej Aleksandrovic si fermò e impallidì. Adesso capiva con chiarezza quanto intensamente avesse desiderato la morte di lei. 

- E come va la salute? 

Kornej in grembiule da mattina, scese di corsa dalla scala. 

- Molto male - rispose. - Ieri c'è stato un consulto e ora il dottore è qui. 

- Prendi la roba - disse Aleksej Aleksandrovic e, provando un certo sollievo alla notizia che c'era pur sempre speranza di morte, entrò nell'ingresso. 

Sull'attaccapanni c'era un cappotto militare. Aleksej Aleksandrovic lo notò e chiese: 

- Chi c'è qui? 

- Il dottore, la levatrice e il conte Vronskij. 

Aleksej Aleksandrovic passò nelle stanze interne. 

Nel salotto non c'era nessuno: dallo studio, al suono dei passi di lui, uscì la levatrice con una cuffia di nastri lilla. 

Si avvicinò ad Aleksej Aleksandrovic e, con la confidenza che dà la prossimità della morte, presolo per mano, lo tirò verso la camera. 

- Sia lodato Iddio che siete arrivato. Solo di voi, solo di voi domanda - ella disse. 

- E date del ghiaccio, presto! - si sentì dalla camera la voce imperiosa del medico. 

Aleksej Aleksandrovic passò nello studio di lei. Vicino alla tavola, di traverso contro la spalliera, su di una sedia bassa, stava Vronskij e, coperto il viso con le mani, piangeva. Saltò su alla voce del dottore, tolse le mani dal viso e vide Aleksej Aleksandrovic. Visto il marito, si confuse al punto da doversi sedere di nuovo, ritraendo il capo nelle spalle quasi desiderando di sprofondare in qualche posto, poi fece uno sforzo su di sé, si alzò e disse: 

- Muore. I dottori hanno detto che non c'è più speranza. Io sono in vostro potere, ma permettetemi di stare qui\ldots{} del resto, io sono a vostra disposizione. 

Aleksej Aleksandrovic, viste le lacrime di Vronskij, sentì un afflusso di quello sconvolgimento d'animo che produceva in lui la vista della sofferenza altrui, e voltando il viso, senza finir di ascoltare le sue parole, si avviò in fretta verso la porta. Dalla camera si udiva la voce di Anna che diceva qualcosa. La voce era animata, viva, con intonazioni straordinariamente precise. Aleksej Aleksandrovic entrò in camera e si accostò al letto. Ella giaceva col viso rivolto verso di lui. Le guance erano rosse, gli occhi lucidi; le piccole mani bianche, uscendo dal polso della camicia, brancicavano, attorcigliandolo, un angolo del lenzuolo. Sembrava che non solo ella fosse sana e fresca, ma nella migliore disposizione d'animo. Parlava in fretta, ad alta voce, con un'intonazione di voce insolitamente giusta e precisa. 

- Perché Aleksej, io parlo di Aleksej Aleksandrovic (che strano, orribile destino che siano tutte e due Aleksej, non è vero?), Aleksej non mi direbbe di no. Io dimenticherei, lui perdonerebbe\ldots{} Ma come mai non viene? Lui è buono, non lo sa neanche lui quanto è buono. Ah, Dio mio! Che pena! Datemi presto dell'acqua. Ah, ma questo a lei, alla mia bambina, farà male! Su, via, va bene, datele una balia. Su, io consento, è anche meglio. Egli arriverà, gli farà male vederla. Datela via. 

- Anna Arkad'evna, egli è arrivato. Eccolo - diceva la levatrice, cercando di richiamare su Aleksej Aleksandrovic l'attenzione di lei. 

- Ah, che sciocchezza! - continuava Anna, senza scorgere il marito. - Su, datemela, la bambina, datemela. Non è venuto ancora. Voi dite che non perdonerà perché non lo conoscete. Nessuno lo conosce. Io sola, e ne ho tanta pena ora. I suoi occhi, dovete sapere, sono quelli di Serëza e perciò non posso guardarli. Hanno dato da mangiare a Serëza? Perché io so che tutti dimenticano. Egli non avrebbe dimenticato. Bisogna trasferire Serëza nella stanza d'angolo e pregare Mariette di dormire con lui. 

Improvvisamente si contrasse, ammutolì e con spavento, come se aspettasse un colpo e volesse difendersene, portò le mani al volto. Aveva visto il marito. 

- No, no - prese a dire - non ho paura di lui, ho paura della morte. Aleksej, avvicinati qua. Ho fretta perché non ho tempo, mi resta poco da vivere, subito ricomincerà la febbre e non capirò più nulla. Adesso capisco tutto e vedo tutto. 

Il viso corrugato di Aleksej Aleksandrovic prese un'espressione martoriata; le afferrò la mano e voleva dire qualcosa, ma non poté articolare parola; il suo labbro inferiore tremava, egli lottava ancora sempre con la propria agitazione e solo di rado la guardava. E ogni volta che la guardava, scorgeva che i suoi occhi lo guardavano con una tenerezza così commossa e incantata quale egli non aveva mai vista in lei. 

- Aspetta, tu non sai\ldots{} Fermatevi, fermatevi\ldots{} - ella si fermò, come per raccogliere le idee. - Ecco quello che volevo dire. Non meravigliarti di me. Io sono sempre la stessa. Ma in me c'è un'altra donna; lei ha cominciato ad amare quell'altro e io volevo odiarti e non potevo dimenticare quello che c'era stato prima. Quella non sono io. Adesso io sono la vera, sono una. Adesso muoio, so che morirò, chiedilo a lui. Io ora sento, ecco, i pesi sulle braccia, sulle gambe, sulle dita. Le dita\ldots{} ecco come sono enormi! Ma tutto questo finirà presto\ldots{} Solo una cosa mi è necessaria: perdonami, perdonami di tutto! Sono detestabile, ma la mia njanja mi diceva che una santa martire\ldots{} come si chiamava?\ldots{} era stata peggiore di me. Andrò a Roma, là ci sono degli eremi, e allora non darò fastidio a nessuno, prenderò solo Serëza e la bambina\ldots{} No, tu non puoi perdonare! no, va' via, sei troppo buono! - Ella teneva con una mano che scottava la mano di lui, con l'altra lo respingeva. 

Lo sconvolgimento d'animo di Aleksej Aleksandrovic si faceva sempre più forte ed era giunto a un tale punto che egli aveva già smesso di dominarlo; a un tratto sentì che quello che egli considerava uno sconvolgimento, era, al contrario, un beato stato d'animo che gli dava a un tratto una nuova felicità mai prima provata. Egli non pensava più che la legge cristiana che avrebbe voluto seguire per tutta la vita gli prescriveva di perdonare e di amare i nemici, ma un gioioso sentimento d'amore e di perdono verso i nemici gli riempiva ora l'anima. Stava in ginocchio e, posto il capo sulla giuntura del braccio di lei che lo bruciava come fuoco attraverso la camiciola, singhiozzava come un bambino. Ella abbracciò la sua testa quasi calva, lo accostò a sé e levò gli occhi in su con una espressione di orgoglio e di sfida. 

- Eccolo, io lo sapevo! Ora addio a tutti, addio!\ldots{} Sono venuti di nuovo, perché non vanno via? Ma toglietemi queste pellicce! 

Il dottore le sciolse le braccia e, dopo averla adagiata con precauzione sul guanciale, le coprì le spalle. Ella si sdraiò docile e guardò dinanzi a sé con uno sguardo raggiante. 

- Ricordati una cosa sola, che avevo bisogno solo del perdono e che non voglio altro\ldots{} E perché lui non viene? - cominciò a dire volgendosi attraverso la porta a Vronskij. - Avvicinati, avvicinati, dagli la mano. 

Vronskij si accostò alla sponda del letto e, guardandola si coprì di nuovo il viso con le mani. 

- Scopri il viso, guardalo! È un santo - ella disse. - Ma scopri, scopri il viso - prese a dire con rabbia. - Aleksej Aleksandrovic, scoprigli il viso. Lo voglio vedere. 

Aleksej Aleksandrovic prese le mani di Vronskij e le allontanò dal viso sfigurato dall'espressione di tormento e di vergogna. 

- Dàgli la mano, perdonagli. 

Aleksej Aleksandrovic gli dette la mano, senza trattenere le lacrime che gli correvano giù dagli occhi. 

- Sia lodato Iddio. Sia lodato Iddio - ella cominciò a dire. - Ora tutto è pronto. Bisogna stendere solo un po' le gambe. Ecco, così, benissimo. Come son fatti senza gusto questi fiori! non sono per nulla simili alle viole - diceva mostrando la tappezzeria. - Dio mio, Dio mio! Quando finirà tutto questo? Datemi della morfina. Dottore datemi della morfina. Dio mio, Dio mio! 

E cominciò ad agitarsi sul letto. 

Il medico curante e i dottori dicevano trattarsi di una febbre puerperale nella quale su cento probabilità novantanove erano di morte. Tutto il giorno ella ebbe febbre, delirio e deliquio. A mezzanotte la malata giaceva priva di sensi e quasi senza polso. 

Si aspettava la fine da un momento all'altro. 

Vronskij andò a casa, ma la mattina venne a prendere notizie, e Aleksej Aleksandrovic, incontrandolo nell'ingresso, disse: 

- Restate, può darsi che cerchi di voi - ed egli stesso lo introdusse nello studio della moglie. 

Verso la mattina cominciò di nuovo l'agitazione, l'eccitamento, la velocità del pensiero e del discorso e sopravvenne un nuovo deliquio. Per due giorni avvenne lo stesso, e i dottori dissero che c'era speranza. Quel giorno Aleksej Aleksandrovic andò nello studio dove era Vronskij e, chiusa la porta, sedette di fronte a lui. 

- Aleksej Aleksandrovic - disse Vronskij, sentendo che si avvicinava la spiegazione - io non posso parlare, non posso intendere. Risparmiatemi! Per quanto voi soffriate, credetemi, per me è ancora più orribile. 

Voleva alzarsi. Ma Aleksej Aleksandrovic lo prese per una mano e disse: 

- Vi prego di ascoltarmi, è indispensabile. Devo spiegarvi i miei sentimenti, quelli che mi hanno guidato e che mi guideranno, perché voi non abbiate a sbagliare sul mio conto. Sapete che mi ero deciso a chiedere il divorzio e avevo persino iniziato la causa. Non vi nascondo che, dando inizio a un giudizio, ero indeciso, mi tormentavo; vi confesso che il desiderio di vendicarmi di voi e di lei mi perseguitava. Quando ho ricevuto il telegramma, sono venuto qua con gli stessi sentimenti nell'animo, dirò di più: desideravo la sua morte. Ma\ldots{} - tacque un po' dubbioso se aprirgli o no l'animo suo. - Ma l'ho vista e ho perdonato. E la gioia del perdono mi ha rivelato il mio dovere. Ho perdonato completamente. Voglio porgere l'altra guancia, voglio dare la tunica quando mi si è preso il mantello. Prego Iddio che non allontani da me la gioia del perdono! - Le lacrime erano nei suoi occhi e il suo sguardo chiaro, tranquillo colpì Vronskij. - Ecco la mia situazione. Voi potete calpestarmi nel fango, fare di me lo zimbello del mondo; io non abbandonerò lei, e non dirò mai a voi una parola di recriminazione - continuò. - Il mio dovere è, per me, chiaramente segnato: devo essere con lei e ci sarò. Se desidera di vedervi, ve lo farò sapere, ma ora suppongo che sia meglio per voi allontanarvi. 

Si alzò e i singhiozzi gli spezzarono le parole. Anche Vronskij si era alzato e, restando curvo, lo guardò di sotto in su. Non capiva i sentimenti di Aleksej Aleksandrovic; ma sentiva che vi era in essi qualcosa di molto più alto, e persino di inaccessibile a lui e alla propria visione del mondo. 

\capitolo{XVIII}Dopo il colloquio con Aleksej Aleksandrovic, Vronskij uscì sulla scala di casa Karenin e vi si fermò, ricordando a stento dove si trovasse e se dovesse andare a piedi o in vettura. Si sentiva svergognato, umiliato, colpevole e nella impossibilità di riscattare la propria umiliazione. Si sentiva lanciato al di fuori di quella carreggiata sulla quale aveva finora camminato con tanto orgoglio e tanta disinvoltura. Tutte le abitudini e regole di vita che gli erano parse sempre così salde, gli si erano improvvisamente mostrate mendaci e inapplicabili. Il marito ingannato, che gli si era fino allora presentato come un essere pietoso, un impedimento casuale e un po' ridicolo alla propria felicità, a un tratto era stato invocato proprio da lei e si era elevato tanto in alto da ispirargli rispetto; e quello stesso marito, pur così in alto ora, non s'era mostrato cattivo, falso, ridicolo, ma buono, semplice, generoso. Tutto questo Vronskij non poteva non sentirlo. Le parti si erano improvvisamente cambiate. Vronskij sentiva la superiorità di lui e la umiliazione propria. Sentiva che quel marito era grande anche nel suo dolore, ed egli basso, meschino nel suo inganno. Ma questa coscienza della propria bassezza, di fronte all'uomo che ingiustamente aveva oltraggiato, rappresentava solo una piccola parte del suo dolore. Si sentiva ora indicibilmente infelice, perché la sua passione per Anna, che gli sembrava intiepidita negli ultimi tempi, nel momento in cui sapeva di averla perduta per sempre, era diventata più forte di quanto non lo fosse mai stata. L'aveva vista tutta durante la sua malattia; aveva imparato a conoscerne l'anima, e gli pareva di non averla mai amata fino ad allora. E proprio adesso, quando aveva imparato a conoscerla e l'aveva presa ad amare così come si deve amare, egli era umiliato davanti a lei, e la perdeva per sempre, lasciando di sé solo un ricordo vergognoso. Più terribile di tutto era stata quella ridicola, umiliante situazione, in cui Aleksej Aleksandrovic gli aveva tolto le mani dalla faccia svergognata. Egli stava fermo sulla scala di casa Karenin, come smemorato, senza sapere cosa fare. 

- Volete una vettura? - chiese il portiere. 

- Sì, una vettura. 

Tornato a casa dopo tre notti insonni, Vronskij, senza spogliarsi, si coricò bocconi sopra un divano, incrociando le mani e poggiandovi sopra la testa. La testa gli pesava. Le immagini, le memorie e le idee più strane si susseguivano le une alle altre con straordinaria velocità e chiarezza: ora la medicina che aveva versato all'ammalata e che aveva fatto gocciolare dal cucchiaino, ora le braccia bianche della levatrice, o la strana posizione di Aleksej Aleksandrovic sul pavimento, davanti al letto. 

``Addormentarsi, dimenticare!'' si disse con la calma certezza dell'uomo sano che se è stanco e vuol dormire, s'addormenta subito. E invero, in quello stesso momento, nella sua testa sopraggiunsero confusione e oblio. Le onde della vita subcosciente avevano già cominciato ad affluirgli alla testa. Ma a un tratto, proprio come se una fortissima scarica elettrica si fosse scaraventata su di lui, rabbrividì in modo che tutto il corpo sussultò sulle molle del divano e, poggiatosi con le mani, saltò su in ginocchio, spaventato. I suoi occhi erano spalancati, come se non si fosse mai addormentato. La pesantezza di testa e la debolezza delle membra erano scomparse d'un tratto. 

``Potete calpestarmi nel fango'' riudiva le parole di Aleksej Aleksandrovic e lo vedeva davanti a sé e vedeva il viso di Anna arrossato dalla febbre e gli occhi scintillanti che guardavano con tenerezza e amore, non lui, ma Aleksej Aleksandrovic; vedeva la propria figura, così com'era apparsa, fatua e ridicola, mentre Aleksej Aleksandrovic gli toglieva le mani dal viso. Di nuovo distese le gambe e si gettò sul divano nella posizione di prima e chiuse gli occhi. 

``Addormentarsi, addormentarsi!'' ripeteva a se stesso. Ma con gli occhi chiusi vedeva ancora più chiaramente il viso di Anna così come era quella sera per lui memorabile delle corse. 

``Questo non è e non sarà, ed ella desidera cancellarlo dalla sua mente. Io invece non posso vivere senza questo. Come potremo mai fare pace, come potremo mai fare pace'' diceva ad alta voce e inconsciamente cominciò a ripetere queste parole. Questa ripetizione di parole tratteneva il sorgere di nuove immagini e di nuovi ricordi che, lo sentiva, gli si affollavano in capo. Ma quella ripetizione di parole non trattenne a lungo l'immaginazione. Di nuovo uno dopo l'altro cominciarono ad apparire i momenti migliori e con essi la recente umiliazione. ``Togli le mani'' diceva la voce di Anna. E lui toglieva le mani e sentiva l'espressione confusa e stupida della propria faccia. 

Era sempre disteso, cercando di addormentarsi, sebbene sentisse che fosse vano, e ripeteva sempre sottovoce le parole casuali di qualche pensiero, tentando così di trattenere il sorgere di nuove immagini. Si pose in ascolto e sentì ripetere con uno strano, pazzo mormorio: ``non hai saputo apprezzare, non hai saputo profittare''. 

``Cos'è, divento pazzo forse? - si disse. - Forse. E del resto, non si diventa forse pazzi, non ci si ammazza, addirittura?'' rispose a se stesso, e, aperti gli occhi, vide con sorpresa, sotto la testa, il cuscino ricamato da Varja, la moglie del fratello. Toccò il fiocco del cuscino e cercò di ricordarsi di Varja, di quando l'aveva vista l'ultima volta. Ma pensare a qualcosa di estraneo era tormentoso. ``No, bisogna addormentarsi''. Accostò il cuscino e si strinse ad esso colla testa, ma doveva fare uno sforzo per tenere gli occhi chiusi. Saltò su e sedette. ``È finita per me - disse. - Bisogna riflettere quel che occorre fare. Che cosa è rimasto?''. Il suo pensiero percorse veloce la propria vita all'infuori del suo amore per Anna. 

L'ambizione? Serpuchovskoj? Il bel mondo? La Corte? Non poteva fermare la mente su nessuna di queste cose. Tutto questo aveva un senso prima, ma ora non esisteva più nulla. Si alzò dal divano, si tolse la finanziera, slacciò la cintura e, scoperto il petto villoso, per respirare più liberamente, fece un giro per la stanza. ``È così che s'impazzisce, è così che ci si spara\ldots{} per non sentire la vergogna'' aggiunse lentamente. 

Si avvicinò alla porta e la chiuse; dopo, con uno sguardo fisso e coi denti fortemente stretti, si accostò alla tavola e, presa la rivoltella, la esaminò, la rigirò dalla parte della canna carica e si fece pensieroso. Rimase due minuti così, col capo chino e l'espressione nel volto di uno sforzo mentale, con la rivoltella in mano, immobile, e pensava. ``S'intende'' disse, come se un logico, prolungato, chiaro passaggio di idee lo avesse condotto a una conclusione indubitabile. In realtà questo ``s'intende'' per lui persuasivo, non era che la conseguenza della ripetizione di un giro sempre identico di ricordi e di figurazioni attraverso il quale era già passato, diecine di volte, nello spazio di un'ora. Sempre gli stesi ricordi della felicità perduta, sempre identica la rappresentazione della mancanza di senso di tutto quello che gli offriva la vita, identica la coscienza della propria umiliazione. Identica era anche la successione di queste immagini e di questi sentimenti. 

``S'intende'' ripeté, quando per la terza volta il pensiero si volse verso la stessa cerchia incantata di ricordi e di pensieri e, poggiata la rivoltella al lato sinistro del petto, strettala forte con tutta la mano, come se d'un tratto l'avesse impugnata, tirò il grilletto. Non sentì il rumore dello sparo, ma un colpo violento nel petto lo buttò a terra. Voleva reggersi all'orlo della tavola, lasciò cadere la rivoltella, vacillò e si sedette per terra, guardandosi intorno con sorpresa. Non riconosceva la sua camera, guardando dal basso i piedi curvi della tavola, il cestino per le carte e la pelle di tigre. I passi veloci scricchiolanti del servitore che camminava nel salotto lo fecero tornare in sé. Fece uno sforzo per pensare e capì che era a terra e, visto il sangue sulla pelle di tigre e sulla mano, capì che s'era sparato. 

- È sciocco! Ho fallito il colpo! - esclamò, con la mano che tastava in cerca della rivoltella. La rivoltella era vicino a lui; lui cercava più in là. Continuando a cercare si protese dall'altra parte e, non avendo la forza di mantenere l'equilibrio, cadde, perdendo sangue. 

L'elegante servitore con le fedine, che più di una volta s'era lamentato con gli amici per la propria debolezza di nervi, si spaventò a tal punto nel vedere il padrone per terra, che lo lasciò a perdere sangue e corse a chiedere aiuto. Dopo un'ora Varja, la moglie del fratello, giunta insieme con tre dottori chiamati da ogni parte e arrivati nello stesso momento, adagiò sul letto il ferito e rimase da lui per curarlo. 

\capitolo{XIX}L'errore di Aleksej Aleksandrovic di non aver pensato, nel rivedere la moglie, e nel caso che il rimorso di lei fosse sincero e ch'egli perdonasse, alla eventualità che ella non morisse, questo errore, due mesi dopo il suo ritorno da Mosca, gli si parò innanzi in tutta la sua gravità. Ma l'errore suo era derivato non solo dal non aver supposto questa eventualità, ma anche dal non aver mai, fino a quel giorno dell'incontro con la moglie morente, conosciuto il proprio cuore. Egli per la prima volta in vita sua, presso il letto della moglie malata, si era abbandonato a quel sentimento di commossa compassione che in lui suscitavano le sofferenze altrui e di cui prima si vergognava come di una debolezza nociva; e la pena verso di lei e il rimorso di averne desiderato la morte e soprattutto la stessa gioia del perdono avevano fatto sì ch'egli, improvvisamente, avesse sentito non solo un lenimento alle proprie pene, ma anche una tranquillità d'animo che non aveva mai provato prima. Improvvisamente aveva sentito che proprio quello che era la causa delle sue pene, diveniva la sorgente della sua gioia spirituale, quello che pareva insolubile, quando egli rimproverava, recriminava e odiava, era divenuto semplice e chiaro, ora che perdonava e amava. 

Aveva perdonato alla moglie e aveva avuto pena di lei per le sofferenze sue e per il suo rimorso. Aveva perdonato Vronskij e lo commiserava, specialmente dopo che erano giunte a lui le voci del suo atto insano. Anche del figlio aveva più pena di prima e si rimproverava ora di essersi tanto poco occupato di lui. Per la piccola neonata, poi, provava un sentimento particolare, non solo di pena, ma di tenerezza. In principio, per pura compassione egli si era occupato di quella fragile creatura appena nata che non era figlia sua e che era stata trascurata durante la malattia della madre, e che certamente sarebbe morta se egli non si fosse preoccupato di lei; ma non s'era accorto neppur lui che aveva cominciato a volerle bene. Varie volte al giorno andava nella camera dei bambini e a lungo restava là a sedere, tanto che la balia e la njanja, prima timide per la sua presenza, s'erano poi abituate a lui. A volte guardava in silenzio per una mezz'ora intera il visino addormentato rosso-zafferano, lanuginoso e grinzoso della bambina e osservava i movimenti della fronte aggrottata e le manine paffute con le dita ripiegate che col dorso si fregavano gli occhietti e la radice del naso. Proprio in quei momenti Aleksej Aleksandrovic si sentiva completamente tranquillo e in armonia con se stesso, e non vedeva nella sua posizione nulla di eccezionale, nulla che fosse da cambiare. 

Ma quanto più tempo passava, tanto più chiaramente egli scorgeva che, per quanto ora questa posizione gli paresse naturale, non gli avrebbero consentito di permanervi. Sentiva che oltre alla felice forza spirituale che aveva guidato la sua anima, c'era un'altra forza di natura materiale, ma altrettanto e ancor più potente, che dirigeva la sua vita e che non gli consentiva di fermarsi in quella umile tranquillità che desiderava. Sentiva che tutti guardavano a lui con interrogativo stupore, che non lo capivano e che si aspettavano qualcosa da lui. Sentiva inoltre l'instabilità e, in modo particolare, la falsità dei suoi rapporti con la moglie. 

Appena dileguata quella commozione prodotta in lei dalla prossimità della morte, Aleksej Aleksandrovic cominciò a notare che Anna lo temeva, che si sentiva oppressa da lui e che non poteva guardarlo dritto negli occhi. Era come s'ella desiderasse qualcosa e non si decidesse a dirglielo, e anche per lei era come se intuisse che quei rapporti non potevano durare, e attendesse qualcosa da lui. 

Alla fine di febbraio, la neonata di Anna, chiamata anch'essa Anna, s'ammalò. Aleksej Aleksandrovic era stato la mattina nella camera dei bambini e, dato ordine di mandare a chiamare il medico, era andato al ministero. Finiti i suoi affari, tornò a casa dopo le tre. Entrando in anticamera, vide un servitore, un bel giovane con una pellegrina d'orso e alamari, che reggeva un mantello bianco di cane americano. 

- Chi è qui? - domandò Aleksej Aleksandrovic. 

- La principessa Elizaveta Fëdorovna Tverskaja - rispose il servitore con un sorriso, come parve ad Aleksej Aleksandrovic. 

Durante quel penoso periodo, Aleksej Aleksandrovic notava che le sue conoscenze mondane, in particolare le donne, s'interessavano vivamente a lui e a sua moglie. Notava in tutti questi amici come una certa gioia a stento contenuta, quella stessa che egli aveva sorpreso negli occhi dell'avvocato e che ora scorgeva negli occhi del servitore. Come se tutti fossero in una certa esaltazione, come se si dovesse sposare qualcuno. Quando lo incontravano domandavano della salute di sua moglie con una gioia appena celata. 

La presenza della principessa Tverskaja, e per i ricordi legati a lei, e perché in complesso non gli era simpatica, non era gradita ad Aleksej Aleksandrovic, ed egli andò di filato nella camera dei bambini. Nella prima stanza Serëza, disteso col petto sulla tavola e con i piedi su di una sedia, disegnava qualcosa commentando allegramente. L'inglese, che durante la malattia di Anna aveva sostituito la governante francese, seduta accanto al ragazzo, con un lavoro di merletto a maglia tra le mani, si alzò in fretta, fece un inchino e scosse Serëza. 

Aleksej Aleksandrovic carezzò con la mano il figlio sui capelli, rispose alla governante sulla salute della moglie, e domandò che cosa avesse detto il medico della baby. 

- Il dottore ha detto che non c'è nulla da impensierirsi e ha ordinato dei bagni, signore. 

- Ma lei soffre sempre - disse Aleksej Aleksandrovic, prestando orecchio al piagnucolio della bambina nella camera accanto. 

- Io penso che la balia non sia adatta, signore - disse decisa l'inglese. 

- Perché lo pensate? - egli domandò fermandosi. 

- È successo così dalla contessa Pol', signore. Curavano il bambino e poi capirono che il bambino aveva semplicemente fame: la balia non aveva latte, signore. 

Aleksej Aleksandrovic si fece pensieroso e, fermatosi per alcuni secondi, entrò nell'altra stanza. La bambina era lì supina, con la testa rovesciata all'indietro, e, dibattendosi in braccio alla balia, non voleva attaccarsi al petto pieno che le veniva offerto, né acquietarsi malgrado il doppio zittio della balia e della njanja, curve su di lei. 

- Nessun miglioramento ancora? - disse Aleksej Aleksandrovic. 

- È molto inquieta - rispose sottovoce la njanja. 

- Miss Edward dice che forse la balia non ha latte - egli disse. 

- Anch'io lo penso, Aleksej Aleksandrovic. 

- Come mai non lo dite? 

- A chi dirlo? Anna Arkad'evna è sempre ammalata - disse la njanja scontenta. 

La njanja era una vecchia donna di casa. Anche in queste sue semplici parole parve ad Aleksej Aleksandrovic di scorgere un'allusione alla propria situazione. 

La bambina gridava ancora più forte rimanendo senza fiato e rantolando. La njanja fece un gesto sconsolato con la mano, le si avvicinò, la prese dalle braccia della balia e cominciò a cullarla camminando. 

- Bisogna pregare il dottore di visitare la balia - disse Aleksej Aleksandrovic. 

La nutrice, tutta adorna, sana all'aspetto, spaventata all'idea di essere licenziata, mormorò qualcosa fra i denti e, nascondendo il vasto petto, sorrise sprezzante del dubbio sull'allattamento. Anche in questo sorriso parve ad Aleksej Aleksandrovic di scorgere una certa ironia verso la propria situazione. 

- Povera bambina! - disse la njanja acquietando la piccola, e seguitò a camminare. 

Aleksej Aleksandrovic si era seduto su di una sedia e col viso abbattuto e sofferente, guardava la njanja che andava avanti e indietro. 

Quando alla fine posero la bambina, finalmente acquietata, nel lettino profondo, e la njanja, accomodato il guancialino, se ne fu allontanata, Aleksej Aleksandrovic si alzò e, avanzando cauto in punta di piedi, si accostò alla piccola. Per un momento stette in silenzio, e con quello stesso viso triste guardò la bambina: ma improvvisamente un sorriso, che gli aggrinzò i capelli e la pelle sulla fronte, gli salì al viso, ed egli ugualmente piano uscì dalla camera. 

In sala da pranzo, bussò e ordinò al cameriere di mandare di nuovo per il dottore. Era irritato con la moglie che non si preoccupava di quella bambina deliziosa e non voleva in tale disposizione di spirito andare da lei, né vedere la principessa Betsy; ma la moglie avrebbe potuto meravigliarsi ch'egli non passasse da lei come al solito, e perciò, dominandosi, si diresse in camera. Si avvicinò sul tappeto morbido accanto alla porta, e sentì involontariamente una conversazione che non avrebbe voluto ascoltare. 

- Se non partisse, capirei il vostro rifiuto e anche quello di lui. Ma vostro marito deve essere superiore a questo - diceva Betsy. 

- Io non lo voglio, né per mio marito, né per me. Non me ne parlate - rispondeva la voce agitata di Anna. 

- Già, ma voi non potete non desiderare di perdonare a un uomo che si è sparato per voi\ldots{} 

- Appunto per questo non voglio. 

Aleksej Aleksandrovic si fermò con una espressione di spavento, come se fosse colpevole, e fece per tornare indietro; ma questo gli parve indegno, si voltò di nuovo e, dopo aver tossito, si diresse verso la camera. Le voci tacquero ed egli entrò. 

Anna in vestaglia grigia, coi capelli neri tagliati corti a spazzola sulla testa rotonda, sedeva su di un letto basso. Come sempre, alla vista del marito, l'animazione sul suo viso scomparve; abbassò la testa e guardò inquieta Betsy. Betsy vestiva all'ultima moda, stravagante: con un cappellino che si poggiava chissà dove sulla testa, come un paralume su di una lampada, e con un abito nero-azzurro a strisce marcate trasversali che correvano sulla vita in un senso e sulla gonna nell'altro, sedeva accanto ad Anna, drizzando il busto alto e piatto, e, chinando il capo, accolse con un sorriso ironico Aleksej Aleksandrovic. 

- Ah - disse con sorpresa. - Sono molto contenta che siate in casa. Non vi si vede in nessun posto, e io non vi ho visto dal tempo della malattia di Anna. Ho saputo tutto\ldots{} le vostre premure! Sì, siete un marito straordinario! - disse con un fare significativo e affabile, come se gli concedesse l'ordine cavalleresco della magnanimità per il suo comportamento verso la moglie. 

Aleksej Aleksandrovic si inchinò con freddezza e, baciata la mano alla moglie, le chiese della sua salute. 

- Sto meglio, mi pare - ella disse, evitando lo sguardo di lui. 

- Ma avete un colorito di febbre - egli disse, calcando la parola ``febbre''. 

- Abbiamo parlato troppo io e lei - disse Betsy; - sento che è egoismo da parte mia e me ne vado. - Si alzò, ma Anna, divenuta rossa d'un tratto, le afferrò in fretta il braccio. 

- No, rimanete ancora, vi prego. Ho bisogno di dire a voi\ldots{} no, a voi - si rivolse ad Aleksej Aleksandrovic, e il rosso le coprì il collo e la fronte. - Io non voglio e non posso avere nulla che vi sia nascosto - ella disse. 

Aleksej Aleksandrovic fece scricchiolare le dita e abbassò il capo. 

- Betsy diceva che il conte Vronskij desiderava venire da noi per salutare prima della sua partenza per Taškent. - Ella non guardava il marito e, evidentemente, si affrettava a dire tutto, per quanto questo le fosse penoso. - Io ho detto che non posso riceverlo. 

- Voi avete detto, amica mia, che questo dipendeva da Aleksej Aleksandrovic - corresse Betsy. 

- Ma no, non lo posso ricevere, e questo non avrà nessun\ldots{} - D'un tratto si fermò e guardò interrogativamente il marito (egli non la guardava). - In una parola, non voglio. 

Aleksej Aleksandrovic si mosse e fece per prenderle la mano. Seguendo il primo impulso, ella ritirò la mano da quella di lui, umida, dalle grosse vene gonfie, che cercava la sua, ma, facendo uno sforzo evidente su di sé, gliela strinse. 

- Vi ringrazio molto per la vostra fiducia, ma\ldots{} - disse, sentendo turbamento e irritazione per il fatto che quello che avrebbe potuto decidere facilmente e chiaramente da solo, non lo poteva fare in presenza della principessa Tverskaja che gli si presentava come la personificazione di quella tale forza volgare che avrebbe dovuto dirigere la sua vita agli occhi del mondo e che gli impediva di abbandonarsi al suo sentimento di amore e di perdono. Si fermò, guardando la principessa Tverskaja. 

- Su, addio, tesoro mio - disse Betsy, alzandosi. Baciò Anna e uscì. Aleksej Aleksandrovic la accompagnò. 

- Aleksej Aleksandrovic! Io vi conosco per un uomo veramente generoso - disse Betsy fermandosi nel piccolo salotto e stringendogli in modo particolarmente forte ancora una volta la mano. - Io sono una persona estranea, ma voglio tanto bene a lei e stimo tanto voi che mi permetto darvi un consiglio. Ricevetelo; Aleksej, Vronskij è l'onore personificato, ed egli parte per Taškent. 

- Vi ringrazio, principessa, per il vostro interessamento e per i vostri consigli. Ma la questione se mia moglie possa o non possa ricevere qualcuno la risolverà lei stessa. 

Disse ciò, sollevando per abitudine le sopracciglia e subito pensò che, quali che fossero le sue parole, non poteva esserci dignità nella sua posizione. E questo egli scorse nel sorriso contenuto, cattivo e ironico, col quale Betsy lo guardò dopo questa frase. 

\capitolo{XX}Aleksej Aleksandrovic fece un inchino a Betsy nella sala e andò dalla moglie. Ella s'era sdraiata, ma, sentendo i passi di lui, riprese in fretta la posizione di prima, e lo guardò con spavento. Egli s'accorse che aveva pianto. 

- Ti sono molto grato per la fiducia che hai in me - ripeté sommessamente in russo la frase detta in francese davanti a Betsy, e sedette accanto a lei. Quando egli parlava in russo e le dava del ``tu'', questo ``tu'' irritava irresistibilmente Anna. - E molto grato per la tua decisione. Anch'io ritengo che, poiché parte, non c'è nessun bisogno per il conte Vronskij di venire qua. Del resto\ldots{} 

- Ma l'ho già detto, perché ripeterlo? - l'interruppe Anna con un'irritazione che non fece in tempo a contenere. ``Nessun bisogno - ella pensava - per un uomo di venire a salutare la donna che ama, per la quale voleva morire e s'è rovinato, per la donna che non può vivere senza di lui. Nessuna necessità!''. Strinse le labbra e abbassò gli occhi scintillanti sulle mani di lui dalle vene gonfie che lentamente si fregavano l'una contro l'altra. - Non ne parliamo più - soggiunse, più calma. 

- Io ti ho lasciata libera di decidere da sola questa faccenda, e sono molto contento di vedere\ldots{} - cominciò a dire Aleksej Aleksandrovic. 

- \ldots{}che il mio desiderio si incontra col vostro - finì svelta lei, irritata dal fatto ch'egli parlasse così lentamente, quando ella sapeva già in precedenza quello che avrebbe detto. 

- Già - confermò lui - e la principessa Tverskaja s'intromette del tutto a sproposito nelle più delicate situazioni familiari. Proprio lei\ldots{} 

- Io non credo affatto a quello che si dice sul suo conto - disse in fretta Anna - so soltanto che mi vuole bene sinceramente. 

Aleksej Aleksandrovic sospirò e tacque. Ella giocava nervosamente con le nappine della vestaglia, guardandolo con quel tormentoso senso di repulsione fisica che si rimproverava, ma che non riusciva a vincere. Adesso ella desiderava una cosa sola: essere liberata dalla presenza spiacevole di lui. 

- Ora ho mandato a chiamare il medico - disse Aleksej Aleksandrovic. 

- Io sto bene; perché il medico per me? 

- No, la piccola grida e dicono che la balia abbia poco latte. 

- Perché non mi hai consentito di allattarla, quando io lo volevo tanto? Sempre lo stesso - Aleksej Aleksandrovic capì cosa significava questo ``sempre lo stesso''; - è una bambina, e la fanno morire. - Sonò e ordinò di portare la bambina. - Ho chiesto di allattarla, non me l'hanno permesso, e ora si rimprovera proprio me. 

- Io non rimprovero\ldots{} 

- No, voi rimproverate. Dio mio! Perché non sono morta! - E si mise a singhiozzare. - Perdonami, sono irritata, sono ingiusta - disse rientrando in sé. - Ma va'\ldots{} 

``No, così non può durare'' si disse deciso Aleksej Aleksandrovic, uscendo dalla camera della moglie. 

L'assurdità della sua posizione agli occhi del mondo e l'odio di sua moglie verso di lui e, più di tutto, la prepotenza di quella forza volgare che, pur nell'orientamento del suo spirito, guidava la sua vita pratica e chiedeva l'adempimento delle sue esigenze, il cambiamento, cioè, dei suoi rapporti con la moglie, non gli si erano mai finora presentati innanzi alla mente con tanta evidenza come in quel momento. Vedeva chiaramente che tutto il mondo e la moglie pretendevano da lui qualche cosa, ma che cosa precisamente pretendessero non gli riusciva di capire. Sentiva che, per questo, nell'animo suo si faceva strada un sentimento cattivo che distruggeva la sua calma e tutto il merito della sua azione. Considerava che per Anna sarebbe stato meglio spezzare i rapporti con Vronskij, ma se tutti gli altri trovavano che questo era possibile, era pronto persino ad ammettere di nuovo questi rapporti, pur di non coprire di vergogna i bambini, pur di non esserne privato e di non cambiare la sua posizione. Per quanto questo fosse male, era sempre preferibile a una rottura per cui ella sarebbe rimasta in una posizione senza via d'uscita, umiliante e lui stesso sarebbe stato privato di tutto quello che amava. Ma si sentiva senza forze; sapeva già che tutti erano contro di lui e che non gli avrebbero consentito di fare quello che sembrava così naturale e buono e che l'avrebbero obbligato a fare quello che era un male, ma che credevano si dovesse fare. 

\capitolo{XXI}Betsy non aveva ancora fatto in tempo a uscire dalla sala che Stepan Arkad'ic, venuto or ora da Eliseev, dove erano arrivate le ostriche fresche, le venne incontro sulla porta. 

- Oh, principessa, quale felice incontro! - cominciò a dire. - E io che sono stato da voi. 

- Incontro di un attimo, perché vado via - disse Betsy, sorridendo e infilando un guanto. 

- Aspettate, principessa, a infilarvi il guanto, datemi a baciare la vostra manina. Per nessuna cosa sono così grato al ritorno delle vecchie mode, quanto per il bacio delle mani. - E baciò la mano di Betsy. - Quando ci vedremo allora? 

- Voi non lo meritate - rispose Betsy sorridendo. 

- No, lo merito molto, perché sono diventato più serio. Non solo metto a posto le mie, ma anche le faccende familiari altrui - disse con un'espressione significativa del viso. 

- Ah, sono molto contenta! - rispose Betsy, avendo subito capito che parlava di Anna. E, tornati in sala, stettero in piedi in un angolo. - Egli la farà morire - disse Betsy con un mormorio significativo. - È impossibile, impossibile\ldots{} 

- Sono molto contento che voi pensiate così - disse Stepan Arkad'ic scotendo il capo con un'espressione seria e piena di compassione; - sono venuto per questo a Pietroburgo. 

- Tutta la città ne parla - ella disse. - È una situazione impossibile. Lei si consuma. Egli non capisce che lei è una di quelle donne che non possono scherzare con i loro sentimenti. Una delle due: o egli la porta via con un atto energico, o dà il divorzio. Ma questo stato la soffoca. 

- Sì, sì, proprio\ldots{} - disse Oblonskij, sospirando. - Io perciò sono venuto. Cioè non proprio per questo\ldots{} Mi hanno fatto ciambellano, via, bisogna pure ringraziare. Ma, soprattutto, bisogna accomodare questa faccenda. 

- Che Iddio vi aiuti - disse Betsy. 

Accompagnata la principessa Betsy fino all'ingresso, baciatale ancora una volta la mano più su del guanto, là dove batte il polso, e, lanciatele ancora delle amenità tanto poco convenienti ch'ella non sapeva più se arrabbiarsi o riderne, Stepan Arkad'ic entrò dalla sorella. La trovò in lacrime. 

Malgrado la disposizione d'animo sprizzante allegria in cui si trovava, Stepan Arkad'ic passò subito con naturalezza a quel tono compassionevole, poeticamente eccitato che si confaceva all'umore di lei. Le chiese della sua salute e come avesse passato la mattina. 

- Molto, molto male. E così il giorno e la mattina e tutti i giorni passati e futuri - ella disse. 

- Mi pare che tu ti abbandoni alla tetraggine. Bisogna scuotersi; bisogna guardare in faccia la vita. Lo so che è penoso, ma\ldots{} 

- Ho sentito che le donne amano gli uomini anche per i loro vizi - cominciò improvvisamente Anna - ma io lo odio per la sua virtù. Io non posso vivere con lui. Intendimi, il suo aspetto fisico agisce su di me, esco fuori di me. Non posso, non posso vivere con lui. Che fare mai? Ero infelice e pensavo non si potesse essere ancora più infelice di così, ma questa orribile posizione nella quale ora sono, non potevo immaginarla. Lo credi che, pur sapendo che egli è un uomo buono, eccellente, che io non valgo una sua unghia, tuttavia, io lo odio? Lo odio proprio per la sua generosità. Non mi resta nulla, tranne\ldots{} 

Voleva dire la morte, ma Stepan Arkad'ic non le dette il tempo di finire. 

- Sei malata ed eccitata - egli disse, - credimi, esageri terribilmente. Qui non c'è nulla di così spaventoso. 

E Stepan Arkad'ic sorrise. Nessuno al posto di Stepan Arkad'ic dinanzi a una simile disperazione, si sarebbe permesso di sorridere (il sorriso sarebbe parso volgare), ma nel sorriso di lui v'era tanta bontà e quasi una tenerezza femminile, che invece di offendere, raddolciva e calmava. 

I suoi discorsi, calmi e rasserenanti, e i suoi sorrisi agivano in modo da ammorbidire come olio di mandorla. E Anna ne sentì subito l'effetto. 

- No, Stiva - ella disse. - Sono perduta, sono perduta! Peggio che perduta. Ancora non sono perduta, non posso dire che tutto sia finito; al contrario, sento che non è finito. Sono come una corda tesa che deve spezzarsi. Ma ancora non è finito\ldots{} e finirà in modo orribile. 

- Ma no, si può adagio adagio allentare la corda. Non vi è situazione dalla quale non si possa uscire. 

- Ho pensato e ho pensato. Soltanto una\ldots{} 

Di nuovo egli capì dal suo sguardo spaventato che quest'unica via d'uscita, per lei, era la morte, e non le permise di finire. 

- Nient'affatto - disse. - Permetti. Tu non puoi vedere la situazione come me. Permettimi di dire apertamente la mia opinione. - Di nuovo egli sorrise timido col suo sorriso all'olio di mandorla. - Comincio dal principio; ti sei sposata con un uomo che aveva vent'anni più di te. Ti sei sposata senz'amore o senza conoscere l'amore. Questo è stato un errore, ammettiamolo. 

- Terribile errore! - disse Anna. 

- Ma io ripeto: quel che è fatto è fatto. Dopo hai avuto, diciamo pure, la sventura di amare chi non era tuo marito. È una sventura, ma anche questo è un fatto compiuto. E tuo marito l'ha riconosciuto come tale e ti ha perdonato. - Egli si fermava dopo ogni frase, aspettando le obiezioni di lei, ma lei non rispondeva nulla. - Così è. Adesso la questione è questa: puoi continuare a vivere con tuo marito? Lo desideri? Lo desidera lui? 

- Io non so nulla, nulla. 

- Ma tu stessa hai detto che non puoi sopportarlo. 

- No, non l'ho detto. Lo ritratto. Io non so nulla e non capisco nulla. 

- Sì, ma permetti\ldots{} 

- Tu non puoi capire. Io sento che precipito con la testa in giù in un abisso, ma che non devo salvarmi, e non posso! 

- Non è niente, noi stenderemo qualcosa sotto di te e ti afferreremo. Ti capisco, capisco che non ti senti di assumere la responsabilità di esprimere il tuo desiderio, il tuo sentimento. 

- Io non desidero nulla, nulla\ldots{} solo che tutto finisca. 

- Ma egli lo vede questo e lo sa. E credi forse che non ne senta, quanto te, tutta la pena? Tu ti tormenti, lui si tormenta e che ne viene fuori? Mentre il divorzio risolverebbe tutto - disse Stepan Arkad'ic, manifestando non senza sforzo il proprio pensiero preminente e guardandola con intenzione. 

Ella non rispose nulla e scosse negativamente il capo dai capelli corti. Ma, dall'espressione del viso che improvvisamente s'era acceso della bellezza d'un tempo, egli capiva ch'ella rifiutava tale soluzione solo perché le pareva una felicità irraggiungibile. 

- Mi fate tanta tanta pena! E come sarei felice se potessi compier tutto questo! - disse Stepan Arkad'ic sorridendo ormai più coraggiosamente. - Non dire, non dire nulla. Se Dio mi concedesse solo di parlare così come sento. Andrò da lui. 

Anna con gli occhi pensosi e splendenti lo guardò, e non disse nulla. 

\capitolo{XXII}Stepan Arkad'ic, con quell'aria alquanto solenne con cui prendeva posto nella poltrona presidenziale del suo ufficio, entrò nello studio di Aleksej Aleksandrovic. Aleksej Aleksandrovic camminava con le mani dietro la schiena su e giù per la stanza e pensava alle stesse cose di cui Stepan Arkad'ic aveva parlato con la moglie. 

- Non ti disturbo? - disse Stepan Arkad'ic, provando un insolito senso di smarrimento alla vista del cognato. Per nascondere il turbamento, cacciò fuori un portasigarette, da poco acquistato, munito di un nuovo sistema di apertura e, annusandone la pelle, ne trasse fuori una sigaretta. 

- No. Ti occorre qualcosa? - rispose di malavoglia Aleksej Aleksandrovic. 

- Sì, volevo\ldots{} ho bisogno\ldots{} già, ho bisogno di parlarti - disse Stepan Arkad'ic, sentendo con sorpresa un'insolita timidezza. 

E ciò era così insolito e strano per lui che Stepan Arkad'ic non volle pensare che potesse essere dovuto alla voce della coscienza che gli presentava come male quello che stava per fare. Fece uno sforzo su di sé e vinse la timidezza che lo aveva pervaso. 

- Spero che tu creda al mio affetto per mia sorella e al sincero legame e alla stima che ho per te - disse, arrossendo. 

Aleksej Aleksandrovic si fermò e non rispose nulla, ma il suo volto colpì Stepan Arkad'ic per l'espressione di vittima sottomessa. 

- Avevo intenzione, volevo parlare di mia sorella, della vostra reciproca situazione - disse Stepan Arkad'ic, lottando ancora con l'insolita timidezza. 

Aleksej Aleksandrovic sorrise triste, guardò il cognato e, senza rispondere, si accostò al tavolo, ne trasse fuori una lettera cominciata e la dette al cognato. 

- Penso continuamente alla stessa cosa. Ed ecco quello che avevo cominciato a scrivere, ritenendo più opportuno parlarle per lettera, evitando così che la mia presenza la irriti - disse, porgendo la lettera. 

Stepan Arkad'ic prese la lettera, e con stupore e perplessità guardò gli occhi appannati, immobilmente fissi su di lui, e cominciò a leggere. 

\begin{quote}
``Vedo che la mia presenza vi è di peso. Per quanto possa essere doloroso per me convincermene, vedo che è così e che non può essere diversamente. Io non vi accuso, e Dio mi è testimone che, da quando vi ho visto durante la vostra malattia, ho deciso con tutta l'anima di dimenticare tutto quello che era stato tra di noi e di cominciare una nuova vita. Io non mi pento e non mi pentirò mai di quello che ho fatto; desideravo una cosa sola: il vostro bene, il bene della vostra anima, e adesso vedo che questo non l'ho raggiunto. Ditemi voi stessa che cosa può dare pace e felicità all'anima vostra. Io mi rimetto alla vostra volontà e al vostro senso di giustizia''.
\end{quote} 

Stepan Arkad'ic restituì la lettera e con la stessa perplessità di prima continuò a guardare il cognato senza sapere che dire. Questo silenzio era per entrambi così penoso che le labbra di Stepan Arkad'ic ebbero un tremito morboso, mentre taceva senza levar gli occhi dal viso di Karenin. 

- Ecco quello che io volevo dirle - disse Aleksej Aleksandrovic, voltandosi da un'altra parte. 

- Sì, sì\ldots{} - disse Stepan Arkad'ic, senza avere la forza di rispondere, giacché le lacrime gli venivano alla gola. - Sì, sì, vi capisco - pronunciò alla fine. 

- Io desidero sapere quello ch'ella vuole - disse Aleksej Aleksandrovic. 

- Penso ch'ella stessa non intenda la propria situazione. Ella non può esserne l'arbitra - diceva Stepan Arkad'ic, riprendendosi. - È schiacciata, proprio schiacciata dalla tua generosità. Se leggerà questa lettera non avrà la forza di nulla, abbasserà solo più in giù il capo. 

- Già, ma che fare, dunque, in un caso simile? Come intuire, come conoscere i suoi desideri? 

- Se mi permetti di dire la mia opinione, io penso che dipenda da te indicare quella misura che ritieni necessaria per far cessare questo stato di cose. 

- Allora tu ritieni che è necessario far cessare? - lo interruppe Aleksej Aleksandrovic. - Ma come? - aggiunse, dopo aver fatto con le mani un insolito gesto davanti agli occhi. - Non vedo possibile nessuna via d'uscita. 

- In qualsiasi situazione c'è sempre una via d'uscita - disse, alzandosi e animandosi, Stepan Arkad'ic. - Vi è stato un momento quando tu volevi rompere\ldots{} Se adesso ti convincerai che non potete formare l'uno la felicità dell'altra\ldots{} 

- La felicità si può intendere in vari modi. Ma poniamo che io sia d'accordo, che non voglia nulla. Quale via d'uscita dalla nostra posizione? 

- Se tu vuoi la mia opinione - disse Stepan Arkad'ic con lo stesso tenero, dolce sorriso all'olio di mandorla, col quale aveva parlato ad Anna. Quel buon sorriso era così suadente che Aleksej Aleksandrovic senza volere, sentendo la propria debolezza e sottomettendovisi, era pronto a credere quello che avrebbe detto Stepan Arkad'ic. - Ella non lo dirà mai. Ma una cosa sola può desiderare - continuò Stepan Arkad'ic - questa: che cessino i rapporti e i ricordi ad essi collegati. Secondo me, nella vostra situazione, è indispensabile la chiarificazione di nuovi rapporti. E questi rapporti possono stabilirsi solo con la libertà delle due parti. 

- Il divorzio - interruppe con avversione Aleksej Aleksandrovic. 

- Sì, io ritengo che sia necessario il divorzio. Sì, il divorzio - ripeté, arrossendo, Stepan Arkad'ic. - Sotto tutti gli spetti è la via d'uscita più ragionevole per coniugi che siano in rapporti come i vostri. Che fare mai se i coniugi sentono che per loro la vita in comune è impossibile? Questo può sempre accadere. - Aleksej Aleksandrovic sospirò pesantemente e chiuse gli occhi. - Qui c'è solo una considerazione da fare: desidera uno dei coniugi contrarre altro matrimonio? Se no, la cosa è molto semplice - disse Stepan Arkad'ic liberandosi sempre più dal disagio. 

Aleksej Aleksandrovic, corrugato il viso per l'emozione, mormorò qualcosa fra sé e sé e non rispose nulla. Tutto quello che a Stepan Arkad'ic pareva così semplice, Aleksej Aleksandrovic lo aveva pensato mille volte. E tutto questo gli pareva, non solo tutt'altro che semplice, ma del tutto impossibile. Il divorzio, i cui particolari già conosceva, gli pareva impossibile perché il sentimento di dignità personale e il rispetto per la religione non gli consentivano di assumere l'accusa fittizia di adulterio e ancora meno di ammettere che la moglie da lui perdonata e amata fosse riconosciuta colpevole e perduta. Il divorzio si presentava, inoltre, impossibile per altre, ancora più importanti ragioni. 

Che ne sarebbe stato del figlio in caso di divorzio? Lasciarlo con la madre non era possibile. La madre divorziata avrebbe avuto una sua famiglia illegittima nella quale la situazione di figliastro e la sua educazione sarebbero state, con ogni probabilità, poco buone. Tenerlo con sé? Questo, lo sapeva, sarebbe stata una cattiva vendetta da parte sua e non voleva. Inoltre, il divorzio sembrava ad Aleksej Aleksandrovic la cosa più inopportuna, perché, acconsentendo al divorzio, egli avrebbe proprio con questo fatto la rovina di Anna. Gli era rimasta nell'animo la frase detta da Dar'ja Aleksandrovna, a Mosca, che, decidendosi al divorzio, avrebbe pensato soltanto a sé, ma che avrebbe rovinato lei irrimediabilmente. E quella frase, congiunta nel suo pensiero, al perdono, all'affetto per i suoi bambini, egli la intendeva, ora, a modo suo. Consentire al divorzio, dare a lei la libertà significava troncare l'ultimo suo legame con la vita dei bambini che amava e privar lei dell'ultimo sostegno sulla via del bene, e gettarla nella rovina. S'ella fosse diventata una moglie divorziata, si sarebbe unita, egli lo sapeva, a Vronskij, e questo legame era illegale e colpevole perché per la moglie, secondo la legge della Chiesa, non può esservi altro matrimonio finché è vivo il marito. ``Si unirà a lui e, dopo un anno o due, egli l'abbandonerà e lei contrarrà un nuovo legame - pensava Aleksej Aleksandrovic. - E io, avendo acconsentito a un divorzio al di fuori della legge, sarò il responsabile della sua perdizione''. Aveva riflettuto a questo centinaia di volte ed era convinto che la soluzione del divorzio, non solo non era molto semplice come diceva il cognato, ma del tutto inaccettabile. Non credeva neppure a una parola di quello che diceva Stepan Arkad'ic, per ogni sua parola aveva migliaia di obiezioni, ma l'ascoltava, perché sentiva che nelle parole di lui trovava espressione quella tale prepotente forza volgare che guidava la sua vita e alla quale avrebbe dovuto sottomettersi. 

- La questione è solo nello stabilire a quali condizioni tu acconsentirai al divorzio. Ella non vuole nulla, e non osa chiederti nulla, si rimette completamente alla tua generosità. 

``Dio mio! Dio mio! Perché?'' pensava Aleksej Aleksandrovic ricordando i particolari di quel tale divorzio nel quale il marito assumeva la colpa dell'adulterio, e con quello stesso gesto con quale Vronskij si era coperto il viso, egli coprì con le mani il suo, sopraffatto dalla vergogna. 

- Tu sei agitato, capisco. Ma se rifletterai\ldots{} 

``E a colui che avrà percosso la tua guancia destra, tendi la sinistra; e a colui che ti avrà tolto il mantello da' la tunica'' diceva a se stesso Aleksej Aleksandrovic. 

- Sì, sì - gridò con voce stridula - prenderò su di me il disonore, darò anche mio figlio, ma\ldots{} non sarebbe meglio? Del resto, fa' quello che vuoi\ldots{} 

E voltatosi in modo che il cognato non potesse vederlo, sedette su di una seggiola accanto alla finestra. Quanta amarezza, quanta vergogna! ma insieme con quest'amarezza e con questa vergogna erano in lui la gioia e la commozione che gli venivano dall'altezza della propria umiltà. 

Stepan Arkad'ic era commosso. Tacque per un po'. 

- Aleksej Aleksandrovic, credimi, ella apprezzerà la tua generosità - egli disse. - Ma si vede, era la volontà di Dio - soggiunse e, detto questo, sentì d'essere stato idiota, e trattenne a stento un sorriso sulla propria idiozia. 

Aleksej Aleksandrovic voleva rispondere qualcosa, ma le lacrime glielo impedirono. 

- È una sventura del destino, ci si deve rassegnare. Riconosco questa sventura come un fatto compiuto e cerco di venire in aiuto a lei e a te - disse Stepan Arkad'ic . 

Quando Stepan Arkad'ic uscì dalla stanza del cognato, era commosso, ma questo non gl'impedì d'essere soddisfatto d'aver condotto a termine felicemente la faccenda, giacché era convinto che Aleksej Aleksandrovic non avrebbe ritrattato le sue parole. A questa soddisfazione si frammischiava l'idea venutagli in mente che, ad affare concluso, avrebbe potuto chiedere alla moglie e agli intimi:``Che differenza c'è fra me e un monarca? Il monarca fa il cambio della guardia e di questo nessuno si avvantaggia ed io, invece, ho portato a termine un divorzio e tre persone ne trarranno vantaggio''. Oppure: ``Che somiglianza c'è fra un monarca e me? Quando\ldots{} Ma\ldots{} ci penserò meglio'' disse fra sé con un sorriso. 

\capitolo{XXIII}La ferita di Vronskij era stata pericolosa, pur avendo risparmiato il cuore. E per alcuni giorni fu tra la vita e la morte. Quando per la prima volta si sentì in condizioni di parlare, nella camera c'era solamente Varja, la moglie del fratello. 

- Varja - egli disse, guardandola - mi sono sparato addosso inavvertitamente. E, ti prego, cerca di non parlarne; ad ogni modo di' così a tutti. Altrimenti, è troppo sciocco! 

Senza rispondere alle sue parole, Varja si chinò su di lui e ne scrutò il viso. Gli occhi erano limpidi, non più febbrili, ma la loro espressione era dura. 

- Via, sia lodato Iddio! - ella disse. - Non ti fa male? 

- Un poco, qui. - E indicò il petto. 

- Allora da' qua, che ti fascio. 

Stringendo in silenzio gli zigomi larghi, Vronskij la guardava mentre ella lo fasciava. Quando ebbe finito, disse: 

- Non sono in stato di delirio; ti prego, fa' in modo che non corra voce che mi sono sparato di proposito. 

- Nessuno lo dice neppure. Soltanto spero che non sparerai più inavvertitamente - ella disse con un sorriso interrogativo. 

- Così dev'essere, non lo farò più, ma sarebbe stato meglio\ldots{} 

E sorrise cupo. 

Malgrado queste parole e il sorriso che avevano spaventato Varja, quando passò l'infiammazione e cominciò a rimettersi, sentì d'essersi completamente liberato di una parte del suo dolore. Era come se, con quell'atto, si fosse liberato della vergogna e dell'umiliazione che aveva provato prima. Ora poteva tranquillamente pensare al Aleksej Aleksandrovic. Riconosceva tutta la generosità di lui, ma non sentiva più la propria umiliazione. Inoltre s'era rimesso di nuovo sulla carreggiata della sua vita precedente. Vedeva che poteva guardare senza vergogna la gente negli occhi e vivere, facendosi guidare dalle sue abitudini. L'unica cosa che non poteva strappare dal proprio cuore, malgrado lottasse senza tregua con quel sentimento, era il rimpianto, che giungeva fino alla disperazione, di averla perduta per sempre. Che egli adesso, riscattata davanti al marito la propria colpa, dovesse rinunciare a lei e non porsi mai più fra il rimorso di lei e il marito, era cosa fermamente decisa nel suo cuore; ma dal cuore non poteva strappare il rimpianto d'aver perduto l'amore di lei, non poteva cancellare dalla memoria quei momenti di felicità che aveva conosciuto con lei e che aveva così poco apprezzato allora, mentre adesso lo perseguitavano con tutto il loro fascino. 

Serpuchovskoj aveva escogitato per lui la nomina a Taškent, e Vronskij, senza la minima esitazione, aveva acconsentito a questa proposta. Ma quanto più si avvicinava il momento della partenza, tanto più penoso si faceva per lui il sacrificio che compiva perché lo considerava doveroso. 

La ferita era cicatrizzata ed egli già usciva per fare i preparativi per la partenza per Taškent. 

``Vederla una volta e poi scomparire, morire'' pensava e, facendo le visite di congedo, espresse questo pensiero a Betsy. Con questo incarico Betsy era andata da Anna e aveva riferito la risposta negativa. 

``Tanto meglio - aveva pensato Vronskij, dopo aver ricevuto questa notizia. - Era una debolezza che avrebbe rovinato le mie ultime risorse''. 

Il giorno dopo la stessa Betsy andò da lui la mattina e gli disse di aver ricevuto per mezzo di Oblonskij la sicura notizia che Aleksej Aleksandrovic consentiva al divorzio e che perciò egli poteva vederla. 

Senza curarsi neppure di accompagnare a casa Betsy, dimentico di tutte le proprie decisioni, senza chiedere quando si poteva, e dove fosse il marito, Vronskij andò subito dai Karenin. Corse per la scala, senza veder nulla e nessuno e, a passo veloce, entrò, trattenendosi appena dal correre, nella camera di lei. E senza pensare e senza notare se ci fosse qualcuno in camera, abbracciò e coprì di baci il viso, le mani, il collo di lei. 

Anna si era preparata a quest'incontro, aveva pensato a quello che gli avrebbe detto, ma non ebbe il tempo di dire nulla: la passione l'afferrò. Voleva calmar lui, calmare se stessa, ma ormai era troppo tardi. L'eccitazione di lui le si era comunicata. Le sue labbra tremavano tanto che a lungo non poté articolar parola. 

- Sì, ti sei impadronito di me e io sono tua - pronunciò alla fine, stringendosi al petto le mani di lui. 

- Così doveva essere - egli disse. - Finché siamo vivi, così deve essere. È questo che io so, adesso. 

- È vero - diceva lei, impallidendo sempre più e abbracciandogli la testa. - Tuttavia c'è qualcosa di tremendo in questo, dopo quello che è stato. 

- Tutto passerà, tutto passerà: noi saremo tanto felici! Il nostro amore, se potesse, diverrebbe più forte, perché in esso c'è qualcosa di tremendo - egli disse, alzando la testa e scoprendo nel sorriso i suoi denti forti. 

Ella non poté non rispondere con un sorriso, non alle parole, ma agli occhi innamorati di lui. Gli prese una mano e si carezzò con questa le guance divenute fredde e i capelli tagliati. 

- Non ti riconosco con questi capelli corti. Così sei più bella. Sembri un ragazzo. Ma come sei pallida! 

- Sì, sono molto debole - ella disse, arrossendo. E le sue labbra di nuovo tremarono. 

- Andremo in Italia, ti rimetterai - egli disse. 

- Ma è possibile che noi siamo come marito e moglie, soli, con una famiglia nostra, io e tu? - disse lei, guardandolo da vicino negli occhi. 

- Mi stupiva soltanto, come una volta potesse essere diversamente. 

- Stiva dice che lui acconsente a tutto, ma io non posso accettare la sua generosità - ella disse, guardando pensosa al di là del viso di Vronskij. - Non voglio il divorzio; ora per me è lo stesso. Solo non so cosa deciderà per Serëza. 

Egli non riusciva in nessun modo a capire come ella potesse, nel momento del loro incontro, ricordarsi del figlio, e pensare al divorzio. Non era forse lo stesso? 

- Non parlare di questo, non pensare - disse, rigirando le mani di lei nella sua e cercando di attirarne l'attenzione; ma lei non lo guardava. 

- Ah, perché non sono morta, sarebbe stato meglio! - disse e, senza singhiozzi, le lacrime le colarono giù per le guance; ma cercava di sorridere per non addolorarlo. 

Secondo le idee che poco prima erano nella mente di Vronskij, rinunziare alla lusinghiera e pericolosa nomina a Taškent, sarebbe stato disonorevole e impossibile. Ma adesso, senza pensarci un attimo, vi rinunziò e, avendo notato che i suoi superiori disapprovavano il gesto, diede immediatamente le dimissioni. 

Dopo un mese, Aleksej Aleksandrovic rimase solo col figlio nel suo appartamento, e Anna partì per l'estero con Vronskij, senza aver ottenuto il divorzio e avendovi decisamente rinunciato. 
