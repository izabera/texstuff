\documentclass{article}
\usepackage[a4paper,margin=2cm,marginparwidth=2cm,marginparsep=0.2cm]{geometry}
\usepackage{vwcol}
\usepackage{mathpazo}
\usepackage[hyperfootnotes=false,pdftex]{hyperref}
\hypersetup{
    pdftitle={A Silvia},
    pdfauthor={Marco Giannaccari},%typeset by iza :P
    pdfsubject={Letteratura},
    pdfkeywords={Poesia A Silvia Giacomo Leopardi},
    pdfproducer={LaTeX},
    pdfcreator={pdflatex},
    pdfstartview={XYZ null null 1.00},
    pdfremotestartview={XYZ null null 1.00},
    pdfdisplaydoctitle={True}
}
\title{A Silvia}
\author{}
\date{}

\usepackage[italian]{babel}
\usepackage[utf8]{inputenc}
\usepackage{marginnote}

\usepackage{xcolor}
\definecolor{grigio}{RGB}{90,90,90}

\newcommand{\notaparentesi}[1]{{\color{grigio}\textsf{(#1)}}}
\newcommand{\notamargine}[1]{\marginnote{\color{grigio}\small#1}}
\renewcommand\thefootnote{\textcolor{grigio}{\arabic{footnote}}}

\begin{document}
\maketitle
\thispagestyle{empty}
\begin{vwcol}[widths={0.43,0.57},justify=flush,rule=0pt,indent=0em]
\reversemarginpar
Silvia, rimembri ancora\\
quel tempo della tua vita mortale,\\
quando beltà splendea\\
negli occhi tuoi ridenti e fuggitivi,\\
\notamargine{5.}
e tu, lieta e pensosa, il limitare\\
di gioventù salivi?\\

Sonavan le quiete\\
stanze, e le vie d'intorno,\\
al tuo perpetuo canto,\\
\notamargine{10.}
allor che all'opre femminili intenta\\
sedevi, assai contenta.\\
di quel vago avvenir che in mente avevi.\\
Era il maggio odoroso: e tu solevi\\
così menare il giorno.\\

\notamargine{15.}
Io gli studi leggiadri\\
talor lasciando e le sudate carte,\\
ove il tempo mio primo\\
e di me si spendea la miglior parte,\\
d’in su i veroni del paterno ostello\\
\notamargine{20.}
porgea gli orecchi al suon della tua voce,\\
ed alla man veloce\\
che percorrea la faticosa tela.\\
Miravo il ciel sereno,\\
le vie dorate e gli orti,\\
\notamargine{25.}
e quinci il mar da lungi, e quindi il monte.\\
Lingua mortal non dice\\
quel ch'io sentiva in seno.\\

Che pensieri soavi,\\
che speranze, che cori, o Silvia mia!\\
\notamargine{30.}
Quale allor ci apparia\\
la vita umana e il fato!\\
Quando sovviemmi di cotanta speme,\\
un affetto mi preme\\
acerbo e sconsolato,\\
\notamargine{35.}
e tornami a doler di mia sventura.\\
O natura, o natura,\\
perché non rendi poi\\
quel che prometti allor? Perché di tanto\\
inganni i figli tuoi?\\



Silvia, ricordi ancora\\
il tempo in cui eri viva\\
quando bellezza spendeva\\
nei tuoi occhi allegri e schivi,\\
e tu, lieta e pensosa \notaparentesi{ossimoro} oltrepassavi\\
il limite della gioventù? \notaparentesi{similitudine}\footnote[1]{}\\

Risuonava la casa quieta\\
e le strade che la circondano\\
al tuo canto eterno,\\
quando eri affaccendata nei lavori femminili \notaparentesi{anastrofe}\\
ti felicitavi\\
del tuo futuro di cui avevi una vaga idea.\\
era Maggio ed eri solita\\
trascorrere le tue giornate in tal modo.\\

Interrompevo a volte i piacevoli studi\\
lasciando le impegnative carte \notaparentesi{metonimia},\\
dove spendevo la mia giovinezza\\
e la parte migliore di me,\\
da sopra i balconi della casa paterna\\
tendevo le orecchie al suono della tua voce\\
e a quello della tua mano veloce\\
che percorreva la tela \notaparentesi{metonimia}.\\
Guardavo il cielo sereno,\\
le vie soleggiate soleggiate e i giardini,\\
e da una parte il mare da me lontano, dall'altra la montagna.\\
Nessun uomo può dire\\
quel che io provavo dentro.\\

Che pensieri soavi,\\
che speranze, che sentimenti\footnote[2]{}, o mia Silvia!\\
Come ci apparivano, in quei tempi,\\
la vita umana e il destino!\\
Quando ricordo le nostre speranze\\
l'angoscia mi sopprime,\\
crudele e inconsolabile,\\
e ritorna il dolore per le mie sventure.\\
O natura, o natura,\\
perché non mantieni\\
le promesse fatte in passato? Perché\\
inganni i tuoi figli in questo modo?
\end{vwcol}
\footnotetext[1]{L'autore utilizza la figura del cuore (``cori'') per indicare i sentimenti, poiché il cuore stesso ne è la sede.}
\footnotetext[2]{La vita di Silvia viene vista dal Leopardi alla stregua di un sentiero, sul quale la ragazza attraversa le diverse fasi della propria esistenza.}

\end{document}