\documentclass[12pt]{article}
%\usepackage{fontspec}
\usepackage{amssymb, amsfonts, amsmath, amsthm}
\usepackage[italian]{babel}
\usepackage[utf8]{inputenc}
\usepackage{graphicx}
\usepackage{mathpazo}

\newcommand\rea{\mathbb R}
\newcommand\ins[1]{\mathbf #1}
\newcommand\pom{di Pompeiu}
\newcommand\inte[3]{\int\limits_{#1}^{#2}\!#3(x)\text dx}
\newcommand\intes[2]{\int\limits_{#1}\!#2\text dx}

\newtheorem*{prop*}{Proposizione}
\newtheorem{prop}{Proposizione}
\newtheorem{teo}{Teorema}
\theoremstyle{definition}
\newtheorem*{defn}{Definizione}
\newtheorem{esem}{Esempio}

%\setmainfont[Ligatures=TeX]{Linux Libertine O}
%\setdefaultlanguage{italian}

\title{Sottoinsiemi \pom{} di $\rea$}% di misura infinita}
\author{Isabella Bosia}
\date{}

\begin{document}
\maketitle
%()&
In questo documento, $\Sigma(\ins A)$ è l'insieme delle traslazioni e delle riflessioni $\sigma$ di $\ins A$. $\ins A+t$ indicherà l'insieme $\ins A$ traslato di $t$. 
\begin{defn}$\ins A\subseteq\rea$ è \pom{} se data $f:\rea\to\rea$ continua tale che $\intes {\sigma(\ins A)} f=k\in\rea$ non dipende da  $\sigma\in\Sigma$ si ha che $f$ è costante.
\end{defn}

Sia $\ins A$ un insieme di misura infinita, sia $\sigma(\ins A)$ una sua traslazione di $t$ tale che $\mu(\ins A\cap\sigma(\ins A))=0$ (dove $\mu$ è la misura di Lebesgue) e sia $\ins B=\ins A\cup\sigma(\ins A)$.
\begin{teo}\label{teo:B-pom-sse-A-pom}$\ins B$ è \pom{} $\iff$ $\ins A$ è \pom{}.
\end{teo}
\begin{proof}Dimostriamo prima che $\ins B$ è \pom{}$\implies\ins A$ è \pom{}.

Supponiamo che $\ins A$ non sia \pom{}. Allora $\exists g$ non costante tale che $\intes {\sigma(\ins A)}{g(x)}=k$ non dipende da $\sigma$.

Poiché $\ins A$ e $\ins A+t$ sono quasi disgiunti, abbiamo che
\begin{equation}\label{eq:quasi-disgiunti}
\intes{\ins B}{g(x)}=\intes{\ins A}{g(x)}+\intes{\ins A+t} {g(x)}=2k
\end{equation}
Quindi $\exists g$ non costante con integrale costante sulle traslazioni di $\ins B$, il che contraddice il fatto che $\ins B$ è \pom{}.

Si noti che questa implicazione non è limitata al caso in cui $\ins A$ sia di misura infinita.

Vediamo ora l'implicazione inversa, cioè $\ins A$ è \pom{}$\implies\ins B$ è \pom{}.

Poiché $\ins A$ è \pom{} abbiamo 
\begin{equation}\label{eq:A-pom}
F(t)=\intes{\ins A}{f(x+t)}=k\quad\forall t ,
\end{equation}
quindi \[\intes{\ins B}{f(x)}=\intes{\ins A}{f(x)} + \intes{\ins A+t}{f(x)}=F(0)+F(t)=2k . \]

Prendiamo una $g$ continua per cui valga \begin{equation}\label{eq:B-non-pom}
\intes {\ins B+t}{g(x)}=q\quad\forall t .
\end{equation}

Se $g$ non è costante, allora $\exists t'$ tale che $\intes{\ins A+t'}{f(x)}=q'\not=\frac q2$.% perché $\ins A$ è un \pom{}

Sia $\ins{A'}=\ins A+t'$. Allora poiché la somma è costante, \[\intes{\ins{A'}+t}{g(x)}=q-q' . \]
Senza perdita di generalità, assumiamo $q=0$. %wlog perché possiamo traslare la funzione in alto o in basso
\[\intes{\ins{A'}}{g(x)}+\intes{\ins{A'}+t}{g(x)}=0\] per cui
\[G(0)+G(t)=0\quad\quad\quad\text{con }G(t)=\intes {A'} {g(x+t)} . \]
\[\intes{\ins B}{g(x)}=q\text{ (costante)}\implies G(0+y)+G(t+y)=0\quad\forall y\]
dunque $G$ è periodica di periodo $2t$.
\begin{equation}\label{eq:g-period}
\begin{split}
\intes{\ins B+y}{g(x)}=\intes{\ins{A'}+y}{g(x)}+\intes{\ins{A'}+t+y}{g(x)}=\\
\intes{\ins{A'}}{g(x+y)+g(x+t+y)}=0\quad\forall y
\end{split}\end{equation}
Per la \eqref{eq:g-period} abbiamo che $g(x)+g(x+t)$ è periodica.

Se $g(x)=k$, allora $g(x+t)=g(x)-k$, perciò $g(x+t)+g(x+t+t)=k$, dunque anche $g$ è periodica.

Per concludere la dimostrazione, l'unica funzione periodica con integrale convergente su un insieme di misura infinita e su tutte le sue traslazioni è la costante nulla, il che contraddice la \eqref{eq:B-non-pom}.
\end{proof}
Ecco alcuni esempi di applicazioni del teorema.
\begin{esem}\label{esem:semiretta}
$(a,+\infty)$ è \pom{} per ogni $a\in\rea$.
\end{esem}
\begin{proof}
Per ipotesi, abbiamo \[\displaystyle\inte a\infty f=\inte{a+t}\infty f\] per ogni $t\in\rea$
e dunque $\displaystyle\inte a{a+t}f=0$ per ogni $t\in\rea$.

%L'integrale non aumenta quindi blabla $f$ è costante.

%{\tiny
Poiché $f$ è continua, se è positiva in un certo punto $x_0$
è positiva in tutto un intervallo $(x_0-\varepsilon,x_0+\varepsilon)$, quindi
l'integrale su quell'intervallo non è $0$.
Dunque deve essere identicamente nulla su tutto $\rea$.%}
\end{proof}

\begin{esem}Insiemi come $\displaystyle\ins X=\bigcup_{k\in\mathbb Z}(a+2kb,b+2kb)$ non sono \pom{}.
\end{esem}
\begin{proof}Ovviamente esiste una traslazione $\sigma$ tramite cui due copie di $\ins X$ ricoprono quasi tutto $\rea$.
Un semplice controesempio è:
\begin{equation*}
f(x)=\left\{\begin{matrix}
0 & x<0 \\ 
x & x\in(0,1]\\
-x+2 & x\in(1,3]\\
x-4 & x\in(3,4]\\
0 & x>4
\end{matrix}\right.
\end{equation*}
con $a=0$, $b=1$ e $k=1$, per cui l'integrale è uguale a $0$ indipendentemente dalla traslazione $\sigma$.
\begin{center}
\includegraphics*[scale=0.8]{ex1.pdf}
\end{center}
\end{proof}

\begin{esem}\label{es:semiretta-tratti}
Insiemi come $\ins Y=\ins X\cap(a,\infty)$ sono \pom{}.
\end{esem}
\begin{proof}
Esiste una traslazione $\sigma$ tramite cui due copie di $\ins Y$ formano una semiretta, per cui la tesi segue dall'esempio \ref{esem:semiretta}.
\end{proof}

\begin{esem}Se $a>0$ allora $\ins X=(-\infty,a)\cup(2a,3a)$ è \pom{}.
\end{esem}
\begin{proof}Poiché la lunghezza dell'intervallo $(a,2a)$ è la stessa dell'intervallo $(2a,3a)$, possiamo dividere $\ins X$ in due sottoinsiemi che sono l'uno la copia dell'altro, ed essi sono \pom{} come mostrato nell'esempio \ref{es:semiretta-tratti}. Il discorso si può estendere anche a insiemi come $(-\infty,a)\bigcup_{n<m}(2na,(2n+1)a)$ per $m\in\mathbb N$.
\end{proof}

Alcune osservazioni non correlate al teorema \ref{teo:B-pom-sse-A-pom} ma al problema in generale.
\begin{esem}Esiste una funzione non periodica con integrale indipendente da traslazioni su un compatto:
\begin{equation*}
f(x)=\cos\left(2\pi x\right)+\cos\left(\frac{\sqrt2}2\pi x\right)\qquad\ins K=\left[0,1\right]\cup\left[\sqrt2,\sqrt2+1\right]
\end{equation*}
\end{esem}
\begin{center}
\includegraphics*[scale=0.8]{ex4.pdf}
\end{center}
\begin{proof}L'integrale è costante perché
\begin{equation}\label{eq:int-funz-non-per}
\intes{\ins K+t}{\cos\left(2\pi x\right)+\cos\left(\frac{\sqrt2}2\pi x\right)}=
\end{equation}
\begin{equation*}
\underbrace{\intes{\ins K}{\cos\left(2\pi\left(x+t\right)\right)}}_0+\intes{\ins K}{\cos\left(\frac{\sqrt2}2\pi\left(x+t\right)\right)}=
\end{equation*}
\begin{equation*}
=\frac{\sqrt2}\pi\left[\left.\sin\left(\frac{\sqrt2}2\pi x\right)\right|_0^1+\left.\sin\left(\frac{\sqrt2}2\pi x\right)\right|_{\sqrt2}^{\sqrt2+1}\right]=
\end{equation*}
\begin{equation*}\begin{split}
=\frac{\sqrt2}\pi\left[\sin\left(\frac{\sqrt2}2\pi\left(1+t\right)\right)\right.&
-\sin\left(\frac{\sqrt2}2\pi t\right)+\\
+\sin\left(\frac{\sqrt2}2\pi\left(\sqrt2+1+t\right)\right)&
\left.-\sin\left(\frac{\sqrt2}2\pi\left(\sqrt2+t\right)\right)\right]
\end{split}\end{equation*}
Il primo seno diventa $\sin\left(\frac{\sqrt2}2\pi+\frac{\sqrt2}2\pi t\right)$, il terzo $-\sin\left(\frac{\sqrt2}2\pi+\frac{\sqrt2}2\pi t\right)$ e il quarto $-\sin\left(\frac{\sqrt2}2\pi t\right)$ per cui l'integrale nella \eqref{eq:int-funz-non-per} è uguale a $0$.
 
Naturalmente la funzione non è periodica perché il periodo di $\cos(2\pi x)$ è $1$ mentre quello di $\cos\left(\frac{\sqrt2}2\pi x\right)$ è $2\sqrt2$, il che risulta in qualche modo controintuitivo.
\end{proof}
Gli strumenti dell'analisi di Fourier risultano poco applicabili, in quanto servono condizioni molto forti sull'insieme in esame perché ne consegua la convergenza $L^1(\rea)$ della funzione.
\begin{esem}Il fatto che $\mu(\ins X)=\infty$ e che $\intes{\ins X+t}{f(x)}=k\quad\forall t\in\rea$ non è condizione sufficiente perché $f\in L^1(\rea)$.
\end{esem}
\begin{proof}Sia $b_0=-1$, $a_n=b_{n-1}+1$ e $b_n=a_n+\frac1n$. Ora consideriamo gli insiemi $\ins A=\bigcup_{n=1}^\infty(a_n,b_n)$, $\ins B=\bigcup_{n=1}^\infty(-b_n,-a_n)$ e $\ins C=\ins A\cup\ins B$. Nonostante la misura di $\ins C$ sia infinita, l'unione finita di traslati di $\ins C$ non ricopre mai tutto $\rea$.

La funzione
\begin{equation*}
f(x)=\left\{\begin{matrix}1&-1<x<1\\\frac1{|x|}&\text {altrimenti}
\end{matrix}\right.
\end{equation*}
ha integrale convergente su $\ins C$ e su tutte le sue traslazioni ma naturalmente non su $\rea$.
\end{proof}
\end{document}