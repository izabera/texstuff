\documentclass[10pt]{article}
%\usepackage{fontspec}
\usepackage{amssymb, amsfonts, amsmath, amsthm}
\usepackage[italian]{babel}
\usepackage[utf8]{inputenc}
\usepackage{graphicx}
\usepackage{mathpazo}

\usepackage{pgf,tikz}
\usetikzlibrary{arrows}

\newcommand\rea{\mathbb R}

\newtheorem*{prop*}{Proposizione}
\newtheorem{prop}{Proposizione}
\newtheorem{teo}{Teorema}
\newtheorem{lemma}{Lemma}
\theoremstyle{definition}
\newtheorem*{defn}{Definizione}
\newtheorem{esem}{Esempio}
\newenvironment{dafinire}{\hrulefill}{\hrulefill}

\title{Triangoli in un piano colorato}
\author{Isabella Bosia}
\date{}
\definecolor{qqqqff}{rgb}{0,0,1}
\definecolor{ffqqqq}{rgb}{1,0,0}
\begin{document}
\maketitle
%-1 5 -2 5
\begin{teo}
Su un piano colorato di rosso e blu esiste un triangolo equilatero
con i vertici dello stesso colore.
\end{teo}

\begin{proof}
Supponiamo che la tesi sia falsa e prendiamo una tassellazione triangolare del piano.
Si considerino i seguenti punti $A$, $B$, $C$, $D$, $E$, $F$, $G$ e $H$.
\begin{center}
%\fbox{
\begin{tikzpicture}[line cap=round,line join=round,>=triangle 45,x=1.0cm,y=1.0cm]
\clip(-0.5,-1.3) rectangle (4,4.5);
\begin{scriptsize}
\fill [color=black] (0,1) circle (2.5pt);
\draw[color=black] (0.14,1.36) node {$A$};
\fill [color=black] (1.72,0) circle (2.5pt);
\draw[color=black] (1.88,0.36) node {$B$};
\fill [color=black] (3.45,0.99) circle (2.5pt);
\draw[color=black] (3.6,1.34) node {$D$};
\fill [color=black] (3.45,2.98) circle (2.5pt);
\draw[color=black] (3.62,3.34) node {$E$};
\fill [color=black] (1.73,3.98) circle (2.5pt);
\draw[color=black] (1.9,4.34) node {$F$};
\fill [color=black] (0.01,2.99) circle (2.5pt);
\draw[color=black] (0.14,3.34) node {$G$};
\fill [color=black] (1.73,1.99) circle (2.5pt);
\draw[color=black] (1.88,2.34) node {$C$};
\fill [color=black] (-0.01,-0.99) circle (2.5pt);
\draw[color=black] (0.16,-0.62) node {$H$};
\end{scriptsize}
\end{tikzpicture}
%}
\end{center}

Consideriamo i punti $A$, $B$ e $C$: se fossero dello stesso colore sarebbe assurdo.
Allora, siano $A$ e $B$ rossi e $C$ blu e consideriamo il punto $D$.
\begin{center}
%\fbox{
\begin{tikzpicture}[line cap=round,line join=round,>=triangle 45,x=1.0cm,y=1.0cm]
\clip(-0.5,-0.3) rectangle (4,2.5);
\begin{scriptsize}
\fill [color=ffqqqq] (0,1) circle (2.5pt);
\draw[color=ffqqqq] (0.14,1.36) node {$A$};
\fill [color=ffqqqq] (1.72,0) circle (2.5pt);
\draw[color=ffqqqq] (1.88,0.36) node {$B$};
\fill [color=qqqqff] (1.73,1.99) circle (2.5pt);
\draw[color=qqqqff] (1.88,2.34) node {$C$};
\fill [color=black] (3.45,0.99) circle (2.5pt);
\draw[color=black] (3.6,1.34) node {$D$};
\end{scriptsize}
\end{tikzpicture}
%}
\end{center}

Abbiamo i casi $D$ rosso e $D$ blu.

Nel caso $D$ sia rosso, allora $F$ deve essere blu,
altrimenti $ADF$ sarebbe un triangolo equilatero
con i vertici dello stesso colore, dunque assurdo. Ma se $F$ è blu,
allora $E$ e $G$ devono essere rossi per evitare $CEF$ e $CFG$.
Quindi il controesempio è il triangolo $BEG$.

\begin{center}
%\fbox{
\begin{tikzpicture}[line cap=round,line join=round,>=triangle 45,x=1.0cm,y=1.0cm]
\clip(-0.5,-0.3) rectangle (4,4.5);
\begin{scriptsize}
\fill [color=ffqqqq] (0,1) circle (2.5pt);
\draw[color=ffqqqq] (0.14,1.36) node {$A$};
\fill [color=ffqqqq] (1.72,0) circle (2.5pt);
\draw[color=ffqqqq] (1.88,0.36) node {$B$};
\fill [color=ffqqqq] (3.45,0.99) circle (2.5pt);
\draw[color=ffqqqq] (3.6,1.34) node {$D$};
\fill [color=ffqqqq] (3.45,2.98) circle (2.5pt);
\draw[color=ffqqqq] (3.62,3.34) node {$E$};
\fill [color=qqqqff] (1.73,3.98) circle (2.5pt);
\draw[color=qqqqff] (1.88,4.34) node {$F$};
\fill [color=ffqqqq] (0.01,2.99) circle (2.5pt);
\draw[color=ffqqqq] (0.16,3.34) node {$G$};
\fill [color=qqqqff] (1.73,1.99) circle (2.5pt);
\draw[color=qqqqff] (1.88,2.34) node {$C$};
\end{scriptsize}
\end{tikzpicture}
%}
\end{center}

Nel caso $D$ sia blu, allora $E$ deve essere rosso per evitare $CDE$. Allora
$E$ deve essere rosso per $CDE$, dunque $G$ deve essere blu per $BEG$, e dunque
$F$ deve essere rosso per $CFG$.

\begin{center}
%\fbox{
\begin{tikzpicture}[line cap=round,line join=round,>=triangle 45,x=1.0cm,y=1.0cm]
\clip(-0.5,-0.3) rectangle (4,4.5);
\begin{scriptsize}
\fill [color=ffqqqq] (0,1) circle (2.5pt);
\draw[color=ffqqqq] (0.14,1.36) node {$A$};
\fill [color=ffqqqq] (1.72,0) circle (2.5pt);
\draw[color=ffqqqq] (1.88,0.36) node {$B$};
\fill [color=qqqqff] (3.45,0.99) circle (2.5pt);
\draw[color=qqqqff] (3.6,1.34) node {$D$};
\fill [color=ffqqqq] (3.45,2.98) circle (2.5pt);
\draw[color=ffqqqq] (3.62,3.34) node {$E$};
\fill [color=ffqqqq] (1.73,3.98) circle (2.5pt);
\draw[color=ffqqqq] (1.88,4.34) node {$F$};
\fill [color=qqqqff] (0.01,2.99) circle (2.5pt);
\draw[color=qqqqff] (0.16,3.34) node {$G$};
\fill [color=qqqqff] (1.73,1.99) circle (2.5pt);
\draw[color=qqqqff] (1.88,2.34) node {$C$};
\end{scriptsize}
\end{tikzpicture}
%}
\end{center}

Allora $H$ deve essere blu per $ABH$ e rosso per $DGH$, assurdo.
\begin{center}
%\fbox{
\begin{tikzpicture}[line cap=round,line join=round,>=triangle 45,x=1.0cm,y=1.0cm]
\clip(-0.5,-1.3) rectangle (4,4.5);
\begin{scriptsize}
\fill [color=ffqqqq] (0,1) circle (2.5pt);
\draw[color=ffqqqq] (0.14,1.36) node {$A$};
\fill [color=ffqqqq] (1.72,0) circle (2.5pt);
\draw[color=ffqqqq] (1.88,0.36) node {$B$};
\fill [color=qqqqff] (3.45,0.99) circle (2.5pt);
\draw[color=qqqqff] (3.6,1.34) node {$D$};
\fill [color=ffqqqq] (3.45,2.98) circle (2.5pt);
\draw[color=ffqqqq] (3.62,3.34) node {$E$};
\fill [color=ffqqqq] (1.73,3.98) circle (2.5pt);
\draw[color=ffqqqq] (1.88,4.34) node {$F$};
\fill [color=qqqqff] (0.01,2.99) circle (2.5pt);
\draw[color=qqqqff] (0.16,3.34) node {$G$};
\fill [color=qqqqff] (1.73,1.99) circle (2.5pt);
\draw[color=qqqqff] (1.88,2.34) node {$C$};
\fill [color=black] (-0.01,-0.99) circle (2.5pt);
\draw[color=black] (0.16,-0.62) node {$H$};
\end{scriptsize}
\end{tikzpicture}
%}
\end{center}
\end{proof}

\end{document}