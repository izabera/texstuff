\documentclass[10pt]{article}
%\usepackage{fontspec}
\usepackage{amssymb, amsfonts, amsmath, amsthm}
\usepackage[italian]{babel}
\usepackage[utf8]{inputenc}
\usepackage{graphicx}
\usepackage{mathpazo}

\usepackage{pgf,tikz}
\usetikzlibrary{arrows}

\newcommand\rea{\mathbb R}

\newtheorem*{prop*}{Proposizione}
\newtheorem{prop}{Proposizione}
\newtheorem{teo}{Teorema}
\newtheorem{lemma}{Lemma}
\theoremstyle{definition}
\newtheorem*{defn}{Definizione}
\newtheorem{esem}{Esempio}
\newenvironment{dafinire}{\hrulefill}{\hrulefill}

\title{Triangoli in un piano colorato}
\author{Isabella Bosia}
\date{}
\definecolor{qqqqff}{rgb}{0,0,1}
\definecolor{ffqqqq}{rgb}{1,0,0}
\begin{document}
\maketitle
%-1 5 -2 5
\begin{teo}\label{2-colori}
Su un piano colorato di rosso e blu esiste un triangolo equilatero con i vertici dello stesso colore.
\end{teo}

\begin{proof}
Supponiamo che la tesi sia falsa e prendiamo una tassellazione triangolare del piano. Si considerino i seguenti punti $A$, $B$, $C$, $D$, $E$, $F$, $G$ e $H$.
\begin{center}
%\fbox{
\begin{tikzpicture}[line cap=round,line join=round,>=triangle 45,x=1.0cm,y=1.0cm]
\clip(-0.5,-1.3) rectangle (4,4.5);
\begin{scriptsize}
\fill [color=black] (0,1) circle (2.5pt);
\draw[color=black] (0.14,1.36) node {$A$};
\fill [color=black] (1.72,0) circle (2.5pt);
\draw[color=black] (1.88,0.36) node {$B$};
\fill [color=black] (3.45,0.99) circle (2.5pt);
\draw[color=black] (3.6,1.34) node {$D$};
\fill [color=black] (3.45,2.98) circle (2.5pt);
\draw[color=black] (3.62,3.34) node {$E$};
\fill [color=black] (1.73,3.98) circle (2.5pt);
\draw[color=black] (1.9,4.34) node {$F$};
\fill [color=black] (0.01,2.99) circle (2.5pt);
\draw[color=black] (0.14,3.34) node {$G$};
\fill [color=black] (1.73,1.99) circle (2.5pt);
\draw[color=black] (1.88,2.34) node {$C$};
\fill [color=black] (-0.01,-0.99) circle (2.5pt);
\draw[color=black] (0.16,-0.62) node {$H$};
\end{scriptsize}
\end{tikzpicture}
%}
\end{center}

Consideriamo i punti $A$, $B$ e $C$: se fossero dello stesso colore sarebbe assurdo. Allora, siano $A$ e $B$ rossi e $C$ blu e consideriamo il punto $D$.
\begin{center}
%\fbox{
\begin{tikzpicture}[line cap=round,line join=round,>=triangle 45,x=1.0cm,y=1.0cm]
\clip(-0.5,-0.3) rectangle (4,2.5);
\begin{scriptsize}
\fill [color=ffqqqq] (0,1) circle (2.5pt);
\draw[color=ffqqqq] (0.14,1.36) node {$A$};
\fill [color=ffqqqq] (1.72,0) circle (2.5pt);
\draw[color=ffqqqq] (1.88,0.36) node {$B$};
\fill [color=qqqqff] (1.73,1.99) circle (2.5pt);
\draw[color=qqqqff] (1.88,2.34) node {$C$};
\fill [color=black] (3.45,0.99) circle (2.5pt);
\draw[color=black] (3.6,1.34) node {$D$};
\end{scriptsize}
\end{tikzpicture}
%}
\end{center}

Abbiamo i casi $D$ rosso e $D$ blu.

Nel caso $D$ sia rosso, allora $F$ deve essere blu, altrimenti $ADF$ sarebbe un triangolo equilatero con i vertici dello stesso colore, dunque assurdo. Ma se $F$ è blu, allora $E$ e $G$ devono essere rossi per evitare $CEF$ e $CFG$. Quindi il controesempio è il triangolo $BEG$.

\begin{center}
%\fbox{
\begin{tikzpicture}[line cap=round,line join=round,>=triangle 45,x=1.0cm,y=1.0cm]
\clip(-0.5,-0.3) rectangle (4,4.5);
\begin{scriptsize}
\fill [color=ffqqqq] (0,1) circle (2.5pt);
\draw[color=ffqqqq] (0.14,1.36) node {$A$};
\fill [color=ffqqqq] (1.72,0) circle (2.5pt);
\draw[color=ffqqqq] (1.88,0.36) node {$B$};
\fill [color=ffqqqq] (3.45,0.99) circle (2.5pt);
\draw[color=ffqqqq] (3.6,1.34) node {$D$};
\fill [color=ffqqqq] (3.45,2.98) circle (2.5pt);
\draw[color=ffqqqq] (3.62,3.34) node {$E$};
\fill [color=qqqqff] (1.73,3.98) circle (2.5pt);
\draw[color=qqqqff] (1.88,4.34) node {$F$};
\fill [color=ffqqqq] (0.01,2.99) circle (2.5pt);
\draw[color=ffqqqq] (0.16,3.34) node {$G$};
\fill [color=qqqqff] (1.73,1.99) circle (2.5pt);
\draw[color=qqqqff] (1.88,2.34) node {$C$};
\end{scriptsize}
\end{tikzpicture}
%}
\end{center}

Nel caso $D$ sia blu, allora $E$ deve essere rosso per evitare $CDE$. Allora $E$ deve essere rosso per $CDE$, dunque $G$ deve essere blu per $BEG$, e dunque $F$ deve essere rosso per $CFG$.

\begin{center}
%\fbox{
\begin{tikzpicture}[line cap=round,line join=round,>=triangle 45,x=1.0cm,y=1.0cm]
\clip(-0.5,-0.3) rectangle (4,4.5);
\begin{scriptsize}
\fill [color=ffqqqq] (0,1) circle (2.5pt);
\draw[color=ffqqqq] (0.14,1.36) node {$A$};
\fill [color=ffqqqq] (1.72,0) circle (2.5pt);
\draw[color=ffqqqq] (1.88,0.36) node {$B$};
\fill [color=qqqqff] (3.45,0.99) circle (2.5pt);
\draw[color=qqqqff] (3.6,1.34) node {$D$};
\fill [color=ffqqqq] (3.45,2.98) circle (2.5pt);
\draw[color=ffqqqq] (3.62,3.34) node {$E$};
\fill [color=ffqqqq] (1.73,3.98) circle (2.5pt);
\draw[color=ffqqqq] (1.88,4.34) node {$F$};
\fill [color=qqqqff] (0.01,2.99) circle (2.5pt);
\draw[color=qqqqff] (0.16,3.34) node {$G$};
\fill [color=qqqqff] (1.73,1.99) circle (2.5pt);
\draw[color=qqqqff] (1.88,2.34) node {$C$};
\end{scriptsize}
\end{tikzpicture}
%}
\end{center}

Allora $H$ deve essere blu per $ABH$ e rosso per $DGH$, assurdo.
\begin{center}
%\fbox{
\begin{tikzpicture}[line cap=round,line join=round,>=triangle 45,x=1.0cm,y=1.0cm]
\clip(-0.5,-1.3) rectangle (4,4.5);
\begin{scriptsize}
\fill [color=ffqqqq] (0,1) circle (2.5pt);
\draw[color=ffqqqq] (0.14,1.36) node {$A$};
\fill [color=ffqqqq] (1.72,0) circle (2.5pt);
\draw[color=ffqqqq] (1.88,0.36) node {$B$};
\fill [color=qqqqff] (3.45,0.99) circle (2.5pt);
\draw[color=qqqqff] (3.6,1.34) node {$D$};
\fill [color=ffqqqq] (3.45,2.98) circle (2.5pt);
\draw[color=ffqqqq] (3.62,3.34) node {$E$};
\fill [color=ffqqqq] (1.73,3.98) circle (2.5pt);
\draw[color=ffqqqq] (1.88,4.34) node {$F$};
\fill [color=qqqqff] (0.01,2.99) circle (2.5pt);
\draw[color=qqqqff] (0.16,3.34) node {$G$};
\fill [color=qqqqff] (1.73,1.99) circle (2.5pt);
\draw[color=qqqqff] (1.88,2.34) node {$C$};
\fill [color=black] (-0.01,-0.99) circle (2.5pt);
\draw[color=black] (0.16,-0.62) node {$H$};
\end{scriptsize}
\end{tikzpicture}
%}
\end{center}
\end{proof}
Chiamiamo questa figura molto caratteristica ``esagono con cappello'', figura che se è colorata con $2$ colori ha la proprietà di contenere un triangolo equilatero con i vertici dello stesso colore.

\begin{teo}
Su un piano colorato con $3$ colori esiste un triangolo equilatero con i vertici dello stesso colore.
\end{teo}

\begin{proof}
Per il teorema di Van der Waerden, per ogni partizione finita di $\mathbb N$ esiste un sottoinsieme che contiene progressioni aritmetiche di lunghezza arbitraria. Perciò possiamo trovare $5$ punti allineati dello stesso colore.
\begin{center}
%\fbox{
\begin{tikzpicture}[line cap=round,line join=round,>=triangle 45,x=1.0cm,y=1.0cm]
\clip(-1.98,-0.92) rectangle (3.8,3.98);
\draw (-1.54,-0.58)-- (0.9,3.65);
\draw (0.9,3.65)-- (3.34,-0.58);
\draw (0.29,2.59)-- (2.12,-0.58);
\draw (3.34,-0.58)-- (-1.54,-0.58);
\draw (-0.93,0.48)-- (2.73,0.48);
\draw (2.12,1.53)-- (-0.32,1.53);
\draw (0.29,2.59)-- (1.51,2.59);
\draw (2.12,1.53)-- (0.9,-0.58);
\draw (-0.32,-0.58)-- (1.51,2.59);
\draw (2.73,0.48)-- (2.12,-0.58);
\draw (0.9,-0.58)-- (-0.32,1.53);
\draw (-0.93,0.48)-- (-0.32,-0.58);
\begin{scriptsize}
\fill [color=qqqqff] (-1.54,-0.58) circle (2.5pt);
\fill [color=qqqqff] (-0.32,-0.58) circle (2.5pt);
\fill [color=qqqqff] (0.9,-0.58) circle (2.5pt);
\fill [color=qqqqff] (2.12,-0.58) circle (2.5pt);
\fill [color=qqqqff] (3.34,-0.58) circle (2.5pt);
\fill [color=black] (-0.93,0.48) circle (2.5pt);
\fill [color=black] (0.29,0.48) circle (2.5pt);
\fill [color=black] (1.51,0.48) circle (2.5pt);
\fill [color=black] (2.73,0.48) circle (2.5pt);
\fill [color=black] (-0.32,1.53) circle (2.5pt);
\fill [color=black] (0.9,1.53) circle (2.5pt);
\fill [color=black] (2.12,1.53) circle (2.5pt);
\fill [color=black] (0.29,2.59) circle (2.5pt);
\fill [color=black] (1.51,2.59) circle (2.5pt);
\fill [color=black] (0.9,3.65) circle (2.5pt);
\end{scriptsize}
\end{tikzpicture}
%}
\end{center}
Nessuno dei punti sopra quei $5$ può essere blu, perché formerebbe un triangolo equilatero con due dei punti della base. Ma il triangolo dei punti neri contiene un esagono con cappello, che per il teorema \ref{2-colori} non può essere colorato con $2$ colori senza triangoli equilateri.
\end{proof}
Chiaramente possiamo generalizzarlo a qualsiasi $n$ finito, basta trovare una figura che non può essere riempita con $n-1$ colori senza contenere triangoli equilateri con i vertici dello stesso colore, poi la si chiude in un triangolo e lo si allunga di una riga.
\end{document}