\documentclass[10pt]{article}
%\usepackage{fontspec}
\usepackage{amssymb, amsfonts, amsmath, amsthm}
\usepackage[italian]{babel}
\usepackage[utf8]{inputenc}
\usepackage{graphicx}
\usepackage{mathpazo}

\newcommand\rea{\mathbb R}
\newcommand\nat{\mathbb N}
\newcommand\ins[1]{\mathbf #1}
\newcommand\inte[3]{\int\limits_{#1}^{#2}\!#3\text dx}
\newcommand\intex[3]{\int\limits_{#1}^{#2}\!#3(x)\text dx}
\newcommand\intes[2]{\int\limits_{#1}\!#2\text dx}
\newcommand\intesx[2]{\int\limits_{#1}\!#2(x)\text dx}

\newtheorem*{prop*}{Proposizione}
\newtheorem{prop}{Proposizione}
\newtheorem{teo}{Teorema}
\newtheorem{lemma}{Lemma}
\theoremstyle{definition}
\newtheorem*{defn}{Definizione}
\newtheorem{esem}{Esempio}
\newenvironment{dafinire}{\hrulefill}{\hrulefill}

%\setmainfont[Ligatures=TeX]{Linux Libertine O}
%\setdefaultlanguage{italian}

\title{Ragionamenti}
\author{Isabella Bosia}
\date{}

\begin{document}
\maketitle

Digressioni varie su argomenti interessosi.

\section{Filtri e ultrafiltri}

\subsection{FIP e non oltre}

Sia $\mathcal F$ un ultrafiltro che estende il filtro cofinito. Abbiamo che $\mathcal F$ contiene il complementare di ogni singoletto, quindi l'intersezione di tutti gli insiemi di $\mathcal F$ è vuota.

\subsection{Assioma della scelta?}

Sia $\mathcal F$ un filtro non principale su $\mathcal M$. Possiamo trovare una catena di filtri che contengono $\mathcal F$ e per il lemma di Zorn la catena ha un massimale, che è un ultrafiltro che estende $\mathcal F$. Si dimostra che l'esistenza di un ultrafiltro non principale non può essere provata senza l'assioma della scelta.

Procediamo in un altro modo: al posto che estendere il filtro verso l'alto, riduciamo l'insieme verso il basso. Preso $\mathcal F$, consideriamo l'insieme dei complementari $\mathcal G=\{x\in\mathcal M:x^c\in\mathcal F\}$. Ora $\mathcal F$ è un ultrafiltro su $\mathcal F\cup\mathcal G$, e continua ad essere non principale(?) Dov'è stato usato l'assioma della scelta?
\end{document}